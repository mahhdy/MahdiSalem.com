% ═══════════════════════════════════════════════════════════════════════════════
%                    تاریخ تحولات فرانسه - فصل [X]
% ═══════════════════════════════════════════════════════════════════════════════

\documentclass[12pt,a4paper]{book}

% ─────────────────────────── پکیج‌ها ───────────────────────────
\usepackage{amsmath,amssymb}
\usepackage{geometry}
\geometry{top=2.5cm, bottom=2.5cm, left=2cm, right=2.5cm, headheight=15pt}
\usepackage{graphicx}
\usepackage{array,booktabs,longtable,multirow,colortbl}
\usepackage{xcolor}
\usepackage{tikz}
\usetikzlibrary{shapes.geometric, arrows.meta, positioning, calc, backgrounds, 
	fit, decorations.pathmorphing, shadows, patterns}
\usepackage{pgfplots}
\pgfplotsset{compat=1.18}
\usepackage{tcolorbox}
\tcbuselibrary{skins,breakable}
\usepackage{enumitem}
\usepackage{fancyhdr}
\usepackage{setspace}
\usepackage{titlesec}
\usepackage{float}
\usepackage{pdfpages}
\usepackage{pdflscape}  % برای صفحات landscape
\usepackage{hyperref}

% ─────────────────────────── رنگ‌ها ───────────────────────────
\definecolor{bleurepublique}{RGB}{0, 35, 149}
\definecolor{rougerevolution}{RGB}{237, 41, 57}
\definecolor{orroyal}{RGB}{255, 215, 0}
\definecolor{vertnapoleon}{RGB}{0, 100, 0}
\definecolor{violetempire}{RGB}{128, 0, 128}
\definecolor{fondclair}{RGB}{255, 253, 240}
\definecolor{gris}{RGB}{128, 128, 128}
\definecolor{grisclair}{RGB}{245, 245, 245}
\definecolor{noirsombre}{RGB}{30, 30, 30}

% رنگ‌های کمکی
\definecolor{bleulight}{RGB}{230, 235, 250}
\definecolor{rougelight}{RGB}{253, 235, 237}
\definecolor{vertlight}{RGB}{235, 250, 235}
\definecolor{violetlight}{RGB}{245, 235, 250}
\definecolor{orroyallight}{RGB}{255, 250, 230}
\definecolor{grislight}{RGB}{248, 248, 248}
\definecolor{bleumid}{RGB}{180, 195, 230}
\definecolor{rougemid}{RGB}{245, 180, 185}
\definecolor{vertmid}{RGB}{180, 220, 180}
\definecolor{violetmid}{RGB}{210, 180, 220}
\definecolor{orroyalmid}{RGB}{255, 240, 180}
\definecolor{orroyaldark}{RGB}{200, 170, 0}

% ─────────────────────────── فونت فارسی ───────────────────────────
\usepackage{fontspec}
\setmainfont{Vazirmatn}
\usepackage{xepersian}
\settextfont{Vazirmatn}
\setdigitfont{Vazirmatn}

% ─────────────────────────── هایپرلینک ───────────────────────────
\hypersetup{
	colorlinks=true,
	linkcolor=bleurepublique,
	urlcolor=bleurepublique,
	citecolor=vertnapoleon
}

% ─────────────────────────── کادرها ───────────────────────────
\newtcolorbox{kholasebox}[1][]{enhanced,breakable,colback=bleulight,
	colframe=bleurepublique,coltitle=white,fonttitle=\bfseries\large,
	title={#1},boxrule=2pt,arc=4pt,left=10pt,right=10pt,top=10pt,bottom=10pt,
	drop shadow={opacity=0.3}}

\newtcolorbox{naghlbox}[1][]{enhanced,breakable,colback=orroyallight,
	colframe=orroyaldark,coltitle=black,fonttitle=\bfseries,title={#1},
	boxrule=1.5pt,arc=3pt,borderline west={4pt}{0pt}{orroyal},
	left=15pt,right=10pt,top=8pt,bottom=8pt}

\newtcolorbox{olgoobox}[1][]{enhanced,breakable,colback=vertlight,
	colframe=vertnapoleon,coltitle=white,fonttitle=\bfseries,title={#1},
	boxrule=1.5pt,arc=4pt,left=10pt,right=10pt,top=8pt,bottom=8pt,
	before upper={\parindent15pt}}

\newtcolorbox{enghelabbox}[1][]{enhanced,breakable,colback=rougelight,
	colframe=rougerevolution,coltitle=white,fonttitle=\bfseries,title={#1},
	boxrule=2pt,arc=4pt,left=10pt,right=10pt,top=8pt,bottom=8pt}

\newtcolorbox{empirebox}[1][]{enhanced,breakable,colback=violetlight,
	colframe=violetempire,coltitle=white,fonttitle=\bfseries,title={#1},
	boxrule=1.5pt,arc=4pt,left=10pt,right=10pt,top=8pt,bottom=8pt}

\newtcolorbox{noktebox}[1][]{enhanced,colback=grisclair,colframe=gris,
	fonttitle=\bfseries,title={#1},boxrule=1pt,arc=3pt,left=8pt,right=8pt}

% ─────────────────────────── صفحه‌آرایی ───────────────────────────
\pagestyle{fancy}
\fancyhf{}
\fancyhead[RO]{\leftmark}
\fancyhead[LE]{\rightmark}
\fancyfoot[C]{\thepage}
\renewcommand{\headrulewidth}{1pt}
\renewcommand{\footrulewidth}{0.5pt}
\setstretch{1.5}

\titleformat{\chapter}[display]
{\normalfont\huge\bfseries\color{bleurepublique}}
{\chaptertitlename\ \thechapter}{20pt}{\Huge}
\titleformat{\section}
{\normalfont\Large\bfseries\color{bleurepublique}}{\thesection}{1em}{}
\titleformat{\subsection}
{\normalfont\large\bfseries\color{bleurepublique}}{\thesubsection}{1em}{}

% ═══════════════════════════════════════════════════════════════════════════════
\begin{document}
	% ██████████████████████████████████████████████████████████████████████████████
	%
	%                    فصل ۲: رژیم قدیم و ریشه‌های انقلاب
	%
	% ██████████████████████████████████████████████████████████████████████████████
	
	\chapter{رژیم قدیم و ریشه‌های انقلاب}
	
	\begin{kholasebox}[خلاصه فصل دوم]
		این فصل به تحلیل ساختار «رژیم قدیم» (Ancien Régime) می‌پردازد: نظامی که از قرن شانزدهم تا ۱۷۸۹ بر فرانسه حاکم بود. ما ساختار سه‌گانه طبقاتی، نظام مالیاتی ناعادلانه، بحران مالی سلطنت، و تنش‌های فکری-اجتماعی را بررسی می‌کنیم. پرسش کلیدی: چرا این نظام قابل اصلاح نبود؟
	\end{kholasebox}
	
	\section{رژیم قدیم چیست؟}
	
	اصطلاح «رژیم قدیم» را انقلابیون برای توصیف نظامی به کار بردند که می‌خواستند نابود کنند. این نظام سه ویژگی محوری داشت:
	
	\begin{enumerate}
		\item \textbf{سلطنت مطلقه}: شاه منبع همه قدرت‌ها بود
		\item \textbf{جامعه طبقاتی}: سه طبقه با حقوق نابرابر
		\item \textbf{امتیازات موروثی}: حقوق بر اساس تولد، نه شایستگی
	\end{enumerate}
	
	\begin{naghlbox}[توکویل درباره رژیم قدیم]
		«من نه از افرادی سخن می‌گویم که انقلاب کردند، بلکه از وضعیتی سخن می‌گویم که انقلاب را ممکن ساخت. انقلاب فرانسه به اندازه‌ای که مولود آنچه انجام شد بود، مولود آنچه دیگر تحمل نمی‌شد نیز بود.»
		\begin{flushright}
			— الکسی دو توکویل، رژیم قدیم و انقلاب (۱۸۵۶)
		\end{flushright}
	\end{naghlbox}
	
	\section{ساختار سه‌گانه جامعه}
	
	جامعه فرانسه به سه «طبقه» (État) تقسیم می‌شد که هر کدام جایگاه حقوقی متفاوتی داشتند:
	
	\begin{figure}[H]
		\centering
		\begin{tikzpicture}[
			scale=1,
			transform shape
			]
			% هرم اجتماعی
			% طبقه اول - روحانیون (بالا)
			\fill[orroyallight, draw=orroyaldark, line width=2pt] 
			(0,8) -- (-1.5,6) -- (1.5,6) -- cycle;
			\node[font=\small\bfseries] at (0,7) {طبقه اول};
			\node[font=\scriptsize] at (0,6.5) {روحانیون};
			\node[font=\tiny] at (0,6.1) {۱۳۰,۰۰۰ نفر};
			
			% طبقه دوم - اشراف (وسط)
			\fill[violetlight, draw=violetempire, line width=2pt] 
			(-1.5,6) -- (-3,4) -- (3,4) -- (1.5,6) -- cycle;
			\node[font=\small\bfseries] at (0,5.2) {طبقه دوم};
			\node[font=\scriptsize] at (0,4.7) {اشراف};
			\node[font=\tiny] at (0,4.3) {۴۰۰,۰۰۰ نفر};
			
			% طبقه سوم - عوام (پایین - بزرگ)
			\fill[bleulight, draw=bleurepublique, line width=2pt] 
			(-3,4) -- (-6,0) -- (6,0) -- (3,4) -- cycle;
			\node[font=\small\bfseries] at (0,2.8) {طبقه سوم};
			\node[font=\scriptsize] at (0,2.3) {عوام (بورژوا، دهقان، کارگر)};
			\node[font=\tiny] at (0,1.8) {۲۷ میلیون نفر (۹۷٪)};
			
			% درصدها در کنار
			\node[right, font=\scriptsize, text=orroyaldark] at (2,7) {۰.۵٪ جمعیت};
			\node[right, font=\scriptsize, text=orroyaldark] at (2,6.5) {۱۰٪ زمین};
			
			\node[right, font=\scriptsize, text=violetempire] at (3.5,5) {۱.۵٪ جمعیت};
			\node[right, font=\scriptsize, text=violetempire] at (3.5,4.5) {۲۵٪ زمین};
			
			\node[right, font=\scriptsize, text=bleurepublique] at (6.5,2.5) {۹۷٪ جمعیت};
			\node[right, font=\scriptsize, text=bleurepublique] at (6.5,2) {۶۵٪ زمین};
			\node[right, font=\scriptsize, text=rougerevolution] at (6.5,1.5) {۱۰۰٪ مالیات!};
			
		\end{tikzpicture}
		\caption{هرم اجتماعی فرانسه در رژیم قدیم}
	\end{figure}
	
	\subsection{طبقه اول: روحانیون (Le Clergé)}
	
	\begin{table}[H]
		\centering
		\caption{ساختار طبقه روحانیون}
		\begin{tabular}{|>{\bfseries}r|c|p{5cm}|p{4cm}|}
			\hline
			\rowcolor{orroyallight}
			\textbf{گروه} & \textbf{تعداد} & \textbf{ویژگی} & \textbf{موضع سیاسی} \\
			\hline
			روحانیون بالا & ۱۰,۰۰۰ & اسقف‌ها، رؤسای صومعه‌ها، از اشراف & محافظه‌کار، طرفدار سلطنت \\
			\hline
			\rowcolor{grisclair}
			روحانیون پایین & ۱۲۰,۰۰۰ & کشیشان روستا، فقیر & اصلاح‌طلب، نزدیک به مردم \\
			\hline
		\end{tabular}
	\end{table}
	
	\begin{noktebox}[امتیازات روحانیون]
		\begin{itemize}[nosep]
			\item معاف از مالیات مستقیم
			\item دریافت «عُشر» (ده‌یک) از محصولات
			\item مالکیت ۱۰٪ زمین‌های فرانسه
			\item دادگاه‌های اختصاصی (محاکم کلیسایی)
			\item انحصار آموزش و ثبت احوال
		\end{itemize}
	\end{noktebox}
	
	\subsection{طبقه دوم: اشراف (La Noblesse)}
	
	اشراف فرانسه خود به چند دسته تقسیم می‌شدند:
	
	\begin{table}[H]
		\centering
		\caption{انواع اشراف فرانسه}
		\begin{tabular}{|>{\bfseries}r|p{4cm}|p{4cm}|p{3.5cm}|}
			\hline
			\rowcolor{violetlight}
			\textbf{نوع} & \textbf{ریشه} & \textbf{ویژگی} & \textbf{موضع سیاسی} \\
			\hline
			اشراف شمشیر & نجیب‌زادگان قدیمی & امتیازات فئودالی، غرور نسب & محافظه‌کار \\
			\hline
			\rowcolor{grisclair}
			اشراف ردا & خریداران مناصب قضایی & بورژوازی اشرافی‌شده & میانه‌رو \\
			\hline
			اشراف دربار & ساکنان ورسای & وابسته به شاه & طرفدار وضع موجود \\
			\hline
			\rowcolor{grisclair}
			اشراف لیبرال & متأثر از روشنگری & لافایت، میرابو & اصلاح‌طلب \\
			\hline
		\end{tabular}
	\end{table}
	
	\begin{enghelabbox}[«واکنش اشرافی» قرن هجدهم]
		در نیمه دوم قرن هجدهم، اشراف تلاش کردند امتیازات ازدست‌رفته خود را بازپس گیرند:
		\begin{itemize}[nosep]
			\item احیای حقوق فئودالی فراموش‌شده
			\item انحصار مناصب نظامی و کلیسایی برای اشراف
			\item مقاومت در برابر اصلاحات مالیاتی
		\end{itemize}
		این «واکنش اشرافی» تضاد با بورژوازی را تشدید کرد.
	\end{enghelabbox}
	
	\subsection{طبقه سوم: عوام (Le Tiers État)}
	
	طبقه سوم ۹۷٪ جمعیت را شامل می‌شد، اما خود بسیار ناهمگن بود:
	
	\begin{figure}[H]
		\centering
		\begin{tikzpicture}[
			scale=0.95,
			transform shape,
			bourgeois/.style={
				rectangle, draw=bleurepublique, line width=1.5pt, fill=bleulight,
				text=black, minimum width=4cm, minimum height=1.5cm,
				align=center, font=\small, rounded corners=3pt
			},
			peuple/.style={
				rectangle, draw=rougerevolution, line width=1.5pt, fill=rougelight,
				text=black, minimum width=4cm, minimum height=1.5cm,
				align=center, font=\small, rounded corners=3pt
			},
			paysan/.style={
				rectangle, draw=vertnapoleon, line width=1.5pt, fill=vertlight,
				text=black, minimum width=4cm, minimum height=1.5cm,
				align=center, font=\small, rounded corners=3pt
			}
			]
			% عنوان
			\node[font=\large\bfseries, text=bleurepublique] at (0,5) {طبقه سوم: ترکیب داخلی};
			
			% بورژوازی
			\node[bourgeois] (burg1) at (-5,3) {
				\begin{tabular}{c}
					\textbf{بورژوازی بالا}\\
					{\scriptsize بانکداران، بازرگانان بزرگ}
				\end{tabular}
			};
			\node[bourgeois] (burg2) at (-5,1) {
				\begin{tabular}{c}
					\textbf{بورژوازی حرفه‌ای}\\
					{\scriptsize وکلا، پزشکان، نویسندگان}
				\end{tabular}
			};
			\node[bourgeois] (burg3) at (-5,-1) {
				\begin{tabular}{c}
					\textbf{بورژوازی کوچک}\\
					{\scriptsize دکانداران، کارفرمایان کوچک}
				\end{tabular}
			};
			
			% توده شهری
			\node[peuple] (peup1) at (0,2) {
				\begin{tabular}{c}
					\textbf{صنعتگران}\\
					{\scriptsize استادکاران، کارگران ماهر}
				\end{tabular}
			};
			\node[peuple] (peup2) at (0,0) {
				\begin{tabular}{c}
					\textbf{سانکولوت‌ها}\\
					{\scriptsize کارگران روزمزد، فقرای شهری}
				\end{tabular}
			};
			
			% دهقانان
			\node[paysan] (pays1) at (5,3) {
				\begin{tabular}{c}
					\textbf{دهقانان مالک}\\
					{\scriptsize ۳۰-۴۰٪ زمین}
				\end{tabular}
			};
			\node[paysan] (pays2) at (5,1) {
				\begin{tabular}{c}
					\textbf{مستأجران}\\
					{\scriptsize اجاره‌نشینان زمین}
				\end{tabular}
			};
			\node[paysan] (pays3) at (5,-1) {
				\begin{tabular}{c}
					\textbf{کارگران کشاورزی}\\
					{\scriptsize بی‌زمین‌ها}
				\end{tabular}
			};
			
			% درصدها
			\node[below, font=\scriptsize, text=bleurepublique] at (-5,-2) {۸-۱۰٪ طبقه سوم};
			\node[below, font=\scriptsize, text=rougerevolution] at (0,-1) {۱۰-۱۵٪ طبقه سوم};
			\node[below, font=\scriptsize, text=vertnapoleon] at (5,-2) {۷۵-۸۰٪ طبقه سوم};
			
			% برچسب‌ها
			\node[above, font=\footnotesize\bfseries, text=bleurepublique] at (-5,4) {بورژوازی};
			\node[above, font=\footnotesize\bfseries, text=rougerevolution] at (0,3) {توده شهری};
			\node[above, font=\footnotesize\bfseries, text=vertnapoleon] at (5,4) {دهقانان};
			
		\end{tikzpicture}
		\caption{ترکیب داخلی طبقه سوم}
	\end{figure}
	
	\begin{naghlbox}[سی‌یس: طبقه سوم چیست؟]
		«طبقه سوم چیست؟ همه‌چیز. تاکنون در نظام سیاسی چه بوده است؟ هیچ. چه می‌خواهد باشد؟ چیزی.»
		\begin{flushright}
			— امانوئل ژوزف سی‌یس، طبقه سوم چیست؟ (ژانویه ۱۷۸۹)
		\end{flushright}
	\end{naghlbox}
	
	\section{نظام سیاسی: سلطنت مطلقه}
	
	\subsection{نظریه سلطنت مطلقه}
	
	\begin{table}[H]
		\centering
		\caption{اصول نظری سلطنت مطلقه فرانسه}
		\begin{tabular}{|>{\bfseries}r|p{5cm}|p{5cm}|}
			\hline
			\rowcolor{orroyallight}
			\textbf{اصل} & \textbf{محتوا} & \textbf{منبع نظری} \\
			\hline
			حق الهی & شاه نماینده خدا بر زمین است & بوسوئه (۱۶۲۷-۱۷۰۴) \\
			\hline
			\rowcolor{grisclair}
			حاکمیت واحد & همه قدرت در شاه متمرکز است & ژان بدن (۱۵۳۰-۱۵۹۶) \\
			\hline
			فوق قانون & شاه قانون‌گذار است، نه تابع قانون & سنت رومی \\
			\hline
			\rowcolor{grisclair}
			پدرسالاری & شاه پدر ملت است & استعاره سنتی \\
			\hline
		\end{tabular}
	\end{table}
	
	\begin{naghlbox}[لویی چهاردهم]
		«دولت، خود من هستم.» (L'État, c'est moi)
		\begin{flushright}
			— منسوب به لویی چهاردهم (احتمالاً افسانه‌ای)
		\end{flushright}
	\end{naghlbox}
	
	\subsection{واقعیت: محدودیت‌های عملی}
	
	با وجود نظریه «مطلقه»، قدرت شاه در عمل محدود بود:
	
	\begin{figure}[H]
		\centering
		\begin{tikzpicture}[
			scale=0.9,
			transform shape,
			limit/.style={
				rectangle, draw=rougerevolution, line width=1.5pt, fill=rougelight,
				text=black, minimum width=4cm, minimum height=1.2cm,
				align=center, font=\small, rounded corners=3pt
			},
			king/.style={
				ellipse, draw=orroyaldark, line width=2pt, fill=orroyallight,
				text=black, minimum width=3cm, minimum height=2cm,
				align=center, font=\bfseries
			}
			]
			% شاه در مرکز
			\node[king] (king) at (0,0) {شاه};
			
			% محدودیت‌ها
			\node[limit] (parl) at (-5,2) {
				\begin{tabular}{c}
					پارلمان‌ها\\
					{\scriptsize ثبت قوانین، حق اعتراض}
				\end{tabular}
			};
			
			\node[limit] (priv) at (5,2) {
				\begin{tabular}{c}
					امتیازات محلی\\
					{\scriptsize استان‌های مرزی، شهرها}
				\end{tabular}
			};
			
			\node[limit] (eglise) at (-5,-2) {
				\begin{tabular}{c}
					کلیسا\\
					{\scriptsize استقلال نسبی، قوانین کلیسایی}
				\end{tabular}
			};
			
			\node[limit] (corp) at (5,-2) {
				\begin{tabular}{c}
					اصناف و صنفی‌ها\\
					{\scriptsize انحصارات، مقررات}
				\end{tabular}
			};
			
			\node[limit] (noble) at (0,3.5) {
				\begin{tabular}{c}
					اشراف\\
					{\scriptsize امتیازات فئودالی}
				\end{tabular}
			};
			
			\node[limit] (opin) at (0,-3.5) {
				\begin{tabular}{c}
					افکار عمومی\\
					{\scriptsize قدرت جدید قرن ۱۸}
				\end{tabular}
			};
			
			% خطوط محدودکننده
			\draw[->, >=Stealth, line width=1.5pt, rougerevolution] (parl) -- (king);
			\draw[->, >=Stealth, line width=1.5pt, rougerevolution] (priv) -- (king);
			\draw[->, >=Stealth, line width=1.5pt, rougerevolution] (eglise) -- (king);
			\draw[->, >=Stealth, line width=1.5pt, rougerevolution] (corp) -- (king);
			\draw[->, >=Stealth, line width=1.5pt, rougerevolution] (noble) -- (king);
			\draw[->, >=Stealth, line width=1.5pt, rougerevolution] (opin) -- (king);
			
		\end{tikzpicture}
		\caption{محدودیت‌های عملی قدرت شاه}
	\end{figure}
	
	\subsection{پارلمان‌ها: مقاومت نهادی}
	
	\begin{olgoobox}[نقش پارلمان‌های فرانسه]
		پارلمان‌های فرانسه (۱۳ پارلمان در سراسر کشور) دادگاه‌های عالی بودند، نه مجالس قانون‌گذاری. اما قدرت مهمی داشتند:
		
		\begin{enumerate}[nosep]
			\item \textbf{ثبت قوانین}: فرمان‌های شاهی باید ثبت می‌شدند
			\item \textbf{حق اعتراض} (Remonstrance): می‌توانستند اعتراض کنند
			\item \textbf{تفسیر قانون}: در اجرای قوانین نقش داشتند
		\end{enumerate}
		
		پارلمان پاریس قوی‌ترین بود و اغلب با اصلاحات مالیاتی مخالفت می‌کرد — البته برای حفظ امتیازات، نه برای عدالت.
	\end{olgoobox}
	
	\section{نظام اقتصادی: فئودالیسم متأخر}
	
	\subsection{ساختار کشاورزی}
	
	\begin{table}[H]
		\centering
		\caption{توزیع مالکیت زمین در فرانسه ۱۷۸۹}
		\begin{tabular}{|>{\bfseries}r|c|c|c|}
			\hline
			\rowcolor{vertlight}
			\textbf{مالک} & \textbf{درصد زمین} & \textbf{درصد جمعیت} & \textbf{نسبت} \\
			\hline
			کلیسا & ۱۰٪ & ۰.۵٪ & ۲۰:۱ \\
			\hline
			\rowcolor{grisclair}
			اشراف & ۲۵٪ & ۱.۵٪ & ۱۷:۱ \\
			\hline
			بورژوازی & ۳۰٪ & ۸٪ & ۴:۱ \\
			\hline
			\rowcolor{grisclair}
			دهقانان & ۳۵٪ & ۹۰٪ & ۰.۴:۱ \\
			\hline
		\end{tabular}
	\end{table}
	
	\subsection{حقوق فئودالی: بار بر دوش دهقانان}
	
	\begin{enghelabbox}[انواع حقوق فئودالی]
		دهقانان حتی اگر مالک زمین بودند، باید عوارض متعددی می‌پرداختند:
		
		\begin{description}[nosep]
			\item[سانس (Cens):] اجاره سالانه نمادین
			\item[شامپار (Champart):] سهم ارباب از محصول (۱/۸ تا ۱/۴)
			\item[لود و وانت (Lods et ventes):] مالیات انتقال زمین
			\item[بانالیته (Banalités):] انحصار آسیاب، تنور، و پرس شراب
			\item[کوروه (Corvée):] کار اجباری برای ارباب
			\item[حق شکار:] ارباب می‌توانست در زمین دهقان شکار کند
		\end{description}
	\end{enghelabbox}
	
	\begin{naghlbox}[آرتور یانگ، سیاح انگلیسی]
		«از شهر پو تا بایون، زمین‌ها تقریباً همه متعلق به دهقانان کوچک است... اما فقر چنان آشکار است که گویی مردم زیر حکومت ترکان زندگی می‌کنند، نه تحت سلطنت مسیحی.»
		\begin{flushright}
			— آرتور یانگ، سفرنامه فرانسه (۱۷۸۹)
		\end{flushright}
	\end{naghlbox}
	
	\section{نظام مالیاتی: قلب مشکل}
	
	\subsection{انواع مالیات‌ها}
	
	% صفحه افقی برای جدول بزرگ مالیات‌ها
	\begin{landscape}
		\begin{table}[p]
			\centering
			\caption{نظام مالیاتی رژیم قدیم فرانسه}
			\small
			\begin{tabular}{|>{\bfseries}r|p{3cm}|p{3.5cm}|p{3cm}|p{3cm}|p{4cm}|}
				\hline
				\rowcolor{rougelight}
				\textbf{مالیات} & \textbf{نوع} & \textbf{پایه} & \textbf{پرداخت‌کننده} & \textbf{معاف} & \textbf{مشکل اصلی} \\
				\hline
				تای (Taille) & مستقیم & زمین یا شخص & طبقه سوم & روحانیون، اشراف & نابرابری طبقاتی \\
				\hline
				\rowcolor{grisclair}
				کاپیتاسیون (Capitation) & مستقیم & سرانه & همه (نظری) & معافیت‌های عملی & فرار اشراف \\
				\hline
				ونتی‌یم (Vingtième) & مستقیم & ۵٪ درآمد & همه (نظری) & مقاومت اشراف & ناکارآمدی \\
				\hline
				\rowcolor{grisclair}
				گابل (Gabelle) & غیرمستقیم & نمک (اجباری) & همه & مناطق معاف & بسیار منفور \\
				\hline
				اِد (Aides) & غیرمستقیم & نوشیدنی & همه & تفاوت منطقه‌ای & پیچیدگی \\
				\hline
				\rowcolor{grisclair}
				عُشر (Dîme) & کلیسایی & ۱۰٪ محصول & دهقانان & روحانیون، اشراف & سنگینی \\
				\hline
				حقوق فئودالی & فئودالی & متنوع & دهقانان & ارباب‌ها & ظلم آشکار \\
				\hline
			\end{tabular}
		\end{table}
	\end{landscape}
	
	\subsection{نابرابری مالیاتی: ریشه خشم}
	
	\begin{figure}[H]
		\centering
		\begin{tikzpicture}
			\begin{axis}[
				width=14cm,
				height=8cm,
				ybar,
				bar width=20pt,
				ylabel={درصد از درآمد پرداختی به عنوان مالیات},
				xlabel={گروه اجتماعی},
				ymin=0, ymax=60,
				xtick=data,
				xticklabels={
					روحانیون بالا,
					اشراف دربار,
					اشراف محلی,
					بورژوازی بالا,
					صنعتگران,
					دهقان متوسط,
					دهقان فقیر
				},
				x tick label style={rotate=45, anchor=east, font=\small},
				nodes near coords,
				nodes near coords style={font=\scriptsize},
				legend pos=north west,
				grid=major,
				grid style={gris!30}
				]
				\addplot[fill=rougerevolution!70, draw=rougerevolution] coordinates {
					(1,2) (2,5) (3,10) (4,15) (5,25) (6,40) (7,55)
				};
			\end{axis}
		\end{tikzpicture}
		\caption{بار مالیاتی نسبی گروه‌های اجتماعی (تخمین)}
	\end{figure}
	
	\begin{olgoobox}[الگوی کلیدی: نابرابری مالیاتی]
		\textbf{پارادوکس مالیاتی رژیم قدیم}:
		\begin{itemize}[nosep]
			\item فقیرترین‌ها بیشترین مالیات را می‌پرداختند
			\item ثروتمندترین‌ها کمترین مالیات را می‌پرداختند
			\item هرچه کسی قدرتمندتر بود، بهتر می‌توانست فرار کند
		\end{itemize}
		
		این نابرابری، بیش از هر چیز دیگر، خشم مردم را برانگیخت.
	\end{olgoobox}
	
	\section{بحران مالی: شتاب‌دهنده انقلاب}
	
	\subsection{ریشه‌های بحران}
	
	\begin{table}[H]
		\centering
		\caption{عوامل بحران مالی سلطنت فرانسه}
		\begin{tabular}{|>{\bfseries}r|p{5cm}|p{4cm}|}
			\hline
			\rowcolor{rougelight}
			\textbf{عامل} & \textbf{توضیح} & \textbf{هزینه تقریبی} \\
			\hline
			جنگ‌ها & جنگ هفت‌ساله، استقلال آمریکا & ۱.۵ میلیارد لیور \\
			\hline
			\rowcolor{grisclair}
			دربار ورسای & هزینه‌های تجملی & ۶-۸٪ بودجه \\
			\hline
			بهره بدهی & بدهی انباشته & ۵۰٪ درآمد! \\
			\hline
			\rowcolor{grisclair}
			ناکارآمدی & فساد، فرار مالیاتی & غیرقابل محاسبه \\
			\hline
		\end{tabular}
	\end{table}
	
	\subsection{کمک به انقلاب آمریکا: بومرنگ تاریخ}
	
	\begin{empirebox}[پارادوکس مداخله در آمریکا]
		فرانسه برای ضربه زدن به انگلستان، از انقلاب آمریکا حمایت کرد:
		\begin{itemize}[nosep]
			\item هزینه: حدود ۱.۳ میلیارد لیور
			\item نتیجه فوری: شکست انگلستان
			\item نتیجه میان‌مدت: ورشکستگی فرانسه
			\item نتیجه بلندمدت: الهام‌بخشی ایده‌های انقلابی
		\end{itemize}
		
		لافایت و دیگر افسران فرانسوی که در آمریکا جنگیدند، ایده‌های جمهوری‌خواهانه را به فرانسه آوردند.
	\end{empirebox}
	
	\subsection{تلاش‌های اصلاحی ناکام}
	
	\begin{table}[H]
		\centering
		\caption{وزرای اصلاح‌طلب و سرنوشت آنها}
		\begin{tabular}{|>{\bfseries}r|c|p{4cm}|p{4cm}|}
			\hline
			\rowcolor{bleumid}
			\textbf{وزیر} & \textbf{دوره} & \textbf{اصلاح پیشنهادی} & \textbf{سرنوشت} \\
			\hline
			تورگو & ۱۷۷۴-۷۶ & آزادسازی اقتصادی، لغو اصناف & برکنار (مخالفت اشراف) \\
			\hline
			\rowcolor{grisclair}
			نکر & ۱۷۷۶-۸۱ & وام‌گیری، شفافیت & برکنار (انتشار بودجه) \\
			\hline
			کالون & ۱۷۸۳-۸۷ & مالیات بر زمین (همه) & برکنار (مخالفت اشراف) \\
			\hline
			\rowcolor{grisclair}
			بری‌ین & ۱۷۸۷-۸۸ & مشابه کالون & برکنار (بن‌بست) \\
			\hline
			نکر (دوم) & ۱۷۸۸-۸۹ & فراخوان مجلس طبقات & برکنار (۱۱ ژوئیه ۱۷۸۹) \\
			\hline
		\end{tabular}
	\end{table}
	
	\begin{naghlbox}[کالون به مجمع اعیان]
		«امتیازات باید قربانی شوند... آیا می‌توان کشور را با یک طبقه ممتاز نجات داد؟ نه! بدون کمک همگانی، نجاتی نیست.»
		\begin{flushright}
			— شارل الکساندر دو کالون (۱۷۸۷)
		\end{flushright}
	\end{naghlbox}
	
	\section{«شورش اشرافی» ۱۷۸۷-۱۷۸۸}
	
	\begin{enghelabbox}[مرحله فراموش‌شده: انقلاب اشراف]
		قبل از انقلاب مردمی، یک \textbf{شورش اشرافی} رخ داد:
		\begin{enumerate}[nosep]
			\item \textbf{۱۷۸۷}: پارلمان پاریس اصلاحات کالون را رد کرد
			\item \textbf{۱۷۸۸}: پارلمان‌ها خواستار فراخوان «مجلس طبقات» شدند
			\item شاه مجبور شد تسلیم شود
		\end{enumerate}
		
		اشراف فکر می‌کردند مجلس طبقات قدرت آنها را تقویت می‌کند. اشتباه بزرگی بود.
	\end{enghelabbox}
	
	\section{بحران اقتصادی ۱۷۸۸-۱۷۸۹}
	
	\begin{figure}[H]
		\centering
		\begin{tikzpicture}
			\begin{axis}[
				width=14cm,
				height=7cm,
				xlabel={ماه},
				ylabel={قیمت نان (سو در لیور)},
				xmin=1, xmax=12,
				ymin=0, ymax=5,
				xtick={1,2,3,4,5,6,7,8,9,10,11,12},
				xticklabels={ژان,فور,مارس,آور,مه,ژوئن,ژوئیه,اوت,سپت,اکت,نوو,دسا},
				grid=both,
				grid style={gris!30},
				legend pos=north west
				]
				% قیمت نان در ۱۷۸۸
				\addplot[bleurepublique, line width=1.5pt, mark=o] coordinates {
					(1,2) (2,2.1) (3,2.2) (4,2.3) (5,2.5) (6,2.8) 
					(7,3) (8,3.2) (9,3.5) (10,3.7) (11,3.9) (12,4)
				};
				\addlegendentry{۱۷۸۸}
				
				% قیمت نان در ۱۷۸۹
				\addplot[rougerevolution, line width=2pt, mark=*] coordinates {
					(1,4.2) (2,4.3) (3,4.5) (4,4.5) (5,4.3) (6,4.2)
					(7,4.5) (8,3.5) (9,3) (10,2.8) (11,2.5) (12,2.3)
				};
				\addlegendentry{۱۷۸۹}
				
			\end{axis}
			
			% نشانگر انقلاب
			\node[circle, draw=rougerevolution, fill=rougelight, minimum size=8pt] at (9.5,5.8) {};
			\node[right, font=\scriptsize, text=rougerevolution] at (9.7,5.8) {۱۴ ژوئیه};
			
		\end{tikzpicture}
		\caption{قیمت نان در پاریس ۱۷۸۸-۱۷۸۹}
	\end{figure}
	
	\begin{noktebox}[زمستان سخت ۱۷۸۸-۱۷۸۹]
		\begin{itemize}[nosep]
			\item تگرگ ژوئیه ۱۷۸۸ محصول را نابود کرد
			\item زمستان سخت‌ترین در ۸۰ سال
			\item قیمت نان به بالاترین سطح قرن رسید
			\item بیکاری گسترده به دلیل رکود
			\item خانواده‌ها ۸۰-۹۰٪ درآمد را صرف نان می‌کردند
		\end{itemize}
	\end{noktebox}
	
	\section{فراخوان مجلس طبقات}
	
	\subsection{مجلس طبقات چیست؟}
	
	\begin{table}[H]
		\centering
		\caption{مجلس طبقات عمومی (États généraux)}
		\begin{tabular}{|>{\bfseries}r|p{8cm}|}
			\hline
			\rowcolor{bleumid}
			\textbf{ویژگی} & \textbf{توضیح} \\
			\hline
			تعریف & مجلس مشورتی سه طبقه که شاه فرامی‌خواند \\
			\hline
			\rowcolor{grisclair}
			آخرین بار & ۱۶۱۴ (۱۷۵ سال قبل!) \\
			\hline
			ساختار & سه اتاق جداگانه، هر اتاق یک رأی \\
			\hline
			\rowcolor{grisclair}
			مشکل & دو طبقه ممتاز علیه یک طبقه (۲ در برابر ۱) \\
			\hline
		\end{tabular}
	\end{table}
	
	\subsection{بحث «دو برابر شدن»}
	
	\begin{figure}[H]
		\centering
		\begin{tikzpicture}[
			scale=0.9,
			option/.style={
				rectangle, draw=#1, line width=1.5pt, fill=#1,
				text=black, minimum width=5cm, minimum height=3cm,
				align=center, font=\small, rounded corners=5pt
			}
			]
			% دو گزینه
			\node[option=orroyallight] (old) at (-4,0) {
				\begin{tabular}{c}
					\textbf{روش سنتی}\\[5pt]
					۳۰۰ نماینده هر طبقه\\
					رأی‌گیری به تفکیک طبقه\\
					نتیجه: ۲ به ۱ علیه طبقه سوم\\[5pt]
					{\scriptsize طرفداران: اشراف، روحانیون بالا}
				\end{tabular}
			};
			
			\node[option=bleulight] (new) at (4,0) {
				\begin{tabular}{c}
					\textbf{روش جدید}\\[5pt]
					۶۰۰ نماینده طبقه سوم\\
					رأی‌گیری سرانه\\
					نتیجه: امکان پیروزی طبقه سوم\\[5pt]
					{\scriptsize طرفداران: بورژوازی، اشراف لیبرال}
				\end{tabular}
			};
			
			% نتیجه
			\node[rectangle, draw=vertnapoleon, line width=2pt, fill=vertlight,
			minimum width=8cm, minimum height=1.5cm, rounded corners=5pt]
			at (0,-3.5) {
				\begin{tabular}{c}
					\textbf{تصمیم نکر (دسامبر ۱۷۸۸):}\\
					دوبرابر شدن تعداد، اما رأی‌گیری نامشخص!
				\end{tabular}
			};
			
			\draw[->, >=Stealth, line width=1.5pt, gris] (old) -- (0,-2);
			\draw[->, >=Stealth, line width=1.5pt, gris] (new) -- (0,-2);
			
		\end{tikzpicture}
		\caption{بحث ساختار مجلس طبقات}
	\end{figure}
	
	\subsection{دفاتر شکایات (Cahiers de doléances)}
	
	\begin{olgoobox}[منبع بی‌نظیر تاریخی]
		برای انتخاب نمایندگان، از مردم خواسته شد شکایات خود را بنویسند. حدود \textbf{۶۰,۰۰۰ دفتر} تهیه شد که به ما نشان می‌دهد:
		
		\begin{itemize}[nosep]
			\item \textbf{همه خواستار اصلاحات بودند} — حتی اشراف و روحانیون
			\item \textbf{کسی خواستار انقلاب نبود} — فقط اصلاح نظام موجود
			\item \textbf{شاه محبوب بود} — انتقاد از وزرا، نه شاه
		\end{itemize}
		
		خواست‌های مشترک: برابری مالیاتی، قانون اساسی، آزادی مطبوعات، لغو حقوق فئودالی
	\end{olgoobox}
	
	\section{چرا رژیم قدیم اصلاح‌ناپذیر بود؟}
	
	\begin{figure}[H]
		\centering
		\begin{tikzpicture}[
			scale=0.95,
			cause/.style={
				rectangle, draw=rougerevolution, line width=1.5pt, fill=rougelight,
				text=black, minimum width=4cm, minimum height=1.5cm,
				align=center, font=\small, rounded corners=3pt
			},
			result/.style={
				rectangle, draw=bleurepublique, line width=2pt, fill=bleulight,
				text=black, minimum width=5cm, minimum height=2cm,
				align=center, font=\small\bfseries, rounded corners=5pt
			}
			]
			% علل
			\node[cause] (c1) at (-5,3) {تعارض منافع طبقاتی};
			\node[cause] (c2) at (0,3) {ضعف نهادهای میانجی};
			\node[cause] (c3) at (5,3) {فرهنگ سیاسی مطلق‌گرا};
			
			\node[cause] (c4) at (-5,0) {بحران مالی حاد};
			\node[cause] (c5) at (0,0) {افکار عمومی رادیکال};
			\node[cause] (c6) at (5,0) {بی‌کفایتی رهبری};
			
			% نتیجه
			\node[result] (result) at (0,-3) {
				\begin{tabular}{c}
					اصلاح تدریجی ناممکن\\
					انقلاب اجتناب‌ناپذیر
				\end{tabular}
			};
			
			% پیکان‌ها
			\draw[->, >=Stealth, line width=1.5pt, gris] (c1) -- (result);
			\draw[->, >=Stealth, line width=1.5pt, gris] (c2) -- (result);
			\draw[->, >=Stealth, line width=1.5pt, gris] (c3) -- (result);
			\draw[->, >=Stealth, line width=1.5pt, gris] (c4) -- (result);
% ══════════════════════════════════════════════════════════════════════════════
%                    ادامه فصل ۲ (از نمودار اصلاح‌ناپذیری)
% ══════════════════════════════════════════════════════════════════════════════

% ادامه پیکان‌ها
\draw[->, >=Stealth, line width=1.5pt, gris] (c5) -- (result);
\draw[->, >=Stealth, line width=1.5pt, gris] (c6) -- (result);

\end{tikzpicture}
\caption{چرا رژیم قدیم اصلاح‌ناپذیر بود؟}
\end{figure}

\subsection{تحلیل توکویل: تمرکز بدون نمایندگی}

\begin{naghlbox}[بینش کلیدی توکویل]
«رژیم قدیم همه قدرت‌های محلی و صنفی را نابود کرد، اما چیزی جایگزین نکرد. میان فرد و دولت، هیچ واسطه‌ای نماند. وقتی دولت مرکزی تضعیف شد، هیچ‌چیز نمانده بود — جز خلأ و سقوط.»
\begin{flushright}
— الکسی دو توکویل، رژیم قدیم و انقلاب (۱۸۵۶)
\end{flushright}
\end{naghlbox}

\begin{table}[H]
\centering
\caption{مقایسه فرانسه و انگلستان: چرا انگلستان انقلاب نکرد؟}
\begin{tabular}{|>{\bfseries}r|p{5cm}|p{5cm}|}
\hline
\rowcolor{bleumid}
\textbf{عامل} & \textbf{فرانسه} & \textbf{انگلستان} \\
\hline
پارلمان & مشورتی، غیرفعال از ۱۶۱۴ & قانون‌گذار، فعال \\
\hline
\rowcolor{grisclair}
اشراف & بسته، امتیازات حقوقی & باز، ادغام با بورژوازی \\
\hline
مالیات & نابرابر، استثنائات & نسبتاً برابر \\
\hline
\rowcolor{grisclair}
حکومت محلی & متمرکز، ضعیف & غیرمتمرکز، قوی \\
\hline
کلیسا & گالیکن، ثروتمند & آنگلیکن، اصلاح‌شده \\
\hline
\rowcolor{grisclair}
روشنفکران & رادیکال، بیرون از قدرت & معتدل، درون نظام \\
\hline
\end{tabular}
\end{table}

\subsection{تحلیل مارکسیستی: تعارض طبقاتی}

\begin{enghelabbox}[دیدگاه مارکسیستی (خلاصه)]
از منظر مارکسیستی (ژرژ لوفور، آلبر سوبول):
\begin{itemize}[nosep]
\item فئودالیسم مانع رشد سرمایه‌داری شده بود
\item بورژوازی قدرت اقتصادی داشت، اما قدرت سیاسی نه
\item انقلاب = پیروزی بورژوازی بر فئودالیسم
\item طبقه سوم «طبقه انقلابی» بود
\end{itemize}

\textbf{نقد}: این تحلیل تنوع درونی طبقات را نادیده می‌گیرد.
\end{enghelabbox}

\subsection{تحلیل «تجدیدنظرطلب»: فرهنگ سیاسی}

\begin{olgoobox}[دیدگاه تجدیدنظرطلب (فوره، بیکار)]
از دهه ۱۹۷۰، مورخانی مانند فرانسوا فوره استدلال کردند:
\begin{itemize}[nosep]
\item انقلاب محصول «فرهنگ سیاسی» خاص بود
\item ایده «اراده عمومی» راه سازش را بست
\item فرهنگ مطلق‌گرا: یا همه‌چیز یا هیچ‌چیز
\item انقلاب گسست بود، نه تداوم
\end{itemize}

\textbf{نقد}: این تحلیل عوامل اقتصادی و اجتماعی را کم‌اهمیت می‌کند.
\end{olgoobox}

\section{لویی شانزدهم: شاه نامناسب برای زمانه}

\begin{figure}[H]
\centering
\begin{tikzpicture}[
scale=0.95,
strength/.style={
	rectangle, draw=vertnapoleon, line width=1.5pt, fill=vertlight,
	text=black, minimum width=4.5cm, minimum height=1cm,
	align=center, font=\small, rounded corners=3pt
},
weakness/.style={
	rectangle, draw=rougerevolution, line width=1.5pt, fill=rougelight,
	text=black, minimum width=4.5cm, minimum height=1cm,
	align=center, font=\small, rounded corners=3pt
},
king/.style={
	ellipse, draw=orroyaldark, line width=2pt, fill=orroyallight,
	text=black, minimum width=4cm, minimum height=2.5cm,
	align=center, font=\bfseries
}
]
% شاه در مرکز
\node[king] (king) at (0,0) {
	\begin{tabular}{c}
		لویی شانزدهم\\
		{\scriptsize (۱۷۵۴-۱۷۹۳)}
	\end{tabular}
};

% نقاط قوت
\node[font=\bfseries, text=vertnapoleon] at (-5,3) {نقاط قوت};
\node[strength] (s1) at (-5,2) {اخلاق شخصی، پرهیزکاری};
\node[strength] (s2) at (-5,1) {علاقه به علم و صنعت};
\node[strength] (s3) at (-5,0) {حسن نیت در اصلاحات};
\node[strength] (s4) at (-5,-1) {محبوبیت اولیه};

% نقاط ضعف
\node[font=\bfseries, text=rougerevolution] at (5,3) {نقاط ضعف};
\node[weakness] (w1) at (5,2) {بی‌تصمیمی مزمن};
\node[weakness] (w2) at (5,1) {تأثیرپذیری از ماری آنتوانت};
\node[weakness] (w3) at (5,0) {ناتوانی در رهبری};
\node[weakness] (w4) at (5,-1) {عدم درک تغییرات};

% خطوط
\draw[vertnapoleon, line width=1pt] (s1.east) -- (king.west);
\draw[vertnapoleon, line width=1pt] (s2.east) -- (king.west);
\draw[vertnapoleon, line width=1pt] (s3.east) -- (king.west);
\draw[vertnapoleon, line width=1pt] (s4.east) -- (king.west);

\draw[rougerevolution, line width=1pt] (w1.west) -- (king.east);
\draw[rougerevolution, line width=1pt] (w2.west) -- (king.east);
\draw[rougerevolution, line width=1pt] (w3.west) -- (king.east);
\draw[rougerevolution, line width=1pt] (w4.west) -- (king.east);

% نتیجه
\node[rectangle, draw=gris, line width=1.5pt, fill=grisclair,
minimum width=10cm, minimum height=1cm, rounded corners=3pt]
at (0,-3) {
	شاهی که برای زمان صلح مناسب بود، نه برای بحران
};

\end{tikzpicture}
\caption{ارزیابی لویی شانزدهم}
\end{figure}

\begin{naghlbox}[لویی شانزدهم در خاطراتش]
«امروز هیچ.» (Rien)
\begin{flushright}
— یادداشت روزانه لویی شانزدهم، ۱۴ ژوئیه ۱۷۸۹ \\
{\scriptsize (او از شکار برگشته بود و «هیچ» شکار نکرده بود — بی‌خبر از سقوط باستیل)}
\end{flushright}
\end{naghlbox}

\section{ماری آنتوانت: ملکه منفور}

\begin{table}[H]
\centering
\caption{تصویر ماری آنتوانت: واقعیت و افسانه}
\begin{tabular}{|>{\bfseries}r|p{5cm}|p{5cm}|}
\hline
\rowcolor{violetlight}
\textbf{موضوع} & \textbf{تصویر عمومی (تبلیغات)} & \textbf{واقعیت تاریخی} \\
\hline
ولخرجی & «مادام کسری» & هزینه‌ها بالا، اما نه علت بحران \\
\hline
\rowcolor{grisclair}
«نان بخورند» & گفته که کیک بخورند! & هرگز این را نگفت \\
\hline
وفاداری & جاسوس اتریش & وفادار به فرانسه، اما ناشیانه \\
\hline
\rowcolor{grisclair}
نفوذ سیاسی & کنترل شاه & نفوذ واقعی، اما محدود \\
\hline
اخلاق & فاسد و بی‌بندوبار & احتمالاً بی‌گناه \\
\hline
\end{tabular}
\end{table}

\begin{enghelabbox}[«رسوایی گردنبند» (۱۷۸۵)]
یک کلاهبرداری که ملکه در آن بی‌گناه بود، اما افکار عمومی او را مقصر دانست:
\begin{itemize}[nosep]
\item کاردینال روهان فریب خورد و گردنبندی گران‌قیمت خرید
\item کلاهبرداران ادعا کردند از طرف ملکه عمل می‌کنند
\item ملکه بی‌گناه بود، اما کسی باور نکرد
\end{itemize}

این رسوایی نشان داد که اعتماد به سلطنت چقدر فرسایش یافته بود.
\end{enghelabbox}

\section{سال سرنوشت‌ساز: ۱۷۸۸}

\begin{figure}[H]
\centering
\begin{tikzpicture}[
scale=0.85,
transform shape,
event/.style={
	rectangle, draw=#1, line width=1.5pt, fill=#1,
	text=black, minimum width=3.5cm, minimum height=1.5cm,
	align=center, font=\scriptsize, rounded corners=3pt
}
]
% خط زمان
\draw[line width=2pt, gris] (0,0) -- (15,0);

% ماه‌ها
\foreach \x/\month in {0/ژان, 2/مارس, 4/مه, 6/ژوئیه, 8/اوت, 10/اکت, 12/دسا} {
	\draw[black, line width=1pt] (\x,-0.2) -- (\x,0.2);
	\node[below, font=\tiny] at (\x,-0.3) {\month};
}

% رویدادها - بالا
\node[event=rougelight] at (1,2) {
	\begin{tabular}{c}
		بحران مالی\\
		تشدید می‌شود
	\end{tabular}
};

\node[event=violetlight] at (4,2) {
	\begin{tabular}{c}
		برکناری بری‌ین\\
		بازگشت نکر
	\end{tabular}
};

\node[event=orroyallight] at (8,2) {
	\begin{tabular}{c}
		تعلیق پرداخت‌ها\\
		دولت ورشکسته
	\end{tabular}
};

\node[event=bleulight] at (12,2) {
	\begin{tabular}{c}
		اعلام فراخوان\\
		مجلس طبقات
	\end{tabular}
};

% رویدادها - پایین
\node[event=vertlight] at (2.5,-2) {
	\begin{tabular}{c}
		شورش‌های محلی\\
		گرسنگی
	\end{tabular}
};

\node[event=rougelight] at (6,-2) {
	\begin{tabular}{c}
		تگرگ ویرانگر\\
		نابودی محصول
	\end{tabular}
};

\node[event=violetlight] at (10,-2) {
	\begin{tabular}{c}
		بحث «دوبرابر»\\
		تنش سیاسی
	\end{tabular}
};

% پیکان‌ها
\draw[->, >=Stealth, line width=1pt, gris] (1,1.2) -- (1,0.3);
\draw[->, >=Stealth, line width=1pt, gris] (4,1.2) -- (4,0.3);
\draw[->, >=Stealth, line width=1pt, gris] (8,1.2) -- (8,0.3);
\draw[->, >=Stealth, line width=1pt, gris] (12,1.2) -- (12,0.3);

\draw[->, >=Stealth, line width=1pt, gris] (2.5,-1.2) -- (2.5,-0.3);
\draw[->, >=Stealth, line width=1pt, gris] (6,-1.2) -- (6,-0.3);
\draw[->, >=Stealth, line width=1pt, gris] (10,-1.2) -- (10,-0.3);

\end{tikzpicture}
\caption{رویدادهای کلیدی سال ۱۷۸۸}
\end{figure}

\section{آستانه انقلاب: زمستان ۱۷۸۸-۸۹}

\begin{noktebox}[وضعیت فرانسه در آستانه ۱۷۸۹]
\begin{itemize}[nosep]
\item \textbf{اقتصادی}: قیمت نان در اوج، بیکاری گسترده
\item \textbf{مالی}: دولت عملاً ورشکسته
\item \textbf{سیاسی}: مجلس طبقات فراخوانده شده، انتظارات بالا
\item \textbf{فکری}: ایده‌های روشنگری در اوج انتشار
\item \textbf{اجتماعی}: نفرت از امتیازات، خشم از نابرابری
\end{itemize}
\end{noktebox}

\begin{olgoobox}[جمع‌بندی: چرا انقلاب در ۱۷۸۹؟]
انقلاب نتیجه همزمانی چند عامل بود:

\begin{enumerate}[nosep]
\item \textbf{بحران ساختاری}: نظامی که نمی‌توانست اصلاح شود
\item \textbf{بحران مالی}: دولتی که نمی‌توانست بپردازد
\item \textbf{بحران معیشتی}: مردمی که نمی‌توانستند نان بخرند
\item \textbf{بحران مشروعیت}: اعتمادی که از دست رفته بود
\item \textbf{فرصت سیاسی}: مجلسی که می‌توانست کانون تغییر شود
\end{enumerate}

هیچ‌کدام به تنهایی کافی نبود. همه با هم، انقلاب را ممکن و اجتناب‌ناپذیر کردند.
\end{olgoobox}

\section{منابع فصل دوم}

\subsection{منابع اولیه}

\begin{enumerate}[nosep]
\item سی‌یس، امانوئل ژوزف. طبقه سوم چیست؟ ۱۷۸۹.
\item یانگ، آرتور. سفرنامه فرانسه. ۱۷۹۲.
\item دفاتر شکایات (Cahiers de doléances). ۱۷۸۹.
\item نکر، ژاک. گزارش به شاه درباره مالیه. ۱۷۸۱.
\end{enumerate}

\subsection{منابع ثانویه}

\begin{enumerate}[nosep]
\item Tocqueville, Alexis de. \textit{The Old Regime and the Revolution}. 1856.
\item Doyle, William. \textit{Origins of the French Revolution}. Oxford UP, 1999.
\item Furet, François. \textit{Interpreting the French Revolution}. Cambridge UP, 1981.
\item Lefebvre, Georges. \textit{The Coming of the French Revolution}. Princeton UP, 1947.
\item Hampson, Norman. \textit{A Social History of the French Revolution}. Routledge, 1963.
\item Jones, Peter. \textit{The Peasantry in the French Revolution}. Cambridge UP, 1988.
\item Kwass, Michael. \textit{Privilege and the Politics of Taxation in 18th-Century France}. Cambridge UP, 2000.
\item Mousnier, Roland. \textit{The Institutions of France under the Absolute Monarchy}. Chicago UP, 1979.
\item Price, Munro. \textit{The Fall of the French Monarchy}. Macmillan, 2002.
\item Swann, Julian. \textit{Politics and the Parlement of Paris, 1754-1774}. Cambridge UP, 1995.
\end{enumerate}

\begin{kholasebox}[جمع‌بندی فصل دوم]
در این فصل دیدیم که:

\begin{itemize}[nosep]
\item \textbf{رژیم قدیم} نظامی نابرابر با سه طبقه حقوقی بود
\item \textbf{سلطنت مطلقه} در نظریه مطلق، اما در عمل محدود بود
\item \textbf{نظام مالیاتی} ناعادلانه، ریشه خشم مردم بود
\item \textbf{بحران مالی} سلطنت را به فراخوان مجلس طبقات واداشت
\item \textbf{اصلاحات} به دلیل مقاومت ذی‌نفعان شکست خورد
\item \textbf{بحران ۱۷۸۸} همه عوامل را در یک نقطه متمرکز کرد
\end{itemize}

\textbf{در فصل بعد}: انقلاب کبیر (۱۷۸۹-۱۷۹۹) — از سقوط باستیل تا کودتای ناپلئون
\end{kholasebox}

% ══════════════════════════════════════════════════════════════════════════════
%                    پایان فصل ۲
% ══════════════════════════════════════════════════════════════════════════════

% ══════════════════════════════════════════════════════════════════════════════
%                    خط زمانی جامع - صفحه افقی
% ══════════════════════════════════════════════════════════════════════════════

\begin{landscape}
	\begin{figure}[p]
		\centering
		\begin{tikzpicture}[
			scale=0.75,
			transform shape
			]
			% خط اصلی زمان
			\draw[line width=3pt, gris] (0,0) -- (24,0);
			
			% دوره‌ها
			% لویی چهاردهم
			\fill[orroyalmid] (0,0.4) rectangle (4,0.8);
			\node[above, font=\scriptsize\bfseries, text=orroyaldark] at (2,0.9) {لویی ۱۴};
			
			% لویی پانزدهم
			\fill[orroyallight] (4,0.4) rectangle (14,0.8);
			\node[above, font=\scriptsize\bfseries, text=orroyaldark] at (9,0.9) {لویی ۱۵};
			
			% لویی شانزدهم
			\fill[violetlight] (14,0.4) rectangle (24,0.8);
			\node[above, font=\scriptsize\bfseries, text=violetempire] at (19,0.9) {لویی ۱۶};
			
			% سال‌ها
			\foreach \x/\year in {0/۱۶۴۳, 4/۱۷۱۵, 8/۱۷۴۰, 12/۱۷۶۰, 14/۱۷۷۴, 18/۱۷۸۵, 22/۱۷۸۸, 24/۱۷۸۹} {
				\draw[black, line width=1.5pt] (\x,-0.3) -- (\x,0.3);
				\node[below, font=\tiny] at (\x,-0.4) {\year};
			}
			
			% رویدادهای بالا (سیاسی/نظامی)
			\node[rectangle, draw=bleurepublique, fill=bleulight, 
			minimum width=2cm, minimum height=0.8cm, font=\tiny, align=center]
			at (2,2.5) {ورسای\\ساخته می‌شود};
			\draw[->, >=Stealth, bleurepublique] (2,2) -- (2,0.5);
			
			\node[rectangle, draw=rougerevolution, fill=rougelight,
			minimum width=2cm, minimum height=0.8cm, font=\tiny, align=center]
			at (6,2.5) {جنگ‌های\\جانشینی};
			\draw[->, >=Stealth, rougerevolution] (6,2) -- (6,0.5);
			
			\node[rectangle, draw=rougerevolution, fill=rougelight,
			minimum width=2cm, minimum height=0.8cm, font=\tiny, align=center]
			at (10,2.5) {جنگ\\هفت‌ساله};
			\draw[->, >=Stealth, rougerevolution] (10,2) -- (10,0.5);
			
			\node[rectangle, draw=bleurepublique, fill=bleulight,
			minimum width=2cm, minimum height=0.8cm, font=\tiny, align=center]
			at (16,2.5) {جنگ استقلال\\آمریکا};
			\draw[->, >=Stealth, bleurepublique] (16,2) -- (16,0.5);
			
			\node[rectangle, draw=violetempire, fill=violetlight,
			minimum width=2cm, minimum height=0.8cm, font=\tiny, align=center]
			at (20,2.5) {مجمع اعیان\\شکست};
			\draw[->, >=Stealth, violetempire] (20,2) -- (20,0.5);
			
			\node[rectangle, draw=rougerevolution, fill=rougelight,
			minimum width=2.5cm, minimum height=0.8cm, font=\tiny\bfseries, align=center]
			at (24,2.5) {مجلس طبقات\\۵ مه ۱۷۸۹};
			\draw[->, >=Stealth, rougerevolution, line width=2pt] (24,2) -- (24,0.5);
			
			% رویدادهای پایین (اقتصادی/اجتماعی/فکری)
			\node[rectangle, draw=vertnapoleon, fill=vertlight,
			minimum width=2cm, minimum height=0.8cm, font=\tiny, align=center]
			at (4,-2.5) {مرگ لویی ۱۴\\بحران مالی};
			\draw[->, >=Stealth, vertnapoleon] (4,-2) -- (4,-0.5);
			
			\node[rectangle, draw=vertnapoleon, fill=vertlight,
			minimum width=2.5cm, minimum height=0.8cm, font=\tiny, align=center]
			at (8,-2.5) {دایره‌المعارف\\۱۷۵۱-۷۲};
			\draw[->, >=Stealth, vertnapoleon] (8,-2) -- (8,-0.5);
			
			\node[rectangle, draw=vertnapoleon, fill=vertlight,
			minimum width=2.5cm, minimum height=0.8cm, font=\tiny, align=center]
			at (11,-2.5) {قرارداد اجتماعی\\روسو ۱۷۶۲};
			\draw[->, >=Stealth, vertnapoleon] (11,-2) -- (11,-0.5);
			
			\node[rectangle, draw=orroyaldark, fill=orroyallight,
			minimum width=2cm, minimum height=0.8cm, font=\tiny, align=center]
			at (14.5,-2.5) {اصلاحات تورگو\\شکست};
			\draw[->, >=Stealth, orroyaldark] (14.5,-2) -- (14.5,-0.5);
			
			\node[rectangle, draw=rougerevolution, fill=rougelight,
			minimum width=2cm, minimum height=0.8cm, font=\tiny, align=center]
			at (18,-2.5) {رسوایی\\گردنبند};
			\draw[->, >=Stealth, rougerevolution] (18,-2) -- (18,-0.5);
			
			\node[rectangle, draw=rougerevolution, fill=rougelight,
			minimum width=2cm, minimum height=0.8cm, font=\tiny, align=center]
			at (22,-2.5) {ورشکستگی\\زمستان سخت};
			\draw[->, >=Stealth, rougerevolution] (22,-2) -- (22,-0.5);
			
			% راهنما
			\node[right, font=\footnotesize] at (0,-4) {
				\begin{tabular}{r@{\hspace{3pt}}l@{\hspace{15pt}}r@{\hspace{3pt}}l@{\hspace{15pt}}r@{\hspace{3pt}}l}
					\textcolor{bleurepublique}{$\blacksquare$} & سیاسی/نهادی &
					\textcolor{rougerevolution}{$\blacksquare$} & بحران/جنگ &
					\textcolor{vertnapoleon}{$\blacksquare$} & فکری/اقتصادی \\
				\end{tabular}
			};
			
			% عنوان
			\node[font=\large\bfseries, text=bleurepublique] at (12,4) {
				خط زمانی رژیم قدیم: از لویی چهاردهم تا آستانه انقلاب
			};
			
		\end{tikzpicture}
		\caption{خط زمانی جامع رژیم قدیم (۱۶۴۳-۱۷۸۹)}
	\end{figure}
\end{landscape}
% ══════════════════════════════════════════════════════════════════════════════
%                    نقشه مفهومی جامع - صفحه افقی
% ══════════════════════════════════════════════════════════════════════════════

\begin{landscape}
	\begin{figure}[p]
		\centering
		\begin{tikzpicture}[
			scale=0.7,
			transform shape,
			mainbox/.style={
				rectangle, draw=bleurepublique, line width=2pt, fill=bleulight,
				text=black, minimum width=4cm, minimum height=2cm,
				align=center, font=\small\bfseries, rounded corners=5pt
			},
			catbox/.style={
				rectangle, draw=#1, line width=1.5pt, fill=#1,
				text=black, minimum width=3.5cm, minimum height=1.5cm,
				align=center, font=\footnotesize, rounded corners=3pt
			},
			subbox/.style={
				rectangle, draw=#1, line width=1pt, fill=white,
				text=black, minimum width=2.8cm, minimum height=0.8cm,
				align=center, font=\scriptsize, rounded corners=2pt
			}
			]
			% عنوان
			\node[font=\Large\bfseries, text=bleurepublique] at (12,10) {
				ریشه‌های انقلاب فرانسه: نقشه مفهومی
			};
			
			% مرکز
			\node[mainbox] (center) at (12,5) {
				\begin{tabular}{c}
					انقلاب ۱۷۸۹\\
					چرا رخ داد؟
				\end{tabular}
			};
			
			% ───────────── ساختار اجتماعی (چپ بالا) ─────────────
			\node[catbox=orroyallight] (social) at (3,8) {
				\begin{tabular}{c}
					\textbf{ساختار اجتماعی}\\
					نابرابری حقوقی
				\end{tabular}
			};
			
			\node[subbox=orroyaldark] at (0,6.5) {روحانیون: معاف};
			\node[subbox=orroyaldark] at (3,6.5) {اشراف: ممتاز};
			\node[subbox=orroyaldark] at (6,6.5) {طبقه سوم: محروم};
			
			\draw[->, >=Stealth, line width=1.5pt, orroyaldark] (social) -- (center);
			
			% ───────────── ساختار سیاسی (چپ پایین) ─────────────
			\node[catbox=violetlight] (politic) at (3,2) {
				\begin{tabular}{c}
					\textbf{ساختار سیاسی}\\
					سلطنت مطلقه ناکارآمد
				\end{tabular}
			};
			
			\node[subbox=violetempire] at (0,0.5) {تمرکز بدون نمایندگی};
			\node[subbox=violetempire] at (3,0.5) {پارلمان‌ها: مقاومت};
			\node[subbox=violetempire] at (6,0.5) {ضعف رهبری};
			
			\draw[->, >=Stealth, line width=1.5pt, violetempire] (politic) -- (center);
			
			% ───────────── ساختار اقتصادی (راست بالا) ─────────────
			\node[catbox=vertlight] (econ) at (21,8) {
				\begin{tabular}{c}
					\textbf{ساختار اقتصادی}\\
					فئودالیسم و بحران
				\end{tabular}
			};
			
			\node[subbox=vertnapoleon] at (18,6.5) {مالیات ناعادلانه};
			\node[subbox=vertnapoleon] at (21,6.5) {حقوق فئودالی};
			\node[subbox=vertnapoleon] at (24,6.5) {ورشکستگی دولت};
			
			\draw[->, >=Stealth, line width=1.5pt, vertnapoleon] (econ) -- (center);
			
			% ───────────── بحران ۱۷۸۸ (راست پایین) ─────────────
			\node[catbox=rougelight] (crisis) at (21,2) {
				\begin{tabular}{c}
					\textbf{بحران ۱۷۸۸}\\
					شتاب‌دهنده نهایی
				\end{tabular}
			};
			
			\node[subbox=rougerevolution] at (18,0.5) {قحطی و گرسنگی};
			\node[subbox=rougerevolution] at (21,0.5) {فراخوان مجلس};
			\node[subbox=rougerevolution] at (24,0.5) {انتظارات بالا};
			
			\draw[->, >=Stealth, line width=1.5pt, rougerevolution] (crisis) -- (center);
			
			% ───────────── فرهنگ و ایده‌ها (بالای مرکز) ─────────────
			\node[catbox=bleulight] (ideas) at (12,8.5) {
				\begin{tabular}{c}
					\textbf{ایده‌ها و فرهنگ}\\
					روشنگری و افکار عمومی
				\end{tabular}
			};
			
			\draw[->, >=Stealth, line width=1.5pt, bleurepublique] (ideas) -- (center);
			
			% ───────────── نتیجه (پایین مرکز) ─────────────
			\node[rectangle, draw=rougerevolution, line width=2pt, fill=rougelight,
			minimum width=8cm, minimum height=1.5cm, font=\small\bfseries,
			rounded corners=5pt]
			at (12,1.5) {
				\begin{tabular}{c}
					همزمانی بحران‌ها + ناتوانی اصلاح = انقلاب
				\end{tabular}
			};
			
			\draw[->, >=Stealth, line width=2pt, rougerevolution] (center) -- (12,2.3);
			
		\end{tikzpicture}
		\caption{نقشه مفهومی جامع: ریشه‌های انقلاب فرانسه}
	\end{figure}
\end{landscape}

\end{document}