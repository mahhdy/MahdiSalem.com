% ═══════════════════════════════════════════════════════════════════════════════
%                    تاریخ تحولات فرانسه - فصل [X]
% ═══════════════════════════════════════════════════════════════════════════════

\documentclass[12pt,a4paper]{book}

% ─────────────────────────── پکیج‌ها ───────────────────────────
\usepackage{amsmath,amssymb}
\usepackage{geometry}
\geometry{top=2.5cm, bottom=2.5cm, left=2cm, right=2.5cm, headheight=15pt}
\usepackage{graphicx}
\usepackage{array,booktabs,longtable,multirow,colortbl}
\usepackage{xcolor}
\usepackage{tikz}
\usetikzlibrary{shapes.geometric, arrows.meta, positioning, calc, backgrounds, 
	fit, decorations.pathmorphing, shadows, patterns}
\usepackage{pgfplots}
\pgfplotsset{compat=1.18}
\usepackage{tcolorbox}
\tcbuselibrary{skins,breakable}
\usepackage{enumitem}
\usepackage{fancyhdr}
\usepackage{pdflscape}
\usepackage{setspace}
\usepackage{titlesec}
\usepackage{float}
\usepackage{pdfpages}
\usepackage{pdflscape}  % برای صفحات landscape
\usepackage{hyperref}

% ─────────────────────────── رنگ‌ها ───────────────────────────
\definecolor{bleurepublique}{RGB}{0, 35, 149}
\definecolor{rougerevolution}{RGB}{237, 41, 57}
\definecolor{orroyal}{RGB}{255, 215, 0}
\definecolor{vertnapoleon}{RGB}{0, 100, 0}
\definecolor{violetempire}{RGB}{128, 0, 128}
\definecolor{fondclair}{RGB}{255, 253, 240}
\definecolor{gris}{RGB}{128, 128, 128}
\definecolor{grisclair}{RGB}{245, 245, 245}
\definecolor{noirsombre}{RGB}{30, 30, 30}

% رنگ‌های کمکی
\definecolor{bleulight}{RGB}{230, 235, 250}
\definecolor{rougelight}{RGB}{253, 235, 237}
\definecolor{vertlight}{RGB}{235, 250, 235}
\definecolor{violetlight}{RGB}{245, 235, 250}
\definecolor{orroyallight}{RGB}{255, 250, 230}
\definecolor{grislight}{RGB}{248, 248, 248}
\definecolor{bleumid}{RGB}{180, 195, 230}
\definecolor{rougemid}{RGB}{245, 180, 185}
\definecolor{vertmid}{RGB}{180, 220, 180}
\definecolor{violetmid}{RGB}{210, 180, 220}
\definecolor{orroyalmid}{RGB}{255, 240, 180}
\definecolor{orroyaldark}{RGB}{200, 170, 0}

% ─────────────────────────── فونت فارسی ───────────────────────────
\usepackage{fontspec}
\setmainfont{Vazirmatn}
\usepackage{xepersian}
\settextfont{Vazirmatn}
\setdigitfont{Vazirmatn}

% ─────────────────────────── هایپرلینک ───────────────────────────
\hypersetup{
	colorlinks=true,
	linkcolor=bleurepublique,
	urlcolor=bleurepublique,
	citecolor=vertnapoleon
}

% ─────────────────────────── کادرها ───────────────────────────
\newtcolorbox{kholasebox}[1][]{enhanced,breakable,colback=bleulight,
	colframe=bleurepublique,coltitle=white,fonttitle=\bfseries\large,
	title={#1},boxrule=2pt,arc=4pt,left=10pt,right=10pt,top=10pt,bottom=10pt,
	drop shadow={opacity=0.3}}

\newtcolorbox{naghlbox}[1][]{enhanced,breakable,colback=orroyallight,
	colframe=orroyaldark,coltitle=black,fonttitle=\bfseries,title={#1},
	boxrule=1.5pt,arc=3pt,borderline west={4pt}{0pt}{orroyal},
	left=15pt,right=10pt,top=8pt,bottom=8pt}

\newtcolorbox{olgoobox}[1][]{enhanced,breakable,colback=vertlight,
	colframe=vertnapoleon,coltitle=white,fonttitle=\bfseries,title={#1},
	boxrule=1.5pt,arc=4pt,left=10pt,right=10pt,top=8pt,bottom=8pt,
	before upper={\parindent15pt}}

\newtcolorbox{enghelabbox}[1][]{enhanced,breakable,colback=rougelight,
	colframe=rougerevolution,coltitle=white,fonttitle=\bfseries,title={#1},
	boxrule=2pt,arc=4pt,left=10pt,right=10pt,top=8pt,bottom=8pt}

\newtcolorbox{empirebox}[1][]{enhanced,breakable,colback=violetlight,
	colframe=violetempire,coltitle=white,fonttitle=\bfseries,title={#1},
	boxrule=1.5pt,arc=4pt,left=10pt,right=10pt,top=8pt,bottom=8pt}

\newtcolorbox{noktebox}[1][]{enhanced,colback=grisclair,colframe=gris,
	fonttitle=\bfseries,title={#1},boxrule=1pt,arc=3pt,left=8pt,right=8pt}

% ─────────────────────────── صفحه‌آرایی ───────────────────────────
\pagestyle{fancy}
\fancyhf{}
\fancyhead[RO]{\leftmark}
\fancyhead[LE]{\rightmark}
\fancyfoot[C]{\thepage}
\renewcommand{\headrulewidth}{1pt}
\renewcommand{\footrulewidth}{0.5pt}
\setstretch{1.5}

\titleformat{\chapter}[display]
{\normalfont\huge\bfseries\color{bleurepublique}}
{\chaptertitlename\ \thechapter}{20pt}{\Huge}
\titleformat{\section}
{\normalfont\Large\bfseries\color{bleurepublique}}{\thesection}{1em}{}
\titleformat{\subsection}
{\normalfont\large\bfseries\color{bleurepublique}}{\thesubsection}{1em}{}

% ═══════════════════════════════════════════════════════════════════════════════
\begin{document}

%══════════════════════════════════════════════════════════════════════════════
% فصل ۱۰: نتیجه‌گیری و الگوها
%══════════════════════════════════════════════════════════════════════════════

\chapter{نتیجه‌گیری و الگوها}
\label{chap:conclusion}

\begin{kholasebox}
	\textbf{خلاصه فصل:}
	
	این فصل پایانی، ۲۳۵ سال تاریخ سیاسی فرانسه را جمع‌بندی می‌کند و الگوهای کلان تحول سیاسی را استخراج می‌نماید. از سقوط باستیل در ۱۷۸۹ تا چالش‌های جمهوری پنجم در ۲۰۲۴، فرانسه آزمایشگاهی بوده برای تقریباً هر شکل ممکن حکومت: سلطنت مطلقه، سلطنت مشروطه، جمهوری، امپراتوری، دیکتاتوری، و دموکراسی ریاستی.
	
	\textbf{پرسش‌های نهایی:}
	\begin{itemize}[nosep,rightmargin=0pt]
		\item آیا فرانسه سرانجام به ثبات رسیده است؟
		\item چه درس‌هایی از این تاریخ پرفراز و نشیب می‌توان گرفت؟
		\item «استثنای فرانسوی» چه جایگاهی در جهان امروز دارد؟
		\item آینده جمهوری پنجم چیست؟
	\end{itemize}
\end{kholasebox}

%──────────────────────────────────────────────────────────────────────────────
\section{سفری که طی کردیم}
%──────────────────────────────────────────────────────────────────────────────

\subsection{مروری بر مسیر}

\begin{tikzpicture}[
	every node/.style={font=\small},
	era/.style={rectangle, rounded corners, minimum width=2.8cm, minimum height=1.5cm, align=center}
	]
	% Title
	\node[font=\bfseries\large] at (7,9) {۲۳۵ سال تاریخ سیاسی فرانسه در یک نگاه};
	
	% Timeline
	\draw[ultra thick, black] (0,7) -- (14,7);
	
	% Era blocks
	\node[era, draw=orroyaldark, fill=orroyallight] at (0.8,5.5) {\footnotesize رژیم قدیم\\تا ۱۷۸۹};
	
	\node[era, draw=rougerevolution, fill=rougelight] at (2.8,5.5) {\footnotesize انقلاب\\۱۷۸۹-۹۹};
	
	\node[era, draw=violetempire, fill=violetlight] at (4.8,5.5) {\footnotesize ناپلئون\\۱۷۹۹-۱۸۱۵};
	
	\node[era, draw=gris, fill=grisclair] at (6.8,5.5) {\footnotesize قرن ناآرام\\۱۸۱۵-۱۸۷۰};
	
	\node[era, draw=bleurepublique, fill=bleulight] at (8.8,5.5) {\footnotesize جمهوری سوم\\۱۸۷۰-۱۹۴۰};
	
	\node[era, draw=rougerevolution, fill=rougelight] at (10.8,5.5) {\footnotesize ویشی/چهارم\\۱۹۴۰-۵۸};
	
	\node[era, draw=bleurepublique, fill=bleumid, text=white] at (12.8,5.5) {\footnotesize جمهوری پنجم\\۱۹۵۸-...};
	
	% Key characteristics below
	\node[font=\tiny, align=center] at (0.8,3.8) {سلطنت مطلقه\\امتیازات\\بحران};
	\node[font=\tiny, align=center] at (2.8,3.8) {گسست\\حقوق بشر\\ترور};
	\node[font=\tiny, align=center] at (4.8,3.8) {تثبیت\\قانون\\جنگ};
	\node[font=\tiny, align=center] at (6.8,3.8) {۵ رژیم\\۳ انقلاب\\طبقه کارگر};
	\node[font=\tiny, align=center] at (8.8,3.8) {لائیسیته\\جنگ بزرگ\\دریفوس};
	\node[font=\tiny, align=center] at (10.8,3.8) {اشغال\\مقاومت\\استعمارزدایی};
	\node[font=\tiny, align=center] at (12.8,3.8) {ثبات\\ریاست قوی\\اروپا};
	
	% Arrow of progress?
	\draw[->, ultra thick, vertnapoleon] (1,2) -- (13,2);
	\node[font=\footnotesize, vertnapoleon] at (7,1.5) {آیا این پیشرفت است؟ یا فقط تغییر؟};
	
\end{tikzpicture}

\subsection{آمار کلیدی}

\begin{table}[htbp]
	\centering
	\caption{تاریخ سیاسی فرانسه در اعداد (۱۷۸۹-۲۰۲۴)}
	\label{tab:france-numbers}
	\begin{tabular}{|r|c|p{6cm}|}
		\hline
		\rowcolor{bleumid}
		\textcolor{white}{\textbf{شاخص}} & \textcolor{white}{\textbf{عدد}} & \textcolor{white}{\textbf{توضیح}} \\
		\hline
		تعداد رژیم‌ها & ۱۵ & از سلطنت مشروطه ۱۷۸۹ تا جمهوری پنجم \\
		\hline
		\rowcolor{bleulight}
		تعداد قوانین اساسی & ۱۵+ & نوشته‌شده یا اصلاح‌شده اساسی \\
		\hline
		تعداد جمهوری‌ها & ۵ & اول تا پنجم \\
		\hline
		\rowcolor{bleulight}
		تعداد امپراتوری‌ها & ۲ & ناپلئون اول و سوم \\
		\hline
		تعداد سلطنت‌ها (پس از انقلاب) & ۳ & بازگشت، ژوئیه، (ویشی شبه‌سلطنتی) \\
		\hline
		\rowcolor{bleulight}
		انقلاب‌های بزرگ & ۴ & ۱۷۸۹، ۱۸۳۰، ۱۸۴۸، ۱۸۷۱ (کمون) \\
		\hline
		کودتاها & ۳ & ۱۷۹۹، ۱۸۵۱، ۱۹۵۸ (شبه‌کودتا) \\
		\hline
		\rowcolor{bleulight}
		شکست‌های نظامی تغییردهنده & ۴ & ۱۸۱۴/۱۵، ۱۸۷۰، ۱۹۴۰ \\
		\hline
		طولانی‌ترین رژیم & ۷۰ سال & جمهوری سوم \\
		\hline
		\rowcolor{bleulight}
		کوتاه‌ترین رژیم & ۱۰۰ روز & صد روز ناپلئون \\
		\hline
		روسای جمهور جمهوری پنجم & ۸ & دوگل تا ماکرون \\
		\hline
	\end{tabular}
\end{table}

%──────────────────────────────────────────────────────────────────────────────
\section{هفت الگوی کلان}
%──────────────────────────────────────────────────────────────────────────────

از بررسی تاریخ فرانسه، هفت الگوی کلان قابل استخراج است:

\subsection{الگوی اول: دیالکتیک انقلاب و واکنش}

\begin{tikzpicture}[
	every node/.style={font=\small},
	phase/.style={circle, draw=rougerevolution, fill=rougelight, 
		minimum size=2cm, align=center}
	]
	% Title
	\node[font=\bfseries\large] at (7,7) {الگوی اول: دیالکتیک انقلاب-واکنش};
	
	% Cycle
	\node[phase] (rev) at (3,4) {\textbf{انقلاب}\\گسست\\تغییر};
	
	\node[phase, draw=orroyaldark, fill=orroyallight] (react) at (11,4) {\textbf{واکنش}\\بازگشت\\نظم};
	
	\node[phase, draw=bleurepublique, fill=bleulight] (synth) at (7,1) {\textbf{ترکیب}\\سازش\\نهادینگی};
	
	% Arrows
	\draw[->, ultra thick, rougerevolution] (rev) -- (react) 
	node[midway, above] {افراط → ترس};
	\draw[->, ultra thick, orroyaldark] (react) -- (synth) 
	node[midway, right] {سختی → شکست};
	\draw[->, ultra thick, bleurepublique] (synth) -- (rev) 
	node[midway, left] {فرسایش → بحران};
	
	% Examples
	\node[font=\footnotesize, align=center] at (3,2.5) {۱۷۸۹\\۱۸۳۰\\۱۸۴۸};
	\node[font=\footnotesize, align=center] at (11,2.5) {۱۷۹۹\\۱۸۱۵\\۱۸۵۱};
	\node[font=\footnotesize, align=center] at (7,-0.5) {جمهوری سوم\\جمهوری پنجم};
	
\end{tikzpicture}

\begin{olgoobox}
	\textbf{الگوی اول: دیالکتیک انقلاب و واکنش}
	
	تاریخ فرانسه را می‌توان به عنوان نوسان مداوم میان دو قطب دید:
	
	\begin{itemize}
		\item \textbf{قطب انقلابی:} تغییر رادیکال، گسست با گذشته، آرمان‌گرایی
		\item \textbf{قطب واکنشی:} بازگشت به نظم، تداوم، واقع‌گرایی
	\end{itemize}
	
	هر انقلاب واکنشی می‌آورد، هر واکنشی زمینه انقلاب بعدی را فراهم می‌کند—تا سرانجام ترکیبی پایدار یافته شود.
	
	\textbf{نمونه‌ها:}
	\begin{itemize}[nosep]
		\item ۱۷۸۹ (انقلاب) → ۱۷۹۹ (ناپلئون) → بازگشت → ۱۸۳۰ → ...
		\item جمهوری سوم و پنجم = ترکیب‌های نسبتاً پایدار
	\end{itemize}
\end{olgoobox}

\subsection{الگوی دوم: جستجوی تعادل قدرت}

\begin{table}[htbp]
	\centering
	\caption{الگوی دوم: نوسان در تعادل قوا}
	\label{tab:power-balance}
	\begin{tabular}{|r|c|c|c|}
		\hline
		\rowcolor{bleumid}
		\textcolor{white}{\textbf{رژیم}} & \textcolor{white}{\textbf{قوه مجریه}} & \textcolor{white}{\textbf{قوه مقننه}} & \textcolor{white}{\textbf{نتیجه}} \\
		\hline
		سلطنت مطلقه & بسیار قوی & ضعیف/غایب & انقلاب \\
		\hline
		\rowcolor{bleulight}
		کنوانسیون & ضعیف & بسیار قوی & بی‌ثباتی \\
		\hline
		ناپلئون & بسیار قوی & ضعیف & ثبات موقت \\
		\hline
		\rowcolor{bleulight}
		جمهوری سوم & ضعیف & قوی & بی‌ثباتی دولت‌ها \\
		\hline
		جمهوری چهارم & ضعیف & قوی & بی‌ثباتی شدید \\
		\hline
		\rowcolor{bleulight}
		جمهوری پنجم & قوی & متوسط & ثبات نسبی \\
		\hline
	\end{tabular}
\end{table}

\begin{noktebox}
	\textbf{درس الگوی دوم:}
	
	نه مجریه بسیار قوی (استبداد) و نه مقننه بسیار قوی (بی‌ثباتی) به تنهایی کار نمی‌کند. جمهوری پنجم تلاش کرده تعادلی بیابد—هرچند منتقدان می‌گویند کفه به نفع رئیس‌جمهور سنگین است.
\end{noktebox}

\subsection{الگوی سوم: مسئله مشروعیت}

\begin{tikzpicture}[
	every node/.style={font=\small},
	source/.style={rectangle, rounded corners, minimum width=3.5cm, minimum height=1.5cm, align=center}
	]
	% Title
	\node[font=\bfseries\large] at (7,8) {الگوی سوم: منابع متعارض مشروعیت};
	
	% Sources
	\node[source, draw=orroyaldark, fill=orroyallight] (divine) at (2,5.5) 
	{\textbf{الهی/سنتی}\\«حق الهی شاهان»\\وراثت، سنت};
	
	\node[source, draw=bleurepublique, fill=bleulight] (popular) at (7,5.5) 
	{\textbf{مردمی/دموکراتیک}\\«حاکمیت ملت»\\انتخابات، رأی};
	
	\node[source, draw=violetempire, fill=violetlight] (charisma) at (12,5.5) 
	{\textbf{کاریزماتیک/قهرمانی}\\«ناجی ملت»\\شخصیت، عمل};
	
	% Conflict zone
	\node[rectangle, draw=rougerevolution, fill=rougelight, rounded corners,
	minimum width=10cm, minimum height=1.5cm, align=center] at (7,2.5) {
		\textbf{منطقه تعارض:} تاریخ فرانسه = نبرد میان این منابع\\
		جمهوری پنجم = ترکیب مردمی + کاریزماتیک
	};
	
	% Arrows to conflict
	\draw[->, thick, orroyaldark] (divine) -- (5,3.3);
	\draw[->, thick, bleurepublique] (popular) -- (7,3.3);
	\draw[->, thick, violetempire] (charisma) -- (9,3.3);
	
	% Historical examples
	\node[font=\footnotesize] at (2,4) {لویی‌ها، شارل دهم};
	\node[font=\footnotesize] at (7,4) {جمهوری‌ها};
	\node[font=\footnotesize] at (12,4) {ناپلئون‌ها، دوگل};
	
\end{tikzpicture}

\subsection{الگوی چهارم: تنش مرکز-پیرامون}

\begin{olgoobox}
	\textbf{الگوی چهارم: پاریس و «بقیه»}
	
	یکی از تنش‌های دائمی تاریخ فرانسه، رابطه میان پایتخت و شهرستان‌هاست:
	
	\textbf{پاریس:}
	\begin{itemize}[nosep]
		\item مرکز انقلاب‌ها (۱۷۸۹، ۱۸۳۰، ۱۸۴۸، ۱۸۷۱، ۱۹۶۸)
		\item رادیکال‌تر، تغییرخواه‌تر
		\item طبقه کارگر متمرکز
		\item روشنفکران و رسانه‌ها
	\end{itemize}
	
	\textbf{شهرستان‌ها:}
	\begin{itemize}[nosep]
		\item محافظه‌کارتر، ثبات‌خواه‌تر
		\item پایگاه رأی راست و میانه
		\item واندی (۱۷۹۳)، رأی به لویی-ناپلئون (۱۸۴۸)، جلیقه‌زردها (۲۰۱۸)
	\end{itemize}
	
	\textbf{پارادوکس:} پاریس انقلاب می‌کند، شهرستان رأی می‌دهد—و معمولاً رأی شهرستان تعیین‌کننده است.
\end{olgoobox}

\subsection{الگوی پنجم: مسئله اجتماعی حل‌نشده}

\begin{tikzpicture}[
	every node/.style={font=\small},
	period/.style={rectangle, rounded corners, draw=vertnapoleon, fill=vertlight,
		minimum width=3.5cm, minimum height=1.5cm, align=center}
	]
	% Title
	\node[font=\bfseries\large] at (7,7.5) {الگوی پنجم: بازگشت مداوم «مسئله اجتماعی»};
	
	% Periods
	\node[period] (p1) at (2,5) {\textbf{۱۷۸۹-۱۸۴۸}\\فقر روستایی\\بحران معیشت};
	
	\node[period] (p2) at (7,5) {\textbf{۱۸۴۸-۱۹۳۶}\\مسئله کارگری\\صنعتی‌شدن};
	
	\node[period] (p3) at (12,5) {\textbf{۱۹۳۶-۱۹۸۰}\\دولت رفاه\\حل نسبی};
	
	\node[period, fill=rougelight, draw=rougerevolution] (p4) at (7,2) {\textbf{۱۹۸۰-حال}\\بازگشت نابرابری\\بیکاری، حاشیه‌نشینی};
	
	% Arrows
	\draw[->, thick] (p1) -- (p2);
	\draw[->, thick] (p2) -- (p3);
	\draw[->, thick] (p3) -- (p4);
	
	% Events
	\node[font=\footnotesize, rougerevolution] at (2,3.5) {ژوئن ۱۸۴۸};
	\node[font=\footnotesize, rougerevolution] at (7,3.5) {کمون، جبهه مردمی};
	\node[font=\footnotesize, rougerevolution] at (12,3.5) {مه ۶۸};
	\node[font=\footnotesize, rougerevolution] at (7,0.5) {جلیقه‌زردها، اصلاح بازنشستگی};
	
\end{tikzpicture}

\subsection{الگوی ششم: بحران خارجی = تغییر داخلی}

\begin{table}[htbp]
	\centering
	\caption{الگوی ششم: تأثیر رویدادهای خارجی}
	\label{tab:external-events}
	\begin{tabular}{|c|p{4cm}|p{5cm}|}
		\hline
		\rowcolor{rougemid}
		\textcolor{white}{\textbf{سال}} & \textcolor{white}{\textbf{رویداد خارجی}} & \textcolor{white}{\textbf{تأثیر داخلی}} \\
		\hline
		۱۷۹۲ & جنگ با اروپا & رادیکالیزاسیون انقلاب \\
		\hline
		\rowcolor{rougelight}
		۱۸۱۴-۱۵ & شکست ناپلئون & بازگشت بوربون‌ها \\
		\hline
		۱۸۷۰ & جنگ پروس & جمهوری سوم، کمون \\
		\hline
		\rowcolor{rougelight}
		۱۹۱۴-۱۸ & جنگ جهانی اول & اتحاد مقدس، تحول اجتماعی \\
		\hline
		۱۹۴۰ & شکست از آلمان & ویشی، مقاومت \\
		\hline
		\rowcolor{rougelight}
		۱۹۵۴-۶۲ & جنگ‌های استعماری & سقوط جمهوری چهارم \\
		\hline
		۱۹۸۹ & پایان جنگ سرد & تحول سیاست خارجی \\
		\hline
	\end{tabular}
\end{table}

\subsection{الگوی هفتم: نقش شخصیت‌ها}

\begin{tikzpicture}[
	every node/.style={font=\small},
	person/.style={rectangle, rounded corners, draw=violetempire, fill=violetlight,
		minimum width=2.8cm, minimum height=1.8cm, align=center}
	]
	% Title
	\node[font=\bfseries\large] at (7,8) {الگوی هفتم: شخصیت‌های تاریخ‌ساز};
	
	% Persons
	\node[person] (n1) at (1.5,5.5) {\textbf{ناپلئون}\\ساختن از\\ویرانه};
	
	\node[person] (tt) at (4.5,5.5) {\textbf{تی‌یر}\\نجات از\\بحران};
	
	\node[person] (clem) at (7.5,5.5) {\textbf{کلمانسو}\\پیروزی در\\جنگ};
	
	\node[person] (dg) at (10.5,5.5) {\textbf{دوگل}\\بازسازی\\ملت};
	
	\node[person] (mit) at (13.5,5.5) {\textbf{میتران}\\نهادینه‌سازی\\چرخش};
	
	% Common trait
	\node[rectangle, draw=bleurepublique, fill=bleulight, rounded corners,
	minimum width=12cm, minimum height=1.5cm, align=center] at (7.5,2.5) {
		\textbf{ویژگی مشترک:} توانایی ترکیب اقتدار با مشروعیت\\
		درک لحظه تاریخی + اراده عمل + سازش حداقلی
	};
	
\end{tikzpicture}

\begin{naghlbox}
	«مردان بزرگ تاریخ را نمی‌سازند، اما در لحظات سرنوشت‌ساز می‌توانند جهت آن را تغییر دهند.»
	
	\hfill --- \textit{مارک بلوک}
\end{naghlbox}

%──────────────────────────────────────────────────────────────────────────────
\section{پرسش‌های باز}
%──────────────────────────────────────────────────────────────────────────────

\subsection{آیا جمهوری پنجم پایدار خواهد ماند؟}

\begin{tikzpicture}[
	every node/.style={font=\small},
	factor/.style={rectangle, rounded corners, minimum width=4.5cm, minimum height=1.5cm, align=center}
	]
	% Title
	\node[font=\bfseries\large] at (7,8) {چشم‌انداز جمهوری پنجم: عوامل مثبت و منفی};
	
	% Positive factors
	\node[font=\bfseries, vertnapoleon] at (3,6.5) {عوامل ثبات};
	
	\node[factor, draw=vertnapoleon, fill=vertlight] at (3,5) {طول عمر\\۶۶ سال و همچنان ادامه};
	\node[factor, draw=vertnapoleon, fill=vertlight] at (3,3) {چرخش موفق\\چپ و راست متعدد};
	\node[factor, draw=vertnapoleon, fill=vertlight] at (3,1) {انعطاف‌پذیری\\۲۴ اصلاحیه، همزیستی};
	
	% Negative factors
	\node[font=\bfseries, rougerevolution] at (11,6.5) {عوامل تهدید};
	
	\node[factor, draw=rougerevolution, fill=rougelight] at (11,5) {صعود راست افراطی\\۴۰٪+ آرا};
	\node[factor, draw=rougerevolution, fill=rougelight] at (11,3) {بحران نمایندگی\\بی‌اعتمادی به سیاست};
	\node[factor, draw=rougerevolution, fill=rougelight] at (11,1) {تنش اجتماعی\\نابرابری رو به رشد};
	
	% Question mark
	\node[font=\Huge] at (7,3) {?};
	
\end{tikzpicture}

\begin{noktebox}
	\textbf{سناریوهای ممکن:}
	
	\begin{enumerate}[nosep]
		\item \textbf{تداوم:} جمهوری پنجم با اصلاحات تدریجی ادامه می‌یابد
		\item \textbf{تحول:} جمهوری «ششم» با تغییرات اساسی (پارلمانی‌تر؟)
		\item \textbf{بحران:} پیروزی راست افراطی و دگرگونی ماهیت نظام
		\item \textbf{فروپاشی:} بحران شدید (اقتصادی؟ اجتماعی؟) و گسست
	\end{enumerate}
	
	اکثر تحلیلگران سناریوی اول را محتمل‌تر می‌دانند، اما تاریخ فرانسه نشان داده که غافلگیری همیشه ممکن است.
\end{noktebox}

\subsection{آیا «استثنای فرانسوی» پایان یافته؟}

\begin{table}[htbp]
	\centering
	\caption{استثنای فرانسوی: باقی‌مانده یا محوشده؟}
	\label{tab:french-exception}
	\begin{tabular}{|r|c|c|}
		\hline
		\rowcolor{bleumid}
		\textcolor{white}{\textbf{ویژگی}} & \textcolor{white}{\textbf{هنوز متمایز}} & \textcolor{white}{\textbf{همگرا با دیگران}} \\
		\hline
		لائیسیته & \checkmark & \\
		\hline
		\rowcolor{bleulight}
		دولت متمرکز & \checkmark & (تمرکززدایی) \\
		\hline
		فرهنگ اعتصاب & \checkmark & \\
		\hline
		\rowcolor{bleulight}
		نقش روشنفکران & & \checkmark (کاهش) \\
		\hline
		ایدئولوژی چپ-راست & & \checkmark (ضعیف‌شده) \\
		\hline
		\rowcolor{bleulight}
		ریاست‌جمهوری قوی & \checkmark & \\
		\hline
		ضدآمریکایی‌گرایی & & \checkmark (کاهش) \\
		\hline
	\end{tabular}
\end{table}

%──────────────────────────────────────────────────────────────────────────────
\section{درس‌ها برای مطالعه تطبیقی}
%──────────────────────────────────────────────────────────────────────────────

\begin{olgoobox}
	\textbf{درس‌هایی از تاریخ فرانسه برای دیگر کشورها}
	
	\begin{enumerate}
		\item \textbf{انقلاب لزوماً به دموکراسی نمی‌انجامد:}
		\begin{itemize}[nosep]
			\item فرانسه پس از ۱۷۸۹ به ترور، امپراتوری، و بازگشت رسید
			\item دموکراسی پایدار نیازمند زمان، نهادسازی، و فرهنگ است
		\end{itemize}
		
		\item \textbf{ثبات نیازمند سازش است:}
		\begin{itemize}[nosep]
			\item رژیم‌های ایدئولوژیک «خالص» دوام نیاوردند
			\item جمهوری سوم و پنجم = سازش‌های عملی
		\end{itemize}
		
		\item \textbf{نهادها مهم‌اند، اما کافی نیستند:}
		\begin{itemize}[nosep]
			\item قانون اساسی خوب ضامن موفقیت نیست
			\item فرهنگ سیاسی، اجماع، و رهبری هم لازم است
		\end{itemize}
		
		\item \textbf{گذشته هرگز نمی‌میرد:}
		\begin{itemize}[nosep]
			\item شکاف‌های ۱۷۸۹ هنوز در فرانسه زنده‌اند
			\item حافظه تاریخی سیاست را شکل می‌دهد
		\end{itemize}
		
		\item \textbf{هر کشور مسیر خود را دارد:}
		\begin{itemize}[nosep]
			\item الگوی فرانسه قابل کپی نیست
			\item زمینه تاریخی، فرهنگی، اجتماعی تعیین‌کننده است
		\end{itemize}
	\end{enumerate}
\end{olgoobox}

%──────────────────────────────────────────────────────────────────────────────
\section{کلام آخر}
%──────────────────────────────────────────────────────────────────────────────

\begin{naghlbox}
	«تاریخ فرانسه را نمی‌توان فهمید مگر آنکه پذیرفت که این کشور همواره در جستجوی شکلی از حکومت بوده که هم آزادی را تضمین کند و هم نظم را، هم برابری را و هم کارایی را، هم وحدت را و هم تنوع را. این جستجو هنوز ادامه دارد.»
	
	\hfill --- \textit{نتیجه‌گیری این پژوهش}
\end{naghlbox}

فرانسه در ۲۳۵ سال گذشته تقریباً هر شکل ممکن حکومت را تجربه کرده است. این کشور آزمایشگاهی بوده برای ایده‌های سیاسی مدرن: حقوق بشر، حاکمیت ملی، جمهوری، سوسیالیسم، ناسیونالیسم، فاشیسم، و دموکراسی لیبرال. برخی از این آزمایش‌ها فاجعه‌بار بودند (ترور، ویشی)، برخی ناقص ماندند (جمهوری‌های اول، دوم، چهارم)، و برخی به نتایج پایداری رسیدند (جمهوری سوم و پنجم).

جمهوری پنجم، با ۶۶ سال عمر، طولانی‌ترین رژیم پس از جمهوری سوم است. این رژیم توانسته از بحران‌های متعددی—از الجزایر تا مه ۶۸، از بحران‌های اقتصادی تا صعود راست افراطی—جان سالم به در ببرد. اما آینده، همچون همیشه، نامعلوم است.

آنچه مسلم است این است که تاریخ فرانسه به ما یادآوری می‌کند: دموکراسی نه هدیه‌ای آسمانی است و نه دستاوردی یک‌باره. دموکراسی فرآیندی است دائمی، نیازمند مراقبت، بازسازی، و گاه مبارزه.

%──────────────────────────────────────────────────────────────────────────────
\section{نمودار نهایی: سیر تحول سیاسی فرانسه}
%──────────────────────────────────────────────────────────────────────────────

\begin{landscape}
	\begin{tikzpicture}[
		every node/.style={font=\footnotesize},
		scale=0.9
		]
		% Title
		\node[font=\bfseries\Large] at (12,12) {سیر تحول سیاسی فرانسه: از سلطنت مطلقه تا جمهوری پنجم};
		
		% Axes
		\draw[->, thick] (0,0) -- (24,0) node[right] {زمان};
		\draw[->, thick] (0,0) -- (0,10) node[above] {دموکراسی};
		
		% Time labels
		\node at (0,-0.5) {۱۷۸۹};
		\node at (6,-0.5) {۱۸۴۸};
		\node at (12,-0.5) {۱۹۰۰};
		\node at (18,-0.5) {۱۹۵۸};
		\node at (24,-0.5) {۲۰۲۴};
		
		% Democracy level labels
		\node[left] at (0,2) {استبداد};
		\node[left] at (0,5) {نیمه‌دموکراسی};
		\node[left] at (0,8) {دموکراسی};
		
		% Path
		\draw[ultra thick, bleurepublique] 
		(0,1) -- % رژیم قدیم
		(0.5,4) -- % انقلاب
		(1,6) -- % جمهوری اول
		(1.5,3) -- % ترور
		(2.5,2) -- % ناپلئون
		(4,1) -- % بازگشت
		(4.5,4) -- % ۱۸۳۰
		(5.5,4.5) -- % ژوئیه
		(6,6) -- % ۱۸۴۸
		(6.5,4) -- % ژوئن
		(7.5,2) -- % ناپلئون سوم
		(9,3) -- % امپراتوری لیبرال
		(9.5,5) -- % جمهوری سوم اول
		(12,6.5) -- % تثبیت
		(15,7) -- % جنگ
		(16,2) -- % ویشی
		(17,6) -- % جمهوری چهارم
		(18,7.5) -- % جمهوری پنجم
		(24,8); % حال
		
		% Key points
		\fill[rougerevolution] (0.5,4) circle (0.15);
		\node[above] at (0.5,4.2) {۱۷۸۹};
		
		\fill[rougerevolution] (4.5,4) circle (0.15);
		\node[above] at (4.5,4.2) {۱۸۳۰};
		
		\fill[rougerevolution] (6,6) circle (0.15);
		\node[above] at (6,6.2) {۱۸۴۸};
		
		\fill[rougerevolution] (9.5,5) circle (0.15);
		\node[above] at (9.5,5.2) {۱۸۷۰};
		
		\fill[rougerevolution] (16,2) circle (0.15);
		\node[below] at (16,1.8) {۱۹۴۰};
		
		\fill[bleurepublique] (18,7.5) circle (0.15);
		\node[above] at (18,7.7) {۱۹۵۸};
		
		% Regime labels
		\node[fill=white] at (1,7) {ج. اول};
		\node[fill=white] at (2.5,1) {ناپلئون};
		\node[fill=white] at (5,5.5) {ژوئیه};
		\node[fill=white] at (6.5,7) {ج. دوم};
		\node[fill=white] at (8,2.5) {امپ. دوم};
		\node[fill=white] at (12,7.5) {جمهوری سوم};
		\node[fill=white] at (16,1) {ویشی};
		\node[fill=white] at (17.5,5.5) {ج.۴};
		\node[fill=white] at (21,8.5) {جمهوری پنجم};
		
		% Trend line
		\draw[dashed, thick, vertnapoleon] (0,1) -- (24,8);
		\node[vertnapoleon, rotate=15] at (12,5.5) {روند کلی: به سوی دموکراسی};
		
	\end{tikzpicture}
\end{landscape}

%──────────────────────────────────────────────────────────────────────────────
\section{جمع‌بندی نهایی}
%──────────────────────────────────────────────────────────────────────────────

\begin{kholasebox}
	\textbf{جمع‌بندی نهایی: تاریخ تحولات سیاسی فرانسه}
	
	\textbf{مسیر طی‌شده:}
	\begin{itemize}[nosep]
		\item از سلطنت مطلقه (قبل از ۱۷۸۹) به جمهوری دموکراتیک (۲۰۲۴)
		\item ۱۵ رژیم مختلف در ۲۳۵ سال
		\item ۴ انقلاب بزرگ، ۳ کودتا، ۴ شکست نظامی تغییردهنده
	\end{itemize}
	
	\textbf{هفت الگوی کلان:}
	\begin{enumerate}[nosep]
		\item دیالکتیک انقلاب و واکنش
		\item جستجوی تعادل قوا
		\item تعارض منابع مشروعیت
		\item تنش پاریس-شهرستان
		\item بازگشت مداوم مسئله اجتماعی
		\item تأثیر رویدادهای خارجی
		\item نقش شخصیت‌های تاریخ‌ساز
	\end{enumerate}
	
	\textbf{دستاوردهای ماندگار:}
	\begin{itemize}[nosep]
		\item اعلامیه حقوق بشر (۱۷۸۹)
		\item مفهوم شهروندی مدرن
		\item لائیسیته و جدایی دین و دولت
		\item رأی همگانی
		\item تأمین اجتماعی و دولت رفاه
		\item نظام نیمه‌ریاستی جمهوری پنجم
	\end{itemize}
	
	\textbf{چالش‌های پیش رو:}
	\begin{itemize}[nosep]
		\item صعود راست افراطی
		\item بحران نمایندگی سیاسی
		\item نابرابری اقتصادی
		\item هویت و مهاجرت
		\item جایگاه در اروپا و جهان
	\end{itemize}
	
	\textbf{پیام نهایی:}
	
	تاریخ فرانسه نشان می‌دهد که راه دموکراسی پرپیچ‌وخم است، اما رفتنی. هر نسل باید این راه را از نو طی کند، از اشتباهات گذشته بیاموزد، و آینده را بسازد. جمهوری پنجم میراث‌دار ۲۳۵ سال آزمایش و خطاست—و داوری نهایی درباره آن بر عهده آیندگان است.
\end{kholasebox}

%──────────────────────────────────────────────────────────────────────────────
\section*{کتاب‌شناسی جامع}
%──────────────────────────────────────────────────────────────────────────────
\addcontentsline{toc}{section}{کتاب‌شناسی جامع}

\subsection*{منابع عمومی تاریخ فرانسه}

\begin{itemize}[nosep]
	\item Braudel, Fernand. \textit{The Identity of France}. 2 vols. New York: Harper \& Row, 1988-1990.
	\item Furet, François. \textit{Revolutionary France, 1770-1880}. Oxford: Blackwell, 1992.
	\item Jones, Colin. \textit{The Cambridge Illustrated History of France}. Cambridge: Cambridge UP, 1994.
	\item McPhee, Peter. \textit{A Social History of France, 1789-1914}. 2nd ed. London: Palgrave, 2004.
	\item Popkin, Jeremy D. \textit{A History of Modern France}. 4th ed. London: Routledge, 2015.
	\item Price, Roger. \textit{A Concise History of France}. 3rd ed. Cambridge: Cambridge UP, 2014.
\end{itemize}

\subsection*{انقلاب و دوره ناپلئونی}

\begin{itemize}[nosep]
	\item Doyle, William. \textit{The Oxford History of the French Revolution}. 2nd ed. Oxford: Oxford UP, 2002.
	\item Furet, François. \textit{Interpreting the French Revolution}. Cambridge: Cambridge UP, 1981.
	\item Hunt, Lynn. \textit{Politics, Culture, and Class in the French Revolution}. Berkeley: UC Press, 1984.
	\item Lefebvre, Georges. \textit{The Coming of the French Revolution}. Princeton: Princeton UP, 1947.
	\item Schama, Simon. \textit{Citizens: A Chronicle of the French Revolution}. New York: Knopf, 1989.
	\item Englund, Steven. \textit{Napoleon: A Political Life}. Cambridge: Harvard UP, 2004.
\end{itemize}

\subsection*{قرن نوزدهم}

\begin{itemize}[nosep]
	\item Agulhon, Maurice. \textit{The Republican Experiment, 1848-1852}. Cambridge: Cambridge UP, 1983.
	\item Magraw, Roger. \textit{France 1815-1914: The Bourgeois Century}. Oxford: Fontana, 1983.
	\item Pinkney, David. \textit{The French Revolution of 1830}. Princeton: Princeton UP, 1972.
	\item Price, Roger. \textit{The French Second Empire: An Anatomy of Political Power}. Cambridge: Cambridge UP, 2001.
	\item Tombs, Robert. \textit{France 1814-1914}. London: Longman, 1996.
\end{itemize}

\subsection*{جمهوری سوم و جنگ‌های جهانی}

\begin{itemize}[nosep]
	\item Kedward, H.R. \textit{France and the French: La Vie en Bleu since 1900}. New York: Overlook, 2006.
	\item Mayeur, Jean-Marie, and Madeleine Rebérioux. \textit{The Third Republic from its Origins to the Great War}. Cambridge: Cambridge UP, 1984.
	\item Paxton, Robert O. \textit{Vichy France: Old Guard and New Order, 1940-1944}. New York: Columbia UP, 1972.
	\item Weber, Eugen. \textit{Peasants into Frenchmen: The Modernization of Rural France, 1870-1914}. Stanford: Stanford UP, 1976.
	\item Jackson, Julian. \textit{France: The Dark Years, 1940-1944}. Oxford: Oxford UP, 2001.
\end{itemize}

\subsection*{جمهوری چهارم و پنجم}

\begin{itemize}[nosep]
	\item Berstein, Serge. \textit{The Republic of de Gaulle, 1958-1969}. Cambridge: Cambridge UP, 1993.
	\item Cole, Alistair. \textit{French Politics and Society}. 3rd ed. London: Routledge, 2017.
	\item Gildea, Robert. \textit{France Since 1945}. 2nd ed. Oxford: Oxford UP, 2002.
	\item Hazareesingh, Sudhir. \textit{In the Shadow of the General: Modern France and the Myth of De Gaulle}. Oxford: Oxford UP, 2012.
	\item Jackson, Julian. \textit{De Gaulle}. Cambridge: Harvard UP, 2018.
	\item Rioux, Jean-Pierre. \textit{The Fourth Republic, 1944-1958}. Cambridge: Cambridge UP, 1987.
\end{itemize}

\subsection*{تحلیل سیاسی و تطبیقی}

\begin{itemize}[nosep]
	\item Arendt, Hannah. \textit{On Revolution}. New York: Viking, 1963.
	\item Hoffmann, Stanley, et al. \textit{In Search of France}. Cambridge: Harvard UP, 1963.
	\item Moore, Barrington. \textit{Social Origins of Dictatorship and Democracy}. Boston: Beacon, 1966.
	\item Rosanvallon, Pierre. \textit{The Demands of Liberty: Civil Society in France since the Revolution}. Cambridge: Harvard UP, 2007.
	\item Skocpol, Theda. \textit{States and Social Revolutions}. Cambridge: Cambridge UP, 1979.
	\item Tocqueville, Alexis de. \textit{The Old Regime and the Revolution}. New York: Anchor, 1955.
\end{itemize}

%══════════════════════════════════════════════════════════════════════════════
% پایان فصل ۱۰ و پایان کتاب
%══════════════════════════════════════════════════════════════════════════════
\end{document}