% ═══════════════════════════════════════════════════════════════════════════════
%                    تاریخ تحولات فرانسه - فصل [X]
% ═══════════════════════════════════════════════════════════════════════════════

\documentclass[12pt,a4paper]{book}

% ─────────────────────────── پکیج‌ها ───────────────────────────
\usepackage{amsmath,amssymb}
\usepackage{geometry}
\geometry{top=2.5cm, bottom=2.5cm, left=2cm, right=2.5cm, headheight=15pt}
\usepackage{graphicx}
\usepackage{array,booktabs,longtable,multirow,colortbl}
\usepackage{xcolor}
\usepackage{tikz}
\usetikzlibrary{shapes.geometric, arrows.meta, positioning, calc, backgrounds, 
	fit, decorations.pathmorphing, shadows, patterns}
\usepackage{pgfplots}
\pgfplotsset{compat=1.18}
\usepackage{tcolorbox}
\tcbuselibrary{skins,breakable}
\usepackage{enumitem}
\usepackage{fancyhdr}
\usepackage{pdflscape}
\usepackage{setspace}
\usepackage{titlesec}
\usepackage{float}
\usepackage{pdfpages}
\usepackage{pdflscape}  % برای صفحات landscape
\usepackage{hyperref}

% ─────────────────────────── رنگ‌ها ───────────────────────────
\definecolor{bleurepublique}{RGB}{0, 35, 149}
\definecolor{rougerevolution}{RGB}{237, 41, 57}
\definecolor{orroyal}{RGB}{255, 215, 0}
\definecolor{vertnapoleon}{RGB}{0, 100, 0}
\definecolor{violetempire}{RGB}{128, 0, 128}
\definecolor{fondclair}{RGB}{255, 253, 240}
\definecolor{gris}{RGB}{128, 128, 128}
\definecolor{grisclair}{RGB}{245, 245, 245}
\definecolor{noirsombre}{RGB}{30, 30, 30}

% رنگ‌های کمکی
\definecolor{bleulight}{RGB}{230, 235, 250}
\definecolor{rougelight}{RGB}{253, 235, 237}
\definecolor{vertlight}{RGB}{235, 250, 235}
\definecolor{violetlight}{RGB}{245, 235, 250}
\definecolor{orroyallight}{RGB}{255, 250, 230}
\definecolor{grislight}{RGB}{248, 248, 248}
\definecolor{bleumid}{RGB}{180, 195, 230}
\definecolor{rougemid}{RGB}{245, 180, 185}
\definecolor{vertmid}{RGB}{180, 220, 180}
\definecolor{violetmid}{RGB}{210, 180, 220}
\definecolor{orroyalmid}{RGB}{255, 240, 180}
\definecolor{orroyaldark}{RGB}{200, 170, 0}

% ─────────────────────────── فونت فارسی ───────────────────────────
\usepackage{fontspec}
\setmainfont{Vazirmatn}
\usepackage{xepersian}
\settextfont{Vazirmatn}
\setdigitfont{Vazirmatn}

% ─────────────────────────── هایپرلینک ───────────────────────────
\hypersetup{
	colorlinks=true,
	linkcolor=bleurepublique,
	urlcolor=bleurepublique,
	citecolor=vertnapoleon
}

% ─────────────────────────── کادرها ───────────────────────────
\newtcolorbox{kholasebox}[1][]{enhanced,breakable,colback=bleulight,
	colframe=bleurepublique,coltitle=white,fonttitle=\bfseries\large,
	title={#1},boxrule=2pt,arc=4pt,left=10pt,right=10pt,top=10pt,bottom=10pt,
	drop shadow={opacity=0.3}}

\newtcolorbox{naghlbox}[1][]{enhanced,breakable,colback=orroyallight,
	colframe=orroyaldark,coltitle=black,fonttitle=\bfseries,title={#1},
	boxrule=1.5pt,arc=3pt,borderline west={4pt}{0pt}{orroyal},
	left=15pt,right=10pt,top=8pt,bottom=8pt}

\newtcolorbox{olgoobox}[1][]{enhanced,breakable,colback=vertlight,
	colframe=vertnapoleon,coltitle=white,fonttitle=\bfseries,title={#1},
	boxrule=1.5pt,arc=4pt,left=10pt,right=10pt,top=8pt,bottom=8pt,
	before upper={\parindent15pt}}

\newtcolorbox{enghelabbox}[1][]{enhanced,breakable,colback=rougelight,
	colframe=rougerevolution,coltitle=white,fonttitle=\bfseries,title={#1},
	boxrule=2pt,arc=4pt,left=10pt,right=10pt,top=8pt,bottom=8pt}

\newtcolorbox{empirebox}[1][]{enhanced,breakable,colback=violetlight,
	colframe=violetempire,coltitle=white,fonttitle=\bfseries,title={#1},
	boxrule=1.5pt,arc=4pt,left=10pt,right=10pt,top=8pt,bottom=8pt}

\newtcolorbox{noktebox}[1][]{enhanced,colback=grisclair,colframe=gris,
	fonttitle=\bfseries,title={#1},boxrule=1pt,arc=3pt,left=8pt,right=8pt}

% ─────────────────────────── صفحه‌آرایی ───────────────────────────
\pagestyle{fancy}
\fancyhf{}
\fancyhead[RO]{\leftmark}
\fancyhead[LE]{\rightmark}
\fancyfoot[C]{\thepage}
\renewcommand{\headrulewidth}{1pt}
\renewcommand{\footrulewidth}{0.5pt}
\setstretch{1.5}

\titleformat{\chapter}[display]
{\normalfont\huge\bfseries\color{bleurepublique}}
{\chaptertitlename\ \thechapter}{20pt}{\Huge}
\titleformat{\section}
{\normalfont\Large\bfseries\color{bleurepublique}}{\thesection}{1em}{}
\titleformat{\subsection}
{\normalfont\large\bfseries\color{bleurepublique}}{\thesubsection}{1em}{}

% ═══════════════════════════════════════════════════════════════════════════════
\begin{document}
% ██████████████████████████████████████████████████████████████████████████████
%
%                    فصل ۴: دوره ناپلئونی (۱۷۹۹-۱۸۱۵)
%
% ██████████████████████████████████████████████████████████████████████████████

\chapter{دوره ناپلئونی (۱۷۹۹-۱۸۱۵)}

\begin{kholasebox}[خلاصه فصل چهارم]
	این فصل به شانزده سال حکومت ناپلئون بناپارت می‌پردازد: از کودتای برومر (۱۷۹۹) تا واترلو (۱۸۱۵). ناپلئون انقلاب را «تثبیت» کرد — اما به چه قیمتی؟ او فرانسه را به امپراتوری بزرگی تبدیل کرد، اصلاحات ماندگار انجام داد، اما میلیون‌ها نفر در جنگ‌هایش کشته شدند. پرسش محوری: ناپلئون وارث انقلاب بود یا خائن به آن؟
\end{kholasebox}

\section{ناپلئون کیست؟}

\begin{figure}[H]
	\centering
	\begin{tikzpicture}[
		scale=0.95,
		phase/.style={
			rectangle, draw=violetempire, line width=1.5pt, fill=violetlight,
			text=black, minimum width=3.5cm, minimum height=2cm,
			align=center, font=\small, rounded corners=5pt
		}
		]
		% خط زمان زندگی
		\draw[line width=3pt, gris] (0,0) -- (16,0);
		
		% مراحل
		\node[phase] at (1.5,2.5) {
			\begin{tabular}{c}
				\textbf{کرس}\\
				{\scriptsize ۱۷۶۹-۱۷۸۵}\\
				تولد، تحصیل
			\end{tabular}
		};
		\draw[->, >=Stealth, violetempire] (1.5,1.5) -- (1.5,0.3);
		
		\node[phase] at (4.5,2.5) {
			\begin{tabular}{c}
				\textbf{افسر جوان}\\
				{\scriptsize ۱۷۸۵-۱۷۹۵}\\
				انقلاب، تولون
			\end{tabular}
		};
		\draw[->, >=Stealth, violetempire] (4.5,1.5) -- (4.5,0.3);
		
		\node[phase] at (7.5,2.5) {
			\begin{tabular}{c}
				\textbf{ژنرال}\\
				{\scriptsize ۱۷۹۶-۱۷۹۹}\\
				ایتالیا، مصر
			\end{tabular}
		};
		\draw[->, >=Stealth, violetempire] (7.5,1.5) -- (7.5,0.3);
		
		\node[phase] at (10.5,2.5) {
			\begin{tabular}{c}
				\textbf{کنسول/امپراتور}\\
				{\scriptsize ۱۷۹۹-۱۸۱۴}\\
				اوج قدرت
			\end{tabular}
		};
		\draw[->, >=Stealth, violetempire] (10.5,1.5) -- (10.5,0.3);
		
		\node[phase] at (14,2.5) {
			\begin{tabular}{c}
				\textbf{تبعید}\\
				{\scriptsize ۱۸۱۴-۱۸۲۱}\\
				اِلبا، سنت‌هلن
			\end{tabular}
		};
		\draw[->, >=Stealth, violetempire] (14,1.5) -- (14,0.3);
		
		% سال‌ها
		\foreach \x/\year in {0/۱۷۶۹, 3/۱۷۸۵, 6/۱۷۹۶, 9/۱۷۹۹, 12/۱۸۱۴, 16/۱۸۲۱} {
			\draw[black, line width=1pt] (\x,-0.2) -- (\x,0.2);
			\node[below, font=\tiny] at (\x,-0.3) {\year};
		}
		
	\end{tikzpicture}
	\caption{خط زمانی زندگی ناپلئون بناپارت}
\end{figure}

\begin{table}[H]
	\centering
	\caption{ناپلئون بناپارت: مشخصات}
	\begin{tabular}{|>{\bfseries}r|p{9cm}|}
		\hline
		\rowcolor{violetlight}
		\textbf{مشخصه} & \textbf{توضیح} \\
		\hline
		تولد & ۱۵ اوت ۱۷۶۹، آژاکسیو، کرس (یک سال پس از الحاق به فرانسه) \\
		\hline
		\rowcolor{grisclair}
		خانواده & اشراف جزئی ایتالیایی‌تبار، هشت فرزند \\
		\hline
		تحصیلات & مدرسه نظامی بری‌ین، سپس پاریس (توپخانه) \\
		\hline
		\rowcolor{grisclair}
		ویژگی‌ها & هوش استثنایی، حافظه عکس‌برداری، انرژی خستگی‌ناپذیر \\
		\hline
		ضعف‌ها & خودبزرگ‌بینی، عدم پذیرش شکست، بی‌اعتمادی به دیگران \\
		\hline
		\rowcolor{grisclair}
		وفات & ۵ مه ۱۸۲۱، سنت‌هلن (احتمالاً سرطان معده) \\
		\hline
	\end{tabular}
\end{table}

\begin{naghlbox}[ناپلئون درباره خودش]
	«من انقلاب را بستم، نه به‌عنوان شاه، بلکه به‌عنوان مجری اراده ملت.»
	
	«قدرت من بر ترس مبتنی نیست، بلکه بر افتخار فرانسه.»
	
	«من جزیره‌ای کوچک را به تخت و تاج بزرگ‌ترین امپراتوری جهان تبدیل کردم.»
	\begin{flushright}
		— از یادداشت‌های سنت‌هلن
	\end{flushright}
\end{naghlbox}

\section{کنسولا (۱۷۹۹-۱۸۰۴)}

\subsection{ساختار قانون اساسی سال هشتم}

\begin{figure}[H]
	\centering
	\begin{tikzpicture}[
		scale=0.9,
		organ/.style={
			rectangle, draw=#1, line width=2pt, fill=#1,
			text=black, minimum width=4.5cm, minimum height=1.8cm,
			align=center, font=\small, rounded corners=5pt
		}
		]
		% عنوان
		\node[font=\large\bfseries, text=violetempire] at (0,6) {
			ساختار کنسولا (قانون اساسی سال هشتم)
		};
		
		% کنسول اول
		\node[organ=violetlight] (consul) at (0,4) {
			\begin{tabular}{c}
				\textbf{کنسول اول: ناپلئون}\\
				{\scriptsize تمام قدرت اجرایی، پیشنهاد قوانین}\\
				{\scriptsize انتصاب همه مقامات}
			\end{tabular}
		};
		
		% دو کنسول دیگر
		\node[organ=grislight] at (-5,4) {
			\begin{tabular}{c}
				کنسول دوم\\
				{\scriptsize کامباسرس}\\
				{\scriptsize فقط مشورتی}
			\end{tabular}
		};
		
		\node[organ=grislight] at (5,4) {
			\begin{tabular}{c}
				کنسول سوم\\
				{\scriptsize لوبرون}\\
				{\scriptsize فقط مشورتی}
			\end{tabular}
		};
		
		% مجالس
		\node[organ=bleulight] (senat) at (-4,1) {
			\begin{tabular}{c}
				\textbf{سنا}\\
				{\scriptsize ۸۰ عضو، مادام‌العمر}\\
				{\scriptsize حافظ قانون اساسی}
			\end{tabular}
		};
		
		\node[organ=bleulight] (trib) at (0,1) {
			\begin{tabular}{c}
				\textbf{تریبونا}\\
				{\scriptsize ۱۰۰ عضو، بحث قوانین}\\
				{\scriptsize بدون حق رأی!}
			\end{tabular}
		};
		
		\node[organ=bleulight] (legis) at (4,1) {
			\begin{tabular}{c}
				\textbf{مجلس قانون‌گذار}\\
				{\scriptsize ۳۰۰ عضو، رأی}\\
				{\scriptsize بدون حق بحث!}
			\end{tabular}
		};
		
		% شورای دولتی
		\node[organ=vertlight] (conseil) at (0,-1.5) {
			\begin{tabular}{c}
				\textbf{شورای دولتی}\\
				{\scriptsize تهیه پیش‌نویس قوانین}\\
				{\scriptsize مشاوران ناپلئون}
			\end{tabular}
		};
		
		% پیکان‌ها
		\draw[->, >=Stealth, line width=1.5pt, violetempire] (consul) -- (senat);
		\draw[->, >=Stealth, line width=1.5pt, violetempire] (consul) -- (trib);
		\draw[->, >=Stealth, line width=1.5pt, violetempire] (consul) -- (legis);
		\draw[->, >=Stealth, line width=1.5pt, violetempire] (consul) -- (conseil);
		
		% نتیجه
		\node[rectangle, draw=rougerevolution, fill=rougelight,
		minimum width=12cm, minimum height=1cm, font=\small, rounded corners=3pt]
		at (0,-3.5) {
			\textbf{نتیجه}: دموکراسی ظاهری، دیکتاتوری واقعی — همه قدرت در دست کنسول اول
		};
		
	\end{tikzpicture}
	\caption{ساختار نظام کنسولا}
\end{figure}

\subsection{«همه‌پرسی» و مشروعیت‌سازی}

\begin{table}[H]
	\centering
	\caption{همه‌پرسی‌های دوره ناپلئون}
	\begin{tabular}{|c|p{4.5cm}|c|c|}
		\hline
		\rowcolor{violetlight}
		\textbf{سال} & \textbf{موضوع} & \textbf{آری} & \textbf{نه} \\
		\hline
		۱۸۰۰ & قانون اساسی سال هشتم & ۳,۰۱۱,۰۰۷ & ۱,۵۶۲ \\
		\hline
		\rowcolor{grisclair}
		۱۸۰۲ & کنسولی مادام‌العمر & ۳,۵۶۸,۸۸۵ & ۸,۳۷۴ \\
		\hline
		۱۸۰۴ & امپراتوری موروثی & ۳,۵۷۲,۳۲۹ & ۲,۵۷۹ \\
		\hline
	\end{tabular}
\end{table}

\begin{enghelabbox}[آیا این ارقام واقعی بودند؟]
	ارقام همه‌پرسی‌ها بسیار مشکوک است:
	\begin{itemize}[nosep]
		\item رأی علنی، نه مخفی
		\item فشار اداری و نظامی
		\item دستکاری آشکار اعداد (توسط لوسین بناپارت)
		\item بسیاری اصلاً رأی ندادند
	\end{itemize}
	
	با این حال، محبوبیت واقعی ناپلئون در این دوره غیرقابل انکار است.
\end{enghelabbox}

\subsection{صلح خارجی و داخلی}

\begin{table}[H]
	\centering
	\caption{دستاوردهای اولیه کنسولا}
	\begin{tabular}{|>{\bfseries}r|c|p{6cm}|}
		\hline
		\rowcolor{vertlight}
		\textbf{دستاورد} & \textbf{تاریخ} & \textbf{اهمیت} \\
		\hline
		پیروزی مارنگو & ۱۸۰۰ & شکست اتریش در ایتالیا \\
		\hline
		\rowcolor{grisclair}
		صلح لونویل & ۱۸۰۱ & پایان جنگ با اتریش \\
		\hline
		صلح آمیان & ۱۸۰۲ & صلح با انگلستان (موقت) \\
		\hline
		\rowcolor{grisclair}
		پایان شورش واندé & ۱۸۰۰ & عفو عمومی، آرامش \\
		\hline
		کنکوردا & ۱۸۰۱ & آشتی با کلیسا \\
		\hline
		\rowcolor{grisclair}
		بازگشت مهاجران & ۱۸۰۰-۰۲ & عفو اشراف (مشروط) \\
		\hline
	\end{tabular}
\end{table}

\begin{naghlbox}[ناپلئون درباره صلح]
	«صلح اولین نیاز من بود. فرانسه خسته از جنگ بود. من صلح را به ارمغان آوردم — برای مدتی.»
\end{naghlbox}

\section{اصلاحات ناپلئونی: میراث ماندگار}

\subsection{کد ناپلئون (قانون مدنی ۱۸۰۴)}

\begin{figure}[H]
	\centering
	\begin{tikzpicture}[
		scale=0.95,
		principle/.style={
			rectangle, draw=bleurepublique, line width=1.5pt, fill=bleulight,
			text=black, minimum width=4.5cm, minimum height=1.5cm,
			align=center, font=\small, rounded corners=3pt
		}
		]
		% عنوان
		\node[font=\large\bfseries, text=bleurepublique] at (0,5) {
			اصول کد ناپلئون
		};
		
		% مرکز
		\node[ellipse, draw=violetempire, line width=3pt, fill=violetlight,
		minimum width=3.5cm, minimum height=2cm, font=\bfseries]
		(center) at (0,2) {
			\begin{tabular}{c}
				کد\\
				ناپلئون
			\end{tabular}
		};
		
		% اصول
		\node[principle] at (-5,3.5) {برابری در برابر قانون};
		\node[principle] at (5,3.5) {حق مالکیت مقدس};
		\node[principle] at (-5,0.5) {آزادی قرارداد};
		\node[principle] at (5,0.5) {سکولاریسم (ازدواج مدنی)};
		\node[principle] at (-5,-2) {وحدت قانونی (پایان قوانین محلی)};
		\node[principle] at (5,-2) {قدرت پدر خانواده};
		
		% خطوط
		\draw[bleurepublique, line width=1pt] (-5,2.8) -- (center);
		\draw[bleurepublique, line width=1pt] (5,2.8) -- (center);
		\draw[bleurepublique, line width=1pt] (-5,1.2) -- (center);
		\draw[bleurepublique, line width=1pt] (5,1.2) -- (center);
		\draw[bleurepublique, line width=1pt] (-5,-1.3) -- (center);
		\draw[bleurepublique, line width=1pt] (5,-1.3) -- (center);
		
		% آمار
		\node[rectangle, draw=gris, fill=grisclair,
		minimum width=10cm, minimum height=0.8cm, font=\scriptsize]
		at (0,-3.5) {
			۲,۲۸۱ ماده | تهیه‌شده توسط شورای دولتی | هنوز پایه حقوق فرانسه
		};
		
	\end{tikzpicture}
	\caption{اصول کد ناپلئون (۱۸۰۴)}
\end{figure}

\begin{table}[H]
	\centering
	\caption{کد ناپلئون: دستاوردها و محدودیت‌ها}
	\begin{tabular}{|>{\bfseries}r|p{5.5cm}|p{5.5cm}|}
		\hline
		\rowcolor{bleumid}
		\textbf{حوزه} & \textbf{دستاورد} & \textbf{محدودیت} \\
		\hline
		برابری & پایان امتیازات موروثی & فقط مردان برابر \\
		\hline
		\rowcolor{grisclair}
		مالکیت & تضمین مالکیت خصوصی & به نفع بورژوازی \\
		\hline
		خانواده & ازدواج مدنی، طلاق & قدرت مطلق شوهر بر زن \\
		\hline
		\rowcolor{grisclair}
		کار & آزادی قرارداد & ممنوعیت اتحادیه کارگری \\
		\hline
		زنان & — & از نظر حقوقی صغیر \\
		\hline
	\end{tabular}
\end{table}

\begin{naghlbox}[ناپلئون درباره کد]
	«افتخار واقعی من پیروزی در چهل نبرد نیست؛ واترلو خاطره همه را پاک خواهد کرد. اما چیزی که هرگز پاک نخواهد شد، قانون مدنی من است.»
	\begin{flushright}
		— ناپلئون، سنت‌هلن
	\end{flushright}
\end{naghlbox}

\begin{enghelabbox}[وضعیت زنان در کد ناپلئون]
	کد ناپلئون زنان را به وضعیت حقوقی تحقیرآمیزی محکوم کرد:
	\begin{itemize}[nosep]
		\item زن متأهل بدون اجازه شوهر نمی‌توانست کار کند، مالک شود، یا قرارداد ببندد
		\item زنا برای زن جرم بود، برای مرد فقط اگر در خانه بود
		\item فرزندان نامشروع حق ارث نداشتند
		\item شوهر می‌توانست زن را زندانی کند
	\end{itemize}
	
	ماده ۲۱۳: «زن باید از شوهر اطاعت کند.»
\end{enghelabbox}

\subsection{کنکوردا با پاپ (۱۸۰۱)}

\begin{figure}[H]
	\centering
	\begin{tikzpicture}[
		scale=0.9,
		party/.style={
			rectangle, draw=#1, line width=2pt, fill=#1,
			text=black, minimum width=5cm, minimum height=3cm,
			align=center, font=\small, rounded corners=5pt
		}
		]
		% دو طرف
		\node[party=violetlight] (nap) at (-4,0) {
			\begin{tabular}{c}
				\textbf{ناپلئون}\\[5pt]
				آشتی با کاتولیک‌ها\\
				کنترل بر کلیسا\\
				پایان شکاف دینی\\
				مشروعیت داخلی
			\end{tabular}
		};
		
		\node[party=orroyallight] (pope) at (4,0) {
			\begin{tabular}{c}
				\textbf{پاپ پیوس هفتم}\\[5pt]
				به رسمیت شناختن\\
				انتصاب اسقف‌ها\\
				حقوق روحانیون\\
				بازگشت به فرانسه
			\end{tabular}
		};
		
		% توافق
		\node[ellipse, draw=vertnapoleon, line width=2pt, fill=vertlight,
		minimum width=3cm, minimum height=1.5cm, font=\bfseries]
		at (0,0) {کنکوردا};
		
		\draw[<->, >=Stealth, line width=2pt, gris] (nap) -- (-1.5,0);
		\draw[<->, >=Stealth, line width=2pt, gris] (1.5,0) -- (pope);
		
		% مفاد
		\node[rectangle, draw=vertnapoleon, fill=vertlight,
		minimum width=12cm, minimum height=2cm, font=\small, rounded corners=5pt]
		at (0,-3) {
			\begin{tabular}{c}
				\textbf{مفاد اصلی}:\\
				کاتولیسیسم «دین اکثریت فرانسویان» (نه دین رسمی) |
				دولت اسقف‌ها را نامزد، پاپ تأیید می‌کند\\
				حقوق روحانیون از دولت | اموال مصادره‌شده برنمی‌گردد
			\end{tabular}
		};
		
	\end{tikzpicture}
	\caption{کنکوردا ۱۸۰۱}
\end{figure}

\begin{olgoobox}[اهمیت کنکوردا]
	کنکوردا یکی از مهم‌ترین دستاوردهای ناپلئون بود:
	\begin{enumerate}[nosep]
		\item پایان شکاف دینی ده‌ساله
		\item آشتی با کاتولیک‌های سنتی
		\item کنترل دولت بر کلیسا (نه برعکس)
		\item مشروعیت برای نظام جدید
	\end{enumerate}
	
	اما ناپلئون «مواد ارگانیک» را بدون موافقت پاپ اضافه کرد که قدرت دولت را افزایش داد.
\end{olgoobox}

\subsection{نظام آموزشی}

\begin{table}[H]
	\centering
	\caption{نظام آموزشی ناپلئونی}
	\begin{tabular}{|>{\bfseries}r|p{4cm}|p{4cm}|p{3cm}|}
		\hline
		\rowcolor{bleulight}
		\textbf{سطح} & \textbf{نهاد} & \textbf{هدف} & \textbf{کنترل} \\
		\hline
		ابتدایی & مدارس محلی & سواد پایه & کلیسا/شهرداری \\
		\hline
		\rowcolor{grisclair}
		متوسطه & لیسه & تربیت نخبگان & دولت مرکزی \\
		\hline
		عالی & دانشگاه امپراتوری & کادرسازی & دولت مرکزی \\
		\hline
		\rowcolor{grisclair}
		نظامی & اکول پلی‌تکنیک & افسران، مهندسان & ارتش \\
		\hline
	\end{tabular}
\end{table}

\begin{naghlbox}[ناپلئون درباره آموزش]
	«از همه نهادهای سیاسی، آموزش عمومی مهم‌ترین است. همه چیز به آموزش بستگی دارد: حال و آینده.»
\end{naghlbox}

\subsection{نظام اداری}

\begin{figure}[H]
	\centering
	\begin{tikzpicture}[
		scale=0.9,
		level/.style={
			rectangle, draw=bleurepublique, line width=1.5pt, fill=bleulight,
			text=black, minimum width=4cm, minimum height=1.5cm,
			align=center, font=\small, rounded corners=3pt
		}
		]
		% سلسله‌مراتب
		\node[level] (nap) at (0,4) {
			\begin{tabular}{c}
				\textbf{ناپلئون}\\
				{\scriptsize وزرا، شورای دولتی}
			\end{tabular}
		};
		
		\node[level] (pref) at (0,2) {
			\begin{tabular}{c}
				\textbf{پرفه (استاندار)}\\
				{\scriptsize منصوب توسط ناپلئون، ۱ در هر دپارتمان}
			\end{tabular}
		};
		
		\node[level] (sous) at (0,0) {
			\begin{tabular}{c}
				\textbf{سو-پرفه}\\
				{\scriptsize در هر آروندیسمان}
			\end{tabular}
		};
		
		\node[level] (maire) at (0,-2) {
			\begin{tabular}{c}
				\textbf{شهردار (مِر)}\\
				{\scriptsize منصوب، نه انتخابی}
			\end{tabular}
		};
		
		% پیکان‌ها
		\draw[->, >=Stealth, line width=2pt, bleurepublique] (nap) -- (pref);
		\draw[->, >=Stealth, line width=2pt, bleurepublique] (pref) -- (sous);
		\draw[->, >=Stealth, line width=2pt, bleurepublique] (sous) -- (maire);
		
		% توضیح
		\node[right, font=\small, text=gris] at (3,1) {
			\begin{tabular}{l}
				زنجیره فرمان واحد\\
				همه منصوب، نه انتخابی\\
				گزارش‌دهی به بالا\\
				اجرای دستور از بالا
			\end{tabular}
		};
		
	\end{tikzpicture}
	\caption{سلسله‌مراتب اداری ناپلئونی}
\end{figure}

\begin{olgoobox}[میراث اداری ناپلئون]
	نظام پرفه (préfet) هنوز پایه اداره فرانسه است:
	\begin{itemize}[nosep]
		\item تمرکز شدید
		\item کارآمدی اداری
		\item یکنواختی در سراسر کشور
		\item اما: کاهش خودمختاری محلی
	\end{itemize}
\end{olgoobox}

\subsection{سایر اصلاحات}

\begin{table}[H]
	\centering
	\caption{سایر اصلاحات ناپلئونی}
	\begin{tabular}{|>{\bfseries}r|p{6cm}|p{4cm}|}
		\hline
		\rowcolor{vertlight}
		\textbf{حوزه} & \textbf{اصلاح} & \textbf{ماندگاری} \\
		\hline
		مالی & بانک فرانسه (۱۸۰۰)، فرانک ژرمینال & تا ۱۹۱۴ \\
		\hline
		\rowcolor{grisclair}
		افتخارات & لژیون دونور (۱۸۰۲) & تا امروز \\
		\hline
		زیرساخت & جاده‌ها، پل‌ها، کانال‌ها & تا امروز \\
		\hline
		\rowcolor{grisclair}
		قضایی & کد جزایی، کد تجاری & با اصلاحات \\
		\hline
		متریک & نظام متریک اجباری & جهانی شد \\
		\hline
	\end{tabular}
\end{table}

\section{از کنسولا به امپراتوری (۱۸۰۴)}

\subsection{چرا امپراتوری؟}

\begin{enghelabbox}[توطئه کادودال و بهانه امپراتوری]
	در ۱۸۰۴، توطئه‌ای برای ترور ناپلئون کشف شد (توطئه کادودال). ناپلئون از این فرصت استفاده کرد:
	\begin{itemize}[nosep]
		\item دوک دانگین (شاهزاده بوربون) ربوده و اعدام شد — بدون محاکمه واقعی
		\item استدلال: «باید جانشینی تضمین شود»
		\item سنا از ناپلئون خواست امپراتور شود
		\item همه‌پرسی (مشکوک) تأیید کرد
	\end{itemize}
\end{enghelabbox}

\subsection{تاجگذاری: ۲ دسامبر ۱۸۰۴}

\begin{figure}[H]
	\centering
	\begin{tikzpicture}
		% کادر اصلی
		\node[rectangle, draw=violetempire, line width=3pt, fill=violetlight,
		minimum width=13cm, minimum height=5cm, rounded corners=10pt]
		at (0,0) {};
		
		% عنوان
		\node[font=\large\bfseries, text=violetempire] at (0,1.8) {
			تاجگذاری در نوتردام
		};
		
		% جزئیات
		\node[text width=11cm, align=center, font=\small] at (0,0) {
			پاپ پیوس هفتم به پاریس آمد — اولین بار که پاپ فرانسه را برای تاجگذاری ترک می‌کرد.\\[5pt]
			اما در لحظه حساس، \textbf{ناپلئون تاج را از دست پاپ گرفت و خودش بر سر نهاد}.\\[5pt]
			پیام: قدرت من از خودم است، نه از کلیسا.\\[5pt]
			سپس تاج ژوزفین را نیز خود بر سر او گذاشت.
		};
		
		% نقل قول
		\node[font=\footnotesize\itshape, text=gris] at (0,-1.8) {
			«من تاج را یافتم در جوی آب افتاده، و با شمشیرم آن را برداشتم.»
		};
		
	\end{tikzpicture}
	\caption{تاجگذاری ناپلئون (۲ دسامبر ۱۸۰۴)}
\end{figure}

\begin{naghlbox}[واکنش بتهوون]
	وقتی بتهوون شنید ناپلئون امپراتور شده، صفحه عنوان سمفونی سوم («قهرمانی») را که به ناپلئون تقدیم کرده بود، پاره کرد و گفت: «پس او هم فقط یک انسان عادی است! حالا حقوق بشر را زیر پا خواهد گذاشت و فقط به جاه‌طلبی خود خدمت خواهد کرد.»
\end{naghlbox}

\section{جنگ‌های ناپلئونی}

\subsection{نبردهای اصلی}

% صفحه افقی برای جدول جنگ‌ها
\begin{landscape}
	\begin{table}[p]
		\centering
		\caption{نبردهای اصلی ناپلئونی (۱۸۰۵-۱۸۱۵)}
		\small
		\begin{tabular}{|c|>{\bfseries}r|p{3cm}|p{3cm}|c|p{4cm}|}
			\hline
			\rowcolor{violetlight}
			\textbf{سال} & \textbf{نبرد} & \textbf{دشمن} & \textbf{فرمانده دشمن} & \textbf{نتیجه} & \textbf{پیامد} \\
			\hline
			۱۸۰۵ & ترافالگار & انگلستان & نلسون & شکست & پایان رؤیای تهاجم به انگلستان \\
			\hline
			\rowcolor{grisclair}
			۱۸۰۵ & آستِرلیتس & اتریش، روسیه & کوتوزوف & پیروزی & انحلال امپراتوری مقدس روم \\
			\hline
			۱۸۰۶ & یِنا & پروس & برانشوایگ & پیروزی & سقوط پروس، تسلط بر آلمان \\
			\hline
			\rowcolor{grisclair}
			۱۸۰۷ & فریدلند & روسیه & بنیگسن & پیروزی & صلح تیلسیت، اوج امپراتوری \\
			\hline
			۱۸۰۹ & واگرام & اتریش & کارل & پیروزی & ازدواج با ماری‌لوئیز \\
			\hline
			\rowcolor{grisclair}
			۱۸۱۲ & بورودینو & روسیه & کوتوزوف & پیروزی پیروسی & تصرف مسکو، فاجعه بازگشت \\
			\hline
			۱۸۱۳ & لایپزیگ & ائتلاف & شوارتزنبرگ & شکست & از دست رفتن آلمان \\
			\hline
			\rowcolor{grisclair}
			۱۸۱۵ & واترلو & انگلستان، پروس & ولینگتون، بلوخر & شکست & پایان کار ناپلئون \\
			\hline
		\end{tabular}
	\end{table}
\end{landscape}

\subsection{آستِرلیتس: شاهکار تاکتیکی}

\begin{figure}[H]
	\centering
	\begin{tikzpicture}[
		scale=0.9
		]
		% کادر
		\node[rectangle, draw=violetempire, line width=3pt, fill=violetlight,
		minimum width=14cm, minimum height=6cm, rounded corners=10pt]
		at (0,0) {};
		
		% عنوان
		\node[font=\large\bfseries, text=violetempire] at (0,2.3) {
			نبرد آستِرلیتس — «نبرد سه امپراتور»
		};
		\node[font=\small, text=gris] at (0,1.7) {۲ دسامبر ۱۸۰۵};
		
		% آمار
		\node[text width=12cm, align=center, font=\small] at (0,0.3) {
			\textbf{فرانسه}: ۶۸,۰۰۰ نفر \quad vs \quad \textbf{ائتلاف}: ۸۵,۰۰۰ نفر\\[5pt]
			\textbf{تلفات فرانسه}: ۹,۰۰۰ \quad vs \quad \textbf{تلفات ائتلاف}: ۳۶,۰۰۰\\[10pt]
			ناپلئون دشمن را فریب داد: وانمود کرد جناح راستش ضعیف است.\\
			متحدین به آنجا حمله کردند — دقیقاً همان‌طور که ناپلئون می‌خواست.\\
			سپس مرکز دشمن را شکافت و آنها را نابود کرد.
		};
		
		% نتیجه
		\node[rectangle, draw=vertnapoleon, fill=vertlight,
		minimum width=10cm, minimum height=1cm, font=\small, rounded corners=3pt]
		at (0,-2) {
			\textbf{نتیجه}: پایان امپراتوری مقدس روم، اتریش تحقیر شد، ناپلئون ارباب اروپا
		};
		
	\end{tikzpicture}
	\caption{نبرد آستِرلیتس}
\end{figure}

\subsection{اوج امپراتوری: ۱۸۰۷-۱۸۱۲}

\begin{figure}[H]
	\centering
	\begin{tikzpicture}[
		scale=0.85,
		transform shape,
		territory/.style={
			rectangle, draw=violetempire, line width=1.5pt, fill=violetlight,
			text=black, minimum width=3cm, minimum height=1.2cm,
			align=center, font=\scriptsize, rounded corners=3pt
		}
		]
		% عنوان
		\node[font=\large\bfseries, text=violetempire] at (0,5) {
			امپراتوری ناپلئون در اوج (۱۸۱۲)
		};
		
		% امپراتوری فرانسه
		\node[ellipse, draw=violetempire, line width=3pt, fill=violetlight,
		minimum width=4cm, minimum height=2cm, font=\bfseries]
		(france) at (0,2) {
			\begin{tabular}{c}
				فرانسه\\
				{\scriptsize ۱۳۰ دپارتمان}
			\end{tabular}
		};
		
		% کشورهای وابسته
		\node[territory] (spain) at (-6,3) {اسپانیا\\(ژوزف)};
		\node[territory] (italy) at (-6,1) {ایتالیا\\(اوژن)};
		\node[territory] (naples) at (-6,-1) {ناپل\\(مورا)};
		\node[territory] (holland) at (-3,4) {هلند\\(لویی→الحاق)};
		\node[territory] (westph) at (3,4) {وستفالی\\(ژروم)};
		\node[territory] (confed) at (6,3) {کنفدراسیون راین\\(متحد)};
		\node[territory] (warsaw) at (6,1) {ورشو\\(وابسته)};
		\node[territory] (swiss) at (6,-1) {سوئیس\\(تحت‌الحمایه)};
		
		% خطوط
		\draw[violetempire, line width=1pt] (france) -- (spain);
		\draw[violetempire, line width=1pt] (france) -- (italy);
		\draw[violetempire, line width=1pt] (france) -- (naples);
		\draw[violetempire, line width=1pt] (france) -- (holland);
		\draw[violetempire, line width=1pt] (france) -- (westph);
		\draw[violetempire, line width=1pt] (france) -- (confed);
		\draw[violetempire, line width=1pt] (france) -- (warsaw);
		\draw[violetempire, line width=1pt] (france) -- (swiss);
		
		% دشمنان باقیمانده
		\node[rectangle, draw=rougerevolution, fill=rougelight,
		minimum width=8cm, minimum height=1cm, font=\small, rounded corners=3pt]
		at (0,-2) {
			\textbf{دشمنان}: انگلستان (دریا) | روسیه (شرق) | پرتغال و اسپانیا (مقاومت)
		};
		
	\end{tikzpicture}
	\caption{امپراتوری ناپلئون در اوج (۱۸۱۲)}
\end{figure}

\subsection{نظام قاره‌ای: جنگ اقتصادی با انگلستان}

\begin{table}[H]
	\centering
	\caption{نظام قاره‌ای (Continental System)}
	\begin{tabular}{|>{\bfseries}r|p{9cm}|}
		\hline
		\rowcolor{rougelight}
		\textbf{جنبه} & \textbf{توضیح} \\
		\hline
		هدف & محاصره اقتصادی انگلستان — «بستن قاره» \\
		\hline
		\rowcolor{grisclair}
		اعلام & فرمان برلین (۱۸۰۶)، میلان (۱۸۰۷) \\
		\hline
		مفاد & ممنوعیت واردات کالای انگلیسی به همه اروپا \\
		\hline
		\rowcolor{grisclair}
		مشکلات & قاچاق گسترده، نارضایتی متحدین، ضرر به خود فرانسه \\
		\hline
		شکست & انگلستان دوام آورد، نظام فروپاشید \\
		\hline
	\end{tabular}
\end{table}

\begin{olgoobox}[چرا نظام قاره‌ای شکست خورد؟]
	\begin{enumerate}[nosep]
		\item انگلستان بازارهای دیگر یافت (آمریکا، آسیا)
		\item قاچاق غیرقابل کنترل بود
		\item متحدین ناپلئون از ممنوعیت رنج می‌بردند
		\item حتی فرانسه به کالاهای انگلیسی نیاز داشت
		\item روسیه از نظام خارج شد — بهانه حمله ۱۸۱۲
	\end{enumerate}
\end{olgoobox}

\section{سقوط امپراتوری (۱۸۱۲-۱۸۱۵)}

\subsection{فاجعه روسیه: ۱۸۱۲}

\begin{figure}[H]
	\centering
	\begin{tikzpicture}[
		scale=0.9
		]
		% عنوان
		\node[font=\large\bfseries, text=rougerevolution] at (0,4.5) {
			لشکرکشی به روسیه: فاجعه بزرگ
		};
		
		% آمار شروع و پایان
		\node[rectangle, draw=bleurepublique, fill=bleulight,
		minimum width=5cm, minimum height=2cm, font=\small, rounded corners=5pt]
		at (-4,2) {
			\begin{tabular}{c}
				\textbf{شروع (ژوئن ۱۸۱۲)}\\[5pt]
				۶۸۵,۰۰۰ سرباز\\
				بزرگ‌ترین ارتش تاریخ
			\end{tabular}
		};
		
		\node[rectangle, draw=rougerevolution, fill=rougelight,
		minimum width=5cm, minimum height=2cm, font=\small, rounded corners=5pt]
		at (4,2) {
			\begin{tabular}{c}
				\textbf{پایان (دسامبر ۱۸۱۲)}\\[5pt]
				۲۷,۰۰۰ سرباز\\
				فقط ۴٪ بازگشتند
			\end{tabular}
		};
		
		% پیکان
		\draw[->, >=Stealth, line width=3pt, rougerevolution] (-1,2) -- (1,2);
		
		% علل
		\node[rectangle, draw=gris, fill=grisclair,
		minimum width=12cm, minimum height=2.5cm, font=\small, rounded corners=5pt]
		at (0,-1) {
			\begin{tabular}{c}
				\textbf{علل فاجعه}:\\[3pt]
				عقب‌نشینی روس‌ها و سوزاندن همه چیز (زمین سوخته)\\
				خطوط تدارکاتی بیش از حد طولانی\\
				زمستان روسیه (−۳۰ درجه) | گرسنگی و بیماری\\
				حمله پارتیزانی در بازگشت
			\end{tabular}
		};
		
		% نقل قول
		\node[font=\small\itshape, text=gris] at (0,-3.5) {
			«از مسکو به ویلنا، ارتش بزرگ یک گورستان متحرک بود.»
		};
		
	\end{tikzpicture}
	\caption{فاجعه روسیه (۱۸۱۲)}
\end{figure}

\subsection{فروپاشی: ۱۸۱۳-۱۸۱۴}

\begin{table}[H]
	\centering
	\caption{مراحل سقوط امپراتوری}
	\begin{tabular}{|c|p{4cm}|p{6cm}|}
		\hline
		\rowcolor{rougelight}
		\textbf{تاریخ} & \textbf{رویداد} & \textbf{پیامد} \\
		\hline
		۱۸۱۳ & نبرد لایپزیگ («نبرد ملل») & شکست بزرگ، از دست رفتن آلمان \\
		\hline
		\rowcolor{grisclair}
		ژانویه ۱۸۱۴ & ائتلاف وارد فرانسه شد & جنگ در خاک فرانسه \\
		\hline
		مارس ۱۸۱۴ & سقوط پاریس & ارتش تسلیم شد \\
		\hline
		\rowcolor{grisclair}
		۶ آوریل ۱۸۱۴ & استعفای ناپلئون & تبعید به اِلبا \\
		\hline
		مه ۱۸۱۴ & لویی هجدهم به قدرت رسید & بازگشت بوربون‌ها \\
		\hline
	\end{tabular}
\end{table}

\subsection{صد روز و واترلو}

\begin{figure}[H]
	\centering
	\begin{tikzpicture}[
		scale=0.85,
		event/.style={
			rectangle, draw=violetempire, line width=1.5pt, fill=violetlight,
			text=black, minimum width=3cm, minimum height=1.5cm,
			align=center, font=\scriptsize, rounded corners=3pt
		}
		]
		% خط زمان
		\draw[line width=3pt, gris] (0,0) -- (14,0);
		
		% تاریخ‌ها
		\node[below, font=\tiny] at (0,-0.3) {مارس ۱۵};
		\node[below, font=\tiny] at (3.5,-0.3) {مارس ۲۰};
		\node[below, font=\tiny] at (7,-0.3) {ژوئن ۱};
		\node[below, font=\tiny] at (10.5,-0.3) {ژوئن ۱۸};
		\node[below, font=\tiny] at (14,-0.3) {ژوئن ۲۲};
		
		% رویدادها
		\node[event] at (1.5,2) {
			\begin{tabular}{c}
				فرار از اِلبا\\
				فرود در فرانسه
			\end{tabular}
		};
		\draw[->, >=Stealth, violetempire] (1.5,1.2) -- (1.5,0.3);
		
		\node[event] at (5,2) {
			\begin{tabular}{c}
				ورود به پاریس\\
				فرار لویی ۱۸
			\end{tabular}
		};
		\draw[->, >=Stealth, violetempire] (5,1.2) -- (5,0.3);
		
		\node[event] at (8.5,2) {
			\begin{tabular}{c}
				آغاز جنگ\\
				با ائتلاف هفتم
			\end{tabular}
		};
		\draw[->, >=Stealth, violetempire] (8.5,1.2) -- (8.5,0.3);
		
		\node[event, draw=rougerevolution, fill=rougelight] at (12,2) {
			\begin{tabular}{c}
				\textbf{واترلو}\\
				شکست نهایی
			\end{tabular}
		};
		\draw[->, >=Stealth, rougerevolution, line width=2pt] (12,1.2) -- (12,0.3);
		
		% نتیجه
		\node[rectangle, draw=bleurepublique, fill=bleulight,
		minimum width=10cm, minimum height=1cm, font=\small, rounded corners=3pt]
		at (7,-2) {
			استعفای دوم → تبعید به سنت‌هلن → مرگ ۱۸۲۱
		};
		
	\end{tikzpicture}
	\caption{صد روز (مارس-ژوئن ۱۸۱۵)}
\end{figure}

\begin{naghlbox}[واترلو]
	«واترلو شکستی بود که فتح را تکمیل کرد.»
	\begin{flushright}
		— ویکتور هوگو، بینوایان
	\end{flushright}
\end{naghlbox}

\section{ارزیابی: ناپلئون وارث یا خائن؟}

\begin{figure}[H]
	\centering
	\begin{tikzpicture}[
		scale=0.9,
		arg/.style={
			rectangle, draw=#1, line width=1.5pt, fill=#1,
			text=black, minimum width=5.5cm, minimum height=1.2cm,
			align=center, font=\small, rounded corners=3pt
		}
		]
		% دو ستون
		\node[font=\large\bfseries, text=vertnapoleon] at (-4,5) {وارث انقلاب};
		\node[font=\large\bfseries, text=rougerevolution] at (4,5) {خائن به انقلاب};
		
		% استدلال‌ها
		\node[arg=vertlight] at (-4,3.8) {تثبیت اصلاحات (کد، کنکوردا)};
		\node[arg=vertlight] at (-4,2.4) {برابری حقوقی (مردان)};
		\node[arg=vertlight] at (-4,1) {پایان فئودالیسم در اروپا};
		\node[arg=vertlight] at (-4,-0.4) {شایسته‌سالاری (کارِیر آزاد)};
		\node[arg=vertlight] at (-4,-1.8) {گسترش ایده‌های انقلابی};
		
		\node[arg=rougelight] at (4,3.8) {پایان دادن به آزادی سیاسی};
		\node[arg=rougelight] at (4,2.4) {امپراتوری موروثی};
		\node[arg=rougelight] at (4,1) {سانسور و پلیس مخفی};
		\node[arg=rougelight] at (4,-0.4) {جنگ‌های بی‌پایان، میلیون‌ها کشته};
		\node[arg=rougelight] at (4,-1.8) {محدودیت حقوق زنان};
		
		% خط جداکننده
		\draw[gris, line width=2pt, dashed] (0,4.5) -- (0,-2.5);
		
		% نتیجه
		\node[rectangle, draw=gris, fill=grisclair,
		minimum width=12cm, minimum height=1.2cm, font=\small, rounded corners=5pt]
		at (0,-3.5) {
			\textbf{نتیجه}: ناپلئون هم وارث و هم خائن بود — بسته به معیار ارزیابی
		};
		
	\end{tikzpicture}
	\caption{ناپلئون: وارث یا خائن به انقلاب؟}
\end{figure}

\begin{olgoobox}[الگوی ناپلئونی]
	ناپلئون الگویی ایجاد کرد که بارها تکرار شد:
	\begin{enumerate}[nosep]
		\item \textbf{بحران و هرج‌ومرج} → مردم خواستار نظم
		\item \textbf{مرد قوی} ظهور می‌کند — معمولاً ژنرال
		\item \textbf{همه‌پرسی} مشروعیت می‌دهد
		\item \textbf{اصلاحات} با اقتدارگرایی
		\item \textbf{جنگ خارجی} برای حفظ مشروعیت
		\item \textbf{شکست نهایی} به دلیل توسعه‌طلبی بیش از حد
	\end{enumerate}
	
	این الگو در ناپلئون سوم، موسولینی، و بسیاری دیگر دیده می‌شود.
\end{olgoobox}

\section{منابع فصل چهارم}

% ══════════════════════════════════════════════════════════════════════════════
%                    ادامه فصل ۴ (از منابع)
% ══════════════════════════════════════════════════════════════════════════════

\subsection{منابع اولیه}

\begin{enumerate}[nosep]
	\item ناپلئون بناپارت. یادداشت‌های سنت‌هلن (روایت لاس کازس).
	\item کد ناپلئون (قانون مدنی ۱۸۰۴).
	\item مکاتبات ناپلئون. ۳۲ جلد.
	\item کنکوردا ۱۸۰۱.
	\item بولتن‌های ارتش بزرگ.
\end{enumerate}

\subsection{منابع ثانویه}

\begin{enumerate}[nosep]
	\item Englund, Steven. \textit{Napoleon: A Political Life}. Harvard UP, 2004.
	\item Lefebvre, Georges. \textit{Napoleon: From 18 Brumaire to Tilsit}. Columbia UP, 1969.
	\item Tulard, Jean. \textit{Napoleon: The Myth of the Saviour}. Weidenfeld, 1984.
	\item Broers, Michael. \textit{Napoleon: Soldier of Destiny}. Pegasus, 2015.
	\item Ellis, Geoffrey. \textit{The Napoleonic Empire}. Macmillan, 2003.
	\item Dwyer, Philip. \textit{Napoleon: The Path to Power}. Yale UP, 2008.
	\item Grab, Alexander. \textit{Napoleon and the Transformation of Europe}. Palgrave, 2003.
	\item Woolf, Stuart. \textit{Napoleon's Integration of Europe}. Routledge, 1991.
	\item Zamoyski, Adam. \textit{1812: Napoleon's Fatal March on Moscow}. HarperCollins, 2004.
	\item Roberts, Andrew. \textit{Napoleon: A Life}. Viking, 2014.
\end{enumerate}

\begin{kholasebox}[جمع‌بندی فصل چهارم]
	در این فصل دیدیم که:
	
	\begin{itemize}[nosep]
		\item \textbf{کنسولا (۱۷۹۹-۱۸۰۴)}: تثبیت انقلاب با اقتدارگرایی
		\item \textbf{اصلاحات ماندگار}: کد ناپلئون، کنکوردا، نظام اداری، آموزش
		\item \textbf{امپراتوری (۱۸۰۴-۱۸۱۴)}: تبدیل جمهوری به سلطنت موروثی
		\item \textbf{جنگ‌ها}: از آستِرلیتس تا واترلو — میلیون‌ها کشته
		\item \textbf{سقوط}: توسعه‌طلبی بیش از حد، فاجعه روسیه، شکست نهایی
		\item \textbf{میراث}: هم تثبیت انقلاب، هم خیانت به آن
	\end{itemize}
	
	\textbf{پرسش کلیدی}: آیا دستاوردهای ناپلئون ارزش هزینه‌هایش را داشت؟
	
	\textbf{در فصل بعد}: قرن ناآرام (۱۸۱۵-۱۸۷۰) — بازگشت سلطنت، انقلاب‌های ۱۸۳۰ و ۱۸۴۸، جمهوری دوم، امپراتوری دوم
\end{kholasebox}

% ══════════════════════════════════════════════════════════════════════════════
%                    پایان فصل ۴
% ══════════════════════════════════════════════════════════════════════════════
% ══════════════════════════════════════════════════════════════════════════════
%                    خط زمانی جامع دوره ناپلئونی - صفحه افقی
% ══════════════════════════════════════════════════════════════════════════════

\begin{landscape}
	\begin{figure}[p]
		\centering
		\begin{tikzpicture}[
			scale=0.7,
			transform shape
			]
			% عنوان
			\node[font=\Large\bfseries, text=violetempire] at (12,8) {
				خط زمانی دوره ناپلئونی (۱۷۹۹-۱۸۱۵)
			};
			
			% خط اصلی زمان
			\draw[line width=3pt, gris] (0,0) -- (24,0);
			
			% دوره‌ها
			\fill[bleulight] (0,0.4) rectangle (6,0.8);
			\node[above, font=\scriptsize\bfseries, text=bleurepublique] at (3,0.9) {کنسولا};
			
			\fill[violetlight] (6,0.4) rectangle (20,0.8);
			\node[above, font=\scriptsize\bfseries, text=violetempire] at (13,0.9) {امپراتوری اول};
			
			\fill[orroyallight] (20,0.4) rectangle (22,0.8);
			\node[above, font=\scriptsize\bfseries, text=orroyaldark] at (21,0.9) {اِلبا};
			
			\fill[rougelight] (22,0.4) rectangle (24,0.8);
			\node[above, font=\scriptsize\bfseries, text=rougerevolution] at (23,0.9) {۱۰۰ روز};
			
			% سال‌ها
			\foreach \x/\year in {0/۱۷۹۹, 3/۱۸۰۱, 6/۱۸۰۴, 9/۱۸۰۵, 12/۱۸۰۷, 
				15/۱۸۱۰, 18/۱۸۱۲, 20/۱۸۱۴, 24/۱۸۱۵} {
				\draw[black, line width=1.5pt] (\x,-0.3) -- (\x,0.3);
				\node[below, font=\tiny] at (\x,-0.4) {\year};
			}
			
			% رویدادهای بالا (سیاسی/نظامی)
			\node[rectangle, draw=bleurepublique, fill=bleulight, 
			minimum width=2cm, minimum height=0.8cm, font=\tiny, align=center]
			at (1.5,2.5) {کودتای\\برومر};
			\draw[->, >=Stealth, bleurepublique] (1.5,2) -- (1.5,0.5);
			
			\node[rectangle, draw=vertnapoleon, fill=vertlight, 
			minimum width=2cm, minimum height=0.8cm, font=\tiny, align=center]
			at (4,2.5) {صلح آمیان\\با انگلستان};
			\draw[->, >=Stealth, vertnapoleon] (4,2) -- (4,0.5);
			
			\node[rectangle, draw=violetempire, fill=violetlight, 
			minimum width=2cm, minimum height=0.8cm, font=\tiny, align=center]
			at (7,2.5) {تاجگذاری\\امپراتور};
			\draw[->, >=Stealth, violetempire] (7,2) -- (7,0.5);
			
			\node[rectangle, draw=rougerevolution, fill=rougelight, 
			minimum width=2cm, minimum height=0.8cm, font=\tiny, align=center]
			at (9.5,2.5) {ترافالگار\\(شکست)};
			\draw[->, >=Stealth, rougerevolution] (9.5,2) -- (9.5,0.5);
			
			\node[rectangle, draw=vertnapoleon, fill=vertlight, 
			minimum width=2cm, minimum height=0.8cm, font=\tiny, align=center]
			at (10.5,4) {آستِرلیتس\\(پیروزی)};
			\draw[->, >=Stealth, vertnapoleon] (10.5,3.5) -- (10.5,0.5);
			
			\node[rectangle, draw=violetempire, fill=violetlight, 
			minimum width=2cm, minimum height=0.8cm, font=\tiny, align=center]
			at (13,2.5) {صلح تیلسیت\\اوج قدرت};
			\draw[->, >=Stealth, violetempire] (13,2) -- (13,0.5);
			
			\node[rectangle, draw=rougerevolution, fill=rougelight, 
			minimum width=2cm, minimum height=0.8cm, font=\tiny, align=center]
			at (16,2.5) {جنگ اسپانیا\\شروع زوال};
			\draw[->, >=Stealth, rougerevolution] (16,2) -- (16,0.5);
			
			\node[rectangle, draw=rougerevolution, fill=rougemid, 
			minimum width=2cm, minimum height=0.8cm, font=\tiny, align=center]
			at (18.5,2.5) {فاجعه\\روسیه};
			\draw[->, >=Stealth, rougerevolution, line width=2pt] (18.5,2) -- (18.5,0.5);
			
			\node[rectangle, draw=rougerevolution, fill=rougemid, 
			minimum width=2cm, minimum height=0.8cm, font=\tiny, align=center]
			at (23,2.5) {واترلو\\پایان};
			\draw[->, >=Stealth, rougerevolution, line width=2pt] (23,2) -- (23,0.5);
			
			% رویدادهای پایین (اصلاحات)
			\node[rectangle, draw=bleurepublique, fill=bleulight, 
			minimum width=2cm, minimum height=0.8cm, font=\tiny, align=center]
			at (2.5,-2.5) {بانک فرانسه\\کنکوردا};
			\draw[->, >=Stealth, bleurepublique] (2.5,-2) -- (2.5,-0.5);
			
			\node[rectangle, draw=bleurepublique, fill=bleulight, 
			minimum width=2cm, minimum height=0.8cm, font=\tiny, align=center]
			at (5.5,-2.5) {کد ناپلئون\\لژیون دونور};
			\draw[->, >=Stealth, bleurepublique] (5.5,-2) -- (5.5,-0.5);
			
			\node[rectangle, draw=violetempire, fill=violetlight, 
			minimum width=2cm, minimum height=0.8cm, font=\tiny, align=center]
			at (8,-2.5) {نظام قاره‌ای\\(برلین)};
			\draw[->, >=Stealth, violetempire] (8,-2) -- (8,-0.5);
			
			\node[rectangle, draw=bleurepublique, fill=bleulight, 
			minimum width=2.5cm, minimum height=0.8cm, font=\tiny, align=center]
			at (11,-2.5) {دانشگاه\\امپراتوری};
			\draw[->, >=Stealth, bleurepublique] (11,-2) -- (11,-0.5);
			
			% راهنما
			\node[right, font=\footnotesize] at (0,-4.5) {
				\begin{tabular}{r@{\hspace{3pt}}l@{\hspace{15pt}}r@{\hspace{3pt}}l@{\hspace{15pt}}r@{\hspace{3pt}}l}
					\textcolor{bleurepublique}{$\blacksquare$} & اصلاحات/صلح &
					\textcolor{violetempire}{$\blacksquare$} & امپراتوری/توسعه &
					\textcolor{rougerevolution}{$\blacksquare$} & جنگ/شکست \\
				\end{tabular}
			};
			
		\end{tikzpicture}
		\caption{خط زمانی جامع دوره ناپلئونی (۱۷۹۹-۱۸۱۵)}
	\end{figure}
\end{landscape}
% ══════════════════════════════════════════════════════════════════════════════
%                    نقشه مفهومی میراث ناپلئون - صفحه افقی
% ══════════════════════════════════════════════════════════════════════════════

\begin{landscape}
	\begin{figure}[p]
		\centering
		\begin{tikzpicture}[
			scale=0.7,
			transform shape,
			legacy/.style={
				rectangle, draw=#1, line width=2pt, fill=#1,
				text=black, minimum width=4cm, minimum height=2.5cm,
				align=center, font=\small, rounded corners=5pt
			},
			sub/.style={
				rectangle, draw=#1, line width=1pt, fill=white,
				text=black, minimum width=3cm, minimum height=0.7cm,
				align=center, font=\scriptsize, rounded corners=2pt
			}
			]
			% عنوان
			\node[font=\Large\bfseries, text=violetempire] at (12,8) {
				میراث ناپلئون: تأثیر ماندگار
			};
			
			% مرکز
			\node[ellipse, draw=violetempire, line width=3pt, fill=violetlight,
			minimum width=5cm, minimum height=3cm, font=\large\bfseries]
			(center) at (12,4) {
				\begin{tabular}{c}
					میراث\\
					ناپلئون
				\end{tabular}
			};
			
			% ───────────── میراث حقوقی ─────────────
			\node[legacy=bleulight] (legal) at (3,6) {
				\begin{tabular}{c}
					\textbf{حقوقی}\\[5pt]
					کد ناپلئون\\
					برابری قانونی\\
					سکولاریسم
				\end{tabular}
			};
			\node[sub=bleurepublique] at (3,3.5) {هنوز پایه حقوق فرانسه};
			\node[sub=bleurepublique] at (3,2.7) {الگو برای ۷۰+ کشور};
			
			% ───────────── میراث اداری ─────────────
			\node[legacy=vertlight] (admin) at (21,6) {
				\begin{tabular}{c}
					\textbf{اداری}\\[5pt]
					نظام پرفه\\
					تمرکزگرایی\\
					شایسته‌سالاری
				\end{tabular}
			};
			\node[sub=vertnapoleon] at (21,3.5) {ساختار فعلی فرانسه};
			\node[sub=vertnapoleon] at (21,2.7) {الگوی دولت مدرن};
			
			% ───────────── میراث آموزشی ─────────────
			\node[legacy=orroyallight] (edu) at (3,0) {
				\begin{tabular}{c}
					\textbf{آموزشی}\\[5pt]
					لیسه\\
					دانشگاه دولتی\\
					اکول پلی‌تکنیک
				\end{tabular}
			};
			\node[sub=orroyaldark] at (3,-2.5) {نظام آموزشی فرانسه};
			
			% ───────────── میراث نظامی ─────────────
			\node[legacy=rougelight] (milit) at (21,0) {
				\begin{tabular}{c}
					\textbf{نظامی}\\[5pt]
					ارتش توده‌ای\\
					تاکتیک‌های نوین\\
					سربازگیری عمومی
				\end{tabular}
			};
			\node[sub=rougerevolution] at (21,-2.5) {الگوی جنگ مدرن};
			
			% ───────────── میراث ایدئولوژیک ─────────────
			\node[legacy=violetlight] (ideol) at (12,-1) {
				\begin{tabular}{c}
					\textbf{ایدئولوژیک}\\[5pt]
					ناسیونالیسم\\
					«مرد قوی»\\
					افسانه ناپلئونی
				\end{tabular}
			};
			\node[sub=violetempire] at (12,-3.5) {الگوی بناپارتیسم};
			\node[sub=violetempire] at (12,-4.3) {ناپلئون سوم، دوگل...};
			
			% خطوط اتصال
			\draw[bleurepublique, line width=2pt] (legal) -- (center);
			\draw[vertnapoleon, line width=2pt] (admin) -- (center);
			\draw[orroyaldark, line width=2pt] (edu) -- (center);
			\draw[rougerevolution, line width=2pt] (milit) -- (center);
			\draw[violetempire, line width=2pt] (ideol) -- (center);
			
		\end{tikzpicture}
		\caption{نقشه مفهومی: میراث ماندگار ناپلئون}
	\end{figure}
\end{landscape}	
	
	
\end{document}