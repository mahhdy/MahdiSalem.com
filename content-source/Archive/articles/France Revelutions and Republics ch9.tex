% ═══════════════════════════════════════════════════════════════════════════════
%                    تاریخ تحولات فرانسه - فصل [X]
% ═══════════════════════════════════════════════════════════════════════════════

\documentclass[12pt,a4paper]{book}

% ─────────────────────────── پکیج‌ها ───────────────────────────
\usepackage{amsmath,amssymb}
\usepackage{geometry}
\geometry{top=2.5cm, bottom=2.5cm, left=2cm, right=2.5cm, headheight=15pt}
\usepackage{graphicx}
\usepackage{array,booktabs,longtable,multirow,colortbl}
\usepackage{xcolor}
\usepackage{tikz}
\usetikzlibrary{shapes.geometric, arrows.meta, positioning, calc, backgrounds, 
	fit, decorations.pathmorphing, shadows, patterns}
\usepackage{pgfplots}
\pgfplotsset{compat=1.18}
\usepackage{tcolorbox}
\tcbuselibrary{skins,breakable}
\usepackage{enumitem}
\usepackage{fancyhdr}
\usepackage{pdflscape}
\usepackage{setspace}
\usepackage{titlesec}
\usepackage{float}
\usepackage{pdfpages}
\usepackage{pdflscape}  % برای صفحات landscape
\usepackage{hyperref}

% ─────────────────────────── رنگ‌ها ───────────────────────────
\definecolor{bleurepublique}{RGB}{0, 35, 149}
\definecolor{rougerevolution}{RGB}{237, 41, 57}
\definecolor{orroyal}{RGB}{255, 215, 0}
\definecolor{vertnapoleon}{RGB}{0, 100, 0}
\definecolor{violetempire}{RGB}{128, 0, 128}
\definecolor{fondclair}{RGB}{255, 253, 240}
\definecolor{gris}{RGB}{128, 128, 128}
\definecolor{grisclair}{RGB}{245, 245, 245}
\definecolor{noirsombre}{RGB}{30, 30, 30}

% رنگ‌های کمکی
\definecolor{bleulight}{RGB}{230, 235, 250}
\definecolor{rougelight}{RGB}{253, 235, 237}
\definecolor{vertlight}{RGB}{235, 250, 235}
\definecolor{violetlight}{RGB}{245, 235, 250}
\definecolor{orroyallight}{RGB}{255, 250, 230}
\definecolor{grislight}{RGB}{248, 248, 248}
\definecolor{bleumid}{RGB}{180, 195, 230}
\definecolor{rougemid}{RGB}{245, 180, 185}
\definecolor{vertmid}{RGB}{180, 220, 180}
\definecolor{violetmid}{RGB}{210, 180, 220}
\definecolor{orroyalmid}{RGB}{255, 240, 180}
\definecolor{orroyaldark}{RGB}{200, 170, 0}

% ─────────────────────────── فونت فارسی ───────────────────────────
\usepackage{fontspec}
\setmainfont{Vazirmatn}
\usepackage{xepersian}
\settextfont{Vazirmatn}
\setdigitfont{Vazirmatn}

% ─────────────────────────── هایپرلینک ───────────────────────────
\hypersetup{
	colorlinks=true,
	linkcolor=bleurepublique,
	urlcolor=bleurepublique,
	citecolor=vertnapoleon
}

% ─────────────────────────── کادرها ───────────────────────────
\newtcolorbox{kholasebox}[1][]{enhanced,breakable,colback=bleulight,
	colframe=bleurepublique,coltitle=white,fonttitle=\bfseries\large,
	title={#1},boxrule=2pt,arc=4pt,left=10pt,right=10pt,top=10pt,bottom=10pt,
	drop shadow={opacity=0.3}}

\newtcolorbox{naghlbox}[1][]{enhanced,breakable,colback=orroyallight,
	colframe=orroyaldark,coltitle=black,fonttitle=\bfseries,title={#1},
	boxrule=1.5pt,arc=3pt,borderline west={4pt}{0pt}{orroyal},
	left=15pt,right=10pt,top=8pt,bottom=8pt}

\newtcolorbox{olgoobox}[1][]{enhanced,breakable,colback=vertlight,
	colframe=vertnapoleon,coltitle=white,fonttitle=\bfseries,title={#1},
	boxrule=1.5pt,arc=4pt,left=10pt,right=10pt,top=8pt,bottom=8pt,
	before upper={\parindent15pt}}

\newtcolorbox{enghelabbox}[1][]{enhanced,breakable,colback=rougelight,
	colframe=rougerevolution,coltitle=white,fonttitle=\bfseries,title={#1},
	boxrule=2pt,arc=4pt,left=10pt,right=10pt,top=8pt,bottom=8pt}

\newtcolorbox{empirebox}[1][]{enhanced,breakable,colback=violetlight,
	colframe=violetempire,coltitle=white,fonttitle=\bfseries,title={#1},
	boxrule=1.5pt,arc=4pt,left=10pt,right=10pt,top=8pt,bottom=8pt}

\newtcolorbox{noktebox}[1][]{enhanced,colback=grisclair,colframe=gris,
	fonttitle=\bfseries,title={#1},boxrule=1pt,arc=3pt,left=8pt,right=8pt}

% ─────────────────────────── صفحه‌آرایی ───────────────────────────
\pagestyle{fancy}
\fancyhf{}
\fancyhead[RO]{\leftmark}
\fancyhead[LE]{\rightmark}
\fancyfoot[C]{\thepage}
\renewcommand{\headrulewidth}{1pt}
\renewcommand{\footrulewidth}{0.5pt}
\setstretch{1.5}

\titleformat{\chapter}[display]
{\normalfont\huge\bfseries\color{bleurepublique}}
{\chaptertitlename\ \thechapter}{20pt}{\Huge}
\titleformat{\section}
{\normalfont\Large\bfseries\color{bleurepublique}}{\thesection}{1em}{}
\titleformat{\subsection}
{\normalfont\large\bfseries\color{bleurepublique}}{\thesubsection}{1em}{}

% ═══════════════════════════════════════════════════════════════════════════════
\begin{document}
%══════════════════════════════════════════════════════════════════════════════
% فصل ۹: تحلیل تطبیقی
%══════════════════════════════════════════════════════════════════════════════

\chapter{تحلیل تطبیقی}
\label{chap:comparative-analysis}

\begin{kholasebox}
	\textbf{خلاصه فصل:}
	
	این فصل به تحلیل تطبیقی تاریخ سیاسی فرانسه از سه منظر می‌پردازد: نخست، مقایسه درونی میان رژیم‌های مختلف فرانسه از ۱۷۸۹ تا ۲۰۲۴؛ دوم، شناسایی الگوهای تکرارشونده در این تاریخ پرفرازونشیب؛ و سوم، مقایسه بیرونی با مسیرهای تحول سیاسی در کشورهای دیگر، به‌ویژه انگلستان، آلمان و ایالات متحده.
	
	\textbf{پرسش‌های محوری:}
	\begin{itemize}[nosep,rightmargin=0pt]
		\item چرا فرانسه ۱۵ رژیم مختلف داشته، در حالی که انگلستان همان پارلمان را حفظ کرده؟
		\item آیا الگویی در انقلاب‌های فرانسه وجود دارد؟
		\item چه عواملی ثبات یا بی‌ثباتی رژیم‌ها را تعیین کرده‌اند؟
		\item «استثنای فرانسوی» چیست و آیا واقعی است؟
	\end{itemize}
	
	\textbf{رویکرد:} تحلیل ساختاری، نهادی، فرهنگی و تطبیقی.
\end{kholasebox}

%──────────────────────────────────────────────────────────────────────────────
\section{مقایسه رژیم‌های فرانسه}
%──────────────────────────────────────────────────────────────────────────────

\subsection{فهرست رژیم‌ها از ۱۷۸۹}

\begin{table}[htbp]
	\centering
	\caption{رژیم‌های فرانسه از ۱۷۸۹ تا ۲۰۲۴}
	\label{tab:all-regimes}
	\begin{tabular}{|c|r|c|c|c|}
		\hline
		\rowcolor{bleumid}
		\textcolor{white}{\textbf{ش.}} & \textcolor{white}{\textbf{رژیم}} & \textcolor{white}{\textbf{دوره}} & \textcolor{white}{\textbf{مدت}} & \textcolor{white}{\textbf{پایان}} \\
		\hline
		۱ & سلطنت مشروطه & ۱۷۸۹-۱۷۹۲ & ۳ سال & انقلاب \\
		\hline
		\rowcolor{rougelight}
		۲ & جمهوری اول & ۱۷۹۲-۱۷۹۹ & ۷ سال & کودتا \\
		\hline
		۳ & کنسولگری & ۱۷۹۹-۱۸۰۴ & ۵ سال & تبدیل به امپراتوری \\
		\hline
		\rowcolor{violetlight}
		۴ & امپراتوری اول & ۱۸۰۴-۱۸۱۴ & ۱۰ سال & شکست نظامی \\
		\hline
		۵ & بازگشت بوربون (اول) & ۱۸۱۴-۱۸۱۵ & ۱ سال & صد روز \\
		\hline
		\rowcolor{violetlight}
		۶ & صد روز & ۱۸۱۵ & ۱۰۰ روز & شکست نظامی \\
		\hline
		۷ & بازگشت بوربون (دوم) & ۱۸۱۵-۱۸۳۰ & ۱۵ سال & انقلاب \\
		\hline
		\rowcolor{bleulight}
		۸ & سلطنت ژوئیه & ۱۸۳۰-۱۸۴۸ & ۱۸ سال & انقلاب \\
		\hline
		۹ & جمهوری دوم & ۱۸۴۸-۱۸۵۲ & ۴ سال & کودتا \\
		\hline
		\rowcolor{violetlight}
		۱۰ & امپراتوری دوم & ۱۸۵۲-۱۸۷۰ & ۱۸ سال & شکست نظامی \\
		\hline
		۱۱ & جمهوری سوم & ۱۸۷۰-۱۹۴۰ & ۷۰ سال & شکست نظامی \\
		\hline
		\rowcolor{orroyallight}
		۱۲ & ویشی & ۱۹۴۰-۱۹۴۴ & ۴ سال & آزادی \\
		\hline
		۱۳ & دولت موقت & ۱۹۴۴-۱۹۴۶ & ۲ سال & قانون اساسی \\
		\hline
		\rowcolor{bleulight}
		۱۴ & جمهوری چهارم & ۱۹۴۶-۱۹۵۸ & ۱۲ سال & بحران الجزایر \\
		\hline
		۱۵ & جمهوری پنجم & ۱۹۵۸-... & ۶۶+ سال & ادامه‌دار \\
		\hline
	\end{tabular}
\end{table}

\begin{noktebox}
	\textbf{محاسبه:}
	
	در ۲۳۵ سال (۱۷۸۹-۲۰۲۴)، فرانسه ۱۵ رژیم مختلف داشته است—یعنی به‌طور میانگین هر ۱۵.۷ سال یک رژیم جدید. البته این میانگین گمراه‌کننده است، زیرا جمهوری سوم (۷۰ سال) و پنجم (۶۶+ سال) بخش بزرگی از این دوره را پوشش می‌دهند.
\end{noktebox}

\subsection{مقایسه پنج جمهوری}

\begin{landscape}
	\begin{table}[htbp]
		\centering
		\caption{مقایسه تفصیلی پنج جمهوری فرانسه}
		\label{tab:five-republics-comparison}
		\begin{tabular}{|r|p{3.8cm}|p{3.8cm}|p{3.8cm}|p{3.8cm}|p{3.8cm}|}
			\hline
			\rowcolor{bleumid}
			\textcolor{white}{\textbf{ویژگی}} & \textcolor{white}{\textbf{جمهوری اول}} & \textcolor{white}{\textbf{جمهوری دوم}} & \textcolor{white}{\textbf{جمهوری سوم}} & \textcolor{white}{\textbf{جمهوری چهارم}} & \textcolor{white}{\textbf{جمهوری پنجم}} \\
			\hline
			\textbf{دوره} & ۱۷۹۲-۱۷۹۹ & ۱۸۴۸-۱۸۵۲ & ۱۸۷۰-۱۹۴۰ & ۱۹۴۶-۱۹۵۸ & ۱۹۵۸-... \\
			\hline
			\rowcolor{bleulight}
			\textbf{مدت} & ۷ سال & ۴ سال & ۷۰ سال & ۱۲ سال & ۶۶+ سال \\
			\hline
			\textbf{زایش} & سقوط سلطنت & انقلاب ۱۸۴۸ & شکست سدان & آزادی & بحران الجزایر \\
			\hline
			\rowcolor{bleulight}
			\textbf{پایان} & کودتای ناپلئون & کودتای لویی-ناپلئون & شکست ۱۹۴۰ & کودتای ۱۳ مه & ادامه‌دار \\
			\hline
			\textbf{قوه مجریه} & کمیته‌ها/دیرکتوار & رئیس‌جمهور قوی & رئیس‌جمهور ضعیف & رئیس‌جمهور ضعیف & رئیس‌جمهور قوی \\
			\hline
			\rowcolor{bleulight}
			\textbf{قوه مقننه} & کنوانسیون/شوراها & یک‌مجلسی & دومجلسی قوی & مجلس ملی قوی & پارلمان محدود \\
			\hline
			\textbf{حق رأی} & مردان (متغیر) & همگانی مردان & همگانی مردان & همگانی & همگانی \\
			\hline
			\rowcolor{bleulight}
			\textbf{تعداد قوانین اساسی} & ۳ & ۱ & ۳ قانون ۱۸۷۵ & ۱ & ۱ (۲۴ اصلاحیه) \\
			\hline
			\textbf{بحران اصلی} & ترور، جنگ & روزهای ژوئن & دریفوس، جنگ اول & هندوچین، الجزایر & مه ۶۸، الجزایر \\
			\hline
			\rowcolor{bleulight}
			\textbf{میراث} & نمادها، ایده‌ها & رأی همگانی & لائیسیته، آموزش & تأمین اجتماعی & ثبات نهادی \\
			\hline
		\end{tabular}
	\end{table}
\end{landscape}

\begin{tikzpicture}[
	every node/.style={font=\small}
	]
	% Title
	\node[font=\bfseries\large] at (7,8) {طول عمر جمهوری‌ها (سال)};
	
	% Bars
	\fill[rougelight] (1,0) rectangle (2,0.7);
	\fill[rougelight] (3,0) rectangle (4,0.4);
	\fill[bleulight] (5,0) rectangle (6,7);
	\fill[bleulight] (7,0) rectangle (8,1.2);
	\fill[bleumid] (9,0) rectangle (10,6.6);
	
	% Labels
	\node at (1.5,-0.5) {اول};
	\node at (3.5,-0.5) {دوم};
	\node at (5.5,-0.5) {سوم};
	\node at (7.5,-0.5) {چهارم};
	\node at (9.5,-0.5) {پنجم};
	
	% Values
	\node at (1.5,1) {۷};
	\node at (3.5,0.7) {۴};
	\node at (5.5,7.3) {۷۰};
	\node at (7.5,1.5) {۱۲};
	\node at (9.5,6.9) {۶۶+};
	
	% Axis
	\draw[thick] (0,0) -- (11,0);
	\draw[thick] (0,0) -- (0,7.5);
	
	% Scale
	\foreach \y in {1,2,3,4,5,6,7} {
		\draw (0,\y) -- (-0.1,\y);
		\node[left] at (-0.1,\y) {\footnotesize \y۰};
	}
	
\end{tikzpicture}

\subsection{چرا برخی جمهوری‌ها دوام آوردند؟}

\begin{tikzpicture}[
	every node/.style={font=\small},
	factor/.style={rectangle, rounded corners, draw=vertnapoleon, fill=vertlight,
		minimum width=4cm, minimum height=1.5cm, align=center}
	]
	% Title
	\node[font=\bfseries\large] at (7,8) {عوامل دوام جمهوری سوم و پنجم};
	
	% Third Republic factors
	\node[font=\bfseries, bleurepublique] at (3,6.5) {جمهوری سوم (۷۰ سال)};
	
	\node[factor] at (3,5) {فقدان جایگزین\\اجماعی};
	\node[factor] at (3,3) {انعطاف‌پذیری\\بدون قانون اساسی سخت};
	\node[factor] at (3,1) {ائتلاف دفاعی\\علیه دشمنان مشترک};
	
	% Fifth Republic factors
	\node[font=\bfseries, bleurepublique] at (11,6.5) {جمهوری پنجم (۶۶+ سال)};
	
	\node[factor] at (11,5) {رهبری قوی\\ریاست‌جمهوری مستقیم};
	\node[factor] at (11,3) {ثبات نهادی\\پارلمانتاریسم عقلانی};
	\node[factor] at (11,1) {چرخش موفق\\چپ و راست};
	
	% Common
	\node[rectangle, draw=rougerevolution, fill=rougelight, rounded corners,
	minimum width=5cm, minimum height=1cm, align=center] at (7,-1) {
		\textbf{مشترک:} بقا از بحران‌های بزرگ\\
		(دریفوس/جنگ اول — مه ۶۸/الجزایر)
	};
	
\end{tikzpicture}

%──────────────────────────────────────────────────────────────────────────────
\section{الگوهای تکرارشونده}
%──────────────────────────────────────────────────────────────────────────────

\subsection{الگوی انقلاب فرانسوی}

\begin{tikzpicture}[
	every node/.style={font=\small},
	phase/.style={rectangle, rounded corners, draw=rougerevolution, fill=rougelight,
		minimum width=3.5cm, minimum height=1.5cm, align=center}
	]
	% Title
	\node[font=\bfseries\large] at (7,8.5) {الگوی چرخه‌ای انقلاب‌های فرانسه};
	
	% Cycle
	\node[phase] (p1) at (2,6) {\textbf{۱. بحران}\\اقتصادی/سیاسی\\انسداد نظام};
	
	\node[phase] (p2) at (7,7) {\textbf{۲. انقلاب}\\شورش شهری\\سقوط رژیم};
	
	\node[phase] (p3) at (12,6) {\textbf{۳. رادیکالیزاسیون}\\افراطی‌شدن\\ترور یا سرکوب};
	
	\node[phase] (p4) at (12,2) {\textbf{۴. واکنش}\\خستگی از افراط\\خواست نظم};
	
	\node[phase] (p5) at (7,1) {\textbf{۵. تثبیت}\\رژیم جدید\\سازش یا اقتدار};
	
	\node[phase] (p6) at (2,2) {\textbf{۶. فرسایش}\\کاهش مشروعیت\\انباشت مشکلات};
	
	% Arrows
	\draw[->, very thick, rougerevolution] (p1) -- (p2);
	\draw[->, very thick, rougerevolution] (p2) -- (p3);
	\draw[->, very thick, rougerevolution] (p3) -- (p4);
	\draw[->, very thick, rougerevolution] (p4) -- (p5);
	\draw[->, very thick, rougerevolution] (p5) -- (p6);
	\draw[->, very thick, rougerevolution] (p6) -- (p1);
	
	% Center note
	\node[align=center] at (7,4) {این الگو در\\۱۷۸۹، ۱۸۳۰، ۱۸۴۸\\تکرار شد};
	
\end{tikzpicture}

\begin{table}[htbp]
	\centering
	\caption{مقایسه سه انقلاب بزرگ}
	\label{tab:three-revolutions}
	\begin{tabular}{|r|p{4cm}|p{4cm}|p{4cm}|}
		\hline
		\rowcolor{rougemid}
		\textcolor{white}{\textbf{جنبه}} & \textcolor{white}{\textbf{۱۷۸۹}} & \textcolor{white}{\textbf{۱۸۳۰}} & \textcolor{white}{\textbf{۱۸۴۸}} \\
		\hline
		\textbf{رژیم ساقط‌شده} & سلطنت مطلقه & بازگشت بوربون & سلطنت ژوئیه \\
		\hline
		\rowcolor{rougelight}
		\textbf{علت مستقیم} & بحران مالی + سیاسی & فرمان‌های ژوئیه & ممنوعیت ضیافت \\
		\hline
		\textbf{بازیگران} & بورژوازی + توده & بورژوازی + پیشه‌وران & همه طبقات \\
		\hline
		\rowcolor{rougelight}
		\textbf{مدت درگیری} & ماه‌ها & ۳ روز & ۳ روز \\
		\hline
		\textbf{نتیجه فوری} & سلطنت مشروطه & سلطنت ژوئیه & جمهوری دوم \\
		\hline
		\rowcolor{rougelight}
		\textbf{نتیجه نهایی} & امپراتوری & جمهوری (۱۸۴۸) & امپراتوری دوم \\
		\hline
		\textbf{«ربوده شد» توسط} & ناپلئون & لویی-فیلیپ & لویی-ناپلئون \\
		\hline
	\end{tabular}
\end{table}

\subsection{الگوی «انقلاب ربوده‌شده»}

\begin{olgoobox}
	\textbf{الگوی تکرارشونده: انقلاب ربوده‌شده}
	
	در هر سه انقلاب بزرگ فرانسه (۱۷۸۹، ۱۸۳۰، ۱۸۴۸)، الگوی مشابهی دیده می‌شود:
	
	\begin{enumerate}
		\item \textbf{کسانی که می‌جنگند ≠ کسانی که پیروز می‌شوند}
		\begin{itemize}[nosep]
			\item ۱۷۸۹: توده‌ها باستیل را گرفتند، بورژوازی قدرت را گرفت
			\item ۱۸۳۰: کارگران جنگیدند، بانکداران پادشاه انتخاب کردند
			\item ۱۸۴۸: مردم جمهوری خواستند، ناپلئون امپراتور شد
		\end{itemize}
		
		\item \textbf{«میانه‌روها» در لحظه بحرانی ابتکار را می‌گیرند}
		\begin{itemize}[nosep]
			\item لافایت (۱۷۸۹، ۱۸۳۰)
			\item تی‌یر (۱۸۳۰)
			\item لامارتین (۱۸۴۸)
		\end{itemize}
		
		\item \textbf{ترس از «افراط» به واکنش می‌انجامد}
		\begin{itemize}[nosep]
			\item ترس از ترور → ناپلئون
			\item ترس از جمهوری → لویی-فیلیپ
			\item ترس از سوسیالیسم → لویی-ناپلئون
		\end{itemize}
	\end{enumerate}
\end{olgoobox}

\subsection{الگوی «شکست نظامی = تغییر رژیم»}

\begin{table}[htbp]
	\centering
	\caption{شکست‌های نظامی و تغییر رژیم}
	\label{tab:military-defeats}
	\begin{tabular}{|c|r|r|r|}
		\hline
		\rowcolor{rougemid}
		\textcolor{white}{\textbf{سال}} & \textcolor{white}{\textbf{شکست}} & \textcolor{white}{\textbf{رژیم ساقط‌شده}} & \textcolor{white}{\textbf{رژیم جدید}} \\
		\hline
		۱۸۱۴ & لایپزیگ، پاریس & امپراتوری اول & بازگشت بوربون \\
		\hline
		\rowcolor{rougelight}
		۱۸۱۵ & واترلو & صد روز & بازگشت دوم \\
		\hline
		۱۸۷۰ & سدان & امپراتوری دوم & جمهوری سوم \\
		\hline
		\rowcolor{rougelight}
		۱۹۴۰ & شکست ۶ هفته‌ای & جمهوری سوم & ویشی \\
		\hline
	\end{tabular}
\end{table}

\begin{noktebox}
	\textbf{نکته:}
	
	شکست نظامی تنها در صورتی به تغییر رژیم می‌انجامد که:
	\begin{itemize}[nosep]
		\item رژیم مشروعیت خود را از قدرت نظامی یا افتخار ملی بگیرد (امپراتوری‌ها)
		\item جایگزینی آماده باشد
		\item شکست «تحقیرآمیز» تلقی شود
	\end{itemize}
	
	جمهوری سوم جنگ جهانی اول را با تلفات هولناک برد و بقا یافت—اما شکست سریع ۱۹۴۰ آن را نابود کرد.
\end{noktebox}

\subsection{الگوی «مرد قوی»}

\begin{tikzpicture}[
	every node/.style={font=\small},
	leader/.style={rectangle, rounded corners, draw=violetempire, fill=violetlight,
		minimum width=3.5cm, minimum height=2cm, align=center}
	]
	% Title
	\node[font=\bfseries\large] at (7,8) {الگوی «مرد قوی» در تاریخ فرانسه};
	
	% Leaders
	\node[leader] (n1) at (2,5) {\textbf{ناپلئون اول}\\۱۷۹۹-۱۸۱۵\\پایان انقلاب\\افتخار + نظم};
	
	\node[leader] (n3) at (7,5) {\textbf{ناپلئون سوم}\\۱۸۵۲-۱۸۷۰\\پایان جمهوری دوم\\مدرنیزاسیون + اقتدار};
	
	\node[leader] (dg) at (12,5) {\textbf{دوگل}\\۱۹۵۸-۱۹۶۹\\پایان جمهوری چهارم\\ثبات + استقلال};
	
	% Common pattern
	\node[rectangle, draw=gris, fill=grisclair, rounded corners,
	minimum width=12cm, minimum height=2cm, align=center] at (7,1.5) {
		\textbf{الگوی مشترک:}\\
		بحران → خواست «نجات‌دهنده» → واگذاری قدرت به فرد\\
		تمرکز قدرت + همه‌پرسی‌های مشروعیت‌بخش + «فراتر از احزاب»
	};
	
	% Timeline arrows
	\draw[->, thick, violetempire] (n1) -- (n3) node[midway, above] {\footnotesize میراث نام};
	\draw[->, thick, violetempire] (n3) -- (dg) node[midway, above] {\footnotesize میراث ساختار};
	
\end{tikzpicture}

\subsection{الگوی «پاریس در برابر شهرستان»}

\begin{table}[htbp]
	\centering
	\caption{تنش پاریس-شهرستان در رویدادهای کلیدی}
	\label{tab:paris-province}
	\begin{tabular}{|c|p{5cm}|p{5cm}|}
		\hline
		\rowcolor{bleumid}
		\textcolor{white}{\textbf{رویداد}} & \textcolor{white}{\textbf{پاریس}} & \textcolor{white}{\textbf{شهرستان}} \\
		\hline
		۱۷۸۹-۱۷۹۴ & انقلابی، رادیکال & مقاومت (واندی، فدرالیسم) \\
		\hline
		\rowcolor{bleulight}
		۱۸۴۸ & جمهوری‌خواه، سوسیالیست & محافظه‌کار، رأی به لویی-ناپلئون \\
		\hline
		۱۸۷۱ & کمون & ارتش ورسای \\
		\hline
		\rowcolor{bleulight}
		۱۹۴۰ & اشغال‌شده & ویشی \\
		\hline
		۲۰۱۸ & ماکرونیست & جلیقه‌زردها \\
		\hline
	\end{tabular}
\end{table}

\begin{naghlbox}
	«پاریس انقلاب می‌کند، شهرستان‌ها رأی می‌دهند.»
	
	\hfill --- \textit{ضرب‌المثل سیاسی فرانسوی}
\end{naghlbox}

%──────────────────────────────────────────────────────────────────────────────
\section{مقایسه با کشورهای دیگر}
%──────────────────────────────────────────────────────────────────────────────

\subsection{فرانسه در برابر انگلستان}

\begin{tikzpicture}[
	every node/.style={font=\small},
	country/.style={rectangle, rounded corners, minimum width=6cm, minimum height=4cm, align=center}
	]
	% Title
	\node[font=\bfseries\large] at (7,8) {دو مسیر: فرانسه و انگلستان};
	
	% France
	\node[country, draw=bleurepublique, fill=bleulight] (fr) at (3,4) {
		\textbf{فرانسه}\\[0.3em]
		انقلاب‌های پیاپی\\
		گسست با گذشته\\
		ایدئولوژی‌محور\\
		۱۵ رژیم\\
		جمهوری vs سلطنت
	};
	
	% Britain
	\node[country, draw=orroyaldark, fill=orroyallight] (uk) at (11,4) {
		\textbf{انگلستان}\\[0.3em]
		تحول تدریجی\\
		تداوم با گذشته\\
		عمل‌گرا\\
		همان پارلمان (از ۱۶۸۹)\\
		سلطنت + دموکراسی
	};
	
	% Key difference
	\node[rectangle, draw=rougerevolution, fill=rougelight, rounded corners,
	minimum width=10cm, minimum height=1.5cm, align=center] at (7,0) {
		\textbf{تفاوت کلیدی:} انگلستان انقلاب «شکوهمند» ۱۶۸۸ را داشت\\
		که سازش اشرافیت-بورژوازی را نهادینه کرد—فرانسه نداشت
	};
	
\end{tikzpicture}

\begin{table}[htbp]
	\centering
	\caption{مقایسه تفصیلی فرانسه و انگلستان}
	\label{tab:france-britain-comparison}
	\begin{tabular}{|r|p{5.5cm}|p{5.5cm}|}
		\hline
		\rowcolor{bleumid}
		\textcolor{white}{\textbf{جنبه}} & \textcolor{white}{\textbf{فرانسه}} & \textcolor{white}{\textbf{انگلستان}} \\
		\hline
		\textbf{انقلاب اصلی} & ۱۷۸۹ — رادیکال، خشونت‌آمیز & ۱۶۸۸ — «شکوهمند»، بدون خونریزی \\
		\hline
		\rowcolor{bleulight}
		\textbf{سرنوشت پادشاه} & اعدام لویی شانزدهم & تبعید جیمز دوم \\
		\hline
		\textbf{قانون اساسی} & مدوّن، متعدد & نامدوّن، تکاملی \\
		\hline
		\rowcolor{bleulight}
		\textbf{اشرافیت} & لغو امتیازات، انقلاب ضد آن & ادغام تدریجی با بورژوازی \\
		\hline
		\textbf{کلیسا} & جدایی (۱۹۰۵)، ضدکلریکالیسم & کلیسای دولتی، سازش \\
		\hline
		\rowcolor{bleulight}
		\textbf{مفهوم شهروندی} & انتزاعی، حقوقی، برابر & تجربی، تاریخی، طبقاتی \\
		\hline
		\textbf{نقش دولت} & قوی، متمرکز & محدود، غیرمتمرکز (تا ۱۹۴۵) \\
		\hline
		\rowcolor{bleulight}
		\textbf{نتیجه} & بی‌ثباتی طولانی، سپس ثبات & ثبات طولانی، تحول تدریجی \\
		\hline
	\end{tabular}
\end{table}

\subsection{فرانسه در برابر آلمان}

\begin{table}[htbp]
	\centering
	\caption{مقایسه مسیرهای فرانسه و آلمان}
	\label{tab:france-germany}
	\begin{tabular}{|r|p{5.5cm}|p{5.5cm}|}
		\hline
		\rowcolor{violetmid}
		\textcolor{white}{\textbf{جنبه}} & \textcolor{white}{\textbf{فرانسه}} & \textcolor{white}{\textbf{آلمان}} \\
		\hline
		\textbf{وحدت ملی} & زودهنگام (قرن‌ها پیش) & دیرهنگام (۱۸۷۱) \\
		\hline
		\rowcolor{violetlight}
		\textbf{انقلاب ۱۸۴۸} & موفق (موقتاً) & شکست \\
		\hline
		\textbf{دموکراسی} & از انقلاب (۱۷۸۹+) & از شکست (۱۹۱۸، ۱۹۴۵) \\
		\hline
		\rowcolor{violetlight}
		\textbf{فاشیسم} & ویشی (تحمیلی/همکاری) & نازیسم (بومی) \\
		\hline
		\textbf{نظام فعلی} & ریاستی-پارلمانی & پارلمانی فدرال \\
		\hline
		\rowcolor{violetlight}
		\textbf{رهبری قوی} & ارزش مثبت (دوگل) & ارزش منفی (تجربه هیتلر) \\
		\hline
	\end{tabular}
\end{table}

\begin{noktebox}
	\textbf{پارادوکس فرانسه-آلمان:}
	
	فرانسه که انقلاب دموکراتیک داشت، ۱۵ رژیم عوض کرد. آلمان که انقلاب ۱۸۴۸ را شکست داد و به فاشیسم افتاد، امروز دموکراسی پایداری دارد. 
	
	درس: مسیر رسیدن به دموکراسی، نتیجه نهایی را تعیین نمی‌کند.
\end{noktebox}

\subsection{فرانسه در برابر ایالات متحده}

\begin{table}[htbp]
	\centering
	\caption{مقایسه دو انقلاب: فرانسه و آمریکا}
	\label{tab:france-usa}
	\begin{tabular}{|r|p{5.5cm}|p{5.5cm}|}
		\hline
		\rowcolor{vertmid}
		\textcolor{white}{\textbf{جنبه}} & \textcolor{white}{\textbf{فرانسه (۱۷۸۹)}} & \textcolor{white}{\textbf{آمریکا (۱۷۷۶)}} \\
		\hline
		\textbf{هدف} & تغییر جامعه موجود & ایجاد جامعه جدید \\
		\hline
		\rowcolor{vertlight}
		\textbf{دشمن} & رژیم قدیم داخلی & قدرت استعماری خارجی \\
		\hline
		\textbf{خشونت} & ترور، جنگ داخلی & جنگ استقلال \\
		\hline
		\rowcolor{vertlight}
		\textbf{قانون اساسی} & متعدد، ناپایدار & یکی، پایدار (۱۷۸۷) \\
		\hline
		\textbf{مذهب} & ضدکلریکالیسم، لائیسیته & آزادی مذهبی، تکثر \\
		\hline
		\rowcolor{vertlight}
		\textbf{تمرکز} & دولت متمرکز & فدرالیسم \\
		\hline
		\textbf{میراث} & ایدئولوژیک، جهانی & نهادی، آمریکایی \\
		\hline
	\end{tabular}
\end{table}

\begin{naghlbox}
	«انقلاب آمریکا محافظه‌کارانه بود—حفظ حقوق موجود. انقلاب فرانسه رادیکال بود—ایجاد انسان و جامعه جدید.»
	
	\hfill --- \textit{هانا آرنت، «درباره انقلاب»}
\end{naghlbox}

%──────────────────────────────────────────────────────────────────────────────
\section{نقشه مفهومی: عوامل ثبات و بی‌ثباتی}
%──────────────────────────────────────────────────────────────────────────────

\begin{landscape}
	\begin{tikzpicture}[
		every node/.style={font=\small},
		factor/.style={rectangle, rounded corners, minimum width=3.5cm, minimum height=1.3cm, align=center}
		]
		% Title
		\node[font=\bfseries\large] at (12,10) {عوامل ثبات و بی‌ثباتی رژیم‌های فرانسه};
		
		% Stability factors (left)
		\node[font=\bfseries, vertnapoleon] at (5,8.5) {عوامل ثبات‌بخش};
		
		\node[factor, draw=vertnapoleon, fill=vertlight] at (2,7) {اجماع حداقلی\\بر قواعد بازی};
		\node[factor, draw=vertnapoleon, fill=vertlight] at (6,7) {انعطاف نهادی\\توان سازگاری};
		\node[factor, draw=vertnapoleon, fill=vertlight] at (10,7) {مشروعیت\\پذیرش عمومی};
		
		\node[factor, draw=vertnapoleon, fill=vertlight] at (2,5) {چرخش مسالمت‌آمیز\\قدرت};
		\node[factor, draw=vertnapoleon, fill=vertlight] at (6,5) {کارایی اقتصادی\\رفاه عمومی};
		\node[factor, draw=vertnapoleon, fill=vertlight] at (10,5) {مدیریت بحران\\بدون فروپاشی};
		
		% Instability factors (right)
		\node[font=\bfseries, rougerevolution] at (19,8.5) {عوامل بی‌ثباتی};
		
		\node[factor, draw=rougerevolution, fill=rougelight] at (14,7) {شکاف‌های عمیق\\طبقاتی/ایدئولوژیک};
		\node[factor, draw=rougerevolution, fill=rougelight] at (18,7) {سختی نهادی\\عدم انعطاف};
		\node[factor, draw=rougerevolution, fill=rougelight] at (22,7) {بحران مشروعیت\\چالش اساسی};
		
		\node[factor, draw=rougerevolution, fill=rougelight] at (14,5) {شکست نظامی\\تحقیر ملی};
		\node[factor, draw=rougerevolution, fill=rougelight] at (18,5) {بحران اقتصادی\\نارضایتی گسترده};
		\node[factor, draw=rougerevolution, fill=rougelight] at (22,5) {ناتوانی در\\حل مسائل اصلی};
		
		% Balance
		\node[rectangle, draw=bleurepublique, fill=bleumid, text=white, rounded corners,
		minimum width=8cm, minimum height=1.5cm, align=center] at (12,2) {
			\textbf{تعادل:} ثبات زمانی ممکن است که عوامل مثبت\\
			بر عوامل منفی غلبه کنند—یا بحران‌ها مدیریت شوند
		};
		
		% Example arrows
		\draw[->, thick, vertnapoleon] (6,4.3) -- (10,2.8);
		\draw[->, thick, rougerevolution] (18,4.3) -- (14,2.8);
		
	\end{tikzpicture}
\end{landscape}

%──────────────────────────────────────────────────────────────────────────────
\section{«استثنای فرانسوی» چیست؟}
%──────────────────────────────────────────────────────────────────────────────

\begin{olgoobox}
	\textbf{استثنای فرانسوی (\lr{Exception française})}
	
	برخی ویژگی‌هایی که فرانسه را از دیگر دموکراسی‌های غربی متمایز می‌کند:
	
	\begin{enumerate}
		\item \textbf{میراث انقلابی:}
		\begin{itemize}[nosep]
			\item انقلاب ۱۷۸۹ به عنوان «مبدأ تاریخ»
			\item حق شورش در فرهنگ سیاسی
			\item تقدس‌زدایی رادیکال از سلطنت
		\end{itemize}
		
		\item \textbf{لائیسیته:}
		\begin{itemize}[nosep]
			\item جدایی سخت‌گیرانه دین و دولت
			\item متفاوت از سکولاریسم آمریکایی یا انگلیسی
			\item دین = امر خصوصی
		\end{itemize}
		
		\item \textbf{دولت متمرکز:}
		\begin{itemize}[nosep]
			\item سنت دولت قوی از قرن هفدهم
			\item ژاکوبنیسم: دولت = ملت
			\item مقاومت در برابر فدرالیسم
		\end{itemize}
		
		\item \textbf{ایدئولوژی‌محوری:}
		\begin{itemize}[nosep]
			\item تفکر انتزاعی، اصول کلی
			\item «چپ» و «راست» اختراع فرانسوی است
			\item روشنفکران در سیاست
		\end{itemize}
		
		\item \textbf{فرهنگ تظاهرات:}
		\begin{itemize}[nosep]
			\item اعتصاب و تظاهرات = ابزار عادی سیاست
			\item نه فقط انتخابات
			\item «خیابان» به عنوان بازیگر سیاسی
		\end{itemize}
	\end{enumerate}
\end{olgoobox}

\begin{naghlbox}
	«دیگر کشورها تاریخ دارند؛ فرانسه ایدئولوژی دارد.»
	
	\hfill --- \textit{سخن منسوب به ناپلئون}
\end{naghlbox}

%──────────────────────────────────────────────────────────────────────────────
\section{جمع‌بندی: درس‌های تاریخ فرانسه}
%──────────────────────────────────────────────────────────────────────────────

\begin{kholasebox}
	\textbf{جمع‌بندی: درس‌های تحلیل تطبیقی}
	
	\textbf{درس‌های درونی (از تاریخ فرانسه):}
	\begin{itemize}[nosep]
		\item ثبات نهادی نیازمند اجماع حداقلی است
		\item انعطاف‌پذیری بهتر از سختی است (جمهوری سوم vs دوم)
		\item «مرد قوی» می‌تواند ثبات بیاورد، اما با ریسک اقتدارگرایی
		\item شکست نظامی می‌تواند رژیم را بیندازد
		\item انقلاب‌ها اغلب «ربوده» می‌شوند
	\end{itemize}
	
	\textbf{درس‌های بیرونی (از مقایسه):}
	\begin{itemize}[nosep]
		\item تحول تدریجی (انگلستان) پایدارتر از انقلاب رادیکال است
		\item مسیر رسیدن به دموکراسی، سرنوشت نهایی را تعیین نمی‌کند
		\item قانون اساسی نوشته ≠ ثبات (فرانسه vs آمریکا)
		\item هر کشور مسیر خاص خود را دارد
	\end{itemize}
	
	\textbf{پرسش باقیمانده:}
	
	آیا جمهوری پنجم «پایان تاریخ» فرانسه است؟ یا باز هم چرخه تکرار خواهد شد؟
\end{kholasebox}

%──────────────────────────────────────────────────────────────────────────────
\section*{منابع فصل}
%──────────────────────────────────────────────────────────────────────────────
\addcontentsline{toc}{section}{منابع فصل}

\begin{itemize}[nosep]
	\item Arendt, Hannah. \textit{On Revolution}. New York: Viking, 1963.
	\item Berger, Suzanne. \textit{Peasants against Politics: Rural Organization in Brittany}. Cambridge: Harvard UP, 1972.
	\item Furet, François. \textit{Interpreting the French Revolution}. Cambridge: Cambridge UP, 1981.
	\item Hoffmann, Stanley, et al. \textit{In Search of France}. Cambridge: Harvard UP, 1963.
	\item Lijphart, Arend. \textit{Patterns of Democracy}. 2nd ed. New Haven: Yale UP, 2012.
	\item Moore, Barrington. \textit{Social Origins of Dictatorship and Democracy}. Boston: Beacon, 1966.
	\item Rosanvallon, Pierre. \textit{The Demands of Liberty: Civil Society in France since the Revolution}. Cambridge: Harvard UP, 2007.
	\item Skocpol, Theda. \textit{States and Social Revolutions}. Cambridge: Cambridge UP, 1979.
	\item Tocqueville, Alexis de. \textit{The Old Regime and the Revolution}. Trans. Stuart Gilbert. New York: Anchor, 1955.
	\item Weber, Eugen. \textit{Peasants into Frenchmen: The Modernization of Rural France, 1870-1914}. Stanford: Stanford UP, 1976.
\end{itemize}

%══════════════════════════════════════════════════════════════════════════════
% پایان فصل ۹
%══════════════════════════════════════════════════════════════════════════════
\end{document}