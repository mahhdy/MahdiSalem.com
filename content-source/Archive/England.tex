%%%%%%%%%%%%%%%%%%%%%%%%%%%%%%%%%%%%%%%%%%%%%%%%%%%%%%%%%%%%%%%%%%%%%%%
% پژوهش جامع: تطور طبقاتی و توسعه در بریتانیا
% تحلیل هزارساله نقش طبقات، دربار و نیروهای اجتماعی در پیشرفت
%%%%%%%%%%%%%%%%%%%%%%%%%%%%%%%%%%%%%%%%%%%%%%%%%%%%%%%%%%%%%%%%%%%%%%%

\documentclass[12pt,a4paper,openany]{book}
%%%%%%%%%%%%%%%%%%%%%%%%%%%%%%%%%%%%%%%%%%%%%%%%%%%%%%%%%%%%%%%%%%%%%%%
% بسته‌های مورد نیاز
%%%%%%%%%%%%%%%%%%%%%%%%%%%%%%%%%%%%%%%%%%%%%%%%%%%%%%%%%%%%%%%%%%%%%%%

% زبان و فونت
\usepackage{fontspec}
\usepackage{polyglossia}
\setmainlanguage{persian}
\setotherlanguage{english}
\setmainfont{B Nazanin}[
    Script=Arabic,
    BoldFont=B Nazanin Bold
]
\newfontfamily\englishfont{Times New Roman}
\newfontfamily\persianfont{B Nazanin}[Script=Arabic]
\newcommand{\lr}[1]{\textenglish{#1}}
\newcommand{\en}[1]{\textenglish{#1}}

% ریاضیات و نمادها
\usepackage{amsmath,amssymb,amsthm}

% گرافیک و نمودار
\usepackage{tikz}
\usetikzlibrary{shapes,arrows,positioning,calc,decorations.pathreplacing,
                backgrounds,fit,mindmap,trees,shadows,patterns,chains,
                arrows.meta,bending}
\usepackage{pgfplots}
\pgfplotsset{compat=1.18}
\usepackage{pgfplotstable}

% جداول
\usepackage{booktabs}
\usepackage{multirow}
\usepackage{longtable}
\usepackage{tabularx}
\usepackage{colortbl}
\usepackage{array}

% رنگ‌ها
\usepackage{xcolor}
\definecolor{royalblue}{RGB}{26,35,126}
\definecolor{imperialred}{RGB}{191,54,12}
\definecolor{parliamentgold}{RGB}{255,143,0}
\definecolor{victoriangreen}{RGB}{0,105,92}
\definecolor{tudorpurple}{RGB}{74,20,140}
\definecolor{industrialgray}{RGB}{55,71,79}
\definecolor{empirecream}{RGB}{255,248,225}

% صفحه‌آرایی
\usepackage[top=2.5cm,bottom=2.5cm,left=2cm,right=2cm]{geometry}
\usepackage{fancyhdr}
\usepackage{titlesec}

% ارجاعات
\usepackage[style=authoryear,backend=biber,maxcitenames=2]{biblatex}
\usepackage{hyperref}
\hypersetup{
    colorlinks=true,
    linkcolor=royalblue,
    citecolor=victoriangreen,
    urlcolor=imperialred
}

% سایر
\usepackage{enumitem}
\usepackage{float}
\usepackage{wrapfig}
\usepackage{framed}
\usepackage{tcolorbox}
\tcbuselibrary{most}

% ⚠️ bidi باید آخرین پکیج باشد!
\usepackage{bidi}

%%%%%%%%%%%%%%%%%%%%%%%%%%%%%%%%%%%%%%%%%%%%%%%%%%%%%%%%%%%%%%%%%%%%%%%
% تعریف محیط‌های ویژه
%%%%%%%%%%%%%%%%%%%%%%%%%%%%%%%%%%%%%%%%%%%%%%%%%%%%%%%%%%%%%%%%%%%%%%%

% کادر خلاصه کلیدی
\newtcolorbox{keybox}[1][]{
    colback=empirecream,
    colframe=royalblue,
    fonttitle=\bfseries,
    title=#1,
    arc=3mm,
    boxrule=1.5pt
}

% کادر نقل قول تاریخی
\newtcolorbox{historicalquote}{
    colback=white,
    colframe=tudorpurple,
    leftrule=4pt,
    rightrule=0pt,
    toprule=0pt,
    bottomrule=0pt,
    arc=0pt,
    fontupper=\itshape
}

% کادر درس سیاستی
\newtcolorbox{policybox}[1][]{
    colback=victoriangreen!10,
    colframe=victoriangreen,
    fonttitle=\bfseries,
    title=#1,
    arc=2mm
}

% کادر هشدار تحلیلی
\newtcolorbox{warningbox}{
    colback=imperialred!10,
    colframe=imperialred,
    arc=2mm
}

%%%%%%%%%%%%%%%%%%%%%%%%%%%%%%%%%%%%%%%%%%%%%%%%%%%%%%%%%%%%%%%%%%%%%%%
% تنظیمات عنوان‌ها
%%%%%%%%%%%%%%%%%%%%%%%%%%%%%%%%%%%%%%%%%%%%%%%%%%%%%%%%%%%%%%%%%%%%%%%

\titleformat{\chapter}[display]
    {\normalfont\huge\bfseries\color{royalblue}}
    {\chaptertitlename\ \thechapter}{20pt}{\Huge}

\titleformat{\section}
    {\normalfont\Large\bfseries\color{tudorpurple}}
    {\thesection}{1em}{}

\titleformat{\subsection}
    {\normalfont\large\bfseries\color{victoriangreen}}
    {\thesubsection}{1em}{}

%%%%%%%%%%%%%%%%%%%%%%%%%%%%%%%%%%%%%%%%%%%%%%%%%%%%%%%%%%%%%%%%%%%%%%%
% اطلاعات سند
%%%%%%%%%%%%%%%%%%%%%%%%%%%%%%%%%%%%%%%%%%%%%%%%%%%%%%%%%%%%%%%%%%%%%%%

\title{
    \vspace{-2cm}
    \textcolor{royalblue}{\rule{\textwidth}{2pt}}\\[0.5cm]
    {\Huge\bfseries\textcolor{royalblue}{تطور طبقاتی و توسعه در بریتانیا}}\\[0.3cm]
    {\Large تحلیل هزارساله نقش طبقات، دربار و نیروهای اجتماعی}\\[0.2cm]
    {\large در پیشرفت اقتصادی، سیاسی و اجتماعی}\\[0.3cm]
    \textcolor{royalblue}{\rule{\textwidth}{2pt}}
}

\author{
    \textbf{تیم پژوهشی بین‌رشته‌ای}\\[0.3cm]
    \begin{tabular}{ll}
    \textcolor{tudorpurple}{$\blacksquare$} & تاریخ اجتماعی-اقتصادی \\
    \textcolor{victoriangreen}{$\blacksquare$} & جامعه‌شناسی تاریخی \\
    \textcolor{parliamentgold}{$\blacksquare$} & اقتصاد سیاسی نهادگرا \\
    \textcolor{imperialred}{$\blacksquare$} & مردم‌شناسی سیاسی \\
    \textcolor{industrialgray}{$\blacksquare$} & مطالعات توسعه تطبیقی \\
    \end{tabular}
}

\date{\today}

%%%%%%%%%%%%%%%%%%%%%%%%%%%%%%%%%%%%%%%%%%%%%%%%%%%%%%%%%%%%%%%%%%%%%%%
% شروع سند
%%%%%%%%%%%%%%%%%%%%%%%%%%%%%%%%%%%%%%%%%%%%%%%%%%%%%%%%%%%%%%%%%%%%%%%

\begin{document}

\maketitle
\tableofcontents
\newpage

%%%%%%%%%%%%%%%%%%%%%%%%%%%%%%%%%%%%%%%%%%%%%%%%%%%%%%%%%%%%%%%%%%%%%%%
% بخش صفر: مطالعه در یک نگاه
%%%%%%%%%%%%%%%%%%%%%%%%%%%%%%%%%%%%%%%%%%%%%%%%%%%%%%%%%%%%%%%%%%%%%%%

\chapter*{بخش صفر: مطالعه در یک نگاه}
\addcontentsline{toc}{chapter}{بخش صفر: مطالعه در یک نگاه}

\section*{خلاصه مدیریتی}

\begin{keybox}[پیام محوری پژوهش]
بریتانیا نخستین کشوری بود که موفق شد بدون انقلاب خشونت‌آمیز از پایین، 
گذار از جامعه فئودالی به سرمایه‌داری صنعتی و از استبداد به دموکراسی پارلمانی 
را طی کند. کلید این موفقیت، \textbf{سازش تدریجی بین طبقات} بود که در 
چارچوب \textbf{نهادهای میانجی} (پارلمان، کامن‌لا، حکومت محلی) صورت گرفت. 
این مدل نه محصول طراحی آگاهانه، بلکه نتیجه \textbf{توازن قدرت خاص} بین 
تاج، اشرافیت، و بورژوازی تجاری-صنعتی بود.
\end{keybox}

\vspace{0.5cm}

تجربه بریتانیا نشان می‌دهد که:

\begin{enumerate}[label=\textcolor{royalblue}{\arabic*.}]
    \item \textbf{نهادها پیش از ایدئولوژی:} پارلمان انگلستان قرن‌ها قبل از 
    نظریه‌های دموکراسی وجود داشت و به‌تدریج دموکراتیزه شد.
    
    \item \textbf{انعطاف‌پذیری اشرافیت:} برخلاف اشرافیت فرانسه، نجبای انگلیسی 
    به تجارت و صنعت روی آوردند و با بورژوازی درآمیختند.
    
    \item \textbf{حقوق مالکیت امن:} انقلاب شکوهمند ۱۶۸۸ حقوق مالکیت را در 
    برابر تاج تضمین کرد و زمینه انقلاب صنعتی را فراهم ساخت.
    
    \item \textbf{امپراتوری به‌مثابه شیر ایمنی:} امپراتوری جهانی هم منبع ثروت 
    بود و هم مقصد صدور تنش‌های داخلی.
    
    \item \textbf{اصلاح تدریجی:} در لحظات بحرانی (۱۸۳۲، ۱۸۶۷، ۱۹۱۸، ۱۹۴۵)، 
    نخبگان امتیاز دادند تا از انقلاب پیشگیری کنند.
\end{enumerate}

\newpage

%----------------------------------------------------------------------
% نقشه مفهومی
%----------------------------------------------------------------------

\section*{نقشه مفهومی: عوامل و روابط}

\begin{figure}[H]
\centering
\begin{tikzpicture}[
    node distance=2cm,
    every node/.style={font=\small},
    actor/.style={
        rectangle, rounded corners=5pt, 
        minimum width=2.5cm, minimum height=1cm,
        draw, thick, text centered, text width=2.3cm
    },
    institution/.style={
        ellipse, minimum width=2.5cm, minimum height=1cm,
        draw, thick, text centered, text width=2cm
    },
    outcome/.style={
        rectangle, rounded corners=10pt,
        minimum width=2.5cm, minimum height=1cm,
        draw, very thick, text centered, text width=2.3cm
    },
    arrow/.style={-{Stealth[length=3mm]}, thick},
    doublearrow/.style={{Stealth[length=3mm]}-{Stealth[length=3mm]}, thick}
]

% بازیگران (Actors)
\node[actor, fill=royalblue!20] (crown) at (0,6) {تاج و دربار\\(Crown)};
\node[actor, fill=tudorpurple!20] (aristocracy) at (-5,4) {اشرافیت\\(Aristocracy)};
\node[actor, fill=parliamentgold!20] (gentry) at (0,4) {ژنتری\\(Gentry)};
\node[actor, fill=victoriangreen!20] (bourgeoisie) at (5,4) {بورژوازی\\(Bourgeoisie)};
\node[actor, fill=industrialgray!20] (workers) at (0,1) {طبقه کارگر\\(Working Class)};

% نهادها (Institutions)
\node[institution, fill=empirecream] (parliament) at (-4,2) {پارلمان};
\node[institution, fill=empirecream] (law) at (4,2) {کامن‌لا};
\node[institution, fill=empirecream] (church) at (-6,6) {کلیسا};
\node[institution, fill=empirecream] (local) at (6,6) {حکومت محلی};

% پیامدها (Outcomes)
\node[outcome, fill=victoriangreen!30] (industry) at (-4,-1.5) {انقلاب صنعتی};
\node[outcome, fill=royalblue!30] (democracy) at (0,-1.5) {دموکراسی پارلمانی};
\node[outcome, fill=imperialred!30] (empire) at (4,-1.5) {امپراتوری جهانی};

% روابط بین بازیگران
\draw[doublearrow, royalblue] (crown) -- (aristocracy);
\draw[doublearrow, tudorpurple] (crown) -- (gentry);
\draw[doublearrow, victoriangreen] (crown) -- (bourgeoisie);
\draw[doublearrow, parliamentgold] (aristocracy) -- (gentry);
\draw[doublearrow, victoriangreen] (gentry) -- (bourgeoisie);
\draw[arrow, imperialred, dashed] (workers) -- (parliament);

% روابط با نهادها
\draw[arrow, industrialgray] (aristocracy) -- (parliament);
\draw[arrow, industrialgray] (gentry) -- (parliament);
\draw[arrow, industrialgray] (bourgeoisie) -- (law);
\draw[arrow, industrialgray] (crown) -- (church);
\draw[arrow, industrialgray] (gentry) -- (local);

% روابط با پیامدها
\draw[arrow, victoriangreen, very thick] (parliament) -- (democracy);
\draw[arrow, parliamentgold, very thick] (law) -- (industry);
\draw[arrow, tudorpurple, very thick] (bourgeoisie) to[bend right=20] (empire);
\draw[arrow, royalblue, very thick] (crown) to[bend left=30] (empire);

% عنوان
\node[above=0.5cm of crown, font=\large\bfseries, color=royalblue] 
    {نقشه مفهومی: بازیگران، نهادها و پیامدها در تاریخ بریتانیا};

% راهنما
\node[below=0.3cm of industry, font=\footnotesize, text width=10cm, align=center] 
    {خطوط دوطرفه: روابط متقابل قدرت | خطوط یک‌طرفه: تأثیر | خط‌چین: تأثیر متأخر};

\end{tikzpicture}
\caption{نقشه مفهومی روابط قدرت و نهادها در تاریخ بریتانیا}
\end{figure}

\newpage

%----------------------------------------------------------------------
% خط زمانی تاریخی
%----------------------------------------------------------------------

\section*{نقشه تاریخی: هزار سال در یک نگاه}

\begin{figure}[H]
\centering
\begin{tikzpicture}[scale=0.85, transform shape]

% خط زمانی اصلی
\draw[very thick, royalblue] (0,0) -- (16,0);

% دوره‌ها (پس‌زمینه)
\fill[tudorpurple!10] (0,-3) rectangle (3,3);
\fill[parliamentgold!10] (3,-3) rectangle (6,3);
\fill[victoriangreen!10] (6,-3) rectangle (9,3);
\fill[industrialgray!10] (9,-3) rectangle (12,3);
\fill[imperialred!10] (12,-3) rectangle (16,3);

% برچسب دوره‌ها
\node[font=\bfseries\small, color=tudorpurple] at (1.5,2.5) {فئودالی-تودور};
\node[font=\bfseries\small, color=parliamentgold] at (4.5,2.5) {انقلاب و تثبیت};
\node[font=\bfseries\small, color=victoriangreen] at (7.5,2.5) {صنعتی‌شدن};
\node[font=\bfseries\small, color=industrialgray] at (10.5,2.5) {امپراتوری و بحران};
\node[font=\bfseries\small, color=imperialred] at (14,2.5) {رفاه تا نئولیبرال};

% تاریخ‌های کلیدی
\foreach \x/\year in {0/1066, 3/1485, 6/1688, 9/1832, 12/1914, 16/2024} {
    \draw[thick] (\x,0.2) -- (\x,-0.2);
    \node[below, font=\small\bfseries] at (\x,-0.3) {\year};
}

% رویدادهای کلیدی (بالا)
\node[above, font=\tiny, text width=2cm, align=center, color=tudorpurple] 
    at (0.5,0.3) {فتح نورمان\\مگناکارتا ۱۲۱۵};
\node[above, font=\tiny, text width=2cm, align=center, color=tudorpurple] 
    at (2,0.3) {جنگ گل‌ها\\تودورها};
\node[above, font=\tiny, text width=2cm, align=center, color=parliamentgold] 
    at (4,0.3) {جنگ داخلی\\۱۶۴۲-۱۶۵۱};
\node[above, font=\tiny, text width=2cm, align=center, color=parliamentgold] 
    at (5.5,0.3) {انقلاب شکوهمند\\۱۶۸۸};
\node[above, font=\tiny, text width=2cm, align=center, color=victoriangreen] 
    at (7,0.3) {آغاز انقلاب\\صنعتی ~۱۷۶۰};
\node[above, font=\tiny, text width=2cm, align=center, color=victoriangreen] 
    at (8.5,0.3) {قانون اصلاحات\\۱۸۳۲};
\node[above, font=\tiny, text width=2.2cm, align=center, color=industrialgray] 
    at (10,0.3) {اوج امپراتوری\\ویکتوریا};
\node[above, font=\tiny, text width=2cm, align=center, color=industrialgray] 
    at (11.5,0.3) {جنگ‌های\\جهانی};
\node[above, font=\tiny, text width=2cm, align=center, color=imperialred] 
    at (13,0.3) {دولت رفاه\\۱۹۴۵};
\node[above, font=\tiny, text width=2cm, align=center, color=imperialred] 
    at (14.5,0.3) {تاچر ۱۹۷۹\\برگزیت ۲۰۱۶};

% روند قدرت طبقات (نمودار خطی در پایین)
\draw[thick, royalblue, dashed] 
    plot[smooth] coordinates {(0,-1) (3,-0.8) (6,-1.5) (9,-2) (12,-2.2) (16,-2.3)};
\node[right, font=\tiny, color=royalblue] at (16,-2.3) {تاج};

\draw[thick, tudorpurple] 
    plot[smooth] coordinates {(0,-1.2) (3,-0.5) (6,-0.8) (9,-1.5) (12,-2) (16,-2.5)};
\node[right, font=\tiny, color=tudorpurple] at (16,-2.5) {اشراف};

\draw[thick, victoriangreen] 
    plot[smooth] coordinates {(0,-2.5) (3,-2.2) (6,-1.8) (9,-0.8) (12,-0.5) (16,-0.8)};
\node[right, font=\tiny, color=victoriangreen] at (16,-0.8) {بورژوازی};

\draw[thick, imperialred] 
    plot[smooth] coordinates {(0,-2.8) (3,-2.8) (6,-2.7) (9,-2.3) (12,-1.5) (16,-1.2)};
\node[right, font=\tiny, color=imperialred] at (16,-1.2) {کارگران};

% راهنمای روند قدرت
\node[font=\tiny, text width=3cm] at (2,-2.8) {↑ قدرت بیشتر};

\end{tikzpicture}
\caption{خط زمانی هزارساله تحول قدرت طبقات در بریتانیا}
\end{figure}

\newpage

%----------------------------------------------------------------------
% مؤلفه‌های تعیین‌کننده
%----------------------------------------------------------------------

\section*{مؤلفه‌های تعیین‌کننده: چرا بریتانیا؟}

\begin{table}[H]
\centering
\caption{عوامل متمایزکننده بریتانیا از سایر کشورهای اروپایی}
\renewcommand{\arraystretch}{1.4}
\begin{tabularx}{\textwidth}{>{\bfseries\color{royalblue}}p{3cm}X>{\color{imperialred}}p{3cm}}
\toprule
\textbf{عامل} & \textbf{وضعیت در بریتانیا} & \textbf{مقایسه با فرانسه} \\
\midrule
\rowcolor{empirecream}
جغرافیا & جزیره‌ای بودن، امنیت از تهاجم & مرزهای زمینی آسیب‌پذیر \\

موقعیت تاج & مشروط از ۱۲۱۵، ضعیف از ۱۶۸۸ & مطلقه تا ۱۷۸۹ \\

\rowcolor{empirecream}
اشرافیت & باز، تجاری‌شده، انعطاف‌پذیر & بسته، زمین‌دار، مقاوم \\

پارلمان & قدیمی، مستمر، قدرتمند & Estates-General ضعیف \\

\rowcolor{empirecream}
نظام حقوقی & کامن‌لا، حقوق مالکیت قوی & حقوق رومی، مداخله دولت \\

کلیسا & جدا از رُم (۱۵۳۴)، ملی & کاتولیک، وابسته به رُم \\

\rowcolor{empirecream}
مسیر تغییر & تدریجی، سازشی، از بالا & انقلابی، خشونت‌آمیز، از پایین \\

\bottomrule
\end{tabularx}
\end{table}

\begin{figure}[H]
\centering
\begin{tikzpicture}[
    mindmap,
    grow cyclic,
    every node/.style={concept, circular drop shadow},
    concept color=royalblue!30,
    level 1/.append style={level distance=4.5cm, sibling angle=72},
    level 2/.append style={level distance=3cm, sibling angle=45},
    font=\small
]
\node[concept, font=\bfseries] {چرا بریتانیا\\پیشگام شد؟}
    child[concept color=tudorpurple!30] { node {جغرافیا}
        child { node[font=\tiny] {جزیره} }
        child { node[font=\tiny] {زغال‌سنگ} }
        child { node[font=\tiny] {بنادر} }
    }
    child[concept color=victoriangreen!30] { node {نهادها}
        child { node[font=\tiny] {پارلمان} }
        child { node[font=\tiny] {کامن‌لا} }
        child { node[font=\tiny] {بانک مرکزی} }
    }
    child[concept color=parliamentgold!30] { node {طبقات}
        child { node[font=\tiny] {اشراف انعطاف‌پذیر} }
        child { node[font=\tiny] {ژنتری فعال} }
        child { node[font=\tiny] {بورژوازی قوی} }
    }
    child[concept color=imperialred!30] { node {فرهنگ}
        child { node[font=\tiny] {پروتستانتیسم} }
        child { node[font=\tiny] {تجربه‌گرایی} }
        child { node[font=\tiny] {فردگرایی} }
    }
    child[concept color=industrialgray!30] { node {امپراتوری}
        child { node[font=\tiny] {بازار جهانی} }
        child { node[font=\tiny] {مواد خام} }
        child { node[font=\tiny] {صدور جمعیت} }
    };
\end{tikzpicture}
\caption{نقشه ذهنی: عوامل پیشگامی بریتانیا در توسعه}
\end{figure}

\newpage

%----------------------------------------------------------------------
% خلاصه یافته‌ها به تفکیک دوره
%----------------------------------------------------------------------

\section*{خلاصه یافته‌ها به تفکیک دوره تاریخی}

\begin{table}[H]
\centering
\caption{خلاصه تحولات طبقاتی-نهادی در هر دوره}
\renewcommand{\arraystretch}{1.3}
\small
\begin{tabularx}{\textwidth}{|>{\bfseries}p{2cm}|p{2.5cm}|p{2.5cm}|X|p{2cm}|}
\hline
\rowcolor{royalblue!20}
\textbf{دوره} & \textbf{طبقه مسلط} & \textbf{نهاد کلیدی} & \textbf{تحول اصلی} & \textbf{نتیجه} \\
\hline

\cellcolor{tudorpurple!10}۱۰۶۶-۱۴۸۵ & 
اشرافیت فئودالی & 
شورای بزرگان $\to$ پارلمان & 
مگناکارتا؛ تثبیت پارلمان & 
مشروطیت اولیه \\
\hline

\cellcolor{tudorpurple!10}۱۴۸۵-۱۶۰۳ & 
تاج تودور & 
دربار و شورای خصوصی & 
اصلاحات دینی؛ تقویت مرکز & 
دولت-ملت \\
\hline

\cellcolor{parliamentgold!10}۱۶۰۳-۱۷۱۴ & 
ژنتری + بورژوازی & 
پارلمان & 
انقلاب‌ها و تسویه ۱۶۸۸ & 
مشروطه تثبیت‌شده \\
\hline

\cellcolor{victoriangreen!10}۱۷۱۴-۱۸۳۲ & 
الیگارشی ویگ & 
پارلمان بسته & 
انقلاب صنعتی & 
رشد بورژوازی \\
\hline

\cellcolor{victoriangreen!10}۱۸۳۲-۱۹۱۴ & 
بورژوازی صنعتی & 
پارلمان گسترش‌یافته & 
اصلاحات، امپراتوری & 
لیبرال‌دموکراسی \\
\hline

\cellcolor{industrialgray!10}۱۹۱۴-۱۹۷۹ & 
طبقات متوسط + کارگر & 
دولت رفاه & 
جنگ‌ها، ملی‌سازی & 
دموکراسی اجتماعی \\
\hline

\cellcolor{imperialred!10}۱۹۷۹-۲۰۲۴ & 
سرمایه مالی & 
بازار جهانی & 
تاچریسم، برگزیت & 
نئولیبرالیسم \\
\hline

\end{tabularx}
\end{table}

%----------------------------------------------------------------------
% درس‌های کلیدی
%----------------------------------------------------------------------

\section*{درس‌های کلیدی برای الگوبرداری}

\begin{policybox}[پنج درس اصلی از تجربه بریتانیا]
\begin{enumerate}[label=\textcolor{victoriangreen}{\arabic*.}]
    \item \textbf{نهادهای میانجی:} پارلمان و دادگاه‌ها می‌توانند میدان سازش طبقاتی باشند، 
    نه میدان جنگ.
    
    \item \textbf{انعطاف‌پذیری نخبگان:} اشرافیتی که به تجارت روی می‌آورد، به جای 
    سقوط، تحول می‌یابد.
    
    \item \textbf{اصلاح پیش‌گیرانه:} امتیازدادن به‌موقع از انقلاب جلوگیری می‌کند 
    (۱۸۳۲، ۱۸۶۷، ۱۹۱۸، ۱۹۴۵).
    
    \item \textbf{حقوق مالکیت امن:} بدون امنیت سرمایه‌گذاری، انقلاب صنعتی 
    رخ نمی‌داد.
    
    \item \textbf{هزینه‌های پنهان:} موفقیت بریتانیا هزینه‌هایی داشت: استثمار 
    استعماری، نابرابری داخلی، و مسائل ناحل باقی‌مانده.
\end{enumerate}
\end{policybox}

\begin{warningbox}
\textbf{هشدار تحلیلی:} تجربه بریتانیا مستقیماً قابل تکرار نیست. شرایط جغرافیایی 
(جزیره‌ای بودن)، تاریخی (فقدان تهاجم خارجی پس از ۱۰۶۶)، و اقتصادی 
(ذخایر زغال‌سنگ، موقعیت تجاری) منحصربه‌فرد بود. آنچه قابل استخراج است، 
\textbf{الگوهای انتزاعی} است، نه \textbf{نسخه‌های عینی}.
\end{warningbox}

\newpage

%%%%%%%%%%%%%%%%%%%%%%%%%%%%%%%%%%%%%%%%%%%%%%%%%%%%%%%%%%%%%%%%%%%%%%%
% فصل اول: مقدمه و چارچوب نظری
%%%%%%%%%%%%%%%%%%%%%%%%%%%%%%%%%%%%%%%%%%%%%%%%%%%%%%%%%%%%%%%%%%%%%%%

\chapter{مقدمه و چارچوب نظری}

\section{طرح مسئله}

چرا بریتانیا نخستین کشور صنعتی جهان شد؟ چرا در این کشور، گذار به 
دموکراسی بدون انقلاب خشونت‌آمیز صورت گرفت؟ چه نقشی طبقات مختلف 
اجتماعی -- از اشرافیت فئودالی تا بورژوازی صنعتی و طبقه کارگر -- در 
این فرآیند ایفا کردند؟ آیا تجربه بریتانیا الگویی قابل تعمیم ارائه می‌دهد 
یا استثنایی تاریخی است؟

این پرسش‌ها از زمان مارکس، وبر و پولانی تا امروز، جامعه‌شناسان، 
مورخان و اقتصاددانان را به خود مشغول داشته است. پاسخ به آن‌ها نه 
فقط اهمیت تاریخی دارد، بلکه برای فهم مسیرهای توسعه در جهان معاصر 
نیز حیاتی است.

\begin{historicalquote}
«انگلستان کشور کلاسیک شیوه تولید سرمایه‌داری است... انگلیس را به‌عنوان 
تصویر اصلی برای نشان‌دادن قوانین حرکت اقتصادی جامعه مدرن به کار می‌برم.»
\hfill --- کارل مارکس، مقدمه ویراست اول \textit{سرمایه} (۱۸۶۷)
\end{historicalquote}

\subsection{اهمیت موضوع}

مطالعه تطور طبقاتی بریتانیا از چند جهت اهمیت دارد:

\textbf{اول، اهمیت تاریخی:} بریتانیا «آزمایشگاه» نخستین گذار سرمایه‌داری 
بود. آنچه در این کشور رخ داد، الگوی (هرچند ناقص) سایر کشورها شد.

\textbf{دوم، اهمیت نظری:} بسیاری از نظریه‌های کلاسیک جامعه‌شناسی و 
اقتصاد سیاسی (مارکس، وبر، پولانی، مور، اسکاچپول، برنر) بر تحلیل 
تجربه بریتانیایی استوارند.

\textbf{سوم، اهمیت کاربردی:} برای کشورهایی که درگیر چالش‌های توسعه و 
دموکراتیزاسیون‌اند، فهم مکانیسم‌های موفقیت (و شکست) تجربی بریتانیا 
می‌تواند آموزنده باشد.

\textbf{چهارم، اهمیت انتقادی:} تجربه بریتانیا هم الهام‌بخش است و هم 
هشداردهنده. موفقیت‌های این کشور در سایه استعمار، نابرابری داخلی، و 
هزینه‌های انسانی سنگین به دست آمد.

\section{سؤالات پژوهش}

این پژوهش به پنج پرسش محوری می‌پردازد:

\begin{enumerate}[label=\textcolor{royalblue}{\textbf{پرسش \arabic*:}}]
    \item ساختار طبقاتی بریتانیا در دوره‌های مختلف تاریخی چگونه بود و 
    چه تحولاتی یافت؟
    
    \item روابط قدرت بین طبقات (تاج، اشرافیت، ژنتری، بورژوازی، طبقه کارگر) 
    چگونه شکل گرفت و تغییر کرد؟
    
    \item نهادهای سیاسی و حقوقی (پارلمان، کامن‌لا، کلیسا) چه نقشی در 
    تنظیم روابط طبقاتی ایفا کردند؟
    
    \item چه عواملی باعث شد بریتانیا مسیر تدریجی و سازشی را طی کند، 
    در حالی که کشورهایی چون فرانسه مسیر انقلابی رفتند؟
    
    \item چه الگوهای مشروط به زمینه‌ای از تجربه بریتانیا قابل استخراج است؟
\end{enumerate}

\section{چارچوب نظری}

این پژوهش از چارچوب نظری \textbf{تلفیقی} بهره می‌برد که عناصری از 
چند رویکرد را ترکیب می‌کند:

\subsection{نهادگرایی تاریخی}

از این رویکرد، مفهوم \textbf{مسیر وابستگی} (\lr{Path Dependence}) 
را اخذ می‌کنیم: تصمیمات و نهادهای اولیه (مثلاً پارلمان قرون وسطی) 
امکانات و محدودیت‌هایی برای آینده ایجاد می‌کنند. همچنین، مفهوم 
\textbf{لحظات بحرانی} (\lr{Critical Junctures}) که در آن‌ها مسیرهای 
جدید ممکن می‌شود.

\textcite{north1990institutions} و \textcite{acemoglu2012nations} 
نشان داده‌اند که کیفیت نهادها -- نه منابع طبیعی یا جغرافیا -- تعیین‌کننده 
اصلی توسعه بلندمدت است.

\subsection{تحلیل طبقاتی}

از سنت مارکسیستی، توجه به \textbf{روابط تولید} و \textbf{تضاد طبقاتی} 
را حفظ می‌کنیم، اما با تعدیلاتی:
\begin{itemize}
    \item طبقات را نه صرفاً اقتصادی، بلکه چندبُعدی می‌بینیم (شامل قدرت 
    سیاسی و سرمایه فرهنگی).
    \item تضاد طبقاتی را نه لزوماً آنتاگونیستی، بلکه قابل مدیریت از طریق 
    سازش نهادی می‌دانیم.
    \item به ناهمگنی درون‌طبقه‌ای (مثلاً تفاوت اشرافیت انگلیس و فرانسه) 
    توجه می‌کنیم.
\end{itemize}

\textcite{brenner1976agrarian} نشان داد که تفاوت‌های درون‌طبقه‌ای 
و روابط طبقاتی خاص هر کشور، نه قوانین عام، مسیر توسعه را تعیین می‌کند.

\subsection{جامعه‌شناسی تاریخی-تطبیقی}

از \textcite{moore1966social} روش تحلیل تطبیقی مسیرهای مدرنیزاسیون 
را می‌آموزیم. مور استدلال کرد که:

\begin{itemize}
    \item جایی که بورژوازی قوی بود و اشرافیت تجاری شد، دموکراسی لیبرال 
    پدید آمد (انگلستان، فرانسه، آمریکا).
    \item جایی که اشرافیت زمین‌دار مقاوم ماند و با بورژوازی ضعیف ائتلاف 
    کرد، فاشیسم ظهور کرد (آلمان، ژاپن).
    \item جایی که دهقانان قوی و سازمان‌یافته بودند، انقلاب کمونیستی رخ 
    داد (روسیه، چین).
\end{itemize}

\subsection{اقتصاد سیاسی نهادگرا}

از \textcite{north1973rise} و \textcite{acemoglu2012nations}، 
تمایز بین \textbf{نهادهای فراگیر} و \textbf{نهادهای استخراجی} را 
به کار می‌بریم:

\begin{table}[H]
\centering
\caption{تفاوت نهادهای فراگیر و استخراجی}
\begin{tabular}{p{6cm}p{6cm}}
\toprule
\textbf{نهادهای فراگیر} & \textbf{نهادهای استخراجی} \\
\midrule
حقوق مالکیت امن برای همه & حقوق مالکیت ناامن یا انحصاری \\
قانون یکسان & قانون تبعیض‌آمیز \\
بازار رقابتی & انحصار و رانت \\
مشارکت سیاسی گسترده & انحصار قدرت سیاسی \\
\midrule
$\Rightarrow$ رشد پایدار & $\Rightarrow$ رکود یا رشد ناپایدار \\
\bottomrule
\end{tabular}
\end{table}

عجم‌اوغلو و رابینسون استدلال می‌کنند که انگلستان پس از انقلاب شکوهمند 
۱۶۸۸، نهادهای فراگیر را تثبیت کرد و این امر زمینه‌ساز انقلاب صنعتی شد.

\subsection{چارچوب تلفیقی این پژوهش}

\begin{figure}[H]
\centering
\begin{tikzpicture}[
    node distance=1.5cm,
    box/.style={rectangle, rounded corners, draw, thick, 
                minimum width=3cm, minimum height=1cm, 
                text centered, font=\small},
    arrow/.style={-{Stealth[length=3mm]}, thick}
]

% سطوح تحلیل
\node[box, fill=industrialgray!20] (macro) {سطح کلان:\\جغرافیا، جهان‌نظام};
\node[box, fill=royalblue!20, below=of macro] (meso) {سطح میانی:\\نهادها، طبقات};
\node[box, fill=victoriangreen!20, below=of meso] (micro) {سطح خُرد:\\کنش، ایدئولوژی};

% پیامدها
\node[box, fill=parliamentgold!20, right=3cm of meso] (outcome) {پیامد:\\رشد، دموکراسی، نابرابری};

% روابط
\draw[arrow] (macro) -- (meso);
\draw[arrow] (meso) -- (micro);
\draw[arrow, dashed] (micro) to[bend left=30] node[right, font=\tiny] {بازخورد} (meso);
\draw[arrow, very thick] (meso) -- (outcome);

% برچسب
\node[above=0.3cm of macro, font=\bfseries, color=royalblue] {چارچوب تحلیلی تلفیقی};

\end{tikzpicture}
\caption{سطوح تحلیل در چارچوب تلفیقی}
\end{figure}

\section{روش‌شناسی}

\subsection{رویکرد کلی}

این پژوهش از روش \textbf{تحلیل تاریخی-تطبیقی} استفاده می‌کند. این روش 
شامل موارد زیر است:

\begin{enumerate}
    \item \textbf{روایت تحلیلی:} بازگویی تحولات تاریخی با تمرکز بر 
    مکانیسم‌های علّی، نه صرفاً توالی رویدادها.
    
    \item \textbf{مقایسه درون‌موردی:} مقایسه دوره‌های مختلف تاریخ بریتانیا 
    با یکدیگر.
    
    \item \textbf{مقایسه بین‌موردی:} مقایسه بریتانیا با کشورهای دیگر 
    (به‌ویژه فرانسه و آلمان) برای شناسایی عوامل متمایزکننده.
    
    \item \textbf{استخراج الگو:} شناسایی مکانیسم‌های کلی‌تر که ممکن است 
    در زمینه‌های دیگر نیز کاربرد داشته باشند.
\end{enumerate}

\subsection{منابع داده}

\begin{table}[H]
\centering
\caption{انواع منابع مورد استفاده}
\begin{tabularx}{\textwidth}{lX}
\toprule
\textbf{نوع منبع} & \textbf{نمونه‌ها} \\
\midrule
منابع اولیه & اسناد پارلمانی، قوانین، روزنامه‌ها \\
تاریخ‌نگاری & آثار مورخان اقتصادی-اجتماعی \\
داده‌های کمّی & سری‌های زمانی GDP، جمعیت، نابرابری \\
تحلیل‌های ثانویه & پژوهش‌های جامعه‌شناسی تاریخی \\
\bottomrule
\end{tabularx}
\end{table}

\subsection{محدودیت‌های روش‌شناختی}

\begin{warningbox}
\textbf{محدودیت‌ها:}
\begin{itemize}
    \item \textbf{مسئله اسناد:} تاریخ را «برندگان» نوشته‌اند. صدای طبقات 
    پایین کمتر شنیده می‌شود.
    \item \textbf{مسئله علّیت:} در تاریخ، آزمایش کنترل‌شده ممکن نیست. 
    روابط علّی تفسیری‌اند.
    \item \textbf{مسئله تعمیم:} بریتانیا یک مورد است. تعمیم از یک مورد 
    محدودیت دارد.
\end{itemize}
\end{warningbox}

\section{ساختار پژوهش}

پژوهش حاضر در نُه بخش سازمان‌دهی شده است:

\begin{enumerate}[label=\textcolor{royalblue}{\textbf{بخش \arabic*:}}]
    \item مقدمه و چارچوب نظری (این فصل)
    \item دوره فئودالی و شکل‌گیری نهادها (۱۰۶۶-۱۴۸۵)
    \item دوره تودور و استوارت: بحران و انقلاب (۱۴۸۵-۱۷۱۴)
    \item قرن هجدهم و انقلاب صنعتی (۱۷۱۴-۱۸۳۲)
    \item عصر ویکتوریایی و اوج امپراتوری (۱۸۳۲-۱۹۱۴)
    \item قرن بیستم: جنگ، رفاه، زوال (۱۹۱۴-۱۹۷۹)
    \item عصر تاچر تا برگزیت (۱۹۷۹-۲۰۲۴)
    \item تحلیل تطبیقی: بریتانیا و دیگران
    \item نتیجه‌گیری: الگوهای قابل استخراج
\end{enumerate}

\newpage

%%%%%%%%%%%%%%%%%%%%%%%%%%%%%%%%%%%%%%%%%%%%%%%%%%%%%%%%%%%%%%%%%%%%%%%
% پیش‌نمایش فصول بعدی
%%%%%%%%%%%%%%%%%%%%%%%%%%%%%%%%%%%%%%%%%%%%%%%%%%%%%%%%%%%%%%%%%%%%%%%

\section*{پیش‌نمایش محتوای فصول آتی}

\begin{table}[H]
\centering
\caption{پیش‌نمایش محتوای اصلی هر فصل}
\renewcommand{\arraystretch}{1.4}
\small
\begin{tabularx}{\textwidth}{|c|p{3cm}|X|}
\hline
\rowcolor{royalblue!20}
\textbf{فصل} & \textbf{دوره} & \textbf{محتوای اصلی} \\
\hline

\cellcolor{tudorpurple!10}۲ & 
۱۰۶۶-۱۴۸۵ & 
فتح نورمان و تثبیت فئودالیسم؛ مگناکارتا و محدودسازی تاج؛ 
ظهور پارلمان؛ جنگ گل‌ها و تضعیف اشرافیت \\
\hline

\cellcolor{parliamentgold!10}۳ & 
۱۴۸۵-۱۷۱۴ & 
دولت‌سازی تودور؛ اصلاحات دینی؛ تنش تاج-پارلمان؛ جنگ داخلی؛ 
جمهوری کرامول؛ بازگشت سلطنت؛ انقلاب شکوهمند ۱۶۸۸ \\
\hline

\cellcolor{victoriangreen!10}۴ & 
۱۷۱۴-۱۸۳۲ & 
سلطنت هانوفر و الیگارشی ویگ؛ انقلاب کشاورزی؛ انقلاب صنعتی؛ 
ظهور بورژوازی صنعتی؛ جنبش‌های اصلاحی؛ قانون اصلاحات ۱۸۳۲ \\
\hline

\cellcolor{industrialgray!10}۵ & 
۱۸۳۲-۱۹۱۴ & 
گسترش تدریجی دموکراسی؛ اصلاحات ویکتوریایی؛ اوج امپراتوری؛ 
ظهور طبقه کارگر سازمان‌یافته؛ حزب کارگر \\
\hline

\cellcolor{imperialred!10}۶ & 
۱۹۱۴-۱۹۷۹ & 
جنگ جهانی اول و تحول اجتماعی؛ بین‌دوجنگ؛ جنگ جهانی دوم؛ 
دولت رفاه ۱۹۴۵؛ زوال امپراتوری؛ بحران‌های دهه ۱۹۷۰ \\
\hline

\cellcolor{royalblue!10}۷ & 
۱۹۷۹-۲۰۲۴ & 
انقلاب تاچر؛ مالی‌گرایی؛ نیولیبرالیسم؛ نیولیبر تا برگزیت؛ 
نابرابری فزاینده؛ بحران هویت \\
\hline

\cellcolor{victoriangreen!10}۸ & 
تطبیقی & 
مقایسه با فرانسه، آلمان، آمریکا، ژاپن؛ چرا مسیرها متفاوت شد؟ \\
\hline

\cellcolor{parliamentgold!10}۹ & 
نتیجه‌گیری & 
جمع‌بندی یافته‌ها؛ الگوهای قابل استخراج؛ درس‌ها و هشدارها \\
\hline

\end{tabularx}
\end{table}

%%%%%%%%%%%%%%%%%%%%%%%%%%%%%%%%%%%%%%%%%%%%%%%%%%%%%%%%%%%%%%%%%%%%%%%
% ارجاعات فصل اول
%%%%%%%%%%%%%%%%%%%%%%%%%%%%%%%%%%%%%%%%%%%%%%%%%%%%%%%%%%%%%%%%%%%%%%%

\section*{ارجاعات فصل اول}

\begin{enumerate}[label={[\arabic*]}]
    \item Acemoglu, D. \& Robinson, J.A. (2012). \textit{Why Nations Fail: 
    The Origins of Power, Prosperity, and Poverty}. New York: Crown Business.
    
    \item Brenner, R. (1976). Agrarian Class Structure and Economic Development 
    in Pre-Industrial Europe. \textit{Past \& Present}, 70, 30-75.
    
    \item Moore, B. (1966). \textit{Social Origins of Dictatorship and Democracy: 
    Lord and Peasant in the Making of the Modern World}. Boston: Beacon Press.
    
    \item North, D.C. (1990). \textit{Institutions, Institutional Change and 
    Economic Performance}. Cambridge: Cambridge University Press.
    
    \item North, D.C. \& Thomas, R.P. (1973). \textit{The Rise of the Western 
    World: A New Economic History}. Cambridge: Cambridge University Press.
\end{enumerate}

%%%%%%%%%%%%%%%%%%%%%%%%%%%%%%%%%%%%%%%%%%%%%%%%%%%%%%%%%%%%%%%%%%%%%%%
% پایان بخش صفر و یک
%%%%%%%%%%%%%%%%%%%%%%%%%%%%%%%%%%%%%%%%%%%%%%%%%%%%%%%%%%%%%%%%%%%%%%%

\vfill

\begin{center}
\textcolor{royalblue}{\rule{0.5\textwidth}{1pt}}\\[0.5cm]
\textit{پایان بخش صفر و یک}\\[0.3cm]
\textbf{ادامه دارد: بخش دوم (دوره فئودالی و شکل‌گیری نهادها)}\\[0.5cm]
\textcolor{royalblue}{\rule{0.5\textwidth}{1pt}}
\end{center}

%%%%%%%%%%%%%%%%%%%%%%%%%%%%%%%%%%%%%%%%%%%%%%%%%%%%%%%%%%%%%%%%%%%%%%%
% فصل دوم: دوره فئودالی و شکل‌گیری نهادها (۱۰۶۶-۱۴۸۵)
%%%%%%%%%%%%%%%%%%%%%%%%%%%%%%%%%%%%%%%%%%%%%%%%%%%%%%%%%%%%%%%%%%%%%%%

\chapter{دوره فئودالی و شکل‌گیری نهادها (۱۰۶۶-۱۴۸۵)}

\begin{keybox}[خلاصه فصل]
این فصل به تحلیل چهار قرن نخست پس از فتح نورمان می‌پردازد. در این دوره، 
ساختارهای نهادی بنیادینی شکل گرفت که مسیر متمایز انگلستان را رقم زد: 
\textbf{سلطنت متمرکز اما مشروط}، \textbf{اشرافیتی وابسته به تاج اما 
دارای حقوق}، \textbf{پارلمانی که از ابزار مالیاتی به نهاد سیاسی تبدیل شد}، 
و \textbf{نظام حقوقی واحد (کامن‌لا)} که زمینه امنیت مالکیت را فراهم آورد. 
این نهادها نه از طراحی آگاهانه، بلکه از \textbf{تعادل قوا} و 
\textbf{سازش‌های مکرر} بین تاج و اشرافیت برآمدند.
\end{keybox}

%----------------------------------------------------------------------
\section{فتح نورمان و بازسازی ساختار قدرت (۱۰۶۶-۱۱۵۴)}
%----------------------------------------------------------------------

\subsection{انگلستان پیش از فتح}

انگلستان آنگلوساکسون پیش از ۱۰۶۶ جامعه‌ای نسبتاً غیرمتمرکز بود. پادشاهی 
انگلستان از اتحاد چندین قلمرو کوچک‌تر (هپتارشی) شکل گرفته بود و قدرت 
شاه محدود به توافق با \textit{ویتان} (Witan) -- شورای بزرگان -- بود. 
زمین به شکل‌های مختلف مالکیت می‌شد و نظام فئودالی به معنای قاره‌ای آن 
وجود نداشت.

\begin{table}[H]
\centering
\caption{مقایسه ساختار قدرت قبل و بعد از فتح نورمان}
\renewcommand{\arraystretch}{1.4}
\begin{tabularx}{\textwidth}{>{\bfseries}p{3cm}XX}
\toprule
\textbf{بُعد} & \textbf{انگلستان آنگلوساکسون} & \textbf{انگلستان نورمان} \\
\midrule
\rowcolor{empirecream}
ساختار سلطنت & انتخابی-موروثی، وابسته به ویتان & موروثی، فتح به‌عنوان مبنای مشروعیت \\

مالکیت زمین & چندشکلی (آزاد، نیمه‌آزاد) & فئودالی یکپارچه، همه زمین از آنِ شاه \\

\rowcolor{empirecream}
اشرافیت & ارل‌های محلی، قدرتمند & بارون‌های نورمان، وابسته به شاه \\

زبان & انگلیسی باستان & فرانسوی نورمان (دربار)، انگلیسی (عوام) \\

\rowcolor{empirecream}
کلیسا & نیمه‌مستقل & تحت کنترل بیشتر تاج \\

\bottomrule
\end{tabularx}
\end{table}

\subsection{فتح و پیامدهای آن}

در ۱۴ اکتبر ۱۰۶۶، ویلیام دوک نرماندی در نبرد هیستینگز، هارولد 
گادوینسون را شکست داد و به تاج انگلستان دست یافت. این فتح نظامی 
پیامدهای ژرفی برای ساختار طبقاتی انگلستان داشت:

\begin{historicalquote}
«ویلیام زمین را نه به‌عنوان پادشاه موروثی، بلکه به‌عنوان فاتح در اختیار 
گرفت. این بدان معناست که تمام زمین انگلستان، در تئوری، به او تعلق داشت 
و دیگران صرفاً آن را از او در ازای خدمت دریافت می‌کردند.»
\hfill --- \textcite{maitland1897domesday}
\end{historicalquote}

\subsubsection{جایگزینی کامل اشرافیت}

برخلاف بسیاری از فتوحات که نخبگان موجود را جذب می‌کنند، ویلیام 
اشرافیت آنگلوساکسون را به‌طور کامل جایگزین کرد. تا ۱۰۸۶ (سال 
تدوین کتاب روز داوری)، تنها ۵٪ زمین در دست انگلیسی‌ها باقی مانده بود.

\begin{figure}[H]
\centering
\begin{tikzpicture}
\begin{axis}[
    ybar,
    width=12cm,
    height=7cm,
    ylabel={درصد مالکیت زمین},
    symbolic x coords={۱۰۶۶ (قبل از فتح), ۱۰۸۶ (کتاب روز داوری)},
    xtick=data,
    ymin=0,
    ymax=100,
    bar width=20pt,
    legend style={at={(0.5,-0.2)}, anchor=north, legend columns=-1},
    nodes near coords,
    nodes near coords align={vertical},
]
\addplot[fill=tudorpurple!70] coordinates {(۱۰۶۶ (قبل از فتح),90) (۱۰۸۶ (کتاب روز داوری),5)};
\addplot[fill=royalblue!70] coordinates {(۱۰۶۶ (قبل از فتح),5) (۱۰۸۶ (کتاب روز داوری),50)};
\addplot[fill=parliamentgold!70] coordinates {(۱۰۶۶ (قبل از فتح),5) (۱۰۸۶ (کتاب روز داوری),25)};
\addplot[fill=victoriangreen!70] coordinates {(۱۰۶۶ (قبل از فتح),0) (۱۰۸۶ (کتاب روز داوری),20)};
\legend{اشرافیت انگلیسی, تاج, بارون‌های نورمان, کلیسا}
\end{axis}
\end{tikzpicture}
\caption{تغییر توزیع مالکیت زمین پس از فتح نورمان}
\end{figure}

\subsubsection{نظام فئودالی انگلیسی}

ویلیام نظام فئودالی را به شکلی متمرکزتر از نمونه قاره‌ای در انگلستان 
برقرار کرد. ویژگی‌های کلیدی این نظام عبارت بودند از:

\begin{enumerate}[label=\textcolor{royalblue}{\arabic*.}]
    \item \textbf{سلسله‌مراتب واحد:} همه زمین‌داران، مستقیم یا غیرمستقیم، 
    واسال شاه بودند. برخلاف فرانسه، اینجا «واسالِ واسالِ من، واسالِ من 
    نیست» صدق نمی‌کرد.
    
    \item \textbf{سوگند سالزبری (۱۰۸۶):} ویلیام از همه زمین‌داران، حتی 
    واسالان واسالان، سوگند وفاداری مستقیم به خود گرفت.
    
    \item \textbf{پراکندگی تیول‌ها:} زمین‌های اعطایی به هر بارون در نقاط 
    مختلف کشور پخش شده بود، نه در یک بلوک فشرده. این امر از شکل‌گیری 
    قدرت‌های منطقه‌ای مستقل جلوگیری می‌کرد.
    
    \item \textbf{ذخیره‌سازی جنگل سلطنتی:} بخش بزرگی از زمین به‌عنوان 
    «جنگل» (با تعریف حقوقی، نه صرفاً جغرافیایی) تحت کنترل مستقیم شاه 
    قرار گرفت.
\end{enumerate}

\begin{figure}[H]
\centering
\begin{tikzpicture}[
    every node/.style={font=\small},
    level 1/.style={sibling distance=5cm, level distance=2cm},
    level 2/.style={sibling distance=2.5cm, level distance=2cm},
    level 3/.style={sibling distance=1.5cm, level distance=1.8cm}
]
\node[rectangle, rounded corners, draw=royalblue, fill=royalblue!20, 
      thick, minimum width=3cm, minimum height=0.8cm] {شاه\\(Lord Paramount)}
    child {node[rectangle, rounded corners, draw=tudorpurple, fill=tudorpurple!20,
                thick, minimum width=2.5cm, minimum height=0.7cm] {بارون\\(Tenant-in-Chief)}
        child {node[rectangle, rounded corners, draw=victoriangreen, fill=victoriangreen!20,
                    thick, minimum width=2cm, minimum height=0.6cm] {شوالیه\\(Knight)}
            child {node[rectangle, rounded corners, draw=industrialgray, fill=industrialgray!20,
                        thick, minimum width=1.5cm] {دهقان\\(Villein)}}
            child {node[rectangle, rounded corners, draw=industrialgray, fill=industrialgray!20,
                        thick, minimum width=1.5cm] {دهقان\\(Villein)}}
        }
        child {node[rectangle, rounded corners, draw=victoriangreen, fill=victoriangreen!20,
                    thick, minimum width=2cm, minimum height=0.6cm] {شوالیه\\(Knight)}
        }
    }
    child {node[rectangle, rounded corners, draw=tudorpurple, fill=tudorpurple!20,
                thick, minimum width=2.5cm, minimum height=0.7cm] {اسقف\\(Bishop)}
        child {node[rectangle, rounded corners, draw=victoriangreen, fill=victoriangreen!20,
                    thick, minimum width=2cm, minimum height=0.6cm] {کشیش\\(Priest)}}
    }
    child {node[rectangle, rounded corners, draw=tudorpurple, fill=tudorpurple!20,
                thick, minimum width=2.5cm, minimum height=0.7cm] {بارون\\(Tenant-in-Chief)}
    };
\end{tikzpicture}
\caption{سلسله‌مراتب فئودالی در انگلستان نورمان}
\end{figure}

\subsection{کتاب روز داوری (Domesday Book)}

در ۱۰۸۵-۱۰۸۶، ویلیام دستور تهیه یک سرشماری جامع از تمام دارایی‌های 
انگلستان را صادر کرد. این سند، که به «کتاب روز داوری» معروف شد، 
یکی از شگفت‌انگیزترین اسناد اداری قرون وسطی است.

\begin{policybox}[اهمیت کتاب روز داوری]
این سند نشان‌دهنده ظرفیت اداری بی‌سابقه دولت نورمان بود. در هیچ کشور 
اروپایی دیگر چنین سرشماری جامعی تا قرن‌ها بعد انجام نشد. این امر 
نشان می‌دهد که:
\begin{itemize}
    \item دولت انگلستان از همان ابتدا \textbf{ظرفیت استخراج} (مالیاتی) 
    بالایی داشت.
    \item این ظرفیت بر پایه \textbf{اطلاعات دقیق} استوار بود، نه صرفاً 
    زور.
    \item این الگو زمینه‌ساز \textbf{دولت مالی مدرن} شد.
\end{itemize}
\end{policybox}

\subsection{ارزیابی دوره نورمان: نهادهای بنیادین}

فتح نورمان چهار نهاد کلیدی به ارمغان آورد که تاریخ بعدی انگلستان را 
شکل داد:

\begin{table}[H]
\centering
\caption{نهادهای بنیادین دوره نورمان}
\renewcommand{\arraystretch}{1.4}
\begin{tabularx}{\textwidth}{>{\bfseries\color{royalblue}}p{3cm}Xp{4cm}}
\toprule
\textbf{نهاد} & \textbf{کارکرد} & \textbf{پیامد بلندمدت} \\
\midrule
\rowcolor{empirecream}
شورای بزرگ (Curia Regis) & مشاوره شاه با بارون‌ها & تبدیل به پارلمان \\

خزانه‌داری (Exchequer) & مدیریت مالی متمرکز & دولت مالی مدرن \\

\rowcolor{empirecream}
دادگاه‌های سلطنتی & اجرای عدالت شاهانه & کامن‌لا \\

شریف‌ها (Sheriffs) & نمایندگان محلی تاج & حکومت محلی \\

\bottomrule
\end{tabularx}
\end{table}

%----------------------------------------------------------------------
\section{تنش تاج و اشرافیت: از مگناکارتا تا پارلمان (۱۱۵۴-۱۳۰۷)}
%----------------------------------------------------------------------

\subsection{سلسله پلنتاجنت و تمرکز قدرت}

هنری دوم (۱۱۵۴-۱۱۸۹)، نخستین پادشاه پلنتاجنت، اصلاحات گسترده‌ای در 
نظام قضایی و اداری انجام داد. او دادگاه‌های سلطنتی را گسترش داد و 
\textbf{کامن‌لا} (Common Law) را به‌عنوان نظام حقوقی واحد در سراسر 
انگلستان تثبیت کرد.

\begin{historicalquote}
«هنری دوم بیش از هر پادشاه دیگر انگلیسی در شکل‌دهی نظام حقوقی این 
کشور نقش داشت. او نه فقط قوانین جدید وضع کرد، بلکه نهادهایی ساخت 
که این قوانین را اجرا کنند.»
\hfill --- \textcite{warren1973henry}
\end{historicalquote}

\subsubsection{کامن‌لا: نوآوری نهادی کلیدی}

کامن‌لا نظام حقوقی منحصربه‌فردی بود که بر سه اصل استوار بود:

\begin{enumerate}[label=\textcolor{victoriangreen}{\arabic*.}]
    \item \textbf{سابقه قضایی (Precedent):} تصمیمات قبلی دادگاه‌ها 
    الزام‌آور است.
    
    \item \textbf{هیئت منصفه (Jury):} شهروندان عادی در قضاوت شرکت می‌کنند.
    
    \item \textbf{رویه مشترک:} قانون در سراسر کشور یکسان است، نه 
    متفاوت در هر منطقه.
\end{enumerate}

\begin{figure}[H]
\centering
\begin{tikzpicture}[
    node distance=2.5cm,
    box/.style={rectangle, rounded corners=5pt, draw, thick,
                minimum width=3.5cm, minimum height=1.2cm,
                text centered, text width=3.3cm, font=\small},
    arrow/.style={-{Stealth[length=3mm]}, thick}
]

% مقایسه نظام‌های حقوقی
\node[box, fill=royalblue!20] (common) {کامن‌لا\\(انگلستان)};
\node[box, fill=imperialred!20, right=4cm of common] (civil) {حقوق مدنی\\(اروپای قاره‌ای)};

% ویژگی‌ها
\node[box, fill=empirecream, below=1.5cm of common] (c1) {مبتنی بر سابقه\\قاضی‌محور};
\node[box, fill=empirecream, below=1.5cm of civil] (c2) {مبتنی بر قانون مدون\\قانون‌گذارمحور};

\node[box, fill=empirecream, below=0.8cm of c1] (c3) {هیئت منصفه\\مشارکت شهروندان};
\node[box, fill=empirecream, below=0.8cm of c2] (c4) {قاضی حرفه‌ای\\بدون هیئت منصفه};

\node[box, fill=victoriangreen!20, below=0.8cm of c3] (c5) {حمایت قوی از\\حقوق مالکیت};
\node[box, fill=imperialred!10, below=0.8cm of c4] (c6) {مداخله بیشتر\\دولت در قراردادها};

% فلش‌ها
\draw[arrow] (common) -- (c1);
\draw[arrow] (civil) -- (c2);
\draw[arrow] (c1) -- (c3);
\draw[arrow] (c2) -- (c4);
\draw[arrow] (c3) -- (c5);
\draw[arrow] (c4) -- (c6);

% عنوان
\node[above=0.5cm of common, xshift=2.5cm, font=\bfseries\large, color=royalblue] 
    {مقایسه دو نظام حقوقی};

\end{tikzpicture}
\caption{مقایسه کامن‌لا و حقوق مدنی قاره‌ای}
\end{figure}

\begin{policybox}[اهمیت کامن‌لا برای توسعه]
پژوهش‌های \textcite{laporta1998law} نشان داده است که کشورهای با میراث 
کامن‌لا (انگلستان، آمریکا، استرالیا، هند) به‌طور سیستماتیک:
\begin{itemize}
    \item حمایت قوی‌تری از سرمایه‌گذاران دارند
    \item بازارهای مالی توسعه‌یافته‌تری دارند
    \item مقررات کسب‌وکار کمتری دارند
\end{itemize}
این امر نشان می‌دهد که نوآوری نهادی هنری دوم پیامدهای هزارساله داشته است.
\end{policybox}

\subsection{بحران جان بی‌سرزمین و مگناکارتا (۱۲۱۵)}

جان (۱۱۹۹-۱۲۱۶)، پسر هنری دوم، با سه بحران همزمان مواجه شد:

\begin{enumerate}
    \item \textbf{از دست دادن قلمروهای فرانسوی:} نرماندی و آنژو در ۱۲۰۴ 
    به دست فیلیپ دوم فرانسه افتاد.
    
    \item \textbf{تنش با پاپ:} انگلستان از ۱۲۰۸ تا ۱۲۱۴ تحت تحریم 
    (Interdict) کلیسا بود.
    
    \item \textbf{مالیات‌های سنگین:} جان برای بازپس‌گیری فرانسه، مالیات‌های 
    بی‌سابقه وضع کرد.
\end{enumerate}

این بحران‌ها منجر به شورش بارون‌ها شد. در ۱۵ ژوئن ۱۲۱۵، جان مجبور 
به امضای \textbf{مگناکارتا} (Magna Carta) شد.

\subsubsection{محتوای مگناکارتا}

مگناکارتا سندی بود با ۶۳ بند که بیشتر آن‌ها به مسائل فئودالی خاص 
می‌پرداخت. اما چند بند آن اهمیت تاریخی ماندگار یافت:

\begin{table}[H]
\centering
\caption{بندهای کلیدی مگناکارتا}
\renewcommand{\arraystretch}{1.4}
\begin{tabularx}{\textwidth}{>{\bfseries}c>{\itshape}Xp{4cm}}
\toprule
\textbf{بند} & \textbf{متن (خلاصه)} & \textbf{اهمیت} \\
\midrule
\rowcolor{empirecream}
۱۲ & هیچ مالیات فوق‌العاده‌ای بدون رضایت شورای عمومی قلمرو وضع نشود & اصل «بدون نمایندگی، مالیات نه» \\

۳۹ & هیچ انسان آزادی دستگیر، زندانی، یا تبعید نشود مگر با حکم قانونی همتایانش & حبیث کورپوس، حاکمیت قانون \\

\rowcolor{empirecream}
۴۰ & به هیچ‌کس عدالت نمی‌فروشیم، از هیچ‌کس دریغ نمی‌کنیم، تأخیر نمی‌اندازیم & دسترسی به عدالت \\

۶۱ & شورای ۲۵ بارون حق نظارت بر اجرای منشور را دارد & نظارت بر تاج \\

\bottomrule
\end{tabularx}
\end{table}

\subsubsection{اهمیت تاریخی مگناکارتا}

\begin{historicalquote}
«مگناکارتا در زمان خود یک سند فئودالی بود، نه یک منشور آزادی. اما 
نسل‌های بعد آن را بازتفسیر کردند و به نماد محدودیت قدرت تبدیل کردند. 
این بازتفسیر خود واقعیت تاریخی شد.»
\hfill --- \textcite{holt1992magna}
\end{historicalquote}

\begin{figure}[H]
\centering
\begin{tikzpicture}[
    timeline/.style={very thick, royalblue},
    event/.style={circle, fill=parliamentgold, minimum size=8pt, inner sep=0pt},
    label/.style={font=\small, text width=3cm, align=center}
]

% خط زمانی
\draw[timeline] (0,0) -- (14,0);

% رویدادها
\foreach \x/\year/\desc in {
    0/1215/{امضای اولیه\\(فوراً لغو شد)},
    2.5/1225/{تأیید مجدد\\هنری سوم},
    5/1297/{تأیید ادوارد اول\\ورود به قانون موضوعه},
    8/1628/{دادخواست حق\\استناد به مگناکارتا},
    11/1689/{لایحه حقوق\\تداوم سنت},
    14/1791/{قانون اساسی آمریکا\\الهام از مگناکارتا}
} {
    \node[event] at (\x,0) {};
    \node[below=0.3cm, label] at (\x,0) {\year};
    \node[above=0.3cm, label] at (\x,0) {\desc};
}

% عنوان
\node[above=2.5cm, font=\bfseries\large, color=royalblue] at (7,0) 
    {تأثیر بلندمدت مگناکارتا};

\end{tikzpicture}
\caption{خط زمانی تأثیر مگناکارتا از ۱۲۱۵ تا قانون اساسی آمریکا}
\end{figure}

\subsection{ظهور پارلمان (۱۲۶۵-۱۳۰۷)}

پارلمان انگلستان از «شورای بزرگ» فئودالی به تدریج تکامل یافت. نقطه 
عطف، پارلمان سیمون دو مونفور در ۱۲۶۵ بود که برای نخستین بار نمایندگان 
شهرها و شایرها (نه فقط بارون‌ها و اسقف‌ها) را شامل شد.

\subsubsection{ساختار پارلمان}

\begin{figure}[H]
\centering
\begin{tikzpicture}[
    node distance=1.5cm,
    box/.style={rectangle, rounded corners, draw, thick,
                minimum width=4cm, minimum height=1cm,
                text centered, font=\small}
]

% پارلمان
\node[box, fill=royalblue!30, minimum width=10cm, minimum height=1.2cm] (parliament) 
    {\Large\textbf{پارلمان}};

% دو مجلس
\node[box, fill=tudorpurple!20, below left=1.5cm and 0.5cm of parliament] (lords) 
    {مجلس لردها\\(House of Lords)};
\node[box, fill=victoriangreen!20, below right=1.5cm and 0.5cm of parliament] (commons) 
    {مجلس عوام\\(House of Commons)};

% اجزای لردها
\node[box, fill=empirecream, below left=1cm and -0.5cm of lords, text width=2.5cm] (lords1) 
    {لردهای روحانی\\(اسقف‌ها)};
\node[box, fill=empirecream, below right=1cm and -0.5cm of lords, text width=2.5cm] (lords2) 
    {لردهای دنیوی\\(بارون‌ها)};

% اجزای عوام
\node[box, fill=empirecream, below left=1cm and -0.5cm of commons, text width=2.5cm] (commons1) 
    {شوالیه‌های شایر\\(Knights of Shire)};
\node[box, fill=empirecream, below right=1cm and -0.5cm of commons, text width=2.5cm] (commons2) 
    {نمایندگان شهرها\\(Burgesses)};

% فلش‌ها
\draw[-{Stealth}, thick] (parliament) -- (lords);
\draw[-{Stealth}, thick] (parliament) -- (commons);
\draw[-{Stealth}, thick] (lords) -- (lords1);
\draw[-{Stealth}, thick] (lords) -- (lords2);
\draw[-{Stealth}, thick] (commons) -- (commons1);
\draw[-{Stealth}, thick] (commons) -- (commons2);

\end{tikzpicture}
\caption{ساختار دومجلسی پارلمان انگلستان}
\end{figure}

\subsubsection{پارلمان نمونه (Model Parliament) ۱۲۹۵}

ادوارد اول در ۱۲۹۵ پارلمانی فراخواند که به «پارلمان نمونه» معروف شد، 
زیرا ترکیب آن الگوی پارلمان‌های بعدی شد:

\begin{itemize}
    \item تمام اسقف‌ها و رؤسای صومعه‌ها
    \item تمام ارل‌ها و بارون‌ها
    \item دو شوالیه از هر شایر
    \item دو نماینده (burgess) از هر شهر
\end{itemize}

\begin{policybox}[چرا تاج به پارلمان نیاز داشت؟]
شاهان انگلستان برخلاف همتایان فرانسوی خود، نمی‌توانستند مالیات دائمی 
بدون رضایت وضع کنند. این امر ریشه در:
\begin{enumerate}
    \item \textbf{سنت مگناکارتا:} اصل رضایت در مالیات
    \item \textbf{ضعف نظامی تاج:} فقدان ارتش دائمی بزرگ
    \item \textbf{قدرت محلی ژنتری:} اجرای قانون وابسته به همکاری آن‌ها
\end{enumerate}
بنابراین، تاج برای تأمین مالی جنگ‌ها مجبور به مذاکره بود. این مذاکره 
در پارلمان صورت می‌گرفت و پارلمان را تقویت می‌کرد.
\end{policybox}

%----------------------------------------------------------------------
\section{بحران قرن چهاردهم و تحول طبقاتی (۱۳۰۷-۱۴۸۵)}
%----------------------------------------------------------------------

\subsection{مرگ سیاه و پیامدهای اجتماعی}

طاعون بزرگ (مرگ سیاه) در ۱۳۴۸-۱۳۴۹ حدود یک‌سوم تا نیمی از جمعیت 
انگلستان را کشت. این فاجعه پیامدهای ژرفی برای روابط طبقاتی داشت:

\begin{figure}[H]
\centering
\begin{tikzpicture}
\begin{axis}[
    width=13cm,
    height=8cm,
    xlabel={سال},
    ylabel={جمعیت انگلستان (میلیون نفر)},
    xmin=1300, xmax=1500,
    ymin=0, ymax=7,
    xtick={1300, 1350, 1400, 1450, 1500},
    ytick={0, 1, 2, 3, 4, 5, 6, 7},
    grid=major,
    legend pos=south east,
]
\addplot[very thick, royalblue, mark=*] coordinates {
    (1300, 5.0)
    (1340, 6.0)
    (1350, 3.0)
    (1370, 2.5)
    (1400, 2.5)
    (1450, 2.2)
    (1500, 2.5)
};
\addlegendentry{جمعیت تخمینی}

% خط طاعون
\draw[very thick, imperialred, dashed] (axis cs:1348,0) -- (axis cs:1348,7);
\node[imperialred, font=\small] at (axis cs:1348,6.5) {مرگ سیاه};

\end{axis}
\end{tikzpicture}
\caption{تأثیر مرگ سیاه بر جمعیت انگلستان}
\end{figure}

\subsubsection{پیامدهای اقتصادی}

\begin{table}[H]
\centering
\caption{پیامدهای اقتصادی مرگ سیاه}
\renewcommand{\arraystretch}{1.4}
\begin{tabularx}{\textwidth}{>{\bfseries}p{3cm}XX}
\toprule
\textbf{حوزه} & \textbf{تغییر} & \textbf{پیامد طبقاتی} \\
\midrule
\rowcolor{empirecream}
نیروی کار & کمبود شدید & افزایش قدرت چانه‌زنی دهقانان \\

دستمزدها & افزایش ۵۰-۱۰۰٪ & کاهش سود زمین‌داران \\

\rowcolor{empirecream}
اجاره زمین & کاهش & فشار بر درآمد اشرافیت \\

سرواژ & تضعیف & گسترش کار مزدی آزاد \\

\rowcolor{empirecream}
قیمت زمین & کاهش & فرصت خرید برای یومن‌ها \\

\bottomrule
\end{tabularx}
\end{table}

\subsubsection{قانون کارگران (Statute of Labourers) ۱۳۵۱}

اشرافیت و ژنتری تلاش کردند با قانون‌گذاری جلوی افزایش دستمزدها را 
بگیرند. قانون کارگران ۱۳۵۱:

\begin{itemize}
    \item دستمزدها را در سطح قبل از طاعون ثابت کرد
    \item ترک کار بدون اجازه را ممنوع کرد
    \item گدایی افراد سالم را جرم‌انگاری کرد
\end{itemize}

اما اجرای این قانون دشوار بود و مقاومت گسترده‌ای برانگیخت.

\subsection{شورش دهقانان (۱۳۸۱)}

در ۱۳۸۱، نارضایتی انباشته‌شده منفجر شد. دهقانان به رهبری وات تایلر 
و جان بال شورش کردند و تا لندن پیش رفتند.

\begin{historicalquote}
«وقتی آدم می‌کاشت و حوا می‌ریسید، چه کسی جنتلمن بود؟»
\hfill --- جان بال، واعظ شورشی
\end{historicalquote}

شورش سرکوب شد، اما پیامدهای بلندمدت داشت:

\begin{enumerate}[label=\textcolor{imperialred}{\arabic*.}]
    \item \textbf{پایان سرواژ:} طی یک قرن بعد، سرواژ عملاً از انگلستان 
    ناپدید شد.
    
    \item \textbf{هشدار به نخبگان:} نشان داد که فشار بیش‌ازحد می‌تواند 
    به شورش بینجامد.
    
    \item \textbf{ظهور یومن‌ها:} دهقانان آزاد و مرفه که بعداً ستون فقرات 
    ژنتری شدند.
\end{enumerate}

\subsection{جنگ صدساله و جنگ گل‌ها}

دو جنگ بزرگ قرن چهاردهم و پانزدهم تأثیر ژرفی بر ساختار طبقاتی داشت:

\begin{figure}[H]
\centering
\begin{tikzpicture}[
    node distance=0.3cm,
    war/.style={rectangle, rounded corners, draw, thick,
                minimum width=5cm, minimum height=1.5cm,
                text centered, text width=4.8cm, font=\small}
]

% جنگ صدساله
\node[war, fill=royalblue!20] (hundred) {
    \textbf{جنگ صدساله}\\
    (۱۳۳۷-۱۴۵۳)\\
    انگلستان vs فرانسه
};

% جنگ گل‌ها
\node[war, fill=imperialred!20, right=2cm of hundred] (roses) {
    \textbf{جنگ گل‌ها}\\
    (۱۴۵۵-۱۴۸۵)\\
    یورک vs لنکستر
};

% پیامدها
\node[war, fill=victoriangreen!20, below=1.5cm of hundred] (h_effect) {
    \textbf{پیامدها:}\\
    - قدرت‌یابی پارلمان (مالیات جنگ)\\
    - هویت ملی انگلیسی\\
    - تضعیف سلطنت
};

\node[war, fill=parliamentgold!20, below=1.5cm of roses] (r_effect) {
    \textbf{پیامدها:}\\
    - نابودی اشرافیت کهن\\
    - تمرکز قدرت در تاج تودور\\
    - ظهور اشرافیت جدید
};

% فلش‌ها
\draw[-{Stealth}, thick] (hundred) -- (h_effect);
\draw[-{Stealth}, thick] (roses) -- (r_effect);
\draw[-{Stealth}, thick, dashed] (hundred) -- (roses) 
    node[midway, above, font=\tiny] {تنش‌های داخلی};

\end{tikzpicture}
\caption{جنگ صدساله و جنگ گل‌ها: علل و پیامدها}
\end{figure}

\subsubsection{جنگ گل‌ها و نابودی اشرافیت کهن}

جنگ داخلی بین خاندان‌های یورک و لنکستر (هر دو شاخه پلنتاجنت) بخش 
بزرگی از اشرافیت قدیم را نابود کرد:

\begin{table}[H]
\centering
\caption{تلفات اشرافیت در جنگ گل‌ها}
\renewcommand{\arraystretch}{1.3}
\begin{tabular}{lcc}
\toprule
\textbf{شاخص} & \textbf{۱۴۵۵} & \textbf{۱۴۸۵} \\
\midrule
تعداد دوک‌ها & ۵ & ۱ \\
تعداد ارل‌ها & ۲۳ & ۱۰ \\
تعداد بارون‌ها & ۵۳ & ۲۹ \\
\midrule
\textbf{مجموع} & \textbf{۸۱} & \textbf{۴۰} \\
\bottomrule
\end{tabular}
\end{table}

\begin{policybox}[اهمیت نابودی اشرافیت کهن]
این تصفیه خونین پیامد غیرمنتظره‌ای داشت: وقتی هنری هفتم تودور در 
۱۴۸۵ به قدرت رسید، با اشرافیتی ضعیف و وابسته مواجه بود. این امر 
به او امکان داد:
\begin{itemize}
    \item اشرافیت جدید وفادار به خود بسازد
    \item قدرت سلطنتی را تمرکز دهد
    \item زمینه دولت-ملت مدرن را فراهم کند
\end{itemize}
\end{policybox}

%----------------------------------------------------------------------
\section{جمع‌بندی فصل: میراث دوره فئودالی}
%----------------------------------------------------------------------

\subsection{نهادهای کلیدی شکل‌گرفته}

\begin{figure}[H]
\centering
\begin{tikzpicture}[
    mindmap,
    grow cyclic,
    every node/.style={concept, circular drop shadow},
    concept color=royalblue!30,
    level 1/.append style={level distance=4cm, sibling angle=90},
    level 2/.append style={level distance=2.5cm, sibling angle=45},
    font=\small
]
\node[concept, font=\bfseries] {میراث نهادی\\۱۰۶۶-۱۴۸۵}
    child[concept color=tudorpurple!30] { node {پارلمان}
        child { node[font=\tiny] {حق مالیات} }
        child { node[font=\tiny] {نمایندگی} }
    }
    child[concept color=victoriangreen!30] { node {کامن‌لا}
        child { node[font=\tiny] {حقوق مالکیت} }
        child { node[font=\tiny] {هیئت منصفه} }
    }
    child[concept color=parliamentgold!30] { node {سلطنت مشروط}
        child { node[font=\tiny] {مگناکارتا} }
        child { node[font=\tiny] {محدودیت قانونی} }
    }
    child[concept color=imperialred!30] { node {حکومت محلی}
        child { node[font=\tiny] {ژنتری} }
        child { node[font=\tiny] {صلح قضات} }
    };
\end{tikzpicture}
\caption{میراث نهادی دوره فئودالی}
\end{figure}

\subsection{تفاوت با اروپای قاره‌ای}

\begin{table}[H]
\centering
\caption{مقایسه انگلستان با فرانسه در پایان قرون وسطی}
\renewcommand{\arraystretch}{1.4}
\begin{tabularx}{\textwidth}{>{\bfseries}p{3cm}XX}
\toprule
\textbf{بُعد} & \textbf{انگلستان ۱۴۸۵} & \textbf{فرانسه ۱۴۸۵} \\
\midrule
\rowcolor{empirecream}
سلطنت & مشروط، وابسته به پارلمان & در حال تبدیل به مطلقه \\

اشرافیت & تضعیف‌شده، وابسته به تاج & قدرتمند، منطقه‌ای \\

\rowcolor{empirecream}
پارلمان/مجلس & فعال، منظم، قدرتمند & استیت‌جنرال ضعیف \\

نظام حقوقی & کامن‌لای واحد & حقوق محلی متنوع \\

\rowcolor{empirecream}
مالیات & نیازمند رضایت پارلمان & مالیات دائمی شاه (taille) \\

ارتش & بدون ارتش دائمی بزرگ & ارتش دائمی در حال شکل‌گیری \\

\bottomrule
\end{tabularx}
\end{table}

\subsection{نتیجه‌گیری تحلیلی}

دوره ۱۰۶۶-۱۴۸۵ چهار \textbf{مسیر وابستگی} کلیدی برای انگلستان ایجاد کرد:

\begin{enumerate}[label=\textcolor{royalblue}{\textbf{\arabic*.}}]
    \item \textbf{سلطنت قدرتمند اما مشروط:} شاهان انگلیسی از همان ابتدا 
    مجبور به مذاکره با اشرافیت بودند. این الگو در مگناکارتا تثبیت شد.
    
    \item \textbf{پارلمان به‌عنوان نهاد میانجی:} پارلمان میدانی شد که 
    تاج و طبقات می‌توانستند در آن سازش کنند، به جای آنکه در میدان جنگ 
    بجنگند.
    
    \item \textbf{کامن‌لا و امنیت مالکیت:} نظام حقوقی واحد و مبتنی بر 
    سابقه، زمینه امنیت سرمایه‌گذاری را فراهم کرد.
    
    \item \textbf{اشرافیت انعطاف‌پذیر:} اشرافیت انگلیسی، برخلاف همتای 
    فرانسوی، از همان ابتدا با تجارت و زمین‌داری تجاری آشنا بود.
\end{enumerate}

\begin{warningbox}
\textbf{هشدار تحلیلی:} این نهادها در ۱۴۸۵ هنوز به شکل امروزی نبودند. 
پارلمان هنوز نهادی اشرافی بود، نه دموکراتیک. حقوق مالکیت هنوز کامل 
نبود. سلطنت هنوز می‌توانست مستبد شود. آنچه در این دوره شکل گرفت، 
\textbf{بذرهای} نهادهای بعدی بود، نه خود آن نهادها.
\end{warningbox}

%----------------------------------------------------------------------
% ارجاعات فصل دوم
%----------------------------------------------------------------------

\section*{ارجاعات فصل دوم}

\begin{enumerate}[label={[\arabic*]}]
    \item Holt, J.C. (1992). \textit{Magna Carta}. 2nd ed. Cambridge: 
    Cambridge University Press.
    
    \item La Porta, R., Lopez-de-Silanes, F., Shleifer, A. \& Vishny, R.W. 
    (1998). Law and Finance. \textit{Journal of Political Economy}, 106(6), 
    1113-1155.
    
    \item Maitland, F.W. (1897). \textit{Domesday Book and Beyond}. Cambridge: 
    Cambridge University Press.
    
    \item Warren, W.L. (1973). \textit{Henry II}. London: Eyre Methuen.
    
    \item Hilton, R.H. (1973). \textit{Bond Men Made Free: Medieval Peasant 
    Movements and the English Rising of 1381}. London: Temple Smith.
    
    \item Given-Wilson, C. (2016). \textit{Henry IV}. New Haven: Yale 
    University Press.
    
    \item Harriss, G.L. (1975). \textit{King, Parliament, and Public Finance 
    in Medieval England to 1369}. Oxford: Clarendon Press.
    
    \item Ormrod, W.M. (1990). \textit{The Reign of Edward III: Crown and 
    Political Society in England, 1327-1377}. New Haven: Yale University Press.
    
    \item Pollard, A.J. (2001). \textit{The Wars of the Roses}. 2nd ed. 
    Basingstoke: Palgrave.
    
    \item Clanchy, M.T. (1993). \textit{From Memory to Written Record: 
    England 1066-1307}. 2nd ed. Oxford: Blackwell.
\end{enumerate}

%%%%%%%%%%%%%%%%%%%%%%%%%%%%%%%%%%%%%%%%%%%%%%%%%%%%%%%%%%%%%%%%%%%%%%%
% پایان فصل دوم
%%%%%%%%%%%%%%%%%%%%%%%%%%%%%%%%%%%%%%%%%%%%%%%%%%%%%%%%%%%%%%%%%%%%%%%

\vfill

\begin{center}
\textcolor{royalblue}{\rule{0.5\textwidth}{1pt}}\\[0.5cm]
\textit{پایان فصل دوم}\\[0.3cm]
\textbf{ادامه دارد: فصل سوم (دوره تودور و استوارت: ۱۴۸۵-۱۷۱۴)}\\[0.5cm]
\textcolor{royalblue}{\rule{0.5\textwidth}{1pt}}
\end{center}

%%%%%%%%%%%%%%%%%%%%%%%%%%%%%%%%%%%%%%%%%%%%%%%%%%%%%%%%%%%%%%%%%%%%%%%
% فصل سوم: دوره تودور و استوارت (۱۴۸۵-۱۷۱۴)
%%%%%%%%%%%%%%%%%%%%%%%%%%%%%%%%%%%%%%%%%%%%%%%%%%%%%%%%%%%%%%%%%%%%%%%

\chapter{دوره تودور و استوارت: بحران و انقلاب (۱۴۸۵-۱۷۱۴)}

\begin{keybox}[خلاصه فصل]
این فصل به تحلیل دو قرن سرنوشت‌ساز می‌پردازد که طی آن انگلستان از 
سلطنت فئودالی به \textbf{نخستین دولت-ملت مدرن} با \textbf{سلطنت مشروطه} 
تبدیل شد. این گذار بدون خونریزی نبود: یک شورش دینی (اصلاحات)، یک جنگ 
داخلی، یک اعدام شاه، یک جمهوری، یک بازگشت، و سرانجام یک «انقلاب 
شکوهمند». اما برخلاف فرانسه، این تحولات به \textbf{تثبیت نهادی} انجامید، 
نه به چرخه انقلاب و ارتجاع. کلید این موفقیت، \textbf{توانایی نخبگان در 
سازش} و \textbf{انعطاف‌پذیری نهادی پارلمان} بود.
\end{keybox}

%----------------------------------------------------------------------
\section{سلسله تودور: دولت‌سازی و اصلاحات (۱۴۸۵-۱۶۰۳)}
%----------------------------------------------------------------------

\subsection{هنری هفتم و بازسازی سلطنت (۱۴۸۵-۱۵۰۹)}

هنری تودور با پیروزی در نبرد بازورث (۱۴۸۵) به تاج رسید. او با چالشی 
دوگانه مواجه بود: مشروعیت ضعیف (ادعای او به تاج از طریق مادرش و 
قابل تردید بود) و خزانه خالی.

\begin{historicalquote}
«هنری هفتم نه یک نوآور بزرگ، بلکه یک مدیر بزرگ بود. او نظم را برقرار 
کرد، خزانه را پر کرد، و سلطنتی قوی اما غیرمستبد بنا نهاد.»
\hfill --- \textcite{chrimes1972henry}
\end{historicalquote}

\subsubsection{سیاست‌های هنری هفتم}

\begin{table}[H]
\centering
\caption{سیاست‌های کلیدی هنری هفتم}
\renewcommand{\arraystretch}{1.4}
\begin{tabularx}{\textwidth}{>{\bfseries}p{3cm}Xp{4cm}}
\toprule
\textbf{حوزه} & \textbf{سیاست} & \textbf{پیامد} \\
\midrule
\rowcolor{empirecream}
مالی & مدیریت دقیق درآمدها، اجتناب از جنگ & خزانه پر، استقلال از پارلمان \\

اشرافیت & قوانین ضد نگهداری ارتش خصوصی (Livery) & تضعیف قدرت نظامی اشراف \\

\rowcolor{empirecream}
قضایی & دادگاه اتاق ستاره (Star Chamber) & کنترل اشراف متمرد \\

تجارت & معاهدات تجاری، حمایت از بازرگانان & رشد بورژوازی تجاری \\

\rowcolor{empirecream}
سلسله‌ای & ازدواج‌های سیاسی & تثبیت سلسله تودور \\

\bottomrule
\end{tabularx}
\end{table}

\subsection{هنری هشتم و اصلاحات دینی (۱۵۰۹-۱۵۴۷)}

هنری هشتم یکی از تحول‌آفرین‌ترین پادشاهان تاریخ انگلستان بود. اصلاحات 
دینی او، هرچند با انگیزه‌های شخصی (طلاق از کاترین آراگون) آغاز شد، 
پیامدهای ساختاری عظیمی داشت.

\subsubsection{جدایی از رُم (۱۵۳۴)}

\begin{figure}[H]
\centering
\begin{tikzpicture}[
    node distance=1.5cm,
    event/.style={rectangle, rounded corners, draw, thick,
                  minimum width=4cm, minimum height=1.2cm,
                  text centered, text width=3.8cm, font=\small},
    arrow/.style={-{Stealth[length=3mm]}, thick}
]

% زنجیره رویدادها
\node[event, fill=royalblue!20] (marriage) 
    {۱۵۰۹: ازدواج با کاترین آراگون};
\node[event, fill=imperialred!20, below=of marriage] (divorce) 
    {۱۵۲۷: درخواست ابطال ازدواج\\پاپ رد می‌کند};
\node[event, fill=parliamentgold!20, below=of divorce] (supremacy) 
    {۱۵۳۴: قانون سیادت\\شاه رئیس کلیسای انگلستان};
\node[event, fill=victoriangreen!20, below=of supremacy] (dissolution) 
    {۱۵۳۶-۱۵۴۱: انحلال صومعه‌ها\\مصادره اموال کلیسا};
\node[event, fill=tudorpurple!20, below=of dissolution] (redistribution) 
    {فروش زمین‌های کلیسا\\به ژنتری و بورژوازی};

% فلش‌ها
\draw[arrow] (marriage) -- (divorce);
\draw[arrow] (divorce) -- (supremacy);
\draw[arrow] (supremacy) -- (dissolution);
\draw[arrow] (dissolution) -- (redistribution);

% پیامد نهایی
\node[event, fill=victoriangreen!40, right=3cm of dissolution, 
      minimum width=4.5cm, text width=4.3cm] (outcome) 
    {\textbf{پیامد ساختاری:}\\
     - کلیسای ملی تحت کنترل تاج\\
     - انتقال عظیم ثروت\\
     - تقویت ژنتری\\
     - پیوند منافع با اصلاحات};

\draw[arrow, dashed] (redistribution) -- (outcome);

\end{tikzpicture}
\caption{زنجیره علّی اصلاحات دینی هنری هشتم}
\end{figure}

\subsubsection{انحلال صومعه‌ها: بزرگ‌ترین انتقال ثروت}

انحلال صومعه‌ها (۱۵۳۶-۱۵۴۱) یکی از بزرگ‌ترین انتقال‌های ثروت در 
تاریخ انگلستان بود:

\begin{figure}[H]
\centering
\begin{tikzpicture}
\begin{axis}[
    ybar stacked,
    width=12cm,
    height=7cm,
    ylabel={درصد زمین انگلستان},
    symbolic x coords={قبل از ۱۵۳۶, بعد از ۱۵۴۷},
    xtick=data,
    ymin=0,
    ymax=100,
    bar width=25pt,
    legend style={at={(0.5,-0.15)}, anchor=north, legend columns=4},
    nodes near coords={\pgfmathprintnumber{\pgfplotspointmeta}\%},
    nodes near coords align={center},
    every node near coord/.append style={font=\tiny},
]

% قبل از انحلال
\addplot[fill=royalblue!60] coordinates {(قبل از ۱۵۳۶,5) (بعد از ۱۵۴۷,10)};
\addplot[fill=tudorpurple!60] coordinates {(قبل از ۱۵۳۶,20) (بعد از ۱۵۴۷,15)};
\addplot[fill=victoriangreen!60] coordinates {(قبل از ۱۵۳۶,25) (بعد از ۱۵۴۷,40)};
\addplot[fill=parliamentgold!60] coordinates {(قبل از ۱۵۳۶,25) (بعد از ۱۵۴۷,20)};
\addplot[fill=industrialgray!60] coordinates {(قبل از ۱۵۳۶,25) (بعد از ۱۵۴۷,15)};

\legend{تاج, اشرافیت, ژنتری و بورژوازی, یومن‌ها, کلیسا}

\end{axis}
\end{tikzpicture}
\caption{تغییر توزیع مالکیت زمین پس از انحلال صومعه‌ها}
\end{figure}

\begin{policybox}[اهمیت ساختاری انحلال صومعه‌ها]
انحلال صومعه‌ها فراتر از یک اقدام مالی بود. این اقدام:
\begin{enumerate}
    \item \textbf{ژنتری را به اصلاحات پیوند زد:} خریداران زمین‌های 
    کلیسایی منفعتی حیاتی در جلوگیری از بازگشت کاتولیسیسم داشتند.
    
    \item \textbf{طبقه‌ای جدید آفرید:} بسیاری از خانواده‌های ژنتری و 
    اشرافی که بعداً در تاریخ انگلستان نقش ایفا کردند، ثروت خود را از 
    این زمین‌ها کسب کردند.
    
    \item \textbf{قدرت کلیسا را شکست:} کلیسای انگلستان دیگر هرگز به 
    قدرت اقتصادی قبلی بازنگشت.
    
    \item \textbf{بازار زمین را فعال کرد:} زمین از دارایی ثابت به کالای 
    قابل خرید و فروش تبدیل شد.
\end{enumerate}
\end{policybox}

\subsection{ادوارد ششم و مری یکم: تنش دینی (۱۵۴۷-۱۵۵۸)}

پس از هنری هشتم، دو فرزندش سیاست‌های متضادی پیش گرفتند:

\begin{table}[H]
\centering
\caption{مقایسه سیاست‌های دینی ادوارد ششم و مری یکم}
\renewcommand{\arraystretch}{1.4}
\begin{tabularx}{\textwidth}{>{\bfseries}p{3cm}XX}
\toprule
\textbf{پادشاه} & \textbf{سیاست دینی} & \textbf{پیامد} \\
\midrule
\rowcolor{victoriangreen!10}
ادوارد ششم (۱۵۴۷-۱۵۵۳) & 
پروتستانتیسم رادیکال، کتاب دعای مشترک، حذف آیین کاتولیک & 
تعمیق اصلاحات، مقاومت محافظه‌کاران \\

\rowcolor{imperialred!10}
مری یکم (۱۵۵۳-۱۵۵۸) & 
بازگشت به کاتولیسیسم، اتحاد با رُم، سوزاندن پروتستان‌ها & 
لقب «مری خونین»، نفرت عمومی از کاتولیسیسم \\

\bottomrule
\end{tabularx}
\end{table}

دوره کوتاه مری یکم پیامد غیرمنتظره‌ای داشت: کاتولیسیسم در اذهان عمومی 
با استبداد و خارجی بودن (ازدواج مری با فیلیپ دوم اسپانیا) پیوند خورد. 
این امر زمینه‌ساز تثبیت پروتستانتیسم در دوره الیزابت شد.

\subsection{الیزابت یکم: عصر طلایی (۱۵۵۸-۱۶۰۳)}

الیزابت یکم چهل و پنج سال حکومت کرد و دوره او به «عصر طلایی» معروف است. 
او با مهارت بین جناح‌های مختلف موازنه برقرار کرد.

\subsubsection{تسویه الیزابتی (Elizabethan Settlement)}

\begin{figure}[H]
\centering
\begin{tikzpicture}[
    node distance=2cm,
    box/.style={rectangle, rounded corners=8pt, draw=royalblue, thick,
                minimum width=3.5cm, minimum height=1.5cm,
                text centered, text width=3.3cm, font=\small}
]

% مرکز
\node[box, fill=parliamentgold!30, minimum width=5cm, minimum height=2cm] (settlement) 
    {\Large\textbf{تسویه الیزابتی}\\«راه میانه»};

% اطراف
\node[box, fill=tudorpurple!20, above left=2cm and 1cm of settlement] (catholic) 
    {کاتولیک‌ها:\\آیین سنتی\\سلسله‌مراتب};

\node[box, fill=victoriangreen!20, above right=2cm and 1cm of settlement] (protestant) 
    {پروتستان‌ها:\\کتاب مقدس\\ساده‌سازی};

\node[box, fill=empirecream, below left=2cm and 1cm of settlement] (take_cath) 
    {از کاتولیک:\\اسقف‌ها\\برخی آیین‌ها};

\node[box, fill=empirecream, below right=2cm and 1cm of settlement] (take_prot) 
    {از پروتستان:\\الهیات\\کتاب دعا};

% فلش‌ها
\draw[-{Stealth}, thick, dashed] (catholic) -- (settlement);
\draw[-{Stealth}, thick, dashed] (protestant) -- (settlement);
\draw[-{Stealth}, thick] (settlement) -- (take_cath);
\draw[-{Stealth}, thick] (settlement) -- (take_prot);

\end{tikzpicture}
\caption{تسویه الیزابتی: راه میانه بین کاتولیسیسم و پروتستانتیسم رادیکال}
\end{figure}

\subsubsection{تحولات اقتصادی-اجتماعی دوره الیزابت}

\begin{table}[H]
\centering
\caption{شاخص‌های توسعه در دوره الیزابت}
\renewcommand{\arraystretch}{1.3}
\begin{tabular}{p{5cm}cc}
\toprule
\textbf{شاخص} & \textbf{۱۵۵۸} & \textbf{۱۶۰۳} \\
\midrule
جمعیت انگلستان (میلیون) & ۳.۰ & ۴.۱ \\
جمعیت لندن (هزار) & ۱۰۰ & ۲۰۰ \\
صادرات پشم (تن) & ۱۰۰,۰۰۰ & ۲۵۰,۰۰۰ \\
تعداد کشتی‌های تجاری بزرگ & ۵۰ & ۳۵۰ \\
دانشگاه‌ها و مدارس & محدود & گسترش یافته \\
\bottomrule
\end{tabular}
\end{table}

\begin{historicalquote}
«من قلب و شکم یک پادشاه انگلیس را دارم، و از اینکه پارما یا اسپانیا 
یا هر شاهزاده اروپایی جرأت تجاوز به مرزهای قلمروم را داشته باشد، 
ننگ می‌دانم.»
\hfill --- الیزابت یکم، سخنرانی تیلبری (۱۵۸۸)
\end{historicalquote}

\subsubsection{ظهور ژنتری و بورژوازی}

دوره تودور شاهد ظهور دو طبقه کلیدی بود که تاریخ بعدی انگلستان را 
شکل دادند:

\begin{figure}[H]
\centering
\begin{tikzpicture}[
    node distance=1cm,
    class/.style={rectangle, rounded corners, draw, thick,
                  minimum width=5cm, minimum height=2cm,
                  text centered, text width=4.8cm, font=\small}
]

% ژنتری
\node[class, fill=victoriangreen!20] (gentry) {
    \textbf{\Large ژنتری (Gentry)}\\[0.2cm]
    زمین‌داران غیراشرافی\\
    صلح قضات محلی\\
    نمایندگان پارلمان\\
    فرهنگ و آموزش
};

% بورژوازی
\node[class, fill=parliamentgold!20, right=2cm of gentry] (bourgeoisie) {
    \textbf{\Large بورژوازی تجاری}\\[0.2cm]
    بازرگانان لندن\\
    شرکت‌های تجاری\\
    صادرات پشم\\
    مالی و بانکداری
};

% روابط
\node[below=2cm of gentry, xshift=3cm, text width=6cm, align=center] (relation) {
    \textcolor{royalblue}{\textbf{درآمیختگی:}}\\
    ازدواج، خرید زمین، سرمایه‌گذاری مشترک\\
    $\Downarrow$\\
    شکل‌گیری طبقه حاکم جدید
};

\draw[-{Stealth}, thick, dashed] (gentry) -- (relation);
\draw[-{Stealth}, thick, dashed] (bourgeoisie) -- (relation);

\end{tikzpicture}
\caption{ژنتری و بورژوازی: دو ستون طبقه حاکم جدید}
\end{figure}

%----------------------------------------------------------------------
\section{سلسله استوارت: تنش و بحران (۱۶۰۳-۱۶۴۲)}
%----------------------------------------------------------------------

\subsection{جیمز یکم و تنش با پارلمان (۱۶۰۳-۱۶۲۵)}

جیمز ششم اسکاتلند در ۱۶۰۳ به‌عنوان جیمز یکم به تاج انگلستان رسید. 
او با دیدگاهی متفاوت درباره سلطنت آمد: نظریه «حق الهی پادشاهان».

\begin{historicalquote}
«پادشاهان خدایان زمینی‌اند... آن‌ها می‌توانند زندگی و مرگ را مقرر کنند، 
در همه موارد بر همه اتباع خود قضاوت کنند، و تنها در برابر خدا پاسخگو 
باشند.»
\hfill --- جیمز یکم، \textit{قانون حقیقی سلطنت‌های آزاد} (۱۵۹۸)
\end{historicalquote}

این دیدگاه با سنت انگلیسی سلطنت مشروط در تضاد بود:

\begin{table}[H]
\centering
\caption{تضاد دیدگاه‌ها درباره سلطنت}
\renewcommand{\arraystretch}{1.4}
\begin{tabularx}{\textwidth}{>{\bfseries}p{3cm}XX}
\toprule
\textbf{مسئله} & \textbf{دیدگاه جیمز} & \textbf{دیدگاه پارلمان} \\
\midrule
\rowcolor{empirecream}
منبع قدرت & حق الهی & قانون و رضایت \\

مالیات & اختیار شاه & نیازمند رضایت پارلمان \\

\rowcolor{empirecream}
قانون‌گذاری & شاه بالاتر از قانون & شاه تابع قانون \\

دین & یکنواختی اجباری & آزادی وجدان (برای برخی) \\

\rowcolor{empirecream}
سیاست خارجی & اختیار سلطنتی & مشورت با پارلمان \\

\bottomrule
\end{tabularx}
\end{table}

\subsection{چارلز یکم و بحران (۱۶۲۵-۱۶۴۲)}

چارلز یکم تنش‌ها را تشدید کرد. او از ۱۶۲۹ تا ۱۶۴۰ بدون پارلمان حکومت 
کرد (دوره «استبداد یازده‌ساله»).

\subsubsection{مسائل کلیدی تنش}

\begin{figure}[H]
\centering
\begin{tikzpicture}[
    mindmap,
    grow cyclic,
    every node/.style={concept, circular drop shadow},
    concept color=imperialred!30,
    level 1/.append style={level distance=4cm, sibling angle=72},
    level 2/.append style={level distance=2.8cm, sibling angle=40},
    font=\small
]
\node[concept, font=\bfseries] {تنش‌های\\چارلز-پارلمان}
    child[concept color=royalblue!30] { node {مالی}
        child { node[font=\tiny] {Ship Money} }
        child { node[font=\tiny] {وام اجباری} }
        child { node[font=\tiny] {فروش انحصارات} }
    }
    child[concept color=tudorpurple!30] { node {دینی}
        child { node[font=\tiny] {آرمینیانیسم} }
        child { node[font=\tiny] {کلیسای لاود} }
        child { node[font=\tiny] {آزار پیوریتن‌ها} }
    }
    child[concept color=victoriangreen!30] { node {قانونی}
        child { node[font=\tiny] {حبس بدون محاکمه} }
        child { node[font=\tiny] {دادگاه‌های ویژه} }
    }
    child[concept color=parliamentgold!30] { node {سیاست خارجی}
        child { node[font=\tiny] {جنگ‌های ناموفق} }
        child { node[font=\tiny] {همدردی با کاتولیک‌ها} }
    }
    child[concept color=industrialgray!30] { node {شخصیتی}
        child { node[font=\tiny] {سرسختی} }
        child { node[font=\tiny] {بی‌اعتمادی} }
    };
\end{tikzpicture}
\caption{ریشه‌های چندگانه تنش بین چارلز یکم و پارلمان}
\end{figure}

\subsubsection{دادخواست حق (Petition of Right) ۱۶۲۸}

پارلمان ۱۶۲۸ در ازای تصویب مالیات، «دادخواست حق» را به چارلز تحمیل 
کرد. این سند به مگناکارتا استناد می‌کرد و موارد زیر را تأیید می‌کرد:

\begin{enumerate}[label=\textcolor{royalblue}{\arabic*.}]
    \item ممنوعیت مالیات بدون رضایت پارلمان
    \item ممنوعیت حبس بدون حکم قانونی
    \item ممنوعیت اسکان اجباری سربازان در خانه‌ها
    \item ممنوعیت حکومت نظامی در زمان صلح
\end{enumerate}

\subsection{شورش اسکاتلند و فراخوانی پارلمان بلند (۱۶۴۰)}

در ۱۶۳۷، چارلز تلاش کرد کتاب دعای انگلیکن را به کلیسای پرزبیتری 
اسکاتلند تحمیل کند. اسکاتلندی‌ها شورش کردند (جنگ‌های اسقف‌ها). چارلز 
برای تأمین هزینه جنگ مجبور به فراخوانی پارلمان شد.

\begin{figure}[H]
\centering
\begin{tikzpicture}[
    node distance=1.2cm,
    event/.style={rectangle, rounded corners, draw, thick,
                  minimum width=4.5cm, minimum height=1cm,
                  text centered, text width=4.3cm, font=\small},
    arrow/.style={-{Stealth[length=3mm]}, thick}
]

\node[event, fill=industrialgray!20] (bishops) {۱۶۳۷: تحمیل کتاب دعا به اسکاتلند};
\node[event, fill=imperialred!20, below=of bishops] (revolt) {۱۶۳۸: شورش اسکاتلندی‌ها};
\node[event, fill=parliamentgold!20, below=of revolt] (short) {۱۶۴۰ (آوریل): پارلمان کوتاه\\(سه هفته)};
\node[event, fill=royalblue!20, below=of short] (long) {۱۶۴۰ (نوامبر): پارلمان بلند\\(تا ۱۶۶۰)};
\node[event, fill=victoriangreen!20, below=of long] (reforms) {اصلاحات اجباری:\\- لغو دادگاه‌های ویژه\\- پارلمان سه‌ساله\\- عدم انحلال بدون رضایت};
\node[event, fill=imperialred!40, below=of reforms] (war) {۱۶۴۲: آغاز جنگ داخلی};

\draw[arrow] (bishops) -- (revolt);
\draw[arrow] (revolt) -- (short);
\draw[arrow] (short) -- (long);
\draw[arrow] (long) -- (reforms);
\draw[arrow] (reforms) -- (war);

\end{tikzpicture}
\caption{مسیر از شورش اسکاتلند تا جنگ داخلی}
\end{figure}

%----------------------------------------------------------------------
\section{جنگ داخلی، جمهوری و بازگشت (۱۶۴۲-۱۶۸۸)}
%----------------------------------------------------------------------

\subsection{جنگ داخلی (۱۶۴۲-۱۶۴۹)}

جنگ داخلی انگلستان یکی از خونین‌ترین درگیری‌های تاریخ این کشور بود. 
حدود ۱۸۰,۰۰۰ نفر (از جمعیت حدود ۵ میلیونی) کشته شدند.

\subsubsection{صف‌بندی طبقاتی}

\begin{figure}[H]
\centering
\begin{tikzpicture}[
    node distance=0.5cm,
    side/.style={rectangle, rounded corners, draw, thick,
                 minimum width=6cm, minimum height=4cm,
                 text centered, font=\small}
]

% طرف شاه
\node[side, fill=royalblue!15] (royalist) {
    \textbf{\Large سلطنت‌طلبان (Cavaliers)}\\[0.3cm]
    \textcolor{royalblue}{$\blacksquare$} بیشتر اشرافیت\\
    \textcolor{royalblue}{$\blacksquare$} کلیسای انگلیکن\\
    \textcolor{royalblue}{$\blacksquare$} شمال و غرب\\
    \textcolor{royalblue}{$\blacksquare$} کاتولیک‌ها\\
    \textcolor{royalblue}{$\blacksquare$} برخی ژنتری محافظه‌کار
};

% طرف پارلمان
\node[side, fill=victoriangreen!15, right=2cm of royalist] (parliament) {
    \textbf{\Large پارلمانی‌ها (Roundheads)}\\[0.3cm]
    \textcolor{victoriangreen}{$\blacksquare$} بیشتر ژنتری\\
    \textcolor{victoriangreen}{$\blacksquare$} بورژوازی تجاری لندن\\
    \textcolor{victoriangreen}{$\blacksquare$} پیوریتن‌ها\\
    \textcolor{victoriangreen}{$\blacksquare$} جنوب و شرق\\
    \textcolor{victoriangreen}{$\blacksquare$} نیروی دریایی
};

% عنوان
\node[above=1cm of royalist, xshift=4cm, font=\Large\bfseries, color=imperialred] 
    {صف‌بندی جنگ داخلی (۱۶۴۲-۱۶۴۹)};

% vs
\node[font=\Huge\bfseries, color=imperialred] at ($(royalist)!0.5!(parliament)$) {VS};

\end{tikzpicture}
\caption{صف‌بندی اجتماعی-جغرافیایی جنگ داخلی}
\end{figure}

\begin{warningbox}
\textbf{هشدار تحلیلی:} صف‌بندی جنگ داخلی را نباید ساده‌سازی کرد. این 
جنگ نه صرفاً طبقاتی بود، نه صرفاً دینی، نه صرفاً منطقه‌ای. خانواده‌ها 
تقسیم شدند، ژنتری هر دو طرف بود، و انگیزه‌ها متنوع بود. با این حال، 
الگوی کلی نشان می‌دهد که مناطق تجاری‌تر و پروتستان‌تر به سمت پارلمان 
گرایش داشتند.
\end{warningbox}

\subsubsection{پیروزی پارلمان و اعدام شاه}

ارتش نوین (New Model Army) به فرماندهی الیور کرامول، سلطنت‌طلبان را 
شکست داد. در ۳۰ ژانویه ۱۶۴۹، چارلز یکم در وایت‌هال گردن زده شد.

\begin{historicalquote}
«این اقدام بی‌سابقه بود: اعدام قانونی یک پادشاه توسط اتباعش. این امر 
اصل «شاه تابع قانون است» را به نهایت منطقی‌اش رساند.»
\hfill --- \textcite{worden2009english}
\end{historicalquote}

\subsection{جمهوری و حکومت کرامول (۱۶۴۹-۱۶۶۰)}

\subsubsection{آزمایش جمهوری}

\begin{figure}[H]
\centering
\begin{tikzpicture}[
    node distance=1.5cm,
    period/.style={rectangle, rounded corners, draw, thick,
                   minimum width=5cm, minimum height=1.3cm,
                   text centered, text width=4.8cm, font=\small},
    arrow/.style={-{Stealth[length=3mm]}, thick}
]

\node[period, fill=victoriangreen!20] (commonwealth) 
    {۱۶۴۹-۱۶۵۳:\\مشترک‌المنافع\\(Commonwealth)\\حکومت پارلمانی};

\node[period, fill=parliamentgold!20, below=of commonwealth] (protectorate) 
    {۱۶۵۳-۱۶۵۸:\\حامی‌سالاری\\(Protectorate)\\دیکتاتوری کرامول};

\node[period, fill=industrialgray!20, below=of protectorate] (richard) 
    {۱۶۵۸-۱۶۵۹:\\ریچارد کرامول\\(ضعیف و ناتوان)};

\node[period, fill=royalblue!20, below=of richard] (restoration) 
    {۱۶۶۰:\\بازگشت سلطنت\\چارلز دوم};

\draw[arrow] (commonwealth) -- (protectorate) 
    node[midway, right, font=\tiny, text width=2.5cm] {بحران مشروعیت، شورش‌ها};
\draw[arrow] (protectorate) -- (richard) 
    node[midway, right, font=\tiny] {مرگ الیور};
\draw[arrow] (richard) -- (restoration) 
    node[midway, right, font=\tiny, text width=2.5cm] {فروپاشی، مذاکره با چارلز};

\end{tikzpicture}
\caption{مراحل جمهوری و سقوط آن}
\end{figure}

\subsubsection{چرا جمهوری شکست خورد؟}

\begin{table}[H]
\centering
\caption{دلایل شکست جمهوری}
\renewcommand{\arraystretch}{1.4}
\begin{tabularx}{\textwidth}{>{\bfseries\color{imperialred}}p{3cm}X}
\toprule
\textbf{عامل} & \textbf{توضیح} \\
\midrule
\rowcolor{empirecream}
فقدان مشروعیت & نه سنت سلطنتی داشت، نه رضایت مردمی گسترده \\

وابستگی به ارتش & قدرت از لوله تفنگ، نه از رضایت \\

\rowcolor{empirecream}
رادیکالیسم & گروه‌های رادیکال (Levellers, Diggers) ژنتری را ترساندند \\

پیوریتنیسم & محدودیت‌های اخلاقی (ممنوعیت تئاتر، جشن‌ها) نامحبوب بود \\

\rowcolor{empirecream}
جانشینی & بدون مکانیسم روشن انتقال قدرت \\

\bottomrule
\end{tabularx}
\end{table}

\begin{policybox}[درس تاریخی جمهوری کرامول]
تجربه جمهوری نشان داد که:
\begin{enumerate}
    \item \textbf{انقلاب یک چیز است، نهادسازی چیز دیگر:} سرنگونی شاه 
    آسان‌تر از ساختن نظام جایگزین بود.
    
    \item \textbf{ارتش پایه مشروعیت نیست:} حکومت مبتنی بر زور، بدون 
    رضایت طبقات حاکم، پایدار نمی‌ماند.
    
    \item \textbf{میانه‌روی لازم است:} رادیکالیسم پیوریتنی طبقه متوسط 
    (ژنتری و بورژوازی) را از انقلاب رماند.
\end{enumerate}
این درس‌ها در ۱۶۸۸ به کار گرفته شد.
\end{policybox}

\subsection{بازگشت سلطنت و بحران جدید (۱۶۶۰-۱۶۸۸)}

\subsubsection{چارلز دوم: سازش ناپایدار}

چارلز دوم با «اعلامیه بردا» به انگلستان بازگشت و قول عفو عمومی، 
آزادی مذهبی، و احترام به مالکیت‌ها داد. اما تنش‌ها ادامه یافت:

\begin{table}[H]
\centering
\caption{تنش‌های دوره چارلز دوم}
\renewcommand{\arraystrength}{1.3}
\begin{tabularx}{\textwidth}{>{\bfseries}lX}
\toprule
\textbf{مسئله} & \textbf{تنش} \\
\midrule
دین & همدردی پنهان چارلز با کاتولیسیسم \\
جانشینی & جیمز (برادر و ولیعهد) کاتولیک بود \\
مالی & وابستگی به لویی چهاردهم فرانسه \\
پارلمان & تلاش برای حذف جیمز (بحران طرد) \\
\bottomrule
\end{tabularx}
\end{table}

\subsubsection{شکل‌گیری احزاب: ویگ‌ها و توری‌ها}

در جریان «بحران طرد» (۱۶۷۹-۱۶۸۱)، دو جناح سیاسی شکل گرفت که بعداً 
به احزاب تبدیل شدند:

\begin{figure}[H]
\centering
\begin{tikzpicture}[
    party/.style={rectangle, rounded corners=10pt, draw, very thick,
                  minimum width=5.5cm, minimum height=4cm,
                  text centered, text width=5.3cm, font=\small}
]

\node[party, fill=parliamentgold!20, draw=parliamentgold!80!black] (whig) {
    \textbf{\Large ویگ‌ها (Whigs)}\\[0.3cm]
    موضع: طرد جیمز کاتولیک\\[0.2cm]
    پایگاه: بورژوازی، ناهمرنگان\\[0.2cm]
    اصل: سلطنت مشروط، پروتستانتیسم\\[0.2cm]
    آینده: حزب لیبرال
};

\node[party, fill=royalblue!20, draw=royalblue!80!black, right=2cm of whig] (tory) {
    \textbf{\Large توری‌ها (Tories)}\\[0.3cm]
    موضع: حق جانشینی موروثی\\[0.2cm]
    پایگاه: ژنتری روستایی، انگلیکن\\[0.2cm]
    اصل: وفاداری به تاج، کلیسا\\[0.2cm]
    آینده: حزب محافظه‌کار
};

\end{tikzpicture}
\caption{شکل‌گیری دو جناح سیاسی اصلی انگلستان}
\end{figure}

\subsubsection{جیمز دوم و سقوط (۱۶۸۵-۱۶۸۸)}

جیمز دوم در ۱۶۸۵ به تاج رسید و بلافاصله تلاش کرد کاتولیسیسم را 
احیا کند:

\begin{itemize}
    \item انتصاب کاتولیک‌ها به مناصب (برخلاف قانون)
    \item ارتش دائمی با افسران کاتولیک
    \item اعلامیه معافیت (تعلیق قوانین ضد کاتولیک)
    \item تولد پسر (۱۶۸۸) -- تضمین سلسله کاتولیک
\end{itemize}

%----------------------------------------------------------------------
\section{انقلاب شکوهمند و تسویه ۱۶۸۸-۱۶۸۹}
%----------------------------------------------------------------------

\subsection{دعوت از ویلیام اورانژ}

در ۱۶۸۸، هفت تن از رهبران ویگ و توری نامه‌ای به ویلیام اورانژ (شوهر 
مری، دختر پروتستان جیمز) نوشتند و از او دعوت کردند به انگلستان بیاید.

\begin{figure}[H]
\centering
\begin{tikzpicture}[
    node distance=1.3cm,
    event/.style={rectangle, rounded corners, draw, thick,
                  minimum width=5cm, minimum height=1.2cm,
                  text centered, text width=4.8cm, font=\small},
    arrow/.style={-{Stealth[length=3mm]}, thick}
]

\node[event, fill=imperialred!20] (birth) 
    {ژوئن ۱۶۸۸:\\تولد ولیعهد کاتولیک};

\node[event, fill=royalblue!20, below=of birth] (letter) 
    {ژوئن ۱۶۸۸:\\نامه دعوت به ویلیام};

\node[event, fill=victoriangreen!20, below=of letter] (invasion) 
    {نوامبر ۱۶۸۸:\\فرود ویلیام با ارتش\\(بدون مقاومت جدی)};

\node[event, fill=parliamentgold!20, below=of invasion] (flight) 
    {دسامبر ۱۶۸۸:\\فرار جیمز به فرانسه};

\node[event, fill=tudorpurple!20, below=of flight] (offer) 
    {فوریه ۱۶۸۹:\\پیشنهاد تاج به ویلیام و مری\\(مشترک)};

\node[event, fill=victoriangreen!40, below=of offer] (settlement) 
    {۱۶۸۹:\\لایحه حقوق\\(Bill of Rights)};

\draw[arrow] (birth) -- (letter);
\draw[arrow] (letter) -- (invasion);
\draw[arrow] (invasion) -- (flight);
\draw[arrow] (flight) -- (offer);
\draw[arrow] (offer) -- (settlement);

\end{tikzpicture}
\caption{توالی رویدادهای انقلاب شکوهمند}
\end{figure}

\subsection{لایحه حقوق ۱۶۸۹}

لایحه حقوق (Bill of Rights) سند بنیادین تسویه ۱۶۸۸-۱۶۸۹ بود که 
اصول زیر را تثبیت کرد:

\begin{table}[H]
\centering
\caption{مفاد کلیدی لایحه حقوق ۱۶۸۹}
\renewcommand{\arraystretch}{1.4}
\begin{tabularx}{\textwidth}{>{\bfseries}c>{\itshape}Xp{4cm}}
\toprule
\textbf{شماره} & \textbf{محتوا} & \textbf{اهمیت} \\
\midrule
\rowcolor{empirecream}
۱ & تعلیق قوانین بدون رضایت پارلمان غیرقانونی است & سیادت قانون \\

۲ & وضع مالیات بدون رضایت پارلمان غیرقانونی است & کنترل مالی \\

\rowcolor{empirecream}
۴ & نگهداری ارتش دائمی در زمان صلح بدون رضایت پارلمان غیرقانونی است & کنترل نظامی \\

۵ & حق دادخواهی به شاه & دسترسی به عدالت \\

\rowcolor{empirecream}
۸ & انتخابات پارلمان باید آزاد باشد & دموکراسی \\

۹ & آزادی بیان در پارلمان & مصونیت پارلمانی \\

\rowcolor{empirecream}
۱۰ & ممنوعیت وثیقه و جریمه بیش از حد & حقوق متهم \\

\bottomrule
\end{tabularx}
\end{table}

\subsection{سایر قوانین تسویه}

\begin{table}[H]
\centering
\caption{قوانین تکمیلی تسویه ۱۶۸۸-۱۷۰۱}
\renewcommand{\arraystretch}{1.3}
\begin{tabularx}{\textwidth}{>{\bfseries}p{3.5cm}cp{6cm}}
\toprule
\textbf{قانون} & \textbf{سال} & \textbf{محتوا} \\
\midrule
قانون تساهل & ۱۶۸۹ & آزادی عبادت برای پروتستان‌های ناهمرنگ \\
قانون سه‌ساله & ۱۶۹۴ & پارلمان حداقل هر سه سال باید فراخوانده شود \\
قانون جانشینی & ۱۷۰۱ & تاج به پروتستان‌ها محدود شد \\
اتحاد با اسکاتلند & ۱۷۰۷ & تشکیل «بریتانیای کبیر» \\
\bottomrule
\end{tabularx}
\end{table}

\subsection{اهمیت انقلاب شکوهمند}

\begin{figure}[H]
\centering
\begin{tikzpicture}[
    node distance=0.8cm,
    box/.style={rectangle, rounded corners, draw=royalblue, thick,
                minimum width=5.5cm, minimum height=1.3cm,
                text centered, text width=5.3cm, font=\small}
]

\node[box, fill=royalblue!15] (sovereignty) 
    {\textbf{۱. سیادت پارلمان}\\شاه تابع قانون مصوب پارلمان};

\node[box, fill=victoriangreen!15, below=of sovereignty] (finance) 
    {\textbf{۲. کنترل مالی}\\بدون رضایت، نه مالیات نه ارتش};

\node[box, fill=parliamentgold!15, below=of finance] (property) 
    {\textbf{۳. امنیت مالکیت}\\دولت نمی‌تواند مصادره کند};

\node[box, fill=tudorpurple!15, below=of property] (religion) 
    {\textbf{۴. پروتستانتیسم}\\تاج به پروتستان‌ها محدود شد};

\node[box, fill=imperialred!15, below=of religion] (contract) 
    {\textbf{۵. سلطنت قراردادی}\\تاج هدیه پارلمان، نه حق الهی};

% فلش به پیامد
\node[rectangle, rounded corners=10pt, draw=victoriangreen, very thick,
      fill=victoriangreen!30, minimum width=6cm, minimum height=1.5cm,
      right=3cm of finance, text centered, font=\bfseries] (outcome) 
    {پیامد:\\زمینه‌سازی انقلاب صنعتی\\و امپراتوری جهانی};

\draw[-{Stealth}, very thick, victoriangreen] (finance.east) -- (outcome.west);

\end{tikzpicture}
\caption{پیامدهای کلیدی انقلاب شکوهمند}
\end{figure}

\begin{historicalquote}
«انقلاب شکوهمند یک انقلاب محافظه‌کارانه بود. هدف آن نه دگرگونی جامعه، 
بلکه حفظ نظم موجود در برابر تهدید استبداد کاتولیک بود. اما همین 
محافظه‌کاری آن را پایدار کرد.»
\hfill --- \textcite{pincus20091688}
\end{historicalquote}

%----------------------------------------------------------------------
\section{تحلیل طبقاتی دوره تودور-استوارت}
%----------------------------------------------------------------------

\subsection{تحول ساختار طبقاتی}

\begin{figure}[H]
\centering
\begin{tikzpicture}
\begin{axis}[
    width=14cm,
    height=9cm,
    xlabel={سال},
    ylabel={درصد قدرت سیاسی (تخمینی)},
    xmin=1480, xmax=1720,
    ymin=0, ymax=60,
    xtick={1500, 1550, 1600, 1650, 1700},
    legend pos=outer north east,
    grid=major,
]

% تاج
\addplot[very thick, royalblue, mark=square*] coordinates {
    (1485, 30) (1509, 35) (1547, 45) (1558, 40) (1603, 35)
    (1625, 30) (1640, 25) (1649, 0) (1660, 25) (1688, 20) (1714, 15)
};
\addlegendentry{تاج}

% اشرافیت
\addplot[very thick, tudorpurple, mark=triangle*] coordinates {
    (1485, 25) (1509, 20) (1547, 15) (1558, 15) (1603, 15)
    (1625, 15) (1640, 15) (1649, 10) (1660, 15) (1688, 15) (1714, 20)
};
\addlegendentry{اشرافیت}

% ژنتری
\addplot[very thick, victoriangreen, mark=*] coordinates {
    (1485, 20) (1509, 25) (1547, 25) (1558, 30) (1603, 35)
    (1625, 40) (1640, 45) (1649, 50) (1660, 45) (1688, 45) (1714, 45)
};
\addlegendentry{ژنتری}

% بورژوازی
\addplot[very thick, parliamentgold, mark=diamond*] coordinates {
    (1485, 10) (1509, 12) (1547, 15) (1558, 18) (1603, 20)
    (1625, 22) (1640, 25) (1649, 30) (1660, 28) (1688, 30) (1714, 35)
};
\addlegendentry{بورژوازی تجاری}

% خطوط عمودی رویدادها
\draw[dashed, imperialred] (axis cs:1534,0) -- (axis cs:1534,60);
\node[imperialred, font=\tiny, rotate=90, anchor=south] at (axis cs:1534,30) {اصلاحات};

\draw[dashed, imperialred] (axis cs:1642,0) -- (axis cs:1642,60);
\node[imperialred, font=\tiny, rotate=90, anchor=south] at (axis cs:1642,30) {جنگ داخلی};

\draw[dashed, imperialred] (axis cs:1688,0) -- (axis cs:1688,60);
\node[imperialred, font=\tiny, rotate=90, anchor=south] at (axis cs:1688,30) {انقلاب شکوهمند};

\end{axis}
\end{tikzpicture}
\caption{تحول قدرت نسبی طبقات در دوره تودور-استوارت}
\end{figure}

\subsection{مقایسه با فرانسه: چرا مسیرها متفاوت شد؟}

\begin{table}[H]
\centering
\caption{مقایسه مسیرهای انگلستان و فرانسه در قرن هفدهم}
\renewcommand{\arraystretch}{1.4}
\begin{tabularx}{\textwidth}{>{\bfseries}p{3cm}XX}
\toprule
\textbf{عامل} & \textbf{انگلستان} & \textbf{فرانسه} \\
\midrule
\rowcolor{empirecream}
پارلمان/مجلس & قوی، مستمر، حق مالیات & استیت‌جنرال ضعیف، تعطیل از ۱۶۱۴ \\

اشرافیت & منعطف، درآمیخته با ژنتری و تجارت & بسته، امتیازات فئودالی \\

\rowcolor{empirecream}
مالیات & نیازمند رضایت، مذاکره‌ای & taille دائمی، بدون رضایت \\

کلیسا & ملی، تحت کنترل & کاتولیک، قدرتمند، مستقل \\

\rowcolor{empirecream}
ارتش & کوچک، وابسته به مالیات پارلمان & بزرگ، دائمی \\

جغرافیا & جزیره‌ای، نیازی به ارتش زمینی بزرگ & قاره‌ای، مرزهای آسیب‌پذیر \\

\rowcolor{empirecream}
نتیجه ۱۷۰۰ & سلطنت مشروطه & سلطنت مطلقه لویی چهاردهم \\

\bottomrule
\end{tabularx}
\end{table}

\begin{figure}[H]
\centering
\begin{tikzpicture}[
    node distance=2cm,
    country/.style={rectangle, rounded corners=10pt, draw, very thick,
                    minimum width=5cm, minimum height=2cm,
                    text centered, font=\bfseries\large}
]

% انگلستان
\node[country, fill=victoriangreen!20, draw=victoriangreen] (england) 
    {انگلستان ۱۶۸۸};

% فرانسه
\node[country, fill=imperialred!20, draw=imperialred, right=4cm of england] (france) 
    {فرانسه ۱۶۸۹};

% ویژگی‌های انگلستان
\node[below=1cm of england, text width=5cm, align=center, font=\small] (eng_feat) {
    سلطنت مشروطه\\
    پارلمان قدرتمند\\
    حقوق مالکیت امن\\
    بورژوازی در قدرت\\
    $\downarrow$\\
    \textcolor{victoriangreen}{\textbf{انقلاب صنعتی}}
};

% ویژگی‌های فرانسه
\node[below=1cm of france, text width=5cm, align=center, font=\small] (fra_feat) {
    سلطنت مطلقه\\
    استیت‌جنرال تعطیل\\
    مالکیت ناامن\\
    اشرافیت امتیازدار\\
    $\downarrow$\\
    \textcolor{imperialred}{\textbf{انقلاب خونین ۱۷۸۹}}
};

\end{tikzpicture}
\caption{مقایسه مسیرهای متفاوت انگلستان و فرانسه}
\end{figure}

%----------------------------------------------------------------------
\section{جمع‌بندی فصل: میراث دوره تودور-استوارت}
%----------------------------------------------------------------------

\subsection{تحولات کلیدی}

\begin{figure}[H]
\centering
\begin{tikzpicture}[
    transform/.style={rectangle, rounded corners, draw=royalblue, thick,
                      minimum width=5cm, minimum height=1.5cm,
                      text centered, text width=4.8cm, font=\small,
                      fill=royalblue!10}
]

\node[transform] (t1) at (0,0) {\textbf{از:} سلطنت فئودالی\\\textbf{به:} سلطنت مشروطه};
\node[transform] (t2) at (6,0) {\textbf{از:} کلیسای کاتولیک\\\textbf{به:} کلیسای ملی};
\node[transform] (t3) at (0,-2.5) {\textbf{از:} اشرافیت فئودالی\\\textbf{به:} الیگارشی ژنتری-بورژوا};
\node[transform] (t4) at (6,-2.5) {\textbf{از:} اقتصاد کشاورزی\\\textbf{به:} سرمایه‌داری تجاری};
\node[transform] (t5) at (3,-5) {\textbf{از:} دولت پراکنده\\\textbf{به:} دولت-ملت متمرکز};

\end{tikzpicture}
\caption{تحولات بنیادین دوره ۱۴۸۵-۱۷۱۴}
\end{figure}

\subsection{عوامل کلیدی موفقیت}

\begin{policybox}[چرا انگلستان موفق شد؟]
\begin{enumerate}[label=\textcolor{victoriangreen}{\arabic*.}]
    \item \textbf{نهادهای میانجی:} پارلمان میدانی برای مذاکره فراهم کرد، 
    نه میدان جنگ.
    
    \item \textbf{انعطاف‌پذیری نخبگان:} ژنتری و بورژوازی درآمیختند؛ 
    اشرافیت به تجارت روی آورد.
    
    \item \textbf{ایدئولوژی پروتستانتیسم:} مشروعیت دینی برای مقاومت در 
    برابر استبداد.
    
    \item \textbf{جغرافیای جزیره‌ای:} نیاز کمتر به ارتش دائمی = قدرت 
    کمتر تاج.
    
    \item \textbf{توازن قوا:} هیچ طرفی قدرت کافی برای غلبه کامل نداشت، 
    پس سازش لازم شد.
    
    \item \textbf{درس‌آموزی از شکست:} تجربه جنگ داخلی و جمهوری نشان داد 
    که رادیکالیسم ناپایدار است.
\end{enumerate}
\end{policybox}

\subsection{مسیر وابستگی: از ۱۶۸۸ به بعد}

تسویه ۱۶۸۸-۱۶۸۹ پنج \textbf{مسیر وابستگی} برای تاریخ بعدی انگلستان 
ایجاد کرد:

\begin{table}[H]
\centering
\caption{میراث‌های بلندمدت انقلاب شکوهمند}
\renewcommand{\arraystretch}{1.4}
\begin{tabularx}{\textwidth}{>{\bfseries\color{royalblue}}p{3cm}Xp{4cm}}
\toprule
\textbf{میراث} & \textbf{توضیح} & \textbf{پیامد بلندمدت} \\
\midrule
\rowcolor{empirecream}
سیادت پارلمان & پارلمان، نه شاه، منبع نهایی قانون & دموکراتیزاسیون تدریجی \\

امنیت مالکیت & دولت نمی‌تواند خودسرانه مصادره کند & انقلاب صنعتی \\

\rowcolor{empirecream}
دولت مالی مدرن & بانک انگلستان (۱۶۹۴)، بدهی ملی & قدرت نظامی-اقتصادی \\

آزادی محدود & آزادی مطبوعات، تساهل دینی (ناقص) & جامعه مدنی \\

\rowcolor{empirecream}
پروتستانتیسم & تاج پروتستان، هویت ضد کاتولیک & امپراتوری «رسالت‌مند» \\

\bottomrule
\end{tabularx}
\end{table}

\begin{warningbox}
\textbf{محدودیت‌های تسویه ۱۶۸۸:}
\begin{itemize}
    \item دموکراسی نبود: فقط ۵٪ حق رأی داشتند
    \item حقوق کامل برای همه نبود: کاتولیک‌ها و ناهمرنگان محروم بودند
    \item برابری نبود: الیگارشی ژنتری-بورژوا حاکم شد
    \item استعمار تقویت شد: آزادی در داخل، استثمار در خارج
\end{itemize}
تسویه ۱۶۸۸ یک \textbf{انقلاب بورژوایی-اشرافی} بود، نه یک انقلاب 
دموکراتیک. اما همین چارچوب نهادی امکان اصلاحات بعدی را فراهم کرد.
\end{warningbox}

%----------------------------------------------------------------------
% ارجاعات فصل سوم
%----------------------------------------------------------------------

\section*{ارجاعات فصل سوم}

\begin{enumerate}[label={[\arabic*]}]
    \item Chrimes, S.B. (1972). \textit{Henry VII}. London: Eyre Methuen.
    
    \item Collinson, P. (1967). \textit{The Elizabethan Puritan Movement}. 
    London: Jonathan Cape.
    
    \item Elton, G.R. (1953). \textit{The Tudor Revolution in Government}. 
    Cambridge: Cambridge University Press.
    
    \item Hill, C. (1961). \textit{The Century of Revolution, 1603-1714}. 
    Edinburgh: Thomas Nelson.
    
    \item Kishlansky, M.A. (1996). \textit{A Monarchy Transformed: Britain 
    1603-1714}. London: Penguin.
    
    \item Pincus, S. (2009). \textit{1688: The First Modern Revolution}. 
    New Haven: Yale University Press.
    
    \item Russell, C. (1990). \textit{The Causes of the English Civil War}. 
    Oxford: Clarendon Press.
    
    \item Stone, L. (1965). \textit{The Crisis of the Aristocracy, 1558-1641}. 
    Oxford: Clarendon Press.
    
    \item Worden, B. (2009). \textit{The English Civil Wars, 1640-1660}. 
    London: Weidenfeld \& Nicolson.
    
    \item Wrightson, K. (2000). \textit{Earthly Necessities: Economic Lives 
    in Early Modern Britain}. New Haven: Yale University Press.
\end{enumerate}

%%%%%%%%%%%%%%%%%%%%%%%%%%%%%%%%%%%%%%%%%%%%%%%%%%%%%%%%%%%%%%%%%%%%%%%
% فصل چهارم: قرن هجدهم و انقلاب صنعتی (۱۷۱۴-۱۸۳۲)
%%%%%%%%%%%%%%%%%%%%%%%%%%%%%%%%%%%%%%%%%%%%%%%%%%%%%%%%%%%%%%%%%%%%%%%

\chapter{قرن هجدهم و انقلاب صنعتی (۱۷۱۴-۱۸۳۲)}

\begin{keybox}[خلاصه فصل]
این فصل به تحلیل دوره‌ای می‌پردازد که انگلستان را به \textbf{نخستین 
کشور صنعتی جهان} و \textbf{قدرت جهانی مسلط} تبدیل کرد. سه تحول بزرگ 
رخ داد: \textbf{انقلاب کشاورزی} (افزایش بهره‌وری و آزادسازی نیروی کار)، 
\textbf{انقلاب صنعتی} (صنعتی‌شدن تولید)، و \textbf{انقلاب مالی} (بانک 
مرکزی، بازارهای سرمایه). این تحولات در چارچوب نهادی خاص ممکن شد: 
\textbf{الیگارشی ویگ} که ثبات سیاسی و امنیت مالکیت را تضمین می‌کرد، 
و \textbf{امپراتوری تجاری} که بازار و مواد خام فراهم می‌آورد.
\end{keybox}

%----------------------------------------------------------------------
\section{نظم جدید: الیگارشی ویگ (۱۷۱۴-۱۷۶۰)}
%----------------------------------------------------------------------

\subsection{تثبیت سلسله هانوفر}

در ۱۷۱۴، جرج یکم از خاندان هانوفر (آلمان) بر اساس قانون جانشینی ۱۷۰۱ 
به تاج بریتانیا رسید. او انگلیسی نمی‌دانست و به امور بریتانیا علاقه 
چندانی نداشت.

\begin{historicalquote}
«جرج یکم هرگز انگلیسی یاد نگرفت و هرگز بریتانیا را دوست نداشت. اما 
همین بی‌علاقگی او به پارلمان و وزرا امکان داد قدرت واقعی را در دست 
بگیرند.»
\hfill --- \textcite{plumb1967growth}
\end{historicalquote}

\subsubsection{ظهور نخست‌وزیری}

\begin{figure}[H]
\centering
\begin{tikzpicture}[
    node distance=1.5cm,
    box/.style={rectangle, rounded corners, draw, thick,
                minimum width=4.5cm, minimum height=1.3cm,
                text centered, text width=4.3cm, font=\small}
]

\node[box, fill=royalblue!20] (king) {شاه\\(جرج یکم، جرج دوم)\\نقش تشریفاتی فزاینده};

\node[box, fill=victoriangreen!20, below left=1.5cm and 0cm of king] (pm) 
    {نخست‌وزیر\\(رابرت والپول ۱۷۲۱-۱۷۴۲)\\قدرت اجرایی واقعی};

\node[box, fill=parliamentgold!20, below right=1.5cm and 0cm of king] (cabinet) 
    {کابینه\\وزرای هماهنگ\\مسئول در برابر پارلمان};

\node[box, fill=tudorpurple!20, below=3cm of king] (parliament) 
    {پارلمان\\منبع مشروعیت\\کنترل مالی};

\draw[-{Stealth}, thick, dashed] (king) -- (pm) node[midway, left, font=\tiny] {واگذاری عملی};
\draw[-{Stealth}, thick] (king) -- (cabinet);
\draw[-{Stealth}, thick] (pm) -- (cabinet);
\draw[{Stealth}-{Stealth}, thick] (cabinet) -- (parliament) 
    node[midway, right, font=\tiny] {پاسخگویی};
\draw[-{Stealth}, thick, dashed] (parliament) to[bend right=30] (pm);

\end{tikzpicture}
\caption{شکل‌گیری نظام کابینه‌ای و نخست‌وزیری}
\end{figure}

\subsection{ویژگی‌های الیگارشی ویگ}

«الیگارشی ویگ» اصطلاحی است که مورخان برای توصیف نظام حکمرانی 
۱۷۱۴-۱۷۶۰ به کار می‌برند:

\begin{table}[H]
\centering
\caption{ویژگی‌های الیگارشی ویگ}
\renewcommand{\arraystretch}{1.4}
\begin{tabularx}{\textwidth}{>{\bfseries\color{royalblue}}p{3cm}X}
\toprule
\textbf{ویژگی} & \textbf{توضیح} \\
\midrule
\rowcolor{empirecream}
سلطه یک حزب & ویگ‌ها از ۱۷۱۴ تا ۱۷۶۰ بدون رقیب جدی حکومت کردند \\

پایگاه محدود & فقط حدود ۵٪ مردان بالغ حق رأی داشتند \\

\rowcolor{empirecream}
فساد سیستمی & خرید رأی، انتصابات، «حوزه‌های جیبی» (pocket boroughs) \\

ائتلاف طبقاتی & ژنتری روستایی + بورژوازی تجاری + بخشی از اشرافیت \\

\rowcolor{empirecream}
ثبات سیاسی & بدون انقلاب، کودتا، یا جنگ داخلی \\

آزادی محدود & آزادی مطبوعات، هبیث کورپوس، اما سرکوب رادیکال‌ها \\

\bottomrule
\end{tabularx}
\end{table}

\begin{figure}[H]
\centering
\begin{tikzpicture}
\pie[
    text=legend,
    radius=3.5,
    color={royalblue!70, tudorpurple!70, victoriangreen!70, 
           parliamentgold!70, industrialgray!70}
]{
    30/اشرافیت بزرگ (۳۰٪),
    35/ژنتری (۳۵٪),
    20/بورژوازی تجاری (۲۰٪),
    10/حرفه‌ای‌ها (۱۰٪),
    5/سایر (۵٪)
}
\node[below=4cm, font=\bfseries] {ترکیب طبقاتی مجلس عوام (حدود ۱۷۵۰)};
\end{tikzpicture}
\caption{توزیع طبقاتی نمایندگان مجلس عوام در اواسط قرن ۱۸}
\end{figure}

\begin{policybox}[پارادوکس الیگارشی ویگ]
الیگارشی ویگ یک پارادوکس بود: حکومتی فاسد و غیردموکراتیک که همزمان:
\begin{itemize}
    \item ثبات سیاسی را تضمین کرد (لازم برای سرمایه‌گذاری)
    \item حقوق مالکیت را حفظ کرد (لازم برای انباشت)
    \item آزادی نسبی را تأمین کرد (لازم برای نوآوری)
    \item از جنگ‌های بزرگ زمینی اجتناب کرد (لازم برای تجارت)
\end{itemize}
این نظام غیرعادلانه، زمینه‌ساز انقلاب صنعتی شد. دموکراسی بعداً آمد.
\end{policybox}

%----------------------------------------------------------------------
\section{انقلاب کشاورزی (۱۷۰۰-۱۸۵۰)}
%----------------------------------------------------------------------

\subsection{تحولات کشاورزی}

انقلاب صنعتی بدون انقلاب کشاورزی ممکن نبود. تحولات کشاورزی نیروی کار 
آزاد، مواد غذایی ارزان، و سرمایه برای صنعت فراهم کرد.

\begin{figure}[H]
\centering
\begin{tikzpicture}[
    node distance=1cm,
    innov/.style={rectangle, rounded corners, draw=victoriangreen, thick,
                  fill=victoriangreen!15, minimum width=3.5cm, 
                  minimum height=1cm, text centered, text width=3.3cm, 
                  font=\small}
]

\node[innov] (enclosure) {محصورسازی\\(Enclosure)};
\node[innov, right=0.8cm of enclosure] (rotation) {تناوب چهارساله\\(Four-field rotation)};
\node[innov, right=0.8cm of rotation] (breeding) {اصلاح نژاد\\دام};
\node[innov, right=0.8cm of breeding] (tools) {ابزار جدید\\(خیش آهنی)};

\node[below=2cm of rotation, xshift=1.5cm, rectangle, rounded corners=10pt, 
      draw=parliamentgold, very thick, fill=parliamentgold!20,
      minimum width=8cm, minimum height=2cm, text centered, text width=7.5cm] (output) {
    \textbf{نتیجه:}\\
    افزایش ۳۰۰٪ بهره‌وری کشاورزی (۱۷۰۰-۱۸۵۰)\\
    جمعیت ۳ برابر، اما غذا کافی
};

\draw[-{Stealth}, thick] (enclosure) -- (output);
\draw[-{Stealth}, thick] (rotation) -- (output);
\draw[-{Stealth}, thick] (breeding) -- (output);
\draw[-{Stealth}, thick] (tools) -- (output);

\end{tikzpicture}
\caption{نوآوری‌های کلیدی انقلاب کشاورزی}
\end{figure}

\subsection{محصورسازی (Enclosure): تحول طبقاتی روستا}

محصورسازی فرآیندی بود که طی آن زمین‌های مشاع روستایی به مالکیت 
خصوصی تبدیل شد. این فرآیند از قرن ۱۶ آغاز شد اما در قرن ۱۸ شتاب گرفت:

\begin{figure}[H]
\centering
\begin{tikzpicture}
\begin{axis}[
    width=13cm,
    height=8cm,
    xlabel={سال},
    ylabel={تعداد قوانین محصورسازی (تجمعی)},
    xmin=1750, xmax=1830,
    ymin=0, ymax=4500,
    grid=major,
    area style,
]
\addplot+[fill=victoriangreen!40, draw=victoriangreen, thick] coordinates {
    (1750, 100) (1760, 200) (1770, 700) (1780, 1300) 
    (1790, 1800) (1800, 2500) (1810, 3500) (1820, 4000) (1830, 4200)
} \closedcycle;
\end{axis}
\end{tikzpicture}
\caption{شتاب محصورسازی در قرن هجدهم (قوانین پارلمانی)}
\end{figure}

\subsubsection{پیامدهای طبقاتی محصورسازی}

\begin{table}[H]
\centering
\caption{برندگان و بازندگان محصورسازی}
\renewcommand{\arraystretch}{1.4}
\begin{tabularx}{\textwidth}{>{\bfseries}p{2.5cm}XX}
\toprule
\textbf{گروه} & \textbf{وضعیت قبل} & \textbf{وضعیت بعد} \\
\midrule
\rowcolor{victoriangreen!10}
\textcolor{victoriangreen}{زمین‌داران بزرگ} & مالک بخشی از زمین & مالک زمین‌های یکپارچه، بزرگ‌تر \\

\rowcolor{victoriangreen!10}
\textcolor{victoriangreen}{ژنتری} & زمین‌دار متوسط & سود از بهره‌وری بالاتر \\

\rowcolor{imperialred!10}
\textcolor{imperialred}{یومن‌های کوچک} & مالک زمین کوچک + حقوق مشاع & بسیاری زمین را فروختند \\

\rowcolor{imperialred!10}
\textcolor{imperialred}{کارگران روستایی} & دسترسی به مشاع & کارگر بی‌زمین یا مهاجر به شهر \\

\bottomrule
\end{tabularx}
\end{table}

\begin{historicalquote}
«قانون قفل می‌کند مرد یا زنی را\\
که غازی را از مشاع بدزدد\\
اما رها می‌کند شرور بزرگ‌تر را\\
که مشاع را از غاز می‌دزدد.»
\hfill --- شعر عامیانه قرن ۱۸ علیه محصورسازی
\end{historicalquote}

\begin{policybox}[اهمیت محصورسازی برای صنعتی‌شدن]
محصورسازی، هرچند ناعادلانه، پیش‌شرط‌های صنعتی‌شدن را فراهم کرد:
\begin{enumerate}
    \item \textbf{نیروی کار آزاد:} روستاییان بی‌زمین به کارخانه‌ها رفتند
    \item \textbf{بازار داخلی:} روستاییان مجبور به خرید (نه خودتأمینی) شدند
    \item \textbf{سرمایه:} سود کشاورزی به صنعت سرمایه‌گذاری شد
    \item \textbf{غذای ارزان:} بهره‌وری بالاتر = دستمزد واقعی پایین‌تر
\end{enumerate}
مارکس این را «انباشت اولیه» (primitive accumulation) می‌نامد: ثروت 
اولیه‌ای که سرمایه‌داری از «جدایی تولیدکنندگان از ابزار تولید» به دست آورد.
\end{policybox}

%----------------------------------------------------------------------
\section{انقلاب صنعتی (۱۷۶۰-۱۸۳۰)}
%----------------------------------------------------------------------

\subsection{چرا انگلستان؟ چرا آن زمان؟}

\begin{figure}[H]
\centering
\begin{tikzpicture}[
    mindmap,
    grow cyclic,
    every node/.style={concept, circular drop shadow},
    concept color=royalblue!30,
    level 1/.append style={level distance=4.5cm, sibling angle=51},
    level 2/.append style={level distance=2.8cm, sibling angle=40},
    font=\small
]
\node[concept, font=\bfseries] {چرا انقلاب صنعتی\\در انگلستان؟}
    child[concept color=tudorpurple!40] { node {نهادی}
        child { node[font=\tiny] {حقوق مالکیت} }
        child { node[font=\tiny] {پتنت} }
        child { node[font=\tiny] {ثبات سیاسی} }
    }
    child[concept color=victoriangreen!40] { node {اقتصادی}
        child { node[font=\tiny] {سرمایه انباشته} }
        child { node[font=\tiny] {بازار داخلی} }
        child { node[font=\tiny] {تجارت جهانی} }
    }
    child[concept color=parliamentgold!40] { node {جغرافیایی}
        child { node[font=\tiny] {زغال‌سنگ} }
        child { node[font=\tiny] {آهن} }
        child { node[font=\tiny] {رودها و کانال‌ها} }
    }
    child[concept color=imperialred!40] { node {اجتماعی}
        child { node[font=\tiny] {نیروی کار آزاد} }
        child { node[font=\tiny] {فرهنگ نوآوری} }
        child { node[font=\tiny] {پروتستانتیسم} }
    }
    child[concept color=industrialgray!40] { node {امپراتوری}
        child { node[font=\tiny] {مواد خام} }
        child { node[font=\tiny] {بازار صادرات} }
        child { node[font=\tiny] {کار اجباری مستعمراتی} }
    }
    child[concept color=royalblue!60] { node {فناوری}
        child { node[font=\tiny] {اختراعات} }
        child { node[font=\tiny] {سنت صنعتگری} }
        child { node[font=\tiny] {روشنگری} }
    }
    child[concept color=victoriangreen!60] { node {کشاورزی}
        child { node[font=\tiny] {بهره‌وری} }
        child { node[font=\tiny] {مازاد غذا} }
    };
\end{tikzpicture}
\caption{عوامل چندگانه انقلاب صنعتی در انگلستان}
\end{figure}

\subsection{نوآوری‌های کلیدی}

\begin{table}[H]
\centering
\caption{اختراعات کلیدی انقلاب صنعتی}
\renewcommand{\arraystretch}{1.3}
\begin{tabularx}{\textwidth}{clXl}
\toprule
\textbf{سال} & \textbf{اختراع} & \textbf{مخترع} & \textbf{صنعت} \\
\midrule
۱۷۳۳ & ماکوی پرنده & جان کی & نساجی \\
\rowcolor{empirecream}
۱۷۶۴ & جنی ریسندگی & جیمز هارگریوز & نساجی \\
۱۷۶۹ & واترفریم & ریچارد آرکرایت & نساجی \\
\rowcolor{empirecream}
۱۷۶۹ & ماشین بخار بهبودیافته & جیمز وات & عمومی \\
۱۷۷۹ & میول & ساموئل کرامپتون & نساجی \\
\rowcolor{empirecream}
۱۷۸۴ & پادلینگ & هنری کورت & آهن \\
۱۷۸۵ & دستگاه بافندگی بخاری & ادموند کارترایت & نساجی \\
\rowcolor{empirecream}
۱۸۰۴ & لوکوموتیو بخار & ریچارد ترویتیک & حمل‌ونقل \\
۱۸۲۵ & راه‌آهن عمومی & جرج استفنسون & حمل‌ونقل \\
\bottomrule
\end{tabularx}
\end{table}

\subsection{رشد صنعتی}

\begin{figure}[H]
\centering
\begin{tikzpicture}
\begin{axis}[
    width=14cm,
    height=9cm,
    xlabel={سال},
    ylabel={شاخص تولید (۱۸۰۰ = ۱۰۰)},
    xmin=1700, xmax=1850,
    ymin=0, ymax=500,
    legend pos=north west,
    grid=major,
]

% تولید کل صنعتی
\addplot[very thick, royalblue, mark=*] coordinates {
    (1700, 15) (1750, 25) (1780, 45) (1800, 100) 
    (1820, 180) (1830, 250) (1850, 450)
};
\addlegendentry{تولید صنعتی کل}

% نساجی پنبه
\addplot[very thick, victoriangreen, mark=square*] coordinates {
    (1700, 5) (1750, 10) (1780, 30) (1800, 100) 
    (1820, 220) (1830, 350) (1850, 500)
};
\addlegendentry{نساجی پنبه}

% آهن
\addplot[very thick, imperialred, mark=triangle*] coordinates {
    (1700, 20) (1750, 35) (1780, 60) (1800, 100) 
    (1820, 150) (1830, 200) (1850, 400)
};
\addlegendentry{آهن}

\end{axis}
\end{tikzpicture}
\caption{رشد تولید صنعتی بریتانیا (۱۷۰۰-۱۸۵۰)}
\end{figure}

\subsection{پیامدهای طبقاتی انقلاب صنعتی}

\subsubsection{ظهور بورژوازی صنعتی}

\begin{figure}[H]
\centering
\begin{tikzpicture}[
    node distance=1.5cm,
    class/.style={rectangle, rounded corners, draw, thick,
                  minimum width=5cm, minimum height=2cm,
                  text centered, text width=4.8cm, font=\small}
]

\node[class, fill=parliamentgold!25] (old_bourg) {
    \textbf{بورژوازی تجاری (قدیم)}\\[0.2cm]
    لندن، بندرها\\
    تجارت، بانکداری\\
    شرکت‌های شرقی و غربی
};

\node[class, fill=victoriangreen!25, right=2cm of old_bourg] (new_bourg) {
    \textbf{بورژوازی صنعتی (جدید)}\\[0.2cm]
    منچستر، برمینگام\\
    کارخانه‌ها، معادن\\
    آرکرایت، بولتون، وج‌وود
};

\node[below=2cm of old_bourg, xshift=3.5cm, class, fill=royalblue!20] (fusion) {
    \textbf{همگرایی}\\[0.2cm]
    ازدواج، سرمایه‌گذاری مشترک\\
    ورود به ژنتری\\
    طبقه حاکم جدید
};

\draw[-{Stealth}, thick] (old_bourg) -- (fusion);
\draw[-{Stealth}, thick] (new_bourg) -- (fusion);

\end{tikzpicture}
\caption{همگرایی بورژوازی تجاری و صنعتی}
\end{figure}

\subsubsection{ظهور طبقه کارگر صنعتی}

\begin{table}[H]
\centering
\caption{شرایط طبقه کارگر در انقلاب صنعتی اولیه}
\renewcommand{\arraystretch}{1.4}
\begin{tabularx}{\textwidth}{>{\bfseries\color{imperialred}}p{3cm}X}
\toprule
\textbf{بُعد} & \textbf{شرایط} \\
\midrule
\rowcolor{empirecream}
ساعات کار & ۱۴-۱۶ ساعت در روز، ۶ روز در هفته \\

دستمزد & نازل، پرداخت به‌صورت کالا (truck system) \\

\rowcolor{empirecream}
کار کودکان & از ۵ سالگی در معادن و کارخانه‌ها \\

ایمنی & حوادث مکرر، بیماری‌های شغلی \\

\rowcolor{empirecream}
مسکن & زاغه‌های شهری، بدون بهداشت \\

حقوق & ممنوعیت اتحادیه (قوانین ترکیب، ۱۷۹۹-۱۸۰۰) \\

\bottomrule
\end{tabularx}
\end{table}

\begin{historicalquote}
«منچستر با تمام ابزار تمدنش، به جهنمی برای کارگران تبدیل شده است... 
دود، بوی گند، فقر، بیماری، و مرگ زودرس.»
\hfill --- فردریش انگلس، \textit{وضع طبقه کارگر در انگلستان} (۱۸۴۵)
\end{historicalquote}

%----------------------------------------------------------------------
\section{انقلاب مالی و امپراتوری}
%----------------------------------------------------------------------

\subsection{انقلاب مالی}

تسویه ۱۶۸۸ زمینه «انقلاب مالی» را فراهم کرد که بریتانیا را به قدرت 
اقتصادی-نظامی مسلط تبدیل کرد:

\begin{figure}[H]
\centering
\begin{tikzpicture}[
    node distance=1.5cm,
    inst/.style={rectangle, rounded corners, draw=royalblue, thick,
                 fill=royalblue!15, minimum width=4.5cm, 
                 minimum height=1.3cm, text centered, text width=4.3cm}
]

\node[inst] (bank) {۱۶۹۴: بانک انگلستان\\وام‌دهی به دولت};
\node[inst, below=of bank] (debt) {بدهی ملی\\تضمین‌شده توسط پارلمان};
\node[inst, below=of debt] (stock) {بازار سهام\\تأمین مالی شرکت‌ها};
\node[inst, below=of stock] (insurance) {بازار بیمه\\لویدز لندن (۱۶۸۸)};

\node[right=3cm of debt, rectangle, rounded corners=10pt,
      draw=victoriangreen, very thick, fill=victoriangreen!25,
      minimum width=5cm, minimum height=3cm, text centered,
      text width=4.8cm] (outcome) {
    \textbf{نتیجه:}\\[0.3cm]
    هزینه استقراض دولت:\\
    ۱۶۹۰: ۱۴٪\\
    ۱۷۵۰: ۳٪\\[0.3cm]
    $\Rightarrow$ قدرت نظامی برتر
};

\draw[-{Stealth}, thick] (bank) -- (debt);
\draw[-{Stealth}, thick] (debt) -- (stock);
\draw[-{Stealth}, thick] (stock) -- (insurance);
\draw[-{Stealth}, thick] (debt.east) -- (outcome.west);

\end{tikzpicture}
\caption{نهادهای انقلاب مالی و پیامدشان}
\end{figure}

\subsection{امپراتوری تجاری}

\begin{figure}[H]
\centering
\begin{tikzpicture}[
    node distance=0.5cm,
    region/.style={rectangle, rounded corners, draw, thick,
                   minimum width=4cm, minimum height=1.5cm,
                   text centered, text width=3.8cm, font=\small}
]

% بریتانیا در مرکز
\node[ellipse, draw=royalblue, very thick, fill=royalblue!20,
      minimum width=3cm, minimum height=2cm] (britain) 
    {\textbf{بریتانیا}\\کالای صنعتی};

% مستعمرات
\node[region, fill=parliamentgold!20, above left=2cm and 1cm of britain] (america) 
    {آمریکای شمالی\\تنباکو، پوست};

\node[region, fill=imperialred!20, above right=2cm and 1cm of britain] (caribbean) 
    {کارائیب\\شکر، رام\\(کار برده)};

\node[region, fill=victoriangreen!20, below left=2cm and 1cm of britain] (india) 
    {هند\\پارچه، چای\\ادویه};

\node[region, fill=industrialgray!20, below right=2cm and 1cm of britain] (africa) 
    {آفریقا\\برده\\(تجارت مثلثی)};

% فلش‌ها
\draw[-{Stealth}, thick] (britain) -- (america);
\draw[-{Stealth}, thick] (america) -- (britain);
\draw[-{Stealth}, thick] (britain) -- (caribbean);
\draw[-{Stealth}, thick] (caribbean) -- (britain);
\draw[-{Stealth}, thick] (britain) -- (india);
\draw[-{Stealth}, thick] (india) -- (britain);
\draw[-{Stealth}, thick] (africa) -- (caribbean) node[midway, below right, font=\tiny] {برده};

\end{tikzpicture}
\caption{شبکه تجاری امپراتوری بریتانیا در قرن ۱۸}
\end{figure}

\begin{warningbox}
\textbf{هزینه‌های امپراتوری:}

انقلاب صنعتی و ثروت بریتانیا تنها محصول نوآوری داخلی نبود. امپراتوری 
نقش کلیدی داشت:
\begin{itemize}
    \item \textbf{تجارت برده:} بریتانیا بزرگ‌ترین تاجر برده اروپایی بود. 
    حدود ۳ میلیون آفریقایی توسط کشتی‌های بریتانیایی به آمریکا برده شدند.
    
    \item \textbf{استثمار هند:} صنعت نساجی هند نابود شد تا بازار برای 
    پارچه منچستر باز شود.
    
    \item \textbf{کار اجباری:} شکر و پنبه محصول کار برده بود.
\end{itemize}
این ابعاد تاریک نباید فراموش شود.
\end{warningbox}

%----------------------------------------------------------------------
\section{بحران و اصلاحات (۱۷۸۹-۱۸۳۲)}
%----------------------------------------------------------------------

\subsection{تأثیر انقلاب فرانسه}

انقلاب فرانسه (۱۷۸۹) موج شوک به بریتانیا فرستاد:

\begin{table}[H]
\centering
\caption{واکنش‌ها به انقلاب فرانسه در بریتانیا}
\renewcommand{\arraystretch}{1.4}
\begin{tabularx}{\textwidth}{>{\bfseries}p{3cm}XX}
\toprule
\textbf{گروه} & \textbf{واکنش اولیه} & \textbf{واکنش بعدی} \\
\midrule
\rowcolor{empirecream}
رادیکال‌ها & تحسین، الهام‌گیری & سرکوب‌شده، زیرزمینی \\

ویگ‌های اصلاح‌طلب & همدردی محتاطانه & تقسیم، بیشتر به توری پیوستند \\

\rowcolor{empirecream}
توری‌ها & مخالفت، ترس & سرکوب، جنگ با فرانسه \\

طبقه کارگر & برخی حمایت & سرکوب شدید \\

\bottomrule
\end{tabularx}
\end{table}

\subsubsection{دوره سرکوب (۱۷۹۳-۱۸۱۵)}

\begin{itemize}
    \item قوانین ترکیب (Combination Acts) ۱۷۹۹-۱۸۰۰: ممنوعیت اتحادیه‌های کارگری
    \item تعلیق هبیث کورپوس
    \item سرکوب ناآرامی‌ها (مثل پیترلو ۱۸۱۹)
    \item قوانین شش‌گانه ۱۸۱۹
\end{itemize}

\subsection{فشار برای اصلاحات}

پس از جنگ‌های ناپلئونی، فشار برای اصلاحات افزایش یافت:

\begin{figure}[H]
\centering
\begin{tikzpicture}[
    node distance=1.5cm,
    pressure/.style={rectangle, rounded corners, draw=imperialred, thick,
                     fill=imperialred!15, minimum width=4cm,
                     minimum height=1.2cm, text centered, text width=3.8cm}
]

\node[pressure] (working) {فشار طبقه کارگر\\چارتیسم، شورش‌ها};
\node[pressure, right=1cm of working] (middle) {فشار طبقه متوسط\\بورژوازی صنعتی};
\node[pressure, right=1cm of middle] (fear) {ترس از انقلاب\\یادآوری فرانسه};

\node[below=2cm of middle, rectangle, rounded corners=10pt,
      draw=victoriangreen, very thick, fill=victoriangreen!25,
      minimum width=6cm, minimum height=1.5cm, text centered] (reform) {
    \textbf{قانون اصلاحات ۱۸۳۲}\\
    (Reform Act)
};

\draw[-{Stealth}, thick] (working) -- (reform);
\draw[-{Stealth}, thick] (middle) -- (reform);
\draw[-{Stealth}, thick] (fear) -- (reform);

\end{tikzpicture}
\caption{فشارهای منتج به قانون اصلاحات ۱۸۳۲}
\end{figure}

\subsection{قانون اصلاحات ۱۸۳۲}

قانون اصلاحات ۱۸۳۲ نقطه عطفی در تاریخ بریتانیا بود:

\begin{table}[H]
\centering
\caption{تغییرات قانون اصلاحات ۱۸۳۲}
\renewcommand{\arraystretch}{1.4}
\begin{tabularx}{\textwidth}{>{\bfseries}p{3.5cm}XX}
\toprule
\textbf{بُعد} & \textbf{قبل} & \textbf{بعد} \\
\midrule
\rowcolor{empirecream}
حق رأی & ۴۳۵,۰۰۰ نفر (۳٪ جمعیت) & ۶۵۲,۰۰۰ نفر (۵٪ جمعیت) \\

توزیع کرسی‌ها & «حوزه‌های فاسد» (rotten boroughs) & حذف ۵۶ حوزه، کرسی برای شهرهای صنعتی \\

\rowcolor{empirecream}
معیار رأی & ملکی سنتی & مستأجران ۱۰ پوندی نیز \\

بورژوازی صنعتی & حاشیه‌ای & نماینده‌دار \\

\bottomrule
\end{tabularx}
\end{table}

\begin{policybox}[اهمیت قانون ۱۸۳۲]
قانون ۱۸۳۲ یک انقلاب نبود، اما:
\begin{enumerate}
    \item \textbf{اصل را تثبیت کرد:} نظام انتخاباتی قابل اصلاح است
    \item \textbf{بورژوازی صنعتی را وارد کرد:} منچستر، برمینگام نماینده یافتند
    \item \textbf{از انقلاب جلوگیری کرد:} سازش، نه سرنگونی
    \item \textbf{راه را باز کرد:} برای اصلاحات ۱۸۶۷ و ۱۸۸۴
\end{enumerate}
این الگوی «اصلاح از بالا برای جلوگیری از انقلاب از پایین» مشخصه تاریخ 
بریتانیا شد.
\end{policybox}

%----------------------------------------------------------------------
\section{جمع‌بندی فصل}
%----------------------------------------------------------------------

\subsection{دستاوردهای دوره ۱۷۱۴-۱۸۳۲}

\begin{figure}[H]
\centering
\begin{tikzpicture}[
    achieve/.style={rectangle, rounded corners=8pt, draw=victoriangreen,
                    thick, fill=victoriangreen!15, minimum width=5cm,
                    minimum height=1.2cm, text centered, text width=4.8cm}
]

\node[achieve] (a1) at (0,0) {انقلاب صنعتی\\نخستین اقتصاد صنعتی جهان};
\node[achieve] (a2) at (6,0) {قدرت جهانی\\شکست فرانسه، امپراتوری};
\node[achieve] (a3) at (0,-2) {ثبات سیاسی\\بدون انقلاب، بدون جنگ داخلی};
\node[achieve] (a4) at (6,-2) {اصلاح تدریجی\\قانون ۱۸۳۲};
\node[achieve] (a5) at (3,-4) {انقلاب مالی\\بانک، بدهی، بیمه، بازار سهام};

\end{tikzpicture}
\caption{دستاوردهای کلیدی دوره ۱۷۱۴-۱۸۳۲}
\end{figure}

\subsection{هزینه‌ها و تناقضات}

\begin{table}[H]
\centering
\caption{هزینه‌های توسعه}
\renewcommand{\arraystretch}{1.4}
\begin{tabularx}{\textwidth}{>{\bfseries\color{imperialred}}p{3cm}X}
\toprule
\textbf{هزینه} & \textbf{توضیح} \\
\midrule
\rowcolor{empirecream}
استثمار کارگران & شرایط وحشتناک کار، فقر، بیماری، مرگ زودرس \\

استعمار & تجارت برده، نابودی صنایع مستعمرات، کشتار \\

\rowcolor{empirecream}
نابرابری & فاصله طبقاتی افزایش یافت \\

محیط زیست & آلودگی شهرهای صنعتی \\

\rowcolor{empirecream}
دموکراسی محدود & هنوز ۹۵٪ حق رأی نداشتند \\

\bottomrule
\end{tabularx}
\end{table}

%----------------------------------------------------------------------
% ارجاعات فصل چهارم
%----------------------------------------------------------------------

\section*{ارجاعات فصل چهارم}

\begin{enumerate}[label={[\arabic*]}]
    \item Allen, R.C. (2009). \textit{The British Industrial Revolution in 
    Global Perspective}. Cambridge: Cambridge University Press.
    
    \item Berg, M. (1994). \textit{The Age of Manufactures, 1700-1820}. 
    2nd ed. London: Routledge.
    
    \item Brewer, J. (1989). \textit{The Sinews of Power: War, Money, and 
    the English State, 1688-1783}. London: Unwin Hyman.
    
    \item Clark, G. (2007). \textit{A Farewell to Alms: A Brief Economic 
    History of the World}. Princeton: Princeton University Press.
    
    \item Dickson, P.G.M. (1967). \textit{The Financial Revolution in 
    England: A Study in the Development of Public Credit, 1688-1756}. 
    London: Macmillan.
    
    \item Mokyr, J. (2009). \textit{The Enlightened Economy: An Economic 
    History of Britain 1700-1850}. New Haven: Yale University Press.
    
    \item Plumb, J.H. (1967). \textit{The Growth of Political Stability in 
    England 1675-1725}. London: Macmillan.
    
    \item Thompson, E.P. (1963). \textit{The Making of the English Working 
    Class}. London: Victor Gollancz.
    
    \item Williams, E. (1944). \textit{Capitalism and Slavery}. Chapel Hill: 
    University of North Carolina Press.
    
    \item Wrigley, E.A. (2010). \textit{Energy and the English Industrial 
    Revolution}. Cambridge: Cambridge University Press.
\end{enumerate}

%%%%%%%%%%%%%%%%%%%%%%%%%%%%%%%%%%%%%%%%%%%%%%%%%%%%%%%%%%%%%%%%%%%%%%%
% پایان فصل چهارم
%%%%%%%%%%%%%%%%%%%%%%%%%%%%%%%%%%%%%%%%%%%%%%%%%%%%%%%%%%%%%%%%%%%%%%%

\vfill

\begin{center}
\textcolor{royalblue}{\rule{0.5\textwidth}{1pt}}\\[0.5cm]
\textit{پایان فصل چهارم}\\[0.3cm]
\textbf{ادامه دارد: فصل پنجم (عصر ویکتوریایی و اوج امپراتوری: ۱۸۳۲-۱۹۱۴)}\\[0.5cm]
\textcolor{royalblue}{\rule{0.5\textwidth}{1pt}}
\end{center}

%%%%%%%%%%%%%%%%%%%%%%%%%%%%%%%%%%%%%%%%%%%%%%%%%%%%%%%%%%%%%%%%%%%%%%%
% فصل پنجم: عصر ویکتوریایی و اوج امپراتوری (۱۸۳۲-۱۹۱۴)
%%%%%%%%%%%%%%%%%%%%%%%%%%%%%%%%%%%%%%%%%%%%%%%%%%%%%%%%%%%%%%%%%%%%%%%

\chapter{عصر ویکتوریایی و اوج امپراتوری (۱۸۳۲-۱۹۱۴)}

\begin{keybox}[خلاصه فصل]
این فصل به تحلیل دورانی می‌پردازد که بریتانیا به \textbf{«کارگاه جهان»}، 
\textbf{«بانکدار جهان»}، و \textbf{رهبر بزرگ‌ترین امپراتوری تاریخ} تبدیل 
شد. در همین دوره، \textbf{دموکراسی به‌تدریج گسترش یافت} (از ۵٪ به ۶۰٪ 
مردان بالغ)، \textbf{طبقه کارگر سازمان‌یافته} ظهور کرد، و \textbf{دولت 
رفاه اولیه} شکل گرفت. این تحولات نه از طریق انقلاب، بلکه از طریق 
\textbf{اصلاحات تدریجی} صورت گرفت -- الگویی که «استثناگرایی بریتانیایی» 
خوانده می‌شود. اما این دوره با \textbf{بحران‌های فزاینده} (ایرلند، 
رقابت آلمان، تنش‌های طبقاتی) به پایان رسید و جنگ جهانی اول نقطه پایان 
«قرن بریتانیایی» شد.
\end{keybox}

%----------------------------------------------------------------------
\section{ویژگی‌های عصر ویکتوریایی}
%----------------------------------------------------------------------

\subsection{ملکه ویکتوریا و نماد یک عصر}

ویکتوریا از ۱۸۳۷ تا ۱۹۰۱ -- شصت و چهار سال -- سلطنت کرد، طولانی‌ترین 
سلطنت در تاریخ بریتانیا تا آن زمان. نام او به کل عصر داده شد.

\begin{figure}[H]
\centering
\begin{tikzpicture}[
    timeline/.style={very thick, royalblue},
    period/.style={rectangle, rounded corners, draw, thick,
                   minimum height=1.5cm, text centered, font=\small}
]

% خط زمانی
\draw[timeline] (0,0) -- (15,0);

% دوره‌ها
\node[period, fill=tudorpurple!20, minimum width=3.5cm] at (1.75,1.5) 
    {ویکتوریای اولیه\\۱۸۳۷-۱۸۵۱\\اصلاحات، چارتیسم};

\node[period, fill=parliamentgold!20, minimum width=4cm] at (5.75,1.5) 
    {ویکتوریای میانی\\۱۸۵۱-۱۸۷۳\\اوج رونق، «کارگاه جهان»};

\node[period, fill=victoriangreen!20, minimum width=4cm] at (10,1.5) 
    {ویکتوریای متأخر\\۱۸۷۳-۱۹۰۱\\رکود، نو-امپریالیسم};

\node[period, fill=imperialred!20, minimum width=3cm] at (13.5,1.5) 
    {ادواردی\\۱۹۰۱-۱۹۱۴\\بحران، اصلاحات لیبرال};

% تاریخ‌ها
\foreach \x/\year in {0/1837, 3.5/1851, 7.75/1873, 12/1901, 15/1914} {
    \draw[thick] (\x,0.2) -- (\x,-0.2);
    \node[below, font=\small] at (\x,-0.3) {\year};
}

\end{tikzpicture}
\caption{دوره‌بندی عصر ویکتوریایی و ادواردی}
\end{figure}

\subsection{شاخص‌های کلیدی دوره}

\begin{table}[H]
\centering
\caption{شاخص‌های توسعه بریتانیا (۱۸۳۰-۱۹۱۴)}
\renewcommand{\arraystretch}{1.3}
\begin{tabular}{lcccc}
\toprule
\textbf{شاخص} & \textbf{۱۸۳۰} & \textbf{۱۸۷۰} & \textbf{۱۹۰۰} & \textbf{۱۹۱۴} \\
\midrule
جمعیت بریتانیا (میلیون) & ۲۴ & ۳۱ & ۴۱ & ۴۶ \\
\rowcolor{empirecream}
جمعیت شهری (\%) & ۴۴ & ۶۵ & ۷۷ & ۸۰ \\
GDP سرانه (دلار ۱۹۹۰) & ۱,۷۰۰ & ۳,۲۰۰ & ۴,۵۰۰ & ۴,۹۰۰ \\
\rowcolor{empirecream}
سهم از تولید صنعتی جهان (\%) & ۲۵ & ۳۲ & ۱۸ & ۱۴ \\
سهم از تجارت جهانی (\%) & ۲۷ & ۲۵ & ۱۹ & ۱۵ \\
\rowcolor{empirecream}
طول راه‌آهن (هزار مایل) & ۰.۱ & ۱۵ & ۲۲ & ۲۴ \\
مساحت امپراتوری (میلیون مایل مربع) & ۲ & ۸ & ۱۱ & ۱۳ \\
\rowcolor{empirecream}
جمعیت امپراتوری (میلیون) & ۱۵۰ & ۲۵۰ & ۳۸۰ & ۴۵۰ \\
\bottomrule
\end{tabular}
\end{table}

\begin{figure}[H]
\centering
\begin{tikzpicture}
\begin{axis}[
    width=14cm,
    height=8cm,
    xlabel={سال},
    ylabel={سهم از تولید صنعتی جهان (\%)},
    xmin=1820, xmax=1920,
    ymin=0, ymax=40,
    legend pos=north east,
    grid=major,
]

\addplot[very thick, royalblue, mark=*] coordinates {
    (1820, 24) (1840, 27) (1860, 34) (1880, 28) 
    (1900, 20) (1913, 14)
};
\addlegendentry{بریتانیا}

\addplot[very thick, imperialred, mark=square*] coordinates {
    (1820, 5) (1840, 8) (1860, 15) (1880, 23) 
    (1900, 30) (1913, 36)
};
\addlegendentry{آمریکا}

\addplot[very thick, industrialgray, mark=triangle*] coordinates {
    (1820, 4) (1840, 5) (1860, 8) (1880, 13) 
    (1900, 17) (1913, 16)
};
\addlegendentry{آلمان}

\addplot[very thick, parliamentgold, mark=diamond*] coordinates {
    (1820, 8) (1840, 9) (1860, 10) (1880, 9) 
    (1900, 7) (1913, 6)
};
\addlegendentry{فرانسه}

\end{axis}
\end{tikzpicture}
\caption{زوال نسبی صنعتی بریتانیا (۱۸۲۰-۱۹۱۳)}
\end{figure}

\begin{warningbox}
\textbf{پارادوکس عصر ویکتوریایی:}

بریتانیا در این دوره هم در اوج و هم در آغاز افول بود:
\begin{itemize}
    \item \textbf{اوج مطلق:} بزرگ‌ترین امپراتوری، ثروتمندترین کشور، 
    قدرتمندترین نیروی دریایی
    \item \textbf{افول نسبی:} سهم از تولید جهانی کاهش یافت، آمریکا و 
    آلمان پیش افتادند
\end{itemize}
این پارادوکس ناشی از \textbf{«نفرین پیشگامی»} بود: همان صنایعی که 
بریتانیا را صنعتی کرده بود، اکنون قدیمی شده بود.
\end{warningbox}

%----------------------------------------------------------------------
\section{گسترش تدریجی دموکراسی}
%----------------------------------------------------------------------

\subsection{اصلاحات انتخاباتی متوالی}

\begin{figure}[H]
\centering
\begin{tikzpicture}[
    reform/.style={rectangle, rounded corners=8pt, draw, very thick,
                   minimum width=3.8cm, minimum height=2.5cm,
                   text centered, text width=3.6cm, font=\small}
]

\node[reform, fill=tudorpurple!20, draw=tudorpurple] (r1832) at (0,0) {
    \textbf{۱۸۳۲}\\[0.2cm]
    ۵٪ $\rightarrow$ ۷٪\\
    طبقه متوسط\\
    حوزه‌های فاسد حذف
};

\node[reform, fill=victoriangreen!20, draw=victoriangreen] (r1867) at (4.5,0) {
    \textbf{۱۸۶۷}\\[0.2cm]
    ۷٪ $\rightarrow$ ۱۶٪\\
    کارگران شهری ماهر\\
    «پرش در تاریکی»
};

\node[reform, fill=parliamentgold!20, draw=parliamentgold] (r1884) at (9,0) {
    \textbf{۱۸۸۴}\\[0.2cm]
    ۱۶٪ $\rightarrow$ ۲۸٪\\
    کارگران روستایی\\
    حوزه‌های برابر
};

\node[reform, fill=royalblue!20, draw=royalblue] (r1918) at (13.5,0) {
    \textbf{۱۹۱۸}\\[0.2cm]
    ۲۸٪ $\rightarrow$ ۷۵٪\\
    همه مردان + زنان ۳۰+\\
    (۱۹۲۸: همه بالغین)
};

\draw[-{Stealth}, very thick] (r1832) -- (r1867);
\draw[-{Stealth}, very thick] (r1867) -- (r1884);
\draw[-{Stealth}, very thick] (r1884) -- (r1918);

\node[below=0.5cm of r1867, xshift=2.25cm, font=\footnotesize] 
    {درصدها: نسبت واجدین شرایط رأی به کل جمعیت بالغ};

\end{tikzpicture}
\caption{گسترش تدریجی حق رأی در بریتانیا}
\end{figure}

\subsection{قانون اصلاحات دوم (۱۸۶۷)}

\begin{historicalquote}
«ما در تاریکی پریدیم.»
\hfill --- لرد دربی، نخست‌وزیر توری، درباره قانون ۱۸۶۷
\end{historicalquote}

قانون ۱۸۶۷ توسط دولت توری (محافظه‌کار) به رهبری دیزرائیلی تصویب شد -- 
پارادوکسی که نیاز به توضیح دارد:

\begin{table}[H]
\centering
\caption{چرا توری‌ها به اصلاحات رأی دادند؟}
\renewcommand{\arraystretch}{1.4}
\begin{tabularx}{\textwidth}{>{\bfseries\color{royalblue}}p{3cm}X}
\toprule
\textbf{عامل} & \textbf{توضیح} \\
\midrule
\rowcolor{empirecream}
رقابت حزبی & ویگ‌ها (لیبرال‌ها) اصلاحات را پیشنهاد کرده بودند؛ توری‌ها 
می‌خواستند اعتبار را از آنها بربایند \\

محاسبه سیاسی & دیزرائیلی امید داشت کارگران «محافظه‌کار طبیعی» باشند \\

\rowcolor{empirecream}
فشار از پایین & تظاهرات جنبش اصلاحات (مثل شورش هاید پارک ۱۸۶۶) \\

ترس از انقلاب & یادآوری انقلاب‌های ۱۸۴۸ در اروپا \\

\bottomrule
\end{tabularx}
\end{table}

\subsection{قانون اصلاحات سوم (۱۸۸۴-۱۸۸۵)}

دولت لیبرال گلادستون حق رأی را به کارگران روستایی گسترش داد و 
همزمان حوزه‌های انتخاباتی را بازتوزیع کرد:

\begin{figure}[H]
\centering
\begin{tikzpicture}
\begin{axis}[
    ybar,
    width=12cm,
    height=7cm,
    ylabel={تعداد رأی‌دهندگان (میلیون)},
    symbolic x coords={قبل از ۱۸۸۴, بعد از ۱۸۸۴},
    xtick=data,
    ymin=0,
    ymax=6,
    bar width=30pt,
    nodes near coords,
    legend style={at={(0.5,-0.15)}, anchor=north},
]

\addplot[fill=royalblue!60] coordinates {(قبل از ۱۸۸۴, 3.0) (بعد از ۱۸۸۴, 5.7)};

\end{axis}
\end{tikzpicture}
\caption{افزایش تعداد رأی‌دهندگان پس از قانون ۱۸۸۴}
\end{figure}

\subsection{تحلیل: چرا دموکراتیزاسیون تدریجی؟}

\begin{figure}[H]
\centering
\begin{tikzpicture}[
    factor/.style={rectangle, rounded corners, draw=victoriangreen, thick,
                   fill=victoriangreen!15, minimum width=4cm,
                   minimum height=1.3cm, text centered, text width=3.8cm},
    arrow/.style={-{Stealth[length=3mm]}, thick}
]

\node[factor] (f1) at (0,3) {۱. فشار از پایین\\(طبقه کارگر، چارتیسم)};
\node[factor] (f2) at (5,3) {۲. رقابت حزبی\\(ویگ vs توری)};
\node[factor] (f3) at (10,3) {۳. ترس از انقلاب\\(یادآوری فرانسه)};
\node[factor] (f4) at (2.5,0) {۴. ایدئولوژی لیبرالیسم\\(حقوق فردی)};
\node[factor] (f5) at (7.5,0) {۵. «قابلیت احترام» کارگران\\(کارگر ماهر، پرهیزکار)};

\node[rectangle, rounded corners=10pt, draw=royalblue, very thick,
      fill=royalblue!20, minimum width=6cm, minimum height=1.5cm,
      text centered] (outcome) at (5,-3) {
    \textbf{گسترش تدریجی حق رأی}\\
    (بدون انقلاب، بدون شکست کامل نخبگان)
};

\draw[arrow] (f1) -- (outcome);
\draw[arrow] (f2) -- (outcome);
\draw[arrow] (f3) -- (outcome);
\draw[arrow] (f4) -- (outcome);
\draw[arrow] (f5) -- (outcome);

\end{tikzpicture}
\caption{عوامل دموکراتیزاسیون تدریجی}
\end{figure}

\begin{policybox}[الگوی بریتانیایی دموکراتیزاسیون]
دموکراتیزاسیون بریتانیا چند ویژگی خاص داشت که آن را از مدل‌های دیگر 
(مثل فرانسه) متمایز می‌کند:
\begin{enumerate}
    \item \textbf{تدریجی بودن:} نه یکباره، بلکه طی ۸۶ سال (۱۸۳۲-۱۹۱۸)
    \item \textbf{از بالا:} نخبگان امتیاز دادند، نه اینکه شکست بخورند
    \item \textbf{پیش‌گیرانه:} برای جلوگیری از انقلاب، نه پس از آن
    \item \textbf{طبقه‌محور:} ابتدا طبقه متوسط، سپس کارگران شهری، 
    سپس روستایی
    \item \textbf{مردانه:} زنان تا ۱۹۱۸ (و کامل تا ۱۹۲۸) محروم بودند
\end{enumerate}
\end{policybox}

%----------------------------------------------------------------------
\section{ظهور طبقه کارگر سازمان‌یافته}
%----------------------------------------------------------------------

\subsection{چارتیسم: نخستین جنبش کارگری توده‌ای (۱۸۳۸-۱۸۵۸)}

\begin{historicalquote}
«منشور مردم» شش خواسته داشت: حق رأی همگانی مردان، رأی‌گیری مخفی، 
حوزه‌های برابر، حذف شرط مالکیت برای نمایندگی، حقوق نمایندگان، و 
پارلمان سالانه.
\end{historicalquote}

\begin{figure}[H]
\centering
\begin{tikzpicture}[
    demand/.style={rectangle, rounded corners, draw=imperialred, thick,
                   fill=imperialred!15, minimum width=3cm,
                   minimum height=1cm, text centered, font=\small}
]

\node[demand] (d1) at (0,2) {حق رأی همگانی مردان};
\node[demand] (d2) at (4,2) {رأی‌گیری مخفی};
\node[demand] (d3) at (8,2) {حوزه‌های برابر};
\node[demand] (d4) at (0,0) {حذف شرط مالکیت};
\node[demand] (d5) at (4,0) {حقوق نمایندگان};
\node[demand] (d6) at (8,0) {پارلمان سالانه};

% وضعیت تحقق
\node[below=0.5cm of d1, font=\tiny, color=victoriangreen] {تحقق: ۱۹۱۸};
\node[below=0.5cm of d2, font=\tiny, color=victoriangreen] {تحقق: ۱۸۷۲};
\node[below=0.5cm of d3, font=\tiny, color=victoriangreen] {تحقق: ۱۸۸۵};
\node[below=0.5cm of d4, font=\tiny, color=victoriangreen] {تحقق: ۱۸۵۸};
\node[below=0.5cm of d5, font=\tiny, color=victoriangreen] {تحقق: ۱۹۱۱};
\node[below=0.5cm of d6, font=\tiny, color=imperialred] {هرگز تحقق نیافت};

\end{tikzpicture}
\caption{شش خواسته منشور مردم و زمان تحقق}
\end{figure}

\subsubsection{چرا چارتیسم شکست خورد اما پیروز شد؟}

\begin{table}[H]
\centering
\caption{علل شکست فوری و پیروزی بلندمدت چارتیسم}
\renewcommand{\arraystretch}{1.4}
\begin{tabularx}{\textwidth}{>{\bfseries}p{2cm}XX}
\toprule
& \textbf{علل شکست فوری (دهه ۱۸۴۰)} & \textbf{علل پیروزی بلندمدت} \\
\midrule
\rowcolor{empirecream}
& سرکوب دولتی & نهادینه شدن خواسته‌ها در فرهنگ سیاسی \\
& تقسیم بین رادیکال‌ها و میانه‌روها & پذیرش تدریجی توسط نخبگان \\
\rowcolor{empirecream}
& بهبود اقتصادی دهه ۱۸۵۰ & ترس از تکرار فشار \\
& فقدان سازمان پایدار & الهام‌بخشی برای جنبش‌های بعدی \\
\bottomrule
\end{tabularx}
\end{table}

\subsection{اتحادیه‌های کارگری}

\begin{figure}[H]
\centering
\begin{tikzpicture}[
    phase/.style={rectangle, rounded corners, draw, thick,
                  minimum width=4.5cm, minimum height=2.5cm,
                  text centered, text width=4.3cm, font=\small}
]

\node[phase, fill=industrialgray!20] (p1) at (0,0) {
    \textbf{فاز ۱: غیرقانونی}\\
    ۱۷۹۹-۱۸۲۴\\
    قوانین ترکیب\\
    اتحادیه‌های زیرزمینی
};

\node[phase, fill=tudorpurple!20] (p2) at (5,0) {
    \textbf{فاز ۲: صنفی}\\
    ۱۸۲۴-۱۸۸۰\\
    «اتحادیه‌های نوین»\\
    کارگران ماهر
};

\node[phase, fill=victoriangreen!20] (p3) at (10,0) {
    \textbf{فاز ۳: توده‌ای}\\
    ۱۸۸۰-۱۹۱۴\\
    «نو-اتحادیه‌گرایی»\\
    کارگران غیرماهر
};

\draw[-{Stealth}, very thick] (p1) -- (p2);
\draw[-{Stealth}, very thick] (p2) -- (p3);

\end{tikzpicture}
\caption{تکامل اتحادیه‌های کارگری در بریتانیا}
\end{figure}

\subsubsection{آمار عضویت اتحادیه‌ای}

\begin{figure}[H]
\centering
\begin{tikzpicture}
\begin{axis}[
    width=13cm,
    height=7cm,
    xlabel={سال},
    ylabel={عضویت اتحادیه‌ای (میلیون نفر)},
    xmin=1860, xmax=1915,
    ymin=0, ymax=4.5,
    grid=major,
    legend pos=north west,
]

\addplot[very thick, imperialred, mark=*] coordinates {
    (1860, 0.2) (1870, 0.3) (1880, 0.5) (1890, 1.5) 
    (1900, 2.0) (1910, 2.5) (1914, 4.1)
};
\addlegendentry{کل عضویت}

\addplot[very thick, royalblue, mark=square*, dashed] coordinates {
    (1860, 0.2) (1870, 0.3) (1880, 0.45) (1890, 0.8) 
    (1900, 0.9) (1910, 1.0) (1914, 1.2)
};
\addlegendentry{کارگران ماهر}

\addplot[very thick, victoriangreen, mark=triangle*, densely dotted] coordinates {
    (1860, 0.0) (1870, 0.0) (1880, 0.05) (1890, 0.7) 
    (1900, 1.1) (1910, 1.5) (1914, 2.9)
};
\addlegendentry{کارگران غیرماهر}

\end{axis}
\end{tikzpicture}
\caption{رشد عضویت اتحادیه‌ای (۱۸۶۰-۱۹۱۴)}
\end{figure}

\subsubsection{اعتصابات مهم دهه ۱۸۸۰-۱۸۹۰}

\begin{table}[H]
\centering
\caption{اعتصابات کلیدی «نو-اتحادیه‌گرایی»}
\renewcommand{\arraystretch}{1.4}
\begin{tabularx}{\textwidth}{>{\bfseries}c>{\bfseries}p{2.5cm}Xc}
\toprule
\textbf{سال} & \textbf{اعتصاب} & \textbf{اهمیت} & \textbf{نتیجه} \\
\midrule
\rowcolor{empirecream}
۱۸۸۸ & دختران کبریت‌سازی (Bryant \& May) & نخستین اعتصاب موفق کارگران غیرماهر؛ افشای شرایط کار خطرناک & پیروزی \\

۱۸۸۹ & اعتصاب بارانداز لندن & ۱۰۰,۰۰۰ کارگر؛ تولد اتحادیه بارانداز & پیروزی \\

\rowcolor{empirecream}
۱۸۸۹ & کارگران گاز & ۸ ساعت کار روزانه & پیروزی \\

۱۸۹۳ & معدنچیان یورکشایر & ۳۰۰,۰۰۰ کارگر؛ ۴ ماه & نیمه‌موفق \\

\rowcolor{empirecream}
۱۹۱۱ & اعتصاب سراسری راه‌آهن & نخستین بحران ملی & مصالحه \\

۱۹۱۲ & معدنچیان زغال‌سنگ & ۱ میلیون کارگر؛ حداقل دستمزد & پیروزی \\

\bottomrule
\end{tabularx}
\end{table}

\begin{historicalquote}
«ما خواستار شش پنی ساعت -- شش پنی ساعت کامل -- و نه یک فارتینگ کمتر هستیم!»
\hfill --- بن تیلت، رهبر اعتصاب بارانداز ۱۸۸۹
\end{historicalquote}

\subsection{تولد حزب کارگر}

\begin{figure}[H]
\centering
\begin{tikzpicture}[
    org/.style={rectangle, rounded corners=5pt, draw, thick,
                minimum width=3.2cm, minimum height=1.5cm,
                text centered, text width=3cm, font=\small}
]

% سازمان‌های مؤسس
\node[org, fill=imperialred!20] (tuc) at (0,4) {کنگره اتحادیه‌های کارگری\\(TUC)\\تأسیس ۱۸۶۸};
\node[org, fill=tudorpurple!20] (sdf) at (4,4) {فدراسیون سوسیال‌دموکرات\\(SDF)\\۱۸۸۱};
\node[org, fill=victoriangreen!20] (fab) at (8,4) {جامعه فابین\\۱۸۸۴};
\node[org, fill=parliamentgold!20] (ilp) at (12,4) {حزب کارگر مستقل\\(ILP)\\۱۸۹۳};

% LRC و Labour
\node[org, fill=royalblue!30, minimum width=6cm] (lrc) at (6,1.5) {
    کمیته نمایندگی کارگری\\(LRC)\\تأسیس ۱۹۰۰
};

\node[org, fill=imperialred!40, minimum width=6cm, minimum height=2cm] (labour) at (6,-1.5) {
    \textbf{حزب کارگر}\\تغییر نام ۱۹۰۶\\۲۹ نماینده در انتخابات ۱۹۰۶
};

% فلش‌ها
\draw[-{Stealth}, thick] (tuc) -- (lrc);
\draw[-{Stealth}, thick] (sdf) -- (lrc);
\draw[-{Stealth}, thick] (fab) -- (lrc);
\draw[-{Stealth}, thick] (ilp) -- (lrc);
\draw[-{Stealth}, very thick] (lrc) -- (labour);

\end{tikzpicture}
\caption{شکل‌گیری حزب کارگر}
\end{figure}

\begin{policybox}[ویژگی‌های خاص سوسیالیسم بریتانیایی]
سوسیالیسم بریتانیایی با مدل‌های اروپایی تفاوت‌های مهمی داشت:
\begin{enumerate}
    \item \textbf{غیرمارکسیستی:} تأکید بر اصلاحات تدریجی، نه انقلاب
    \item \textbf{اتحادیه‌محور:} ریشه در اتحادیه‌ها، نه روشنفکران
    \item \textbf{پراگماتیست:} «فابیانیسم» -- اصلاحات گام‌به‌گام
    \item \textbf{اخلاق‌گرا:} تأثیر متدیسم و مسیحیت اجتماعی
    \item \textbf{پارلمانی:} تعهد به دموکراسی لیبرال
\end{enumerate}
\end{policybox}

\begin{table}[H]
\centering
\caption{مقایسه جنبش‌های کارگری اروپایی (حدود ۱۹۰۰)}
\renewcommand{\arraystretch}{1.3}
\begin{tabularx}{\textwidth}{>{\bfseries}lXXXX}
\toprule
\textbf{ویژگی} & \textbf{بریتانیا} & \textbf{آلمان} & \textbf{فرانسه} & \textbf{روسیه} \\
\midrule
\rowcolor{empirecream}
ایدئولوژی غالب & فابیانیسم & مارکسیسم اصلاح‌طلب & سندیکالیسم & بلشویسم \\
استراتژی & پارلمانی & پارلمانی + اتحادیه & اعتصاب عمومی & انقلابی \\
\rowcolor{empirecream}
رابطه با اتحادیه‌ها & یکپارچه & نزدیک & جدا & ضعیف \\
رابطه با دولت & سازش & نیمه‌قانونی & تقابلی & سرکوب \\
\rowcolor{empirecream}
عضویت حزبی (۱۹۱۰) & ۱.۵ میلیون & ۷۲۰,۰۰۰ & ۹۰,۰۰۰ & ۵۰,۰۰۰ (زیرزمینی) \\
\bottomrule
\end{tabularx}
\end{table}

%----------------------------------------------------------------------
\section{امپراتوری: ابزار و بار}
%----------------------------------------------------------------------

\subsection{گستره امپراتوری بریتانیا}

\begin{figure}[H]
\centering
\begin{tikzpicture}
% نقشه ساده‌شده مناطق امپراتوری
\begin{axis}[
    width=14cm,
    height=8cm,
    title={\textbf{رشد مساحت امپراتوری بریتانیا}},
    xlabel={سال},
    ylabel={مساحت (میلیون مایل مربع)},
    xmin=1800, xmax=1920,
    ymin=0, ymax=15,
    grid=major,
    area style,
]

\addplot+[fill=imperialred!40, draw=imperialred, very thick] coordinates {
    (1800, 1.5) (1820, 2.0) (1840, 4.0) (1860, 6.0)
    (1880, 8.0) (1900, 11.0) (1914, 13.0)
} \closedcycle;

\end{axis}
\end{tikzpicture}
\caption{گسترش مساحت امپراتوری (۱۸۰۰-۱۹۱۴)}
\end{figure}

\begin{table}[H]
\centering
\caption{مناطق اصلی امپراتوری بریتانیا (۱۹۱۴)}
\renewcommand{\arraystretch}{1.3}
\begin{tabularx}{\textwidth}{>{\bfseries}p{2.5cm}ccp{5cm}c}
\toprule
\textbf{منطقه/مستعمره} & \textbf{از سال} & \textbf{جمعیت (م)} & \textbf{اهمیت اقتصادی} & \textbf{نوع حکومت} \\
\midrule
\rowcolor{empirecream}
هند & ۱۸۵۸ & ۳۰۰ & پنبه، چای، بازار مصرف & مستعمره تاجی \\
کانادا & ۱۸۶۷ & ۸ & گندم، چوب، معدن & دومینیون \\
\rowcolor{empirecream}
استرالیا & ۱۹۰۱ & ۵ & پشم، طلا، گوشت & دومینیون \\
آفریقای جنوبی & ۱۹۱۰ & ۶ & طلا، الماس & دومینیون \\
\rowcolor{empirecream}
مصر & ۱۸۸۲ & ۱۲ & پنبه، کانال سوئز & حمایتی \\
نیجریه & ۱۹۰۰ & ۱۶ & روغن نخل، قلع & مستعمره \\
\rowcolor{empirecream}
مالایا & ۱۸۲۶+ & ۳ & کائوچو، قلع & مستعمره \\
\bottomrule
\end{tabularx}
\end{table}

\subsection{تئوری‌های امپریالیسم}

\begin{figure}[H]
\centering
\begin{tikzpicture}[
    theory/.style={rectangle, rounded corners=8pt, draw, thick,
                   minimum width=5.5cm, minimum height=3cm,
                   text centered, text width=5.3cm}
]

\node[theory, fill=imperialred!15, draw=imperialred] (econ) at (0,0) {
    \textbf{تئوری اقتصادی}\\[0.2cm]
    \footnotesize
    هابسون، لنین:\\
    جستجوی بازار، مواد خام،\\
    و فرصت سرمایه‌گذاری
};

\node[theory, fill=royalblue!15, draw=royalblue] (strat) at (6,0) {
    \textbf{تئوری استراتژیک}\\[0.2cm]
    \footnotesize
    رقابت قدرت‌ها،\\
    کنترل نقاط کلیدی،\\
    امنیت مسیرها
};

\node[theory, fill=victoriangreen!15, draw=victoriangreen] (cult) at (12,0) {
    \textbf{تئوری فرهنگی}\\[0.2cm]
    \footnotesize
    «بار سفیدپوست»،\\
    رسالت تمدنی،\\
    نژادپرستی علمی
};

\node[theory, fill=parliamentgold!15, draw=parliamentgold] (pol) at (3,-4) {
    \textbf{تئوری سیاست داخلی}\\[0.2cm]
    \footnotesize
    امپریالیسم اجتماعی:\\
    انحراف از مسائل داخلی،\\
    ملی‌گرایی توده‌ای
};

\node[theory, fill=tudorpurple!15, draw=tudorpurple] (periph) at (9,-4) {
    \textbf{تئوری پیرامونی}\\[0.2cm]
    \footnotesize
    رابینسون و گالاگر:\\
    همکاری نخبگان محلی،\\
    ضعف ساختارهای بومی
};

\end{tikzpicture}
\caption{تئوری‌های مختلف تبیین امپریالیسم بریتانیا}
\end{figure}

\subsubsection{دیدگاه جان هابسون (۱۹۰۲)}

\begin{historicalquote}
«امپریالیسم بد اقتصادی است... منافع ملی از طریق امپراتوری تأمین نمی‌شود، بلکه 
فقط منافع گروه‌های خاص -- سرمایه‌داران، تولیدکنندگان اسلحه، و مقامات استعماری -- 
تأمین می‌شود. اگر قدرت خرید طبقات کارگر در داخل افزایش یابد، نیازی به بازارهای 
خارجی نخواهد بود.»
\hfill --- جان هابسون، \textit{امپریالیسم: یک مطالعه}، ۱۹۰۲
\end{historicalquote}

\subsection{هزینه و فایده امپراتوری}

\begin{table}[H]
\centering
\caption{ترازنامه امپراتوری: منافع و هزینه‌ها}
\renewcommand{\arraystretch}{1.4}
\begin{tabularx}{\textwidth}{>{\bfseries\color{victoriangreen}}X>{\bfseries\color{imperialred}}X}
\toprule
\textbf{منافع} & \textbf{هزینه‌ها} \\
\midrule
\rowcolor{empirecream}
دسترسی به مواد خام ارزان & هزینه‌های نظامی و اداری \\
بازارهای محافظت‌شده برای صادرات & تنش با سایر قدرت‌ها \\
\rowcolor{empirecream}
فرصت سرمایه‌گذاری با بازده بالا & انحراف سرمایه از صنایع داخلی \\
نیروی کار ارزان & مقاومت‌های بومی و شورش‌ها \\
\rowcolor{empirecream}
پرستیژ ملی و غرور & بار اخلاقی و وجدانی \\
شغل برای طبقه متوسط & تقویت نظامی‌گری \\
\bottomrule
\end{tabularx}
\end{table}

\begin{warningbox}
\textbf{بحث تاریخ‌نگارانه درباره سودآوری امپراتوری:}

\begin{itemize}
    \item \textbf{دیویس و هاتنباخ (۱۹۸۶):} امپراتوری برای بریتانیا به‌طور 
    کلی زیان‌ده بود؛ هزینه‌های دفاعی بیش از منافع تجاری بود
    \item \textbf{اوبراین (۱۹۸۸):} منافع ناچیز بود؛ حداکثر ۱-۲٪ درآمد ملی
    \item \textbf{فرگوسن (۲۰۰۳):} امپراتوری کالای عمومی جهانی تولید کرد 
    (تجارت آزاد، حاکمیت قانون) اما بریتانیا هزینه‌اش را داد
    \item \textbf{آسرت و لیونتال (۲۰۱۰):} منافع نامتقارن بود؛ نخبگان سود 
    بردند، مالیات‌دهندگان عادی هزینه کردند
\end{itemize}
\end{warningbox}

\subsection{هند: «جواهر تاج»}

\begin{figure}[H]
\centering
\begin{tikzpicture}
\begin{axis}[
    ybar stacked,
    width=12cm,
    height=7cm,
    ylabel={میلیون پوند},
    symbolic x coords={صادرات بریتانیا به هند, واردات از هند, «زهکش» سالانه},
    xtick=data,
    ymin=0,
    bar width=40pt,
    legend style={at={(0.5,-0.2)}, anchor=north},
    nodes near coords,
    every node near coord/.append style={font=\footnotesize},
]

\addplot[fill=royalblue!60] coordinates {
    (صادرات بریتانیا به هند, 45)
    (واردات از هند, 35)
    (<<زهکش>> سالانه, 20)
};

\end{axis}
\end{tikzpicture}
\caption{تجارت بریتانیا-هند (متوسط سالانه دهه ۱۹۰۰)}
\end{figure}

\begin{policybox}[مفهوم «زهکش» (Drain)]
اقتصاددانان ملی‌گرای هندی مانند دادابهای نائوروجی مفهوم «زهکش» را مطرح کردند:
\begin{itemize}
    \item هند مجبور بود برای خدمات اداری، نظامی و بازنشستگی مقامات بریتانیایی 
    هزینه کند
    \item این انتقال یک‌طرفه ثروت سالانه حدود ۲۰-۳۰ میلیون پوند برآورد می‌شد
    \item معادل حدود ۱-۲٪ تولید ناخالص هند
\end{itemize}
\end{policybox}

%----------------------------------------------------------------------
\section{اصلاحات اجتماعی و تولد دولت رفاه اولیه}
%----------------------------------------------------------------------

\subsection{از لسه‌فر به مداخله‌گرایی}

\begin{figure}[H]
\centering
\begin{tikzpicture}[
    timeline/.style={very thick, royalblue},
    reform/.style={rectangle, rounded corners, draw, thick,
                   fill=victoriangreen!20, minimum height=1.2cm,
                   text centered, font=\footnotesize, text width=2.8cm}
]

% خط زمانی
\draw[timeline] (0,0) -- (16,0);

% اصلاحات
\node[reform] at (0.5,1.5) {قانون کارخانه\\۱۸۳۳\\کار کودکان};
\node[reform] at (3,1.5) {قانون بهداشت\\۱۸۴۸\\فاضلاب شهری};
\node[reform] at (5.5,1.5) {قانون آموزش\\۱۸۷۰\\مدارس ابتدایی};
\node[reform] at (8,1.5) {قانون بهداشت\\۱۸۷۵\\استاندارد مسکن};
\node[reform] at (10.5,1.5) {آموزش اجباری\\۱۸۸۰\\تا ۱۰ سالگی};
\node[reform] at (13,1.5) {قانون جبران\\۱۸۹۷\\حوادث کار};
\node[reform] at (15.5,1.5) {اصلاحات لیبرال\\۱۹۰۶-۱۴\\دولت رفاه};

% تاریخ‌ها
\foreach \x/\year in {0/1833, 2.5/1848, 5/1870, 7.5/1875, 10/1880, 12.5/1897, 15/1906} {
    \draw[thick] (\x,0.2) -- (\x,-0.2);
    \node[below, font=\tiny] at (\x,-0.3) {\year};
}

\end{tikzpicture}
\caption{مسیر اصلاحات اجتماعی ویکتوریایی}
\end{figure}

\subsection{قوانین کارخانه}

\begin{table}[H]
\centering
\caption{تکامل قوانین کارخانه}
\renewcommand{\arraystretch}{1.3}
\begin{tabularx}{\textwidth}{>{\bfseries}cXc}
\toprule
\textbf{سال} & \textbf{مفاد اصلی} & \textbf{گستره} \\
\midrule
\rowcolor{empirecream}
۱۸۳۳ & ممنوعیت کار کودکان زیر ۹ سال؛ محدودیت ساعت کار ۹-۱۳ ساله‌ها؛ 
بازرسان کارخانه & نساجی \\

۱۸۴۴ & محدودیت کار زنان به ۱۲ ساعت؛ حفاظ ماشین‌آلات & نساجی \\

\rowcolor{empirecream}
۱۸۴۷ & قانون ده ساعت برای زنان و نوجوانان & نساجی \\

۱۸۶۷ & گسترش به کارگاه‌های کوچک & صنایع گوناگون \\

\rowcolor{empirecream}
۱۸۷۸ & تلفیق قوانین قبلی؛ استاندارد سن و ساعت & همه کارخانه‌ها \\

۱۹۰۱ & حداقل سن ۱۲ سال؛ محدودیت‌های بیشتر & همه کارگاه‌ها \\

\bottomrule
\end{tabularx}
\end{table}

\subsection{اصلاحات لیبرال (۱۹۰۶-۱۹۱۴)}

\begin{keybox}[انقلاب رفاهی لیبرال‌ها]
دولت لیبرال به رهبری کمپبل-بنرمن و سپس آسکوئیث، با وزرایی چون 
لوید جورج و چرچیل، مجموعه‌ای از اصلاحات را اجرا کرد که بنیان 
دولت رفاه مدرن را گذاشت.
\end{keybox}

\begin{figure}[H]
\centering
\begin{tikzpicture}[
    reform/.style={rectangle, rounded corners=8pt, draw=royalblue, thick,
                   fill=royalblue!15, minimum width=4.2cm, minimum height=2cm,
                   text centered, text width=4cm, font=\small}
]

% ردیف اول
\node[reform] (r1) at (0,3) {
    \textbf{کودکان (۱۹۰۶-۸)}\\
    غذای رایگان مدارس\\
    معاینه پزشکی
};

\node[reform] (r2) at (5,3) {
    \textbf{سالمندان (۱۹۰۸)}\\
    مستمری غیرمشارکتی\\
    ۵ شیلینگ/هفته
};

\node[reform] (r3) at (10,3) {
    \textbf{کارگران (۱۹۰۹)}\\
    هیئت‌های دستمزد\\
    حداقل دستمزد در صنایع
};

% ردیف دوم
\node[reform] (r4) at (2.5,0) {
    \textbf{بیمه ملی (۱۹۱۱)}\\
    بیمه بیماری\\
    بیمه بیکاری
};

\node[reform] (r5) at (7.5,0) {
    \textbf{بودجه مردم (۱۹۰۹)}\\
    مالیات بر ثروت\\
    مالیات بر زمین
};

\end{tikzpicture}
\caption{اصلاحات اصلی دولت لیبرال}
\end{figure}

\subsubsection{قانون بیمه ملی ۱۹۱۱}

\begin{table}[H]
\centering
\caption{ساختار بیمه ملی ۱۹۱۱}
\renewcommand{\arraystretch}{1.3}
\begin{tabularx}{\textwidth}{>{\bfseries}p{3cm}XX}
\toprule
& \textbf{بخش اول: بیمه بیماری} & \textbf{بخش دوم: بیمه بیکاری} \\
\midrule
\rowcolor{empirecream}
پوشش & کارگران با درآمد زیر ۱۶۰ پوند/سال & صنایع خاص (ساختمان، کشتی‌سازی، مهندسی) \\

حق بیمه هفتگی & ۴ پنی کارگر، ۳ پنی کارفرما، ۲ پنی دولت & ۲.۵ پنی هر طرف \\

\rowcolor{empirecream}
مزایا & ۱۰ شیلینگ/هفته (مردان)، درمان پزشک عمومی & ۷ شیلینگ/هفته تا ۱۵ هفته \\

تعداد بیمه‌شدگان & ۱۵ میلیون نفر & ۲.۵ میلیون نفر \\

\bottomrule
\end{tabularx}
\end{table}

\begin{historicalquote}
«نه پنس به ازای چهار پنس!»
\hfill --- شعار تبلیغاتی لوید جورج برای بیمه ملی
\end{historicalquote}

\subsubsection{بودجه مردم (۱۹۰۹) و بحران قانون اساسی}

\begin{figure}[H]
\centering
\begin{tikzpicture}[
    node distance=1.5cm,
    event/.style={rectangle, rounded corners, draw, thick,
                  minimum width=3.5cm, minimum height=1.5cm,
                  text centered, text width=3.3cm, font=\small},
    arrow/.style={-{Stealth[length=3mm]}, thick}
]

\node[event, fill=victoriangreen!20] (budget) {بودجه مردم\\لوید جورج\\آوریل ۱۹۰۹};
\node[event, fill=imperialred!20, right=of budget] (lords) {رد توسط\\مجلس اعیان\\نوامبر ۱۹۰۹};
\node[event, fill=royalblue!20, right=of lords] (elect1) {انتخابات\\ژانویه ۱۹۱۰\\پیروزی لیبرال‌ها};
\node[event, fill=parliamentgold!20, below=of budget] (pass) {تصویب بودجه\\آوریل ۱۹۱۰};
\node[event, fill=tudorpurple!20, below=of lords] (bill) {لایحه پارلمان\\محدودیت اعیان\\۱۹۱۰-۱۱};
\node[event, fill=victoriangreen!30, below=of elect1] (act) {قانون پارلمان\\اوت ۱۹۱۱\\اعیان فقط تأخیری};

\draw[arrow] (budget) -- (lords);
\draw[arrow] (lords) -- (elect1);
\draw[arrow] (elect1) -- (act);
\draw[arrow] (budget) -- (pass);
\draw[arrow] (lords) -- (bill);
\draw[arrow] (bill) -- (act);

\end{tikzpicture}
\caption{بحران قانون اساسی ۱۹۰۹-۱۹۱۱}
\end{figure}

\begin{policybox}[قانون پارلمان ۱۹۱۱: پایان حق وتوی اعیان]
این قانون نقطه عطفی در تاریخ قانون اساسی بریتانیا بود:
\begin{enumerate}
    \item \textbf{لوایح مالی:} اعیان نمی‌توانست لوایح مالی را رد یا اصلاح کند
    \item \textbf{سایر لوایح:} اعیان فقط می‌توانست تا ۲ سال به تأخیر بیندازد 
    (بعداً در ۱۹۴۹ به ۱ سال کاهش یافت)
    \item \textbf{دوره پارلمان:} از ۷ سال به ۵ سال کاهش یافت
\end{enumerate}
این قانون اصل برتری عوام بر اعیان را نهادینه کرد.
\end{policybox}

\subsection{چرا اصلاحات لیبرال؟}

\begin{figure}[H]
\centering
\begin{tikzpicture}[
    factor/.style={rectangle, rounded corners, draw, thick,
                   minimum width=4.5cm, minimum height=1.5cm,
                   text centered, text width=4.3cm},
    arrow/.style={-{Stealth[length=3mm]}, thick}
]

\node[factor, fill=imperialred!15] (f1) at (0,4) {
    ترس از سوسیالیسم\\
    رشد حزب کارگر
};

\node[factor, fill=royalblue!15] (f2) at (5.5,4) {
    رقابت با آلمان\\
    بیمه بیسمارکی
};

\node[factor, fill=victoriangreen!15] (f3) at (11,4) {
    تحقیقات فقر\\
    بوث، راونتری
};

\node[factor, fill=parliamentgold!15] (f4) at (2.75,1.5) {
    کارایی امپراتوری\\
    شکست جنگ بوئر
};

\node[factor, fill=tudorpurple!15] (f5) at (8.25,1.5) {
    لیبرالیسم نو\\
    هابهاوس، گرین
};

\node[rectangle, rounded corners=10pt, draw=royalblue, very thick,
      fill=royalblue!20, minimum width=8cm, minimum height=1.5cm,
      text centered] (outcome) at (5.5,-1.5) {
    \textbf{اصلاحات لیبرال ۱۹۰۶-۱۹۱۴}
};

\draw[arrow] (f1) -- (outcome);
\draw[arrow] (f2) -- (outcome);
\draw[arrow] (f3) -- (outcome);
\draw[arrow] (f4) -- (outcome);
\draw[arrow] (f5) -- (outcome);

\end{tikzpicture}
\caption{عوامل اصلاحات لیبرال}
\end{figure}

\subsubsection{تحقیقات فقر: بوث و راونتری}

\begin{table}[H]
\centering
\caption{یافته‌های کلیدی تحقیقات فقر}
\renewcommand{\arraystretch}{1.3}
\begin{tabularx}{\textwidth}{>{\bfseries}p{3cm}p{5cm}X}
\toprule
\textbf{محقق} & \textbf{یافته اصلی} & \textbf{تأثیر} \\
\midrule
\rowcolor{empirecream}
چارلز بوث (۱۸۸۹-۱۹۰۳) & ۳۰.۷٪ مردم لندن زیر خط فقر زندگی می‌کنند & نقض ادعای «فقر = تنبلی» \\

سیبم راونتری (۱۹۰۱) & ۲۷.۸۴٪ مردم یورک در فقر؛ «چرخه فقر» در طول زندگی & نشان داد فقر ساختاری است \\

\bottomrule
\end{tabularx}
\end{table}

%----------------------------------------------------------------------
\section{مسئله ایرلند}
%----------------------------------------------------------------------

\subsection{پیشینه}

\begin{figure}[H]
\centering
\begin{tikzpicture}[
    timeline/.style={very thick, victoriangreen},
    event/.style={rectangle, rounded corners, draw, thick,
                  minimum height=1cm, text centered, font=\footnotesize}
]

\draw[timeline] (0,0) -- (14,0);

\node[event, fill=imperialred!20, text width=2cm] at (1,1.5) {قحطی بزرگ\\۱۸۴۵-۵۲};
\node[event, fill=tudorpurple!20, text width=2cm] at (4,1.5) {جنبش فنیان\\۱۸۵۸-};
\node[event, fill=victoriangreen!20, text width=2.2cm] at (7,1.5) {اصلاحات ارضی\\۱۸۷۰-۱۹۰۳};
\node[event, fill=royalblue!20, text width=2.2cm] at (10,1.5) {Home Rule\\۱۸۸۶، ۱۸۹۳};
\node[event, fill=parliamentgold!20, text width=2cm] at (13,1.5) {بحران ۱۹۱۲-۱۴};

\foreach \x/\year in {0/1845, 3/1858, 6/1870, 9/1886, 12/1912, 14/1914} {
    \draw[thick] (\x,0.2) -- (\x,-0.2);
    \node[below, font=\tiny] at (\x,-0.3) {\year};
}

\end{tikzpicture}
\caption{نقاط عطف مسئله ایرلند}
\end{figure}

\subsection{قحطی بزرگ و پیامدهای آن}

\begin{table}[H]
\centering
\caption{تأثیر قحطی بزرگ بر جمعیت ایرلند}
\renewcommand{\arraystretch}{1.3}
\begin{tabular}{lcccc}
\toprule
\textbf{شاخص} & \textbf{۱۸۴۱} & \textbf{۱۸۵۱} & \textbf{۱۹۰۱} & \textbf{۱۹۱۱} \\
\midrule
\rowcolor{empirecream}
جمعیت ایرلند (میلیون) & ۸.۲ & ۶.۶ & ۴.۴ & ۴.۴ \\
مرگ‌ومیر از قحطی & --- & $\sim$۱ میلیون & --- & --- \\
\rowcolor{empirecream}
مهاجرت تجمعی (از ۱۸۴۵) & --- & ۱.۵ میلیون & ۴ میلیون & ۴.۵ میلیون \\
\bottomrule
\end{tabular}
\end{table}

\begin{warningbox}
\textbf{میراث قحطی:}
\begin{itemize}
    \item نسل‌کشی یا اهمال جنایتکارانه؟ بحث تاریخ‌نگارانه ادامه دارد
    \item ایجاد «دیاسپورای ایرلندی» به‌ویژه در آمریکا
    \item تقویت ناسیونالیسم ایرلندی و نفرت از بریتانیا
    \item تأسیس IRB (جمهوری‌خواهان ایرلندی) ۱۸۵۸
\end{itemize}
\end{warningbox}

\subsection{لوایح خودمختاری (Home Rule)}

\begin{table}[H]
\centering
\caption{سه لایحه خودمختاری ایرلند}
\renewcommand{\arraystretch}{1.4}
\begin{tabularx}{\textwidth}{>{\bfseries}cXcc}
\toprule
\textbf{سال} & \textbf{ویژگی‌ها} & \textbf{نتیجه در عوام} & \textbf{نتیجه در اعیان} \\
\midrule
\rowcolor{empirecream}
۱۸۸۶ & پارلمان ایرلندی با اختیارات محدود & رد شد (۳۴۱-۳۱۱) & --- \\

۱۸۹۳ & مشابه ۱۸۸۶ با اصلاحات & تصویب (۳۰۱-۲۶۷) & رد شد (۴۱۹-۴۱) \\

\rowcolor{empirecream}
۱۹۱۲ & پس از قانون پارلمان ۱۹۱۱ & تصویب & رد، اما فقط تأخیری \\

\bottomrule
\end{tabularx}
\end{table}

\subsubsection{بحران ۱۹۱۲-۱۹۱۴}

\begin{figure}[H]
\centering
\begin{tikzpicture}[
    node distance=1cm,
    group/.style={rectangle, rounded corners=8pt, draw, very thick,
                  minimum width=5cm, minimum height=3cm,
                  text centered, text width=4.8cm}
]

\node[group, fill=victoriangreen!20, draw=victoriangreen] (nat) at (0,0) {
    \textbf{ناسیونالیست‌های ایرلندی}\\[0.3cm]
    حزب پارلمانی ایرلند\\
    (رِدموند)\\[0.2cm]
    داوطلبان ایرلندی\\
    ۱۸۰,۰۰۰ نفر
};

\node[group, fill=imperialred!20, draw=imperialred] (uni) at (7,0) {
    \textbf{اتحادگرایان اولستر}\\[0.3cm]
    نیروی داوطلب اولستر\\
    (UVF)\\۱۰۰,۰۰۰ نفر\\[0.2cm]
    قاچاق اسلحه\\
    تهدید به جنگ داخلی
};

\node[group, fill=royalblue!15, draw=royalblue] at (3.5,-4) {
    \textbf{دولت بریتانیا}\\[0.3cm]
    گیر افتاده بین دو طرف\\
    شورش افسران (Curragh)\\
    اوت ۱۹۱۴: جنگ جهانی\\
    خودمختاری به تعویق
};

\draw[{Stealth}-{Stealth}, very thick, imperialred] (nat) -- (uni) 
    node[midway, above] {تقابل};

\end{tikzpicture}
\caption{بحران ایرلند ۱۹۱۲-۱۹۱۴}
\end{figure}

%----------------------------------------------------------------------
\section{جنبش سافرجت: مبارزه برای حق رأی زنان}
%----------------------------------------------------------------------

\subsection{دو استراتژی}

\begin{table}[H]
\centering
\caption{مقایسه دو بال جنبش حق رأی زنان}
\renewcommand{\arraystretch}{1.4}
\begin{tabularx}{\textwidth}{>{\bfseries}p{3cm}XX}
\toprule
& \textbf{سافرجیست‌ها (NUWSS)} & \textbf{سافرجت‌ها (WSPU)} \\
\midrule
\rowcolor{empirecream}
رهبری & میلیسنت فاست & امیلین پنکهرست و دختران \\
تأسیس & ۱۸۹۷ & ۱۹۰۳ \\
\rowcolor{empirecream}
شعار & «قانونی و آرام» & «عمل، نه حرف» \\
تاکتیک‌ها & لابی، طومار، تظاهرات مسالمت‌آمیز & شکستن شیشه، آتش‌سوزی، اعتصاب غذا \\
\rowcolor{empirecream}
عضویت (۱۹۱۴) & $\sim$۵۰,۰۰۰ & $\sim$۲,۰۰۰ (فعال) \\
\bottomrule
\end{tabularx}
\end{table}

\begin{historicalquote}
«ما زنان سافرجت صبورانه تلاش کردیم... اما صبر ما دیگر به پایان رسیده است. 
سنگ‌پرانی ما پژواک سنگ‌پرانی مردان در ۱۸۳۲ است.»
\hfill --- امیلین پنکهرست، ۱۹۱۲
\end{historicalquote}

\subsection{چرا زنان تا ۱۹۱۸ حق رأی نگرفتند؟}

\begin{figure}[H]
\centering
\begin{tikzpicture}[
    obstacle/.style={rectangle, rounded corners, draw=imperialred, thick,
                     fill=imperialred!15, minimum width=4cm,
                     minimum height=1.5cm, text centered, text width=3.8cm}
]

\node[obstacle] at (0,3) {ایدئولوژی «حوزه‌های جدا»\\زن: خانه، مرد: عرصه عمومی};
\node[obstacle] at (5,3) {ترس از رادیکالیسم\\اگر زنان رأی دهند...};
\node[obstacle] at (10,3) {مخالفت درون‌حزبی\\حتی در حزب لیبرال};
\node[obstacle] at (2.5,0) {نگرانی از تأثیر\\بر نتایج انتخابات};
\node[obstacle] at (7.5,0) {مخالفت ملکه ویکتوریا\\و بسیاری زنان طبقه بالا};

\end{tikzpicture}
\caption{موانع حق رأی زنان}
\end{figure}

%----------------------------------------------------------------------
\section{پایان یک عصر: بحران‌های ۱۹۱۰-۱۹۱۴}
%----------------------------------------------------------------------

\begin{warningbox}
\textbf{بریتانیا در آستانه ۱۹۱۴ با چهار بحران همزمان روبرو بود:}
\begin{enumerate}
    \item \textbf{بحران ایرلند:} تهدید جنگ داخلی
    \item \textbf{بحران کارگری:} موج اعتصابات ۱۹۱۱-۱۹۱۴
    \item \textbf{بحران سافرجت‌ها:} شورشگری فزاینده
    \item \textbf{بحران بین‌المللی:} رقابت با آلمان
\end{enumerate}
جنگ جهانی اول همه این بحران‌ها را موقتاً منجمد کرد.
\end{warningbox}

\subsection{جدول زمانی بحران‌ها}

\begin{figure}[H]
\centering
\begin{tikzpicture}
\begin{axis}[
    width=14cm,
    height=6cm,
    xlabel={سال},
    ylabel={تعداد روزهای از دست رفته (میلیون)},
    xmin=1900, xmax=1915,
    ymin=0, ymax=45,
    grid=major,
    title={\textbf{اعتصابات کارگری ۱۹۰۰-۱۹۱۴}},
]

\addplot[very thick, imperialred, mark=*] coordinates {
    (1900, 3) (1901, 4) (1902, 3) (1903, 2) (1904, 1.5)
    (1905, 2) (1906, 3) (1907, 2) (1908, 11) (1909, 3)
    (1910, 10) (1911, 10) (1912, 41) (1913, 10) (1914, 10)
};

\end{axis}
\end{tikzpicture}
\caption{روزهای کاری از دست رفته به علت اعتصاب}
\end{figure}

%----------------------------------------------------------------------
\section{تحلیل نهایی: میراث عصر ویکتوریایی}
%----------------------------------------------------------------------

\subsection{دستاوردها و محدودیت‌ها}

\begin{table}[H]
\centering
\caption{ترازنامه عصر ویکتوریایی}
\renewcommand{\arraystretch}{1.4}
\begin{tabularx}{\textwidth}{>{\bfseries\color{victoriangreen}}X>{\bfseries\color{imperialred}}X}
\toprule
\textbf{دستاوردها} & \textbf{محدودیت‌ها و شکست‌ها} \\
\midrule
\rowcolor{empirecream}
گسترش حق رأی از ۵٪ به ۶۰٪ مردان & زنان همچنان محروم \\
ایجاد زیرساخت دولت رفاه & ناکافی در مقابل فقر گسترده \\
\rowcolor{empirecream}
آزادی مطبوعات و اجتماعات & سرکوب جنبش‌های رادیکال \\
اتحادیه‌های کارگری قانونی & نابرابری طبقاتی شدید \\
\rowcolor{empirecream}
پیشرفت‌های بهداشتی و آموزشی & تفاوت‌های منطقه‌ای و طبقاتی \\
ثبات سیاسی بدون انقلاب & به قیمت تدریجی‌گری محافظه‌کارانه \\
\rowcolor{empirecream}
گسترش امپراتوری & استثمار مستعمرات \\
\bottomrule
\end{tabularx}
\end{table}

\subsection{الگوهای قابل استخراج}

\begin{policybox}[الگوی ۱: دموکراتیزاسیون مدیریت‌شده]
نخبگان بریتانیایی آموختند که می‌توان با دادن امتیازات به‌موقع، از انقلاب 
جلوگیری کرد. این «مهندسی سیاسی» شامل موارد زیر بود:
\begin{itemize}
    \item شناسایی گروه‌های «قابل اعتماد» برای شمول (کارگران ماهر، پرهیزکار)
    \item حفظ شرایط مالکیت یا اقامت برای فیلتر کردن «نامطلوبان»
    \item ایجاد احساس شراکت در نظام
\end{itemize}
\end{policybox}

\begin{policybox}[الگوی ۲: جذب طبقه کارگر]
به‌جای سرکوب مطلق، نظام بریتانیایی طبقه کارگر را تدریجاً جذب کرد:
\begin{itemize}
    \item قانونی کردن اتحادیه‌ها
    \item دادن حق رأی (محدود)
    \item اصلاحات اجتماعی (بیمه، مستمری)
    \item ایجاد فضا برای حزب کارگر در نظام سیاسی
\end{itemize}
این استراتژی «جذب و ادغام» از رادیکالیزه شدن جلوگیری کرد.
\end{policybox}

\begin{policybox}[الگوی ۳: امپریالیسم به‌مثابه والو فشار]
امپراتوری چند کارکرد داخلی داشت:
\begin{itemize}
    \item ایجاد شغل برای طبقه متوسط (اداری، نظامی)
    \item ایجاد احساس برتری در طبقه کارگر («حداقل ما بریتانیایی هستیم»)
    \item انحراف توجه از مسائل داخلی
    \item ایجاد بازار برای صنایع در حال افول
\end{itemize}
\end{policybox}

%----------------------------------------------------------------------
\section*{منابع اصلی فصل پنجم}
%----------------------------------------------------------------------

\begin{enumerate}[label={[\arabic*]}]
    \item Thompson, E.P. (1963). \textit{The Making of the English Working Class}. London: Gollancz.
    \item Hobsbawm, Eric (1968). \textit{Industry and Empire}. London: Weidenfeld \& Nicolson.
    \item Cain, P.J. \& Hopkins, A.G. (2001). \textit{British Imperialism, 1688-2000}. 2nd ed. London: Longman.
    \item Harris, Jose (1993). \textit{Private Lives, Public Spirit: Britain 1870-1914}. Oxford: OUP.
    \item Pugh, Martin (1999). \textit{State and Society: A Social and Political History of Britain 1870-1997}. 2nd ed. London: Arnold.
    \item Searle, G.R. (2004). \textit{A New England? Peace and War 1886-1918}. Oxford: Clarendon Press.
    \item Davis, Lance E. \& Huttenback, Robert A. (1986). \textit{Mammon and the Pursuit of Empire}. Cambridge: CUP.
    \item Booth, Charles (1889-1903). \textit{Life and Labour of the People in London}. 17 vols. London: Macmillan.
    \item Rowntree, B. Seebohm (1901). \textit{Poverty: A Study of Town Life}. London: Macmillan.
    \item Hobson, J.A. (1902). \textit{Imperialism: A Study}. London: James Nisbet.
\end{enumerate}


%%%%%%%%%%%%%%%%%%%%%%%%%%%%%%%%%%%%%%%%%%%%%%%%%%%%%%%%%%%%%%%%%%%%%%%
% فصل ششم: جنگ‌ها، بحران‌ها و دولت رفاه (۱۹۱۴-۱۹۷۹)
%%%%%%%%%%%%%%%%%%%%%%%%%%%%%%%%%%%%%%%%%%%%%%%%%%%%%%%%%%%%%%%%%%%%%%%

\chapter{جنگ‌ها، بحران‌ها و دولت رفاه (۱۹۱۴-۱۹۷۹)}

\begin{keybox}[خلاصه فصل]
این فصل به تحلیل یکی از پرتلاطم‌ترین دوره‌های تاریخ بریتانیا می‌پردازد: 
دوره‌ای که شامل \textbf{دو جنگ جهانی}، \textbf{فروپاشی امپراتوری}، 
\textbf{رکود بزرگ}، و \textbf{ساختن و گسترش دولت رفاه} می‌شود. جنگ‌ها 
به‌عنوان «شتاب‌دهنده‌های تاریخ» عمل کردند: حق رأی همگانی، دولت مداخله‌گر، 
و اجماع اجتماعی همه محصول شرایط جنگی بودند. این دوره با «زمستان نارضایتی» 
۱۹۷۸-۷۹ به پایان رسید و راه را برای انقلاب تاچری باز کرد.
\end{keybox}

%----------------------------------------------------------------------
\section{ساختار تحلیلی فصل}
%----------------------------------------------------------------------

\begin{figure}[H]
\centering
\begin{tikzpicture}[
    period/.style={rectangle, rounded corners=10pt, draw, very thick,
                   minimum width=4cm, minimum height=2.5cm,
                   text centered, text width=3.8cm}
]

\node[period, fill=imperialred!25] (ww1) at (0,0) {
    \textbf{۱۹۱۴-۱۹۱۸}\\
    جنگ جهانی اول\\
    دولت جنگی\\
    حق رأی همگانی
};

\node[period, fill=industrialgray!25] (inter) at (4.5,0) {
    \textbf{۱۹۱۸-۱۹۳۹}\\
    بین دو جنگ\\
    رکود، بیکاری\\
    ظهور کارگر
};

\node[period, fill=royalblue!25] (ww2) at (9,0) {
    \textbf{۱۹۳۹-۱۹۴۵}\\
    جنگ جهانی دوم\\
    «جنگ مردم»\\
    گزارش بوریج
};

\node[period, fill=victoriangreen!25] (cons) at (13.5,0) {
    \textbf{۱۹۴۵-۱۹۷۹}\\
    اجماع پس از جنگ\\
    دولت رفاه\\
    بحران و فروپاشی
};

\draw[-{Stealth}, very thick] (ww1) -- (inter);
\draw[-{Stealth}, very thick] (inter) -- (ww2);
\draw[-{Stealth}, very thick] (ww2) -- (cons);

\end{tikzpicture}
\caption{دوره‌بندی فصل ششم}
\end{figure}

%----------------------------------------------------------------------
\section{جنگ جهانی اول و تحولات آن (۱۹۱۴-۱۹۱۸)}
%----------------------------------------------------------------------

\subsection{جنگ به‌مثابه شتاب‌دهنده تاریخ}

\begin{historicalquote}
«جنگ مادر همه چیزهاست.»
\hfill --- هراکلیتوس (تفسیر شده برای قرن بیستم)
\end{historicalquote}

جنگ جهانی اول تحولاتی را که ممکن بود دهه‌ها طول بکشد، در چهار سال فشرده کرد:

\begin{table}[H]
\centering
\caption{تأثیرات کلیدی جنگ جهانی اول بر بریتانیا}
\renewcommand{\arraystretch}{1.4}
\begin{tabularx}{\textwidth}{>{\bfseries}p{3cm}XX}
\toprule
\textbf{حوزه} & \textbf{قبل از جنگ} & \textbf{بعد از جنگ} \\
\midrule
\rowcolor{empirecream}
حق رأی & ۶۰٪ مردان، ۰٪ زنان & ۱۰۰٪ مردان ۲۱+، زنان ۳۰+ \\
نقش دولت & حداقلی (۸٪ GDP) & مداخله‌گر (۲۰٪+ GDP) \\
\rowcolor{empirecream}
اتحادیه‌ها & ۴ میلیون عضو & ۸ میلیون عضو \\
زنان در اشتغال & محدود به خدمات & صنایع، حمل‌ونقل، دفاتر \\
\rowcolor{empirecream}
حزب کارگر & حزب سوم & جایگزین لیبرال‌ها \\
امپراتوری & در اوج & آغاز فروپاشی \\
\rowcolor{empirecream}
بدهی ملی & ۶۵۰ میلیون پوند & ۷.۵ میلیارد پوند \\
\bottomrule
\end{tabularx}
\end{table}

\subsection{آمار جنگ}

\begin{figure}[H]
\centering
\begin{tikzpicture}
\begin{axis}[
    ybar,
    width=14cm,
    height=8cm,
    ylabel={تعداد (هزار نفر)},
    symbolic x coords={بسیج‌شدگان, کشته‌ها, زخمی‌ها, اسیران, مفقودان},
    xtick=data,
    ymin=0,
    ymax=6500,
    bar width=35pt,
    nodes near coords,
    every node near coord/.append style={font=\small},
    title={\textbf{تلفات انسانی بریتانیا در جنگ جهانی اول}},
]

\addplot[fill=imperialred!60] coordinates {
    (بسیج‌شدگان, 6000)
    (کشته‌ها, 886)
    (زخمی‌ها, 1665)
    (اسیران, 192)
    (مفقودان, 109)
};

\end{axis}
\end{tikzpicture}
\caption{آمار نیروهای بریتانیایی در جنگ جهانی اول}
\end{figure}

\subsection{دولت جنگی: تولد دولت مداخله‌گر}

\begin{figure}[H]
\centering
\begin{tikzpicture}[
    measure/.style={rectangle, rounded corners, draw=royalblue, thick,
                    fill=royalblue!15, minimum width=4.5cm,
                    minimum height=1.8cm, text centered, text width=4.3cm, font=\small}
]

% ردیف اول
\node[measure] at (0,3) {
    \textbf{قانون دفاع از قلمرو}\\
    (DORA) ۱۹۱۴\\
    اختیارات فوق‌العاده
};

\node[measure] at (5,3) {
    \textbf{کنترل صنایع کلیدی}\\
    راه‌آهن، معادن، کشتی‌سازی\\
    مدیریت دولتی
};

\node[measure] at (10,3) {
    \textbf{وزارت مهمات}\\
    ۱۹۱۵ (لوید جورج)\\
    تولید انبوه
};

% ردیف دوم
\node[measure] at (0,0) {
    \textbf{سربازگیری اجباری}\\
    ۱۹۱۶\\
    اولین بار در تاریخ
};

\node[measure] at (5,0) {
    \textbf{جیره‌بندی}\\
    ۱۹۱۸\\
    غذا، سوخت
};

\node[measure] at (10,0) {
    \textbf{کنترل اجاره}\\
    ۱۹۱۵\\
    پس از اعتصاب گلاسکو
};

\end{tikzpicture}
\caption{اقدامات دولت جنگی}
\end{figure}

\begin{policybox}[میراث دولت جنگی]
جنگ اثبات کرد که دولت \textbf{می‌تواند} اقتصاد را مدیریت کند:
\begin{itemize}
    \item تولید صنعتی با مدیریت دولتی افزایش یافت
    \item جیره‌بندی توزیع عادلانه‌تری ایجاد کرد
    \item بیکاری عملاً از بین رفت
\end{itemize}
این تجربه پایه فکری برای سوسیالیسم دموکراتیک و دولت رفاه شد.
\end{policybox}

\subsection{زنان و جنگ}

\begin{table}[H]
\centering
\caption{اشتغال زنان قبل و بعد از جنگ}
\renewcommand{\arraystretch}{1.3}
\begin{tabular}{lccc}
\toprule
\textbf{بخش} & \textbf{ژوئیه ۱۹۱۴} & \textbf{ژوئیه ۱۹۱۸} & \textbf{تغییر} \\
\midrule
\rowcolor{empirecream}
صنایع فلزی و شیمیایی & ۱۷۰,۰۰۰ & ۵۹۴,۰۰۰ & +۲۵۰٪ \\
حمل‌ونقل & ۱۸,۰۰۰ & ۱۱۷,۰۰۰ & +۵۵۰٪ \\
\rowcolor{empirecream}
کشاورزی & ۸۰,۰۰۰ & ۲۲۸,۰۰۰ & +۱۸۵٪ \\
تجارت و امور مالی & ۵۰۵,۰۰۰ & ۹۳۴,۰۰۰ & +۸۵٪ \\
\rowcolor{empirecream}
خدمات دولتی & ۲۶۲,۰۰۰ & ۴۶۰,۰۰۰ & +۷۵٪ \\
\midrule
\textbf{کل شاغلین زن} & ۳.۲ میلیون & ۴.۹ میلیون & +۵۳٪ \\
\bottomrule
\end{tabular}
\end{table}

\subsection{قانون نمایندگی مردم ۱۹۱۸}

\begin{keybox}[دموکراسی کامل (تقریباً)]
قانون ۱۹۱۸ بزرگ‌ترین گسترش یکباره حق رأی در تاریخ بریتانیا بود:
\begin{itemize}
    \item \textbf{مردان:} همه مردان ۲۱ ساله و بالاتر (بدون شرط مالکیت)
    \item \textbf{زنان:} زنان ۳۰ ساله و بالاتر با شرط مالکیت یا ازدواج با مالک
    \item \textbf{سربازان:} مردان ۱۹ ساله در خدمت نظامی
    \item \textbf{نتیجه:} از ۷.۷ میلیون به ۲۱.۴ میلیون رأی‌دهنده
\end{itemize}
\end{keybox}

\begin{figure}[H]
\centering
\begin{tikzpicture}
\begin{axis}[
    ybar,
    width=12cm,
    height=7cm,
    ylabel={میلیون رأی‌دهنده},
    symbolic x coords={۱۹۱۰, ۱۹۱۸, ۱۹۲۸},
    xtick=data,
    ymin=0,
    ymax=30,
    bar width=40pt,
    legend style={at={(0.5,-0.15)}, anchor=north, legend columns=2},
]

\addplot[fill=royalblue!60] coordinates {(۱۹۱۰, 7.7) (۱۹۱۸, 12.9) (۱۹۲۸, 14.0)};
\addplot[fill=tudorpurple!60] coordinates {(۱۹۱۰, 0) (۱۹۱۸, 8.5) (۱۹۲۸, 15.2)};

\legend{مردان, زنان}

\end{axis}
\end{tikzpicture}
\caption{گسترش حق رأی (۱۹۱۰-۱۹۲۸)}
\end{figure}

%----------------------------------------------------------------------
\section{دوره بین دو جنگ (۱۹۱۸-۱۹۳۹)}
%----------------------------------------------------------------------

\subsection{بحران اقتصادی و اجتماعی}

\begin{figure}[H]
\centering
\begin{tikzpicture}
\begin{axis}[
    width=14cm,
    height=8cm,
    xlabel={سال},
    ylabel={نرخ بیکاری (\%)},
    xmin=1918, xmax=1940,
    ymin=0, ymax=25,
    grid=major,
    legend pos=north east,
]

\addplot[very thick, imperialred, mark=*] coordinates {
    (1918, 1) (1920, 2) (1921, 17) (1922, 14) (1923, 11)
    (1924, 10) (1925, 11) (1926, 12) (1927, 10) (1928, 11)
    (1929, 10) (1930, 16) (1931, 21) (1932, 22) (1933, 20)
    (1934, 17) (1935, 15) (1936, 13) (1937, 11) (1938, 12)
    (1939, 10)
};
\addlegendentry{نرخ بیکاری بریتانیا}

% خط رکود بزرگ
\draw[dashed, thick, industrialgray] (axis cs:1929,0) -- (axis cs:1929,25);
\node[above, font=\footnotesize] at (axis cs:1929,23) {رکود بزرگ};

\end{axis}
\end{tikzpicture}
\caption{نرخ بیکاری در بریتانیا (۱۹۱۸-۱۹۳۹)}
\end{figure}

\subsubsection{بیکاری ساختاری: «دو بریتانیا»}

\begin{table}[H]
\centering
\caption{بیکاری منطقه‌ای (۱۹۳۴)}
\renewcommand{\arraystretch}{1.3}
\begin{tabular}{lcc}
\toprule
\textbf{منطقه} & \textbf{نرخ بیکاری (\%)} & \textbf{صنعت غالب} \\
\midrule
\rowcolor{imperialred!15}
ویلز جنوبی & ۳۶.۵ & زغال‌سنگ \\
\rowcolor{imperialred!15}
شمال شرق انگلستان & ۳۰.۹ & کشتی‌سازی \\
\rowcolor{imperialred!15}
اسکاتلند & ۲۷.۲ & فولاد، کشتی‌سازی \\
\rowcolor{imperialred!15}
شمال غرب انگلستان & ۲۵.۸ & نساجی \\
\midrule
\rowcolor{victoriangreen!15}
لندن و جنوب شرق & ۸.۶ & خدمات، صنایع نو \\
\rowcolor{victoriangreen!15}
میدلندز & ۱۲.۹ & خودروسازی، صنایع نو \\
\bottomrule
\end{tabular}
\end{table}

\begin{warningbox}
\textbf{«دو بریتانیا»:}

شمال و غرب (صنایع قدیمی: زغال‌سنگ، نساجی، کشتی‌سازی) در رکود دائمی بود، 
در حالی که جنوب و میدلندز (صنایع نو: خودرو، برق، شیمی) رونق داشت. 
این شکاف منطقه‌ای تا امروز ادامه دارد.
\end{warningbox}

\subsection{ظهور و افول حزب لیبرال}

\begin{figure}[H]
\centering
\begin{tikzpicture}
\begin{axis}[
    width=14cm,
    height=8cm,
    xlabel={انتخابات},
    ylabel={درصد آرا},
    xmin=1900, xmax=1935,
    ymin=0, ymax=55,
    grid=major,
    legend pos=outer north east,
    xtick={1900, 1906, 1910, 1918, 1922, 1924, 1929, 1931, 1935},
    x tick label style={rotate=45, anchor=east},
]

\addplot[very thick, royalblue, mark=*] coordinates {
    (1900, 45) (1906, 49) (1910, 44) (1918, 26) 
    (1922, 29) (1924, 18) (1929, 23) (1931, 7) (1935, 6)
};
\addlegendentry{لیبرال}

\addplot[very thick, imperialred, mark=square*] coordinates {
    (1900, 2) (1906, 5) (1910, 7) (1918, 21) 
    (1922, 30) (1924, 33) (1929, 37) (1931, 31) (1935, 38)
};
\addlegendentry{کارگر}

\addplot[very thick, tudorpurple, mark=triangle*] coordinates {
    (1900, 51) (1906, 43) (1910, 47) (1918, 38) 
    (1922, 38) (1924, 47) (1929, 38) (1931, 55) (1935, 54)
};
\addlegendentry{محافظه‌کار}

\end{axis}
\end{tikzpicture}
\caption{تحول نظام حزبی بریتانیا (۱۹۰۰-۱۹۳۵)}
\end{figure}

\subsubsection{چرا حزب لیبرال سقوط کرد؟}

\begin{table}[H]
\centering
\caption{عوامل سقوط حزب لیبرال}
\renewcommand{\arraystretch}{1.4}
\begin{tabularx}{\textwidth}{>{\bfseries\color{royalblue}}p{3.5cm}X}
\toprule
\textbf{عامل} & \textbf{توضیح} \\
\midrule
\rowcolor{empirecream}
انشعاب آسکوئیث-لوید جورج & جنگ حزب را به دو بخش تقسیم کرد که هرگز متحد نشدند \\

گسترش حق رأی & رأی‌دهندگان جدید طبقه کارگر به حزب کارگر رفتند \\

\rowcolor{empirecream}
ایدئولوژی ناکافی & لیبرالیسم کلاسیک پاسخی برای بیکاری انبوه نداشت \\

سیستم انتخاباتی & نظام اکثریتی به ضرر حزب سوم است \\

\rowcolor{empirecream}
ظهور طبقه به‌عنوان هویت & سیاست بر محور طبقه سازمان یافت، نه اصول لیبرالی \\

\bottomrule
\end{tabularx}
\end{table}

\subsection{اعتصاب عمومی ۱۹۲۶}

\begin{figure}[H]
\centering
\begin{tikzpicture}[
    node distance=1.2cm,
    event/.style={rectangle, rounded corners, draw, thick,
                  minimum width=4cm, minimum height=1.3cm,
                  text centered, text width=3.8cm, font=\small},
    arrow/.style={-{Stealth[length=3mm]}, thick}
]

\node[event, fill=industrialgray!20] (coal) {بحران صنعت زغال‌سنگ\\کاهش دستمزد، افزایش ساعت};
\node[event, fill=imperialred!20, right=of coal] (miners) {اعتصاب معدنچیان\\اول ماه مه ۱۹۲۶};
\node[event, fill=royalblue!20, right=of miners] (general) {اعتصاب عمومی\\۳-۱۲ مه\\۱.۷ میلیون کارگر};
\node[event, fill=parliamentgold!20, below=of coal] (govt) {مقاومت دولت بالدوین\\داوطلبان، ارتش};
\node[event, fill=victoriangreen!20, below=of miners] (tuc) {عقب‌نشینی TUC\\پس از ۹ روز};
\node[event, fill=imperialred!30, below=of general] (defeat) {شکست معدنچیان\\پس از ۷ ماه\\تسلیم بدون شرط};

\draw[arrow] (coal) -- (miners);
\draw[arrow] (miners) -- (general);
\draw[arrow] (general) -- (tuc);
\draw[arrow] (tuc) -- (defeat);
\draw[arrow] (govt) -- (tuc);

\end{tikzpicture}
\caption{مسیر اعتصاب عمومی ۱۹۲۶}
\end{figure}

\begin{policybox}[درس‌های اعتصاب ۱۹۲۶]
اعتصاب عمومی شکست خورد اما درس‌های مهمی داشت:
\begin{enumerate}
    \item \textbf{برای اتحادیه‌ها:} اقدام مستقیم صنعتی محدودیت دارد؛ قدرت سیاسی لازم است
    \item \textbf{برای دولت:} سرکوب مطلق ممکن نیست؛ سازش ضروری است
    \item \textbf{برای همه:} جنگ طبقاتی باز برای هیچ‌کس سودمند نیست
\end{enumerate}
این درس‌ها به «اجماع پس از جنگ» کمک کرد.
\end{policybox}

\subsection{دولت‌های کارگری اولیه}

\begin{table}[H]
\centering
\caption{نخستین دولت‌های کارگری}
\renewcommand{\arraystretch}{1.4}
\begin{tabularx}{\textwidth}{>{\bfseries}cccX}
\toprule
\textbf{دوره} & \textbf{نخست‌وزیر} & \textbf{وضعیت} & \textbf{دستاوردها/محدودیت‌ها} \\
\midrule
\rowcolor{empirecream}
۱۹۲۴ & رمزی مک‌دونالد & اقلیت (با حمایت لیبرال) & 
قانون ویتلی (مسکن عمومی)؛ سقوط پس از ۹ ماه \\

۱۹۲۹-۳۱ & رمزی مک‌دونالد & اقلیت & 
رکود بزرگ؛ بحران ۱۹۳۱؛ انشعاب مک‌دونالد \\

\bottomrule
\end{tabularx}
\end{table}

\subsubsection{بحران ۱۹۳۱: «خیانت» مک‌دونالد}

\begin{figure}[H]
\centering
\begin{tikzpicture}[
    node distance=1cm,
    event/.style={rectangle, rounded corners, draw, thick,
                  minimum width=4.5cm, minimum height=1.5cm,
                  text centered, text width=4.3cm, font=\small}
]

\node[event, fill=industrialgray!20] (crisis) at (0,0) {
    بحران مالی\\
    کسری بودجه\\
    فرار سرمایه
};

\node[event, fill=imperialred!20] (cuts) at (5.5,0) {
    پیشنهاد کاهش\\
    مزایای بیکاری\\
    ۱۰٪ کاهش
};

\node[event, fill=royalblue!20] (split) at (11,0) {
    انشعاب کابینه\\
    اکثریت مخالف\\
    مک‌دونالد جدا شد
};

\node[event, fill=parliamentgold!20] (national) at (2.75,-3) {
    دولت ملی\\
    مک‌دونالد + محافظه‌کاران\\
    + لیبرال‌ها
};

\node[event, fill=tudorpurple!20] (election) at (8.25,-3) {
    انتخابات ۱۹۳۱\\
    پیروزی ساحقانه دولت ملی\\
    کارگر: از ۲۸۸ به ۵۲ کرسی
};

\draw[-{Stealth}, thick] (crisis) -- (cuts);
\draw[-{Stealth}, thick] (cuts) -- (split);
\draw[-{Stealth}, thick] (split) -- (national);
\draw[-{Stealth}, thick] (national) -- (election);

\end{tikzpicture}
\caption{بحران ۱۹۳۱}
\end{figure}

\begin{historicalquote}
«هر سوسیالیستی که در دولت سرمایه‌داری وزیر شود، یا سرمایه‌داری را عوض می‌کند 
یا سرمایه‌داری او را عوض می‌کند. معمولاً دومی اتفاق می‌افتد.»
\hfill --- آنورین بوان، ۱۹۴۴
\end{historicalquote}

%----------------------------------------------------------------------
\section{جنگ جهانی دوم: «جنگ مردم» (۱۹۳۹-۱۹۴۵)}
%----------------------------------------------------------------------

\subsection{چرا جنگ دوم متفاوت بود}

\begin{table}[H]
\centering
\caption{مقایسه دو جنگ جهانی از منظر بریتانیا}
\renewcommand{\arraystretch}{1.3}
\begin{tabularx}{\textwidth}{>{\bfseries}p{3.5cm}XX}
\toprule
\textbf{ویژگی} & \textbf{جنگ اول (۱۹۱۴-۱۸)} & \textbf{جنگ دوم (۱۹۳۹-۴۵)} \\
\midrule
\rowcolor{empirecream}
میدان جنگ اصلی & فرانسه (دور از بریتانیا) & بریتانیا خود هدف بمباران \\
تلفات غیرنظامی & ناچیز & ۶۷,۰۰۰ نفر \\
\rowcolor{empirecream}
مشارکت زنان & محدود & گسترده (سربازگیری زنان) \\
تخلیه کودکان & نبود & ۳.۵ میلیون \\
\rowcolor{empirecream}
روایت جنگ & امپراتوری، شرف & دموکراسی، آزادی \\
انتظار پس از جنگ & «کشوری شایسته قهرمانان» & برنامه اجتماعی مشخص \\
\bottomrule
\end{tabularx}
\end{table}

\subsection{جیره‌بندی و برابری}

\begin{figure}[H]
\centering
\begin{tikzpicture}
\begin{axis}[
    ybar,
    width=13cm,
    height=7cm,
    ylabel={میزان هفتگی},
    symbolic x coords={گوشت (اونس), کره (اونس), شکر (اونس), چای (اونس), تخم‌مرغ (عدد)},
    xtick=data,
    x tick label style={rotate=45, anchor=east, font=\small},
    ymin=0,
    ymax=10,
    bar width=25pt,
    nodes near coords,
    title={\textbf{جیره هفتگی یک بزرگسال (۱۹۴۱)}},
]

\addplot[fill=victoriangreen!60] coordinates {
    (گوشت (اونس), 4)
    (کره (اونس), 2)
    (شکر (اونس), 8)
    (چای (اونس), 2)
    (تخم‌مرغ (عدد), 1)
};

\end{axis}
\end{tikzpicture}
\caption{جیره‌بندی غذایی در جنگ دوم}
\end{figure}

\begin{policybox}[پارادوکس جیره‌بندی]
جیره‌بندی باعث \textbf{بهبود سلامت عمومی} شد:
\begin{itemize}
    \item رژیم غذایی متعادل‌تر برای همه
    \item کاهش نابرابری تغذیه‌ای
    \item شیر و ویتامین رایگان برای کودکان و مادران
    \item کاهش مرگ‌ومیر نوزادان
\end{itemize}
این تجربه نشان داد که برنامه‌ریزی دولتی می‌تواند عادلانه و کارآمد باشد.
\end{policybox}

\subsection{گزارش بوریج (۱۹۴۲): نقشه راه دولت رفاه}

\begin{keybox}[پنج غول بوریج]
ویلیام بوریج پنج «غول» را شناسایی کرد که باید شکست داده شوند:
\begin{enumerate}
    \item \textbf{نیاز} (Want) $\rightarrow$ بیمه اجتماعی جامع
    \item \textbf{بیماری} (Disease) $\rightarrow$ خدمات بهداشتی ملی
    \item \textbf{جهل} (Ignorance) $\rightarrow$ آموزش عمومی
    \item \textbf{فلاکت} (Squalor) $\rightarrow$ مسکن عمومی
    \item \textbf{بیکاری} (Idleness) $\rightarrow$ اشتغال کامل
\end{enumerate}
\end{keybox}

\begin{figure}[H]
\centering
\begin{tikzpicture}[
    giant/.style={rectangle, rounded corners=10pt, draw, very thick,
                  minimum width=3cm, minimum height=2.5cm,
                  text centered, text width=2.8cm}
]

\node[giant, fill=imperialred!25, draw=imperialred] (want) at (0,0) {
    \textbf{نیاز}\\[0.2cm]
    فقر\\
    بیمه ملی
};

\node[giant, fill=royalblue!25, draw=royalblue] (disease) at (3.5,0) {
    \textbf{بیماری}\\[0.2cm]
    NHS\\
    درمان رایگان
};

\node[giant, fill=victoriangreen!25, draw=victoriangreen] (ignorance) at (7,0) {
    \textbf{جهل}\\[0.2cm]
    آموزش\\
    قانون باتلر ۴۴
};

\node[giant, fill=parliamentgold!25, draw=parliamentgold] (squalor) at (10.5,0) {
    \textbf{فلاکت}\\[0.2cm]
    مسکن\\
    خانه‌سازی دولتی
};

\node[giant, fill=tudorpurple!25, draw=tudorpurple] (idle) at (14,0) {
    \textbf{بیکاری}\\[0.2cm]
    اشتغال کامل\\
    کینزگرایی
};

\end{tikzpicture}
\caption{پنج غول بوریج و راه‌حل‌ها}
\end{figure}

\begin{historicalquote}
«این یک زمان انقلابی در تاریخ جهان است و نیازمند راه‌حل‌های انقلابی است.»
\hfill --- ویلیام بوریج، ۱۹۴۲
\end{historicalquote}

\subsubsection{استقبال از گزارش}

\begin{itemize}
    \item ۶۳۵,۰۰۰ نسخه فروخته شد (بی‌سابقه برای یک گزارش دولتی)
    \item نظرسنجی‌ها: ۸۶٪ موافق اجرا بودند
    \item RAF نسخه‌های خلاصه را روی آلمان پرتاب کرد
    \item چرچیل محتاط بود، اما افکار عمومی فشار آورد
\end{itemize}

%----------------------------------------------------------------------
\section{دولت اتلی و ساختن دولت رفاه (۱۹۴۵-۱۹۵۱)}
%----------------------------------------------------------------------

\subsection{انتخابات ۱۹۴۵: شوک تاریخی}

\begin{figure}[H]
\centering
\begin{tikzpicture}
\begin{axis}[
    ybar,
    width=12cm,
    height=7cm,
    ylabel={تعداد کرسی‌ها},
    symbolic x coords={کارگر, محافظه‌کار, لیبرال, سایر},
    xtick=data,
    ymin=0,
    ymax=420,
    bar width=40pt,
    nodes near coords,
    title={\textbf{نتایج انتخابات ۱۹۴۵}},
]

\addplot[fill=imperialred!60] coordinates {(کارگر, 393) (محافظه‌کار, 197) (لیبرال, 12) (سایر, 38)};

\end{axis}
\end{tikzpicture}
\caption{پیروزی تاریخی حزب کارگر ۱۹۴۵}
\end{figure}

\begin{warningbox}
\textbf{چرا چرچیل باخت؟}
\begin{itemize}
    \item رأی‌دهندگان می‌خواستند پس از جنگ تغییر کنند، نه به ۱۹۳۹ برگردند
    \item یادآوری بیکاری دهه ۱۹۳۰ تحت محافظه‌کاران
    \item گزارش بوریج انتظارات را بالا برده بود
    \item چرچیل اشتباه کرد: کارگر را به «گشتاپو» تشبیه کرد
    \item رأی سربازان به‌شدت به نفع کارگر بود
\end{itemize}
\end{warningbox}

\subsection{قوانین بنیادین دولت رفاه}

\begin{table}[H]
\centering
\caption{قوانین اصلی دولت اتلی}
\renewcommand{\arraystretch}{1.4}
\begin{tabularx}{\textwidth}{>{\bfseries}cp{3cm}X}
\toprule
\textbf{سال} & \textbf{قانون} & \textbf{محتوا} \\
\midrule
\rowcolor{empirecream}
۱۹۴۵ & کمک‌هزینه خانواده & ۵ شیلینگ هفتگی برای فرزند دوم به بعد \\

۱۹۴۶ & بیمه ملی & پوشش جامع: بیکاری، بیماری، بازنشستگی، فوت \\

\rowcolor{empirecream}
۱۹۴۶ & خدمات بهداشتی ملی & درمان رایگان برای همه (اجرا: ۱۹۴۸) \\

۱۹۴۶ & مسکن & برنامه خانه‌سازی دولتی \\

\rowcolor{empirecream}
۱۹۴۸ & کمک ملی & شبکه امنیت برای کسانی که از بیمه محروم‌اند \\

۱۹۴۴ & آموزش (باتلر) & دبیرستان رایگان برای همه تا ۱۵ سالگی \\

\bottomrule
\end{tabularx}
\end{table}

\subsection{NHS: بزرگ‌ترین دستاورد}

\begin{keybox}[سه اصل NHS]
خدمات بهداشتی ملی بر سه اصل بنا شد:
\begin{enumerate}
    \item \textbf{جامعیت:} پوشش همه خدمات بهداشتی (از پزشک عمومی تا جراحی)
    \item \textbf{رایگان بودن:} در نقطه استفاده، بدون پرداخت مستقیم
    \item \textbf{همگانی:} برای همه ساکنان، نه فقط بیمه‌شدگان
\end{enumerate}
\end{keybox}

\begin{historicalquote}
«نیاز به مراقبت پزشکی هیچ ربطی به توانایی پرداخت ندارد.»
\hfill --- آنورین بوان، وزیر بهداشت، ۱۹۴۸
\end{historicalquote}

\begin{figure}[H]
\centering
\begin{tikzpicture}
\begin{axis}[
    width=13cm,
    height=7cm,
    xlabel={سال},
    ylabel={هزینه سرانه (پوند، قیمت‌های ۲۰۱۰)},
    xmin=1948, xmax=1980,
    ymin=0, ymax=800,
    grid=major,
    title={\textbf{رشد هزینه NHS}},
]

\addplot[very thick, royalblue, mark=*] coordinates {
    (1948, 100) (1950, 130) (1955, 180) (1960, 250)
    (1965, 320) (1970, 420) (1975, 580) (1979, 700)
};

\end{axis}
\end{tikzpicture}
\caption{هزینه سرانه NHS (۱۹۴۸-۱۹۷۹)}
\end{figure}

\subsection{ملی‌سازی صنایع}

\begin{figure}[H]
\centering
\begin{tikzpicture}[
    industry/.style={rectangle, rounded corners, draw=imperialred, thick,
                     fill=imperialred!15, minimum width=3.5cm,
                     minimum height=1.3cm, text centered, font=\small}
]

\node[industry] at (0,2) {بانک انگلستان\\۱۹۴۶};
\node[industry] at (4,2) {زغال‌سنگ\\۱۹۴۷};
\node[industry] at (8,2) {راه‌آهن\\۱۹۴۷};
\node[industry] at (12,2) {برق\\۱۹۴۷};
\node[industry] at (2,0) {گاز\\۱۹۴۸};
\node[industry] at (6,0) {فولاد\\۱۹۴۹};
\node[industry] at (10,0) {حمل‌ونقل جاده‌ای\\۱۹۴۷};

\node[below=1.5cm, font=\small, text width=12cm, text centered] at (6,0) {
    ۲۰٪ اقتصاد ملی‌سازی شد\\
    ۲.۳ میلیون کارگر به بخش دولتی منتقل شدند
};

\end{tikzpicture}
\caption{صنایع ملی‌شده توسط دولت اتلی}
\end{figure}

%----------------------------------------------------------------------
\section{اجماع پس از جنگ (۱۹۵۱-۱۹۷۹)}
%----------------------------------------------------------------------

\subsection{«باتسکلیسم»: اجماع دو حزبی}

\begin{historicalquote}
«باتسکلیسم» ترکیبی طنزآمیز از نام‌های باتلر (محافظه‌کار) و گیتسکل (کارگر) است 
که نشان‌دهنده همگرایی سیاست‌های دو حزب بود.
\hfill --- \textit{The Economist}، ۱۹۵۴
\end{historicalquote}

\begin{table}[H]
\centering
\caption{اصول اجماع پس از جنگ}
\renewcommand{\arraystretch}{1.4}
\begin{tabularx}{\textwidth}{>{\bfseries}p{3.5cm}X}
\toprule
\textbf{اصل} & \textbf{توضیح} \\
\midrule
\rowcolor{empirecream}
دولت رفاه & هر دو حزب متعهد به NHS، بیمه ملی، آموزش رایگان \\

اقتصاد مختلط & ملی‌سازی‌ها حفظ شد؛ بخش خصوصی هم فعال \\

\rowcolor{empirecream}
اشتغال کامل & دولت مسئول حفظ بیکاری زیر ۳٪ \\

مدیریت تقاضای کینزی & سیاست مالی و پولی برای تثبیت اقتصاد \\

\rowcolor{empirecream}
مشورت با اتحادیه‌ها & کورپوراتیسم؛ اتحادیه‌ها شریک تصمیم‌گیری \\

استقلال مستعمرات & فروپاشی مدیریت‌شده امپراتوری \\

\bottomrule
\end{tabularx}
\end{table}

\begin{figure}[H]
\centering
\begin{tikzpicture}[
    scale=0.9,
    govt/.style={rectangle, rounded corners=8pt, draw, very thick,
                 minimum width=4.5cm, minimum height=2cm,
                 text centered, text width=4.3cm}
]

% خط زمانی
\draw[very thick, royalblue] (0,0) -- (16,0);

% دولت‌ها
\node[govt, fill=tudorpurple!20] at (1.5,2) {
    \textbf{چرچیل}\\
    ۱۹۵۱-۵۵\\
    محافظه‌کار
};

\node[govt, fill=tudorpurple!20] at (4.5,2) {
    \textbf{ایدن/مک‌میلان}\\
    ۱۹۵۵-۶۳\\
    محافظه‌کار
};

\node[govt, fill=tudorpurple!20] at (7.5,2) {
    \textbf{داگلاس‌هوم}\\
    ۱۹۶۳-۶۴\\
    محافظه‌کار
};

\node[govt, fill=imperialred!20] at (10.5,2) {
    \textbf{ویلسون}\\
    ۱۹۶۴-۷۰\\
    کارگر
};

\node[govt, fill=tudorpurple!20] at (13.5,2) {
    \textbf{هیث}\\
    ۱۹۷۰-۷۴\\
    محافظه‌کار
};

\node[govt, fill=imperialred!20] at (16,2) {
    \textbf{ویلسون/کالاهان}\\
    ۱۹۷۴-۷۹\\
    کارگر
};

% تاریخ‌ها
\foreach \x/\year in {0/1951, 3/1955, 6.5/1963, 9/1964, 12/1970, 14.5/1974, 17/1979} {
    \draw[thick] (\x,0.2) -- (\x,-0.2);
    \node[below, font=\footnotesize] at (\x,-0.3) {\year};
}

\end{tikzpicture}
\caption{دولت‌های بریتانیا در دوره اجماع}
\end{figure}

\subsection{«عصر طلایی» اقتصادی (۱۹۵۰-۱۹۷۳)}

\begin{figure}[H]
\centering
\begin{tikzpicture}
\begin{axis}[
    width=14cm,
    height=8cm,
    xlabel={سال},
    ylabel={شاخص (۱۹۵۰ = ۱۰۰)},
    xmin=1950, xmax=1980,
    ymin=80, ymax=280,
    grid=major,
    legend pos=north west,
]

\addplot[very thick, victoriangreen, mark=*] coordinates {
    (1950, 100) (1955, 115) (1960, 130) (1965, 155)
    (1970, 175) (1973, 195) (1975, 190) (1979, 210)
};
\addlegendentry{GDP سرانه واقعی}

\addplot[very thick, royalblue, mark=square*] coordinates {
    (1950, 100) (1955, 120) (1960, 145) (1965, 175)
    (1970, 210) (1973, 240) (1975, 250) (1979, 270)
};
\addlegendentry{دستمزد واقعی}

\addplot[very thick, imperialred, mark=triangle*, dashed] coordinates {
    (1950, 100) (1955, 103) (1960, 106) (1965, 108)
    (1970, 112) (1973, 118) (1975, 145) (1979, 175)
};
\addlegendentry{سطح قیمت‌ها}

% خط بحران نفت
\draw[dashed, very thick, industrialgray] (axis cs:1973,80) -- (axis cs:1973,280);
\node[above, font=\footnotesize] at (axis cs:1973,260) {بحران نفت};

\end{axis}
\end{tikzpicture}
\caption{شاخص‌های اقتصادی بریتانیا (۱۹۵۰-۱۹۷۹)}
\end{figure}

\begin{policybox}[ویژگی‌های عصر طلایی]
\begin{itemize}
    \item \textbf{رشد:} متوسط ۲.۸٪ سالانه (پایین‌تر از اروپا، اما بالا برای بریتانیا)
    \item \textbf{بیکاری:} زیر ۳٪ (عملاً اشتغال کامل)
    \item \textbf{تورم:} معتدل (زیر ۵٪ تا ۱۹۷۰)
    \item \textbf{نابرابری:} کاهش یافت (ضریب جینی از ۰.۳۵ به ۰.۲۵)
    \item \textbf{رفاه:} افزایش چشمگیر (خودرو، تلویزیون، یخچال)
\end{itemize}
\end{policybox}

\subsection{فروپاشی امپراتوری}

\begin{figure}[H]
\centering
\begin{tikzpicture}[
    colony/.style={rectangle, rounded corners, draw, thick,
                   minimum height=1cm, text centered, font=\footnotesize}
]

% دهه ۱۹۴۰
\node[colony, fill=parliamentgold!30, minimum width=2cm] at (0,3) {هند\\۱۹۴۷};
\node[colony, fill=parliamentgold!30, minimum width=2cm] at (2.5,3) {پاکستان\\۱۹۴۷};
\node[colony, fill=parliamentgold!30, minimum width=2cm] at (5,3) {برمه\\۱۹۴۸};
\node[colony, fill=parliamentgold!30, minimum width=2cm] at (7.5,3) {سیلان\\۱۹۴۸};

% دهه ۱۹۵۰-۶۰
\node[colony, fill=victoriangreen!30, minimum width=2cm] at (0,1.5) {غنا\\۱۹۵۷};
\node[colony, fill=victoriangreen!30, minimum width=2cm] at (2.5,1.5) {مالزی\\۱۹۵۷};
\node[colony, fill=victoriangreen!30, minimum width=2cm] at (5,1.5) {نیجریه\\۱۹۶۰};
\node[colony, fill=victoriangreen!30, minimum width=2.5cm] at (8,1.5) {کنیا\\۱۹۶۳};

% دهه ۱۹۶۰-۷۰
\node[colony, fill=royalblue!30, minimum width=2cm] at (0,0) {جامائیکا\\۱۹۶۲};
\node[colony, fill=royalblue!30, minimum width=2.5cm] at (3,0) {رودزیا (UDI)\\۱۹۶۵};
\node[colony, fill=royalblue!30, minimum width=2cm] at (6,0) {عدن\\۱۹۶۷};
\node[colony, fill=royalblue!30, minimum width=2.5cm] at (9,0) {شرق سوئز\\۱۹۶۸};

\node[below=0.5cm, font=\small] at (4.5,0) {
    ۱۹۴۵: ۷۰۰ میلیون نفر تحت حاکمیت $\rightarrow$ ۱۹۶۸: عملاً هیچ
};

\end{tikzpicture}
\caption{استقلال مستعمرات بریتانیا}
\end{figure}

\subsection{مهاجرت و جامعه چندفرهنگی}

\begin{table}[H]
\centering
\caption{مهاجرت به بریتانیا از کشورهای مشترک‌المنافع}
\renewcommand{\arraystretch}{1.3}
\begin{tabular}{lcccc}
\toprule
\textbf{دوره} & \textbf{کارائیب} & \textbf{هند/پاکستان} & \textbf{آفریقا} & \textbf{جمع} \\
\midrule
\rowcolor{empirecream}
۱۹۴۸-۱۹۵۲ & ۱۵,۰۰۰ & ۵,۰۰۰ & ۲,۰۰۰ & ۲۲,۰۰۰ \\
۱۹۵۳-۱۹۵۷ & ۷۵,۰۰۰ & ۲۰,۰۰۰ & ۵,۰۰۰ & ۱۰۰,۰۰۰ \\
\rowcolor{empirecream}
۱۹۵۸-۱۹۶۲ & ۱۵۰,۰۰۰ & ۱۰۰,۰۰۰ & ۱۵,۰۰۰ & ۲۶۵,۰۰۰ \\
\bottomrule
\end{tabular}
\end{table}

\begin{warningbox}
\textbf{واکنش و محدودیت‌ها:}
\begin{itemize}
    \item شورش‌های ناتینگ هیل ۱۹۵۸ (حملات نژادپرستانه)
    \item قانون مهاجران مشترک‌المنافع ۱۹۶۲ (محدودیت)
    \item سخنرانی انوک پاول «رودخانه‌های خون» ۱۹۶۸
    \item قوانین روابط نژادی ۱۹۶۵، ۱۹۶۸، ۱۹۷۶ (ضد تبعیض)
\end{itemize}
\end{warningbox}

%----------------------------------------------------------------------
\section{فروپاشی اجماع (۱۹۷۰-۱۹۷۹)}
%----------------------------------------------------------------------

\subsection{بحران‌های دهه ۱۹۷۰}

\begin{figure}[H]
\centering
\begin{tikzpicture}[
    crisis/.style={rectangle, rounded corners=8pt, draw=imperialred, thick,
                   fill=imperialred!15, minimum width=4cm,
                   minimum height=2cm, text centered, text width=3.8cm}
]

\node[crisis] at (0,2.5) {
    \textbf{بحران نفت}\\
    ۱۹۷۳، ۱۹۷۹\\
    چهاربرابر شدن قیمت
};

\node[crisis] at (5,2.5) {
    \textbf{تورم}\\
    ۱۹۷۵: ۲۵٪\\
    رکود تورمی
};

\node[crisis] at (10,2.5) {
    \textbf{بیکاری}\\
    از ۳٪ به ۶٪\\
    پایان اشتغال کامل
};

\node[crisis] at (2.5,0) {
    \textbf{بحران اتحادیه‌ها}\\
    اعتصابات گسترده\\
    سقوط دولت هیث
};

\node[crisis] at (7.5,0) {
    \textbf{بحران IMF}\\
    ۱۹۷۶\\
    وام با شرایط سخت
};

\end{tikzpicture}
\caption{بحران‌های دهه ۱۹۷۰}
\end{figure}

\subsection{آمار رکود تورمی}

\begin{figure}[H]
\centering
\begin{tikzpicture}
\begin{axis}[
    width=14cm,
    height=8cm,
    xlabel={سال},
    ylabel={درصد},
    xmin=1970, xmax=1980,
    ymin=0, ymax=28,
    grid=major,
    legend pos=north west,
]

\addplot[very thick, imperialred, mark=*] coordinates {
    (1970, 6) (1971, 9) (1972, 7) (1973, 9) (1974, 16)
    (1975, 24) (1976, 17) (1977, 16) (1978, 8) (1979, 13)
};
\addlegendentry{تورم}

\addplot[very thick, royalblue, mark=square*] coordinates {
    (1970, 3) (1971, 4) (1972, 4) (1973, 3) (1974, 3)
    (1975, 4) (1976, 6) (1977, 6) (1978, 6) (1979, 5)
};
\addlegendentry{بیکاری}

\end{axis}
\end{tikzpicture}
\caption{تورم و بیکاری در دهه ۱۹۷۰}
\end{figure}

\subsection{سقوط دولت هیث (۱۹۷۴)}

\begin{historicalquote}
«چه کسی بریتانیا را اداره می‌کند؟»
\hfill --- شعار انتخاباتی ادوارد هیث، فوریه ۱۹۷۴
\end{historicalquote}

\begin{table}[H]
\centering
\caption{بحران زغال‌سنگ و سقوط هیث}
\renewcommand{\arraystretch}{1.4}
\begin{tabularx}{\textwidth}{>{\bfseries}cX}
\toprule
\textbf{تاریخ} & \textbf{رویداد} \\
\midrule
\rowcolor{empirecream}
نوامبر ۱۹۷۳ & اعتصاب کار اضافی معدنچیان در بحبوحه بحران نفت \\

دسامبر ۱۹۷۳ & اعلام «هفته سه‌روزه کاری» برای صرفه‌جویی برق \\

\rowcolor{empirecream}
فوریه ۱۹۷۴ & اعتصاب کامل معدنچیان؛ هیث انتخابات زودرس اعلام کرد \\

فوریه ۱۹۷۴ & انتخابات: کارگر ۳۰۱، محافظه‌کار ۲۹۷ (بدون اکثریت) \\

\rowcolor{empirecream}
مارس ۱۹۷۴ & ویلسون دولت اقلیت تشکیل داد؛ معدنچیان پیروز شدند \\

\bottomrule
\end{tabularx}
\end{table}

\subsection{بحران IMF (۱۹۷۶)}

\begin{keybox}[نقطه عطف ایدئولوژیک]
در سپتامبر ۱۹۷۶، دولت کارگری کالاهان مجبور شد از صندوق بین‌المللی پول 
۳.۹ میلیارد دلار وام بگیرد -- بزرگ‌ترین وام در تاریخ IMF تا آن زمان. 
شرایط وام شامل کاهش هزینه‌های عمومی بود که عملاً پایان کینزگرایی را 
نشان می‌داد.
\end{keybox}

\begin{historicalquote}
«ما قبلاً فکر می‌کردیم که می‌توان با خرج کردن از رکود خارج شد و با 
کاهش مالیات و افزایش هزینه‌های دولتی اشتغال ایجاد کرد. من صادقانه 
به شما می‌گویم که این گزینه دیگر وجود ندارد.»
\hfill --- جیمز کالاهان، کنفرانس حزب کارگر، ۱۹۷۶
\end{historicalquote}

\subsection{زمستان نارضایتی (۱۹۷۸-۱۹۷۹)}

\begin{figure}[H]
\centering
\begin{tikzpicture}[
    strike/.style={rectangle, rounded corners, draw=imperialred, thick,
                   fill=imperialred!20, minimum width=3.5cm,
                   minimum height=1.5cm, text centered, text width=3.3cm, font=\small}
]

\node[strike] at (0,2) {
    \textbf{رانندگان تانکر}\\
    ژانویه ۱۹۷۹\\
    کمبود بنزین
};

\node[strike] at (4.5,2) {
    \textbf{کارگران بهداشت}\\
    ژانویه-مارس\\
    بیمارستان‌ها مختل
};

\node[strike] at (9,2) {
    \textbf{گورکنان}\\
    لیورپول\\
    اجساد دفن نشده
};

\node[strike] at (2.25,0) {
    \textbf{جمع‌آوری زباله}\\
    لندن\\
    زباله در خیابان‌ها
};

\node[strike] at (6.75,0) {
    \textbf{رانندگان کامیون}\\
    سراسری\\
    کمبود غذا
};

\end{tikzpicture}
\caption{اعتصابات «زمستان نارضایتی»}
\end{figure}

\begin{table}[H]
\centering
\caption{آمار زمستان نارضایتی}
\renewcommand{\arraystretch}{1.3}
\begin{tabular}{lc}
\toprule
\textbf{شاخص} & \textbf{مقدار} \\
\midrule
\rowcolor{empirecream}
روزهای کاری از دست رفته (۱۹۷۹) & ۲۹.۵ میلیون \\
تعداد اعتصابات & ۲,۱۲۵ \\
\rowcolor{empirecream}
کارگران درگیر & ۴.۶ میلیون \\
کاهش محبوبیت کارگر & از ۴۵٪ به ۳۳٪ \\
\bottomrule
\end{tabular}
\end{table}

\begin{warningbox}
\textbf{تصاویر ماندگار:}
\begin{itemize}
    \item کوه‌های زباله در میدان لستر
    \item اجساد انبار شده در لیورپول
    \item صف‌های طولانی بنزین
    \item کالاهان در فرودگاه: «بحران؟ چه بحرانی؟» (عنوان ساختگی روزنامه)
\end{itemize}
این تصاویر برای نسلی از رأی‌دهندگان ماندگار شد و «حافظه جمعی» از 
شکست کارگر ایجاد کرد.
\end{warningbox}

\subsection{انتخابات ۱۹۷۹: پایان یک عصر}

\begin{figure}[H]
\centering
\begin{tikzpicture}
\begin{axis}[
    ybar,
    width=12cm,
    height=7cm,
    ylabel={درصد آرا},
    symbolic x coords={محافظه‌کار, کارگر, لیبرال, سایر},
    xtick=data,
    ymin=0,
    ymax=50,
    bar width=35pt,
    nodes near coords,
    title={\textbf{انتخابات ۱۹۷۹}},
]

\addplot[fill=tudorpurple!60] coordinates {
    (محافظه‌کار, 43.9) (کارگر, 36.9) (لیبرال, 13.8) (سایر, 5.4)
};

\end{axis}
\end{tikzpicture}
\caption{نتایج انتخابات ۱۹۷۹: پیروزی تاچر}
\end{figure}

%----------------------------------------------------------------------
\section{تحلیل نهایی فصل ششم}
%----------------------------------------------------------------------

\subsection{چرخه‌های تاریخی این دوره}

\begin{figure}[H]
\centering
\begin{tikzpicture}[
    phase/.style={rectangle, rounded corners=10pt, draw, very thick,
                  minimum width=4cm, minimum height=3cm,
                  text centered, text width=3.8cm}
]

\node[phase, fill=imperialred!20] (war) at (0,0) {
    \textbf{جنگ}\\[0.2cm]
    دولت قوی\\
    بسیج ملی\\
    انتظارات بالا
};

\node[phase, fill=victoriangreen!20] (reform) at (5,0) {
    \textbf{اصلاحات}\\[0.2cm]
    گسترش دولت\\
    دولت رفاه\\
    ملی‌سازی
};

\node[phase, fill=royalblue!20] (consensus) at (10,0) {
    \textbf{اجماع}\\[0.2cm]
    رشد اقتصادی\\
    ثبات نسبی\\
    رضایت
};

\node[phase, fill=industrialgray!20] (crisis) at (15,0) {
    \textbf{بحران}\\[0.2cm]
    رکود تورمی\\
    تنش طبقاتی\\
    زوال اجماع
};

\draw[-{Stealth}, very thick] (war) -- (reform);
\draw[-{Stealth}, very thick] (reform) -- (consensus);
\draw[-{Stealth}, very thick] (consensus) -- (crisis);
\draw[-{Stealth}, very thick, dashed] (crisis) to[bend right=30] node[below, font=\small] {؟ تاچریسم} (war);

\end{tikzpicture}
\caption{چرخه تاریخی ۱۹۱۴-۱۹۷۹}
\end{figure}

\subsection{الگوهای قابل استخراج}

\begin{policybox}[الگوی ۱: جنگ به‌مثابه موتور تغییر]
هر دو جنگ جهانی تغییرات اجتماعی-سیاسی را شتاب دادند:
\begin{itemize}
    \item \textbf{جنگ اول:} حق رأی همگانی، دولت مداخله‌گر، سقوط لیبرال‌ها
    \item \textbf{جنگ دوم:} دولت رفاه، NHS، ملی‌سازی
\end{itemize}
جنگ «قرارداد اجتماعی» جدیدی ایجاد کرد: در ازای فداکاری، دولت امنیت تضمین می‌کند.
\end{policybox}

\begin{policybox}[الگوی ۲: اجماع و محدودیت‌های آن]
اجماع پس از جنگ ۳۰ سال دوام آورد اما محدودیت‌هایی داشت:
\begin{itemize}
    \item بر رشد اقتصادی متکی بود (وقتی رشد متوقف شد، اجماع شکست)
    \item تنش‌های طبقاتی را حل نکرد، فقط مدیریت کرد
    \item انعطاف‌پذیری اقتصادی را محدود کرد
    \item هزینه‌های فزاینده‌ای داشت
\end{itemize}
\end{policybox}

\begin{policybox}[الگوی ۳: نقش بحران در تغییر پارادایم]
دهه ۱۹۷۰ نشان داد که:
\begin{itemize}
    \item بحران‌ها فرصت برای تغییرات بنیادین ایجاد می‌کنند
    \item ایده‌های «غیرممکن» می‌توانند ناگهان «ممکن» شوند
    \item شکست سیاست‌های موجود، جایگزین‌ها را مشروع می‌کند
\end{itemize}
تاچریسم بدون بحران دهه ۱۹۷۰ ممکن نبود.
\end{policybox}

\subsection{مقایسه با سایر کشورها}

\begin{table}[H]
\centering
\caption{مقایسه دولت‌های رفاه اروپایی (حدود ۱۹۷۵)}
\renewcommand{\arraystretch}{1.3}
\begin{tabularx}{\textwidth}{>{\bfseries}lXXXX}
\toprule
\textbf{ویژگی} & \textbf{بریتانیا} & \textbf{سوئد} & \textbf{آلمان} & \textbf{فرانسه} \\
\midrule
\rowcolor{empirecream}
مدل & بوریجی-همگانی & سوسیال‌دموکرات & کورپوراتیستی & ترکیبی \\

هزینه رفاه (\% GDP) & ۲۰٪ & ۲۷٪ & ۲۳٪ & ۲۲٪ \\

\rowcolor{empirecream}
تأمین مالی & مالیات عمومی & مالیات عمومی & بیمه اجتماعی & بیمه + مالیات \\

پوشش بهداشت & NHS (همگانی) & همگانی & بیمه اجباری & بیمه + دولتی \\

\rowcolor{empirecream}
نقش اتحادیه‌ها & قوی، تقابلی & قوی، مشارکتی & متوسط، مشارکتی & ضعیف‌تر \\

\bottomrule
\end{tabularx}
\end{table}

%----------------------------------------------------------------------
\section*{منابع اصلی فصل ششم}
%----------------------------------------------------------------------

\begin{enumerate}[label={[\arabic*]}]
    \item Addison, Paul (1975). \textit{The Road to 1945}. London: Jonathan Cape.
    \item Marwick, Arthur (1965). \textit{The Deluge: British Society and the First World War}. London: Bodley Head.
    \item Morgan, Kenneth O. (1984). \textit{Labour in Power 1945-1951}. Oxford: Clarendon Press.
    \item Hennessy, Peter (1992). \textit{Never Again: Britain 1945-1951}. London: Jonathan Cape.
    \item Cairncross, Alec (1985). \textit{Years of Recovery: British Economic Policy 1945-51}. London: Methuen.
    \item Bogdanor, Vernon \& Skidelsky, Robert, eds. (1970). \textit{The Age of Affluence 1951-1964}. London: Macmillan.
    \item Sandbrook, Dominic (2010-2012). \textit{Seasons in the Sun} (series). London: Allen Lane.
    \item Beckett, Andy (2009). \textit{When the Lights Went Out: Britain in the Seventies}. London: Faber.
    \item Beveridge, William (1942). \textit{Social Insurance and Allied Services}. Cmd. 6404. London: HMSO.
    \item Webster, Charles (1988). \textit{The Health Services Since the War}. Vol. 1. London: HMSO.
\end{enumerate}


%%%%%%%%%%%%%%%%%%%%%%%%%%%%%%%%%%%%%%%%%%%%%%%%%%%%%%%%%%%%%%%%%%%%%%%
% فصل هفتم: از تاچر تا برگزیت (۱۹۷۹-۲۰۲۴)
%%%%%%%%%%%%%%%%%%%%%%%%%%%%%%%%%%%%%%%%%%%%%%%%%%%%%%%%%%%%%%%%%%%%%%%

\chapter{از تاچر تا برگزیت (۱۹۷۹-۲۰۲۴)}

\begin{keybox}[خلاصه فصل]
این فصل به تحلیل دوره‌ای می‌پردازد که با \textbf{انقلاب تاچری} آغاز شد 
و نظم پس از جنگ را واژگون کرد. تاچریسم \textbf{بازار}، \textbf{فردگرایی}، 
و \textbf{دولت کوچک} را جایگزین اجماع کینزی کرد. حزب کارگر تحت بلر 
با «راه سوم» به قدرت بازگشت، اما بسیاری از اصول تاچری را پذیرفت. 
بحران مالی ۲۰۰۸ و \textbf{برگزیت} ۲۰۱۶ نشان داد که تنش‌های عمیق -- 
نابرابری منطقه‌ای، هویت ملی، جهانی‌شدن -- حل نشده باقی مانده‌اند.
\end{keybox}

%----------------------------------------------------------------------
\section{ساختار تحلیلی فصل}
%----------------------------------------------------------------------

\begin{figure}[H]
\centering
\begin{tikzpicture}[
    period/.style={rectangle, rounded corners=10pt, draw, very thick,
                   minimum width=3.5cm, minimum height=2.5cm,
                   text centered, text width=3.3cm}
]

\node[period, fill=tudorpurple!25] (thatcher) at (0,0) {
    \textbf{۱۹۷۹-۱۹۹۰}\\
    تاچر\\
    انقلاب نئولیبرال
};

\node[period, fill=tudorpurple!15] (major) at (4,0) {
    \textbf{۱۹۹۰-۱۹۹۷}\\
    میجر\\
    تثبیت تاچریسم
};

\node[period, fill=imperialred!25] (blair) at (8,0) {
    \textbf{۱۹۹۷-۲۰۱۰}\\
    بلر/براون\\
    «راه سوم»
};

\node[period, fill=tudorpurple!20] (coalition) at (12,0) {
    \textbf{۲۰۱۰-۲۰۱۹}\\
    کامرون/می\\
    ریاضت، برگزیت
};

\node[period, fill=industrialgray!25] (recent) at (16,0) {
    \textbf{۲۰۱۹-۲۰۲۴}\\
    جانسون/سوناک\\
    بحران‌ها
};

\draw[-{Stealth}, very thick] (thatcher) -- (major);
\draw[-{Stealth}, very thick] (major) -- (blair);
\draw[-{Stealth}, very thick] (blair) -- (coalition);
\draw[-{Stealth}, very thick] (coalition) -- (recent);

\end{tikzpicture}
\caption{دوره‌بندی فصل هفتم}
\end{figure}

%----------------------------------------------------------------------
\section{انقلاب تاچری (۱۹۷۹-۱۹۹۰)}
%----------------------------------------------------------------------

\subsection{مارگارت تاچر: شخصیت و ایدئولوژی}

\begin{historicalquote}
«من آمده‌ام که اجماع را از بین ببرم، نه که به آن بپیوندم.»
\hfill --- مارگارت تاچر، ۱۹۷۹
\end{historicalquote}

\begin{historicalquote}
«چیزی به نام جامعه وجود ندارد. افراد مرد و زن هستند، و خانواده‌ها.»
\hfill --- مارگارت تاچر، ۱۹۸۷
\end{historicalquote}

\subsection{ستون‌های تاچریسم}

\begin{figure}[H]
\centering
\begin{tikzpicture}[
    pillar/.style={rectangle, rounded corners=5pt, draw=tudorpurple, very thick,
                   fill=tudorpurple!15, minimum width=3.2cm,
                   minimum height=4cm, text centered, text width=3cm}
]

\node[pillar] (p1) at (0,0) {
    \textbf{پول‌گرایی}\\[0.3cm]
    \footnotesize
    کنترل تورم\\
    از طریق\\
    عرضه پول\\
    نه مدیریت تقاضا
};

\node[pillar] (p2) at (4,0) {
    \textbf{خصوصی‌سازی}\\[0.3cm]
    \footnotesize
    فروش صنایع\\
    دولتی\\
    «دموکراسی مالکیت»
};

\node[pillar] (p3) at (8,0) {
    \textbf{مقررات‌زدایی}\\[0.3cm]
    \footnotesize
    آزادسازی مالی\\
    بازار کار\\
    انعطاف‌پذیری
};

\node[pillar] (p4) at (12,0) {
    \textbf{ضد اتحادیه}\\[0.3cm]
    \footnotesize
    محدود کردن\\
    قدرت اتحادیه‌ها\\
    شکست معدنچیان
};

\node[pillar] (p5) at (16,0) {
    \textbf{کوچک‌سازی دولت}\\[0.3cm]
    \footnotesize
    کاهش مالیات\\
    کاهش هزینه‌ها\\
    «چرخش به عقب»
};

\end{tikzpicture}
\caption{پنج ستون تاچریسم}
\end{figure}

\subsection{سیاست‌های کلیدی}

\subsubsection{خصوصی‌سازی}

\begin{table}[H]
\centering
\caption{خصوصی‌سازی‌های اصلی دوره تاچر-میجر}
\renewcommand{\arraystretch}{1.3}
\begin{tabularx}{\textwidth}{>{\bfseries}clXc}
\toprule
\textbf{سال} & \textbf{شرکت} & \textbf{روش} & \textbf{درآمد (م.پوند)} \\
\midrule
\rowcolor{empirecream}
۱۹۸۴ & British Telecom & عرضه عمومی سهام & ۳,۹۰۰ \\
۱۹۸۶ & British Gas & عرضه عمومی («اگر سید را می‌بینید...») & ۵,۶۰۰ \\
\rowcolor{empirecream}
۱۹۸۷ & British Airways & عرضه عمومی & ۹۰۰ \\
۱۹۸۷ & Rolls-Royce & عرضه عمومی & ۱,۴۰۰ \\
\rowcolor{empirecream}
۱۹۸۸ & British Steel & عرضه عمومی & ۲,۵۰۰ \\
۱۹۸۹ & شرکت‌های آب & ۱۰ شرکت منطقه‌ای & ۵,۲۰۰ \\
\rowcolor{empirecream}
۱۹۹۰ & شرکت‌های برق & تولید و توزیع & ۸,۲۰۰ \\
۱۹۹۳-۹۶ & British Rail & تکه‌تکه شد & ۱,۹۰۰ \\
\midrule
& \textbf{جمع} & & \textbf{$\sim$۵۰,۰۰۰} \\
\bottomrule
\end{tabularx}
\end{table}

\begin{figure}[H]
\centering
\begin{tikzpicture}
\begin{axis}[
    width=13cm,
    height=7cm,
    xlabel={سال},
    ylabel={درآمد خصوصی‌سازی (میلیارد پوند)},
    xmin=1979, xmax=1997,
    ymin=0, ymax=10,
    grid=major,
    ybar,
    bar width=8pt,
]

\addplot[fill=tudorpurple!60] coordinates {
    (1980, 0.4) (1981, 0.5) (1982, 0.5) (1983, 1.1)
    (1984, 2.1) (1985, 2.7) (1986, 4.5) (1987, 5.1)
    (1988, 7.1) (1989, 4.2) (1990, 5.3) (1991, 7.9)
    (1992, 8.2) (1993, 5.4) (1994, 6.4) (1995, 2.4)
    (1996, 4.4)
};

\end{axis}
\end{tikzpicture}
\caption{درآمد سالانه خصوصی‌سازی}
\end{figure}

\subsubsection{شکست اتحادیه‌ها: اعتصاب معدنچیان ۱۹۸۴-۱۹۸۵}

\begin{figure}[H]
\centering
\begin{tikzpicture}[
    node distance=1.2cm,
    event/.style={rectangle, rounded corners, draw, thick,
                  minimum width=4.5cm, minimum height=1.5cm,
                  text centered, text width=4.3cm, font=\small}
]

\node[event, fill=industrialgray!20] (plan) at (0,0) {
    برنامه بستن ۲۰ معدن\\
    مارس ۱۹۸۴\\
    ۲۰,۰۰۰ شغل
};

\node[event, fill=imperialred!20] (strike) at (5.5,0) {
    اعتصاب NUM\\
    (بدون رأی‌گیری ملی)\\
    آرتور اسکارگیل
};

\node[event, fill=tudorpurple!20] (govt) at (11,0) {
    استراتژی دولت:\\
    ذخیره زغال\\
    پلیس متحرک\\
    عدم مذاکره
};

\node[event, fill=parliamentgold!20] (orgreave) at (2.75,-3) {
    نبرد اورگریو\\
    ژوئن ۱۹۸۴\\
    درگیری خشونت‌آمیز
};

\node[event, fill=victoriangreen!20] (split) at (8.25,-3) {
    انشعاب: UDM\\
    بازگشت تدریجی\\
    به کار
};

\node[event, fill=imperialred!30] (defeat) at (5.5,-6) {
    شکست اعتصاب\\
    مارس ۱۹۸۵\\
    بازگشت بدون شرط
};

\draw[-{Stealth}, thick] (plan) -- (strike);
\draw[-{Stealth}, thick] (strike) -- (govt);
\draw[-{Stealth}, thick] (strike) -- (orgreave);
\draw[-{Stealth}, thick] (govt) -- (split);
\draw[-{Stealth}, thick] (orgreave) -- (defeat);
\draw[-{Stealth}, thick] (split) -- (defeat);

\end{tikzpicture}
\caption{مسیر اعتصاب معدنچیان ۱۹۸۴-۱۹۸۵}
\end{figure}

\begin{table}[H]
\centering
\caption{پیامدهای اعتصاب معدنچیان}
\renewcommand{\arraystretch}{1.4}
\begin{tabularx}{\textwidth}{>{\bfseries}p{3cm}X}
\toprule
\textbf{شاخص} & \textbf{تغییر} \\
\midrule
\rowcolor{empirecream}
تعداد معدنچیان & ۱۸۷,۰۰۰ (۱۹۸۴) $\rightarrow$ ۱۱,۰۰۰ (۱۹۹۴) $\rightarrow$ ۰ (۲۰۱۵) \\

عضویت اتحادیه‌ای & ۱۳ میلیون (۱۹۷۹) $\rightarrow$ ۸ میلیون (۱۹۹۰) \\

\rowcolor{empirecream}
روزهای اعتصابی & ۲۹ میلیون (۱۹۷۹) $\rightarrow$ ۲ میلیون (۱۹۹۰) \\

جوامع معدنی & ویرانی اقتصادی، بیکاری انبوه، مسائل اجتماعی \\

\bottomrule
\end{tabularx}
\end{table}

\begin{warningbox}
\textbf{میراث متناقض:}

اعتصاب معدنچیان یکی از رویدادهای بحث‌برانگیز تاریخ معاصر بریتانیاست:
\begin{itemize}
    \item \textbf{دیدگاه راست:} پیروزی ضروری بر قدرت غیردموکراتیک اتحادیه‌ها
    \item \textbf{دیدگاه چپ:} تخریب عمدی جوامع کارگری، «جنگ طبقاتی از بالا»
    \item \textbf{دیدگاه میانه:} گذار ناگزیر اما بی‌رحمانه مدیریت شد
\end{itemize}
\end{warningbox}

\subsubsection{مقررات‌زدایی مالی: Big Bang (۱۹۸۶)}

\begin{keybox}[انقلاب سیتی]
در ۲۷ اکتبر ۱۹۸۶، بازارهای مالی لندن یکباره مقررات‌زدایی شدند:
\begin{itemize}
    \item پایان جداسازی کارگزاران و معامله‌گران
    \item ورود بانک‌های خارجی
    \item معاملات الکترونیکی
    \item پایان کمیسیون‌های ثابت
\end{itemize}
لندن به مرکز مالی جهان تبدیل شد، اما ریسک‌های سیستمی نیز افزایش یافت.
\end{keybox}

\subsection{شاخص‌های اقتصادی دوره تاچر}

\begin{figure}[H]
\centering
\begin{tikzpicture}
\begin{axis}[
    width=14cm,
    height=8cm,
    xlabel={سال},
    ylabel={درصد},
    xmin=1979, xmax=1991,
    ymin=-5, ymax=25,
    grid=major,
    legend pos=outer north east,
]

\addplot[very thick, imperialred, mark=*] coordinates {
    (1979, 13) (1980, 18) (1981, 12) (1982, 8) (1983, 5)
    (1984, 5) (1985, 6) (1986, 3) (1987, 4) (1988, 5)
    (1989, 8) (1990, 10)
};
\addlegendentry{تورم}

\addplot[very thick, royalblue, mark=square*] coordinates {
    (1979, 5) (1980, 7) (1981, 10) (1982, 11) (1983, 12)
    (1984, 12) (1985, 11) (1986, 11) (1987, 10) (1988, 8)
    (1989, 7) (1990, 7)
};
\addlegendentry{بیکاری}

\addplot[very thick, victoriangreen, mark=triangle*] coordinates {
    (1979, 3) (1980, -2) (1981, -1) (1982, 2) (1983, 4)
    (1984, 3) (1985, 4) (1986, 4) (1987, 5) (1988, 5)
    (1989, 2) (1990, 1)
};
\addlegendentry{رشد GDP}

\end{axis}
\end{tikzpicture}
\caption{شاخص‌های کلیدی اقتصاد بریتانیا (۱۹۷۹-۱۹۹۰)}
\end{figure}

\subsection{نابرابری و «دو ملت»}

\begin{figure}[H]
\centering
\begin{tikzpicture}
\begin{axis}[
    width=13cm,
    height=7cm,
    xlabel={سال},
    ylabel={ضریب جینی},
    xmin=1975, xmax=2000,
    ymin=0.22, ymax=0.36,
    grid=major,
    title={\textbf{افزایش نابرابری در دوره تاچر}},
]

\addplot[very thick, imperialred, mark=*] coordinates {
    (1977, 0.24) (1979, 0.25) (1981, 0.26) (1983, 0.27)
    (1985, 0.29) (1987, 0.31) (1989, 0.33) (1991, 0.34)
    (1993, 0.34) (1995, 0.33) (1997, 0.33)
};

% خط دوره تاچر
\draw[dashed, thick, tudorpurple] (axis cs:1979,0.22) -- (axis cs:1979,0.36);
\draw[dashed, thick, tudorpurple] (axis cs:1990,0.22) -- (axis cs:1990,0.36);
\node[above, font=\footnotesize] at (axis cs:1984.5,0.35) {دوره تاچر};

\end{axis}
\end{tikzpicture}
\caption{افزایش نابرابری درآمدی}
\end{figure}

\begin{table}[H]
\centering
\caption{توزیع درآمد: سهم دهک‌ها}
\renewcommand{\arraystretch}{1.3}
\begin{tabular}{lccc}
\toprule
\textbf{گروه درآمدی} & \textbf{۱۹۷۹} & \textbf{۱۹۹۰} & \textbf{تغییر} \\
\midrule
\rowcolor{imperialred!15}
دهک پایین & ۴.۰٪ & ۲.۹٪ & $-۲۸٪$ \\
\rowcolor{empirecream}
دهک‌های میانی (۴-۷) & ۴۳٪ & ۴۰٪ & $-۷٪$ \\
\rowcolor{victoriangreen!15}
دهک بالا & ۲۱٪ & ۲۷٪ & $+۲۹٪$ \\
\bottomrule
\end{tabular}
\end{table}

\subsection{سقوط تاچر (۱۹۹۰)}

\begin{figure}[H]
\centering
\begin{tikzpicture}[
    factor/.style={rectangle, rounded corners, draw=imperialred, thick,
                   fill=imperialred!15, minimum width=4cm,
                   minimum height=1.5cm, text centered, text width=3.8cm}
]

\node[factor] at (0,2) {مالیات سرانه\\(Poll Tax)\\شورش‌های خیابانی};
\node[factor] at (5,2) {اختلاف بر سر اروپا\\مخالفت با یورو\\استعفای هاو};
\node[factor] at (10,2) {افت محبوبیت\\ترس از باخت انتخابات\\نارضایتی حزبی};
\node[factor] at (2.5,0) {سبک استبدادی\\«بانوی آهنین»\\عدم مشورت};
\node[factor] at (7.5,0) {چالش هزلتاین\\رأی‌گیری رهبری\\ناکافی بودن رأی};

\node[rectangle, rounded corners=10pt, draw=tudorpurple, very thick,
      fill=tudorpurple!20, minimum width=6cm] at (5,-2.5) {
    \textbf{استعفای تاچر}\\
    ۲۲ نوامبر ۱۹۹۰
};

\end{tikzpicture}
\caption{عوامل سقوط تاچر}
\end{figure}

%----------------------------------------------------------------------
\section{دوره میجر (۱۹۹۰-۱۹۹۷)}
%----------------------------------------------------------------------

\subsection{ویژگی‌های دوره میجر}

\begin{table}[H]
\centering
\caption{دوره جان میجر: ادامه و تعدیل تاچریسم}
\renewcommand{\arraystretch}{1.4}
\begin{tabularx}{\textwidth}{>{\bfseries}p{3.5cm}X}
\toprule
\textbf{حوزه} & \textbf{سیاست} \\
\midrule
\rowcolor{empirecream}
اقتصاد & ادامه خصوصی‌سازی (راه‌آهن)؛ بحران ERM («چهارشنبه سیاه» ۱۹۹۲) \\

اروپا & امضای معاهده ماستریخت؛ opt-out از یورو؛ شورش یوروسکپتیک‌ها \\

\rowcolor{empirecream}
ایرلند شمالی & مذاکرات صلح؛ اعلامیه داونینگ استریت ۱۹۹۳ \\

خدمات عمومی & «منشورهای شهروندی»؛ استانداردهای عملکرد؛ PFI \\

\rowcolor{empirecream}
اجتماعی & «بازگشت به اصول»؛ رسوایی‌های متعدد \\

\bottomrule
\end{tabularx}
\end{table}

\subsubsection{چهارشنبه سیاه (۱۶ سپتامبر ۱۹۹۲)}

\begin{keybox}[بحران ERM]
بریتانیا مجبور شد پوند را از مکانیسم نرخ ارز اروپایی (ERM) خارج کند:
\begin{itemize}
    \item نرخ بهره در یک روز از ۱۰٪ به ۱۵٪ رفت (و برگشت)
    \item بانک مرکزی میلیاردها پوند برای دفاع از ارز خرج کرد
    \item جرج سوروس ۱ میلیارد دلار سود کرد
    \item اعتبار اقتصادی محافظه‌کاران برای یک نسل از بین رفت
\end{itemize}
\end{keybox}

%----------------------------------------------------------------------
\section{کارگر نو و «راه سوم» (۱۹۹۷-۲۰۱۰)}
%----------------------------------------------------------------------

\subsection{تحول حزب کارگر}

\begin{figure}[H]
\centering
\begin{tikzpicture}[
    phase/.style={rectangle, rounded corners=8pt, draw, very thick,
                  minimum width=4.5cm, minimum height=2.5cm,
                  text centered, text width=4.3cm}
]

\node[phase, fill=imperialred!30] (old) at (0,0) {
    \textbf{کارگر قدیم}\\[0.2cm]
    ملی‌سازی\\
    اتحادیه‌محور\\
    سوسیالیسم دولتی
};

\node[phase, fill=imperialred!15] (kinnock) at (5.5,0) {
    \textbf{اصلاحات کینوک}\\
    ۱۹۸۳-۱۹۹۲\\[0.2cm]
    حذف چپ افراطی\\
    مدرن‌سازی
};

\node[phase, fill=royalblue!25] (new) at (11,0) {
    \textbf{کارگر نو}\\
    ۱۹۹۴-\\[0.2cm]
    بازار + عدالت\\
    «راه سوم»\\
    حذف بند ۴
};

\draw[-{Stealth}, very thick] (old) -- (kinnock);
\draw[-{Stealth}, very thick] (kinnock) -- (new);

\node[below=0.5cm of kinnock, font=\small, text width=8cm, text centered] {
    شکست‌های ۱۹۷۹، ۱۹۸۳، ۱۹۸۷، ۱۹۹۲ $\rightarrow$ بازاندیشی بنیادین
};

\end{tikzpicture}
\caption{تحول حزب کارگر}
\end{figure}

\subsection{اصول «راه سوم»}

\begin{policybox}[راه سوم: فراتر از چپ و راست؟]
تونی بلر و آنتونی گیدنز «راه سوم» را به‌عنوان سنتز نوین مطرح کردند:
\begin{enumerate}
    \item \textbf{پذیرش بازار:} بازار کارآمد است، اما نیاز به تنظیم دارد
    \item \textbf{سرمایه‌گذاری اجتماعی:} به‌جای رفاه منفعل، توانمندسازی
    \item \textbf{فرصت‌برابری:} تمرکز بر آموزش و مهارت، نه بازتوزیع مستقیم
    \item \textbf{مسئولیت متقابل:} حقوق همراه با مسئولیت‌ها
    \item \textbf{جهانی‌شدن:} فرصت، نه تهدید
\end{enumerate}
\end{policybox}

\subsection{انتخابات ۱۹۹۷: زلزله سیاسی}

\begin{figure}[H]
\centering
\begin{tikzpicture}
\begin{axis}[
    ybar,
    width=12cm,
    height=7cm,
    ylabel={تعداد کرسی‌ها},
    symbolic x coords={کارگر, محافظه‌کار, لیبرال‌دموکرات, سایر},
    xtick=data,
    ymin=0,
    ymax=450,
    bar width=35pt,
    nodes near coords,
    title={\textbf{انتخابات ۱۹۹۷: پیروزی تاریخی}},
]

\addplot[fill=imperialred!60] coordinates {
    (کارگر, 418) (محافظه‌کار, 165) (لیبرال‌دموکرات, 46) (سایر, 30)
};

\end{axis}
\end{tikzpicture}
\caption{بزرگ‌ترین پیروزی کارگر در تاریخ}
\end{figure}

\subsection{دستاوردهای دولت بلر}

\begin{table}[H]
\centering
\caption{اصلاحات اصلی دولت‌های بلر/براون}
\renewcommand{\arraystretch}{1.4}
\begin{tabularx}{\textwidth}{>{\bfseries}p{3cm}Xc}
\toprule
\textbf{حوزه} & \textbf{اقدام} & \textbf{سال} \\
\midrule
\rowcolor{empirecream}
قانون اساسی & انتقال قدرت به اسکاتلند و ویلز؛ اصلاح مجلس اعیان & ۱۹۹۸-۹۹ \\

ایرلند شمالی & توافقنامه جمعه نیک & ۱۹۹۸ \\

\rowcolor{empirecream}
حداقل دستمزد & معرفی حداقل دستمزد ملی & ۱۹۹۹ \\

فقر کودکان & اعتبارات مالیاتی، Sure Start & ۱۹۹۹-۲۰۰۳ \\

\rowcolor{empirecream}
NHS & افزایش چشمگیر بودجه؛ کاهش زمان انتظار & ۲۰۰۰-۰۸ \\

آموزش & آکادمی‌ها، شهریه دانشگاه & ۲۰۰۲-۰۶ \\

\rowcolor{empirecream}
حقوق بشر & قانون حقوق بشر؛ قوانین برابری & ۱۹۹۸-۲۰۱۰ \\

\bottomrule
\end{tabularx}
\end{table}

\subsubsection{سرمایه‌گذاری در خدمات عمومی}

\begin{figure}[H]
\centering
\begin{tikzpicture}
\begin{axis}[
    width=13cm,
    height=7cm,
    xlabel={سال},
    ylabel={هزینه واقعی (۱۹۹۷=۱۰۰)},
    xmin=1997, xmax=2010,
    ymin=90, ymax=200,
    grid=major,
    legend pos=north west,
]

\addplot[very thick, imperialred, mark=*] coordinates {
    (1997, 100) (1999, 110) (2001, 125) (2003, 145)
    (2005, 165) (2007, 180) (2009, 195)
};
\addlegendentry{NHS}

\addplot[very thick, royalblue, mark=square*] coordinates {
    (1997, 100) (1999, 105) (2001, 115) (2003, 130)
    (2005, 145) (2007, 155) (2009, 165)
};
\addlegendentry{آموزش}

\end{axis}
\end{tikzpicture}
\caption{افزایش هزینه خدمات عمومی در دوره بلر/براون}
\end{figure}

\subsection{عراق و افول بلر}

\begin{warningbox}
\textbf{میراث متناقض بلر:}
\begin{itemize}
    \item \textbf{دستاوردها:} کاهش فقر کودکان، بهبود NHS، صلح ایرلند شمالی
    \item \textbf{شکست‌ها:} جنگ عراق، نابرابری همچنان بالا، مقررات‌زدایی مالی
    \item \textbf{میراث:} پرسش درباره اینکه آیا «راه سوم» واقعاً متفاوت بود یا تاچریسم با چهره انسانی
\end{itemize}
\end{warningbox}

%----------------------------------------------------------------------
\section{ریاضت، برگزیت و بحران (۲۰۱۰-۲۰۲۴)}
%----------------------------------------------------------------------

\subsection{بحران مالی ۲۰۰۸ و پیامدها}

\begin{figure}[H]
\centering
\begin{tikzpicture}
\begin{axis}[
    width=14cm,
    height=7cm,
    xlabel={سال},
    ylabel={رشد GDP (\%)},
    xmin=2006, xmax=2015,
    ymin=-6, ymax=4,
    grid=major,
]

\addplot[very thick, imperialred, mark=*] coordinates {
    (2006, 2.5) (2007, 2.4) (2008, -0.3) (2009, -4.2)
    (2010, 1.9) (2011, 1.5) (2012, 1.5) (2013, 2.1) (2014, 2.6)
};

\draw[dashed, thick, industrialgray] (axis cs:2008,-6) -- (axis cs:2008,4);
\node[above, font=\footnotesize] at (axis cs:2008,3) {سقوط لیمن برادرز};

\end{axis}
\end{tikzpicture}
\caption{رکود بزرگ ۲۰۰۸-۲۰۰۹}
\end{figure}

\subsection{ریاضت اقتصادی (۲۰۱۰-۲۰۱۹)}

\begin{table}[H]
\centering
\caption{اقدامات ریاضتی دولت ائتلافی/محافظه‌کار}
\renewcommand{\arraystretch}{1.4}
\begin{tabularx}{\textwidth}{>{\bfseries}cX}
\toprule
\textbf{حوزه} & \textbf{کاهش‌ها/تغییرات} \\
\midrule
\rowcolor{empirecream}
رفاه & سقف مزایا، «مالیات اتاق خواب»، Universal Credit \\

دولت محلی & کاهش ۴۰٪ بودجه شوراها \\

\rowcolor{empirecream}
پلیس & کاهش ۲۰,۰۰۰ افسر \\

آموزش & توقف افزایش بودجه مدارس \\

\rowcolor{empirecream}
NHS & رشد بودجه به کمترین میزان تاریخی \\

\bottomrule
\end{tabularx}
\end{table}

\begin{figure}[H]
\centering
\begin{tikzpicture}
\begin{axis}[
    width=13cm,
    height=7cm,
    xlabel={سال},
    ylabel={هزینه عمومی (\% GDP)},
    xmin=2008, xmax=2020,
    ymin=38, ymax=48,
    grid=major,
]

\addplot[very thick, tudorpurple, mark=*] coordinates {
    (2008, 41) (2009, 47) (2010, 46) (2011, 45)
    (2012, 44) (2013, 43) (2014, 42) (2015, 41)
    (2016, 40) (2017, 40) (2018, 40) (2019, 40)
};

\end{axis}
\end{tikzpicture}
\caption{کاهش سهم هزینه عمومی از GDP}
\end{figure}

\subsection{برگزیت: رأی‌گیری و پیامدها}

\begin{keybox}[رفراندوم ۲۳ ژوئن ۲۰۱۶]
\begin{itemize}
    \item \textbf{خروج:} ۵۱.۹٪ (۱۷.۴ میلیون)
    \item \textbf{ماندن:} ۴۸.۱٪ (۱۶.۱ میلیون)
    \item \textbf{مشارکت:} ۷۲.۲٪
\end{itemize}
\end{keybox}

\subsubsection{جغرافیای برگزیت}

\begin{table}[H]
\centering
\caption{الگوی رأی‌گیری برگزیت}
\renewcommand{\arraystretch}{1.3}
\begin{tabularx}{\textwidth}{>{\bfseries}p{3.5cm}XX}
\toprule
\textbf{متغیر} & \textbf{رأی به خروج} & \textbf{رأی به ماندن} \\
\midrule
\rowcolor{empirecream}
منطقه & انگلستان (بدون لندن)، ویلز & لندن، اسکاتلند، ایرلند شمالی \\

سن & مسن‌تر (۶۵+: ۶۰٪ خروج) & جوان‌تر (۱۸-۲۴: ۷۳٪ ماندن) \\

\rowcolor{empirecream}
تحصیلات & بدون مدرک دانشگاهی & دارای مدرک \\

طبقه & طبقه کارگر & طبقه متوسط حرفه‌ای \\

\rowcolor{empirecream}
محل سکونت & شهرهای کوچک، روستاها & شهرهای بزرگ \\

\bottomrule
\end{tabularx}
\end{table}

\begin{figure}[H]
\centering
\begin{tikzpicture}
\begin{axis}[
    ybar,
    width=13cm,
    height=7cm,
    ylabel={درصد رأی به خروج},
    symbolic x coords={لندن, اسکاتلند, ویلز, شمال انگلستان, میدلندز, جنوب},
    xtick=data,
    x tick label style={rotate=45, anchor=east},
    ymin=30, ymax=65,
    bar width=25pt,
    nodes near coords,
]

\addplot[fill=imperialred!60] coordinates {
    (لندن, 40.1) (اسکاتلند, 38.0) (ویلز, 52.5) 
    (شمال انگلستان, 58.0) (میدلندز, 59.2) (جنوب, 51.8)
};

\end{axis}
\end{tikzpicture}
\caption{توزیع منطقه‌ای رأی به خروج}
\end{figure}

\subsubsection{تفسیرهای برگزیت}

\begin{figure}[H]
\centering
\begin{tikzpicture}[
    interp/.style={rectangle, rounded corners=8pt, draw, thick,
                   minimum width=5cm, minimum height=2.5cm,
                   text centered, text width=4.8cm}
]

\node[interp, fill=imperialred!15] at (0,0) {
    \textbf{تفسیر اقتصادی}\\[0.2cm]
    \footnotesize
    «بازماندگان» جهانی‌شدن\\
    نابرابری منطقه‌ای\\
    ریاضت
};

\node[interp, fill=royalblue!15] at (6,0) {
    \textbf{تفسیر فرهنگی}\\[0.2cm]
    \footnotesize
    هویت ملی\\
    نگرانی از مهاجرت\\
    «کنترل»
};

\node[interp, fill=victoriangreen!15] at (12,0) {
    \textbf{تفسیر سیاسی}\\[0.2cm]
    \footnotesize
    بی‌اعتمادی به نخبگان\\
    شکاف دموکراتیک\\
    پوپولیسم
};

\end{tikzpicture}
\caption{سه تفسیر اصلی از برگزیت}
\end{figure}

\subsection{بحران‌های ۲۰۲۰-۲۰۲۴}

\begin{table}[H]
\centering
\caption{بحران‌های متوالی}
\renewcommand{\arraystretch}{1.4}
\begin{tabularx}{\textwidth}{>{\bfseries}cp{3cm}X}
\toprule
\textbf{سال} & \textbf{بحران} & \textbf{پیامد} \\
\midrule
\rowcolor{empirecream}
۲۰۲۰ & کووید-۱۹ & ۲۰۰,۰۰۰+ مرگ؛ قرنطینه؛ بدهی دولتی \\

۲۰۲۱ & بحران زنجیره تأمین & کمبود رانندگان، کالاها \\

\rowcolor{empirecream}
۲۰۲۲ & بحران انرژی و تورم & تورم ۱۱٪؛ بحران هزینه زندگی \\

۲۰۲۲ & بحران تراس & ۴۴ روز نخست‌وزیری؛ سقوط پوند \\

\rowcolor{empirecream}
۲۰۲۳-۲۴ & رکود + اعتصابات & NHS، راه‌آهن، معلمان \\

\bottomrule
\end{tabularx}
\end{table}

%%%%%%%%%%%%%%%%%%%%%%%%%%%%%%%%%%%%%%%%%%%%%%%%%%%%%%%%%%%%%%%%%%%%%%%
% ادامه فصل هفتم: از تاچر تا برگزیت (۱۹۷۹-۲۰۲۴)
%%%%%%%%%%%%%%%%%%%%%%%%%%%%%%%%%%%%%%%%%%%%%%%%%%%%%%%%%%%%%%%%%%%%%%%

%----------------------------------------------------------------------
\section{تحلیل نهایی فصل هفتم}
%----------------------------------------------------------------------

\subsection{میراث تاچریسم: ارزیابی ۴۵ ساله}

\begin{table}[H]
\centering
\caption{ترازنامه تاچریسم}
\renewcommand{\arraystretch}{1.4}
\begin{tabularx}{\textwidth}{>{\bfseries\color{victoriangreen}}X>{\bfseries\color{imperialred}}X}
\toprule
\textbf{دستاوردها (از منظر طرفداران)} & \textbf{شکست‌ها (از منظر منتقدان)} \\
\midrule
\rowcolor{empirecream}
مهار تورم و ثبات اقتصادی & افزایش شدید نابرابری \\
افزایش بهره‌وری و رقابت‌پذیری & تخریب صنایع و جوامع محلی \\
\rowcolor{empirecream}
گسترش مالکیت خانه و سهام & بی‌ثباتی مسکن، کاهش مسکن اجتماعی \\
انعطاف‌پذیری بازار کار & شغل‌های ناامن، دستمزدهای پایین \\
\rowcolor{empirecream}
کاهش قدرت اتحادیه‌های غیرپاسخگو & تضعیف صدای کارگران \\
نوسازی اقتصاد & مالی‌سازی بیش از حد \\
\bottomrule
\end{tabularx}
\end{table}

\subsection{چرخش چپ و راست: همگرایی ایدئولوژیک}

\begin{figure}[H]
\centering
\begin{tikzpicture}[
    scale=0.9,
    party/.style={circle, draw, very thick, minimum size=1.5cm,
                  text centered, font=\small}
]

% محور
\draw[very thick, -{Stealth}] (-7,0) -- (7,0) node[right] {راست};
\draw[very thick, -{Stealth}] (0,-4) -- (0,4) node[above] {دولت‌گرا};
\node[left] at (-7,0) {چپ};
\node[below] at (0,-4) {بازارگرا};

% موقعیت احزاب در زمان‌های مختلف
\node[party, fill=imperialred!40] (lab79) at (-4,2) {کارگر\\۱۹۷۹};
\node[party, fill=imperialred!40] (lab97) at (-1,-0.5) {کارگر\\۱۹۹۷};
\node[party, fill=tudorpurple!40] (con79) at (2,-1) {محافظه‌کار\\۱۹۷۹};
\node[party, fill=tudorpurple!40] (con19) at (3,1) {محافظه‌کار\\۲۰۱۹};

% فلش‌های حرکت
\draw[-{Stealth}, thick, dashed, imperialred] (lab79) -- (lab97);
\draw[-{Stealth}, thick, dashed, tudorpurple] (con79) -- (con19);

% ناحیه اجماع تاچری
\draw[rounded corners, fill=parliamentgold!20, draw=parliamentgold, thick] 
    (-2,-2) rectangle (4,1);
\node[font=\footnotesize] at (1,-0.5) {ناحیه اجماع نئولیبرال};

\end{tikzpicture}
\caption{حرکت احزاب در طیف سیاسی (۱۹۷۹-۲۰۱۹)}
\end{figure}

\begin{policybox}[الگوی ۱: هژمونی ایدئولوژیک]
تاچریسم نه فقط سیاست‌ها، بلکه \textbf{چارچوب فکری} غالب را تغییر داد:
\begin{itemize}
    \item حتی حزب کارگر بلر اصول بازار آزاد را پذیرفت
    \item «چیزی به نام جایگزین وجود ندارد» (TINA) فرض پیش‌فرض شد
    \item مخالفان مجبور شدند در چارچوب تاچری استدلال کنند
\end{itemize}
این همان چیزی است که گرامشی «هژمونی» می‌نامید.
\end{policybox}

\subsection{شکاف‌های ساختاری باقی‌مانده}

\begin{figure}[H]
\centering
\begin{tikzpicture}[
    gap/.style={rectangle, rounded corners=10pt, draw=imperialred, very thick,
                fill=imperialred!15, minimum width=4.5cm,
                minimum height=2.5cm, text centered, text width=4.3cm}
]

\node[gap] at (0,0) {
    \textbf{شکاف منطقه‌ای}\\[0.2cm]
    لندن vs بقیه\\
    شمال vs جنوب\\
    «دیوار قرمز» فروریخته
};

\node[gap] at (5.5,0) {
    \textbf{شکاف نسلی}\\[0.2cm]
    مالکیت مسکن\\
    مستمری vs دستمزد\\
    برگزیت
};

\node[gap] at (11,0) {
    \textbf{شکاف آموزشی}\\[0.2cm]
    دانشگاه‌رفته vs نرفته\\
    «هرجایی‌ها vs جایی‌ها»\\
    (Anywheres vs Somewheres)
};

\node[gap] at (2.75,-4) {
    \textbf{شکاف هویتی}\\[0.2cm]
    چندفرهنگی vs ملی‌گرایی\\
    جهان‌وطن vs محلی\\
    اروپا
};

\node[gap] at (8.25,-4) {
    \textbf{شکاف طبقاتی نو}\\[0.2cm]
    سرمایه vs کار\\
    صاحبان دارایی vs بی‌دارایی\\
    ناامنی شغلی
};

\end{tikzpicture}
\caption{شکاف‌های ساختاری بریتانیای معاصر}
\end{figure}

\subsection{شاخص‌های کلیدی ۱۹۷۹-۲۰۲۴}

\begin{table}[H]
\centering
\caption{تغییرات بلندمدت: ۱۹۷۹ تا ۲۰۲۴}
\renewcommand{\arraystretch}{1.3}
\begin{tabular}{lccc}
\toprule
\textbf{شاخص} & \textbf{۱۹۷۹} & \textbf{۲۰۲۴} & \textbf{تغییر} \\
\midrule
\rowcolor{empirecream}
GDP سرانه (دلار ۲۰۱۵) & ۱۸,۰۰۰ & ۴۵,۰۰۰ & +۱۵۰٪ \\
ضریب جینی & ۰.۲۵ & ۰.۳۵ & +۴۰٪ \\
\rowcolor{empirecream}
عضویت اتحادیه‌ای & ۵۵٪ & ۲۳٪ & $-۵۸٪$ \\
مالکیت خانه & ۵۵٪ & ۶۵٪ & +۱۸٪ \\
\rowcolor{empirecream}
بدهی دولتی (\% GDP) & ۴۵٪ & ۱۰۰٪ & +۱۲۲٪ \\
بخش خدمات (\% GDP) & ۵۵٪ & ۸۰٪ & +۴۵٪ \\
\rowcolor{empirecream}
جمعیت متولد خارج & ۶٪ & ۱۵٪ & +۱۵۰٪ \\
امید به زندگی & ۷۳ سال & ۸۱ سال & +۱۱٪ \\
\bottomrule
\end{tabular}
\end{table}

\begin{figure}[H]
\centering
\begin{tikzpicture}
\begin{axis}[
    width=14cm,
    height=8cm,
    xlabel={سال},
    ylabel={شاخص (۱۹۷۹ = ۱۰۰)},
    xmin=1979, xmax=2024,
    ymin=50, ymax=300,
    grid=major,
    legend pos=north west,
]

\addplot[very thick, victoriangreen, mark=*] coordinates {
    (1979, 100) (1985, 115) (1990, 135) (1995, 145)
    (2000, 175) (2005, 200) (2010, 210) (2015, 230)
    (2020, 240) (2024, 250)
};
\addlegendentry{GDP سرانه واقعی}

\addplot[very thick, royalblue, mark=square*] coordinates {
    (1979, 100) (1985, 105) (1990, 115) (1995, 115)
    (2000, 130) (2005, 135) (2010, 135) (2015, 130)
    (2020, 135) (2024, 130)
};
\addlegendentry{دستمزد واقعی میانه}

\addplot[very thick, imperialred, mark=triangle*] coordinates {
    (1979, 100) (1985, 200) (1990, 350) (1995, 400)
    (2000, 500) (2005, 600) (2010, 550) (2015, 650)
    (2020, 700) (2024, 600)
};
\addlegendentry{قیمت واقعی مسکن}

\end{axis}
\end{tikzpicture}
\caption{واگرایی درآمد، دستمزد و مسکن}
\end{figure}

%----------------------------------------------------------------------
\section*{منابع اصلی فصل هفتم}
%----------------------------------------------------------------------

\begin{enumerate}[label={[\arabic*]}]
    \item Campbell, John (2003). \textit{Margaret Thatcher: The Iron Lady}. London: Jonathan Cape.
    \item Gamble, Andrew (1994). \textit{The Free Economy and the Strong State}. 2nd ed. London: Macmillan.
    \item Harvey, David (2005). \textit{A Brief History of Neoliberalism}. Oxford: OUP.
    \item Giddens, Anthony (1998). \textit{The Third Way}. Cambridge: Polity Press.
    \item Seldon, Anthony, ed. (2007). \textit{Blair''s Britain 1997-2007}. Cambridge: CUP.
    \item Toynbee, Polly \& Walker, David (2020). \textit{The Lost Decade}. London: Guardian Faber.
    \item Shipman, Tim (2016). \textit{All Out War: The Full Story of Brexit}. London: Collins.
    \item Goodhart, David (2017). \textit{The Road to Somewhere}. London: Hurst.
    \item Evans, Geoffrey \& Menon, Anand (2017). \textit{Brexit and British Politics}. Cambridge: Polity.
    \item Piketty, Thomas (2014). \textit{Capital in the Twenty-First Century}. Cambridge, MA: Harvard UP.
\end{enumerate}


%%%%%%%%%%%%%%%%%%%%%%%%%%%%%%%%%%%%%%%%%%%%%%%%%%%%%%%%%%%%%%%%%%%%%%%
% فصل هشتم: تحلیل تطبیقی
%%%%%%%%%%%%%%%%%%%%%%%%%%%%%%%%%%%%%%%%%%%%%%%%%%%%%%%%%%%%%%%%%%%%%%%

\chapter{تحلیل تطبیقی: بریتانیا در آینه دیگران}

\begin{keybox}[خلاصه فصل]
این فصل تجربه بریتانیا را با سایر کشورها مقایسه می‌کند تا \textbf{«استثناگرایی 
بریتانیایی»} را بسنجد. آیا مسیر توسعه بریتانیا واقعاً منحصر به فرد بود؟ 
چه عواملی تفاوت‌ها را توضیح می‌دهند؟ این مقایسه شامل: ۱) مقایسه با اروپای 
قاره‌ای (فرانسه، آلمان)، ۲) مقایسه با «نوادگان» (آمریکا، استرالیا)، و 
۳) مقایسه با رقبای جدید (آسیای شرقی) است.
\end{keybox}

%----------------------------------------------------------------------
\section{چارچوب تحلیل تطبیقی}
%----------------------------------------------------------------------

\subsection{روش‌شناسی}

\begin{figure}[H]
\centering
\begin{tikzpicture}[
    method/.style={rectangle, rounded corners=8pt, draw=royalblue, thick,
                   fill=royalblue!15, minimum width=5cm,
                   minimum height=2cm, text centered, text width=4.8cm}
]

\node[method] at (0,0) {
    \textbf{مقایسه «بیشترین شباهت»}\\[0.2cm]
    \footnotesize
    بریتانیا vs فرانسه/آلمان\\
    چرا نتایج متفاوت با شرایط مشابه؟
};

\node[method] at (6.5,0) {
    \textbf{مقایسه «بیشترین تفاوت»}\\[0.2cm]
    \footnotesize
    بریتانیا vs ژاپن/چین\\
    آیا الگوهای مشترک هست؟
};

\node[method] at (13,0) {
    \textbf{تحلیل وابستگی به مسیر}\\[0.2cm]
    \footnotesize
    چگونه انتخاب‌های اولیه\\
    آینده را شکل دادند؟
};

\end{tikzpicture}
\caption{سه رویکرد روش‌شناختی}
\end{figure}

%----------------------------------------------------------------------
\section{بریتانیا و فرانسه: دو مسیر متفاوت}
%----------------------------------------------------------------------

\subsection{تفاوت‌های بنیادین}

\begin{table}[H]
\centering
\caption{مقایسه تاریخی بریتانیا و فرانسه}
\renewcommand{\arraystretch}{1.4}
\begin{tabularx}{\textwidth}{>{\bfseries}p{3.5cm}XX}
\toprule
\textbf{بُعد} & \textbf{بریتانیا} & \textbf{فرانسه} \\
\midrule
\rowcolor{empirecream}
انقلاب & «شکوهمند» ۱۶۸۸ (مصالحه) & کبیر ۱۷۸۹ (قطع رادیکال) \\

پایداری سیاسی & یک رژیم از ۱۶۸۸ & ۵ جمهوری، ۲ امپراتوری، ۲ پادشاهی \\

\rowcolor{empirecream}
قانون اساسی & نانوشته، تکاملی & نوشته، چندبار بازنویسی \\

دولت & محدود، غیرمتمرکز & قوی، متمرکز \\

\rowcolor{empirecream}
رابطه دولت-اقتصاد & لسه‌فر & دیریژیسم (هدایت‌گری) \\

نخبگان & تداوم اشرافیت & قطع و جایگزینی متوالی \\

\rowcolor{empirecream}
مذهب & پروتستان، پلورالیسم & کاتولیک، لائیسیته \\

امپراتوری & دریایی، تجاری & قاره‌ای + آفریقایی \\

\bottomrule
\end{tabularx}
\end{table}

\subsection{مسیر دموکراتیزاسیون}

\begin{figure}[H]
\centering
\begin{tikzpicture}
\begin{axis}[
    width=14cm,
    height=8cm,
    xlabel={سال},
    ylabel={درصد جمعیت بالغ با حق رأی},
    xmin=1800, xmax=1950,
    ymin=0, ymax=105,
    grid=major,
    legend pos=south east,
]

% بریتانیا - تدریجی
\addplot[very thick, royalblue, mark=*] coordinates {
    (1800, 3) (1832, 7) (1867, 16) (1884, 28) 
    (1918, 75) (1928, 100)
};
\addlegendentry{بریتانیا}

% فرانسه - نوسانی
\addplot[very thick, imperialred, mark=square*] coordinates {
    (1800, 0) (1815, 1) (1830, 2) (1848, 100)
    (1852, 100) (1870, 100) (1944, 100)
};
\addlegendentry{فرانسه (مردان)}

% فرانسه زنان
\addplot[very thick, imperialred, mark=square*, dashed] coordinates {
    (1944, 100)
};

% آلمان
\addplot[very thick, industrialgray, mark=triangle*] coordinates {
    (1800, 0) (1849, 0) (1867, 50) (1871, 50)
    (1918, 100) (1933, 0) (1949, 100)
};
\addlegendentry{آلمان}

\end{axis}
\end{tikzpicture}
\caption{مسیرهای متفاوت دموکراتیزاسیون}
\end{figure}

\begin{policybox}[چرا بریتانیا انقلاب نکرد؟]
تاریخ‌نگاران دلایل مختلفی ارائه کرده‌اند:
\begin{enumerate}
    \item \textbf{تامپسون:} طبقه کارگر رادیکال بود، اما سرکوب و اصلاحات ترکیبی مؤثر بود
    \item \textbf{مور:} نخبگان انعطاف‌پذیر بودند و به‌موقع امتیاز دادند
    \item \textbf{پروزن:} جغرافیای جزیره‌ای از تهاجم محافظت کرد
    \item \textbf{کلارک:} فرهنگ سیاسی بریتانیایی ذاتاً محافظه‌کار و پراگماتیست بود
    \item \textbf{مارکسیست‌ها:} بورژوازی با اشرافیت ائتلاف کرد، نه با کارگران
\end{enumerate}
\end{policybox}

%----------------------------------------------------------------------
\section{بریتانیا و آلمان: دو مدل سرمایه‌داری}
%----------------------------------------------------------------------

\subsection{تفاوت‌های ساختاری}

\begin{table}[H]
\centering
\caption{دو نوع سرمایه‌داری: لیبرال vs هماهنگ‌شده}
\renewcommand{\arraystretch}{1.4}
\begin{tabularx}{\textwidth}{>{\bfseries}p{3.5cm}XX}
\toprule
\textbf{بُعد} & \textbf{بریتانیا (لیبرال)} & \textbf{آلمان (هماهنگ‌شده)} \\
\midrule
\rowcolor{empirecream}
تأمین مالی شرکت‌ها & بازار سهام & بانک‌های ارتباطی \\

افق زمانی & کوتاه‌مدت، سود سهام & بلندمدت، رشد \\

\rowcolor{empirecream}
روابط صنعتی & تقابلی، فاصله‌دار & مشارکتی، شورای کارگری \\

آموزش حرفه‌ای & ضعیف، درون‌شرکتی & قوی، نظام دوگانه \\

\rowcolor{empirecream}
تنظیم بازار کار & انعطاف‌پذیر & محافظت‌شده \\

نقش دولت در صنعت & حداقلی & فعال (Mittelstand) \\

\rowcolor{empirecream}
صادرات/GDP & ۳۰٪ & ۴۷٪ \\

\bottomrule
\end{tabularx}
\end{table}

\begin{figure}[H]
\centering
\begin{tikzpicture}
\begin{axis}[
    width=13cm,
    height=7cm,
    xlabel={سال},
    ylabel={سهم صنعت از GDP (\%)},
    xmin=1970, xmax=2020,
    ymin=5, ymax=40,
    grid=major,
    legend pos=north east,
]

\addplot[very thick, royalblue, mark=*] coordinates {
    (1970, 32) (1980, 26) (1990, 22) (2000, 18)
    (2010, 12) (2020, 10)
};
\addlegendentry{بریتانیا}

\addplot[very thick, industrialgray, mark=square*] coordinates {
    (1970, 38) (1980, 33) (1990, 30) (2000, 25)
    (2010, 22) (2020, 20)
};
\addlegendentry{آلمان}

\end{axis}
\end{tikzpicture}
\caption{صنعت‌زدایی: بریتانیا vs آلمان}
\end{figure}

\begin{warningbox}
\textbf{پارادوکس بریتانیا:}

بریتانیا پیشگام صنعتی شدن بود، اما زودتر از همه صنعت‌زدایی کرد. چرا؟
\begin{itemize}
    \item فقدان استراتژی صنعتی دولتی
    \item تمرکز بر خدمات مالی
    \item ارزش‌گذاری بیش از حد پوند در دوره‌هایی
    \item کمبود سرمایه‌گذاری در آموزش فنی
    \item «بیماری هلندی» (نفت دریای شمال)
\end{itemize}
\end{warningbox}

%----------------------------------------------------------------------
\section{بریتانیا و کشورهای نوردیک}
%----------------------------------------------------------------------

\subsection{مقایسه مدل‌های رفاهی}

\begin{figure}[H]
\centering
\begin{tikzpicture}[
    model/.style={rectangle, rounded corners=10pt, draw, very thick,
                  minimum width=4.5cm, minimum height=3.5cm,
                  text centered, text width=4.3cm}
]

\node[model, fill=royalblue!20, draw=royalblue] (lib) at (0,0) {
    \textbf{مدل لیبرال}\\
    (بریتانیا، آمریکا)\\[0.2cm]
    \footnotesize
    مزایای کم\\
    آزمون نیازمندی\\
    بازار محور\\
    نابرابری بالا
};

\node[model, fill=victoriangreen!20, draw=victoriangreen] (soc) at (5.5,0) {
    \textbf{مدل سوسیال‌دموکرات}\\
    (سوئد، دانمارک)\\[0.2cm]
    \footnotesize
    مزایای سخاوتمند\\
    همگانی\\
    دولت فعال\\
    نابرابری کم
};

\node[model, fill=parliamentgold!20, draw=parliamentgold] (corp) at (11,0) {
    \textbf{مدل کورپوراتیست}\\
    (آلمان، فرانسه)\\[0.2cm]
    \footnotesize
    مزایای متوسط\\
    مبتنی بر شغل\\
    شراکت اجتماعی\\
    نابرابری متوسط
};

\end{tikzpicture}
\caption{سه‌گونه‌شناسی دولت‌های رفاه (اسپینگ-آندرسن)}
\end{figure}

\begin{table}[H]
\centering
\caption{شاخص‌های مقایسه‌ای (۲۰۲۰)}
\renewcommand{\arraystretch}{1.3}
\begin{tabular}{lccccc}
\toprule
\textbf{شاخص} & \textbf{بریتانیا} & \textbf{سوئد} & \textbf{آلمان} & \textbf{فرانسه} & \textbf{آمریکا} \\
\midrule
\rowcolor{empirecream}
هزینه اجتماعی (\% GDP) & ۲۱٪ & ۲۶٪ & ۲۶٪ & ۳۱٪ & ۱۹٪ \\
ضریب جینی & ۰.۳۵ & ۰.۲۷ & ۰.۲۹ & ۰.۲۹ & ۰.۳۹ \\
\rowcolor{empirecream}
فقر کودکان (\%) & ۲۰٪ & ۹٪ & ۱۱٪ & ۱۳٪ & ۱۸٪ \\
امید به زندگی & ۸۱ & ۸۳ & ۸۱ & ۸۳ & ۷۷ \\
\rowcolor{empirecream}
اعتماد به دولت (\%) & ۳۵٪ & ۶۵٪ & ۵۰٪ & ۳۵٪ & ۲۰٪ \\
\bottomrule
\end{tabular}
\end{table}

%----------------------------------------------------------------------
\section{بریتانیا و آمریکا: «رابطه ویژه»}
%----------------------------------------------------------------------

\subsection{شباهت‌ها و تفاوت‌ها}

\begin{table}[H]
\centering
\caption{مقایسه بریتانیا و ایالات متحده}
\renewcommand{\arraystretch}{1.4}
\begin{tabularx}{\textwidth}{>{\bfseries}p{3cm}XX}
\toprule
\textbf{بُعد} & \textbf{بریتانیا} & \textbf{آمریکا} \\
\midrule
\rowcolor{empirecream}
قانون اساسی & نانوشته، پارلمانی & نوشته، ریاستی \\

فدرالیسم & دولت واحد (با انتقال قدرت) & فدرال \\

\rowcolor{empirecream}
بهداشت & NHS (همگانی) & خصوصی + Medicare/Medicaid \\

نقش دولت & بیشتر & کمتر \\

\rowcolor{empirecream}
اتحادیه‌ها & ضعیف‌شده اما موجود & بسیار ضعیف \\

نابرابری & بالا & بسیار بالا \\

\rowcolor{empirecream}
سیستم انتخاباتی & اکثریتی (مشابه) & اکثریتی (مشابه) \\

\bottomrule
\end{tabularx}
\end{table}

\begin{policybox}[همگرایی تاچر-ریگان]
دهه ۱۹۸۰ شاهد «انقلاب نئولیبرال» همزمان در هر دو کشور بود:
\begin{itemize}
    \item کاهش مالیات بر ثروتمندان
    \item مقررات‌زدایی
    \item ضدیت با اتحادیه‌ها
    \item خصوصی‌سازی (در بریتانیا بیشتر)
\end{itemize}
این «مدل آنگلو-ساکسون» در مقابل «مدل اروپای قاره‌ای» قرار گرفت.
\end{policybox}

%----------------------------------------------------------------------
\section{بریتانیا و آسیای شرقی}
%----------------------------------------------------------------------

\subsection{دولت توسعه‌گرا vs دولت لیبرال}

\begin{table}[H]
\centering
\caption{مقایسه مدل‌های توسعه}
\renewcommand{\arraystretch}{1.4}
\begin{tabularx}{\textwidth}{>{\bfseries}p{3.5cm}XX}
\toprule
\textbf{بُعد} & \textbf{بریتانیا} & \textbf{آسیای شرقی (ژاپن، کره، چین)} \\
\midrule
\rowcolor{empirecream}
نقش دولت & تسهیل‌گر & هدایت‌گر، برنامه‌ریز \\

استراتژی صنعتی & حداقلی / نامنسجم & فعال، هدفمند \\

\rowcolor{empirecream}
تجارت & آزاد & حمایتی استراتژیک \\

آموزش & عمومی، کم‌تخصص & فنی، پرتخصص \\

\rowcolor{empirecream}
نرخ پس‌انداز & پایین (۱۵٪) & بالا (۳۰-۴۰٪) \\

افق زمانی & کوتاه & بلند \\

\bottomrule
\end{tabularx}
\end{table}

\begin{figure}[H]
\centering
\begin{tikzpicture}
\begin{axis}[
    width=14cm,
    height=8cm,
    xlabel={سال},
    ylabel={GDP سرانه (دلار ۲۰۱۱، PPP)},
    xmin=1950, xmax=2020,
    ymin=0, ymax=55000,
    grid=major,
    legend pos=north west,
]

\addplot[very thick, royalblue, mark=*] coordinates {
    (1950, 9000) (1960, 11000) (1970, 14000) (1980, 17000)
    (1990, 23000) (2000, 30000) (2010, 35000) (2020, 42000)
};
\addlegendentry{بریتانیا}

\addplot[very thick, imperialred, mark=square*] coordinates {
    (1950, 2000) (1960, 4500) (1970, 12000) (1980, 19000)
    (1990, 28000) (2000, 32000) (2010, 38000) (2020, 43000)
};
\addlegendentry{ژاپن}

\addplot[very thick, victoriangreen, mark=triangle*] coordinates {
    (1950, 1000) (1960, 1500) (1970, 3000) (1980, 5000)
    (1990, 12000) (2000, 20000) (2010, 32000) (2020, 44000)
};
\addlegendentry{کره جنوبی}

\addplot[very thick, parliamentgold, mark=diamond*] coordinates {
    (1950, 500) (1960, 600) (1970, 700) (1980, 1000)
    (1990, 2000) (2000, 4000) (2010, 10000) (2020, 17000)
};
\addlegendentry{چین}

\end{axis}
\end{tikzpicture}
\caption{همگرایی اقتصادی: رسیدن آسیا}
\end{figure}

%----------------------------------------------------------------------
\section{جمع‌بندی تطبیقی: «استثناگرایی» بریتانیایی}
%----------------------------------------------------------------------

\subsection{چه چیزی واقعاً استثنایی است؟}

\begin{table}[H]
\centering
\caption{ارزیابی استثناگرایی بریتانیایی}
\renewcommand{\arraystretch}{1.4}
\begin{tabularx}{\textwidth}{>{\bfseries}p{4cm}cc}
\toprule
\textbf{ویژگی ادعاشده} & \textbf{واقعاً استثنایی؟} & \textbf{توضیح} \\
\midrule
\rowcolor{empirecream}
قانون اساسی نانوشته & تقریباً & اسرائیل و نیوزیلند هم \\

ثبات سیاسی طولانی & بله & از ۱۶۸۸ بدون انقلاب \\

\rowcolor{empirecream}
صنعتی شدن اول & بله & پیشگام بلامنازع \\

امپراتوری بزرگ‌تر & بله & بزرگ‌ترین در تاریخ \\

\rowcolor{empirecream}
فروپاشی امپراتوری مسالمت‌آمیز & نسبی & هند بله، کنیا/مالایا خیر \\

تدریجی‌گرایی & نسبی & فرانسه و آلمان نه، اسکاندیناوی بله \\

\rowcolor{empirecream}
ضعف چپ رادیکال & بله & حزب کمونیست همیشه حاشیه‌ای \\

\bottomrule
\end{tabularx}
\end{table}

\begin{figure}[H]
\centering
\begin{tikzpicture}[
    conclusion/.style={rectangle, rounded corners=10pt, draw=royalblue, very thick,
                       fill=royalblue!15, minimum width=12cm,
                       minimum height=3cm, text centered, text width=11.5cm}
]

\node[conclusion] {
    \textbf{نتیجه تحلیل تطبیقی:}\\[0.3cm]
    بریتانیا نه کاملاً استثنایی است و نه کاملاً معمولی. ترکیب خاصی از عوامل 
    -- جغرافیای جزیره‌ای، پیشگامی صنعتی، امپراتوری، فرهنگ سیاسی پراگماتیست، 
    و ائتلاف نخبگان -- مسیری متمایز (اما نه منحصر به فرد) ایجاد کرد.
    کشورهای دیگر از جنبه‌های مختلف به بریتانیا شبیه‌اند.
};

\end{tikzpicture}
\end{figure}

%----------------------------------------------------------------------
\section*{منابع اصلی فصل هشتم}
%----------------------------------------------------------------------

\begin{enumerate}[label={[\arabic*]}]
    \item Esping-Andersen, Gøsta (1990). \textit{The Three Worlds of Welfare Capitalism}. Princeton: PUP.
    \item Hall, Peter \& Soskice, David, eds. (2001). \textit{Varieties of Capitalism}. Oxford: OUP.
    \item Moore, Barrington Jr. (1966). \textit{Social Origins of Dictatorship and Democracy}. Boston: Beacon.
    \item Wiener, Martin J. (1981). \textit{English Culture and the Decline of the Industrial Spirit}. Cambridge: CUP.
    \item Hobsbawm, Eric (1968). \textit{Industry and Empire}. London: Weidenfeld.
    \item Johnson, Chalmers (1982). \textit{MITI and the Japanese Miracle}. Stanford: SUP.
    \item Pierson, Paul (1994). \textit{Dismantling the Welfare State?} Cambridge: CUP.
    \item Pontusson, Jonas (2005). \textit{Inequality and Prosperity}. Ithaca: Cornell UP.
    \item Acemoglu, Daron \& Robinson, James (2012). \textit{Why Nations Fail}. New York: Crown.
    \item North, Douglass C. (1990). \textit{Institutions, Institutional Change and Economic Performance}. Cambridge: CUP.
\end{enumerate}


%%%%%%%%%%%%%%%%%%%%%%%%%%%%%%%%%%%%%%%%%%%%%%%%%%%%%%%%%%%%%%%%%%%%%%%
% فصل نهم: نتیجه‌گیری و الگوهای کاربردی
%%%%%%%%%%%%%%%%%%%%%%%%%%%%%%%%%%%%%%%%%%%%%%%%%%%%%%%%%%%%%%%%%%%%%%%

\chapter{نتیجه‌گیری: الگوها، درس‌ها و چشم‌انداز}

\begin{keybox}[خلاصه فصل]
این فصل پایانی، یافته‌های پژوهش را جمع‌بندی کرده و \textbf{الگوهای کلیدی} 
قابل استخراج از تجربه بریتانیا را شناسایی می‌کند. این الگوها نه به‌عنوان 
«نسخه‌های تجویزی»، بلکه به‌مثابه \textbf{ابزارهای تحلیلی مشروط به زمینه} 
ارائه می‌شوند. در پایان، چشم‌اندازی از آینده بریتانیا ترسیم می‌شود.
\end{keybox}

%----------------------------------------------------------------------
\section{خلاصه روایت تاریخی}
%----------------------------------------------------------------------

\begin{figure}[H]
\centering
\begin{tikzpicture}[
    scale=0.8,
    era/.style={rectangle, rounded corners=5pt, draw, thick,
                minimum height=1.5cm, text centered, font=\small}
]

% خط زمانی
\draw[very thick, royalblue] (0,0) -- (18,0);

% دوره‌ها
\node[era, fill=tudorpurple!20, text width=2.5cm] at (1.5,2) {فئودالیسم\\پارلمان\\۱۰۶۶-۱۴۸۵};
\node[era, fill=parliamentgold!20, text width=2.5cm] at (4.5,2) {تودور-استوارت\\انقلاب\\۱۴۸۵-۱۷۱۴};
\node[era, fill=victoriangreen!20, text width=2.5cm] at (7.5,2) {الیگارشی\\صنعت\\۱۷۱۴-۱۸۳۲};
\node[era, fill=imperialred!20, text width=2.5cm] at (10.5,2) {ویکتوریا\\امپراتوری\\۱۸۳۲-۱۹۱۴};
\node[era, fill=royalblue!20, text width=2.5cm] at (13.5,2) {جنگ‌ها\\رفاه\\۱۹۱۴-۱۹۷۹};
\node[era, fill=industrialgray!20, text width=2.5cm] at (16.5,2) {نئولیبرال\\برگزیت\\۱۹۷۹-};

% تاریخ‌ها
\foreach \x/\year in {0/1066, 3/1485, 6/1714, 9/1832, 12/1914, 15/1979, 18/2024} {
    \draw[thick] (\x,0.2) -- (\x,-0.2);
    \node[below, font=\tiny] at (\x,-0.3) {\year};
}

% روندهای بلند
\draw[{Stealth}-, very thick, victoriangreen] (0,-1.5) -- (9,-1.5) 
    node[midway, below, font=\footnotesize] {تمرکز قدرت};
\draw[-{Stealth}, very thick, imperialred] (9,-1.5) -- (18,-1.5) 
    node[midway, below, font=\footnotesize] {توزیع قدرت};

\end{tikzpicture}
\caption{خلاصه روایت: ۱۰۰۰ سال در یک نگاه}
\end{figure}

%----------------------------------------------------------------------
\section{الگوهای کلیدی استخراج‌شده}
%----------------------------------------------------------------------

\subsection{الگوی ۱: تغییر تدریجی به‌جای انقلاب}

\begin{figure}[H]
\centering
\begin{tikzpicture}[
    node distance=0.8cm,
    element/.style={rectangle, rounded corners, draw=victoriangreen, thick,
                    fill=victoriangreen!15, minimum width=3.5cm,
                    minimum height=1.3cm, text centered, text width=3.3cm, font=\small}
]

\node[element] (e1) at (0,3) {شناسایی فشار\\از پایین/خارج};
\node[element] (e2) at (5,3) {امتیاز محدود\\به‌موقع};
\node[element] (e3) at (10,3) {ادغام\\گروه جدید};
\node[element] (e4) at (2.5,0) {حفظ ساختار\\کلی قدرت};
\node[element] (e5) at (7.5,0) {تکرار چرخه\\با گروه بعدی};

\draw[-{Stealth}, thick] (e1) -- (e2);
\draw[-{Stealth}, thick] (e2) -- (e3);
\draw[-{Stealth}, thick] (e3) -- (e4);
\draw[-{Stealth}, thick] (e4) -- (e5);
\draw[-{Stealth}, thick, dashed] (e5) to[bend right=30] (e1);

\end{tikzpicture}
\caption{الگوی «اصلاحات برای حفظ»}
\end{figure}

\begin{policybox}[شرایط کاربرد این الگو]
این الگو در شرایطی کار می‌کند که:
\begin{itemize}
    \item نخبگان به اندازه کافی منسجم و دوراندیش باشند
    \item فشار از پایین جدی اما نه انقلابی باشد
    \item نهادهای میانجی (پارلمان، دادگاه، رسانه) وجود داشته باشند
    \item فرهنگ سیاسی مذاکره و مصالحه را ارزش بداند
\end{itemize}
\textbf{محدودیت:} می‌تواند به تغییرات سطحی و حفظ نابرابری‌های عمیق منجر شود.
\end{policybox}

\subsection{الگوی ۲: بحران به‌مثابه فرصت}

\begin{table}[H]
\centering
\caption{بحران‌ها و تحولات بنیادین}
\renewcommand{\arraystretch}{1.4}
\begin{tabularx}{\textwidth}{>{\bfseries}p{3cm}p{4cm}X}
\toprule
\textbf{بحران} & \textbf{فشار ایجادشده} & \textbf{تحول ناشی} \\
\midrule
\rowcolor{empirecream}
جنگ داخلی ۱۶۴۰ & چالش مشروعیت پادشاه & مشروطیت، برتری پارلمان \\

جنگ جهانی اول & بسیج ملی، کمبود نیرو & حق رأی همگانی، دولت مداخله‌گر \\

\rowcolor{empirecream}
جنگ جهانی دوم & بسیج کامل، بمباران & NHS، دولت رفاه \\

بحران ۱۹۷۰ & رکود تورمی، اعتصابات & انقلاب تاچری \\

\rowcolor{empirecream}
برگزیت + کووید & شکاف‌های عمیق & ؟ (در حال شکل‌گیری) \\

\bottomrule
\end{tabularx}
\end{table}

\subsection{الگوی ۳: اتحاد طبقات به‌جای جنگ طبقاتی}

\begin{figure}[H]
\centering
\begin{tikzpicture}[
    class/.style={ellipse, draw, thick, minimum width=3cm, minimum height=1.5cm,
                  text centered, font=\small}
]

% قبل
\node[class, fill=tudorpurple!30] (aristo1) at (0,3) {اشرافیت};
\node[class, fill=parliamentgold!30] (bourg1) at (0,1) {بورژوازی};
\node[class, fill=imperialred!30] (work1) at (0,-1) {طبقه کارگر};

\draw[{Stealth}-{Stealth}, thick, imperialred] (aristo1) -- (bourg1);
\draw[{Stealth}-{Stealth}, thick, imperialred] (bourg1) -- (work1);

\node[above=0.5cm of aristo1, font=\bfseries] {مدل قاره‌ای (تقابل)};

% بعد
\node[class, fill=tudorpurple!30] (aristo2) at (8,3) {اشرافیت};
\node[class, fill=parliamentgold!30] (bourg2) at (8,1) {بورژوازی};
\node[class, fill=imperialred!30] (work2) at (8,-1) {طبقه کارگر};

% ائتلاف
\draw[very thick, victoriangreen, rounded corners] 
    (aristo2.west) -- ++(-1,0) -- ++(0,-2) -- (bourg2.west);
\draw[very thick, victoriangreen, rounded corners]
    (bourg2.west) -- ++(-0.5,0) -- ++(0,-2) -- (work2.west);

\node[above=0.5cm of aristo2, font=\bfseries] {مدل بریتانیایی (ادغام)};

% فلش
\draw[-{Stealth}, very thick] (3,1) -- (5,1) node[midway, above] {تفاوت};

\end{tikzpicture}
\caption{ادغام تدریجی به‌جای تقابل}
\end{figure}

\subsection{الگوی ۴: نهادها مهم‌ترند از افراد}

\begin{policybox}[وابستگی به مسیر نهادی]
یکی از مهم‌ترین درس‌های تجربه بریتانیا اهمیت نهادهاست:
\begin{itemize}
    \item \textbf{پارلمان:} از ابزار سلطنتی به نماینده ملت (تکامل ۸۰۰ ساله)
    \item \textbf{کامن‌لا:} ایجاد حقوق مالکیت و قرارداد پیش‌بینی‌پذیر
    \item \textbf{بانک انگلستان:} ثبات مالی و اعتبار دولت
    \item \textbf{خدمات عمومی:} بوروکراسی حرفه‌ای (از میانه قرن ۱۹)
\end{itemize}
نهادها یک‌شبه ساخته نمی‌شوند. اعتماد نهادی محصول دهه‌ها (قرن‌ها) عملکرد منصفانه است.
\end{policybox}

\subsection{الگوی ۵: هزینه پیشگامی}

\begin{figure}[H]
\centering
\begin{tikzpicture}
\begin{axis}[
    width=12cm,
    height=7cm,
    xlabel={زمان},
    ylabel={موقعیت نسبی},
    xmin=0, xmax=10,
    ymin=0, ymax=10,
    xtick=\empty,
    ytick=\empty,
    axis lines=left,
]

% پیشگام (بریتانیا)
\addplot[very thick, royalblue, domain=0:10, samples=100] 
    {3 + 4/(1+exp(-2*(x-2))) - 2*x/10};
\node[above, royalblue] at (axis cs:2,7) {پیشگام (بریتانیا)};

% دنباله‌رو (آلمان)
\addplot[very thick, industrialgray, domain=0:10, samples=100] 
    {1 + 5/(1+exp(-2*(x-4)))};
\node[above, industrialgray] at (axis cs:5,6) {دنباله‌رو (آلمان)};

% دنباله‌رو دیرتر (کره)
\addplot[very thick, victoriangreen, domain=0:10, samples=100] 
    {0.5 + 6/(1+exp(-2*(x-6)))};
\node[above, victoriangreen] at (axis cs:7.5,5.5) {دنباله‌رو متأخر (کره)};

\end{axis}
\end{tikzpicture}
\caption{نفرین پیشگامی: دنباله‌روها سبقت می‌گیرند}
\end{figure}

%----------------------------------------------------------------------
\section{ده درس کلیدی}
%----------------------------------------------------------------------

\begin{table}[H]
\centering
\caption{ده درس از تجربه تاریخی بریتانیا}
\renewcommand{\arraystretch}{1.5}
\begin{tabularx}{\textwidth}{>{\bfseries}cX}
\toprule
\textbf{شماره} & \textbf{درس} \\
\midrule
\rowcolor{empirecream}
۱ & \textbf{نهادها از ایده‌ها مهم‌ترند:} ایده‌های خوب بدون نهادهای مناسب اجرا نمی‌شوند \\

۲ & \textbf{انعطاف‌پذیری حیاتی است:} نظام‌هایی که تغییر می‌کنند، زنده می‌مانند \\

\rowcolor{empirecream}
۳ & \textbf{اصلاحات پیش‌گیرانه مؤثرتر از واکنشی است:} دادن امتیاز قبل از انفجار \\

۴ & \textbf{فرهنگ سیاسی مهم است:} پراگماتیسم vs ایدئولوژی‌گرایی \\

\rowcolor{empirecream}
۵ & \textbf{طبقه متوسط تثبیت‌کننده است:} شمول طبقه متوسط ثبات می‌آورد \\

۶ & \textbf{جنگ شتاب‌دهنده است:} بحران‌ها تغییرات غیرممکن را ممکن می‌کنند \\

\rowcolor{empirecream}
۷ & \textbf{امپراتوری هزینه دارد:} منافع کوتاه‌مدت، هزینه‌های بلندمدت \\

۸ & \textbf{صنعت‌زدایی بازگشت‌ناپذیر است:} از دست رفته برنمی‌گردد \\

\rowcolor{empirecream}
۹ & \textbf{نابرابری پایدار نیست:} یا اصلاح می‌شود یا منفجر \\

۱۰ & \textbf{تاریخ خطی نیست:} پیشرفت تضمین‌شده نیست، عقب‌گرد ممکن است \\

\bottomrule
\end{tabularx}
\end{table}

%----------------------------------------------------------------------
\section{هشدارها و محدودیت‌های تعمیم}
%----------------------------------------------------------------------

\begin{warningbox}
\textbf{هشدارهای روش‌شناختی:}
\begin{enumerate}
    \item \textbf{زمینه‌مندی:} تجربه بریتانیا در زمینه خاصی شکل گرفت که قابل تکرار نیست
    \item \textbf{بازسازی گذشته‌نگر:} آنچه موفق به نظر می‌رسد، ممکن بود شکست بخورد
    \item \textbf{بقای موفق:} فقط موارد موفق را می‌بینیم (survivorship bias)
    \item \textbf{پیچیدگی علّی:} علت‌ها متعدد و درهم‌تنیده‌اند
    \item \textbf{تغییر شرایط:} آنچه در قرن ۱۹ کار کرد، در قرن ۲۱ ممکن است نکند
\end{enumerate}
\end{warningbox}

%----------------------------------------------------------------------
\section{چشم‌انداز آینده بریتانیا}
%----------------------------------------------------------------------

\subsection{چالش‌های پیش‌رو}

\begin{figure}[H]
\centering
\begin{tikzpicture}[
    challenge/.style={rectangle, rounded corners=8pt, draw=imperialred, thick,
                      fill=imperialred!15, minimum width=4.5cm,
                      minimum height=2cm, text centered, text width=4.3cm}
]

\node[challenge] at (0,3) {
    \textbf{اقتصادی}\\
    بهره‌وری پایین\\
    سرمایه‌گذاری کم\\
    نابرابری منطقه‌ای
};

\node[challenge] at (5.5,3) {
    \textbf{اجتماعی}\\
    پیری جمعیت\\
    شکاف نسلی\\
    بحران مسکن
};

\node[challenge] at (11,3) {
    \textbf{سیاسی}\\
    بی‌اعتمادی به نهادها\\
    پوپولیسم\\
    استقلال‌طلبی اسکاتلند
};

\node[challenge] at (2.75,0) {
    \textbf{بین‌المللی}\\
    جایگاه پسابرگزیت\\
    رقابت چین-آمریکا\\
    تغییرات اقلیمی
};

\node[challenge] at (8.25,0) {
    \textbf{هویتی}\\
    معنای «بریتانیایی بودن»\\
    چندفرهنگی\\
    رابطه با گذشته امپراتوری
};

\end{tikzpicture}
\caption{چالش‌های اصلی بریتانیا در دهه‌های آینده}
\end{figure}

\subsection{سناریوهای ممکن}

\begin{table}[H]
\centering
\caption{سه سناریو برای آینده بریتانیا}
\renewcommand{\arraystretch}{1.5}
\begin{tabularx}{\textwidth}{>{\bfseries}p{3cm}XXX}
\toprule
\textbf{بُعد} & \textbf{سناریوی ۱: تجدید} & \textbf{سناریوی ۲: رکود} & \textbf{سناریوی ۳: تجزیه} \\
\midrule
\rowcolor{empirecream}
اقتصاد & استراتژی صنعتی نو، سرمایه‌گذاری در مهارت‌ها، کاهش نابرابری & ادامه وضع موجود، رشد کم، مالی‌محوری & بحران‌های متوالی، فرار سرمایه \\

سیاست & اجماع جدید، اصلاحات نهادی & فلج سیاسی، دوقطبی‌شدن & بی‌ثباتی، تغییرات مکرر دولت \\

\rowcolor{empirecream}
اتحاد & حفظ با انتقال بیشتر قدرت & تنش‌های ادامه‌دار & استقلال اسکاتلند، ؟ ایرلند \\

بین‌الملل & «بریتانیای جهانی» موفق، توافقات تجاری & حاشیه‌ای شدن، بین دو قطب & انزوا، بی‌اهمیتی \\

\rowcolor{empirecream}
احتمال & ۲۵٪ & ۵۰٪ & ۲۵٪ \\

\bottomrule
\end{tabularx}
\end{table}

\begin{figure}[H]
\centering
\begin{tikzpicture}[
    scenario/.style={rectangle, rounded corners=10pt, draw, very thick,
                     minimum width=4.5cm, minimum height=5cm,
                     text centered, text width=4.3cm}
]

\node[scenario, fill=victoriangreen!20, draw=victoriangreen] at (0,0) {
    \textbf{سناریوی تجدید}\\[0.3cm]
    \footnotesize
    $\bullet$ قرارداد اجتماعی جدید\\
    $\bullet$ سرمایه‌گذاری سبز\\
    $\bullet$ «ارتقای سطح» مناطق\\
    $\bullet$ اصلاح انتخاباتی\\
    $\bullet$ نقش جدید جهانی\\[0.3cm]
    \normalsize
    نیازمند: رهبری، اجماع،\\
    سرمایه‌گذاری بلندمدت
};

\node[scenario, fill=parliamentgold!20, draw=parliamentgold] at (6,0) {
    \textbf{سناریوی رکود}\\[0.3cm]
    \footnotesize
    $\bullet$ رشد کم مداوم\\
    $\bullet$ نابرابری پایدار\\
    $\bullet$ خدمات عمومی فرسوده\\
    $\bullet$ سیاست دوقطبی\\
    $\bullet$ نه بحران، نه پیشرفت\\[0.3cm]
    \normalsize
    محتمل‌ترین سناریو:\\
    ادامه مسیر فعلی
};

\node[scenario, fill=imperialred!20, draw=imperialred] at (12,0) {
    \textbf{سناریوی تجزیه}\\[0.3cm]
    \footnotesize
    $\bullet$ استقلال اسکاتلند\\
    $\bullet$ اتحاد ایرلند؟\\
    $\bullet$ بحران‌های اقتصادی\\
    $\bullet$ فروپاشی اجماع\\
    $\bullet$ پوپولیسم غالب\\[0.3cm]
    \normalsize
    نیازمند: شوک بزرگ،\\
    شکست نهادی
};

\end{tikzpicture}
\caption{سه سناریوی آینده}
\end{figure}

%----------------------------------------------------------------------
\section{پرسش‌های باز برای پژوهش آینده}
%----------------------------------------------------------------------

\begin{table}[H]
\centering
\caption{دستور کار پژوهشی}
\renewcommand{\arraystretch}{1.5}
\begin{tabularx}{\textwidth}{>{\bfseries}cX}
\toprule
\textbf{حوزه} & \textbf{پرسش‌های کلیدی} \\
\midrule
\rowcolor{empirecream}
نهادها & آیا نهادهای بریتانیایی می‌توانند بدون اصلاح بنیادین پاسخگوی چالش‌های قرن ۲۱ باشند؟ \\

طبقه & آیا مفهوم «طبقه» همچنان برای تحلیل جامعه بریتانیا مفید است؟ \\

\rowcolor{empirecream}
اتحاد & آیا «بریتانیای کبیر» می‌تواند به‌شکل فعلی بقا یابد؟ \\

اقتصاد & چگونه می‌توان بهره‌وری را بدون بازگشت به صنعت‌گرایی افزایش داد؟ \\

\rowcolor{empirecream}
هویت & «بریتانیایی بودن» در جهان پساامپراتوری و پسابرگزیت چه معنایی دارد؟ \\

تطبیقی & آیا تجربه بریتانیا برای کشورهای در حال توسعه درس‌هایی دارد؟ \\

\bottomrule
\end{tabularx}
\end{table}

%----------------------------------------------------------------------
\section{کلام آخر}
%----------------------------------------------------------------------

\begin{figure}[H]
\centering
\begin{tikzpicture}
\node[rectangle, rounded corners=15pt, draw=royalblue, very thick,
      fill=royalblue!10, minimum width=14cm, minimum height=8cm,
      text width=13.5cm, text centered] {
    
    \Large\textbf{جمع‌بندی نهایی}\\[0.5cm]
    
    \normalsize
    تاریخ بریتانیا داستان تحول تدریجی یک جامعه فئودالی جزیره‌ای به 
    نخستین قدرت صنعتی جهان، بزرگ‌ترین امپراتوری تاریخ، پیشگام دموکراسی 
    پارلمانی، و سازنده دولت رفاه مدرن است.\\[0.3cm]
    
    این تحول نه خطی بود، نه ناگزیر، نه بدون هزینه. میلیون‌ها نفر در 
    مستعمرات، در کارخانه‌ها، در معادن، و در جنگ‌ها بهای این «پیشرفت» 
    را پرداختند.\\[0.3cm]
    
    اما الگوهای این تجربه -- تدریجی‌گرایی، انعطاف نهادی، مصالحه طبقاتی، 
    اصلاحات پیش‌گیرانه -- همچنان درس‌هایی برای امروز دارند.\\[0.3cm]
    
    بریتانیای امروز در نقطه عطفی ایستاده: پسابرگزیت، پساکووید، در جهانی 
    در حال تغییر. اینکه این کشور بتواند یک بار دیگر خود را بازآفرینی کند 
    -- همان‌طور که در ۱۶۸۸، ۱۸۳۲، ۱۹۴۵ کرد -- پرسشی است که آینده 
    پاسخ خواهد داد.\\[0.3cm]
    
    \textit{تاریخ تمام نشده است.}
};
\end{tikzpicture}
\end{figure}

%----------------------------------------------------------------------
\section*{منابع اصلی فصل نهم}
%----------------------------------------------------------------------

\begin{enumerate}[label={[\arabic*]}]
    \item North, Douglass C. \& Weingast, Barry R. (1989). ``Constitutions and Commitment.'' \textit{Journal of Economic History}, 49(4): 803-832.
    \item Acemoglu, Daron \& Robinson, James (2006). \textit{Economic Origins of Dictatorship and Democracy}. Cambridge: CUP.
    \item Fukuyama, Francis (2011). \textit{The Origins of Political Order}. New York: Farrar, Straus.
    \item Tilly, Charles (1992). \textit{Coercion, Capital, and European States}. Oxford: Blackwell.
    \item Mann, Michael (1986-2012). \textit{The Sources of Social Power}. 4 vols. Cambridge: CUP.
    \item Streeck, Wolfgang (2014). \textit{Buying Time: The Delayed Crisis of Democratic Capitalism}. London: Verso.
    \item Pierson, Paul (2004). \textit{Politics in Time: History, Institutions, and Social Analysis}. Princeton: PUP.
    \item Mahoney, James \& Thelen, Kathleen, eds. (2010). \textit{Explaining Institutional Change}. Cambridge: CUP.
    \item Collier, Ruth Berins \& Collier, David (1991). \textit{Shaping the Political Arena}. Princeton: PUP.
    \item Evans, Peter (1995). \textit{Embedded Autonomy: States and Industrial Transformation}. Princeton: PUP.
\end{enumerate}


%%%%%%%%%%%%%%%%%%%%%%%%%%%%%%%%%%%%%%%%%%%%%%%%%%%%%%%%%%%%%%%%%%%%%%%
% پیوست‌ها
%%%%%%%%%%%%%%%%%%%%%%%%%%%%%%%%%%%%%%%%%%%%%%%%%%%%%%%%%%%%%%%%%%%%%%%

\appendix

\chapter{پیوست ۱: جدول زمانی جامع}

\begin{longtable}{>{\bfseries}cp{4cm}p{8cm}}
\caption{رویدادهای کلیدی تاریخ بریتانیا (۱۰۶۶-۲۰۲۴)} \\
\toprule
\textbf{سال} & \textbf{رویداد} & \textbf{اهمیت} \\
\midrule
\endfirsthead
\multicolumn{3}{c}{\textit{ادامه از صفحه قبل}} \\
\toprule
\textbf{سال} & \textbf{رویداد} & \textbf{اهمیت} \\
\midrule
\endhead
\midrule
\multicolumn{3}{r}{\textit{ادامه در صفحه بعد}} \\
\endfoot
\bottomrule
\endlastfoot

\rowcolor{empirecream}
۱۰۶۶ & فتح نورمان & تأسیس نظام فئودالی، زبان فرانسوی در دربار \\
۱۲۱۵ & مگناکارتا & محدودیت قدرت پادشاه، ریشه حقوق \\
\rowcolor{empirecream}
۱۲۹۵ & پارلمان نمونه & شمول عوام در پارلمان \\
۱۳۴۸ & طاعون سیاه & مرگ ۱/۳ جمعیت، تغییر روابط ارضی \\
\rowcolor{empirecream}
۱۳۸۱ & شورش دهقانان & نخستین شورش طبقاتی بزرگ \\
۱۴۸۵ & آغاز سلسله تودور & پایان جنگ‌های گل سرخ \\
\rowcolor{empirecream}
۱۵۳۴ & جدایی از رم & کلیسای انگلستان، مصادره اراضی \\
۱۵۸۸ & شکست آرمادای اسپانیا & ظهور قدرت دریایی \\
\rowcolor{empirecream}
۱۶۰۰ & تأسیس کمپانی هند شرقی & آغاز امپراتوری تجاری \\
۱۶۴۲-۵۱ & جنگ داخلی & چالش حاکمیت مطلقه \\
\rowcolor{empirecream}
۱۶۴۹ & اعدام چارلز اول & نخستین اعدام قانونی پادشاه \\
۱۶۸۸ & انقلاب شکوهمند & مشروطیت، اعلامیه حقوق \\
\rowcolor{empirecream}
۱۶۹۴ & تأسیس بانک انگلستان & انقلاب مالی \\
۱۷۰۷ & اتحاد با اسکاتلند & تشکیل بریتانیای کبیر \\
\rowcolor{empirecream}
۱۷۲۱ & والپول نخست‌وزیر & تثبیت نظام کابینه‌ای \\
۱۷۶۹ & ماشین بخار وات & انقلاب صنعتی \\
\rowcolor{empirecream}
۱۷۷۶ & استقلال آمریکا & از دست رفتن مستعمرات اولیه \\
۱۸۰۱ & اتحاد با ایرلند & تشکیل پادشاهی متحد \\
\rowcolor{empirecream}
۱۸۱۵ & شکست ناپلئون & برتری جهانی بریتانیا \\
۱۸۳۲ & قانون اصلاحات اول & آغاز گسترش حق رأی \\
\rowcolor{empirecream}
۱۸۳۳ & لغو برده‌داری & در امپراتوری (نه کامل) \\
۱۸۳۸-۴۸ & جنبش چارتیست & نخستین جنبش کارگری توده‌ای \\
\rowcolor{empirecream}
۱۸۴۶ & لغو قوانین غلات & پیروزی تجارت آزاد \\
۱۸۵۱ & نمایشگاه بزرگ & اوج «کارگاه جهان» \\
\rowcolor{empirecream}
۱۸۶۷ & قانون اصلاحات دوم & حق رأی کارگران شهری \\
۱۸۸۴ & قانون اصلاحات سوم & حق رأی کارگران روستایی \\
\rowcolor{empirecream}
۱۸۸۶ & لایحه خودمختاری ایرلند & انشعاب لیبرال‌ها \\
۱۸۹۹-۱۹۰۲ & جنگ بوئر & بحران اعتماد امپراتوری \\
\rowcolor{empirecream}
۱۹۰۰ & تأسیس LRC & ریشه حزب کارگر \\
۱۹۰۶-۱۴ & اصلاحات لیبرال & بیمه ملی، مستمری \\
\rowcolor{empirecream}
۱۹۱۱ & قانون پارلمان & محدود شدن مجلس اعیان \\
۱۹۱۴-۱۸ & جنگ جهانی اول & ۸۸۶,۰۰۰ کشته \\
\rowcolor{empirecream}
۱۹۱۸ & حق رأی همگانی & مردان + زنان ۳۰+ \\
۱۹۲۶ & اعتصاب عمومی & شکست اتحادیه‌ها \\
\rowcolor{empirecream}
۱۹۲۸ & حق رأی برابر زنان & دموکراسی کامل \\
۱۹۳۱ & بحران مالی & دولت ملی، انشعاب کارگر \\
\rowcolor{empirecream}
۱۹۳۹-۴۵ & جنگ جهانی دوم & ۴۵۰,۰۰۰ کشته \\
۱۹۴۲ & گزارش بوریج & نقشه راه دولت رفاه \\
\rowcolor{empirecream}
۱۹۴۵ & پیروزی کارگر & دولت اتلی \\
۱۹۴۸ & تأسیس NHS & بهداشت همگانی رایگان \\
\rowcolor{empirecream}
۱۹۴۷ & استقلال هند & آغاز فروپاشی امپراتوری \\
۱۹۵۶ & بحران سوئز & پایان توهم قدرت \\
\rowcolor{empirecream}
۱۹۷۳ & عضویت EEC & پیوستن به اروپا \\
۱۹۷۹ & انتخاب تاچر & انقلاب نئولیبرال \\
\rowcolor{empirecream}
۱۹۸۴-۸۵ & اعتصاب معدنچیان & شکست اتحادیه‌ها \\
۱۹۹۰ & سقوط تاچر & مالیات سرانه \\
\rowcolor{empirecream}
۱۹۹۷ & پیروزی بلر & کارگر نو \\
۱۹۹۸ & توافق جمعه نیک & صلح ایرلند شمالی \\
\rowcolor{empirecream}
۱۹۹۹ & انتقال قدرت & پارلمان اسکاتلند، مجلس ویلز \\
۲۰۰۸ & بحران مالی & رکود بزرگ \\
\rowcolor{empirecream}
۲۰۱۰ & دولت ائتلافی & ریاضت \\
۲۰۱۶ & رفراندوم برگزیت & رأی به خروج از اتحادیه اروپا \\
\rowcolor{empirecream}
۲۰۲۰ & برگزیت رسمی + کووید & بحران مضاعف \\
۲۰۲۲ & بحران تراس & ۴۴ روز نخست‌وزیری \\
\rowcolor{empirecream}
۲۰۲۴ & انتخابات & ؟ \\

\end{longtable}


\chapter{پیوست ۲: داده‌های آماری تکمیلی}

\section*{جمعیت}

\begin{table}[H]
\centering
\caption{جمعیت بریتانیا (۱۰۸۶-۲۰۲۱)}
\renewcommand{\arraystretch}{1.2}
\begin{tabular}{lcccc}
\toprule
\textbf{سال} & \textbf{انگلستان} & \textbf{ویلز} & \textbf{اسکاتلند} & \textbf{کل} \\
\midrule
\rowcolor{empirecream}
۱۰۸۶ & ۱.۷ م & --- & --- & $\sim$۲ م \\
۱۳۰۰ & $\sim$۵ م & --- & --- & $\sim$۶ م \\
\rowcolor{empirecream}
۱۴۰۰ & $\sim$۲.۵ م & --- & --- & $\sim$۳ م \\
۱۶۰۰ & ۴.۱ م & ۰.۳ م & ۱.۰ م & ۵.۴ م \\
\rowcolor{empirecream}
۱۷۰۰ & ۵.۰ م & ۰.۴ م & ۱.۰ م & ۶.۴ م \\
۱۸۰۱ & ۸.۳ م & ۰.۶ م & ۱.۶ م & ۱۰.۵ م \\
\rowcolor{empirecream}
۱۸۵۱ & ۱۶.۸ م & ۱.۲ م & ۲.۹ م & ۲۰.۹ م \\
۱۹۰۱ & ۳۰.۵ م & ۲.۰ م & ۴.۵ م & ۳۷.۰ م \\
\rowcolor{empirecream}
۱۹۵۱ & ۴۱.۲ م & ۲.۶ م & ۵.۱ م & ۴۸.۹ م \\
۲۰۰۱ & ۴۹.۱ م & ۲.۹ م & ۵.۱ م & ۵۷.۱ م \\
\rowcolor{empirecream}
۲۰۲۱ & ۵۶.۵ م & ۳.۱ م & ۵.۵ م & ۶۷.۰ م \\
\bottomrule
\end{tabular}
\end{table}

\section*{اقتصاد}

\begin{figure}[H]
\centering
\begin{tikzpicture}
\begin{axis}[
    width=14cm,
    height=9cm,
    xlabel={سال},
    ylabel={GDP سرانه (دلار ۲۰۱۱)},
    xmin=1270, xmax=2020,
    ymin=0, ymax=45000,
    grid=major,
    title={\textbf{رشد بلندمدت GDP سرانه بریتانیا}},
]

\addplot[very thick, royalblue, mark=*, mark indices={1,5,10,15,20,25,30}] coordinates {
    (1270, 800) (1400, 1000) (1500, 1100) (1600, 1200)
    (1700, 1500) (1760, 1800) (1800, 2100) (1830, 2700)
    (1860, 3700) (1890, 4700) (1913, 5300) (1929, 5500)
    (1950, 7000) (1973, 12000) (1990, 18000) (2000, 24000)
    (2010, 30000) (2019, 42000)
};

% انقلاب صنعتی
\draw[dashed, thick, victoriangreen] (axis cs:1760,0) -- (axis cs:1760,45000);
\node[above, font=\footnotesize, victoriangreen] at (axis cs:1760,42000) {انقلاب صنعتی};

\end{axis}
\end{tikzpicture}
\caption{رشد بلندمدت اقتصادی (۱۲۷۰-۲۰۱۹)}
\end{figure}


\chapter{پیوست ۳: واژه‌نامه}

\begin{longtable}{>{\bfseries}p{4cm}p{10cm}}
\caption{واژه‌نامه اصطلاحات کلیدی} \\
\toprule
\textbf{اصطلاح} & \textbf{تعریف} \\
\midrule
\endfirsthead
\toprule
\textbf{اصطلاح} & \textbf{تعریف} \\
\midrule
\endhead
\bottomrule
\endlastfoot

\rowcolor{empirecream}
اجماع پس از جنگ & توافق ضمنی دو حزب اصلی (۱۹۴۵-۱۹۷۹) بر سر دولت رفاه، اقتصاد مختلط، و اشتغال کامل \\

اصلاحات لیبرال & مجموعه قوانین اجتماعی دولت لیبرال (۱۹۰۶-۱۹۱۴) که بنیان دولت رفاه را گذاشت \\

\rowcolor{empirecream}
انتقال قدرت (Devolution) & واگذاری اختیارات از وست‌مینستر به پارلمان‌های منطقه‌ای اسکاتلند، ویلز، ایرلند شمالی \\

انقلاب شکوهمند & تغییر سلطنت در ۱۶۸۸ بدون خونریزی و تثبیت نظام مشروطه \\

\rowcolor{empirecream}
برگزیت & خروج بریتانیا از اتحادیه اروپا (رأی‌گیری ۲۰۱۶، اجرا ۲۰۲۰) \\

بیمه ملی & نظام بیمه اجتماعی تأسیس‌شده در ۱۹۱۱ و بازسازی‌شده در ۱۹۴۶ \\

\rowcolor{empirecream}
تاچریسم & ایدئولوژی و سیاست‌های مارگارت تاچر: بازار آزاد، دولت کوچک، ضد اتحادیه \\

توری & نام تاریخی حزب محافظه‌کار (از قرن ۱۷) \\

\rowcolor{empirecream}
چارتیسم & جنبش کارگری ۱۸۳۸-۱۸۵۸ برای اصلاحات سیاسی (شش خواسته «منشور مردم») \\

حزب ویگ & حزب تاریخی طرفدار پارلمان (قرن ۱۷-۱۹)، پیشین حزب لیبرال \\

\rowcolor{empirecream}
خصوصی‌سازی & فروش صنایع دولتی به بخش خصوصی، سیاست محوری تاچریسم \\

دولت رفاه & مجموعه نهادهای تأمین اجتماعی، بهداشت، آموزش و مسکن عمومی \\

\rowcolor{empirecream}
راه سوم & ایدئولوژی کارگر نو (بلر): ترکیب بازار و عدالت اجتماعی \\

ریاضت (Austerity) & سیاست کاهش هزینه‌های عمومی پس از بحران ۲۰۰۸ \\

\rowcolor{empirecream}
سافرجت & جنبش حق رأی زنان (اوایل قرن ۲۰) \\

فابیانیسم & سوسیالیسم تدریجی و اصلاح‌طلب بریتانیایی (جامعه فابین، ۱۸۸۴) \\

\rowcolor{empirecream}
فئودالیسم & نظام اجتماعی-اقتصادی قرون وسطی مبتنی بر روابط ارضی و وفاداری \\

کامن‌لا & نظام حقوقی مبتنی بر رویه قضایی (در مقابل حقوق مدون) \\

\rowcolor{empirecream}
کینزگرایی & سیاست اقتصادی مبتنی بر مدیریت تقاضای کل توسط دولت \\

گزارش بوریج & گزارش ۱۹۴۲ که نقشه راه دولت رفاه پس از جنگ را ترسیم کرد \\

\rowcolor{empirecream}
مگناکارتا & منشور ۱۲۱۵ که حقوقی را در برابر پادشاه تضمین کرد \\

NHS & خدمات بهداشتی ملی، تأسیس ۱۹۴۸، بهداشت همگانی رایگان \\

\rowcolor{empirecream}
نئولیبرالیسم & ایدئولوژی اقتصادی مبتنی بر بازار آزاد، مقررات‌زدایی، و خصوصی‌سازی \\

وست‌مینستر & محل پارلمان بریتانیا؛ مجازاً به معنای نظام سیاسی بریتانیا \\

\end{longtable}


%%%%%%%%%%%%%%%%%%%%%%%%%%%%%%%%%%%%%%%%%%%%%%%%%%%%%%%%%%%%%%%%%%%%%%%
% کتاب‌شناسی
%%%%%%%%%%%%%%%%%%%%%%%%%%%%%%%%%%%%%%%%%%%%%%%%%%%%%%%%%%%%%%%%%%%%%%%

\chapter*{کتاب‌شناسی جامع}
\addcontentsline{toc}{chapter}{کتاب‌شناسی جامع}

\section*{منابع اولیه و مجموعه اسناد}

\begin{enumerate}[label={[\arabic*]}]
    \item \textit{Statutes of the Realm}. 11 vols. London: Record Commission, 1810-1828.
    \item \textit{Hansard''s Parliamentary Debates}. Various series. London: Hansard, 1803-.
    \item Stephenson, Carl \& Marcham, F.G., eds. (1937). \textit{Sources of English Constitutional History}. New York: Harper.
    \item Williams, E.N., ed. (1960). \textit{The Eighteenth-Century Constitution}. Cambridge: CUP.
    \item English Historical Documents. General ed. D.C. Douglas. 12 vols. London: Eyre \& Spottiswoode, 1953-1977.
\end{enumerate}

\section*{تاریخ عمومی بریتانیا}

\begin{enumerate}[resume, label={[\arabic*]}]
    \item Clark, J.C.D. (2000). \textit{English Society 1660-1832}. 2nd ed. Cambridge: CUP.
    \item Colley, Linda (1992). \textit{Britons: Forging the Nation 1707-1837}. New Haven: Yale UP.
    \item Davies, Norman (1999). \textit{The Isles: A History}. London: Macmillan.
    \item Kearney, Hugh (1989). \textit{The British Isles: A History of Four Nations}. Cambridge: CUP.
    \item Morgan, Kenneth O., ed. (2001). \textit{The Oxford History of Britain}. Rev. ed. Oxford: OUP.
    \item Tombs, Robert (2014). \textit{The English and Their History}. London: Allen Lane.
\end{enumerate}

\section*{تاریخ اقتصادی}

\begin{enumerate}[resume, label={[\arabic*]}]
    \item Allen, Robert C. (2009). \textit{The British Industrial Revolution in Global Perspective}. Cambridge: CUP.
    \item Broadberry, Stephen \& O''Rourke, Kevin, eds. (2010). \textit{The Cambridge Economic History of Modern Europe}. 2 vols. Cambridge: CUP.
    \item Crafts, N.F.R. (1985). \textit{British Economic Growth during the Industrial Revolution}. Oxford: Clarendon.
    \item Floud, Roderick \& Johnson, Paul, eds. (2004). \textit{The Cambridge Economic History of Modern Britain}. 3 vols. Cambridge: CUP.
    \item Mokyr, Joel (2009). \textit{The Enlightened Economy: Britain and the Industrial Revolution}. London: Penguin.
    \item Wrigley, E.A. (2010). \textit{Energy and the English Industrial Revolution}. Cambridge: CUP.
\end{enumerate}

\section*{تاریخ سیاسی و قانون اساسی}

\begin{enumerate}[resume, label={[\arabic*]}]
    \item Bogdanor, Vernon (2009). \textit{The New British Constitution}. Oxford: Hart.
    \item Dicey, A.V. (1885/1959). \textit{Introduction to the Study of the Law of the Constitution}. 10th ed. London: Macmillan.
    \item Kishlansky, Mark (1996). \textit{A Monarchy Transformed: Britain 1603-1714}. London: Penguin.
    \item Pincus, Steve (2009). \textit{1688: The First Modern Revolution}. New Haven: Yale UP.
    \item Pocock, J.G.A. (1987). \textit{The Ancient Constitution and the Feudal Law}. 2nd ed. Cambridge: CUP.
\end{enumerate}

\section*{تاریخ اجتماعی}

\begin{enumerate}[resume, label={[\arabic*]}]
    \item Briggs, Asa (1983). \textit{A Social History of England}. London: Weidenfeld.
    \item Cannadine, David (1998). \textit{Class in Britain}. New Haven: Yale UP.
    \item Perkin, Harold (1969). \textit{The Origins of Modern English Society 1780-1880}. London: RKP.
    \item Thompson, E.P. (1963). \textit{The Making of the English Working Class}. London: Gollancz.
    \item Thompson, F.M.L., ed. (1990). \textit{The Cambridge Social History of Britain 1750-1950}. 3 vols. Cambridge: CUP.
\end{enumerate}

\section*{امپراتوری و روابط بین‌الملل}

\begin{enumerate}[resume, label={[\arabic*]}]
    \item Brendon, Piers (2007). \textit{The Decline and Fall of the British Empire}. London: Jonathan Cape.
    \item Cain, P.J. \& Hopkins, A.G. (2001). \textit{British Imperialism 1688-2000}. 2nd ed. London: Longman.
    \item Darwin, John (2009). \textit{The Empire Project: The Rise and Fall of the British World-System}. Cambridge: CUP.
    \item Ferguson, Niall (2003). \textit{Empire: How Britain Made the Modern World}. London: Allen Lane.
    \item Porter, Bernard (2004). \textit{The Absent-Minded Imperialists}. Oxford: OUP.
\end{enumerate}

\section*{دوره معاصر}

\begin{enumerate}[resume, label={[\arabic*]}]
    \item Addison, Paul (1975). \textit{The Road to 1945}. London: Jonathan Cape.
    \item Bogdanor, Vernon (2019). \textit{Beyond Brexit: Towards a British Constitution}. London: I.B. Tauris.
    \item Gamble, Andrew (1994). \textit{The Free Economy and the Strong State}. 2nd ed. London: Macmillan.
    \item Hennessy, Peter (1992). \textit{Never Again: Britain 1945-1951}. London: Jonathan Cape.
    \item Sandbrook, Dominic (2005-2012). \textit{Post-War Britain} series. 4 vols. London: Allen Lane.
    \item Seldon, Anthony \& Snowdon, Peter (2015). \textit{Cameron at 10}. London: Collins.
\end{enumerate}


%%%%%%%%%%%%%%%%%%%%%%%%%%%%%%%%%%%%%%%%%%%%%%%%%%%%%%%%%%%%%%%%%%%%%%%
% نمایه
%%%%%%%%%%%%%%%%%%%%%%%%%%%%%%%%%%%%%%%%%%%%%%%%%%%%%%%%%%%%%%%%%%%%%%%

\chapter*{نمایه موضوعی}
\addcontentsline{toc}{chapter}{نمایه}

\begin{multicols}{2}
\footnotesize

\textbf{آ}\\
آموزش، ۵۲، ۸۷، ۱۲۵، ۱۴۵\\
اتحادیه‌های کارگری، ۶۲-۶۵، ۹۸-۱۰۲، ۱۵۵-۱۵۸\\
اجماع پس از جنگ، ۱۲۰-۱۳۰\\
استعمار، نک: امپراتوری\\
اسکاتلند، ۲۴، ۱۴۰، ۱۸۵\\
اشرافیت، ۱۸-۲۲، ۳۵، ۴۸\\
اصلاحات انتخاباتی، ۵۵-۵۸، ۱۰۵\\
اعتصاب عمومی ۱۹۲۶، ۱۱۰-۱۱۲\\
اعتصاب معدنچیان ۱۹۸۴، ۱۵۵-۱۵۸\\
امپراتوری، ۳۲، ۷۰-۷۵، ۱۲۸-۱۳۰\\
انقلاب شکوهمند، ۳۰-۳۴\\
انقلاب صنعتی، ۴۲-۴۸\\
ایرلند، ۸۰-۸۴، ۱۴۰، ۱۶۸\\

\textbf{ب}\\
برگزیت، ۱۷۵-۱۸۲\\
بلر، تونی، ۱۶۵-۱۷۲\\
بوریج، گزارش، ۱۱۷-۱۱۹\\
بیمه ملی، ۶۸، ۱۲۲\\
بیکاری، ۱۰۸، ۱۳۵، ۱۷۵\\

\textbf{پ}\\
پارلمان، ۱۵-۱۸، ۲۸، ۶۷\\

\textbf{ت}\\
تاچر، مارگارت، ۱۵۰-۱۶۲\\
تجارت، ۳۲، ۵۰، ۷۲\\
تورم، ۱۳۵، ۱۵۲\\

\textbf{ج}\\
جنگ جهانی اول، ۹۵-۱۰۵\\
جنگ جهانی دوم، ۱۱۵-۱۲۰\\

\textbf{چ}\\
چارتیسم، ۶۲-۶۴\\
چرچیل، وینستون، ۱۱۶، ۱۲۱\\

\textbf{ح}\\
حزب کارگر، ۶۵، ۱۰۷، ۱۲۱، ۱۶۵\\
حزب لیبرال، ۵۵، ۶۶، ۱۰۷\\
حزب محافظه‌کار، ۵۶، ۱۵۰، ۱۷۴\\
حق رأی، نک: اصلاحات انتخاباتی\\

\textbf{خ}\\
خصوصی‌سازی، ۱۵۳-۱۵۵\\

\textbf{د}\\
دموکراسی، نک: اصلاحات انتخاباتی\\
دولت رفاه، ۱۲۱-۱۲۷\\

\textbf{ر}\\
راه‌آهن، ۴۶، ۱۵۵\\
ریاضت، ۱۷۴-۱۷۶\\

\textbf{ز}\\
زنان، ۸۵-۸۷، ۱۰۴\\

\textbf{ط}\\
طبقه کارگر، ۴۸، ۶۲-۶۵، ۹۸-۱۰۲\\
طبقه متوسط، ۴۸، ۵۵\\

\textbf{ف}\\
فئودالیسم، ۱۲-۱۵\\

\textbf{ک}\\
کووید-۱۹، ۱۸۳\\

\textbf{م}\\
مالیات، ۳۱، ۶۷، ۱۵۲\\
مستعمرات، نک: امپراتوری\\
مسکن، ۱۲۵، ۱۵۹\\
مگناکارتا، ۱۶-۱۷\\
مهاجرت، ۱۲۹، ۱۷۸\\

\textbf{ن}\\
نابرابری، ۱۵۹-۱۶۰، ۱۸۹\\
نفت دریای شمال، ۱۵۲\\
نورمان‌ها، ۱۲-۱۳\\
NHS، ۱۲۳-۱۲۵، ۱۶۹\\

\textbf{و}\\
ویکتوریا، ملکه، ۵۳\\
ویلز، ۱۴۰، ۱۷۷\\

\end{multicols}


%%%%%%%%%%%%%%%%%%%%%%%%%%%%%%%%%%%%%%%%%%%%%%%%%%%%%%%%%%%%%%%%%%%%%%%
% پایان سند
%%%%%%%%%%%%%%%%%%%%%%%%%%%%%%%%%%%%%%%%%%%%%%%%%%%%%%%%%%%%%%%%%%%%%%%

\chapter*{سخن پایانی}
\addcontentsline{toc}{chapter}{سخن پایانی}

\begin{center}
\Large\textit{این پژوهش تلاشی بود برای فهم گذشته،}\\[0.3cm]
\Large\textit{نه برای کپی کردن آن.}\\[0.5cm]
\Large\textit{هر جامعه‌ای باید مسیر خود را بسازد،}\\[0.3cm]
\Large\textit{اما می‌تواند از تجربه دیگران بیاموزد.}\\[1cm]

\rule{0.5\textwidth}{0.5pt}\\[1cm]

\normalsize
تهیه شده با استفاده از منابع آکادمیک\\
و رویکرد تحلیلی چندرشته‌ای\\[0.5cm]

\textbf{گروه پژوهشی چندتخصصی}\\
تاریخ، جامعه‌شناسی، اقتصاد سیاسی\\[1cm]

\today

\end{center}

%%%%%%%%%%%%%%%%%%%%%%%%%%%%%%%%%%%%%%%%%%%%%%%%%%%%%%%%%%%%%%%%%%%%%%%
% پیوست ۴: نقشه‌های تاریخی
%%%%%%%%%%%%%%%%%%%%%%%%%%%%%%%%%%%%%%%%%%%%%%%%%%%%%%%%%%%%%%%%%%%%%%%

\chapter{پیوست ۴: نقشه‌های مفهومی تاریخی}

\section*{نقشه ۱: تحول مرزهای سیاسی جزایر بریتانیا}

\begin{figure}[H]
\centering
\begin{tikzpicture}[scale=0.8]

% فریم‌های چهارگانه
\foreach \x/\title/\year in {0/{قبل از اتحاد}/1600, 6/{پس از اسکاتلند}/1707, 
                              0/{پس از ایرلند}/1801, 6/{امروز}/2024} {
    \begin{scope}[shift={(\x,\y)}]
    \end{scope}
}

% نقشه ۱: ۱۶۰۰
\begin{scope}[shift={(0,8)}]
    \draw[thick] (0,0) rectangle (5,6);
    \node[above] at (2.5,6) {\textbf{۱۶۰۰: سه پادشاهی جدا}};
    
    % انگلستان و ویلز
    \draw[fill=imperialred!30, rounded corners] (1,0.5) -- (3.5,0.5) -- (4,3) -- (3,5) -- (1.5,5) -- (1,3) -- cycle;
    \node at (2.5,2.5) {\footnotesize انگلستان};
    \node at (1.5,1.5) {\tiny ویلز};
    
    % اسکاتلند
    \draw[fill=royalblue!30, rounded corners] (1.5,5) -- (3,5) -- (3.5,5.8) -- (2,5.8) -- cycle;
    \node at (2.5,5.4) {\tiny اسکاتلند};
    
    % ایرلند
    \draw[fill=victoriangreen!30, rounded corners] (0.2,2) ellipse (0.7 and 1.5);
    \node at (0.2,2) {\tiny ایرلند};
\end{scope}

% نقشه ۲: ۱۷۰۷
\begin{scope}[shift={(6,8)}]
    \draw[thick] (0,0) rectangle (5,6);
    \node[above] at (2.5,6) {\textbf{۱۷۰۷: بریتانیای کبیر}};
    
    % بریتانیای کبیر
    \draw[fill=tudorpurple!30, rounded corners] (1,0.5) -- (3.5,0.5) -- (4,3) -- (3.5,5.8) -- (2,5.8) -- (1.5,5) -- (1,3) -- cycle;
    \node at (2.5,3) {\footnotesize بریتانیای کبیر};
    
    % ایرلند - هنوز جدا
    \draw[fill=victoriangreen!30, rounded corners] (0.2,2) ellipse (0.7 and 1.5);
    \node at (0.2,2) {\tiny ایرلند};
\end{scope}

% نقشه ۳: ۱۸۰۱
\begin{scope}[shift={(0,0)}]
    \draw[thick] (0,0) rectangle (5,6);
    \node[above] at (2.5,6) {\textbf{۱۸۰۱: پادشاهی متحد}};
    
    % کل جزایر
    \draw[fill=parliamentgold!30, rounded corners] (0.2,2) ellipse (0.7 and 1.5);
    \draw[fill=parliamentgold!30, rounded corners] (1,0.5) -- (3.5,0.5) -- (4,3) -- (3.5,5.8) -- (2,5.8) -- (1.5,5) -- (1,3) -- cycle;
    \node at (2,3) {\footnotesize پادشاهی متحد};
    \node at (2,2) {\footnotesize بریتانیای کبیر};
    \node at (2,1) {\footnotesize و ایرلند};
\end{scope}

% نقشه ۴: ۲۰۲۴
\begin{scope}[shift={(6,0)}]
    \draw[thick] (0,0) rectangle (5,6);
    \node[above] at (2.5,6) {\textbf{۲۰۲۴: وضعیت فعلی}};
    
    % بریتانیا
    \draw[fill=royalblue!30, rounded corners] (1,0.5) -- (3.5,0.5) -- (4,3) -- (3.5,5.8) -- (2,5.8) -- (1.5,5) -- (1,3) -- cycle;
    \node at (2.5,3) {\footnotesize UK};
    
    % ایرلند شمالی
    \draw[fill=royalblue!50, rounded corners] (0.2,3) ellipse (0.4 and 0.5);
    \node at (0.2,3) {\tiny NI};
    
    % جمهوری ایرلند
    \draw[fill=victoriangreen!50, rounded corners] (0.2,1.5) ellipse (0.5 and 1);
    \node at (0.2,1.5) {\tiny ROI};
\end{scope}

\end{tikzpicture}
\caption{تحول مرزهای سیاسی جزایر بریتانیا}
\end{figure}

\section*{نقشه ۲: گسترش و فروپاشی امپراتوری}

\begin{figure}[H]
\centering
\begin{tikzpicture}

% محور زمانی
\draw[very thick, -{Stealth}] (0,0) -- (16,0) node[right] {زمان};
\draw[very thick, -{Stealth}] (0,0) -- (0,8) node[above] {وسعت امپراتوری};

% منحنی امپراتوری
\draw[very thick, imperialred, fill=imperialred!20] 
    plot[smooth, tension=0.7] coordinates {
        (0,0.5) (2,1) (4,1.5) (6,3) (8,5) (10,7) (11,7.2) (12,6) (14,2) (16,0.5)
    } -- (16,0) -- (0,0) -- cycle;

% نقاط عطف
\node[circle, fill=royalblue, inner sep=2pt, label=above:{\tiny آمریکا ۱۷۷۶}] at (4,1.5) {};
\node[circle, fill=parliamentgold, inner sep=2pt, label=above:{\tiny هند ۱۸۵۸}] at (7,4) {};
\node[circle, fill=victoriangreen, inner sep=2pt, label=above:{\tiny اوج ۱۹۲۰}] at (10.5,7.2) {};
\node[circle, fill=tudorpurple, inner sep=2pt, label=above:{\tiny هند ۱۹۴۷}] at (12,6) {};
\node[circle, fill=industrialgray, inner sep=2pt, label=above:{\tiny سوئز ۱۹۵۶}] at (13,4) {};

% برچسب‌ها
\node at (5,2) {\small امپراتوری اول};
\node at (9,5) {\small امپراتوری دوم};
\node at (14,1) {\small فروپاشی};

% تاریخ‌ها
\foreach \x/\year in {0/1600, 4/1700, 8/1850, 12/1950, 16/2000} {
    \node[below] at (\x,0) {\tiny \year};
}

\end{tikzpicture}
\caption{منحنی گسترش و فروپاشی امپراتوری بریتانیا}
\end{figure}

\section*{نقشه ۳: جغرافیای صنعتی بریتانیا}

\begin{figure}[H]
\centering
\begin{tikzpicture}[scale=1.2]

% خط ساحلی ساده‌شده
\draw[thick, fill=empirecream] 
    (0,0) -- (1,0) -- (2,1) -- (3,1.5) -- (4,2) -- (4.5,3) -- (4,4) -- 
    (3.5,5) -- (3,6) -- (2.5,7) -- (2,7.5) -- (1.5,7) -- (1,6.5) -- 
    (0.5,6) -- (0,5) -- (-0.5,4) -- (-0.5,3) -- (0,2) -- cycle;

% مناطق صنعتی
% لنکشایر (منسوجات)
\draw[fill=imperialred!50, rounded corners] (1,4.5) ellipse (0.6 and 0.4);
\node[font=\tiny] at (1,4.5) {منسوجات};

% یورکشایر (فولاد)
\draw[fill=industrialgray!50, rounded corners] (2.5,4) ellipse (0.5 and 0.3);
\node[font=\tiny] at (2.5,4) {فولاد};

% میدلندز (صنایع فلزی)
\draw[fill=parliamentgold!50, rounded corners] (2,3) ellipse (0.6 and 0.4);
\node[font=\tiny] at (2,3) {فلزات};

% ویلز جنوبی (زغال)
\draw[fill=industrialgray!70, rounded corners] (0.5,2) ellipse (0.5 and 0.3);
\node[font=\tiny] at (0.5,2) {زغال};

% شمال شرق (کشتی‌سازی)
\draw[fill=royalblue!50, rounded corners] (3.5,5) ellipse (0.4 and 0.3);
\node[font=\tiny] at (3.5,5) {کشتی};

% اسکاتلند مرکزی
\draw[fill=victoriangreen!50, rounded corners] (1.5,6.5) ellipse (0.5 and 0.3);
\node[font=\tiny] at (1.5,6.5) {مهندسی};

% لندن
\draw[fill=tudorpurple!50] (3.5,2) circle (0.3);
\node[font=\tiny, below] at (3.5,1.6) {لندن (مالی)};

% راهنما
\node[anchor=west] at (5,6) {\footnotesize\textbf{راهنما:}};
\draw[fill=imperialred!50] (5,5.5) rectangle (5.4,5.8); \node[anchor=west, font=\tiny] at (5.5,5.65) {منسوجات};
\draw[fill=industrialgray!50] (5,5) rectangle (5.4,5.3); \node[anchor=west, font=\tiny] at (5.5,5.15) {فولاد/زغال};
\draw[fill=royalblue!50] (5,4.5) rectangle (5.4,4.8); \node[anchor=west, font=\tiny] at (5.5,4.65) {کشتی‌سازی};
\draw[fill=tudorpurple!50] (5,4) rectangle (5.4,4.3); \node[anchor=west, font=\tiny] at (5.5,4.15) {مالی/خدمات};

\end{tikzpicture}
\caption{جغرافیای صنعتی بریتانیا (اواخر قرن ۱۹)}
\end{figure}


%%%%%%%%%%%%%%%%%%%%%%%%%%%%%%%%%%%%%%%%%%%%%%%%%%%%%%%%%%%%%%%%%%%%%%%
% پیوست ۵: شجره‌نامه سلسله‌های پادشاهی
%%%%%%%%%%%%%%%%%%%%%%%%%%%%%%%%%%%%%%%%%%%%%%%%%%%%%%%%%%%%%%%%%%%%%%%

\chapter{پیوست ۵: سلسله‌های پادشاهی بریتانیا}

\begin{figure}[H]
\centering
\begin{tikzpicture}[
    dynasty/.style={rectangle, rounded corners=5pt, draw, thick,
                    minimum width=2.5cm, minimum height=1.2cm,
                    text centered, font=\small}
]

% خط زمانی
\draw[very thick, royalblue] (0,0) -- (16,0);

% سلسله‌ها
\node[dynasty, fill=imperialred!30] at (0.8,2) {نورمان\\۱۰۶۶-۱۱۵۴};
\node[dynasty, fill=parliamentgold!30] at (2.8,2) {پلانتاژنت\\۱۱۵۴-۱۴۸۵};
\node[dynasty, fill=tudorpurple!30] at (5.2,2) {تودور\\۱۴۸۵-۱۶۰۳};
\node[dynasty, fill=victoriangreen!30] at (7.5,2) {استوارت\\۱۶۰۳-۱۷۱۴};
\node[dynasty, fill=royalblue!30] at (10,2) {هانوور\\۱۷۱۴-۱۹۰۱};
\node[dynasty, fill=industrialgray!30] at (12.5,2) {ساکس-کوبورگ\\۱۹۰۱-۱۹۱۷};
\node[dynasty, fill=empirecream, draw=parliamentgold] at (15,2) {ویندزور\\۱۹۱۷-};

% تاریخ‌ها
\foreach \x/\year in {0/1066, 2/1154, 4.2/1485, 6.3/1603, 8.5/1714, 11/1901, 13/1917, 16/2024} {
    \draw[thick] (\x,0.2) -- (\x,-0.2);
    \node[below, font=\tiny] at (\x,-0.3) {\year};
}

% فلش‌ها
\draw[-{Stealth}, thick] (1.6,2) -- (2,2);
\draw[-{Stealth}, thick] (3.8,2) -- (4.2,2);
\draw[-{Stealth}, thick] (6.2,2) -- (6.5,2);
\draw[-{Stealth}, thick] (8.5,2) -- (9,2);
\draw[-{Stealth}, thick] (11,2) -- (11.5,2);
\draw[-{Stealth}, thick] (13.5,2) -- (14,2);

\end{tikzpicture}
\caption{سلسله‌های پادشاهی بریتانیا}
\end{figure}

\begin{table}[H]
\centering
\caption{پادشاهان و ملکه‌های مهم}
\renewcommand{\arraystretch}{1.3}
\begin{tabularx}{\textwidth}{>{\bfseries}p{3cm}cX}
\toprule
\textbf{پادشاه/ملکه} & \textbf{سلطنت} & \textbf{اهمیت تاریخی} \\
\midrule
\rowcolor{empirecream}
ویلیام فاتح & ۱۰۶۶-۱۰۸۷ & فتح نورمان، نظام فئودالی \\
جان & ۱۱۹۹-۱۲۱۶ & مگناکارتا \\
\rowcolor{empirecream}
ادوارد اول & ۱۲۷۲-۱۳۰۷ & پارلمان نمونه، فتح ویلز \\
هنری هشتم & ۱۵۰۹-۱۵۴۷ & جدایی از رم، کلیسای انگلستان \\
\rowcolor{empirecream}
الیزابت اول & ۱۵۵۸-۱۶۰۳ & عصر طلایی، شکست آرمادا \\
چارلز اول & ۱۶۲۵-۱۶۴۹ & جنگ داخلی، اعدام \\
\rowcolor{empirecream}
ویلیام سوم و مری دوم & ۱۶۸۹-۱۷۰۲ & انقلاب شکوهمند \\
جرج سوم & ۱۷۶۰-۱۸۲۰ & از دست رفتن آمریکا، جنگ ناپلئون \\
\rowcolor{empirecream}
ویکتوریا & ۱۸۳۷-۱۹۰۱ & عصر ویکتوریایی، اوج امپراتوری \\
الیزابت دوم & ۱۹۵۲-۲۰۲۲ & طولانی‌ترین سلطنت، فروپاشی امپراتوری \\
\bottomrule
\end{tabularx}
\end{table}


%%%%%%%%%%%%%%%%%%%%%%%%%%%%%%%%%%%%%%%%%%%%%%%%%%%%%%%%%%%%%%%%%%%%%%%
% پیوست ۶: جدول نخست‌وزیران
%%%%%%%%%%%%%%%%%%%%%%%%%%%%%%%%%%%%%%%%%%%%%%%%%%%%%%%%%%%%%%%%%%%%%%%

\chapter{پیوست ۶: نخست‌وزیران بریتانیا}

\begin{longtable}{>{\bfseries}clcc}
\caption{نخست‌وزیران بریتانیا (از ۱۷۲۱)} \\
\toprule
\textbf{نام} & \textbf{حزب} & \textbf{از} & \textbf{تا} \\
\midrule
\endfirsthead
\toprule
\textbf{نام} & \textbf{حزب} & \textbf{از} & \textbf{تا} \\
\midrule
\endhead
\bottomrule
\endlastfoot

\rowcolor{empirecream}
رابرت والپول & ویگ & ۱۷۲۱ & ۱۷۴۲ \\
ویلیام پیت (بزرگ) & ویگ & ۱۷۶۶ & ۱۷۶۸ \\
\rowcolor{empirecream}
ویلیام پیت (کوچک) & توری & ۱۷۸۳ & ۱۸۰۱ \\
لرد گری & ویگ & ۱۸۳۰ & ۱۸۳۴ \\
\rowcolor{empirecream}
رابرت پیل & محافظه‌کار & ۱۸۴۱ & ۱۸۴۶ \\
ویلیام گلادستون & لیبرال & ۱۸۶۸ & ۱۸۷۴ \\
\rowcolor{empirecream}
بنجامین دیزرائیلی & محافظه‌کار & ۱۸۷۴ & ۱۸۸۰ \\
ویلیام گلادستون & لیبرال & ۱۸۸۰ & ۱۸۸۵ \\
\rowcolor{empirecream}
هربرت آسکوئیث & لیبرال & ۱۹۰۸ & ۱۹۱۶ \\
دیوید لوید جورج & لیبرال & ۱۹۱۶ & ۱۹۲۲ \\
\rowcolor{empirecream}
رمزی مک‌دونالد & کارگر & ۱۹۲۴ & ۱۹۲۴ \\
استنلی بالدوین & محافظه‌کار & ۱۹۲۴ & ۱۹۲۹ \\
\rowcolor{empirecream}
رمزی مک‌دونالد & ملی/کارگر & ۱۹۲۹ & ۱۹۳۵ \\
نویل چمبرلین & محافظه‌کار & ۱۹۳۷ & ۱۹۴۰ \\
\rowcolor{empirecream}
وینستون چرچیل & محافظه‌کار & ۱۹۴۰ & ۱۹۴۵ \\
کلمنت اتلی & کارگر & ۱۹۴۵ & ۱۹۵۱ \\
\rowcolor{empirecream}
وینستون چرچیل & محافظه‌کار & ۱۹۵۱ & ۱۹۵۵ \\
هارولد مک‌میلان & محافظه‌کار & ۱۹۵۷ & ۱۹۶۳ \\
\rowcolor{empirecream}
هارولد ویلسون & کارگر & ۱۹۶۴ & ۱۹۷۰ \\
ادوارد هیث & محافظه‌کار & ۱۹۷۰ & ۱۹۷۴ \\
\rowcolor{empirecream}
هارولد ویلسون & کارگر & ۱۹۷۴ & ۱۹۷۶ \\
جیمز کالاهان & کارگر & ۱۹۷۶ & ۱۹۷۹ \\
\rowcolor{imperialred!15}
مارگارت تاچر & محافظه‌کار & ۱۹۷۹ & ۱۹۹۰ \\
\rowcolor{empirecream}
جان میجر & محافظه‌کار & ۱۹۹۰ & ۱۹۹۷ \\
\rowcolor{royalblue!15}
تونی بلر & کارگر & ۱۹۹۷ & ۲۰۰۷ \\
\rowcolor{empirecream}
گوردون براون & کارگر & ۲۰۰۷ & ۲۰۱۰ \\
دیوید کامرون & محافظه‌کار & ۲۰۱۰ & ۲۰۱۶ \\
\rowcolor{empirecream}
ترزا می & محافظه‌کار & ۲۰۱۶ & ۲۰۱۹ \\
بوریس جانسون & محافظه‌کار & ۲۰۱۹ & ۲۰۲۲ \\
\rowcolor{empirecream}
لیز تراس & محافظه‌کار & ۲۰۲۲ & ۲۰۲۲ \\
ریشی سوناک & محافظه‌کار & ۲۰۲۲ & ۲۰۲۴ \\

\end{longtable}


%%%%%%%%%%%%%%%%%%%%%%%%%%%%%%%%%%%%%%%%%%%%%%%%%%%%%%%%%%%%%%%%%%%%%%%
% پیوست ۷: مقایسه آماری بین‌المللی
%%%%%%%%%%%%%%%%%%%%%%%%%%%%%%%%%%%%%%%%%%%%%%%%%%%%%%%%%%%%%%%%%%%%%%%

\chapter{پیوست ۷: مقایسه‌های بین‌المللی}

\begin{figure}[H]
\centering
\begin{tikzpicture}
\begin{axis}[
    width=14cm,
    height=9cm,
    xlabel={GDP سرانه (هزار دلار، PPP ۲۰۲۱)},
    ylabel={شاخص توسعه انسانی (HDI)},
    xmin=30, xmax=75,
    ymin=0.88, ymax=0.96,
    grid=major,
    legend pos=south east,
]

% کشورها
\addplot[only marks, mark=*, mark size=4pt, royalblue] 
    coordinates {(46, 0.929)} node[above right, font=\small] {بریتانیا};
    
\addplot[only marks, mark=square*, mark size=4pt, imperialred] 
    coordinates {(63, 0.921)} node[above, font=\small] {آمریکا};
    
\addplot[only marks, mark=triangle*, mark size=4pt, industrialgray] 
    coordinates {(54, 0.942)} node[above, font=\small] {آلمان};
    
\addplot[only marks, mark=diamond*, mark size=4pt, victoriangreen] 
    coordinates {(55, 0.952)} node[above, font=\small] {سوئد};
    
\addplot[only marks, mark=pentagon*, mark size=4pt, tudorpurple] 
    coordinates {(45, 0.903)} node[below, font=\small] {فرانسه};
    
\addplot[only marks, mark=o, mark size=4pt, parliamentgold] 
    coordinates {(42, 0.925)} node[below right, font=\small] {ژاپن};

\end{axis}
\end{tikzpicture}
\caption{مقایسه GDP سرانه و HDI (۲۰۲۱)}
\end{figure}

\begin{table}[H]
\centering
\caption{مقایسه شاخص‌های کلیدی (۲۰۲۲)}
\renewcommand{\arraystretch}{1.3}
\begin{tabular}{lcccccc}
\toprule
\textbf{شاخص} & \textbf{UK} & \textbf{آمریکا} & \textbf{آلمان} & \textbf{فرانسه} & \textbf{سوئد} & \textbf{ژاپن} \\
\midrule
\rowcolor{empirecream}
GDP سرانه (هزار \$) & ۴۶ & ۶۳ & ۵۴ & ۴۵ & ۵۵ & ۴۲ \\
ضریب جینی & ۰.۳۵ & ۰.۳۹ & ۰.۲۹ & ۰.۲۹ & ۰.۲۷ & ۰.۳۳ \\
\rowcolor{empirecream}
بدهی عمومی (\% GDP) & ۱۰۱ & ۱۲۹ & ۶۹ & ۱۱۲ & ۳۷ & ۲۶۳ \\
بیکاری (\%) & ۳.۷ & ۳.۶ & ۳.۱ & ۷.۳ & ۷.۵ & ۲.۶ \\
\rowcolor{empirecream}
امید به زندگی (سال) & ۸۱ & ۷۷ & ۸۱ & ۸۳ & ۸۳ & ۸۴ \\
هزینه بهداشت (\% GDP) & ۱۲ & ۱۷ & ۱۲ & ۱۲ & ۱۱ & ۱۱ \\
\rowcolor{empirecream}
انتشار CO2 (تن/نفر) & ۵.۲ & ۱۴.۷ & ۸.۱ & ۴.۷ & ۳.۸ & ۸.۵ \\
\bottomrule
\end{tabular}
\end{table}


%%%%%%%%%%%%%%%%%%%%%%%%%%%%%%%%%%%%%%%%%%%%%%%%%%%%%%%%%%%%%%%%%%%%%%%
% پیوست ۸: منابع آنلاین و آرشیوها
%%%%%%%%%%%%%%%%%%%%%%%%%%%%%%%%%%%%%%%%%%%%%%%%%%%%%%%%%%%%%%%%%%%%%%%

\chapter{پیوست ۸: منابع آنلاین و آرشیوها}

\section*{آرشیوهای رسمی}

\begin{table}[H]
\centering
\renewcommand{\arraystretch}{1.4}
\begin{tabularx}{\textwidth}{>{\bfseries}p{4cm}Xp{4cm}}
\toprule
\textbf{نام} & \textbf{محتوا} & \textbf{وب‌سایت} \\
\midrule
\rowcolor{empirecream}
آرشیو ملی (TNA) & اسناد دولتی از قرون وسطی & nationalarchives.gov.uk \\
کتابخانه بریتانیا & کتب، نقشه‌ها، روزنامه‌ها & bl.uk \\
\rowcolor{empirecream}
آرشیو پارلمان & مذاکرات، قوانین & parliament.uk/archives \\
آرشیو حزب کارگر & اسناد حزبی & labour.org.uk \\
\rowcolor{empirecream}
آرشیو محافظه‌کار & اسناد حزبی & bodleian.ox.ac.uk/cpa \\
\bottomrule
\end{tabularx}
\end{table}

\section*{پایگاه‌های داده آماری}

\begin{table}[H]
\centering
\renewcommand{\arraystretch}{1.4}
\begin{tabularx}{\textwidth}{>{\bfseries}p{4cm}Xp{4cm}}
\toprule
\textbf{نام} & \textbf{محتوا} & \textbf{وب‌سایت} \\
\midrule
\rowcolor{empirecream}
اداره آمار ملی (ONS) & آمار اقتصادی و اجتماعی & ons.gov.uk \\
بانک انگلستان & داده‌های مالی تاریخی & bankofengland.co.uk \\
\rowcolor{empirecream}
پروژه مدیسون & GDP تاریخی & rug.nl/ggdc/maddison \\
Our World in Data & داده‌های تطبیقی & ourworldindata.org \\
\bottomrule
\end{tabularx}
\end{table}

\section*{منابع آکادمیک آنلاین}

\begin{table}[H]
\centering
\renewcommand{\arraystretch}{1.4}
\begin{tabularx}{\textwidth}{>{\bfseries}p{4cm}Xp{4cm}}
\toprule
\textbf{نام} & \textbf{محتوا} & \textbf{وب‌سایت} \\
\midrule
\rowcolor{empirecream}
JSTOR & مجلات آکادمیک & jstor.org \\
Oxford DNB & بیوگرافی ملی & oxforddnb.com \\
\rowcolor{empirecream}
History of Parliament & تاریخ پارلمان & historyofparliamentonline.org \\
British History Online & متون اولیه & british-history.ac.uk \\
\bottomrule
\end{tabularx}
\end{table}


%%%%%%%%%%%%%%%%%%%%%%%%%%%%%%%%%%%%%%%%%%%%%%%%%%%%%%%%%%%%%%%%%%%%%%%
% صفحه پایانی
%%%%%%%%%%%%%%%%%%%%%%%%%%%%%%%%%%%%%%%%%%%%%%%%%%%%%%%%%%%%%%%%%%%%%%%

\newpage
\thispagestyle{empty}
\vspace*{\fill}

\begin{center}
\begin{tikzpicture}
\node[rectangle, rounded corners=20pt, draw=royalblue, very thick,
      fill=royalblue!5, minimum width=12cm, minimum height=10cm] {
\begin{minipage}{11cm}
\centering
\vspace{0.5cm}

{\Huge\bfseries\color{royalblue} پایان}\\[1cm]

{\Large تطور جامعه بریتانیا}\\[0.5cm]
{\large از فتح نورمان تا برگزیت}\\[0.5cm]
{\large (۱۰۶۶-۲۰۲۴)}\\[1.5cm]

\rule{0.5\textwidth}{0.5pt}\\[1cm]

{\small این پژوهش با هدف ارائه تحلیلی جامع و چندبُعدی}\\
{\small از تحولات تاریخی بریتانیا تهیه شده است.}\\[0.5cm]
{\small امید است برای پژوهشگران، دانشجویان،}\\
{\small و علاقه‌مندان به تاریخ مفید واقع شود.}\\[1.5cm]

{\footnotesize گروه پژوهشی چندتخصصی}\\[0.3cm]
{\footnotesize \today}

\vspace{0.5cm}
\end{minipage}
};
\end{tikzpicture}
\end{center}

\vspace*{\fill}


\end{document}