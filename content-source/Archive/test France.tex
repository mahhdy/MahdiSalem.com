%%%%%%%%%%%%%%%%%%%%%%%%%%%%%%%%%%%%%%%%%%%%%%%%%%%%%%%%%%%%%%%%%%%%%%%
%  فایل تست جامع - نسخه ۱.۱ (اصلاح‌شده)
%  کامپایلر: XeLaTeX
%%%%%%%%%%%%%%%%%%%%%%%%%%%%%%%%%%%%%%%%%%%%%%%%%%%%%%%%%%%%%%%%%%%%%%%

\documentclass[12pt, a4paper]{article}

%======================================================================
%                    پکیج‌ها (ترتیب مهم است!)
%======================================================================

\usepackage{amsmath}
\usepackage{amssymb}          % ← این خط را اضافه کنید
\usepackage{booktabs}
\usepackage{tabularx}
\usepackage{colortbl}
\usepackage{array}

\usepackage{tikz}
\usepackage{pgfplots}
\pgfplotsset{compat=1.18}
\usetikzlibrary{
	shapes.geometric,
	arrows.meta,
	positioning,
	calc
}

\usepackage{graphicx}
\usepackage{float}
\usepackage{xcolor}

\usepackage{tcolorbox}
\tcbuselibrary{skins, breakable}

\usepackage{geometry}
\geometry{margin=2.5cm}

%======================================================================
%         فارسی - حتماً آخرین پکیج!
%======================================================================

\usepackage{xepersian}
\settextfont[Scale=1.1]{Vazirmatn}

%======================================================================
%                    تعریف رنگ‌ها
%======================================================================

\definecolor{bleurepublique}{RGB}{0, 35, 149}
\definecolor{rougerevolution}{RGB}{237, 41, 57}
\definecolor{orroyal}{RGB}{255, 215, 0}
\definecolor{vertnapoleon}{RGB}{0, 100, 0}
\definecolor{violetempire}{RGB}{128, 0, 128}
\definecolor{fondclair}{RGB}{255, 253, 240}

%======================================================================
%                    کادرهای سفارشی
%======================================================================

\newtcolorbox{keybox}[1][نکته کلیدی]{
	enhanced,
	colback=bleurepublique!8,
	colframe=bleurepublique,
	fonttitle=\bfseries,
	title=#1,
	boxrule=1.5pt,
	arc=4pt
}

\newtcolorbox{revbox}[1][نقطه عطف]{
	enhanced,
	colback=rougerevolution!8,
	colframe=rougerevolution,
	fonttitle=\bfseries,
	title=#1,
	boxrule=1.5pt,
	arc=4pt
}

\newtcolorbox{patternbox}[1][الگو]{
	enhanced,
	colback=vertnapoleon!8,
	colframe=vertnapoleon,
	fonttitle=\bfseries,
	title=#1,
	boxrule=1.5pt,
	arc=4pt
}

\newtcolorbox{quotebox}{
	enhanced,
	colback=fondclair,
	colframe=orroyal,
	boxrule=1.5pt,
	arc=4pt,
	left=10pt, right=10pt,
	before upper={\itshape}
}

%======================================================================
%                    شروع سند
%======================================================================

\begin{document}
	
	\begin{center}
		{\Huge\bfseries\color{bleurepublique} فایل تست جامع}\\[0.5cm]
		{\large تمام عناصر: جدول، نمودار، نقشه مفهومی، کادر}\\[0.3cm]
		{\small نسخه ۱.۱ --- کامپایلر: XeLaTeX}
	\end{center}
	
	\tableofcontents
	
	%----------------------------------------------------------------------
	\section{تست کادرها}
	%----------------------------------------------------------------------
	
	\begin{keybox}[نکته کلیدی]
		این یک کادر آبی برای نکات مهم است.
		فرانسه بین ۱۷۸۹ تا ۱۹۵۸ پانزده رژیم مختلف داشت!
	\end{keybox}
	
	\begin{revbox}[انقلاب ۱۷۸۹]
		۱۴ ژوئیه ۱۷۸۹: سقوط باستیل.
		این روز به نماد انقلاب فرانسه تبدیل شد.
	\end{revbox}
	
	\begin{patternbox}[الگوی شماره ۱]
		انقلاب‌ها معمولاً در شرایط زیر رخ می‌دهند:
		\begin{itemize}
			\item بحران مالی دولت
			\item شکاف در نخبگان
			\item بسیج توده‌ای
		\end{itemize}
	\end{patternbox}
	
	\begin{quotebox}
		«آزادی، برابری، برادری»\\[0.3cm]
		\hfill --- شعار انقلاب فرانسه
	\end{quotebox}
	
	%----------------------------------------------------------------------
	\section{تست جدول}
	%----------------------------------------------------------------------
	
	\begin{table}[H]
		\centering
		\caption{پنج جمهوری فرانسه}
		\renewcommand{\arraystretch}{1.4}
		\begin{tabular}{|r|c|c|c|}
			\hline
			\rowcolor{bleurepublique!20}
			\textbf{جمهوری} & \textbf{دوره} & \textbf{طول عمر} & \textbf{پایان} \\
			\hline
			اول & ۱۷۹۲-۱۸۰۴ & ۱۲ سال & کودتای ناپلئون \\
			\hline
			\rowcolor{fondclair}
			دوم & ۱۸۴۸-۱۸۵۲ & ۴ سال & کودتای لویی ناپلئون \\
			\hline
			سوم & ۱۸۷۰-۱۹۴۰ & ۷۰ سال & شکست از آلمان \\
			\hline
			\rowcolor{fondclair}
			چهارم & ۱۹۴۶-۱۹۵۸ & ۱۲ سال & بحران الجزایر \\
			\hline
			پنجم & ۱۹۵۸-اکنون & ۶۶+ سال & ادامه دارد \\
			\hline
		\end{tabular}
	\end{table}
	
	%----------------------------------------------------------------------
	\section{تست نمودار خطی}
	%----------------------------------------------------------------------
	
	\begin{figure}[H]
		\centering
		\begin{tikzpicture}
			\begin{axis}[
				width=12cm,
				height=7cm,
				xlabel={سال},
				ylabel={تعداد انقلاب‌ها (تجمعی)},
				xmin=1780, xmax=1880,
				ymin=0, ymax=6,
				grid=major,
				xtick={1789, 1830, 1848, 1871},
				ytick={0,1,2,3,4,5},
				]
				\addplot[very thick, rougerevolution, mark=*, mark size=4pt] coordinates {
					(1789, 1) (1830, 2) (1848, 3) (1871, 4)
				};
			\end{axis}
		\end{tikzpicture}
		\caption{انقلاب‌های فرانسه در قرن نوزدهم}
	\end{figure}
	
	%----------------------------------------------------------------------
	\section{تست نقشه مفهومی}
	%----------------------------------------------------------------------
	
	\begin{figure}[H]
		\centering
		\begin{tikzpicture}[
			node distance=2cm,
			box/.style={
				rectangle, 
				rounded corners=8pt, 
				draw=#1, 
				thick,
				fill=#1!15, 
				minimum width=2.8cm,
				minimum height=1.2cm, 
				text centered,
				font=\small
			}
			]
			
			\node[box=bleurepublique, minimum width=3.5cm, minimum height=1.5cm] 
			(center) {انقلاب فرانسه};
			
			\node[box=rougerevolution, above left=of center] (cause1) {بحران مالی};
			\node[box=rougerevolution, above right=of center] (cause2) {نارضایتی طبقات};
			\node[box=vertnapoleon, below left=of center] (result1) {حقوق بشر};
			\node[box=vertnapoleon, below right=of center] (result2) {جمهوری};
			
			\draw[-{Stealth}, thick, rougerevolution] (cause1) -- (center);
			\draw[-{Stealth}, thick, rougerevolution] (cause2) -- (center);
			\draw[-{Stealth}, thick, vertnapoleon] (center) -- (result1);
			\draw[-{Stealth}, thick, vertnapoleon] (center) -- (result2);
			
			\node[above=3cm of center, font=\bfseries\color{rougerevolution}] {علل};
			\node[below=3cm of center, font=\bfseries\color{vertnapoleon}] {نتایج};
			
		\end{tikzpicture}
		\caption{نقشه مفهومی: علل و نتایج انقلاب}
	\end{figure}
	
	%----------------------------------------------------------------------
	\section{تست نقشه با متن چندخطی}
	%----------------------------------------------------------------------
	
	\begin{figure}[H]
		\centering
		\begin{tikzpicture}[
			multibox/.style={
				rectangle, 
				rounded corners=8pt, 
				draw=#1, 
				thick,
				fill=#1!12,
				text centered,
				font=\small,
				inner sep=8pt
			}
			]
			
			\node[multibox=violetempire] (napoleon) {
				\begin{tabular}{c}
					\textbf{ناپلئون بناپارت}\\
					۱۷۶۹-۱۸۲۱\\
					امپراتور فرانسه
				\end{tabular}
			};
			
			\node[multibox=bleurepublique, right=3cm of napoleon] (legacy) {
				\begin{tabular}{c}
					\textbf{میراث ناپلئون}\\
					قانون مدنی\\
					نظام اداری مدرن
				\end{tabular}
			};
			
			\node[multibox=rougerevolution, left=3cm of napoleon] (rise) {
				\begin{tabular}{c}
					\textbf{ظهور}\\
					انقلاب ۱۷۸۹\\
					جنگ‌های انقلابی
				\end{tabular}
			};
			
			\draw[-{Stealth}, very thick] (rise) -- (napoleon);
			\draw[-{Stealth}, very thick] (napoleon) -- (legacy);
			
		\end{tikzpicture}
		\caption{نقشه مفهومی با متن چندخطی}
	\end{figure}
	
	%----------------------------------------------------------------------
	\section{تست خط زمانی}
	%----------------------------------------------------------------------
	
	\begin{figure}[H]
		\centering
		\begin{tikzpicture}
			
			\draw[very thick, bleurepublique] (0,0) -- (14,0);
			
			\foreach \x/\year/\event/\col in {
				0/1789/انقلاب/rougerevolution,
				3.5/1804/امپراتوری/violetempire,
				7/1848/جمهوری دوم/bleurepublique,
				10.5/1870/جمهوری سوم/bleurepublique,
				14/1958/جمهوری پنجم/bleurepublique
			} {
				\fill[\col] (\x,0) circle (6pt);
				\node[below=8pt, font=\footnotesize] at (\x,0) {\year};
				\node[above=8pt, font=\footnotesize\bfseries, \col] at (\x,0) {\event};
			}
			
		\end{tikzpicture}
		\caption{خط زمانی تحولات فرانسه}
	\end{figure}
	
	%----------------------------------------------------------------------
	\section{تست نمودار ستونی}
	%----------------------------------------------------------------------
	
	\begin{figure}[H]
		\centering
		\begin{tikzpicture}
			\begin{axis}[
				ybar,
				width=12cm,
				height=7cm,
				ylabel={طول عمر (سال)},
				symbolic x coords={اول, دوم, سوم, چهارم, پنجم},
				xtick=data,
				ymin=0, ymax=80,
				bar width=25pt,
				nodes near coords,
				every node near coord/.append style={font=\small},
				]
				\addplot[fill=bleurepublique!60, draw=bleurepublique] coordinates {
					(اول, 12) (دوم, 4) (سوم, 70) (چهارم, 12) (پنجم, 66)
				};
			\end{axis}
		\end{tikzpicture}
		\caption{مقایسه طول عمر جمهوری‌ها}
	\end{figure}
	
	%----------------------------------------------------------------------
	\section*{نتیجه تست}
	%----------------------------------------------------------------------
	
	\begin{keybox}[نتیجه]
		اگر این صفحه را بدون خطا می‌بینید، همه چیز درست کار می‌کند!
		
		چک‌لیست:
		\begin{itemize}
			\item[$\checkmark$] کادرهای رنگی
			\item[$\checkmark$] جدول با رنگ‌بندی
			\item[$\checkmark$] نمودار خطی
			\item[$\checkmark$] نمودار ستونی
			\item[$\checkmark$] نقشه مفهومی
			\item[$\checkmark$] خط زمانی
			\item[$\checkmark$] متن فارسی در همه عناصر
		\end{itemize}
	\end{keybox}
	
\end{document}