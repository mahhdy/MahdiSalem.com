% ============================================================
% تنظیمات فونت — پس از xepersian
% ============================================================
% ╔══════════════════════════════════════════════════════════════════╗
% ║  تنظیمات فونت — پس از xepersian                                 ║
% ╚══════════════════════════════════════════════════════════════════╝

% ═══════════════════════════════════════════════════════
% فونت فارسی اصلی
% ═══════════════════════════════════════════════════════

% گزینه ۱: فونت سیستمی (توصیه‌شده)
\settextfont{XB Zar}

% گزینه ۲: فونت از پوشه
% \settextfont[
%     Scale=1.0,
%     Extension=.ttf,
%     Path=fonts/,
%     BoldFont=XB Zar Bold,
%     ItalicFont=XB Zar Italic
% ]{XB Zar}

% گزینه ۳: فونت‌های جایگزین
% \settextfont{Vazir}
% \settextfont{B Nazanin}
% \settextfont{Iranian Sans}
% \settextfont{Sahel}

% ═══════════════════════════════════════════════════════
% فونت لاتین
% ═══════════════════════════════════════════════════════

% گزینه ۱: فونت سیستمی
\setlatintextfont{Times New Roman}

% گزینه ۲: فونت‌های جایگزین
% \setlatintextfont{Georgia}
% \setlatintextfont{Palatino Linotype}
% \setlatintextfont{Linux Libertine O}

% ═══════════════════════════════════════════════════════
% فونت اعداد
% ═══════════════════════════════════════════════════════
\setdigitfont{XB Zar}

% ═══════════════════════════════════════════════════════
% فونت عناوین (اختیاری)
% ═══════════════════════════════════════════════════════
% \defpersianfont\titlefont[Scale=1.2]{XB Titre}

% ═══════════════════════════════════════════════════════
% فونت کد (Monospace)
% ═══════════════════════════════════════════════════════
\setmonofont[Scale=0.85]{Consolas}
% جایگزین: Courier New, Fira Code, DejaVu Sans Mono|
% ✅ تعریف فونت آمار و ارقام بزرگ
\newfontfamily\statisticfont[Scale=2.5]{XB Zar}

% جایگزین اگر می‌خواهید فونت متفاوت: