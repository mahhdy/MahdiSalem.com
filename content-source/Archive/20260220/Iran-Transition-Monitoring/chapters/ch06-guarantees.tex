% ╔══════════════════════════════════════════════════════════════════╗
% ║  فصل ۶: تضمین‌های موفقیت و پیش‌شرط‌های ساختاری               ║
% ║  شش حوزه‌ی تضمین + ماتریس پیش‌شرط‌ها                         ║
% ╚══════════════════════════════════════════════════════════════════╝

% ---- صفحه‌ی آغازین فصل ----
\chapteropening{۶}
    {تضمین‌های موفقیت و پیش‌شرط‌های ساختاری}
    {MainGreen}
    {دموکراسی بدون پیش‌شرط ممکن نیست. 
    اما پیش‌شرط‌ها بهانه‌ای برای تأخیر نیستند — 
    ابزاری برای آمادگی هستند.}
    {آمارتیا سن، اقتصاددان و فیلسوف هندی، برنده نوبل}

\chapter{تضمین‌های موفقیت و پیش‌شرط‌های ساختاری}
\label{ch:guarantees}
\minitoc

% ---- خلاصه‌ی اجرایی ----
\begin{executivesummary}
نظارت بین‌المللی در خلأ عمل نمی‌کند. 
موفقیت آن مستلزم فراهم بودن مجموعه‌ای 
از \emphgreen{پیش‌شرط‌ها و تضمین‌ها} در 
شش حوزه است: سیاسی، امنیتی، حقوقی، 
اقتصادی، اجتماعی و نهادی. این فصل 
هر حوزه را تشریح می‌کند، پیش‌شرط‌ها 
را اولویت‌بندی می‌کند (حیاتی / مهم / مطلوب) 
و مسئول تأمین هر یک را مشخص می‌سازد. 
هدف آن است که بازیگران بدانند 
\emphgreen{قبل از آغاز، حین اجرا و 
برای تداوم} نظارت چه چیزهایی باید 
فراهم باشد.
\end{executivesummary}

% ============================================================
\section{چارچوب شش‌گانه‌ی تضمین‌ها}
\label{sec:guarantee-framework}
% ============================================================

\begin{figure}[htbp]
    \centering
    \begin{tikzpicture}[
        pillar/.style={
            draw=#1, fill=#1!10,
            rounded corners=3pt,
            minimum height=3cm, minimum width=2.2cm,
            align=center, font=\footnotesize\bfseries
        },
        base/.style={
            draw=DarkGray, fill=VeryLightGray,
            minimum height=0.8cm, minimum width=15cm,
            align=center, font=\small\bfseries
        },
        roof/.style={
            draw=MainPurple, fill=PurpleBG,
            minimum height=1cm, minimum width=15cm,
            align=center, font=\normalsize\bfseries
        }
    ]
    
    % پایه
    \node[base] (foundation) at (0, 0) 
        {مالکیت ملی + اراده‌ی مردم ایران};
    
    % ستون‌ها
    \node[pillar=MainBlue] (pol) at (-6.2, 2.5) 
        {سیاسی\\[4pt]\faUsers\\[4pt]
        \tiny اجماع\\حمایت\\قواعد بازی};
    
    \node[pillar=MainRed] (sec) at (-3.7, 2.5) 
        {امنیتی\\[4pt]\faShieldAlt\\[4pt]
        \tiny سپاه\\DDR\\مرزها};
    
    \node[pillar=MainOrange] (leg) at (-1.2, 2.5) 
        {حقوقی\\[4pt]\faGavel\\[4pt]
        \tiny SOMA\\قانون\\معاهدات};
    
    \node[pillar=MainGreen] (eco) at (1.3, 2.5) 
        {اقتصادی\\[4pt]\faChartLine\\[4pt]
        \tiny تحریم\\مارشال\\صندوق};
    
    \node[pillar=DarkYellow] (soc) at (3.8, 2.5) 
        {اجتماعی\\[4pt]\faHeart\\[4pt]
        \tiny زنان\\اقوام\\آشتی};
    
    \node[pillar=MainPurple] (inst) at (6.3, 2.5) 
        {نهادی\\[4pt]\faUniversity\\[4pt]
        \tiny انتخابات\\قضا\\رسانه};
    
    % سقف
    \node[roof] (success) at (0, 4.8) 
        {\faCheckDouble\hspace{8pt}
        گذار دموکراتیک موفق و پایدار
        \hspace{8pt}\faCheckDouble};
    
    \end{tikzpicture}
    \caption{شش ستون تضمین موفقیت 
    گذار دموکراتیک}
    \label{fig:six-pillars}
\end{figure}

\sectiondivider

% ============================================================
\section{تضمین‌های سیاسی}
\label{sec:political-guarantees}
% ============================================================

\subsection{اجماع حداقلی نیروهای سیاسی}

\begin{definitionbox}{اجماع حداقلی 
(\lr{Minimum Consensus})}
توافق طرف‌های اصلی (نه لزوماً همه) 
بر «قواعد بازی» — یعنی فرایند گذار، 
نقش نظارت بین‌المللی و اصول 
غیرقابل مذاکره (مانند حقوق بشر و 
دموکراسی) — حتی اگر بر محتوای 
سیاسی (نوع نظام آینده) 
اختلاف داشته باشند.
\end{definitionbox}

\begin{table}[htbp]
    \centering
    \caption{عناصر اجماع حداقلی سیاسی}
    \label{tab:political-consensus}
    \begin{tabularx}{\textwidth}{
        C{0.6cm} X L{3cm}
    }
        \toprule
        \headerrow
        \textbf{\#} & 
        \textbf{عنصر اجماع} & 
        \textbf{نمونه‌ی تاریخی} \\
        \midrule
        
        ۱ &
        پذیرش اصل انتخابات آزاد 
        به‌عنوان تنها مکانیزم 
        مشروع تعیین قدرت &
        لهستان: میز گرد ۱۹۸۹ \\
        \altrow
        
        ۲ &
        توافق بر فرایند تدوین 
        قانون اساسی جدید 
        (چه کسی، چگونه، چه وقت) &
        آفریقای جنوبی: \lr{CODESA} \\
        
        ۳ &
        تعهد به عدم خشونت و 
        حل اختلاف از طریق 
        مکانیزم‌های مدنی &
        اسپانیا: پیمان مونکلوا \\
        \altrow
        
        ۴ &
        پذیرش نظارت بین‌المللی 
        به‌عنوان تسهیل‌کننده 
        (نه مداخله‌گر) &
        تیمور شرقی \\
        
        ۵ &
        توافق بر خطوط قرمز مشترک: 
        عدم انتقام‌جویی کور، 
        عدم تجزیه‌طلبی خشونت‌آمیز &
        آفریقای جنوبی \\
        \altrow
        
        ۶ &
        قبول حق مشارکت همه‌ی 
        گروه‌ها (حتی اعضای 
        سابق نظام قدیم که 
        مرتکب جنایت نشده‌اند) &
        اسپانیا: عفو مشروط \\
        
        \bottomrule
    \end{tabularx}
\end{table}

\begin{warningbox}
\textbf{هشدار: اجماع = اتفاق‌نظر کامل نیست}

اجماع حداقلی به معنای توافق بر 
\emphred{قواعد بازی} است، نه بر 
\emphred{نتیجه‌ی بازی}. طرف‌ها 
می‌توانند (و باید) بر سر نوع نظام، 
سیاست‌های اقتصادی و ارزش‌های 
فرهنگی رقابت کنند — به شرط آنکه 
بر قواعد رقابت دموکراتیک توافق 
داشته باشند.

\vspace{4pt}
\textbf{بزرگ‌ترین چالش ایران:} 
اپوزیسیون ایران هنوز حتی بر 
این قواعد حداقلی اجماع ندارد. 
بخشی خواهان جمهوری‌اند، بخشی 
مشروطه، بخشی فدرالیسم و بخشی 
مخالف هر سه. \emphred{بدون حل 
این معضل، هیچ مدل نظارتی 
کار نخواهد کرد.}
\end{warningbox}

\subsection{حمایت بین‌المللی}

\begin{table}[htbp]
    \centering
    \caption{سطوح حمایت بین‌المللی مورد نیاز}
    \label{tab:international-support}
    \begin{tabularx}{\textwidth}{
        L{2.5cm} X C{2cm}
    }
        \toprule
        \headerrow
        \textbf{سطح} & 
        \textbf{محتوا} & 
        \textbf{اولویت} \\
        \midrule
        
        حداقلی &
        عدم مخالفت فعال — 
        روسیه و چین مانع نشوند 
        (حداقل رأی ممتنع در 
        شورای امنیت) &
        \cellred{\textbf{حیاتی}} \\
        \altrow
        
        مطلوب &
        حمایت فعال اکثریت 
        اعضای شورای امنیت + 
        مجمع عمومی &
        مهم \\
        
        ایده‌آل &
        اجماع جهانی + بسته‌ی 
        جامع حمایتی (مالی، 
        فنی، دیپلماتیک) &
        مطلوب \\
        
        \bottomrule
    \end{tabularx}
\end{table}

\sectiondivider

% ============================================================
\section{تضمین‌های امنیتی}
\label{sec:security-guarantees}
% ============================================================

\begin{keypoint}
تضمین‌های امنیتی \emphgreen{مهم‌ترین و 
حساس‌ترین} حوزه هستند. بدون امنیت، 
هیچ فرایند سیاسی ممکن نیست. و مهم‌ترین 
متغیر امنیتی ایران: \emphgreen{سپاه پاسداران}.
\end{keypoint}

\subsection{مدیریت سپاه پاسداران: 
سه رویکرد ممکن}

\begin{table}[htbp]
    \centering
    \caption{سه رویکرد مدیریت سپاه: 
    مقایسه‌ی مزایا و ریسک‌ها}
    \label{tab:irgc-approaches}
    \tablefontsize
    \begin{tabularx}{\textwidth}{
        L{2cm} X X L{2cm}
    }
        \toprule
        \headerrow
        \textbf{رویکرد} & 
        \textbf{مزایا} & 
        \textbf{ریسک‌ها} &
        \textbf{نمونه} \\
        \midrule
        
        \textbf{انحلال کامل}
        \newline\lr{\tiny Full Dissolution} &
        \cellgreen{حذف تهدید، 
        نمادین قوی، پاسخ به 
        خواست مردم} &
        \cellred{۵۰۰K+ نیروی مسلح 
        بیکار و خشمگین، خلأ 
        امنیتی، شورش مسلحانه} &
        عراق ۲۰۰۳ \newline \emphred{فاجعه} \\
        \altrow
        
        \textbf{حفظ با اصلاح}
        \newline\lr{\tiny Reform \& Retain} &
        \cellgreen{ثبات، حفظ ظرفیت 
        دفاعی، کاهش مقاومت} &
        \cellred{خطر کودتا (مصر)، 
        حفظ نفوذ اقتصادی-سیاسی، 
        عدم اعتماد مردم} &
        مصر ۲۰۱۱ \newline \emphred{کودتا} \\
        
        \textbf{بازسازی تدریجی}
        \newline\lr{\tiny Gradual Restructuring} &
        \cellgreen{تعادل بین امنیت و 
        اصلاح، ادغام تدریجی 
        در ارتش حرفه‌ای، 
        جداسازی اقتصاد} &
        \cellorange{زمان‌بر (۱۰-۱۵ سال)، 
        نیاز به نظارت مستمر، 
        مقاومت فرماندهان} &
        اندونزی ۱۹۹۸ \newline \emphgreen{نسبتاً موفق} \\
        
        \bottomrule
    \end{tabularx}
\end{table}

\begin{recommendation}
\textbf{رویکرد پیشنهادی: بازسازی تدریجی 
(مدل اندونزی)}

\begin{enumerate}[itemsep=4pt]
    \item \textbf{فاز ۱ (ماه ۱-۶):} 
    برکناری فرماندهان ارشد دخیل 
    در جنایات + تعلیق فعالیت‌های 
    سیاسی و اقتصادی سپاه + 
    نظارت بین‌المللی بر تأسیسات 
    حساس
    
    \item \textbf{فاز ۲ (ماه ۶-۲۴):} 
    جداسازی بازوی اقتصادی سپاه 
    (واگذاری شرکت‌ها) + ادغام 
    نیروهای پایین‌رتبه در ارتش 
    ملی + آغاز فرایند حسابرسی 
    (\lr{vetting})
    
    \item \textbf{فاز ۳ (سال ۲-۵):} 
    ادغام کامل در ساختار دفاعی 
    حرفه‌ای + حذف سازمان 
    اطلاعات مستقل + نظارت 
    غیرنظامی بر ارتش
    
    \item \textbf{فاز ۴ (سال ۵-۱۵):} 
    حرفه‌ای‌سازی کامل ارتش + 
    آموزش نسل جدید نظامیان 
    در ارزش‌های دموکراتیک
\end{enumerate}
\end{recommendation}

\begin{lessonlearned}
\textbf{از تجربه‌ی اندونزی (اصلاح \lr{TNI}، ۱۹۹۸-۲۰۱۴):}

پس از سقوط سوهارتو (۱۹۹۸)، ارتش 
اندونزی (\lr{TNI}) — مشابه سپاه ایران — 
هم نظامی، هم اقتصادی و هم سیاسی بود. 
اصلاحات:
\begin{enumerate}[itemsep=2pt, font=\small]
    \item حذف نمایندگی نظامی از مجلس (فوری)
    \item جداسازی پلیس از ارتش (سال اول)
    \item واگذاری کسب‌وکارهای نظامی (تدریجی — ۱۰ سال)
    \item حذف ساختار سرزمینی نظامی (تدریجی)
    \item آموزش حقوق بشر به نظامیان (مستمر)
\end{enumerate}

\textbf{نتیجه:} اندونزی امروز بزرگ‌ترین 
دموکراسی مسلمان جهان است و ارتش آن 
تحت نظارت غیرنظامی قرار دارد.

\vspace{4pt}
\emphblue{درس: بازسازی تدریجی ممکن 
است اما نیاز به صبر، منابع و 
نظارت مستمر بین‌المللی دارد.}
\end{lessonlearned}

\subsection{امنیت مرزی و منطقه‌ای}

\begin{table}[htbp]
    \centering
    \caption{اولویت‌های امنیت مرزی 
    در دوره‌ی گذار}
    \label{tab:border-security}
    \begin{tabularx}{\textwidth}{
        L{2.5cm} X C{1.5cm}
    }
        \toprule
        \headerrow
        \textbf{مرز/منطقه} & 
        \textbf{تهدید اصلی} & 
        \textbf{اولویت} \\
        \midrule
        
        غرب (عراق) &
        نفوذ شبه‌نظامیان حشد الشعبی، 
        قاچاق سلاح &
        \cellred{\textbf{بسیار بالا}} \\
        \altrow
        
        شرق (افغانستان) &
        نفوذ طالبان/داعش، 
        مواد مخدر، مهاجرت &
        \cellred{\textbf{بسیار بالا}} \\
        
        جنوب‌شرق (پاکستان) &
        گروه‌های مسلح بلوچ، 
        قاچاق &
        \cellorange{بالا} \\
        \altrow
        
        شمال‌غرب (ترکیه) &
        مسئله‌ی کُرد، 
        تنش مرزی &
        \cellorange{بالا} \\
        
        شمال (آذربایجان) &
        تنش‌های قومی، 
        نفوذ پان‌ترکیسم &
        \cellorange{متوسط} \\
        \altrow
        
        جنوب (خلیج فارس) &
        تنگه‌ی هرمز، 
        جزایر مورد اختلاف &
        \cellred{\textbf{بسیار بالا}} \\
        
        تأسیسات هسته‌ای &
        سرقت مواد، خرابکاری، 
        دسترسی غیرمجاز &
        \cellred{\textbf{حیاتی}} \\
        
        \bottomrule
    \end{tabularx}
\end{table}

\begin{warningbox}
\textbf{هشدار ویژه: تأسیسات هسته‌ای}

ایران حداقل ۱۰ تأسیسات هسته‌ای شناخته‌شده 
دارد (نطنز، فردو، اصفهان، اراک، بوشهر...). 
در هرگونه سناریوی فروپاشی یا بی‌ثباتی:
\begin{itemize}[itemsep=2pt]
    \item \emphred{حفاظت از مواد شکافت‌پذیر} 
    اولویت مطلق است
    \item \lr{IAEA} باید \emphred{فوراً} 
    دسترسی کامل داشته باشد
    \item نیروی حفاظتی مشترک 
    (بین‌المللی + ایرانی) تشکیل شود
    \item هرگونه تلاش برای انتقال 
    مواد هسته‌ای باید جرم‌انگاری شود
\end{itemize}
\end{warningbox}

\sectiondivider

% ============================================================
\section{تضمین‌های حقوقی}
\label{sec:legal-guarantees}
% ============================================================

\begin{table}[htbp]
    \centering
    \caption{تضمین‌های حقوقی مورد نیاز}
    \label{tab:legal-guarantees}
    \tablefontsize
    \begin{tabularx}{\textwidth}{
        L{3cm} X C{1.5cm} C{1.5cm}
    }
        \toprule
        \headerrow
        \textbf{تضمین حقوقی} & 
        \textbf{محتوا} & 
        \textbf{اولویت} &
        \textbf{زمان} \\
        \midrule
        
        \bilingual{توافق‌نامه وضعیت مأموریت}{SOMA} &
        وضعیت حقوقی، مصونیت‌ها و 
        امتیازات ناظران بین‌المللی &
        \cellred{حیاتی} &
        هفته‌ی ۱ \\
        \altrow
        
        چارچوب حقوقی موقت &
        قانون اساسی موقت یا اعلامیه‌ی 
        اصول دوره‌ی گذار &
        \cellred{حیاتی} &
        ماه ۱ \\
        
        قانون انتخابات موقت &
        قواعد ثبت‌نام، نامزدی، رأی‌گیری 
        و شمارش برای انتخابات اولیه &
        \cellred{حیاتی} &
        ماه ۲-۳ \\
        \altrow
        
        قانون احزاب موقت &
        حق تأسیس و فعالیت احزاب، 
        شفافیت مالی &
        مهم &
        ماه ۳-۶ \\
        
        قانون رسانه‌ی موقت &
        آزادی مطبوعات و رسانه، 
        تنظیم‌گری مستقل &
        مهم &
        ماه ۱-۳ \\
        \altrow
        
        مکانیزم عدالت انتقالی &
        قانون تأسیس کمیسیون حقیقت، 
        حدود عفو، حقوق قربانیان &
        مهم &
        ماه ۶-۱۲ \\
        
        الحاق به معاهدات &
        \lr{ICCPR, ICESCR, CAT, CEDAW, 
        CRC, Rome Statute} &
        مطلوب &
        سال ۱-۲ \\
        
        \bottomrule
    \end{tabularx}
\end{table}

\begin{casestudy}{قانون اساسی موقت 
آفریقای جنوبی (۱۹۹۳)}
آفریقای جنوبی قبل از انتخابات ۱۹۹۴ 
یک «قانون اساسی موقت» تدوین کرد 
که ۳۴ اصل غیرقابل مذاکره 
(\lr{Constitutional Principles}) 
تعیین می‌کرد. هر قانون اساسی 
نهایی باید با این اصول سازگار 
می‌بود. دادگاه قانون اساسی 
(\lr{Constitutional Court}) 
ناظر رعایت این اصول بود.

\vspace{4pt}
\emphblue{درس برای ایران: تدوین 
«اعلامیه‌ی اصول گذار» قبل از 
قانون اساسی نهایی — شامل اصول 
غیرقابل مذاکره مانند حقوق بشر، 
تفکیک دین و دولت، برابری جنسیتی 
و حقوق اقلیت‌ها.}
\end{casestudy}

\sectiondivider

% ============================================================
\section{تضمین‌های اقتصادی}
\label{sec:economic-guarantees}
% ============================================================

\begin{keypoint}
\emphgreen{اقتصاد ناموفق = دموکراسی ناموفق.}
اگر مردم در ماه‌های نخست گذار شاهد 
بهبود ملموس اقتصادی نباشند، سرخوردگی 
سیاسی و نوستالژی نظام قدیم رشد می‌کند. 
تجربه‌ی روسیه دهه‌ی ۱۹۹۰ هشدار روشنی است.
\end{keypoint}

\begin{table}[htbp]
    \centering
    \caption{بسته‌ی تضمین‌های اقتصادی}
    \label{tab:economic-guarantees}
    \tablefontsize
    \begin{tabularx}{\textwidth}{
        L{2.5cm} X C{1.5cm} L{2.5cm}
    }
        \toprule
        \headerrow
        \textbf{تضمین} & 
        \textbf{محتوا} & 
        \textbf{اولویت} &
        \textbf{مسئول} \\
        \midrule
        
        رفع تحریم‌ها &
        مرحله‌ای و مشروط به 
        پیشرفت گذار — نه 
        یکباره و نه معطل &
        \cellred{حیاتی} &
        آمریکا + EU \\
        \altrow
        
        بسته‌ی حمایت فوری &
        کمک غذایی و دارویی، 
        تثبیت نرخ ارز، 
        وام اضطراری \lr{IMF} &
        \cellred{حیاتی} &
        \lr{IMF} + دولت‌ها \\
        
        آزادسازی دارایی‌ها &
        دسترسی به دارایی‌های 
        بلوکه‌شده‌ی ایران 
        (تخمین: \$۱۰۰-۱۵۰B) &
        \cellred{حیاتی} &
        آمریکا + EU \\
        \altrow
        
        صندوق امانی نفت &
        مدیریت شفاف درآمد نفت 
        در دوره‌ی گذار — 
        جلوگیری از غارت &
        مهم &
        \lr{UN} + دولت موقت \\
        
        برنامه ضد فساد &
        ردیابی و بازپس‌گیری 
        دارایی‌های غارت‌شده 
        توسط سران نظام قدیم &
        مهم &
        \lr{TI} + نهاد ملی \\
        \altrow
        
        بازسازی زیرساخت &
        آب، برق، حمل‌ونقل، 
        ارتباطات &
        مهم (فاز ۲) &
        \lr{World Bank} + بخش خصوصی \\
        
        حمایت از اشتغال &
        برنامه‌ی اضطراری 
        اشتغال‌زایی — 
        به‌ویژه جوانان &
        مهم &
        \lr{ILO} + دولت موقت \\
        
        \bottomrule
    \end{tabularx}
\end{table}

\begin{lessonlearned}
\textbf{از تجربه‌ی روسیه (شوک‌درمانی دهه‌ی ۱۹۹۰):}

روسیه پس از فروپاشی شوروی 
«شوک‌درمانی» اقتصادی را پیش گرفت: 
آزادسازی ناگهانی قیمت‌ها + خصوصی‌سازی 
سریع. نتیجه:
\begin{itemize}[itemsep=2pt, font=\small]
    \item تورم ۲,۵۰۰٪ در ۱۹۹۲
    \item ظهور الیگارشی (دزدیدن دارایی‌های ملی)
    \item فقر ۴۰٪ جمعیت
    \item سرخوردگی از دموکراسی → ظهور پوتین
\end{itemize}

\emphblue{درس: اصلاحات اقتصادی باید 
تدریجی، عادلانه و با شبکه‌ی 
حمایت اجتماعی باشد. 
«شوک‌درمانی» برای ایران فاجعه‌بار 
خواهد بود.}
\end{lessonlearned}

\sectiondivider

% ============================================================
\section{تضمین‌های اجتماعی-فرهنگی}
\label{sec:social-guarantees}
% ============================================================

\subsection{مشارکت زنان: نه امتیاز بلکه حق}

\begin{statsbox}
\begin{center}
    {\statisticfont حداقل ۳۰\%}\\[4pt]
    {\small مشارکت زنان در تمام نهادهای 
    دوره‌ی گذار — از مجلس مؤسسان 
    تا کمیسیون انتخابات}\\[10pt]
    {\small هدف بلندمدت:}\\[2pt]
    {\statisticfont ۵۰\%}\\[4pt]
    {\small برابری کامل در نمایندگی}
\end{center}
\end{statsbox}

\begin{table}[htbp]
    \centering
    \caption{مکانیزم‌های تضمین مشارکت زنان}
    \label{tab:women-participation}
    \begin{tabularx}{\textwidth}{
        L{3cm} X L{2.5cm}
    }
        \toprule
        \headerrow
        \textbf{مکانیزم} & 
        \textbf{توضیح} & 
        \textbf{نمونه‌ی موفق} \\
        \midrule
        
        سهمیه‌ی قانونی &
        حداقل ۳۰٪ کرسی‌ها 
        برای زنان در مجلس مؤسسان 
        و نهادهای گذار &
        رواندا (۶۱٪!), تونس \\
        \altrow
        
        \lr{UN SCR 1325} &
        اجرای قطعنامه‌ی زنان، صلح 
        و امنیت — مشارکت زنان 
        در مذاکرات صلح/گذار &
        کلمبیا \\
        
        فهرست زیپ &
        فهرست‌های انتخاباتی 
        متناوب زن-مرد &
        بولیوی, اکوادور \\
        \altrow
        
        ممیزی جنسیتی &
        ارزیابی تأثیر جنسیتی 
        هر قانون و سیاست &
        سوئد, کانادا \\
        
        \bottomrule
    \end{tabularx}
\end{table}

\subsection{حقوق اقوام و مدل‌های 
خودمختاری}

\begin{table}[htbp]
    \centering
    \caption{مدل‌های مدیریت تنوع قومی}
    \label{tab:ethnic-models}
    \tablefontsize
    \begin{tabularx}{\textwidth}{
        L{2.5cm} X X C{1.5cm}
    }
        \toprule
        \headerrow
        \textbf{مدل} & 
        \textbf{مزایا} & 
        \textbf{ریسک‌ها} &
        \textbf{نمونه} \\
        \midrule
        
        دولت متمرکز با حقوق فرهنگی &
        \cellgreen{وحدت ملی, سادگی} &
        \cellred{سرکوب تنوع, نارضایتی} &
        فرانسه \\
        \altrow
        
        خودمختاری فرهنگی &
        \cellgreen{حفظ هویت, انعطاف} &
        \cellorange{ناکافی برای اقوام 
        بزرگ با تمرکز جغرافیایی} &
        استونی \\
        
        فدرالیسم جغرافیایی &
        \cellgreen{خودگردانی, نمایندگی} &
        \cellorange{خطر تجزیه, رقابت 
        منابع} &
        آلمان, هند \\
        \altrow
        
        فدرالیسم قومی &
        \cellgreen{نمایندگی مستقیم اقوام} &
        \cellred{قومیت‌سازی سیاست, 
        تجزیه (اتیوپی)} &
        اتیوپی (مشکل‌دار) \\
        
        ترکیبی (پیشنهادی) &
        \cellgreen{خودمختاری استانی + 
        حقوق فرهنگی + تضمین 
        اقلیت‌ها در مرکز} &
        \cellorange{پیچیدگی طراحی} &
        اسپانیا \\
        
        \bottomrule
    \end{tabularx}
\end{table}

\begin{recommendation}
\textbf{مدل پیشنهادی برای ایران: 
ترکیبی-اسپانیایی}

\begin{enumerate}[itemsep=3pt]
    \item \textbf{تقسیمات استانی} 
    (نه قومی) با اختیارات گسترده‌ی 
    خودگردانی (آموزش, فرهنگ, 
    زبان محلی, عمران)
    \item \textbf{زبان‌های رسمی منطقه‌ای} 
    در کنار فارسی به‌عنوان زبان مشترک
    \item \textbf{تضمین نمایندگی} 
    اقوام در نهادهای مرکزی 
    (مجلس دوم/سنا)
    \item \textbf{خطوط قرمز:} تمامیت 
    ارضی غیرقابل مذاکره، 
    خودمختاری ≠ استقلال
\end{enumerate}
\end{recommendation}

\subsection{عدالت انتقالی و آشتی ملی}

\begin{table}[htbp]
    \centering
    \caption{ابزارهای عدالت انتقالی 
    و اولویت‌بندی}
    \label{tab:transitional-justice}
    \begin{tabularx}{\textwidth}{
        L{2.5cm} X C{1.5cm} C{1.5cm}
    }
        \toprule
        \headerrow
        \textbf{ابزار} & 
        \textbf{توضیح} & 
        \textbf{اولویت} &
        \textbf{زمان} \\
        \midrule
        
        کمیسیون حقیقت &
        ثبت شهادات قربانیان, 
        مستندسازی جامع نقض‌ها &
        \cellred{حیاتی} &
        ماه ۶-۱۲ \\
        \altrow
        
        محاکمه‌ی مسئولان &
        محاکمه‌ی عاملان جنایات 
        بزرگ (اعدام‌ها, شکنجه, 
        کشتار ۶۷) &
        \cellred{حیاتی} &
        سال ۱-۳ \\
        
        غرامت به قربانیان &
        جبران مادی و معنوی 
        برای قربانیان و 
        خانواده‌هایشان &
        مهم &
        سال ۱-۵ \\
        \altrow
        
        حسابرسی نهادی &
        بررسی پیشینه‌ی کارکنان 
        دولتی (\lr{vetting}) &
        مهم &
        ماه ۶-۲۴ \\
        
        بزرگداشت و یادبود &
        ساختن حافظه‌ی جمعی, 
        موزه‌ها, روز ملی &
        مطلوب &
        سال ۲-۱۰ \\
        \altrow
        
        آشتی ملی &
        گفت‌وگوی ملی, 
        عذرخواهی رسمی, 
        نمادهای مشترک &
        مهم &
        مستمر \\
        
        \bottomrule
    \end{tabularx}
\end{table}

\begin{warningbox}
\textbf{تعادل حیاتی: عدالت vs ثبات}

بزرگ‌ترین معضل عدالت انتقالی تعادل بین 
خواست عدالت و نیاز به ثبات است:
\begin{itemize}[itemsep=2pt]
    \item \emphred{عدالت بیش از حد → 
    بی‌ثباتی} (عراق: دی‌بعثی‌سازی → 
    داعش)
    \item \emphred{فراموشی → 
    نارضایتی مزمن} (اسپانیا: 
    «پیمان فراموشی» نیم‌قرن بعد 
    هنوز محل بحث)
    \item \emphgreen{تعادل = 
    محاکمه‌ی مسئولان اصلی + 
    آشتی برای بقیه} 
    (آفریقای جنوبی)
\end{itemize}
\end{warningbox}

\sectiondivider

% ============================================================
\section{تضمین‌های نهادی}
\label{sec:institutional-guarantees}
% ============================================================

\begin{table}[htbp]
    \centering
    \caption{نهادهای کلیدی دوره‌ی گذار}
    \label{tab:transitional-institutions}
    \tablefontsize
    \begin{tabularx}{\textwidth}{
        L{3cm} X C{1.5cm}
    }
        \toprule
        \headerrow
        \textbf{نهاد} & 
        \textbf{وظیفه و مشخصات} & 
        \textbf{زمان تأسیس} \\
        \midrule
        
        کمیسیون مستقل انتخابات &
        مدیریت تمام انتخابات و رفراندوم‌ها, 
        استقلال کامل از دولت, 
        ترکیب مشورتی بین‌المللی &
        ماه ۱-۳ \\
        \altrow
        
        دادگاه قانون اساسی &
        نظارت بر انطباق قوانین 
        با اصول دموکراتیک, 
        حکمیت اختلافات &
        ماه ۶-۱۲ \\
        
        نهاد تنظیم‌گر رسانه &
        تضمین آزادی و مسئولیت 
        رسانه, صدور مجوز, 
        استقلال از دولت &
        ماه ۳-۶ \\
        \altrow
        
        کمیسیون ملی حقوق بشر &
        نظارت بر رعایت حقوق بشر, 
        دریافت شکایات, 
        گزارش‌دهی عمومی &
        ماه ۳-۶ \\
        
        نهاد ضد فساد &
        پیشگیری و مبارزه با فساد 
        در دوره‌ی گذار &
        ماه ۳-۶ \\
        \altrow
        
        شورای مشورتی ملی &
        نمایندگی همه‌ی گروه‌ها 
        تا تشکیل مجلس مؤسسان &
        هفته‌ی ۱-۲ \\
        
        \bottomrule
    \end{tabularx}
\end{table}

\sectiondivider

% ============================================================
\section{ماتریس جامع تضمین‌ها}
\label{sec:guarantee-matrix}
% ============================================================

\begin{landscape}
\begin{table}[htbp]
    \centering
    \caption{ماتریس جامع تضمین‌ها 
    و اولویت‌بندی}
    \label{tab:guarantee-matrix}
    \bigtablefontsize
    \setlength{\tabcolsep}{3pt}
    \begin{tabularx}{\linewidth}{
        L{2cm} X X X L{2.5cm}
    }
        \toprule
        \headerrow
        \textbf{حوزه} & 
        \textbf{حیاتی (بدون آن شروع نکنید)} & 
        \textbf{مهم (لازم برای موفقیت)} & 
        \textbf{مطلوب (تقویت‌کننده)} &
        \textbf{مسئول اصلی} \\
        \midrule
        
        سیاسی &
        اجماع حداقلی بر قواعد بازی &
        حمایت فعال بین‌المللی, 
        مکانیزم حل اختلاف &
        مشارکت دیاسپورا &
        نیروهای سیاسی ایرانی \\
        \altrow
        
        امنیتی &
        کنترل سپاه, حفاظت 
        از تأسیسات هسته‌ای &
        امنیت مرزی, \lr{DDR}, 
        مدیریت نیروهای نیابتی &
        اصلاح کامل بخش 
        امنیتی &
        شورای امنیت + 
        بازیگران داخلی \\
        
        حقوقی &
        \lr{SOMA}, چارچوب 
        حقوقی موقت &
        قانون انتخابات/احزاب/رسانه, 
        مکانیزم عدالت انتقالی &
        الحاق به معاهدات 
        بین‌المللی &
        دولت موقت + UN \\
        \altrow
        
        اقتصادی &
        رفع تحریم‌ها, 
        بسته حمایت فوری &
        صندوق امانی نفت, 
        ضد فساد &
        بازسازی زیرساخت, 
        جذب سرمایه &
        قدرت‌های بزرگ + 
        \lr{IFIs} \\
        
        اجتماعی &
        مشارکت زنان (۳۰٪+), 
        نمایندگی اقوام &
        گفت‌وگوی ملی, 
        عدالت انتقالی &
        آشتی کامل, 
        سواد دموکراتیک &
        جامعه مدنی + UN \\
        \altrow
        
        نهادی &
        کمیسیون انتخابات, 
        شورای مشورتی &
        استقلال قضا, 
        آزادی رسانه &
        کمیسیون حقوق بشر, 
        نهاد ضد فساد &
        دولت موقت + 
        مشاوران بین‌المللی \\
        
        \bottomrule
    \end{tabularx}
\end{table}
\end{landscape}

\sectiondivider

% ============================================================
\section{جمع‌بندی فصل}
\label{sec:ch6-summary}
% ============================================================

\begin{chaptersummary}

\textbf{آنچه در این فصل آموختیم:}

\begin{enumerate}[
    label=\textcolor{DarkGray}{\bfseries\arabic*.},
    itemsep=4pt
]
    \item \textbf{شش حوزه‌ی تضمین} 
    شناسایی شد: سیاسی, امنیتی, 
    حقوقی, اقتصادی, اجتماعی و نهادی.
    
    \item \textbf{اجماع حداقلی سیاسی} 
    اولین و مهم‌ترین پیش‌شرط است — 
    و بزرگ‌ترین ضعف فعلی اپوزیسیون ایران.
    
    \item \textbf{مدیریت سپاه} با مدل 
    «بازسازی تدریجی» (الگوی اندونزی) 
    پیشنهاد شد — نه انحلال کامل 
    (خطای عراق) و نه حفظ بدون اصلاح 
    (خطای مصر).
    
    \item \textbf{تأسیسات هسته‌ای} 
    اولویت مطلق امنیتی هستند و 
    \lr{IAEA} باید فوراً دسترسی 
    کامل داشته باشد.
    
    \item \textbf{رفع تحریم‌ها و 
    بسته‌ی حمایت اقتصادی} شرط لازم 
    موفقیت است — «شوک‌درمانی» روسی 
    تکرار نشود.
    
    \item \textbf{حداقل ۳۰٪ مشارکت زنان} 
    در همه‌ی نهادهای گذار باید 
    تضمین شود.
    
    \item \textbf{مدل ترکیبی مدیریت 
    تنوع قومی} (الگوی اسپانیایی) 
    پیشنهاد شد: خودمختاری استانی + 
    حقوق فرهنگی + تضمین نمایندگی.
    
    \item \textbf{عدالت انتقالی} باید 
    آشتی‌محور باشد: محاکمه‌ی مسئولان 
    اصلی + آشتی برای بقیه.
    
    \item \textbf{نهادهای کلیدی} باید 
    در هفته‌ها و ماه‌های نخست 
    تأسیس شوند — از شورای مشورتی 
    تا کمیسیون انتخابات.
\end{enumerate}

\vspace{6pt}
\begin{center}
    \textcolor{MainRed}{
        \faArrowLeft\hspace{8pt}
        \textbf{فصل بعد: آسیب‌شناسی، 
        ریسک‌ها و چالش‌های پیش‌رو}
        \hspace{8pt}\faArrowLeft
    }
\end{center}

\end{chaptersummary}

\chapterend