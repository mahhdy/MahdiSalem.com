%══════════════════════════════════════════════════════════════
% پیوست پ: مطالعه موردی شیلی
% فایل: appendices/app-c-chile.tex
% حجم هدف: ۸-۱۰ صفحه
%══════════════════════════════════════════════════════════════

\chapter{مطالعهٔ موردی: شیلی (۱۹۸۸-۱۹۹۸)}
\label{app:chile}

\begin{executivesummary}
شیلی یکی از موفق‌ترین نمونه‌های گذار دموکراتیک در جهان است که از \textbf{دیکتاتوری نظامی} ژنرال \person{آگوستو پینوشه}{\lr{Augusto Pinochet}} (۱۹۷۳-۱۹۹۰) به دموکراسی باثبات انتقال یافت. ویژگی‌های منحصربه‌فرد این گذار عبارت‌اند از: ۱) استفاده از \textbf{رفراندوم} به‌عنوان ابزار گذار (پلبیسیت ۱۹۸۸)، ۲) \textbf{ائتلاف فراگیر اپوزیسیون} (\lr{Concertación})، ۳) \textbf{عدالت انتقالی تدریجی} (از کمیسیون رتیگ تا بازداشت لندن)، ۴) \textbf{مدل اقتصادی موفق} (رشد پایدار با عدالت اجتماعی)، و ۵) \textbf{مدیریت ریسک نظامی} (پینوشه ماند اما قدرتش کاهش یافت). این تجربه برای ایران در ابعاد رفراندوم، ائتلاف‌سازی، و عدالت تدریجی بسیار آموزنده است.
\end{executivesummary}

%═══════════════════════════════════════════════════════════
\section{زمینه و بافت تاریخی}
\label{app:chile:context}
%═══════════════════════════════════════════════════════════

\subsection{کودتای ۱۹۷۳ و دیکتاتوری پینوشه}

در ۱۱ سپتامبر ۱۹۷۳، ارتش شیلی به فرماندهی ژنرال \person{آگوستو پینوشه}{\lr{Augusto Pinochet}} با حمایت \lr{CIA} علیه دولت منتخب سوسیالیست \person{سالوادور آلنده}{\lr{Salvador Allende}} کودتا کرد. آلنده در کاخ ریاست‌جمهوری (\lr{La Moneda}) جان باخت و ۱۷ سال دیکتاتوری نظامی آغاز شد.

\begin{table}[htbp]
\centering
\caption{مشخصات شیلی در آستانهٔ گذار (۱۹۸۸)}
\label{tab:app-chile-profile}
\begin{tabularx}{\textwidth}{>{\raggedleft\arraybackslash}p{4.5cm} >{\raggedleft\arraybackslash}X}
\toprule
\headerrow \textbf{شاخص} & \textbf{مقدار} \\
\midrule
جمعیت & ۱۳ میلیون نفر \\
\altrow مساحت & ۷۵۶,۰۰۰ \lr{km²} \\
تنوع قومی & پایین (۹۵٪ مستیزو + اسپانیایی‌تبار) \\
\altrow \lr{GDP per capita} & $\sim$\$۲,۵۰۰ \\
طول عمر دیکتاتوری & ۱۷ سال (۱۹۷۳-۱۹۹۰) \\
\altrow اندازهٔ نیروهای مسلح & $\sim$۱۰۰,۰۰۰ \\
تعداد کشته‌شدگان/ناپدیدشدگان & ۳,۲۰۰+ (رتیگ) — تا ۴۰,۰۰۰+ (والش) \\
\altrow تعداد شکنجه‌شدگان & ۲۸,۰۰۰+ (والش) \\
مدل اقتصادی & نئولیبرال (\lr{Chicago Boys}) \\
\altrow تحریم‌های بین‌المللی & محدود (آمریکا حامی بود) \\
\bottomrule
\end{tabularx}
\end{table}

\subsection{ویژگی‌های دیکتاتوری پینوشه}

\begin{enumerate}[nosep]
\item \textbf{شخصی‌سازی قدرت:} پینوشه هم رئیس‌جمهور بود، هم فرمانده کل ارتش — تمرکز قدرت مشابه ساختار ولایت فقیه
\item \textbf{سرکوب سیستماتیک:} \org{پلیس مخفی}{\lr{DINA}} (بعداً \lr{CNI}) — عملیات کندور (\lr{Operation Condor}) در سطح منطقه‌ای
\item \textbf{اصلاحات اقتصادی نئولیبرال:} «پسران شیکاگو» (\lr{Chicago Boys}) اقتصاد را آزادسازی کردند — رشد اقتصادی با نابرابری
\item \textbf{قانون اساسی ۱۹۸۰:} پینوشه قانون اساسی جدیدی تصویب کرد که خودش را برای ۸ سال دیگر تثبیت می‌کرد و سپس رفراندوم (\lr{Plebiscite}) تعیین‌کننده بود
\item \textbf{نقش آمریکا:} دولت ریگان در ابتدا حامی پینوشه بود اما از ۱۹۸۶ فشار برای دموکراتیزه‌شدن آغاز شد
\end{enumerate}

\begin{casestudy}
\textbf{مقایسهٔ ساختاری پینوشه و جمهوری اسلامی:} هر دو نظام ترکیبی از \textbf{شخصی‌سازی قدرت + نیروهای امنیتی قدرتمند + ایدئولوژی (ضدکمونیسم/اسلام سیاسی) + سرکوب سازمان‌یافته} هستند. تفاوت‌ها: ۱) پینوشه فاقد مشروعیت مذهبی بود؛ ۲) اقتصاد شیلی رو به رشد بود (نه بحران)؛ ۳) جمعیت شیلی بسیار کمتر (۱۳M vs ۸۵M)؛ ۴) تنوع قومی پایین بود. نکتهٔ مهم: پینوشه \textbf{خودش} رفراندوم را در قانون اساسی‌اش گنجانده بود — اشتباه راهبردی که اپوزیسیون از آن بهره برد.
\end{casestudy}

\sectiondivider

%═══════════════════════════════════════════════════════════
\section{رفراندوم ۱۹۸۸: نقطهٔ عطف تاریخی}
\label{app:chile:plebiscite}
%═══════════════════════════════════════════════════════════

\subsection{زمینه‌سازی: از تفرقه تا ائتلاف}

مهم‌ترین درس شیلی برای اپوزیسیون ایران: چگونه جریان‌های متفرق و متخاصم توانستند در یک \textbf{ائتلاف فراگیر} متحد شوند.

\begin{table}[htbp]
\centering
\caption{ترکیب ائتلاف «نه» (\lr{Concertación}) و معادل ایرانی}
\label{tab:app-chile-coalition}
\begin{tabularx}{\textwidth}{>{\raggedleft\arraybackslash}p{3.2cm} >{\raggedleft\arraybackslash}X >{\raggedleft\arraybackslash}p{3.5cm}}
\toprule
\headerrow \textbf{جریان شیلیایی} & \textbf{ویژگی} & \textbf{معادل ایرانی احتمالی} \\
\midrule
\lr{PDC} (دموکرات‌مسیحی) & میانه‌رو، بزرگ‌ترین حزب & ملی-مذهبیون \\
\altrow \lr{PS} (سوسیالیست) & چپ میانه (حزب آلنده) & چپ دموکراتیک \\
\lr{PPD} (دموکراسی برای مردم) & لیبرال-چپ & جمهوری‌خواهان \\
\altrow \lr{PR} (رادیکال) & لائیک، لیبرال & سکولارها \\
کمونیست‌ها & چپ رادیکال (خارج ائتلاف رسمی) & چپ رادیکال (حاشیه‌ای) \\
\altrow جنبش زنان & حقوق زنان + مادران ناپدیدشدگان & جنبش زن-زندگی-آزادی \\
اتحادیهٔ کارگری (\lr{CUT}) & کارگران & تشکل‌های کارگری \\
\altrow روشنفکران و هنرمندان & فرهنگ‌سازی + کمپین خلاقانه & روشنفکران و هنرمندان ایرانی \\
\bottomrule
\end{tabularx}
\end{table}

\subsection{کمپین «نه» (\lr{NO}): نبوغ خلاقانه}

کمپین رفراندوم «نه» به پینوشه یکی از \textbf{خلاقانه‌ترین کمپین‌های سیاسی تاریخ} بود:

\begin{itemize}[nosep]
\item \textbf{۱۵ دقیقه تلویزیون شبانه:} طبق قانون، هر طرف ۱۵ دقیقه وقت تبلیغاتی تلویزیونی داشت. تیم «نه» به‌جای نشان دادن سرکوب و خشونت، \textbf{شادی و امید} را برگزید
\item \textbf{شعار «شادی در راه است» (\lr{La Alegría Ya Viene}):} پیام مثبت + رنگین‌کمان به‌جای پیام خشم و انتقام
\item \textbf{ثبت‌نام رأی‌دهندگان:} کمپین گستردهٔ ثبت‌نام ۷.۵ میلیون نفر
\item \textbf{ناظران داخلی:} سازمان \lr{Participa} شبکهٔ ۳۰,۰۰۰ ناظر داوطلب سازماندهی کرد
\item \textbf{شمارش موازی:} اپوزیسیون سیستم شمارش موازی (\lr{Parallel Vote Tabulation}) راه‌اندازی کرد تا تقلب غیرممکن شود
\end{itemize}

\begin{keypoint}
\textbf{نتیجهٔ رفراندوم ۵ اکتبر ۱۹۸۸:} «نه» ۵۵.۹۹٪ — «بله» ۴۴.۰۱٪. مشارکت: ۹۷٪ ثبت‌نام‌شدگان. پینوشه ابتدا نتیجه را نپذیرفت و خواست حکومت نظامی اعلام کند، اما \textbf{فرماندهان ارتش} (به‌ویژه ژنرال \person{ماتیاس}{Matthei}) از اجرای دستور امتناع کردند. این نشان داد که حتی در درون ارتش، \textbf{شکاف نخبگان} وجود داشت.
\end{keypoint}

\subsection{آمار و تحلیل رفراندوم}

\begin{table}[htbp]
\centering
\caption{آمار رفراندوم شیلی ۱۹۸۸ و الگوی ایرانی}
\label{tab:app-chile-referendum}
\begin{tabularx}{\textwidth}{>{\raggedleft\arraybackslash}p{4.5cm} >{\centering\arraybackslash}p{3cm} >{\raggedleft\arraybackslash}X}
\toprule
\headerrow \textbf{شاخص} & \textbf{شیلی ۱۹۸۸} & \textbf{کاربرد ایرانی} \\
\midrule
واجدین شرایط رأی & ۷,۴۳۵,۹۱۳ & $\sim$۶۰ میلیون \\
\altrow مشارکت & ۹۷.۵٪ & هدف: ۷۰٪+ \\
آرای «نه» & ۵۵.۹۹٪ (۳,۹۶۷,۵۷۹) & -- \\
\altrow آرای «بله» & ۴۴.۰۱٪ (۳,۱۱۹,۱۱۰) & -- \\
ناظران بین‌المللی & $\sim$۱,۰۰۰ & نیاز: ۱۰,۰۰۰+ \\
\altrow ناظران داخلی & $\sim$۳۰,۰۰۰ & نیاز: ۲۰۰,۰۰۰+ \\
شمارش موازی & \cmark (حیاتی) & \cmark (ضروری) \\
\altrow خشونت انتخاباتی & \risklow حداقل & باید تضمین شود \\
\bottomrule
\end{tabularx}
\end{table}

\begin{lessonlearned}
\textbf{ابزار رفراندوم برای ایران:} شیلی نشان داد که رفراندوم می‌تواند ابزار قدرتمند گذار باشد، \textbf{اما تنها در شرایطی:} ۱) اپوزیسیون متحد باشد؛ ۲) ناظران داخلی و بین‌المللی کافی وجود داشته باشد؛ ۳) شمارش موازی امکان‌پذیر باشد؛ ۴) رسانهٔ آزاد (حتی محدود) وجود داشته باشد. در سناریوی \lr{D} ایران (تحول از درون)، رفراندوم بر سر قانون اساسی جدید ممکن‌ترین ابزار است (\seeChapter{ch:scenarios}).
\end{lessonlearned}

\sectiondivider

%═══════════════════════════════════════════════════════════
\section{دولت آیلوین و انتقال مسالمت‌آمیز قدرت}
\label{app:chile:aylwin}
%═══════════════════════════════════════════════════════════

\subsection{انتخابات ۱۹۸۹ و محدودیت‌ها}

پس از شکست در رفراندوم، پینوشه انتخابات ریاست‌جمهوری و پارلمانی برگزار کرد (دسامبر ۱۹۸۹). \person{پاتریسیو آیلوین}{\lr{Patricio Aylwin}} از ائتلاف \lr{Concertación} با ۵۵٪ آرا پیروز شد.

اما پینوشه \textbf{قبل از ترک قدرت}، مجموعه‌ای از «قفل‌های نهادی» (\lr{Institutional Locks}) ایجاد کرد:

\begin{table}[htbp]
\centering
\caption{«قفل‌های نهادی» پینوشه و معادل ایرانی}
\label{tab:app-chile-locks}
\begin{tabularx}{\textwidth}{>{\raggedleft\arraybackslash}p{3.5cm} >{\raggedleft\arraybackslash}X >{\raggedleft\arraybackslash}p{3.5cm}}
\toprule
\headerrow \textbf{قفل پینوشه} & \textbf{توضیح} & \textbf{معادل ایرانی احتمالی} \\
\midrule
سناتورهای منصوب & ۹ سناتور منصوب (از جمله خود پینوشه) + مادام‌العمر & شورای نگهبان / مجلس خبرگان \\
\altrow فرمانده‌کل ارتش & پینوشه ۸ سال دیگر فرمانده ماند (تا ۱۹۹۸) & رهبر نظامی-امنیتی باقیمانده \\
عدم قابلیت عزل فرماندهان & رئیس‌جمهور حق عزل فرماندهان نظامی را نداشت & ساختار نظامی مستقل \\
\altrow قانون عفو ۱۹۷۸ & عفو عمومی برای جرایم ۱۹۷۳-۱۹۷۸ & مصونیت‌های قانونی \\
شورای امنیت ملی & ارتش اکثریت داشت (۴ از ۸ عضو) & شورای عالی امنیت ملی \\
\altrow نظام انتخاباتی دوعضوی & (\lr{Binomial}) که اقلیت (راست) را بیش‌نمایندگی می‌کرد & مهندسی انتخابات \\
\bottomrule
\end{tabularx}
\end{table}

\begin{warningbox}
\textbf{درس قفل‌های نهادی:} شیلی نشان داد که حتی پس از پیروزی دموکراتیک، رژیم پیشین می‌تواند از طریق \textbf{قفل‌های نهادی} قدرت خود را حفظ کند. شیلی ۱۵ سال طول کشید تا اکثر این قفل‌ها را بردارد (اصلاحیهٔ قانون اساسی ۲۰۰۵). \emphred{هشدار ایرانی:} در هر سناریوی مذاکره‌ای، باید مراقب ایجاد «قفل‌های نهادی» توسط جمهوری اسلامی بود. مدل ۶ پیشنهادی باید مکانیزم بازبینی و حذف تدریجی این قفل‌ها را پیش‌بینی کند (\seeChapter{ch:risks}).
\end{warningbox}

\subsection{استراتژی آیلوین: «عدالت تا حد ممکن»}

\person{آیلوین}{\lr{Aylwin}} عبارت معروف خود را به‌کار برد: «\textit{\lr{Justicia en la medida de lo posible}}» (عدالت تا حد ممکن). این به معنای \textbf{واقع‌بینی} در مواجهه با محدودیت‌ها بود:

\begin{itemize}[nosep]
\item پینوشه هنوز فرماندهٔ ارتش بود و تهدید کرده بود
\item قانون عفو ۱۹۷۸ مانع محاکمه بود
\item سناتورهای منصوب مانع اصلاح قانون اساسی بودند
\item اما آیلوین \textbf{کمیسیون حقیقت} تشکیل داد — «اگر نمی‌توانیم محاکمه کنیم، حداقل حقیقت را روشن می‌کنیم»
\end{itemize}

\sectiondivider

%═══════════════════════════════════════════════════════════
\section{عدالت انتقالی: مدل تدریجی شیلی}
\label{app:chile:tj}
%═══════════════════════════════════════════════════════════

شیلی مدل منحصربه‌فردی از عدالت انتقالی ارائه کرد: نه عفو کامل (اسپانیا) و نه محاکمهٔ فوری (نورنبرگ)، بلکه \textbf{پیشروی تدریجی} در طول ۳۰ سال.

\subsection{گاه‌شمار عدالت تدریجی}

\begin{table}[htbp]
\centering
\caption{گاه‌شمار عدالت انتقالی تدریجی شیلی (۱۹۹۰-۲۰۲۳)}
\label{tab:app-chile-tj-timeline}
\begin{tabularx}{\textwidth}{>{\centering\arraybackslash}p{2.2cm} >{\raggedleft\arraybackslash}X >{\centering\arraybackslash}p{2cm}}
\toprule
\headerrow \textbf{سال} & \textbf{اقدام} & \textbf{سطح عدالت} \\
\midrule
۱۹۹۱ & \textbf{کمیسیون رتیگ}: شناسایی ۳,۱۹۷ قربانی (کشته/ناپدید) & \statuswarn حقیقت \\
\altrow ۱۹۹۲ & \textbf{برنامهٔ غرامت:} مستمری ماهانه به خانواده‌ها ($\sim$\$۵۰۰/ماه) & \statuswarn غرامت \\
۱۹۹۶ & تفسیر جدید دادگاه عالی: ناپدیدشدن = «جرم مستمر» (مشمول عفو نمی‌شود) & \statuswarn حقوقی \\
\altrow ۱۹۹۸ & \textbf{بازداشت پینوشه در لندن} (حکم قاضی گارسون اسپانیا) & \statusok نقطهٔ عطف \\
۲۰۰۰ & بازگشت پینوشه + لغو مصونیت توسط دادگاه عالی شیلی & \statusok محاکمه \\
\altrow ۲۰۰۴ & \textbf{کمیسیون والش}: شناسایی ۲۸,۰۰۰+ قربانی شکنجه + غرامت & \statusok تعمیق \\
۲۰۰۵ & \textbf{اصلاح قانون اساسی}: حذف سناتورهای منصوب + کنترل مدنی ارتش & \statusok نهادی \\
\altrow ۲۰۰۶ & مرگ پینوشه (بدون محکومیت نهایی) & $\sim$ \\
۲۰۱۰-۲۰۲۳ & ادامهٔ محاکمات: ۵۰۰+ نظامی محکوم، ۱۲۰+ زندانی & \statusok عدالت \\
\altrow ۲۰۲۳ & پنجاهمین سالگرد کودتا + اعلام «برنامهٔ ملی جستجوی ناپدیدشدگان» & \statusok حافظه \\
\bottomrule
\end{tabularx}
\end{table}

\subsection{کمیسیون‌های حقیقت: رتیگ و والش}

\begin{table}[htbp]
\centering
\caption{مقایسهٔ دو کمیسیون حقیقت شیلی}
\label{tab:app-chile-commissions}
\begin{tabularx}{\textwidth}{>{\raggedleft\arraybackslash}p{3.5cm} >{\raggedleft\arraybackslash}X >{\raggedleft\arraybackslash}X}
\toprule
\headerrow \textbf{شاخص} & \textbf{کمیسیون رتیگ (۱۹۹۱)} & \textbf{کمیسیون والش (۲۰۰۳-۲۰۰۴)} \\
\midrule
نام رسمی & \lr{Comisión Nacional de Verdad y Reconciliación} & \lr{Comisión Nacional sobre Prisión Política y Tortura} \\
\altrow رئیس & \person{رائول رتیگ}{\lr{Raúl Rettig}} & \person{سرخیو والش}{\lr{Sergio Valech}} \\
تمرکز & کشته‌شدگان و ناپدیدشدگان & شکنجه‌شدگان و زندانیان سیاسی \\
\altrow تعداد قربانیان شناسایی‌شده & ۳,۱۹۷ (بعداً ۳,۲۰۰+) & ۲۸,۴۵۹ (بعداً ۴۰,۰۰۰+) \\
اختیار عفو & \xmark (فقط حقیقت‌یابی) & \xmark \\
\altrow غرامت & \cmark (مستمری ماهانه) & \cmark (مبلغ یکباره + مستمری) \\
تأثیر & گشایش فضای حقیقت & تعمیق + شناسایی ابعاد جدید \\
\bottomrule
\end{tabularx}
\end{table}

\begin{keypoint}
\textbf{نوآوری شیلی: عدالت تدریجی (\lr{Incremental Justice}).} برخلاف مدل آفریقای جنوبی (عفو مشروط یکباره) یا مدل عراقی (محاکمهٔ فوری)، شیلی مسیر \textbf{سی‌ساله‌ای} را طی کرد: ابتدا حقیقت (رتیگ ۱۹۹۱) ← سپس غرامت (۱۹۹۲) ← سپس تفسیر حقوقی جدید (۱۹۹۶) ← سپس بازداشت بین‌المللی (۱۹۹۸) ← سپس محاکمه‌های داخلی (۲۰۰۰+). این مدل نشان می‌دهد که عدالت \textbf{لزوماً فوری نیست}، اما باید \textbf{مسیر مشخص و رو به جلو} داشته باشد.
\end{keypoint}

\subsection{بازداشت لندن: نقطهٔ عطف جهانی}

در ۱۶ اکتبر ۱۹۹۸، پینوشه در لندن به‌دستور قاضی \person{بالتاسار گارسون}{\lr{Baltasar Garzón}} اسپانیایی بازداشت شد. این اولین بار بود که یک دیکتاتور سابق بر اساس اصل \termfn{صلاحیت جهانی}{Universal Jurisdiction} بازداشت می‌شد.

\begin{lessonlearned}
\textbf{درس بازداشت لندن برای ایران:} ۱) هیچ مصونیتی دائمی نیست — حتی بندهای غروب و عفوهای داخلی مانع صلاحیت جهانی نمی‌شوند؛ ۲) \textbf{زمان به نفع عدالت} است — اگر امروز محاکمه ممکن نیست، فردا ممکن می‌شود؛ ۳) دادگاه‌های بین‌المللی و خارجی \textbf{مکمل} عدالت داخلی هستند؛ ۴) تهدید صلاحیت جهانی \textbf{انگیزهٔ مذاکره} ایجاد می‌کند (مقامات جمهوری اسلامی بهتر است در عدالت انتقالی داخلی مشارکت کنند تا با دادگاه‌های بین‌المللی مواجه شوند).
\end{lessonlearned}

\sectiondivider

%═══════════════════════════════════════════════════════════
\section{اصلاح بخش امنیتی: مدل «حفظ با نظارت تدریجی»}
\label{app:chile:ssr}
%═══════════════════════════════════════════════════════════

\subsection{چالش: ارتشی که نرفت}

برخلاف بسیاری از نمونه‌ها، در شیلی \textbf{ارتش سقوط نکرد} — پینوشه فرماندهٔ ارتش ماند و ساختار نظامی دست‌نخورده باقی ماند. این ریسک بزرگی بود اما دولت آیلوین آن را مدیریت کرد:

\begin{table}[htbp]
\centering
\caption{استراتژی تدریجی نظارت مدنی بر ارتش شیلی}
\label{tab:app-chile-ssr}
\begin{tabularx}{\textwidth}{>{\centering\arraybackslash}p{2cm} >{\raggedleft\arraybackslash}X >{\centering\arraybackslash}p{2cm}}
\toprule
\headerrow \textbf{دوره} & \textbf{اقدام} & \textbf{ریسک} \\
\midrule
۱۹۹۰-۱۹۹۳ & احتیاط: پینوشه هنوز فرمانده بود. دو بار تهدید نظامی (\lr{Ejercicio de Enlace} ۱۹۹۰ + \lr{Boinazo} ۱۹۹۳) & \riskhigh \\
\altrow ۱۹۹۴-۱۹۹۸ & تثبیت: دولت فری (\lr{Frei}) ادامه‌داد. بازداشت پینوشه در لندن (۱۹۹۸) قدرت نمادین ارتش را شکست & \riskmedium \\
۱۹۹۸-۲۰۰۵ & اصلاح: حذف مصونیت پینوشه + محاکمات نظامیان + کاهش بودجهٔ نظامی & \riskmedium \\
\altrow ۲۰۰۵-۲۰۱۰ & نهادسازی: اصلاح قانون اساسی (۲۰۰۵): رئیس‌جمهور حق عزل فرماندهان + حذف شورای امنیت نظامی‌محور & \risklow \\
۲۰۱۰+ & تحکیم: ارتش کاملاً تحت نظارت مدنی. نظامیان در زندان. فرهنگ دموکراتیک نهادینه & \risklow \\
\bottomrule
\end{tabularx}
\end{table}

\begin{warningbox}
\textbf{ریسک مدل شیلی:} «حفظ ارتش با نظارت تدریجی» ذاتاً \textbf{خطرناک} است. پینوشه دو بار نیروهای مسلح را به حالت آماده‌باش درآورد (۱۹۹۰ و ۱۹۹۳) تا مانع تعقیب قضایی شود. فقط به‌دلیل \textbf{عزم سیاسی دولت + فشار بین‌المللی + شکاف درون ارتش}، کودتا رخ نداد. \emphred{هشدار ایرانی:} اگر سپاه بعد از گذار در ساختار باقی بماند، خطر «پینوشه ایرانی» (فرمانده‌ای که از درون تهدید کند) واقعی است. مدل ایران باید ترکیبی از شیلی (نظارت تدریجی) و آفریقای جنوبی (ادغام) باشد (\seeChapter{ch:guarantees}).
\end{warningbox}

\sectiondivider

%═══════════════════════════════════════════════════════════
\section{مدل اقتصادی: رشد با عدالت}
\label{app:chile:economy}
%═══════════════════════════════════════════════════════════

\subsection{«رشد با برابری» (\lr{Growth with Equity})}

یکی از مهم‌ترین درس‌های شیلی، مدیریت اقتصادی دورهٔ گذار بود:

\begin{table}[htbp]
\centering
\caption{شاخص‌های اقتصادی شیلی قبل و بعد از گذار}
\label{tab:app-chile-economy}
\begin{tabularx}{\textwidth}{>{\raggedleft\arraybackslash}p{4.5cm} >{\centering\arraybackslash}p{3cm} >{\centering\arraybackslash}p{3cm}}
\toprule
\headerrow \textbf{شاخص} & \textbf{۱۹۹۰ (گذار)} & \textbf{۲۰۰۰ (یک دهه بعد)} \\
\midrule
\lr{GDP per capita} & \$۲,۵۰۰ & \$۵,۰۰۰ \\
\altrow نرخ رشد \lr{GDP} & ۳.۷٪ & ۵.۴٪ (میانگین دهه) \\
نرخ فقر & ۴۰٪ & ۱۴٪ \\
\altrow نرخ بیکاری & ۱۰٪ & ۹.۲٪ \\
ضریب جینی & ۰.۵۵ & ۰.۵۲ \\
\altrow تورم & ۲۷٪ & ۴.۵٪ \\
سرمایه‌گذاری خارجی & \$۱.۵B & \$۴.۹B \\
\altrow عضویت بین‌المللی & -- & \lr{WTO} (۱۹۹۵), \lr{APEC} \\
\bottomrule
\end{tabularx}
\end{table}

\begin{enumerate}[nosep]
\item \textbf{حفظ ثبات اقتصادی:} دولت آیلوین ساختار اقتصادی نئولیبرال را \textbf{ناگهان} تغییر نداد — اصلاح تدریجی
\item \textbf{افزایش مالیات:} اصلاح مالیاتی (۱۹۹۰) برای تأمین هزینهٔ برنامه‌های اجتماعی
\item \textbf{سرمایه‌گذاری اجتماعی:} افزایش بودجهٔ بهداشت، آموزش، مسکن
\item \textbf{کاهش فقر:} از ۴۰٪ به ۱۴٪ در یک دهه (موفق‌ترین کاهش فقر در آمریکای لاتین)
\item \textbf{ادغام در اقتصاد جهانی:} عضویت \lr{WTO}، توافقات تجارت آزاد
\end{enumerate}

\begin{recommendation}
\textbf{مدل اقتصادی شیلی} برای ایران آموزنده است: ۱) در دورهٔ گذار، ثبات اقتصادی را فدای شعارهای انقلابی نکنید؛ ۲) اصلاح تدریجی اقتصاد (نه شوک‌تراپی لهستان و نه حفظ وضع موجود)؛ ۳) کاهش فقر = مشروعیت دموکراسی نوپا؛ ۴) ادغام در اقتصاد جهانی مهم‌ترین مشوق بلندمدت است. البته ضعف شیلی (نابرابری پایدار: جینی ۰.۵۲) درس منفی هم دارد: \textbf{رشد بدون بازتوزیع} کافی نیست (\seeChapter{ch:guarantees}).
\end{recommendation}

\sectiondivider

%═══════════════════════════════════════════════════════════
\section{ماتریس درس‌آموخته‌ها برای ایران}
\label{app:chile:lessons}
%═══════════════════════════════════════════════════════════

\begin{table}[htbp]
\centering
\caption{ماتریس انتقال درس‌آموخته‌های شیلی به ایران}
\label{tab:app-chile-lessons}
\begin{tabularx}{\textwidth}{
  >{\raggedleft\arraybackslash}p{2.2cm}
  >{\raggedleft\arraybackslash}p{3.5cm}
  >{\raggedleft\arraybackslash}X
  >{\centering\arraybackslash}p{1.5cm}
}
\toprule
\headerrow \textbf{بُعد} & \textbf{درس شیلی} & \textbf{کاربرد ایرانی} & \textbf{انتقال‌پذیری} \\
\midrule
رفراندوم & پلبیسیت ۱۹۸۸: ابزار تغییر & رفراندوم قانون اساسی جدید & \rating{5} \\
\altrow
ائتلاف‌سازی & \lr{Concertación}: ۱۶ حزب متحد شدند & ائتلاف فراگیر اپوزیسیون ایران & \rating{5} \\
کمپین خلاقانه & «شادی در راه است» + امید (نه خشم) & رسانهٔ امیدبخش + فضای مجازی & \rating{4} \\
\altrow
عدالت تدریجی & حقیقت ← غرامت ← محاکمه (۳۰ سال) & مسیر تدریجی اما رو به جلو & \rating{5} \\
قفل‌های نهادی & پینوشه قفل‌ها گذاشت → ۱۵ سال رفع & پیش‌بینی و مقابله با قفل‌های ج.ا. & \rating{5} \\
\altrow
مدیریت ارتش & حفظ + نظارت تدریجی (ریسکی) & ترکیب با مدل ادغام آفریقای جنوبی & \rating{3} \\
مدل اقتصادی & ثبات + اصلاح تدریجی + کاهش فقر & ثبات اقتصادی اولویت + اصلاح تدریجی & \rating{4} \\
\altrow
شمارش موازی & \lr{PVT} + ۳۰,۰۰۰ ناظر داوطلب & شبکهٔ ناظران مدنی + فناوری & \rating{5} \\
صلاحیت جهانی & بازداشت لندن: مصونیت دائمی نیست & هشدار به مقامات: مذاکره کنید & \rating{4} \\
\altrow
اصلاح قانون اساسی & اصلاحیهٔ ۲۰۰۵: ۱۵ سال طول کشید & مجلس مؤسسان از ابتدا (نه اصلاحیه) & \rating{3} \\
\midrule
\headerrow \multicolumn{3}{l}{\textbf{میانگین انتقال‌پذیری}} & \textbf{\rating{4}} \\
\bottomrule
\end{tabularx}
\end{table}

\sectiondivider

%═══════════════════════════════════════════════════════════
\section{نمودار: مقایسهٔ مسیر شیلی و مسیر پیشنهادی ایران}
\label{app:chile:diagram}
%═══════════════════════════════════════════════════════════

\begin{figure}[htbp]
\centering
\begin{tikzpicture}[
  node distance=0.5cm and 0.3cm,
  timeline/.style={very thick, ->, >=stealth},
  event/.style={
    draw, rounded corners=3pt, minimum width=2.2cm,
    minimum height=0.8cm, font=\tiny, align=center
  },
  chile/.style={event, fill=MainGreen!15, draw=MainGreen!60},
  iran/.style={event, fill=MainPurple!15, draw=MainPurple!60},
  danger/.style={event, fill=MainRed!15, draw=MainRed!60},
  success/.style={event, fill=MainGreen!25, draw=MainGreen!80}
]

% عنوان
\node[font=\small\bfseries, MainGreen] at (0,5.5) {مسیر شیلی (۱۹۸۸-۲۰۰۵)};
\node[font=\small\bfseries, MainPurple] at (0,0) {مسیر پیشنهادی ایران};

% خط زمانی شیلی
\draw[timeline, MainGreen!60] (-6,4.5) -- (6,4.5);

\node[chile] at (-5,4.5) {رفراندوم\\۱۹۸۸};
\node[chile] at (-3,4.5) {انتخابات\\۱۹۸۹};
\node[chile] at (-1,4.5) {رتیگ\\۱۹۹۱};
\node[danger] at (0.5,3.3) {تهدید نظامی\\بویناسو ۱۹۹۳};
\node[chile] at (1.5,4.5) {بازداشت\\لندن ۱۹۹۸};
\node[chile] at (3.5,4.5) {والش\\۲۰۰۴};
\node[success] at (5.5,4.5) {اصلاح ق.ا.\\۲۰۰۵};

% خط زمانی ایران
\draw[timeline, MainPurple!60] (-6,-1) -- (6,-1);

\node[iran] at (-5,-1) {رفراندوم\\ق.ا.\ جدید};
\node[iran] at (-3,-1) {انتخابات\\آزاد};
\node[iran] at (-1,-1) {کمیسیون\\حقیقت};
\node[danger] at (0.5,-2.2) {خطر: سپاه\\تهدید می‌کند؟};
\node[iran] at (1.5,-1) {محاکمات\\تدریجی};
\node[iran] at (3.5,-1) {تحکیم\\نهادی};
\node[success] at (5.5,-1) {دموکراسی\\پایدار};

% فلش‌های ارتباطی
\draw[dashed, MainOrange, ->] (-5,3.8) -- (-5,-0.3);
\draw[dashed, MainOrange, ->] (-3,3.8) -- (-3,-0.3);
\draw[dashed, MainOrange, ->] (-1,3.8) -- (-1,-0.3);
\draw[dashed, MainRed, ->] (0.5,3.0) -- (0.5,-1.7);
\draw[dashed, MainOrange, ->] (1.5,3.8) -- (1.5,-0.3);

% راهنما
\node[font=\tiny, MainOrange] at (7, 2) {انتقال درس};
\node[font=\tiny, MainRed] at (7, 1.5) {انتقال هشدار};

\end{tikzpicture}
\caption{مقایسهٔ مسیر شیلی و مسیر پیشنهادی ایران}
\label{fig:app-chile-comparison}
\end{figure}

\sectiondivider

%═══════════════════════════════════════════════════════════
\section{جمع‌بندی پیوست}
\label{app:chile:conclusion}
%═══════════════════════════════════════════════════════════

\begin{chaptersummary}
جمع‌بندی پیوست پ — مطالعهٔ موردی شیلی:

\begin{enumerate}[nosep]
\item شیلی با امتیاز \textbf{۲۹ از ۴۰} دومین الگوی مهم (پس از آفریقای جنوبی) برای ایران است، به‌ویژه در ابعاد \textbf{رفراندوم}، \textbf{ائتلاف‌سازی}، و \textbf{عدالت تدریجی}.
\item \textbf{رفراندوم ۱۹۸۸} نشان داد که ابزار مسالمت‌آمیز می‌تواند دیکتاتوری نظامی را سرنگون کند — بدون خشونت.
\item \textbf{ائتلاف \lr{Concertación}} الگوی ائتلاف‌سازی اپوزیسیون متفرق است: ۱۶ حزب از دموکرات‌مسیحی تا سوسیالیست متحد شدند.
\item \textbf{کمپین «نه»} نشان داد که پیام \textbf{امید و شادی} مؤثرتر از خشم و انتقام است — درس مستقیم برای رسانه‌های اپوزیسیون ایران.
\item \textbf{عدالت تدریجی} مدل منحصربه‌فردی است: حقیقت امروز ← غرامت فردا ← محاکمه پس‌فردا. مسیر ۳۰ ساله اما رو به جلو.
\item \textbf{قفل‌های نهادی} پینوشه هشدار جدی است: رژیم پیشین می‌تواند حتی پس از شکست، قدرت نهادی حفظ کند. مدل ۶ باید مکانیزم رفع تدریجی قفل‌ها داشته باشد.
\item \textbf{مدل اقتصادی} شیلی (ثبات + اصلاح تدریجی + کاهش فقر) الگوی مدیریت اقتصاد دورهٔ گذار است.
\item ضعف شیلی: \textbf{پینوشه ماند} و ۱۵ سال تهدید کرد + \textbf{نابرابری پایدار}. ایران باید مدل شیلی را با ادغام آفریقای جنوبی ترکیب کند.
\end{enumerate}

\vspace{0.3cm}
\textit{مطالعهٔ تکمیلی:}
\begin{itemize}[nosep]
\item مقایسهٔ جامع ۹ نمونه: \seeChapter{app:comparison}
\item آفریقای جنوبی (الگوی مکمل): \seeChapter{app:south-africa}
\item سناریوی رفراندوم: \seeChapter{ch:scenarios}
\item عدالت انتقالی: \seeChapter{ch:guarantees}
\item تونس (الگوی خاورمیانه‌ای): \seeChapter{app:tunisia}
\end{itemize}
\end{chaptersummary}

\chapterend

%══════════════════════════════════════════════════════════════
% پایان پیوست پ
%══════════════════════════════════════════════════════════════