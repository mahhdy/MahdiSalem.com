%══════════════════════════════════════════════════════════════
% پیوست ب: مطالعه موردی آفریقای جنوبی
% فایل: appendices/app-b-south-africa.tex
% حجم هدف: ۸-۱۰ صفحه
%══════════════════════════════════════════════════════════════

\chapter{مطالعهٔ موردی: آفریقای جنوبی (۱۹۹۰-۱۹۹۹)}
\label{app:south-africa}

\begin{executivesummary}
آفریقای جنوبی یکی از موفق‌ترین نمونه‌های گذار دموکراتیک در نیمهٔ دوم قرن بیستم و \textbf{جامع‌ترین الگوی قابل‌انتقال} به پروندهٔ ایران است. گذار از نظام \termfn{آپارتاید}{Apartheid} (۱۹۴۸-۱۹۹۴) به دموکراسی چندنژادی، از طریق \textbf{مذاکرهٔ چندجانبه} (\lr{CODESA})، \textbf{قانون اساسی فراگیر}، \textbf{کمیسیون حقیقت و آشتی} (\lr{TRC})، و \textbf{ادغام نیروهای مسلح} صورت گرفت. این پیوست هفت بُعد کلیدی این تجربه را تحلیل و درس‌آموخته‌های آن را برای ایران استخراج می‌کند. وجوه مشابهت: ایدئولوژی رسمی سرکوبگر، تحریم‌های بین‌المللی، نیروهای امنیتی قدرتمند، تنوع قومی-زبانی، و دیاسپورای فعال.
\end{executivesummary}

%═══════════════════════════════════════════════════════════
\section{زمینه و بافت تاریخی}
\label{app:sa:context}
%═══════════════════════════════════════════════════════════

\subsection{نظام آپارتاید: ساختار و ویژگی‌ها}

نظام \termfn{آپارتاید}{Apartheid} (به معنای «جدایی» در زبان آفریکانس) از ۱۹۴۸ تا ۱۹۹۴ حاکم بود و جامعهٔ آفریقای جنوبی را بر اساس نژاد به چهار دستهٔ سلسله‌مراتبی تقسیم کرده بود:

\begin{itemize}[nosep]
\item \textbf{سفیدپوستان} (۱۳٪ جمعیت): انحصار کامل قدرت سیاسی و اقتصادی
\item \textbf{رنگین‌پوستان (\lr{Coloured})} (۹٪): حقوق محدود
\item \textbf{آسیایی‌تبارها} (۳٪): حقوق محدود
\item \textbf{سیاه‌پوستان} (۷۵٪): محرومیت کامل از حقوق شهروندی
\end{itemize}

\begin{table}[htbp]
\centering
\caption{مشخصات آفریقای جنوبی در آستانهٔ گذار (۱۹۹۰)}
\label{tab:app-sa-profile}
\begin{tabularx}{\textwidth}{>{\raggedleft\arraybackslash}p{4.5cm} >{\raggedleft\arraybackslash}X}
\toprule
\headerrow \textbf{شاخص} & \textbf{مقدار} \\
\midrule
جمعیت & ۴۰ میلیون نفر \\
\altrow مساحت & ۱,۲۲۰,۰۰۰ \lr{km²} \\
تنوع زبانی & ۱۱ زبان رسمی \\
\altrow \lr{GDP per capita} & $\sim$\$۳,۵۰۰ \\
ضریب جینی & ۰.۶۳ (یکی از بالاترین جهان) \\
\altrow نرخ بیکاری & $\sim$۳۰٪ (سیاه‌پوستان: $\sim$۴۵٪) \\
طول عمر رژیم آپارتاید & ۴۶ سال (۱۹۴۸-۱۹۹۴) \\
\altrow اندازهٔ نیروهای مسلح (\lr{SADF}) & $\sim$۱۰۰,۰۰۰ \\
برنامهٔ هسته‌ای & ۶ کلاهک (خلع‌سلاح ۱۹۸۹-۱۹۹۳) \\
\altrow تحریم‌های بین‌المللی & جامع (تسلیحاتی ۱۹۷۷ + اقتصادی ۱۹۸۶) \\
\bottomrule
\end{tabularx}
\end{table}

\subsection{ستون‌های قدرت نظام آپارتاید}

نظام آپارتاید بر پنج ستون استوار بود — مشابهت‌های هر ستون با جمهوری اسلامی در ادامه تحلیل خواهد شد:

\begin{enumerate}[nosep]
\item \textbf{ایدئولوژی نژادی-مذهبی:} کالوینیسم آفریکانر + نظریهٔ «توسعهٔ جداگانه» (\lr{Separate Development}) — مشابه ایدئولوژی ولایت فقیه
\item \textbf{نیروهای امنیتی:} \org{نیروی دفاعی آفریقای جنوبی}{\lr{SADF}} + پلیس امنیتی (\lr{SAP}) + نیروهای ویژه — مشابه سپاه + بسیج + اطلاعات
\item \textbf{نظام حقوقی تبعیض‌آمیز:} بیش از ۳۰۰ قانون نژادپرستانه — مشابه قوانین تبعیض جنسیتی/مذهبی
\item \textbf{کنترل اقتصادی:} شرکت‌های دولتی-نظامی (\lr{ARMSCOR}) + معادن — مشابه خاتم‌الانبیاء + بنیادها
\item \textbf{سرکوب جامعهٔ مدنی:} ممنوعیت \lr{ANC/PAC} + سانسور + حکومت نظامی (\lr{State of Emergency} ۱۹۸۵-۱۹۹۰) — مشابه سرکوب جنبش‌های اعتراضی
\end{enumerate}

\begin{casestudy}
\textbf{مقایسهٔ ساختاری آپارتاید و جمهوری اسلامی:}
هر دو نظام بر \textbf{ایدئولوژی رسمی تبعیض‌آمیز} بنا شده‌اند (نژادی در آفریقا، مذهبی-جنسیتی در ایران). هر دو \textbf{نیروهای امنیتی موازی} با منافع اقتصادی دارند. هر دو تحت \textbf{تحریم‌های بین‌المللی} قرار گرفتند. تفاوت کلیدی: آپارتاید اقلیتِ حاکم بر اکثریت بود (۱۳٪ بر ۸۷٪)؛ در ایران نخبگان حاکم لزوماً اقلیت قومی/نژادی نیستند. تفاوت دوم: آفریقای جنوبی فاقد مسئلهٔ برنامهٔ هسته‌ای فعال در \textbf{زمان} گذار بود (خلع‌سلاح \textbf{قبل} از مذاکرات).
\end{casestudy}

\sectiondivider

%═══════════════════════════════════════════════════════════
\section{مسیر گذار: از بحران تا مذاکره}
\label{app:sa:transition}
%═══════════════════════════════════════════════════════════

\subsection{محرک‌های گذار (۱۹۸۵-۱۹۹۰)}

ترکیبی از فشارهای داخلی و خارجی رژیم آپارتاید را به مذاکره واداشت:

\begin{table}[htbp]
\centering
\caption{محرک‌های چندگانهٔ گذار آفریقای جنوبی}
\label{tab:app-sa-drivers}
\begin{tabularx}{\textwidth}{>{\centering\arraybackslash}p{2.5cm} >{\raggedleft\arraybackslash}X >{\centering\arraybackslash}p{2cm}}
\toprule
\headerrow \textbf{نوع فشار} & \textbf{توضیح} & \textbf{شدت} \\
\midrule
اقتصادی-بین‌المللی & تحریم‌های جامع (اقتصادی ۱۹۸۶ + تسلیحاتی ۱۹۷۷) + خروج سرمایه + کاهش ارزش رَند & \riskhigh \\
\altrow مدنی-داخلی & اعتصابات عمومی + شورش‌های تاونشیپ + کمپین نافرمانی + \lr{UDF} & \riskhigh \\
مسلحانه & عملیات‌های \lr{MK} (بازوی نظامی \lr{ANC}) + تشدید ناامنی & \riskmedium \\
\altrow ژئوپلیتیکی & فروپاشی شوروی = حذف بهانهٔ ضدکمونیستی + استقلال نامیبیا & \riskhigh \\
جمعیت‌شناختی & رشد سریع جمعیت سیاه‌پوست + فشار بر منابع & \riskmedium \\
\altrow شکاف نخبگان & اصلاح‌طلبان درون حزب ملی (\person{دکلرک}{\lr{F.W. de Klerk}}) & \riskhigh \\
\bottomrule
\end{tabularx}
\end{table}

\subsection{لحظهٔ تاریخی: ۲ فوریه ۱۹۹۰}

\person{اف.دبلیو. دکلرک}{\lr{F.W. de Klerk}}، رئیس‌جمهور جدید حزب ملی، در ۲ فوریه ۱۹۹۰ سخنرانی تاریخی خود در پارلمان ایراد کرد و اعلام نمود:

\begin{itemize}[nosep]
\item آزادی \person{نلسون ماندلا}{\lr{Nelson Mandela}} و سایر زندانیان سیاسی
\item رفع ممنوعیت \lr{ANC}، \lr{PAC}، حزب کمونیست و ۳۰+ سازمان
\item لغو حکومت نظامی
\item آغاز مذاکرات برای قانون اساسی جدید
\end{itemize}

\begin{keypoint}
\textbf{درس کلیدی:} دکلرک از موضع \textbf{قدرت نسبی} (نه فروپاشی) مذاکره را آغاز کرد. ارتش هنوز قوی بود، اما او فهمید که \textbf{هزینهٔ ادامهٔ وضع موجود} بیش از هزینهٔ مذاکره است. این مشابه سناریوی \lr{D} (تحول از درون) یا \lr{B} (مذاکره‌ای) برای ایران است (\seeChapter{ch:scenarios}).
\end{keypoint}

\subsection{گاه‌شمار گذار}

\begin{table}[htbp]
\centering
\caption{گاه‌شمار کلیدی گذار آفریقای جنوبی (۱۹۹۰-۱۹۹۹)}
\label{tab:app-sa-timeline}
\begin{tabularx}{\textwidth}{>{\centering\arraybackslash}p{2.5cm} >{\raggedleft\arraybackslash}X >{\centering\arraybackslash}p{2.2cm}}
\toprule
\headerrow \textbf{تاریخ} & \textbf{رویداد} & \textbf{فاز معادل ایران} \\
\midrule
فوریه ۱۹۹۰ & سخنرانی دکلرک + آزادی ماندلا (۱۱ فوریه) & فاز ۱ (تثبیت) \\
\altrow دسامبر ۱۹۹۱ & آغاز \lr{CODESA I} (۱۹ حزب + سازمان) & فاز ۱-۲ \\
مه ۱۹۹۲ & \lr{CODESA II}: بن‌بست بر سر وتو & فاز ۲ (بحران) \\
\altrow ژوئن ۱۹۹۲ & کشتار بویپاتونگ + \lr{ANC} مذاکرات را ترک کرد & بحران \\
سپتامبر ۱۹۹۲ & توافق ریکورد (\lr{Record of Understanding}) بین دکلرک و ماندلا & بازگشت به مذاکره \\
\altrow آوریل ۱۹۹۳ & ترور \person{کریس هانی}{\lr{Chris Hani}} — ماندلا آرامش خواست & آزمون بحران \\
نوامبر ۱۹۹۳ & تصویب قانون اساسی موقت (فصل ۱۵: عفو) & فاز ۲ \\
\altrow آوریل ۱۹۹۴ & \textbf{اولین انتخابات آزاد} (۲۷ آوریل): مشارکت ۸۶٪ & فاز ۲ (نقطهٔ عطف) \\
مه ۱۹۹۴ & ماندلا رئیس‌جمهور + دولت وحدت ملی & فاز ۲-۳ \\
\altrow دسامبر ۱۹۹۵ & تشکیل \lr{TRC} به ریاست اسقف \person{دزموند توتو}{\lr{Desmond Tutu}} & فاز ۳ \\
دسامبر ۱۹۹۶ & تصویب قانون اساسی دائمی (منشور حقوق) & فاز ۳ \\
\altrow اکتبر ۱۹۹۸ & ارائهٔ گزارش نهایی \lr{TRC} (۵ جلد) & فاز ۳ \\
ژوئن ۱۹۹۹ & \textbf{دومین انتخابات آزاد}: انتقال از ماندلا به مبکی & فاز ۴ (تحکیم) \\
\bottomrule
\end{tabularx}
\end{table}

\sectiondivider

%═══════════════════════════════════════════════════════════
\section{مذاکرات \lr{CODESA}: الگوی گفت‌وگوی ملی}
\label{app:sa:codesa}
%═══════════════════════════════════════════════════════════

\subsection{ساختار و شرکت‌کنندگان}

\org{مجمع یک آفریقای جنوبی دموکراتیک}{\lr{Convention for a Democratic South Africa (CODESA)}} در دو مرحله (۱۹۹۱ و ۱۹۹۲) برگزار شد و \textbf{۱۹ حزب و سازمان} در آن شرکت کردند:

\begin{table}[htbp]
\centering
\caption{ترکیب شرکت‌کنندگان \lr{CODESA} و معادل ایرانی}
\label{tab:app-sa-codesa}
\begin{tabularx}{\textwidth}{>{\raggedleft\arraybackslash}p{3.5cm} >{\raggedleft\arraybackslash}X >{\raggedleft\arraybackslash}p{3.5cm}}
\toprule
\headerrow \textbf{جریان آفریقایی} & \textbf{نقش} & \textbf{معادل ایرانی احتمالی} \\
\midrule
حزب ملی (\lr{NP}: دکلرک) & رژیم حاکم (اصلاح‌طلب) & بخش اصلاح‌طلب نظام \\
\altrow \lr{ANC} (ماندلا) & اپوزیسیون اصلی & ائتلاف فراگیر اپوزیسیون \\
حزب آزادی اینکاتا (\lr{IFP}: بوتلزی) & ملی‌گرای زولو & جریان‌های قومی/فدرالیست \\
\altrow حزب کمونیست (\lr{SACP}) & چپ & جریان چپ/سوسیالیست \\
\lr{PAC} & ملی‌گرای رادیکال & جریان‌های رادیکال \\
\altrow \lr{COSATU} & اتحادیهٔ کارگری & تشکل‌های کارگری مستقل \\
نمایندگان بانتوستان‌ها & خودمختاری‌های قومی & نمایندگان اقوام \\
\altrow احزاب لیبرال سفیدپوست (\lr{DP}) & میانه‌رو & اصلاح‌طلبان مستقل \\
نمایندگان مذهبی & کلیساها & نمایندگان ادیان و مذاهب \\
\altrow ناظران بین‌المللی & \lr{UN + OAU + EC} & \lr{UN + EU + AU} \\
\bottomrule
\end{tabularx}
\end{table}

\subsection{اصول کلیدی \lr{CODESA}}

پنج اصل بنیادین \lr{CODESA} که برای ایران الگو هستند:

\begin{enumerate}[nosep]
\item \textbf{اصل «فراگیری کافی» (\lr{Sufficient Inclusivity}):} هیچ جریان مهمی نباید از میز مذاکره حذف شود — حتی جریان‌های ناخوشایند
\item \textbf{اصل «اجماع کافی» (\lr{Sufficient Consensus}):} نه اتفاق آرا (غیرممکن)، بلکه توافق اکثریت قریب به اتفاق
\item \textbf{اصل «هم‌زمانی مذاکره و فشار» (\lr{Rolling Mass Action}):} \lr{ANC} هم‌زمان با مذاکره، اعتراضات خیابانی را ادامه داد
\item \textbf{اصل «غروب» (\lr{Sunset Clauses}):} تضمین‌های موقت برای کاهش ترس حاکمان (مثلاً: ۵ سال دولت وحدت ملی، امنیت شغلی کارمندان)
\item \textbf{اصل «دو مرحله‌ای بودن» (\lr{Two-Phase Constitution}):} ابتدا قانون اساسی موقت (۱۹۹۳)، سپس دائمی (۱۹۹۶)
\end{enumerate}

\begin{lessonlearned}
\textbf{اصل «بندهای غروب» (\lr{Sunset Clauses}) مهم‌ترین نوآوری آفریقای جنوبی} بود و توسط \person{جو اسلوو}{\lr{Joe Slovo}} (رهبر حزب کمونیست!) پیشنهاد شد. این بندها به سفیدپوستان \textbf{تضمین موقت} دادند: ۱) ادامهٔ خدمت کارمندان دولت به مدت ۵ سال، ۲) دولت وحدت ملی (دکلرک معاون ماندلا شد)، ۳) عدم مصادرهٔ اموال. بدون این تضمین‌ها، سفیدپوستان مذاکره نمی‌کردند. \emphorange{کاربرد ایرانی:} تضمین‌های مشابه برای بخش‌هایی از نظام فعلی که حاضر به همکاری شوند (\seeChapter{ch:guarantees}).
\end{lessonlearned}

\sectiondivider

%═══════════════════════════════════════════════════════════
\section{قانون اساسی: مدل مجلس مؤسسان فراگیر}
\label{app:sa:constitution}
%═══════════════════════════════════════════════════════════

\subsection{فرآیند دو مرحله‌ای}

\begin{enumerate}[nosep]
\item \textbf{قانون اساسی موقت (۱۹۹۳):} محصول مذاکرات \lr{CODESA} — ۳۴ اصل قانون اساسی (\lr{Constitutional Principles}) تعیین شد که قانون اساسی دائمی نمی‌توانست نقض کند
\item \textbf{قانون اساسی دائمی (۱۹۹۶):} توسط \textbf{مجلس مؤسسان} (= مجلس ملی + سنا) نوشته شد با مشارکت عمومی گسترده
\end{enumerate}

\subsection{مشارکت عمومی بی‌سابقه}

\begin{table}[htbp]
\centering
\caption{آمار مشارکت عمومی در تدوین قانون اساسی ۱۹۹۶}
\label{tab:app-sa-constitution-participation}
\begin{tabularx}{\textwidth}{>{\raggedleft\arraybackslash}p{5cm} >{\centering\arraybackslash}X}
\toprule
\headerrow \textbf{شاخص مشارکت} & \textbf{آمار} \\
\midrule
تعداد پیشنهادات مردمی دریافتی & بیش از ۲ میلیون \\
\altrow جلسات عمومی در سراسر کشور & ۲۶ کارگاه منطقه‌ای \\
مدت تدوین & ۲ سال (۱۹۹۴-۱۹۹۶) \\
\altrow تأیید دادگاه قانون اساسی & \cmark (بررسی انطباق با ۳۴ اصل) \\
تعداد زبان‌های رسمی در قانون اساسی & ۱۱ زبان \\
\altrow حقوق زنان & مادهٔ ۹: برابری کامل + ممنوعیت تبعیض جنسیتی \\
حقوق اقلیت‌ها & فصل ۲: منشور حقوق جامع (\lr{Bill of Rights}) \\
\bottomrule
\end{tabularx}
\end{table}

\subsection{ویژگی‌های منشور حقوق (\lr{Bill of Rights})}

\textbf{فصل دوم قانون اساسی ۱۹۹۶} یکی از پیشرفته‌ترین منشورهای حقوقی جهان است:

\begin{itemize}[nosep]
\item \textbf{حق حیات} (ممنوعیت اعدام — رأی دادگاه قانون اساسی ۱۹۹۵)
\item \textbf{برابری و ممنوعیت تبعیض} بر اساس نژاد، جنسیت، گرایش جنسی، مذهب، قومیت
\item \textbf{حقوق اجتماعی-اقتصادی:} حق مسکن، آب، غذا، تأمین اجتماعی، بهداشت
\item \textbf{حقوق زبانی:} ۱۱ زبان رسمی + هیئت زبان‌های آفریقای جنوبی (\lr{PanSALB})
\item \textbf{حقوق فرهنگی و مذهبی:} آزادی مذهب + حق آموزش به زبان مادری
\item \textbf{حقوق محیط‌زیستی:} نسل سوم حقوق بشر
\end{itemize}

\begin{recommendation}
\textbf{مدل قانون اساسی آفریقای جنوبی} بهترین الگو برای ایران در بُعد «فراگیری» است. قانون اساسی آیندهٔ ایران باید حداقل شامل: ۱) منشور حقوق جامع (شامل حقوق زنان، اقوام، اقلیت‌های مذهبی)، ۲) ۳+ زبان رسمی یا «زبان‌های ملی»، ۳) دادگاه قانون اساسی مستقل، ۴) حقوق اجتماعی-اقتصادی، ۵) فرآیند تدوین مشارکتی با دریافت پیشنهادات مردمی باشد (\seeChapter{ch:guarantees}).
\end{recommendation}

\sectiondivider

%═══════════════════════════════════════════════════════════
\section{کمیسیون حقیقت و آشتی (\lr{TRC})}
\label{app:sa:trc}
%═══════════════════════════════════════════════════════════

\subsection{ساختار و مأموریت}

\org{کمیسیون حقیقت و آشتی}{\lr{Truth and Reconciliation Commission (TRC)}} در دسامبر ۱۹۹۵ تشکیل شد و تا ۱۹۹۸ (گزارش نهایی ۲۰۰۳) فعالیت کرد:

\begin{table}[htbp]
\centering
\caption{ساختار و آمار \lr{TRC} آفریقای جنوبی}
\label{tab:app-sa-trc}
\begin{tabularx}{\textwidth}{>{\raggedleft\arraybackslash}p{5cm} >{\raggedleft\arraybackslash}X}
\toprule
\headerrow \textbf{شاخص} & \textbf{جزئیات} \\
\midrule
رئیس & اسقف اعظم \person{دزموند توتو}{\lr{Desmond Tutu}} \\
\altrow تعداد اعضا & ۱۷ کمیسیونر \\
سه کمیتهٔ فرعی & ۱) نقض حقوق بشر، ۲) عفو، ۳) غرامت و بازتوانی \\
\altrow تعداد شهادت‌های دریافتی & ۲۱,۰۰۰+ \\
تعداد جلسات علنی & ۸۰+ جلسه در سراسر کشور \\
\altrow درخواست‌های عفو & ۷,۱۱۲ درخواست \\
عفو اعطاشده & ۱,۵۲۷ (۲۱.۵٪) \\
\altrow عفو رد شده & ۵,۳۹۲ (۷۵.۸٪) \\
پخش زندهٔ تلویزیونی & \cmark (بی‌سابقه) \\
\altrow حجم گزارش نهایی & ۵ جلد + ۲ جلد تکمیلی (۲۰۰۳) \\
بودجه & $\sim$\$۱۸ میلیون دلار \\
\bottomrule
\end{tabularx}
\end{table}

\subsection{مکانیزم «عفو مشروط»}

نوآوری محوری \lr{TRC} مکانیزم \textbf{عفو مشروط} (\lr{Conditional Amnesty}) بود:

\begin{enumerate}[nosep]
\item عامل خشونت باید \textbf{شخصاً} درخواست عفو می‌کرد (نه عفو عمومی)
\item باید \textbf{اعتراف کامل و علنی} می‌کرد (نه پشت درهای بسته)
\item عمل باید \textbf{با انگیزهٔ سیاسی} بوده باشد (نه جنایت شخصی)
\item باید \textbf{تناسب} بین عمل و هدف سیاسی وجود می‌داشت
\item \textbf{قربانیان حق بودند} که حضور یابند و سؤال بپرسند
\item تصمیم توسط \textbf{هیئت قضایی مستقل} گرفته می‌شد
\end{enumerate}

\begin{keypoint}
\textbf{فلسفهٔ \lr{TRC}} بر مفهوم آفریقایی \termfn{اوبونتو}{\lr{Ubuntu}} — «من هستم چون ما هستیم» — استوار بود. این فلسفه نه \textbf{عفو بدون حقیقت} (مدل اسپانیا) و نه \textbf{انتقام بدون آشتی} (مدل نورنبرگ) را می‌پذیرفت، بلکه راه سومی ارائه کرد: \textbf{حقیقت در ازای عفو، آشتی در ازای اعتراف}.
\end{keypoint}

\subsection{نقد و محدودیت‌های \lr{TRC}}

با وجود موفقیت‌ها، \lr{TRC} با انتقادات جدی مواجه شد:

\begin{warningbox}
\textbf{محدودیت‌های \lr{TRC} که ایران باید بداند:}
\begin{enumerate}[nosep]
\item \textbf{غرامت ناکافی:} توصیه‌های کمیتهٔ غرامت عمدتاً اجرا نشد — متوسط پرداخت: فقط $\sim$\$۳,۹۰۰ به هر قربانی
\item \textbf{عدم تعقیب مجرمان عفو‌نشده:} دولت فاقد ارادهٔ سیاسی برای پیگیری بود
\item \textbf{تمرکز بر خشونت فیزیکی:} خشونت ساختاری آپارتاید (فقر، بیکاری، محرومیت) کمتر بررسی شد
\item \textbf{نابرابری اقتصادی پایدار:} \lr{TRC} نتوانست عدالت اقتصادی ایجاد کند — ضریب جینی هنوز ۰.۶۳ است
\item \textbf{«آشتی» سطحی:} بسیاری از قربانیان احساس عدالت نکردند
\end{enumerate}
\end{warningbox}

\sectiondivider

%═══════════════════════════════════════════════════════════
\section{اصلاح بخش امنیتی: مدل ادغام}
\label{app:sa:ssr}
%═══════════════════════════════════════════════════════════

\subsection{چالش: ادغام ۷ نیروی مسلح}

یکی از پیچیده‌ترین ابعاد گذار آفریقای جنوبی، ادغام نیروهای مسلح رقیب در یک ارتش واحد بود:

\begin{table}[htbp]
\centering
\caption{ادغام نیروهای مسلح در \lr{SANDF}}
\label{tab:app-sa-sandf}
\begin{tabularx}{\textwidth}{>{\raggedleft\arraybackslash}p{4cm} >{\centering\arraybackslash}p{2cm} >{\raggedleft\arraybackslash}X}
\toprule
\headerrow \textbf{نیرو} & \textbf{تعداد تقریبی} & \textbf{ماهیت} \\
\midrule
\lr{SADF} (ارتش رژیم) & ۹۰,۰۰۰ & حرفه‌ای، سفیدپوست‌محور \\
\altrow \lr{MK} (بازوی نظامی \lr{ANC}) & ۲۸,۰۰۰ & چریکی، تبعیدی \\
\lr{APLA} (بازوی نظامی \lr{PAC}) & ۶,۰۰۰ & چریکی \\
\altrow نیروهای بانتوستان‌ها (۴ ارتش) & ۱۱,۰۰۰ & نیمه‌حرفه‌ای \\
\lr{KZP} (پلیس کوازولو) & ۸,۰۰۰ & وابسته به \lr{IFP} \\
\midrule
\headerrow \textbf{\lr{SANDF} (نیروی جدید)} & \textbf{$\sim$۱۰۰,۰۰۰} & \textbf{ملی-فراگیر} \\
\bottomrule
\end{tabularx}
\end{table}

\subsection{استراتژی ادغام}

\begin{enumerate}[nosep]
\item \textbf{ادغام، نه انحلال:} برخلاف عراق، هیچ نیرویی منحل نشد
\item \textbf{نظارت مدنی:} وزیر دفاع غیرنظامی + نظارت پارلمانی
\item \textbf{کاهش تدریجی:} از ۱۴۳,۰۰۰ (مجموع) به $\sim$۱۰۰,۰۰۰
\item \textbf{آموزش مشترک:} نیروهای \lr{MK} آموزش حرفه‌ای دیدند
\item \textbf{ترکیب رهبری:} فرماندهی ترکیبی (ژنرال \person{گئورگ مرینگ}{\lr{Georg Meiring}} از \lr{SADF} + ژنرال \person{سیفیوه نیاندا}{\lr{Siphiwe Nyanda}} از \lr{MK})
\item \textbf{بازنشستگی افتخاری:} بستهٔ مالی برای نظامیانی که نمی‌خواستند بمانند
\end{enumerate}

\begin{lessonlearned}
\textbf{مدل ادغام آفریقای جنوبی} برای سپاه پاسداران ایران با تعدیل‌هایی قابل‌استفاده است: ۱) سپاه منحل نمی‌شود بلکه در ارتش ملی واحد ادغام می‌شود؛ ۲) فرماندهی ترکیبی (از هر دو طرف)؛ ۳) بُعد اقتصادی سپاه باید \textbf{جداگانه} مدیریت شود (مدل اندونزی، نه آفریقای جنوبی)؛ ۴) بستهٔ مالی بازنشستگی افتخاری برای کاهش مقاومت؛ ۵) زمان‌بندی: ۵-۱۰ سال (\seeChapter{ch:guarantees}).
\end{lessonlearned}

\subsection{خلع‌سلاح هسته‌ای: تنها الگوی موجود}

آفریقای جنوبی تنها کشور جهان است که \textbf{داوطلبانه} برنامهٔ سلاح هسته‌ای خود را کنار گذاشت:

\begin{table}[htbp]
\centering
\caption{خلع‌سلاح هسته‌ای آفریقای جنوبی و مقایسه با ایران}
\label{tab:app-sa-nuclear}
\begin{tabularx}{\textwidth}{>{\raggedleft\arraybackslash}p{4cm} >{\raggedleft\arraybackslash}X >{\raggedleft\arraybackslash}X}
\toprule
\headerrow \textbf{بُعد} & \textbf{آفریقای جنوبی} & \textbf{ایران (وضعیت فعلی)} \\
\midrule
تعداد کلاهک‌ها & ۶ عدد & -- (غنی‌سازی بالای ۶۰٪) \\
\altrow زمان خلع‌سلاح & ۱۹۸۹-۱۹۹۳ (قبل از گذار) & باید در فاز ۱-۲ تعیین شود \\
انگیزهٔ خلع‌سلاح & ترس از دست‌یابی \lr{ANC} + فشار بین‌المللی & پیش‌شرط لغو تحریم؟ \\
\altrow مکانیزم تأیید & بازرسی \lr{IAEA} (پس از الحاق به \lr{NPT} ۱۹۹۱) & پروتکل الحاقی + بازرسی‌های ویژه \\
نقش در مذاکرات & ابزار اعتمادسازی & شرط لازم برای حمایت بین‌المللی \\
\altrow درس کلیدی & خلع‌سلاح \textbf{قبل} از انتقال قدرت & ترجیحاً در فاز ۰-۱ \\
\bottomrule
\end{tabularx}
\end{table}

\begin{warningbox}
تفاوت حیاتی: دکلرک برنامهٔ هسته‌ای را \textbf{قبل از} مذاکرات خلع کرد تا سلاح به دست \lr{ANC} نیفتد. در ایران، جمهوری اسلامی ممکن است برعکس عمل کند و برنامهٔ هسته‌ای را \textbf{ابزار چانه‌زنی} قرار دهد. مدل ایرانی باید خلع هسته‌ای را به \textbf{بسته‌ای از مشوق‌ها} (لغو تحریم + سرمایه‌گذاری + تضمین امنیتی) گره بزند (\seeChapter{ch:guarantees}).
\end{warningbox}

\sectiondivider

%═══════════════════════════════════════════════════════════
\section{نقش بین‌المللی: نظارت حمایتی}
\label{app:sa:international}
%═══════════════════════════════════════════════════════════

\subsection{سطوح مداخلهٔ بین‌المللی}

نقش بین‌المللی در آفریقای جنوبی \textbf{حمایتی-نظارتی} بود (مدل ۳ از فصل ۳)، نه مدیریت مستقیم:

\begin{table}[htbp]
\centering
\caption{نقش بازیگران بین‌المللی در گذار آفریقای جنوبی}
\label{tab:app-sa-intl}
\begin{tabularx}{\textwidth}{>{\raggedleft\arraybackslash}p{3.5cm} >{\raggedleft\arraybackslash}X >{\centering\arraybackslash}p{2cm}}
\toprule
\headerrow \textbf{بازیگر} & \textbf{نقش} & \textbf{اثرگذاری} \\
\midrule
سازمان ملل (\lr{UNOMSA}) & ناظران انتخاباتی (۲,۱۲۰ نفر) + ناظران حقوق بشر & \rating{4} \\
\altrow \lr{OAU/AU} & حمایت دیپلماتیک + مشروعیت‌بخشی & \rating{3} \\
اتحادیهٔ اروپا & تحریم (قبل) + کمک مالی (بعد) + ناظران & \rating{4} \\
\altrow ایالات متحده & تحریم‌ها (\lr{CAAA} ۱۹۸۶) + فشار دیپلماتیک & \rating{4} \\
جنبش ضد آپارتاید جهانی & فشار افکار عمومی + تحریم فرهنگی-ورزشی & \rating{5} \\
\altrow دولت‌های جبهه‌ای (\lr{Frontline States}) & پناهندگی + پایگاه \lr{ANC} & \rating{3} \\
کمنولث & تعلیق عضویت + فشار & \rating{3} \\
\altrow \lr{IAEA} & تأیید خلع‌سلاح هسته‌ای & \rating{4} \\
\lr{NGO}ها (\lr{HRW, AI, IDASA}) & مستندسازی + آموزش مدنی & \rating{4} \\
\bottomrule
\end{tabularx}
\end{table}

\begin{keypoint}
\textbf{کلید موفقیت:} نقش بین‌المللی در آفریقای جنوبی \textbf{مکمل} بود نه \textbf{جایگزین} مالکیت ملی. \lr{UNOMSA} فقط ۲,۱۲۰ ناظر فرستاد (مقایسه کنید با ۸,۰۰۰ نفر تیمور شرقی یا ۱۵۰,۰۰۰ نفر عراق). مذاکرات \lr{CODESA} توسط \textbf{خود آفریقایی‌ها} هدایت شد. این دقیقاً مدلی است که برای ایران توصیه می‌شود: نظارت بین‌المللی حمایتی (مدل ۳-۴ → مدل ۶) با مالکیت ملی ایرانی (\seeChapter{ch:approaches}).
\end{keypoint}

\sectiondivider

%═══════════════════════════════════════════════════════════
\section{نتایج و ارزیابی بلندمدت}
\label{app:sa:outcomes}
%═══════════════════════════════════════════════════════════

\subsection{دستاوردها}

\begin{table}[htbp]
\centering
\caption{دستاوردها و چالش‌های پایدار آفریقای جنوبی}
\label{tab:app-sa-outcomes}
\begin{tabularx}{\textwidth}{>{\centering\arraybackslash}p{1cm} >{\raggedleft\arraybackslash}X >{\centering\arraybackslash}p{2cm}}
\toprule
\headerrow & \textbf{دستاوردها} & \textbf{امتیاز} \\
\midrule
\cmark & گذار مسالمت‌آمیز (بدون جنگ داخلی) & \starrating{5} \\
\altrow \cmark & قانون اساسی فراگیر و مترقی & \starrating{5} \\
\cmark & سه انتقال مسالمت‌آمیز قدرت (۱۹۹۴، ۱۹۹۹، ۲۰۰۹+) & \starrating{5} \\
\altrow \cmark & آزادی مطبوعات و جامعهٔ مدنی فعال & \starrating{4} \\
\cmark & ادغام موفق نیروهای مسلح & \starrating{4} \\
\altrow \cmark & خلع‌سلاح هسته‌ای داوطلبانه & \starrating{5} \\
\cmark & لغو مجازات اعدام & \starrating{5} \\
\midrule
\headerrow & \textbf{چالش‌های پایدار} & \textbf{شدت} \\
\midrule
\xmark & نابرابری اقتصادی شدید (جینی ۰.۶۳) & \riskhigh \\
\altrow \xmark & بیکاری بالا ($\sim$۳۳٪، جوانان: $\sim$۶۰٪) & \riskhigh \\
\xmark & فساد سیستماتیک (دورهٔ زوما: \lr{State Capture}) & \riskhigh \\
\altrow \xmark & خشونت جنسیتی بالا & \riskhigh \\
\xmark & بحران خدمات عمومی (آب، برق) & \riskmedium \\
\altrow \xmark & تسلط تک‌حزبی \lr{ANC} (تا ۲۰۲۴) & \riskmedium \\
\bottomrule
\end{tabularx}
\end{table}

\subsection{شاخص‌های کمّی (۲۰۲۳)}

\begin{center}
\begin{tabularx}{0.85\textwidth}{>{\raggedleft\arraybackslash}X >{\centering\arraybackslash}p{3cm}}
\toprule
\headerrow \textbf{شاخص} & \textbf{مقدار} \\
\midrule
\lr{V-Dem Liberal Democracy Index} & ۰.۷۲ \\
\altrow \lr{Freedom House} & آزاد (۷۹/۱۰۰) \\
\lr{Transparency International CPI} & ۴۳/۱۰۰ (رتبهٔ ۸۳) \\
\altrow \lr{RSF Press Freedom} & رتبهٔ ۳۸ \\
\lr{GDP per capita (PPP)} & $\sim$\$۱۵,۰۰۰ \\
\altrow \lr{HDI} & ۰.۷۱۳ (متوسط-بالا) \\
\bottomrule
\end{tabularx}
\end{center}

\sectiondivider

%═══════════════════════════════════════════════════════════
\section{ماتریس درس‌آموخته‌ها برای ایران}
\label{app:sa:lessons}
%═══════════════════════════════════════════════════════════

\begin{table}[htbp]
\centering
\caption{ماتریس انتقال درس‌آموخته‌های آفریقای جنوبی به ایران}
\label{tab:app-sa-lessons}
\begin{tabularx}{\textwidth}{
  >{\raggedleft\arraybackslash}p{2.5cm}
  >{\raggedleft\arraybackslash}p{3.5cm}
  >{\raggedleft\arraybackslash}X
  >{\centering\arraybackslash}p{1.5cm}
}
\toprule
\headerrow \textbf{بُعد} & \textbf{درس آفریقای جنوبی} & \textbf{کاربرد ایرانی} & \textbf{انتقال‌پذیری} \\
\midrule
مذاکره & \lr{CODESA}: ۱۹ طرف، فراگیر & «کنفرانس ملی ایران» با حضور همهٔ جریان‌ها & \rating{5} \\
\altrow
بندهای غروب & تضمین‌های موقت برای رژیم پیشین & تضمین ۵ ساله برای همکاری‌کنندگان نظام & \rating{5} \\
قانون اساسی & مجلس مؤسسان + مشارکت ۲M + ۱۱ زبان & مجلس مؤسسان + سهمیهٔ اقوام/زنان + ۳+ زبان & \rating{5} \\
\altrow
\lr{TRC} & حقیقت + عفو مشروط + پخش زنده & «کمیسیون حقیقت ایران» + رسانهٔ ملی & \rating{4} \\
ادغام نظامی & ۷ نیرو → \lr{SANDF} & سپاه + ارتش → ارتش ملی واحد & \rating{4} \\
\altrow
هسته‌ای & خلع‌سلاح قبل از گذار & الحاق به پروتکل الحاقی + بازرسی \lr{IAEA} & \rating{3} \\
نظارت بین‌المللی & \lr{UNOMSA} (حمایتی) & مدل ۶: حمایتی-نظارتی & \rating{5} \\
\altrow
عدالت اقتصادی & ضعیف (نابرابری پایدار) & ایران باید از این ضعف بیاموزد & \rating{4} \\
فراگیری & ۱۱ زبان + منشور حقوق & زبان‌های ملی + منشور حقوق اقلیت‌ها & \rating{5} \\
\altrow
ضد فساد & ضعیف (\lr{State Capture}) & نهاد ضد فساد مستقل از روز اول & \rating{4} \\
\midrule
\headerrow \multicolumn{3}{l}{\textbf{میانگین انتقال‌پذیری}} & \textbf{\rating{4}} \\
\bottomrule
\end{tabularx}
\end{table}

\sectiondivider

%═══════════════════════════════════════════════════════════
\section{نمودار: مسیر گذار آفریقای جنوبی و نقاط انتقال به ایران}
\label{app:sa:diagram}
%═══════════════════════════════════════════════════════════

\begin{figure}[htbp]
\centering
\begin{tikzpicture}[
  node distance=1.2cm and 2.5cm,
  phase/.style={
    draw, rounded corners=5pt, minimum width=3.2cm,
    minimum height=1.5cm, font=\small\bfseries, align=center,
    text=white
  },
  arrow/.style={->, thick, >=stealth},
  label/.style={font=\tiny, align=center, text width=3cm},
  iranlabel/.style={font=\tiny\itshape, align=center, text width=3cm, MainPurple}
]

% فازها
\node[phase, fill=MainRed!80] (crisis) {بحران و فشار\\(۱۹۸۵-۱۹۸۹)};
\node[phase, fill=MainOrange!80, right=of crisis] (opening) {گشایش\\(فوریه ۱۹۹۰)};
\node[phase, fill=MainYellow!90!black, right=of opening] (negotiation) {مذاکره\\(\lr{CODESA}\\۱۹۹۱-۱۹۹۳)};
\node[phase, fill=MainGreen!80, below=2cm of crisis] (election) {انتخابات\\(آوریل ۱۹۹۴)};
\node[phase, fill=MainBlue!80, right=of election] (consolidation) {تحکیم\\(۱۹۹۴-۱۹۹۹)};
\node[phase, fill=MainPurple!80, right=of consolidation] (democracy) {دموکراسی\\(۱۹۹۹+)};

% فلش‌ها
\draw[arrow] (crisis) -- (opening);
\draw[arrow] (opening) -- (negotiation);
\draw[arrow] (negotiation) -- (election);
\draw[arrow] (election) -- (consolidation);
\draw[arrow] (consolidation) -- (democracy);

% بحران‌ها (نقاط خطر)
\node[label, MainRed] at ($(opening)!0.5!(negotiation)+(0,1.2)$) {بحران: \lr{CODESA II}\\بن‌بست + بویپاتونگ};
\node[label, MainRed] at ($(negotiation)!0.5!(election)+(0,-0.8)$) {بحران: ترور هانی\\(آوریل ۱۹۹۳)};

% معادل ایرانی
\node[iranlabel] at ($(crisis)+(0,-1.2)$) {ایران: فاز ۰\\(پیش‌گذار)};
\node[iranlabel] at ($(opening)+(0,-1.2)$) {ایران: فاز ۱\\(تثبیت)};
\node[iranlabel] at ($(negotiation)+(0,-1.2)$) {ایران: فاز ۲\\(نهادسازی)};
\node[iranlabel] at ($(election)+(0,1.2)$) {ایران: فاز ۲\\(انتخابات)};
\node[iranlabel] at ($(consolidation)+(0,1.2)$) {ایران: فاز ۳\\(تحکیم)};
\node[iranlabel] at ($(democracy)+(0,1.2)$) {ایران: فاز ۴\\(خروج نظارت)};
\node[label, DarkGray] at ($(crisis)+(-1.5,-2.5)$) {تحریم‌ها\\اعتراضات\\فروپاشی شوروی};
\node[label, DarkGray] at ($(opening)+(0,-2.5)$) {آزادی ماندلا\\رفع ممنوعیت \lr{ANC}\\لغو قوانین آپارتاید};
\node[label, DarkGray] at ($(negotiation)+(1.5,-2.5)$) {۱۹ حزب\\بندهای غروب\\ق.ا.\ موقت};

\end{tikzpicture}
\caption{مسیر گذار آفریقای جنوبی و فازهای معادل ایرانی}
\label{fig:app-sa-path}
\end{figure}

\sectiondivider

%═══════════════════════════════════════════════════════════
\section{تحلیل مقایسه‌ای تفصیلی: آفریقای جنوبی و ایران}
\label{app:sa:comparison}
%═══════════════════════════════════════════════════════════

\subsection{ماتریس مشابهت‌ها و تفاوت‌ها}

\begin{table}[htbp]
\centering
\caption{ماتریس تفصیلی مشابهت‌ها و تفاوت‌های آفریقای جنوبی و ایران}
\label{tab:app-sa-iran-comparison}
\begin{tabularx}{\textwidth}{
  >{\raggedleft\arraybackslash}p{2.5cm}
  >{\raggedleft\arraybackslash}X
  >{\raggedleft\arraybackslash}X
  >{\centering\arraybackslash}p{1.5cm}
}
\toprule
\headerrow \textbf{بُعد} & \textbf{آفریقای جنوبی} & \textbf{ایران} & \textbf{مشابهت} \\
\midrule
ایدئولوژی رسمی & نژادپرستی (\lr{Apartheid}) & ولایت فقیه + تبعیض مذهبی-جنسیتی & \rating{4} \\
\altrow
جمعیت & ۴۰ میلیون & ۸۵+ میلیون & \rating{2} \\
تنوع قومی-زبانی & بسیار بالا (۱۱ زبان) & بالا (۸+ قوم اصلی) & \rating{4} \\
\altrow
تحریم‌های بین‌المللی & جامع و مؤثر & جامع (هسته‌ای + حقوق بشر) & \rating{5} \\
نیروهای امنیتی & \lr{SADF} + پلیس + نیروهای ویژه & سپاه + ارتش + بسیج + اطلاعات & \rating{4} \\
\altrow
منافع اقتصادی نظامیان & \lr{ARMSCOR} + صنایع دفاعی & خاتم‌الانبیاء + بنیادها + قاچاق & \rating{4} \\
برنامهٔ هسته‌ای & ۶ کلاهک (خلع ۱۹۹۳) & غنی‌سازی ۶۰٪+ (فعال) & \rating{3} \\
\altrow
اپوزیسیون سازمان‌یافته & \lr{ANC}: ۱۰۰+ سال، ۳M عضو & پراکنده، فاقد تشکل فراگیر & \rating{1} \\
دیاسپورا & محدود & بسیار بزرگ (۴-۵M) & \rating{2} \\
\altrow
جامعهٔ مدنی & قوی (\lr{UDF, COSATU}) & قوی اما سرکوب‌شده & \rating{3} \\
رهبر کاریزماتیک & ماندلا (نماد جهانی) & فاقد (چالش اصلی) & \rating{1} \\
\altrow
موقعیت ژئوپلیتیکی & منطقه‌ای مهم اما نه حساس & بسیار حساس (هرمز، هسته‌ای، همسایگان) & \rating{2} \\
نقش مذهب در سیاست & محدود (کلیسا حامی آشتی) & مرکزی (دین = قدرت) & \rating{2} \\
\altrow
سابقهٔ دموکراتیک & محدود (سفیدپوستان) & تجربهٔ مشروطه + مصدق (محدود) & \rating{3} \\
\midrule
\headerrow \multicolumn{3}{l}{\textbf{میانگین مشابهت کلی}} & \textbf{\rating{3}} \\
\bottomrule
\end{tabularx}
\end{table}

\subsection{پنج تفاوت حیاتی که تعدیل مدل را ضروری می‌سازد}

\begin{warningbox}
\textbf{پنج تفاوت حیاتی ایران با آفریقای جنوبی} که مانع کپی‌برداری مستقیم می‌شود:

\begin{enumerate}[nosep]
\item \textbf{فقدان ماندلا:} آفریقای جنوبی رهبری کاریزماتیک با مشروعیت جهانی و ۲۷ سال زندان داشت. ایران فاقد چنین شخصیتی است. \emphred{راه‌حل:} جایگزینی رهبر فردی با \textbf{ائتلاف فراگیر} و \textbf{نهادسازی} (رهبری جمعی).

\item \textbf{فقدان \lr{ANC}:} \lr{ANC} تشکیلاتی ۱۰۰+ ساله با میلیون‌ها عضو بود. اپوزیسیون ایرانی پراکنده و متفرق است. \emphred{راه‌حل:} تشکیل \textbf{پلتفرم هماهنگی اپوزیسیون} با ساختار فدرالی (نه تک‌حزبی) قبل از گذار.

\item \textbf{بُعد هسته‌ای فعال:} آفریقای جنوبی هسته‌ای را \textbf{قبل} از مذاکره خلع کرد. ایران برنامهٔ هسته‌ای فعال دارد که ابزار چانه‌زنی است. \emphred{راه‌حل:} گره زدن خلع هسته‌ای به \textbf{بستهٔ جامع مشوق‌ها} (تحریم + سرمایه‌گذاری + امنیت).

\item \textbf{ژئوپلیتیک حساس‌تر:} آفریقای جنوبی در منطقه‌ای نسبتاً باثبات بود. ایران ۱۵ همسایه دارد، تنگهٔ هرمز را کنترل می‌کند، و در ۴ جنگ نیابتی درگیر است. \emphred{راه‌حل:} \textbf{گروه تماس بین‌المللی} گسترده‌تر و حضور دیپلماتیک فعال‌تر.

\item \textbf{دین = قدرت:} در آفریقای جنوبی کلیسا نقش \textbf{میانجی‌گر} داشت. در ایران دین \textbf{ابزار قدرت} است و جدایی دین از دولت چالش اصلی خواهد بود. \emphred{راه‌حل:} تأکید بر \textbf{آزادی مذهب} (نه ضدیت با دین) و جلب حمایت روحانیون مستقل.
\end{enumerate}
\end{warningbox}

\subsection{پنج عنصر مستقیماً قابل‌انتقال}

\begin{recommendation}
\textbf{پنج عنصر از مدل آفریقای جنوبی} که مستقیماً به ایران قابل‌انتقال است:

\begin{enumerate}[nosep]
\item \textbf{بندهای غروب (\lr{Sunset Clauses}):} تضمین‌های موقت ۵ ساله برای کاهش ترس بخش‌های همکاری‌کنندهٔ نظام فعلی — امنیت شغلی کارمندان + عدم مصادره + مشارکت در دولت انتقالی. بدون این بندها، مذاکره آغاز نخواهد شد.

\item \textbf{مدل \lr{CODESA} (کنفرانس ملی فراگیر):} تشکیل «کنفرانس ملی ایران» با حضور همهٔ جریان‌ها — ملی-مذهبی، جمهوری‌خواه، فدرالیست، زنان، اقوام، جوانان، و حتی بخش‌هایی از نظام فعلی. قاعدهٔ «اجماع کافی» (نه اتفاق آرا).

\item \textbf{قانون اساسی دو مرحله‌ای:} ابتدا منشور موقت (اصول بنیادین غیرقابل‌تغییر) در فاز ۱-۲، سپس قانون اساسی دائمی توسط مجلس مؤسسان منتخب در فاز ۲-۳. دادگاه قانون اساسی بررسی انطباق کند.

\item \textbf{مدل \lr{TRC} با تعدیل:} «کمیسیون حقیقت و کرامت ایران» با مکانیزم عفو مشروط + پخش زنده + حقوق قربانیان + \textbf{افزودن بُعد اقتصادی} (برخلاف آفریقای جنوبی که این بُعد را نادیده گرفت).

\item \textbf{ادغام نیروهای مسلح (نه انحلال):} ادغام سپاه و ارتش در نیروی دفاعی ملی واحد + نظارت مدنی + بازنشستگی افتخاری + \textbf{تفکیک اقتصادی} (ترکیب مدل آفریقای جنوبی و اندونزی).
\end{enumerate}
\end{recommendation}

\sectiondivider

%═══════════════════════════════════════════════════════════
\section{نمودار: شبکهٔ انتقال درس‌آموخته‌ها}
\label{app:sa:network}
%═══════════════════════════════════════════════════════════

\begin{figure}[htbp]
\centering
\begin{tikzpicture}[
  node distance=2cm,
  sanode/.style={
    draw=MainGreen, fill=MainGreen!10, rounded corners=5pt,
    minimum width=2.8cm, minimum height=1cm, font=\small,
    align=center, thick
  },
  irannode/.style={
    draw=MainPurple, fill=MainPurple!10, rounded corners=5pt,
    minimum width=2.8cm, minimum height=1cm, font=\small,
    align=center, thick
  },
  transferarrow/.style={->, thick, >=stealth, MainOrange, dashed},
  directarrow/.style={->, very thick, >=stealth, MainGreen}
]

% گره‌های آفریقای جنوبی (بالا)
\node[sanode] (codesa) at (0,4) {\lr{CODESA}\\مذاکرهٔ فراگیر};
\node[sanode] (sunset) at (4,4) {بندهای غروب\\تضمین موقت};
\node[sanode] (trc) at (8,4) {\lr{TRC}\\حقیقت+عفو};
\node[sanode] (sandf) at (0,2) {\lr{SANDF}\\ادغام نیروها};
\node[sanode] (const) at (4,2) {قانون اساسی\\فراگیر};
\node[sanode] (nuclear) at (8,2) {خلع هسته‌ای\\داوطلبانه};

% عنوان آفریقای جنوبی
\node[font=\bfseries\small, MainGreen] at (4,5.3) {درس‌آموخته‌های آفریقای جنوبی};

% گره‌های ایران (پایین)
\node[irannode] (iran-conf) at (0,-1) {کنفرانس ملی\\ایران};
\node[irannode] (iran-sunset) at (4,-1) {تضمین‌ها برای\\همکاری‌کنندگان};
\node[irannode] (iran-trc) at (8,-1) {کمیسیون حقیقت\\و کرامت ایران};
\node[irannode] (iran-army) at (0,-3) {ارتش ملی\\واحد};
\node[irannode] (iran-const) at (4,-3) {مجلس مؤسسان\\+ منشور حقوق};
\node[irannode] (iran-nuke) at (8,-3) {توافق هسته‌ای\\جامع};

% عنوان ایران
\node[font=\bfseries\small, MainPurple] at (4,-4.3) {کاربردهای ایرانی (مدل ۶)};

% فلش‌های انتقال مستقیم
\draw[directarrow] (codesa) -- (iran-conf);
\draw[directarrow] (sunset) -- (iran-sunset);
\draw[directarrow] (trc) -- (iran-trc);
\draw[directarrow] (sandf) -- (iran-army);
\draw[directarrow] (const) -- (iran-const);

% فلش انتقال با تعدیل (هسته‌ای)
\draw[transferarrow] (nuclear) -- node[right, font=\tiny, MainOrange] {نیاز به تعدیل} (iran-nuke);

% خط جداکننده
\draw[gray!30, thick] (-2,0.5) -- (10,0.5);
\node[font=\tiny, gray] at (10.8,0.5) {خط انتقال};

\end{tikzpicture}
\caption{شبکهٔ انتقال درس‌آموخته‌ها: از آفریقای جنوبی به مدل ایران}
\label{fig:app-sa-transfer}
\end{figure}

\sectiondivider

%═══════════════════════════════════════════════════════════
\section{مصاحبه‌ها و منابع اصلی}
\label{app:sa:sources}
%═══════════════════════════════════════════════════════════

\subsection{منابع کلیدی دربارهٔ گذار آفریقای جنوبی}

\begin{enumerate}[nosep]
\item \textbf{ماندلا، نلسون} (۱۹۹۴). \textit{\lr{Long Walk to Freedom}}. لیتل براون. — خودزندگی‌نامهٔ ماندلا با جزئیات مذاکرات.

\item \textbf{اسپارکز، آلیستر} (۱۹۹۵). \textit{\lr{Tomorrow is Another Country: The Inside Story of South Africa's Negotiated Revolution}}. مندارین. — بهترین روایت ژورنالیستی از مذاکرات.

\item \textbf{توتو، دزموند} (۱۹۹۹). \textit{\lr{No Future Without Forgiveness}}. دابلدی. — فلسفهٔ \lr{TRC} از زبان رئیس آن.

\item \textbf{گزارش نهایی \lr{TRC}} (۱۹۹۸-۲۰۰۳). ۷ جلد. قابل‌دسترسی: \lr{justice.gov.za/trc}

\item \textbf{اسلوو، جو} (۱۹۹۲). «\lr{Negotiations: What Room for Compromise?}» — مقالهٔ تاریخی دربارهٔ بندهای غروب.

\item \textbf{واله، فردریک ون زیل} (۲۰۱۲). \textit{\lr{The Last Trek — A New Beginning}}. مکمیلان. — دیدگاه دکلرک.

\item \textbf{ساسکسمن، مارک} (۲۰۱۶). \textit{\lr{The State Capture Report}}. — تحلیل فساد دورهٔ زوما.

\item \textbf{لاند، کریس} (۲۰۰۶). «\lr{War and Peace in the Democratic Republic of Congo and South Africa}». \textit{\lr{Strategic Review for Southern Africa}}. — مقایسهٔ مدل‌های \lr{SSR}.
\end{enumerate}

\subsection{شخصیت‌های کلیدی برای مشاوره}

\begin{casestudy}
\textbf{کارشناسان آفریقای جنوبی که تیم آماده‌سازی ایران باید مشورت کند:}

\begin{itemize}[nosep]
\item \person{سیریل رامافوزا}{\lr{Cyril Ramaphosa}}: رئیس تیم مذاکره‌کنندهٔ \lr{ANC} در \lr{CODESA} (اکنون رئیس‌جمهور) — تجربهٔ مذاکره
\item \person{آلبی ساکس}{\lr{Albie Sachs}}: قاضی دادگاه قانون اساسی — تجربهٔ تدوین منشور حقوق
\item \person{الکس بورین}{\lr{Alex Boraine}}: معاون رئیس \lr{TRC} — تجربهٔ عدالت انتقالی (مؤسس \lr{ICTJ})
\item \person{ویلی استورهوف}{\lr{Willy Esterhuyse}}: واسطهٔ مذاکرات محرمانه — تجربهٔ کانال‌های پنهان
\item \person{هنری باردن}{\lr{Henry Memory Barton}}: مدیر ادغام \lr{SANDF} — تجربهٔ \lr{SSR}
\item \person{یاسمین سوکا}{\lr{Yasmin Sooka}}: کمیسیونر \lr{TRC} — تجربهٔ حقوق زنان در عدالت انتقالی
\end{itemize}

\textbf{توصیه:} در فاز ۰ (پیش‌گذار)، تشکیل \textbf{گروه مشاورهٔ آفریقای جنوبی} (\lr{South Africa Advisory Group}) متشکل از ۵-۱۰ کارشناس فوق برای انتقال تجربه به تیم ایرانی (\seeChapter{ch:timeline}).
\end{casestudy}

\sectiondivider

%═══════════════════════════════════════════════════════════
\section{جمع‌بندی پیوست}
\label{app:sa:conclusion}
%═══════════════════════════════════════════════════════════

\begin{chaptersummary}
جمع‌بندی پیوست ب — مطالعهٔ موردی آفریقای جنوبی:

\begin{enumerate}[nosep]
\item آفریقای جنوبی با امتیاز \textbf{۳۶ از ۴۰} جامع‌ترین الگوی قابل‌انتقال به ایران است.
\item پنج عنصر مستقیماً قابل‌انتقال: \lr{CODESA}، بندهای غروب، قانون اساسی فراگیر، \lr{TRC}، و ادغام نیروهای مسلح.
\item پنج تفاوت حیاتی (فقدان ماندلا، فقدان \lr{ANC}، هسته‌ای فعال، ژئوپلیتیک حساس‌تر، دین=قدرت) کپی‌برداری مستقیم را ناممکن و \textbf{تعدیل هوشمند} را ضروری می‌سازد.
\item مهم‌ترین نوآوری آفریقای جنوبی: \textbf{بندهای غروب} (تضمین موقت برای رژیم پیشین) که مذاکره را ممکن ساخت.
\item ضعف اصلی آفریقای جنوبی: \textbf{عدالت اقتصادی ناکافی} — ایران باید از این شکست بیاموزد و بُعد اقتصادی را از روز اول در مدل عدالت انتقالی بگنجاند.
\item \lr{TRC} الگوی اصلی عدالت انتقالی برای ایران است، اما باید با \textbf{بُعد اقتصادی} (غرامت واقعی) و \textbf{محاکمهٔ تکمیلی} (برای موارد عفو‌نشده) ارتقا یابد.
\item نقش بین‌المللی در آفریقای جنوبی \textbf{حمایتی} بود (مدل ۳) نه مدیریت مستقیم — این دقیقاً مبنای مدل ۶ پیشنهادی است.
\item تشکیل \textbf{گروه مشاورهٔ آفریقای جنوبی} در فاز ۰ پیش‌گذار توصیه می‌شود.
\end{enumerate}

\vspace{0.3cm}
\textit{مطالعهٔ تکمیلی:}
\begin{itemize}[nosep]
\item مقایسهٔ جامع ۹ نمونه: \seeChapter{app:comparison}
\item تضمین‌های موفقیت: \seeChapter{ch:guarantees}
\item عدالت انتقالی و \lr{SSR}: \seeChapter{ch:requirements}
\item سناریوی مذاکره‌ای (\lr{B}): \seeChapter{ch:scenarios}
\item شیلی (الگوی مکمل): \seeChapter{app:chile}
\end{itemize}
\end{chaptersummary}

\chapterend

%══════════════════════════════════════════════════════════════
% پایان پیوست ب
%══════════════════════════════════════════════════════════════