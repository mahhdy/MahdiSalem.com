%══════════════════════════════════════════════════════════════
% پیوست د: فهرست نهادها و سازمان‌های کلیدی
% فایل: appendices/app-j-organizations.tex
% حجم هدف: ۵-۷ صفحه
%══════════════════════════════════════════════════════════════

\chapter{فهرست نهادها و سازمان‌های کلیدی}
\label{app:organizations}

\begin{executivesummary}
این پیوست فهرست جامعی از \textbf{نهادها، سازمان‌ها و نهادهای بین‌المللی} کلیدی مرتبط با فرآیند نظارت بر گذار دموکراتیک ایران ارائه می‌دهد. بیش از \textbf{۸۰ نهاد} در ۹ دسته طبقه‌بندی شده‌اند. هر نهاد با نام فارسی، مخفف انگلیسی، مأموریت مختصر، وب‌سایت، و نقش احتمالی در پروندهٔ ایران معرفی شده است. این فهرست مکمل فصل ۵ (نهادها و بازیگران) است (\seeChapter{ch:actors}).
\end{executivesummary}

%═══════════════════════════════════════════════════════════
\section{نهادهای سازمان ملل متحد}
\label{app:org:un}
%═══════════════════════════════════════════════════════════

\begin{longtable}{>{\raggedleft\arraybackslash}p{3.5cm} >{\centering\arraybackslash}p{1.5cm} >{\raggedleft\arraybackslash}p{5cm} >{\raggedleft\arraybackslash}p{4cm}}
\caption{نهادهای کلیدی سازمان ملل}
\label{tab:org-un} \\

\toprule
\headerrow \textbf{نهاد} & \textbf{مخفف} & \textbf{مأموریت} & \textbf{نقش در ایران} \\
\midrule
\endfirsthead

\multicolumn{4}{c}{\small\textit{ادامهٔ جدول نهادهای سازمان ملل}} \\
\toprule
\headerrow \textbf{نهاد} & \textbf{مخفف} & \textbf{مأموریت} & \textbf{نقش در ایران} \\
\midrule
\endhead

\bottomrule
\endlastfoot

شورای امنیت & \lr{UNSC} & حفظ صلح و امنیت بین‌المللی & قطعنامهٔ تأسیس مأموریت + مجوز \\
\altrow مجمع عمومی & \lr{UNGA} & تصمیم‌گیری عمومی و بودجه & مشروعیت‌بخشی + بودجه \\
دبیرخانه — دپارتمان صلح & \lr{DPPA/DPO} & مأموریت‌های صلح و سیاسی & طراحی و مدیریت \lr{UNMOIT} \\
\altrow \termfn{کمیساریای عالی حقوق بشر}{\lr{OHCHR}} & \lr{OHCHR} & حمایت از حقوق بشر & نظارت حقوق بشری \\
\termfn{برنامهٔ توسعه}{\lr{UNDP}} & \lr{UNDP} & توسعهٔ پایدار + حکمرانی & ظرفیت‌سازی نهادی \\
\altrow \termfn{کمیساریای پناهندگان}{\lr{UNHCR}} & \lr{UNHCR} & حمایت از پناهندگان و آوارگان & بازگشت آوارگان + دیاسپورا \\
\termfn{صندوق کودکان}{\lr{UNICEF}} & \lr{UNICEF} & حقوق کودکان & حمایت از کودکان آسیب‌دیده \\
\altrow \termfn{زنان سازمان ملل}{\lr{UN Women}} & \lr{UN Women} & برابری جنسیتی & نظارت بر سهمیهٔ زنان \\
\termfn{آژانس بین‌المللی انرژی اتمی}{\lr{IAEA}} & \lr{IAEA} & بازرسی هسته‌ای & نظارت بر برنامهٔ هسته‌ای \\
\altrow دیوان بین‌المللی دادگستری & \lr{ICJ} & حل اختلافات بین دولت‌ها & داوری حقوقی \\
دیوان بین‌المللی کیفری & \lr{ICC} & محاکمهٔ جنایات بین‌المللی & صلاحیت تکمیلی (جنایات) \\
\altrow شورای حقوق بشر & \lr{HRC} & بررسی وضعیت حقوق بشر کشورها & گزارشگر ویژهٔ ایران \\
\termfn{دفتر هماهنگی امور انسانی}{\lr{OCHA}} & \lr{OCHA} & هماهنگی کمک‌های انسان‌دوستانه & مدیریت بحران انسانی \\
\altrow \termfn{برنامهٔ جهانی غذا}{\lr{WFP}} & \lr{WFP} & امنیت غذایی & تأمین غذا در فاز بحران \\

\end{longtable}

%═══════════════════════════════════════════════════════════
\section{سازمان‌های منطقه‌ای}
\label{app:org:regional}
%═══════════════════════════════════════════════════════════

\begin{longtable}{>{\raggedleft\arraybackslash}p{3.5cm} >{\centering\arraybackslash}p{1.5cm} >{\raggedleft\arraybackslash}p{5cm} >{\raggedleft\arraybackslash}p{4cm}}
\caption{سازمان‌های منطقه‌ای مرتبط}
\label{tab:org-regional} \\

\toprule
\headerrow \textbf{نهاد} & \textbf{مخفف} & \textbf{مأموریت} & \textbf{نقش احتمالی در ایران} \\
\midrule
\endfirsthead
\bottomrule
\endlastfoot

اتحادیهٔ اروپا & \lr{EU} & ادغام اروپایی + سیاست خارجی & تحریم/مشوق + ناظران + کمک مالی \\
\altrow سازمان امنیت و همکاری اروپا & \lr{OSCE} & امنیت اروپا + نظارت انتخاباتی (\lr{ODIHR}) & ناظران انتخاباتی متخصص \\
شورای اروپا & \lr{CoE} & حقوق بشر + دموکراسی + حاکمیت قانون & کمیسیون ونیز (مشاوره قانون اساسی) \\
\altrow اتحادیهٔ آفریقا & \lr{AU} & صلح و همکاری آفریقا & تجربهٔ نظارت (الگوی آفریقای جنوبی) \\
سازمان همکاری اسلامی & \lr{OIC} & همکاری کشورهای اسلامی & مشروعیت‌بخشی منطقه‌ای \\
\altrow اتحادیهٔ عرب & \lr{AL} & همکاری کشورهای عربی & همسایگان عرب ایران \\
سازمان همکاری شانگهای & \lr{SCO} & همکاری اوراسیا (چین + روسیه) & بازیگر حساس (احتمالاً مقاوم) \\
\altrow اکو & \lr{ECO} & همکاری اقتصادی منطقه‌ای & چارچوب همکاری با همسایگان \\

\end{longtable}

%═══════════════════════════════════════════════════════════
\section{دولت‌های کلیدی}
\label{app:org:states}
%═══════════════════════════════════════════════════════════

\begin{longtable}{>{\raggedleft\arraybackslash}p{3cm} >{\raggedleft\arraybackslash}p{4.5cm} >{\raggedleft\arraybackslash}p{3.5cm} >{\centering\arraybackslash}p{2.5cm}}
\caption{دولت‌های کلیدی و نقش احتمالی}
\label{tab:org-states} \\

\toprule
\headerrow \textbf{کشور} & \textbf{نقش احتمالی} & \textbf{حساسیت/ریسک} & \textbf{اهمیت} \\
\midrule
\endfirsthead
\bottomrule
\endlastfoot

ایالات متحده & لغو تحریم + کمک مالی + فشار & تاریخچهٔ ۱۹۵۳ + بی‌اعتمادی & \rating{5} \\
\altrow بریتانیا & دیپلماسی + \lr{BBC} فارسی + حقوق & تاریخچهٔ استعماری & \rating{4} \\
فرانسه & \lr{EU} + شورای امنیت + فرهنگ & -- & \rating{3} \\
\altrow آلمان & بزرگ‌ترین شریک تجاری اروپایی & ملاحظات اقتصادی & \rating{4} \\
چین & شورای امنیت (وتو) + نفت & احتمالاً مقاوم/معارض & \rating{5} \\
\altrow روسیه & شورای امنیت (وتو) + سلاح & احتمالاً مقاوم/معارض & \rating{5} \\
ترکیه & همسایه + مدل + ناتو & رقابت منطقه‌ای & \rating{4} \\
\altrow عربستان سعودی & رقیب منطقه‌ای + نفت & رقابت ژئوپلیتیکی & \rating{4} \\
امارات & سرمایه‌گذاری + لجستیک & رقابت/فرصت & \rating{3} \\
\altrow عراق & همسایهٔ شیعه + تجربهٔ گذار & بی‌ثباتی + نفوذ ایران & \rating{4} \\
هند & همسایهٔ منطقه‌ای + انرژی & ملاحظات نفتی & \rating{3} \\
\altrow ژاپن + کره & کمک مالی + فناوری & فاصلهٔ جغرافیایی & \rating{3} \\
نروژ + سوئد & میانجی‌گری + مدل & تجربهٔ صلح + مالی & \rating{3} \\
\altrow استرالیا & تجربهٔ تیمور + ظرفیت & فاصلهٔ جغرافیایی & \rating{2} \\

\end{longtable}

%═══════════════════════════════════════════════════════════
\section{سازمان‌های غیردولتی بین‌المللی (\lr{INGO})}
\label{app:org:ngo}
%═══════════════════════════════════════════════════════════

\begin{longtable}{>{\raggedleft\arraybackslash}p{3.5cm} >{\centering\arraybackslash}p{1.5cm} >{\raggedleft\arraybackslash}p{5cm} >{\raggedleft\arraybackslash}p{4cm}}
\caption{سازمان‌های غیردولتی بین‌المللی کلیدی}
\label{tab:org-ngo} \\

\toprule
\headerrow \textbf{نهاد} & \textbf{مخفف} & \textbf{حوزهٔ فعالیت} & \textbf{نقش در ایران} \\
\midrule
\endfirsthead
\bottomrule
\endlastfoot

\termfn{دیده‌بان حقوق بشر}{\lr{Human Rights Watch}} & \lr{HRW} & مستندسازی نقض حقوق بشر & نظارت حقوق بشری \\
\altrow عفو بین‌الملل & \lr{AI} & حقوق بشر + دادخواهی & حمایت از زندانیان سیاسی \\
\termfn{مرکز بین‌المللی عدالت انتقالی}{\lr{ICTJ}} & \lr{ICTJ} & مشاوره در عدالت انتقالی & طراحی کمیسیون حقیقت \\
\altrow \termfn{گروه بحران بین‌المللی}{\lr{ICG}} & \lr{ICG} & تحلیل بحران + پیشگیری & تحلیل سناریوها + هشدار \\
مرکز کارتر & \lr{Carter Center} & نظارت انتخاباتی + میانجی‌گری & ناظران انتخاباتی \\
\altrow مؤسسهٔ صلح آمریکا & \lr{USIP} & تحقیق صلح + میانجی‌گری & مشاوره فرآیند صلح \\
بنیاد بین‌المللی نظام‌های انتخاباتی & \lr{IFES} & فنی انتخاباتی & طراحی سیستم انتخاباتی \\
\altrow \termfn{مؤسسهٔ ملی دموکراسی}{\lr{NDI}} & \lr{NDI} & آموزش دموکراسی + احزاب & ظرفیت‌سازی حزبی \\
مؤسسهٔ بین‌المللی جمهوری‌خواهان & \lr{IRI} & حمایت از دموکراسی & آموزش + نظارت \\
\altrow ایده & \lr{IDEA} & دموکراسی + انتخابات & مشاوره قانون اساسی \\
گزارشگران بدون مرز & \lr{RSF} & آزادی مطبوعات & نظارت رسانه‌ای \\
\altrow شفافیت بین‌الملل & \lr{TI} & ضد فساد & نظارت مالی + ضد فساد \\
\termfn{کمیتهٔ بین‌المللی صلیب سرخ}{\lr{ICRC}} & \lr{ICRC} & حقوق بشردوستانه & بازدید از زندان‌ها + بازگشت \\
\altrow پزشکان بدون مرز & \lr{MSF} & بهداشت در بحران & خدمات پزشکی فاز ۱ \\

\end{longtable}

%═══════════════════════════════════════════════════════════
\section{نهادهای مالی بین‌المللی}
\label{app:org:ifi}
%═══════════════════════════════════════════════════════════

\begin{longtable}{>{\raggedleft\arraybackslash}p{3.5cm} >{\centering\arraybackslash}p{1.5cm} >{\raggedleft\arraybackslash}p{5cm} >{\raggedleft\arraybackslash}p{4cm}}
\caption{نهادهای مالی بین‌المللی}
\label{tab:org-ifi} \\

\toprule
\headerrow \textbf{نهاد} & \textbf{مخفف} & \textbf{مأموریت} & \textbf{نقش در ایران} \\
\midrule
\endfirsthead
\bottomrule
\endlastfoot

بانک جهانی & \lr{WB} & توسعه + کاهش فقر & بازسازی اقتصادی + وام \\
\altrow صندوق بین‌المللی پول & \lr{IMF} & ثبات مالی + اصلاحات & مشاوره اقتصاد کلان \\
بانک توسعهٔ اسلامی & \lr{IsDB} & توسعه در جهان اسلام & تأمین مالی + مشروعیت \\
\altrow بانک سرمایه‌گذاری اروپا & \lr{EIB} & سرمایه‌گذاری توسعه‌ای & زیرساخت \\
سازمان تجارت جهانی & \lr{WTO} & تجارت آزاد & عضویت = مشوق اقتصادی \\
\altrow بانک بازسازی و توسعهٔ اروپا & \lr{EBRD} & بازسازی + خصوصی‌سازی & سرمایه‌گذاری + ظرفیت‌سازی \\

\end{longtable}

%═══════════════════════════════════════════════════════════
\section{نهادهای ایرانی (موجود / پیشنهادی)}
\label{app:org:iranian}
%═══════════════════════════════════════════════════════════

\begin{longtable}{>{\raggedleft\arraybackslash}p{4cm} >{\centering\arraybackslash}p{1.8cm} >{\raggedleft\arraybackslash}p{4.5cm} >{\raggedleft\arraybackslash}p{3.5cm}}
\caption{نهادهای ایرانی موجود و پیشنهادی}
\label{tab:org-iranian} \\

\toprule
\headerrow \textbf{نهاد} & \textbf{وضعیت} & \textbf{نقش} & \textbf{فصل مرتبط} \\
\midrule
\endfirsthead
\bottomrule
\endlastfoot

ارتش ملی واحد & پیشنهادی & ادغام سپاه + ارتش تحت نظارت مدنی & ۶، ب \\
\altrow کمیسیون حقیقت و کرامت ایران & پیشنهادی & حقیقت‌یابی + عفو مشروط + بُعد اقتصادی & ۶، ب، ت \\
مجلس مؤسسان & پیشنهادی & تدوین قانون اساسی جدید & ۹ \\
\altrow کمیسیون ملی انتخابات & پیشنهادی & مدیریت مستقل انتخابات & ۸ \\
مؤسسهٔ حافظهٔ ملی ایران & پیشنهادی & حفظ آرشیو + تحقیق تاریخی & ث \\
\altrow دادگاه ویژهٔ رسیدگی به جرایم رژیم & پیشنهادی & محاکمهٔ فرماندهان ارشد & ۸ \\
نهاد ضد فساد مستقل & پیشنهادی & نظارت مالی + شفافیت & ۸ \\
\altrow کمیسیون مستقل رسانه & پیشنهادی & آزادی مطبوعات + نظارت & ۸ \\
شورای مشورتی ملی & پیشنهادی & مشارکت جامعهٔ مدنی & ۹ \\
\altrow صندوق امانی بازسازی ایران & پیشنهادی & مدیریت کمک‌های مالی & ۱۰ \\
چهارگانهٔ ایرانی & پیشنهادی & میانجی‌گری (الگوی تونس) & ت \\
\altrow کانون وکلای دادگستری & موجود & نظارت حقوقی + دفاع از حقوق & ت \\
تشکل‌های صنفی مستقل & موجود (سرکوب) & ائتلاف‌سازی + فشار مدنی & ث \\
\altrow جنبش زن، زندگی، آزادی & موجود & حقوق زنان + فراگیری & ۲، ۶ \\

\end{longtable}

%═══════════════════════════════════════════════════════════
\section{مراکز تحقیقاتی و آکادمیک}
\label{app:org:academic}
%═══════════════════════════════════════════════════════════

\begin{longtable}{>{\raggedleft\arraybackslash}p{4cm} >{\raggedleft\arraybackslash}p{4cm} >{\raggedleft\arraybackslash}p{6cm}}
\caption{مراکز تحقیقاتی کلیدی}
\label{tab:org-academic} \\

\toprule
\headerrow \textbf{نهاد} & \textbf{حوزه} & \textbf{ارتباط با ایران} \\
\midrule
\endfirsthead
\bottomrule
\endlastfoot

\lr{V-Dem Institute} (گوتنبرگ) & شاخص‌های دموکراسی & سنجش پیشرفت ایران \\
\altrow \lr{Freedom House} & آزادی سیاسی و مدنی & رتبه‌بندی ایران \\
\lr{Chatham House} (لندن) & سیاست خارجی + ایران & برنامهٔ خاورمیانه + ایران \\
\altrow \lr{Brookings Institution} & سیاست عمومی & برنامهٔ ایران و خاورمیانه \\
\lr{Carnegie Endowment} & صلح بین‌المللی & تحلیل سیاسی ایران \\
\altrow \lr{RAND Corporation} & تحلیل راهبردی & سناریوسازی امنیتی \\
دانشگاه‌های \lr{SOAS + Oxford + Harvard} & مطالعات ایرانی & پژوهش + آموزش نخبگان \\
\altrow \lr{Institute for War \& Peace Reporting} & رسانه در بحران & آموزش خبرنگاران ایرانی \\
مؤسسهٔ هاینریش بُل (آلمان) & دموکراسی + محیط‌زیست & حمایت از جامعهٔ مدنی \\
\altrow صندوق ملی دموکراسی (\lr{NED}) & حمایت از دموکراسی & تأمین مالی نهادهای مدنی ایرانی \\

\end{longtable}

%═══════════════════════════════════════════════════════════
\section{رسانه‌های بین‌المللی فارسی‌زبان}
\label{app:org:media}
%═══════════════════════════════════════════════════════════

\begin{longtable}{>{\raggedleft\arraybackslash}p{3.5cm} >{\raggedleft\arraybackslash}p{3cm} >{\raggedleft\arraybackslash}p{4cm} >{\raggedleft\arraybackslash}p{3.5cm}}
\caption{رسانه‌های کلیدی فارسی‌زبان و نقش احتمالی}
\label{tab:org-media} \\

\toprule
\headerrow \textbf{رسانه} & \textbf{مالکیت} & \textbf{مخاطب} & \textbf{نقش احتمالی} \\
\midrule
\endfirsthead
\bottomrule
\endlastfoot

\lr{BBC Persian} & بریتانیا & ایران + دیاسپورا & پوشش خبری + اطلاع‌رسانی \\
\altrow \lr{Iran International} & خصوصی (لندن) & ایران + دیاسپورا & پوشش خبری \\
\lr{VOA Persian} & آمریکا & ایران + دیاسپورا & اطلاع‌رسانی \\
\altrow \lr{Radio Farda / RFE} & آمریکا & ایران & پوشش رادیویی \\
\lr{DW Farsi} & آلمان & ایران + دیاسپورا & اطلاع‌رسانی بی‌طرف \\
\altrow \lr{France 24 Farsi} & فرانسه & ایران + دیاسپورا & تنوع منابع \\
رسانه‌های مستقل ایرانی & خصوصی/مدنی & ایران & ضد اطلاعات نادرست + راستی‌آزمایی \\

\end{longtable}

\begin{warningbox}
\textbf{نکتهٔ حساسیت رسانه‌ای:} رسانه‌های فارسی‌زبان دولتی (آمریکا، بریتانیا) در ایران با بی‌اعتمادی مواجه هستند. مدل ۶ توصیه می‌کند: ۱) از \textbf{رسانه‌های مستقل ایرانی} حمایت مالی شود (نه دولتی)؛ ۲) \textbf{سامانهٔ راستی‌آزمایی} مستقل تأسیس شود؛ ۳) \textbf{استارلینک} و ابزارهای دسترسی آزاد به اینترنت فراهم شود (\seeChapter{ch:timeline}).
\end{warningbox}

\sectiondivider

%═══════════════════════════════════════════════════════════
\section{نمودار: شبکهٔ نهادهای کلیدی مرتبط با ایران}
\label{app:org:network}
%═══════════════════════════════════════════════════════════

\begin{figure}[htbp]
\centering
\begin{tikzpicture}[
  node distance=2cm,
  core/.style={
    draw=MainPurple, fill=MainPurple!15, rounded corners=5pt,
    minimum width=2.5cm, minimum height=1cm, font=\small\bfseries,
    align=center, thick
  },
  un/.style={
    draw=MainBlue, fill=MainBlue!10, rounded corners=3pt,
    minimum width=2cm, minimum height=0.7cm, font=\tiny,
    align=center
  },
  regional/.style={
    draw=MainGreen, fill=MainGreen!10, rounded corners=3pt,
    minimum width=2cm, minimum height=0.7cm, font=\tiny,
    align=center
  },
  ngo/.style={
    draw=MainOrange, fill=MainOrange!10, rounded corners=3pt,
    minimum width=2cm, minimum height=0.7cm, font=\tiny,
    align=center
  },
  iranian/.style={
    draw=MainRed, fill=MainRed!10, rounded corners=3pt,
    minimum width=2cm, minimum height=0.7cm, font=\tiny,
    align=center
  },
  finance/.style={
    draw=MainYellow!80!black, fill=MainYellow!10, rounded corners=3pt,
    minimum width=2cm, minimum height=0.7cm, font=\tiny,
    align=center
  },
  link/.style={-, gray!50, thin},
  stronglink/.style={-, MainPurple!60, thick}
]

% هستهٔ مرکزی
\node[core] (iran) at (0,0) {گذار\\ایران};

% سازمان ملل (بالا)
\node[un] (unsc) at (-3,3.5) {\lr{UNSC}\\شورای امنیت};
\node[un] (srsg) at (0,3.5) {\lr{SRSG/UNMOIT}\\مأموریت ایران};
\node[un] (iaea) at (3,3.5) {\lr{IAEA}\\هسته‌ای};
\node[un] (ohchr) at (-1.5,2.3) {\lr{OHCHR}\\حقوق بشر};
\node[un] (undp) at (1.5,2.3) {\lr{UNDP}\\توسعه};

% منطقه‌ای (چپ)
\node[regional] (eu) at (-5,1) {\lr{EU}\\اتحادیهٔ اروپا};
\node[regional] (osce) at (-5,-0.5) {\lr{OSCE/ODIHR}\\ناظران};
\node[regional] (oic) at (-5,-2) {\lr{OIC}\\همکاری اسلامی};

% دولت‌ها (راست)
\node[ngo] (us) at (5,1.5) {آمریکا\\تحریم/مشوق};
\node[ngo] (china) at (5,0) {چین + روسیه\\وتو/حمایت؟};
\node[ngo] (turkey) at (5,-1.5) {ترکیه + عربستان\\همسایگان};

% مالی (پایین-راست)
\node[finance] (wb) at (3,-3) {\lr{WB/IMF}\\بازسازی};
\node[finance] (fund) at (0,-3.5) {صندوق امانی\\بازسازی ایران};

% ایرانی (پایین-چپ)
\node[iranian] (const) at (-3,-3) {مجلس مؤسسان\\ایران};
\node[iranian] (trc) at (-1.5,-2.3) {کمیسیون حقیقت\\ایران};
\node[iranian] (civil) at (1.5,-2.3) {جامعهٔ مدنی\\ایران};

% NGOها (بالا-راست)
\node[ngo] (hrw) at (4.5,3) {\lr{HRW/AI}\\حقوق بشر};
\node[ngo] (ictj) at (3,2.3) {\lr{ICTJ}\\عدالت انتقالی};

% اتصالات قوی
\draw[stronglink] (iran) -- (srsg);
\draw[stronglink] (iran) -- (const);
\draw[stronglink] (iran) -- (trc);
\draw[stronglink] (iran) -- (civil);
\draw[stronglink] (iran) -- (fund);
\draw[stronglink] (srsg) -- (unsc);

% اتصالات عادی
\draw[link] (iran) -- (eu);
\draw[link] (iran) -- (osce);
\draw[link] (iran) -- (oic);
\draw[link] (iran) -- (us);
\draw[link] (iran) -- (china);
\draw[link] (iran) -- (turkey);
\draw[link] (iran) -- (wb);
\draw[link] (srsg) -- (ohchr);
\draw[link] (srsg) -- (undp);
\draw[link] (srsg) -- (iaea);
\draw[link] (srsg) -- (hrw);
\draw[link] (srsg) -- (ictj);
\draw[link] (fund) -- (wb);
\draw[link] (fund) -- (eu);
\draw[link] (trc) -- (ictj);
\draw[link] (const) -- (eu);

% راهنما
\node[font=\tiny, MainBlue] at (-5.5,3.5) {$\blacksquare$ سازمان ملل};
\node[font=\tiny, MainGreen] at (-5.5,3) {$\blacksquare$ منطقه‌ای};
\node[font=\tiny, MainOrange] at (-5.5,2.5) {$\blacksquare$ دولت‌ها/\lr{NGO}};
\node[font=\tiny, MainRed] at (-5.5,2) {$\blacksquare$ ایرانی};
\node[font=\tiny, MainYellow!80!black] at (-5.5,1.5) {$\blacksquare$ مالی};

\end{tikzpicture}
\caption{شبکهٔ نهادهای کلیدی مرتبط با نظارت بر گذار ایران}
\label{fig:app-org-network}
\end{figure}

\sectiondivider

%═══════════════════════════════════════════════════════════
\section{جمع‌بندی پیوست}
\label{app:org:conclusion}
%═══════════════════════════════════════════════════════════

\begin{chaptersummary}
جمع‌بندی پیوست د — فهرست نهادها و سازمان‌ها:

\begin{enumerate}[nosep]
\item بیش از \textbf{۸۰ نهاد} در ۹ دسته معرفی شدند: سازمان ملل (۱۴)، منطقه‌ای (۸)، دولت‌ها (۱۴)، \lr{NGO}ها (۱۴)، مالی (۶)، ایرانی (۱۴)، آکادمیک (۱۰)، رسانه‌ای (۷).
\item \textbf{شورای امنیت سازمان ملل} مرجع تصمیم‌گیری اصلی است — ریسک وتوی چین و روسیه باید مدیریت شود.
\item \textbf{\lr{SRSG} و مأموریت \lr{UNMOIT}} هستهٔ مرکزی نظارت بین‌المللی خواهد بود.
\item \textbf{نهادهای ایرانی پیشنهادی} (مجلس مؤسسان، کمیسیون حقیقت، ارتش ملی واحد) \textbf{اصلی‌ترین بازیگران} هستند — مالکیت ملی.
\item \textbf{چهارگانهٔ ایرانی} (الگوی تونس) و \textbf{تشکل‌های صنفی} نقش میانجی‌گری خواهند داشت.
\item \textbf{رسانه‌های مستقل ایرانی} باید تقویت شوند — رسانه‌های دولتی خارجی بی‌اعتمادی ایجاد می‌کنند.
\item \lr{IAEA} نقش ویژه‌ای در بُعد هسته‌ای دارد — مکمل فرآیند سیاسی.
\item صندوق امانی بازسازی ایران (\$۲.۵-۵B) با مشارکت بانک جهانی، \lr{EU}، و کشورهای کمک‌دهنده تأمین خواهد شد.
\end{enumerate}

\vspace{0.3cm}
\textit{ارجاعات:}
\begin{itemize}[nosep]
\item نهادها و بازیگران (تفصیلی): \seeChapter{ch:actors}
\item بودجه‌بندی و تأمین مالی: \seeChapter{ch:budget}
\item زمان‌بندی و تیم‌سازی: \seeChapter{ch:timeline}
\item واژه‌نامهٔ تخصصی: \seeChapter{app:glossary}
\end{itemize}
\end{chaptersummary}

\chapterend

%══════════════════════════════════════════════════════════════
% پایان پیوست د
%══════════════════════════════════════════════════════════════
