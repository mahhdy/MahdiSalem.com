%══════════════════════════════════════════════════════════════
% پیوست الف: مقایسه جامع ۹ نمونه تاریخی گذار
% فایل: appendices/app-a-comparison.tex
% رنگ: خاکستری/چندرنگ (پیوست)
% حجم هدف: ۲۰-۲۵ صفحه
%══════════════════════════════════════════════════════════════

\chapter{مقایسه جامع نُه نمونهٔ تاریخی گذار دموکراتیک}
\label{app:comparison}

%-----------------------------------------------------------
\begin{executivesummary}
این پیوست، جامع‌ترین بخش مقایسه‌ای کتاب حاضر است و نُه نمونهٔ تاریخی گذار دموکراتیک را در قالب جداول افقی گسترده بررسی می‌کند. کشورهای مورد مطالعه عبارت‌اند از: \textbf{اسپانیا} (۱۹۷۵-۱۹۸۲)، \textbf{لهستان} (۱۹۸۹-۱۹۹۱)، \textbf{شیلی} (۱۹۸۸-۱۹۹۰)، \textbf{آفریقای جنوبی} (۱۹۹۰-۱۹۹۴)، \textbf{اندونزی} (۱۹۹۸-۲۰۰۴)، \textbf{تیمور شرقی} (۱۹۹۹-۲۰۰۲)، \textbf{عراق} (۲۰۰۳-۲۰۱۰)، \textbf{تونس} (۲۰۱۱-۲۰۱۴) و \textbf{میانمار} (۲۰۱۰-۲۰۲۱). این نمونه‌ها در ۱۰ بُعد تحلیلی و بیش از ۶۰ شاخص مقایسه شده‌اند تا \textbf{درس‌آموخته‌های قابل‌انتقال} به سناریوی گذار ایران استخراج شود.
\end{executivesummary}

%-----------------------------------------------------------
\section{مقدمه و روش‌شناسی مقایسه}
\label{app:a:intro}

\subsection{معیارهای انتخاب نمونه‌ها}

انتخاب نُه نمونهٔ تاریخی بر اساس پنج معیار صورت گرفته است:

\begin{enumerate}[nosep]
\item \textbf{تنوع جغرافیایی:} اروپا (اسپانیا، لهستان)، آمریکای لاتین (شیلی)، آفریقا (آفریقای جنوبی)، آسیای جنوب‌شرقی (اندونزی، تیمور شرقی، میانمار)، خاورمیانه/شمال آفریقا (عراق، تونس)
\item \textbf{تنوع نوع گذار:} مذاکره‌ای (اسپانیا، لهستان، آفریقای جنوبی)، انقلابی/مردمی (تونس، اندونزی)، مداخلهٔ خارجی (عراق، تیمور شرقی)، ترکیبی (شیلی)، ناتمام (میانمار)
\item \textbf{تنوع نتیجه:} موفق (اسپانیا، شیلی)، نسبتاً موفق (لهستان، آفریقای جنوبی، تیمور شرقی)، ناموفق (عراق)، بازگشتی (میانمار، تونس پس از ۲۰۲۱)
\item \textbf{تنوع زمانی:} موج سوم (اسپانیا ۱۹۷۵)، موج چهارم (لهستان ۱۹۸۹)، پسا-یازده‌سپتامبر (عراق ۲۰۰۳)، بهار عربی (تونس ۲۰۱۱)
\item \textbf{مشابهت تحلیلی با ایران:} در حداقل سه بُعد مرتبط با پروندهٔ ایران
\end{enumerate}

\subsection{ابعاد ده‌گانهٔ مقایسه}

\begin{keypoint}
هر نمونه در \textbf{ده بُعد} تحلیل شده است: ۱)مشخصات عمومی، ۲)رژیم پیشین، ۳)مسیر و محرک گذار، ۴)بازیگران اصلی، ۵)نقش بین‌المللی، ۶)نظارت و مدیریت گذار، ۷)عدالت انتقالی، ۸)اصلاحات بخش امنیتی، ۹)نتایج و شاخص‌ها، ۱۰)درس‌آموخته‌ها برای ایران.
\end{keypoint}

\subsection{نظام نمادها و امتیازدهی}

در جداول این پیوست از نظام نمادی زیر استفاده شده است:

\begin{center}
\begin{tabularx}{0.85\textwidth}{>{\centering\arraybackslash}p{2.5cm} >{\raggedleft\arraybackslash}X}
\toprule
\headerrow \textbf{نماد} & \textbf{معنا} \\
\midrule
\cmark & وجود / اجرا / موفقیت \\
\altrow \xmark & عدم وجود / شکست \\
$\sim$ & نسبی / ناقص \\
\altrow \statusok & وضعیت مطلوب \\
\statuswarn & وضعیت هشداری \\
\altrow \statusbad & وضعیت بحرانی \\
\rating{5} & امتیاز ۵ از ۵ (بسیار بالا) \\
\altrow \rating{1} & امتیاز ۱ از ۵ (بسیار پایین) \\
\bottomrule
\end{tabularx}
\end{center}

\sectiondivider

%═══════════════════════════════════════════════════════════
\section{جدول اول: مشخصات عمومی و زمینه‌ای}
\label{app:a:table1}
%═══════════════════════════════════════════════════════════

\begin{landscape}
\pagestyle{empty}
\bigtablefontsize

\begin{longtable}{
  >{\raggedleft\arraybackslash}p{2.1cm}|
  >{\raggedleft\arraybackslash}p{1.7cm}
  >{\raggedleft\arraybackslash}p{1.7cm}
  >{\raggedleft\arraybackslash}p{1.7cm}
  >{\raggedleft\arraybackslash}p{1.7cm}
  >{\raggedleft\arraybackslash}p{1.7cm}
  >{\raggedleft\arraybackslash}p{1.7cm}
  >{\raggedleft\arraybackslash}p{1.7cm}
  >{\raggedleft\arraybackslash}p{1.7cm}
  >{\raggedleft\arraybackslash}p{1.7cm}
}
\caption{مشخصات عمومی نُه نمونهٔ تاریخی گذار}
\label{tab:app-a-general} \\

\toprule
\headerrow
\rot{\textbf{شاخص}} &
\rot{\textbf{اسپانیا}} &
\rot{\textbf{لهستان}} &
\rot{\textbf{شیلی}} &
\rot{\textbf{آفریقای جنوبی}} &
\rot{\textbf{اندونزی}} &
\rot{\textbf{تیمور شرقی}} &
\rot{\textbf{عراق}} &
\rot{\textbf{تونس}} &
\rot{\textbf{میانمار}} \\
\midrule
\endfirsthead

\multicolumn{10}{c}{\small\textit{ادامهٔ جدول \ref{tab:app-a-general}: مشخصات عمومی}} \\
\toprule
\headerrow
\rot{\textbf{شاخص}} &
\rot{\textbf{اسپانیا}} &
\rot{\textbf{لهستان}} &
\rot{\textbf{شیلی}} &
\rot{\textbf{آفریقای جنوبی}} &
\rot{\textbf{اندونزی}} &
\rot{\textbf{تیمور شرقی}} &
\rot{\textbf{عراق}} &
\rot{\textbf{تونس}} &
\rot{\textbf{میانمار}} \\
\midrule
\endhead

\midrule
\multicolumn{10}{l}{\small\textit{ادامه در صفحهٔ بعد...}} \\
\endfoot

\bottomrule
\endlastfoot

%--- ردیف‌ها ---
\textbf{منطقه} &
اروپای جنوبی &
اروپای شرقی &
آمریکای لاتین &
آفریقای جنوبی &
آسیای جنوب‌شرقی &
آسیای جنوب‌شرقی &
خاورمیانه &
شمال آفریقا &
آسیای جنوب‌شرقی \\

\altrow
\textbf{جمعیت (زمان گذار)} &
۳۷ میلیون &
۳۸ میلیون &
۱۳ میلیون &
۴۰ میلیون &
۲۱۰ میلیون &
۰.۸ میلیون &
۲۶ میلیون &
۱۱ میلیون &
۵۲ میلیون \\

\textbf{مساحت (\lr{km²})} &
۵۰۵,۰۰۰ &
۳۱۲,۰۰۰ &
۷۵۶,۰۰۰ &
۱,۲۲۰,۰۰۰ &
۱,۹۰۵,۰۰۰ &
۱۵,۰۰۰ &
۴۳۸,۰۰۰ &
۱۶۳,۰۰۰ &
۶۷۶,۰۰۰ \\

\altrow
\textbf{تنوع قومی-مذهبی} &
متوسط (کاتالان، باسک) &
پایین (۹۷٪ لهستانی) &
پایین &
بسیار بالا (۱۱ زبان رسمی) &
بسیار بالا (۳۰۰+ قوم) &
متوسط &
بالا (عرب/کرد/ ترکمن، شیعه/سنی) &
پایین &
بسیار بالا (۱۳۵ قوم) \\

\textbf{منابع طبیعی کلیدی} &
محدود &
زغال‌سنگ &
مس &
طلا، الماس &
نفت، گاز &
نفت &
نفت (ذخایر ۵) &
فسفات &
یشم، گاز \\

\altrow
\textbf{\lr{GDP per capita} (زمان گذار)} &
$\sim$\$۵,۰۰۰ &
$\sim$\$۱,۷۰۰ &
$\sim$\$۲,۵۰۰ &
$\sim$\$۳,۵۰۰ &
$\sim$\$۱,۰۰۰ &
$\sim$\$۴۰۰ &
$\sim$\$۹۰۰ &
$\sim$\$۴,۲۰۰ &
$\sim$\$۱,۲۰۰ \\

\textbf{سال آغاز گذار} &
۱۹۷۵ &
۱۹۸۹ &
۱۹۸۸ &
۱۹۹۰ &
۱۹۹۸ &
۱۹۹۹ &
۲۰۰۳ &
۲۰۱۱ &
۲۰۱۰ \\

\altrow
\textbf{مدت فاز انتقالی} &
$\sim$۷ سال &
$\sim$۲ سال &
$\sim$۲ سال &
$\sim$۴ سال &
$\sim$۶ سال &
$\sim$۳ سال &
$\sim$۷ سال+ &
$\sim$۳ سال &
$\sim$۵ سال (سپس بازگشت) \\

\textbf{سال اولین انتخابات آزاد} &
۱۹۷۷ &
۱۹۸۹ (نسبی) &
۱۹۸۹ &
۱۹۹۴ &
۱۹۹۹ &
۲۰۰۱ &
۲۰۰۵ &
۲۰۱۱ (مجلس مؤسسان) &
۲۰۱۵ \\

\altrow
\textbf{قانون اساسی جدید} &
۱۹۷۸ &
اصلاحیهٔ ۱۹۹۷ &
اصلاحیهٔ ۱۹۸۹ &
۱۹۹۶ &
اصلاحیه‌های متعدد &
۲۰۰۲ &
۲۰۰۵ &
۲۰۱۴ &
۲۰۰۸ (نظامی) \\

\textbf{\termfn{شاخص دموکراسی فعلی}{V-Dem ۲۰۲۳}} &
\cellgreen{بالا (۰.۸۱)} &
\cellorange{متوسط (۰.۵۵)} &
\cellgreen{بالا (۰.۷۹)} &
\cellgreen{بالا (۰.۷۲)} &
\cellorange{متوسط (۰.۵۲)} &
\cellorange{متوسط (۰.۵۸)} &
\cellred{پایین (۰.۲۵)} &
\cellred{پایین (۰.۱۸)} &
\cellred{بسیار پایین (۰.۰۸)} \\

\altrow
\textbf{ارزیابی کلی نتیجه} &
\statusok موفق &
\statusok موفق (با افت اخیر) &
\statusok موفق &
\statusok نسبتاً موفق &
\statuswarn نسبتاً موفق &
\statuswarn نسبتاً موفق &
\statusbad ناموفق &
\statusbad بازگشت (۲۰۲۱) &
\statusbad بازگشت (۲۰۲۱) \\

\textbf{مقایسه‌پذیری با ایران} &
\rating{4} &
\rating{3} &
\rating{4} &
\rating{5} &
\rating{4} &
\rating{3} &
\rating{3} &
\rating{4} &
\rating{3} \\

\end{longtable}
\end{landscape}

\sectiondivider

%═══════════════════════════════════════════════════════════
\section{جدول دوم: ویژگی‌های رژیم پیشین}
\label{app:a:table2}
%═══════════════════════════════════════════════════════════

\begin{landscape}
\pagestyle{empty}
\bigtablefontsize

\begin{longtable}{
  >{\raggedleft\arraybackslash}p{2.1cm}|
  >{\raggedleft\arraybackslash}p{1.7cm}
  >{\raggedleft\arraybackslash}p{1.7cm}
  >{\raggedleft\arraybackslash}p{1.7cm}
  >{\raggedleft\arraybackslash}p{1.7cm}
  >{\raggedleft\arraybackslash}p{1.7cm}
  >{\raggedleft\arraybackslash}p{1.7cm}
  >{\raggedleft\arraybackslash}p{1.7cm}
  >{\raggedleft\arraybackslash}p{1.7cm}
  >{\raggedleft\arraybackslash}p{1.7cm}
}
\caption{ویژگی‌های رژیم پیشین در نُه نمونه}
\label{tab:app-a-regime} \\

\toprule
\headerrow
\rot{\textbf{شاخص}} &
\rot{\textbf{اسپانیا}} &
\rot{\textbf{لهستان}} &
\rot{\textbf{شیلی}} &
\rot{\textbf{آفریقای جنوبی}} &
\rot{\textbf{اندونزی}} &
\rot{\textbf{تیمور شرقی}} &
\rot{\textbf{عراق}} &
\rot{\textbf{تونس}} &
\rot{\textbf{میانمار}} \\
\midrule
\endfirsthead

\multicolumn{10}{c}{\small\textit{ادامهٔ جدول \ref{tab:app-a-regime}: رژیم پیشین}} \\
\toprule
\headerrow
\rot{\textbf{شاخص}} &
\rot{\textbf{اسپانیا}} &
\rot{\textbf{لهستان}} &
\rot{\textbf{شیلی}} &
\rot{\textbf{آفریقای جنوبی}} &
\rot{\textbf{اندونزی}} &
\rot{\textbf{تیمور شرقی}} &
\rot{\textbf{عراق}} &
\rot{\textbf{تونس}} &
\rot{\textbf{میانمار}} \\
\midrule
\endhead

\bottomrule
\endlastfoot

\textbf{نوع رژیم} &
فاشیست/ اقتدارگرای شخصی &
کمونیست تک‌حزبی &
دیکتاتوری نظامی &
آپارتاید نژادی &
اقتدارگرای شخصی-نظامی &
اشغال خارجی (اندونزی) &
اقتدارگرای تمامیت‌خواه &
اقتدارگرای شخصی &
جونتای نظامی \\

\altrow
\textbf{طول عمر رژیم} &
۳۶ سال (۱۹۳۹-۱۹۷۵) &
۴۵ سال (۱۹۴۴-۱۹۸۹) &
۱۷ سال (۱۹۷۳-۱۹۹۰) &
۴۶ سال (۱۹۴۸-۱۹۹۴) &
۳۲ سال (۱۹۶۶-۱۹۹۸) &
۲۴ سال (۱۹۷۵-۱۹۹۹) &
۳۵ سال (۱۹۶۸-۲۰۰۳) &
۲۳ سال (۱۹۸۷-۲۰۱۱) &
۴۹ سال (۱۹۶۲-۲۰۱۱) \\

\textbf{ایدئولوژی مسلط} &
ناسیونال-کاتولیک &
مارکسیسم-لنینیسم &
ضدکمونیسم/ نئولیبرال &
سفیدبرتری نژادی &
پنچاسیلا/ ضدکمونیسم &
-- (اشغال) &
بعثیسم/ ناسیونالیسم عربی &
سکولاریسم اقتدارگرا &
ناسیونالیسم بامار \\

\altrow
\textbf{نقش ارتش/نیروهای مسلح} &
\riskhigh بالا (ارتش حامی فرانکو) &
\riskmedium متوسط &
\riskhigh بسیار بالا (پینوشه) &
\riskhigh بالا (\lr{SADF}) &
\riskhigh بسیار بالا (\lr{TNI}) &
\riskhigh بسیار بالا &
\riskhigh بسیار بالا (حرس جمهوری) &
\riskmedium متوسط &
\riskhigh بسیار بالا (تاتمادو) \\

\textbf{نقش نهاد مذهبی} &
کلیسای کاتولیک (حامی سپس منتقد) &
کلیسای کاتولیک (حامی جنبش) &
کلیسا (حامی حقوق‌بشر) &
کلیساها (متنوع) &
سازمان‌های اسلامی (متنوع) &
کلیسای کاتولیک (حامی) &
مرجعیت شیعه (دوگانه) &
محدود &
سنگها (بودایی، ملی‌گرا) \\

\altrow
\textbf{سطح سرکوب} &
بالا (اعدام، شکنجه) &
متوسط (حکومت نظامی ۱۹۸۱) &
بسیار بالا (۳,۰۰۰+ ناپدیدشده) &
بسیار بالا (شارپویل، سوتو) &
بالا (تیانمن اندونزیایی) &
بسیار بالا (نسل‌کشی) &
بسیار بالا (انفال، حلبچه) &
بالا (ترور) &
بسیار بالا \\

\textbf{وضعیت جامعهٔ مدنی} &
محدود اما فعال (اواخر) &
قوی (همبستگی) &
بازسازی‌شده (اواخر) &
قوی (\lr{UDF/ANC}) &
نسبتاً فعال &
ضعیف &
سرکوب‌شده &
نسبتاً فعال &
ضعیف \\

\altrow
\textbf{وجود اپوزیسیون سازمان‌یافته} &
\cmark (احزاب مخفی) &
\cmark (همبستگی ۱۰M) &
\cmark (ائتلاف «نه») &
\cmark (\lr{ANC}) &
$\sim$ (پراکنده) &
\cmark (\lr{FRETILIN}) &
$\sim$ (تبعیدی) &
$\sim$ (ضعیف) &
\cmark (\lr{NLD}) \\

\textbf{وضعیت اقتصاد (آستانهٔ گذار)} &
رشد (معجزهٔ اسپانیا) &
بحران شدید &
رشد نسبی &
رکود + تحریم &
بحران مالی آسیا &
فروپاشی &
تحریم + جنگ &
رشد نسبی اما نابرابر &
رکود + تحریم \\

\altrow
\textbf{مشابهت با ج.ا.ایران} &
\rating{3} (شخصی، ارتش، ایدئولوژیک) &
\rating{2} (تک‌حزبی، بحران اقتصادی) &
\rating{4} (نظامی-اقتصادی، رفراندوم) &
\rating{4} (ایدئولوژیک، تحریم، سرکوب) &
\rating{4} (سپاه، تنوع، نفت) &
\rating{2} (اشغال، کوچک) &
\rating{3} (شیعه، نفت، سپاه) &
\rating{3} (خاورمیانه، جوان) &
\rating{3} (نظامی-اقتصادی) \\

\end{longtable}
\end{landscape}

\begin{lessonlearned}
از میان نُه نمونه، \textbf{آفریقای جنوبی} و \textbf{اندونزی} بیشترین مشابهت ساختاری با ایران دارند: هر دو دارای نیروهای امنیتی قدرتمند با منافع اقتصادی، جمعیت بزرگ، تنوع قومی-مذهبی، و ایدئولوژی رسمی سرکوبگر بودند. تفاوت کلیدی ایران: \textbf{ترکیب ایدئولوژی مذهبی + قدرت نظامی-اقتصادی سپاه + برنامه هسته‌ای} این پرونده را منحصربه‌فرد می‌سازد.
\end{lessonlearned}

\sectiondivider

%═══════════════════════════════════════════════════════════
\section{جدول سوم: مسیر و محرک‌های گذار}
\label{app:a:table3}
%═══════════════════════════════════════════════════════════

\begin{landscape}
\pagestyle{empty}
\bigtablefontsize

\begin{longtable}{
  >{\raggedleft\arraybackslash}p{2.1cm}|
  >{\raggedleft\arraybackslash}p{1.7cm}
  >{\raggedleft\arraybackslash}p{1.7cm}
  >{\raggedleft\arraybackslash}p{1.7cm}
  >{\raggedleft\arraybackslash}p{1.7cm}
  >{\raggedleft\arraybackslash}p{1.7cm}
  >{\raggedleft\arraybackslash}p{1.7cm}
  >{\raggedleft\arraybackslash}p{1.7cm}
  >{\raggedleft\arraybackslash}p{1.7cm}
  >{\raggedleft\arraybackslash}p{1.7cm}
}
\caption{مسیر و محرک‌های گذار در نُه نمونه}
\label{tab:app-a-drivers} \\

\toprule
\headerrow
\rot{\textbf{شاخص}} &
\rot{\textbf{اسپانیا}} &
\rot{\textbf{لهستان}} &
\rot{\textbf{شیلی}} &
\rot{\textbf{آفریقای جنوبی}} &
\rot{\textbf{اندونزی}} &
\rot{\textbf{تیمور شرقی}} &
\rot{\textbf{عراق}} &
\rot{\textbf{تونس}} &
\rot{\textbf{میانمار}} \\
\midrule
\endfirsthead

\multicolumn{10}{c}{\small\textit{ادامهٔ جدول \ref{tab:app-a-drivers}: مسیر و محرک‌های گذار}} \\
\toprule
\headerrow
\rot{\textbf{شاخص}} &
\rot{\textbf{اسپانیا}} &
\rot{\textbf{لهستان}} &
\rot{\textbf{شیلی}} &
\rot{\textbf{آفریقای جنوبی}} &
\rot{\textbf{اندونزی}} &
\rot{\textbf{تیمور شرقی}} &
\rot{\textbf{عراق}} &
\rot{\textbf{تونس}} &
\rot{\textbf{میانمار}} \\
\midrule
\endhead

\bottomrule
\endlastfoot

\textbf{نوع گذار (تیپولوژی)} &
\termfn{مذاکره‌ای از بالا}{Pacted} &
\termfn{میزگرد}{Round Table} &
\termfn{ترکیبی}{Hybrid} (رفراندوم) &
\termfn{مذاکره‌ای}{Negotiated} &
\termfn{فروپاشی}{Collapse} + مذاکره &
\termfn{مداخله + رفراندوم}{Intervention} &
\termfn{مداخله نظامی}{Military Intervention} &
\termfn{انقلاب مردمی}{Popular Revolution} &
\termfn{تحول از بالا}{Top-down} \\

\altrow
\textbf{محرک اصلی} &
مرگ فرانکو &
بحران اقتصادی + همبستگی &
فشار مدنی + رفراندوم &
فشار بین‌المللی + بحران &
بحران مالی آسیا &
رفراندوم + مداخله &
حملهٔ نظامی آمریکا &
خودسوزی بوعزیزی &
تصمیم ژنرال‌ها \\

\textbf{نقش مرگ/حذف رهبر} &
\cmark (مرگ طبیعی) &
\xmark &
\xmark (پینوشه ماند) &
\xmark (دکلرک مذاکره کرد) &
\cmark (سقوط سوهارتو) &
-- &
\cmark (سقوط صدام) &
\cmark (فرار بن‌علی) &
\xmark (انتقال تدریجی) \\

\altrow
\textbf{نقش شکاف نخبگان} &
\cmark (اصلاح‌طلبان سوآرز) &
\cmark (یاروزلسکی) &
$\sim$ &
\cmark (دکلرک) &
\cmark (شکاف ارتش) &
$\sim$ &
\xmark &
$\sim$ &
\cmark \\

\textbf{نقش اعتراضات مردمی} &
$\sim$ &
\cmark (قوی) &
\cmark (قوی) &
\cmark (بسیار قوی) &
\cmark (بسیار قوی) &
\cmark &
\xmark &
\cmark (بسیار قوی) &
$\sim$ \\

\altrow
\textbf{نقش فشار خارجی} &
\cmark (\lr{EC}) &
\cmark (واتیکان + آمریکا) &
\cmark (آمریکا) &
\cmark (تحریم‌ها) &
\cmark (\lr{IMF}) &
\cmark (سازمان ملل) &
\cmark (تعیین‌کننده) &
$\sim$ &
$\sim$ \\

\textbf{خشونت فاز گذار} &
\risklow محدود (کودتای ۲۳-F) &
\risklow بسیار محدود &
\risklow محدود &
\riskmedium متوسط (IFP/ANC) &
\riskmedium متوسط (شورش‌ها) &
\riskhigh شدید (میلیشیا) &
\riskhigh بسیار شدید (جنگ داخلی) &
\risklow محدود &
\risklow محدود (در فاز اول) \\

\altrow
\textbf{توافق‌نامهٔ سیاسی رسمی} &
\cmark \lr{Moncloa Pacts} (۱۹۷۷) &
\cmark توافق میزگرد (۱۹۸۹) &
$\sim$ رفراندوم ۱۹۸۸ &
\cmark \lr{CODESA/} قانون اساسی موقت &
$\sim$ توافق ضمنی &
\cmark توافق نیویورک ۱۹۹۹ &
\xmark (اشغال نظامی) &
\cmark گفت‌وگوی ملی &
$\sim$ قانون اساسی ۲۰۰۸ \\

\textbf{درس برای ایران} &
اهمیت اصلاح‌طلب درون نظام &
قدرت جنبش اجتماعی سازمان‌یافته &
ابزار رفراندوم &
مذاکره با فشار &
ارتش منشعب‌شدنی &
نیاز به حضور بین‌المللی &
مداخله نظامی = فاجعه &
سرعت بالا + خطر بازگشت &
گشایش کنترل‌شده ≠ دموکراسی \\

\end{longtable}
\end{landscape}

\begin{warningbox}
از نُه نمونه، تنها \textbf{عراق} با \textbf{مداخلهٔ نظامی خارجی} وارد گذار شد و تنها نمونه‌ای است که به \textbf{جنگ داخلی} منجر شد. هزینهٔ انسانی: بیش از ۲۰۰,۰۰۰ کشته. هزینهٔ مالی: بیش از ۲ تریلیون دلار. این نمونه \textbf{قوی‌ترین شاهد ضد مداخلهٔ نظامی} در پروندهٔ ایران است (\seeChapter{ch:scenarios}).
\end{warningbox}

\sectiondivider

%═══════════════════════════════════════════════════════════
\section{جدول چهارم: بازیگران اصلی داخلی}
\label{app:a:table4}
%═══════════════════════════════════════════════════════════

\begin{landscape}
\pagestyle{empty}
\bigtablefontsize

\begin{longtable}{
  >{\raggedleft\arraybackslash}p{2.1cm}|
  >{\raggedleft\arraybackslash}p{1.7cm}
  >{\raggedleft\arraybackslash}p{1.7cm}
  >{\raggedleft\arraybackslash}p{1.7cm}
  >{\raggedleft\arraybackslash}p{1.7cm}
  >{\raggedleft\arraybackslash}p{1.7cm}
  >{\raggedleft\arraybackslash}p{1.7cm}
  >{\raggedleft\arraybackslash}p{1.7cm}
  >{\raggedleft\arraybackslash}p{1.7cm}
  >{\raggedleft\arraybackslash}p{1.7cm}
}
\caption{بازیگران اصلی داخلی گذار}
\label{tab:app-a-actors} \\

\toprule
\headerrow
\rot{\textbf{شاخص}} &
\rot{\textbf{اسپانیا}} &
\rot{\textbf{لهستان}} &
\rot{\textbf{شیلی}} &
\rot{\textbf{آفریقای جنوبی}} &
\rot{\textbf{اندونزی}} &
\rot{\textbf{تیمور شرقی}} &
\rot{\textbf{عراق}} &
\rot{\textbf{تونس}} &
\rot{\textbf{میانمار}} \\
\midrule
\endfirsthead

\multicolumn{10}{c}{\small\textit{ادامهٔ جدول \ref{tab:app-a-actors}: بازیگران داخلی}} \\
\toprule
\headerrow
\rot{\textbf{شاخص}} &
\rot{\textbf{اسپانیا}} &
\rot{\textbf{لهستان}} &
\rot{\textbf{شیلی}} &
\rot{\textbf{آفریقای جنوبی}} &
\rot{\textbf{اندونزی}} &
\rot{\textbf{تیمور شرقی}} &
\rot{\textbf{عراق}} &
\rot{\textbf{تونس}} &
\rot{\textbf{میانمار}} \\
\midrule
\endhead

\bottomrule
\endlastfoot

\textbf{رهبر کلیدی گذار} &
\person{خوان‌کارلوس + سوآرز}{Juan Carlos \& Suárez} &
\person{والسا + مازوویتسکی}{Wałęsa} &
\person{آیلوین}{Aylwin} &
\person{ماندلا + دکلرک}{Mandela \& de Klerk} &
\person{حبیبی + واحد}{Habibie \& Wahid} &
\person{گوسمائو}{Gusmão} &
\person{برمر (خارجی)}{Bremer} &
\person{السبسی + غنوشی}{Essebsi \& Ghannouchi} &
\person{تئین‌سئین}{Thein Sein} \\

\altrow
\textbf{نقش رهبر پیشین رژیم} &
فرانکو مُرد &
یاروزلسکی تسلیم شد &
پینوشه ماند (سناتور) &
دکلرک شریک شد &
سوهارتو استعفا داد &
-- (اشغالگر رفت) &
صدام اعدام شد &
بن‌علی فرار کرد &
تان‌شوی بازنشسته شد \\

\textbf{حزب/جنبش اصلی اپوزیسیون} &
\lr{PSOE + PCE} &
\lr{Solidarność} &
\lr{Concertación} &
\lr{ANC + COSATU + SACP} &
\lr{PDI-P} + اسلامی‌ها &
\lr{FRETILIN + CNRT} &
احزاب شیعه/کرد &
\lr{Ennahda + Nidaa} &
\lr{NLD} \\

\altrow
\textbf{نقش جنبش کارگری} &
\cmark (\lr{CCOO}) &
\cmark (تعیین‌کننده) &
\cmark (\lr{CUT}) &
\cmark (\lr{COSATU}) &
$\sim$ &
\xmark &
\xmark &
\cmark (\lr{UGTT}) &
\xmark \\

\textbf{نقش زنان} &
$\sim$ (محدود) &
$\sim$ (محدود) &
$\sim$ (محدود) &
\cmark (منشور حقوق) &
$\sim$ &
$\sim$ &
\xmark &
\cmark (مادهٔ ۴۶ قانون اساسی) &
\cmark (\lr{NLD}: سوچی) \\

\altrow
\textbf{نقش دیاسپورا} &
$\sim$ &
\cmark (لابی در غرب) &
\cmark (تبعیدیان) &
\xmark &
$\sim$ &
\cmark (لابی استقلال) &
\cmark (شورای حاکمیتی) &
$\sim$ &
\cmark \\

\textbf{مقایسه با اپوزیسیون ایران} &
ایران: احزاب ضعیف‌تر &
ایران: فاقد تشکل ۱۰M &
ایران: رفراندوم ممکن &
ایران: دیاسپورای قوی‌تر &
ایران: سپاه = \lr{TNI}+ &
ایران: بزرگ‌تر &
ایران: اپوزیسیون پراکنده‌تر &
ایران: جنبش زنان قوی &
ایران: فاقد سوچی \\

\end{longtable}
\end{landscape}

\sectiondivider

%═══════════════════════════════════════════════════════════
\section{جدول پنجم: نقش بین‌المللی و نظارت}
\label{app:a:table5}
%═══════════════════════════════════════════════════════════

\begin{landscape}
\pagestyle{empty}
\bigtablefontsize

\begin{longtable}{
  >{\raggedleft\arraybackslash}p{2.1cm}|
  >{\raggedleft\arraybackslash}p{1.7cm}
  >{\raggedleft\arraybackslash}p{1.7cm}
  >{\raggedleft\arraybackslash}p{1.7cm}
  >{\raggedleft\arraybackslash}p{1.7cm}
  >{\raggedleft\arraybackslash}p{1.7cm}
  >{\raggedleft\arraybackslash}p{1.7cm}
  >{\raggedleft\arraybackslash}p{1.7cm}
  >{\raggedleft\arraybackslash}p{1.7cm}
  >{\raggedleft\arraybackslash}p{1.7cm}
}
\caption{نقش بین‌المللی و ساختار نظارت}
\label{tab:app-a-intl} \\

\toprule
\headerrow
\rot{\textbf{شاخص}} &
\rot{\textbf{اسپانیا}} &
\rot{\textbf{لهستان}} &
\rot{\textbf{شیلی}} &
\rot{\textbf{آفریقای جنوبی}} &
\rot{\textbf{اندونزی}} &
\rot{\textbf{تیمور شرقی}} &
\rot{\textbf{عراق}} &
\rot{\textbf{تونس}} &
\rot{\textbf{میانمار}} \\
\midrule
\endfirsthead

\multicolumn{10}{c}{\small\textit{ادامهٔ جدول \ref{tab:app-a-intl}: نقش بین‌المللی}} \\
\toprule
\headerrow
\rot{\textbf{شاخص}} &
\rot{\textbf{اسپانیا}} &
\rot{\textbf{لهستان}} &
\rot{\textbf{شیلی}} &
\rot{\textbf{آفریقای جنوبی}} &
\rot{\textbf{اندونزی}} &
\rot{\textbf{تیمور شرقی}} &
\rot{\textbf{عراق}} &
\rot{\textbf{تونس}} &
\rot{\textbf{میانمار}} \\
\midrule
\endhead

\bottomrule
\endlastfoot

\textbf{مدل نظارت (از فصل ۳)} &
مدل ۱ (انتخاباتی محدود) &
مدل ۲ (مشورتی) &
مدل ۲ (مشورتی) &
مدل ۳ (ساختاری) &
مدل ۲ (مشورتی) &
مدل ۵ (مدیریت مستقیم) &
مدل ۵ (مدیریت مستقیم) &
مدل ۱-۲ (انتخاباتی + مشورتی) &
مدل ۱ (انتخاباتی محدود) \\

\altrow
\textbf{قطعنامهٔ شورای امنیت} &
\xmark &
\xmark &
\xmark &
\xmark &
\xmark &
\cmark (Res.\ 1272) &
\cmark (Res.\ 1483+) &
\xmark &
\xmark \\

\textbf{مأموریت سازمان ملل} &
\xmark &
\xmark &
\xmark &
ناظران (UNOMSA) &
\xmark &
\cmark (UNTAET/UNMISET) &
\cmark (UNAMI) &
\xmark &
$\sim$ (نمایندهٔ ویژه) \\

\altrow
\textbf{نیروی حافظ صلح} &
\xmark &
\xmark &
\xmark &
\xmark &
\xmark (INTERFET استرالیا) &
\cmark (INTERFET→PKF) &
\cmark (MNF-I) &
\xmark &
\xmark \\

\textbf{ناظران انتخاباتی بین‌المللی} &
$\sim$ (محدود) &
\cmark &
\cmark (قوی) &
\cmark (۲,۱۲۰ ناظر) &
\cmark (کارتر) &
\cmark (سازمان ملل) &
\cmark (محدود) &
\cmark (\lr{EU + Carter}) &
\cmark \\

\altrow
\textbf{نقش سازمان‌های منطقه‌ای} &
\cmark (\lr{EC}→عضویت) &
\cmark (\lr{CSCE/NATO}) &
\cmark (\lr{OAS}) &
\cmark (\lr{OAU/SADC}) &
\cmark (\lr{ASEAN} محدود) &
$\sim$ &
$\sim$ (اتحادیهٔ عرب ضعیف) &
$\sim$ (اتحادیهٔ عرب) &
\cmark (\lr{ASEAN}) \\

\textbf{نقش \lr{NGO}های بین‌المللی} &
$\sim$ &
\cmark (\lr{NED, Soros}) &
\cmark (\lr{HRW, AI}) &
\cmark (بسیار فعال) &
\cmark &
\cmark &
\cmark &
\cmark (بسیار فعال) &
\cmark \\

\altrow
\textbf{تحریم‌ها (قبل از گذار)} &
\xmark &
\xmark &
$\sim$ (محدود) &
\cmark (جامع و مؤثر) &
\xmark &
-- &
\cmark (جامع) &
\xmark &
\cmark (هدفمند) \\

\textbf{کمک‌های بین‌المللی پس از گذار} &
\cmark (صندوق‌های \lr{EC}) &
\cmark (\lr{PHARE, IMF}) &
$\sim$ &
\cmark (\lr{EU + US}) &
\cmark (\lr{IMF} مشروط) &
\cmark ($\sim$\$۵B) &
\cmark ($\sim$\$۶۰B+) &
\cmark ($\sim$\$۱B) &
\cmark (محدود) \\

\altrow
\textbf{مدل پیشنهادی برای ایران (ارجاع)} &
عنصر: مشوق عضویت &
عنصر: حمایت جامعهٔ مدنی &
عنصر: رفراندوم &
مدل ۶ اصلی &
عنصر: \lr{DDR} سپاه &
عنصر: مدیریت مرحله‌ای &
\textbf{ضد الگو} &
عنصر: گفت‌وگوی ملی &
درس: خطر بازگشت \\

\end{longtable}
\end{landscape}

\begin{recommendation}
\textbf{مدل ۶ (ترکیبی-تطبیقی) پیشنهادی برای ایران} بر مبنای مقایسهٔ این جدول طراحی شده: ناظران انتخاباتی از شیلی و آفریقای جنوبی، مشاوره فنی از لهستان، نظارت ساختاری از آفریقای جنوبی، تضمین‌های اجرایی از تیمور شرقی — اما بدون مدیریت مستقیم (ضد الگوی عراق). مالکیت ملی ایرانی: اصل غیرقابل‌مذاکره (\seeChapter{ch:approaches}).
\end{recommendation}

\sectiondivider

%═══════════════════════════════════════════════════════════
\section{جدول ششم: عدالت انتقالی}
\label{app:a:table6}
%═══════════════════════════════════════════════════════════

\begin{landscape}
\pagestyle{empty}
\bigtablefontsize

\begin{longtable}{
  >{\raggedleft\arraybackslash}p{2.1cm}|
  >{\raggedleft\arraybackslash}p{1.7cm}
  >{\raggedleft\arraybackslash}p{1.7cm}
  >{\raggedleft\arraybackslash}p{1.7cm}
  >{\raggedleft\arraybackslash}p{1.7cm}
  >{\raggedleft\arraybackslash}p{1.7cm}
  >{\raggedleft\arraybackslash}p{1.7cm}
  >{\raggedleft\arraybackslash}p{1.7cm}
  >{\raggedleft\arraybackslash}p{1.7cm}
  >{\raggedleft\arraybackslash}p{1.7cm}
}
\caption{مقایسهٔ سازوکارهای عدالت انتقالی}
\label{tab:app-a-tj} \\

\toprule
\headerrow
\rot{\textbf{شاخص}} &
\rot{\textbf{اسپانیا}} &
\rot{\textbf{لهستان}} &
\rot{\textbf{شیلی}} &
\rot{\textbf{آفریقای جنوبی}} &
\rot{\textbf{اندونزی}} &
\rot{\textbf{تیمور شرقی}} &
\rot{\textbf{عراق}} &
\rot{\textbf{تونس}} &
\rot{\textbf{میانمار}} \\
\midrule
\endfirsthead

\multicolumn{10}{c}{\small\textit{ادامهٔ جدول \ref{tab:app-a-tj}: عدالت انتقالی}} \\
\toprule
\headerrow
\rot{\textbf{شاخص}} &
\rot{\textbf{اسپانیا}} &
\rot{\textbf{لهستان}} &
\rot{\textbf{شیلی}} &
\rot{\textbf{آفریقای جنوبی}} &
\rot{\textbf{اندونزی}} &
\rot{\textbf{تیمور شرقی}} &
\rot{\textbf{عراق}} &
%══════════════════════════════════════════════════════════════
% تکمیل جدول ششم + جداول ۷-۱۰ + نمودار حبابی
% این بلوک بین endhead جدول ششم و نمودار حبابی قرار می‌گیرد
%══════════════════════════════════════════════════════════════

% ── ادامهٔ endhead جدول ششم ──
\rot{\textbf{تونس}} &
\rot{\textbf{میانمار}} \\
\midrule
\endhead

\bottomrule
\endlastfoot

%--- ردیف‌های جدول ششم ---

\textbf{کمیسیون حقیقت} &
\xmark (عفو ۱۹۷۷) &
$\sim$ (\lr{IPN} بایگانی) &
\cmark (کمیسیون رِتیگ ۱۹۹۱ + والِش ۲۰۰۴) &
\cmark (\lr{TRC} ۱۹۹۵-۲۰۰۲، ریاست توتو) &
\xmark (لایحه رد شد) &
\cmark (\lr{CAVR} ۲۰۰۲-۲۰۰۵) &
\xmark &
\cmark (\lr{IVD} ۲۰۱۴-۲۰۱۹) &
\xmark \\

\altrow
\textbf{دادگاه ویژه / محاکمات} &
\xmark &
$\sim$ (محدود، اواخر) &
\cmark (پینوشه ۱۹۹۸ بازداشت لندن) &
$\sim$ (محدود، بیشتر عفو) &
\xmark &
\cmark (دادگاه ویژه ترکیبی \lr{UN}) &
\cmark (دادگاه صدام ۲۰۰۶) &
$\sim$ (محاکمهٔ بن‌علی غیابی) &
\xmark \\

\textbf{مکانیزم عفو} &
\cmark (عفو عمومی بلانکت ۱۹۷۷) &
\xmark &
$\sim$ (عفو نظامی ۱۹۷۸ — بعداً لغو) &
\cmark (عفو مشروط فردی در \lr{TRC}) &
\xmark (بدون مکانیزم) &
$\sim$ &
\xmark &
$\sim$ &
\xmark \\

\altrow
\textbf{جبران خسارت قربانیان} &
\cmark (قانون ۲۰۰۷ — ۳۰ سال بعد!) &
$\sim$ (بسیار محدود) &
\cmark (غرامت مالی + نمادین) &
\cmark (برنامهٔ جبران، ناقص) &
$\sim$ (بسیار محدود) &
\cmark (برنامهٔ جبران ملی) &
\xmark &
\cmark (صندوق جبران) &
\xmark \\

\textbf{لوستراسیون / پاک‌سازی} &
\xmark &
\cmark (قانون ۱۹۹۷ — محدود و جنجالی) &
$\sim$ (ممنوعیت مناصب محدود) &
$\sim$ (عمدتاً نه) &
\xmark &
$\sim$ &
\cmark (بعث‌زدایی افراطی — فاجعه) &
$\sim$ (\lr{IVD} توصیه کرد، اجرا نشد) &
\xmark \\

\altrow
\textbf{تعداد تقریبی قربانیان رسیدگی‌شده} &
--- &
$\sim$۲۰,۰۰۰ پرونده &
$\sim$۳,۱۰۰ ناپدید + ۳۰,۰۰۰+ شکنجه &
۲۱,۰۰۰ شهادت، ۷,۱۱۲ درخواست عفو &
$\sim$۱۰۰ پرونده &
$\sim$۸,۰۰۰ شهادت &
$\sim$۳۰,۰۰۰ بعثی محکوم/اخراج &
$\sim$۶۲,۰۰۰ پرونده &
--- \\

\textbf{ارزیابی کلی عدالت انتقالی} &
\cellred{ناکافی: عفو بدون حقیقت} &
\cellorange{متوسط: تأخیر ۸ سال} &
\cellgreen{مؤثر: ترکیب کمیسیون + دادگاه} &
\cellgreen{الگویی: \lr{TRC} مدل جهانی شد} &
\cellred{ناکافی: بدون مکانیزم} &
\cellorange{نسبتاً مؤثر} &
\cellred{فاجعه: بعث‌زدایی = جنگ داخلی} &
\cellorange{مؤثر اما ناتمام} &
\cellred{صفر} \\

\altrow
\textbf{درس برای ایران} &
عفو بدون حقیقت نارضایتی می‌آورد &
تأخیر پذیرفتنی اما نه بیش از ۳ سال &
ترکیب حقیقت + محاکمه بهترین &
مدل \lr{TRC} با تعدیل ایرانی &
بدون عدالت ≠ ثبات &
مدل ترکیبی قابل‌استفاده &
اجتثاث افراطی = خطر مرگبار &
شروع سریع + صبر برای تکمیل &
بدون عدالت = بازگشت \\

\end{longtable}
\end{landscape}

\begin{casestudy}{مقایسهٔ سه مدل عدالت انتقالی}
\textbf{مدل اسپانیایی (فراموشی):} عفو عمومی ۱۹۷۷ باعث «پیمان فراموشی» شد. ثبات کوتاه‌مدت فراهم کرد اما زخم‌های بلندمدت ماند. ۴۰ سال بعد (۲۰۰۷) قانون حافظهٔ تاریخی تصویب شد — خیلی دیر. \textbf{مدل عراقی (انتقام):} بعث‌زدایی بدون تمایز، ۴۰۰,۰۰۰ نفر را اخراج و میلیون‌ها نفر را به‌حاشیه راند. نتیجه: داعش. \textbf{مدل آفریقای جنوبی (آشتی):} عفو فردی مشروط به اعتراف علنی و کامل. حقیقت آشکار شد بدون جنگ داخلی. \textbf{توصیه برای ایران:} مدل سوم با تقویت بُعد محاکمهٔ آمران جنایات (\seeChapter{ch:requirements}).
\end{casestudy}

\sectiondivider

%═══════════════════════════════════════════════════════════
\section{جدول هفتم: اصلاحات بخش امنیتی}
\label{app:a:table7}
%═══════════════════════════════════════════════════════════

\begin{landscape}
\pagestyle{empty}
\bigtablefontsize

\begin{longtable}{
  >{\raggedleft\arraybackslash}p{2.1cm}|
  >{\raggedleft\arraybackslash}p{1.7cm}
  >{\raggedleft\arraybackslash}p{1.7cm}
  >{\raggedleft\arraybackslash}p{1.7cm}
  >{\raggedleft\arraybackslash}p{1.7cm}
  >{\raggedleft\arraybackslash}p{1.7cm}
  >{\raggedleft\arraybackslash}p{1.7cm}
  >{\raggedleft\arraybackslash}p{1.7cm}
  >{\raggedleft\arraybackslash}p{1.7cm}
  >{\raggedleft\arraybackslash}p{1.7cm}
}
\caption{مقایسهٔ اصلاحات بخش امنیتی (\lr{SSR})}
\label{tab:app-a-ssr} \\

\toprule
\headerrow
\rot{\textbf{شاخص}} &
\rot{\textbf{اسپانیا}} &
\rot{\textbf{لهستان}} &
\rot{\textbf{شیلی}} &
\rot{\textbf{آفریقای جنوبی}} &
\rot{\textbf{اندونزی}} &
\rot{\textbf{تیمور شرقی}} &
\rot{\textbf{عراق}} &
\rot{\textbf{تونس}} &
\rot{\textbf{میانمار}} \\
\midrule
\endfirsthead

\multicolumn{10}{c}{\small\textit{ادامهٔ جدول \ref{tab:app-a-ssr}: اصلاحات بخش امنیتی}} \\
\toprule
\headerrow
\rot{\textbf{شاخص}} &
\rot{\textbf{اسپانیا}} &
\rot{\textbf{لهستان}} &
\rot{\textbf{شیلی}} &
\rot{\textbf{آفریقای جنوبی}} &
\rot{\textbf{اندونزی}} &
\rot{\textbf{تیمور شرقی}} &
\rot{\textbf{عراق}} &
\rot{\textbf{تونس}} &
\rot{\textbf{میانمار}} \\
\midrule
\endhead

\bottomrule
\endlastfoot

\textbf{اندازهٔ نیروهای مسلح (زمان گذار)} &
$\sim$۳۰۰K &
$\sim$۴۰۰K &
$\sim$۱۰۰K &
$\sim$۹۰K (\lr{SADF}) + ۲۰K+ (\lr{MK}) &
$\sim$۴۰۰K (\lr{TNI}) &
-- (اشغالگر) &
$\sim$۴۰۰K + ۵۰K حرس &
$\sim$۵۰K &
$\sim$۴۰۰K (تاتمادو) \\

\altrow
\textbf{سهم اقتصادی ارتش} &
\riskmedium متوسط &
\risklow پایین &
\riskmedium متوسط &
\riskmedium متوسط &
\riskhigh بسیار بالا (۲۰-۳۰٪) &
-- &
\riskhigh بالا &
\risklow پایین &
\riskhigh بسیار بالا (۴۰٪+) \\

\textbf{استراتژی اصلاح} &
حفظ + تبعیت تدریجی &
ادغام در \lr{NATO} &
حفظ خودمختاری (تدریجاً کاهش) &
ادغام \lr{SADF+MK+APLA} در \lr{SANDF} &
تفکیک تدریجی نقش نظامی/اقتصادی &
ایجاد نیروی جدید (\lr{F-FDTL}) &
\cellred{انحلال کامل ارتش} &
بی‌طرف‌سازی + اصلاح تدریجی &
\cellred{هیچ اصلاحی (فریب)} \\

\altrow
\textbf{برنامهٔ \lr{DDR}} &
\xmark (نیازی نبود) &
\xmark &
\xmark &
\cmark (ادغام چندنیرویی) &
$\sim$ (محدود) &
\cmark (\lr{FALINTIL} → \lr{F-FDTL}) &
\cellred{\xmark انحلال بدون \lr{DDR}} &
$\sim$ &
\xmark \\

\textbf{نظارت مدنی بر ارتش} &
\cmark (تدریجی — کودتای نافرجام ۲۳-F ۱۹۸۱) &
\cmark (وزیر دفاع غیرنظامی) &
$\sim$ (پینوشه فرمانده ماند تا ۱۹۹۸) &
\cmark (وزیر دفاع غیرنظامی) &
$\sim$ (هنوز نفوذ) &
\cmark (از ابتدا مدنی) &
\cellred{\xmark} &
\cmark &
\cellred{\xmark (ارتش ۲۵٪ پارلمان)} \\

\altrow
\textbf{خلع‌سلاح هسته‌ای} &
-- &
-- &
-- &
\cmark (۶ کلاهک تحویل داوطلبانه ۱۹۹۳) &
-- &
-- &
\cmark (WMD وجود نداشت) &
-- &
-- \\

\textbf{اصلاح پلیس} &
\cmark (حل \lr{Guardia Civil} → پلیس ملی) &
\cmark &
$\sim$ &
\cmark (ادغام \lr{SAP+KZP}→\lr{SAPS}) &
$\sim$ (محدود) &
\cmark (ایجاد \lr{PNTL}) &
$\sim$ &
$\sim$ &
\xmark \\

\altrow
\textbf{نتیجهٔ \lr{SSR}} &
\cellgreen{موفق — ارتش حرفه‌ای غیرسیاسی} &
\cellgreen{موفق — عضو \lr{NATO}} &
\cellorange{ناقص — ارتش تا ۱۹۹۸ خودمختار} &
\cellgreen{موفق — \lr{SANDF} چندنژادی} &
\cellorange{ناقص — نفوذ ادامه دارد} &
\cellorange{شکننده — بحران ۲۰۰۶} &
\cellred{فاجعه — منشأ شورش و داعش} &
\cellorange{نسبی — بدون اصلاح ساختاری} &
\cellred{فاجعه — کودتای ۲۰۲۱} \\

\textbf{درس برای سپاه ایران} &
صبر + تبعیت تدریجی &
مشوق خارجی (معادل \lr{NATO}) &
پذیرش موقت خودمختاری &
ادغام بهترین مدل &
تفکیک نقش اقتصادی حیاتی &
ایجاد نهاد جدید اگر لازم &
\textbf{هرگز انحلال!} &
بی‌طرف‌سازی ممکن اگر ارتش حرفه‌ای &
گشایش کنترل‌شده ≠ اصلاح \\

\end{longtable}
\end{landscape}

\begin{warningbox}
\textbf{درس حیاتی برای سپاه پاسداران:} سپاه با $\sim$۱۹۰,۰۰۰ نفر نیروی نظامی، ۲۰-۴۰٪ اقتصاد، شبکهٔ اطلاعاتی گسترده، و ایدئولوژی تثبیت‌شده، ترکیبی از ارتش اندونزی (\lr{TNI})، حرس جمهوری عراق، و تاتمادوی میانمار است — و به‌مراتب پیچیده‌تر. تنها مدل موفقِ مقایسه‌ای: \textbf{ادغام تدریجی آفریقای جنوبی} + \textbf{تفکیک اقتصادی اندونزی} + \textbf{مشوق بین‌المللی لهستان}. انحلال (مدل عراق) \emphred{به هیچ وجه} نباید تکرار شود (\seeChapter{ch:guarantees}).
\end{warningbox}

\sectiondivider

%═══════════════════════════════════════════════════════════
\section{جدول هشتم: نتایج اقتصادی و اجتماعی گذار}
\label{app:a:table8}
%═══════════════════════════════════════════════════════════

\begin{landscape}
\pagestyle{empty}
\bigtablefontsize

\begin{longtable}{
  >{\raggedleft\arraybackslash}p{2.1cm}|
  >{\raggedleft\arraybackslash}p{1.7cm}
  >{\raggedleft\arraybackslash}p{1.7cm}
  >{\raggedleft\arraybackslash}p{1.7cm}
  >{\raggedleft\arraybackslash}p{1.7cm}
  >{\raggedleft\arraybackslash}p{1.7cm}
  >{\raggedleft\arraybackslash}p{1.7cm}
  >{\raggedleft\arraybackslash}p{1.7cm}
  >{\raggedleft\arraybackslash}p{1.7cm}
  >{\raggedleft\arraybackslash}p{1.7cm}
}
\caption{نتایج اقتصادی و اجتماعی گذار}
\label{tab:app-a-econ} \\

\toprule
\headerrow
\rot{\textbf{شاخص}} &
\rot{\textbf{اسپانیا}} &
\rot{\textbf{لهستان}} &
\rot{\textbf{شیلی}} &
\rot{\textbf{آفریقای جنوبی}} &
\rot{\textbf{اندونزی}} &
\rot{\textbf{تیمور شرقی}} &
\rot{\textbf{عراق}} &
\rot{\textbf{تونس}} &
\rot{\textbf{میانمار}} \\
\midrule
\endfirsthead

\multicolumn{10}{c}{\small\textit{ادامهٔ جدول \ref{tab:app-a-econ}: نتایج اقتصادی}} \\
\toprule
\headerrow
\rot{\textbf{شاخص}} &
\rot{\textbf{اسپانیا}} &
\rot{\textbf{لهستان}} &
\rot{\textbf{شیلی}} &
\rot{\textbf{آفریقای جنوبی}} &
\rot{\textbf{اندونزی}} &
\rot{\textbf{تیمور شرقی}} &
\rot{\textbf{عراق}} &
\rot{\textbf{تونس}} &
\rot{\textbf{میانمار}} \\
\midrule
\endhead

\bottomrule
\endlastfoot

\textbf{رشد \lr{GDP} ۵ سال اول} &
\cellgreen{+۳.۵٪ میانگین} &
\cellred{-۷٪ (۱۹۹۰) سپس +۵٪} &
\cellgreen{+۷٪ میانگین} &
\cellorange{+۲.۵٪ میانگین} &
\cellred{-۱۳٪ (۱۹۹۸) سپس +۴٪} &
\cellorange{+۲٪ (وابسته)} &
\cellred{-۳۰٪ (۲۰۰۳) سپس نوسانی} &
\cellorange{+۱.۵٪ ضعیف} &
\cellorange{+۶٪ (آمارهای مشکوک)} \\

\altrow
\textbf{تورم فاز انتقالی} &
بالا (۲۰٪) — تدریجاً کاهش &
بسیار بالا (۵۸۶٪ ۱۹۹۰) — شوک‌درمانی &
پایین (کنترل‌شده) &
متوسط (۹٪) &
بالا (۷۸٪ ۱۹۹۸) &
بالا &
بسیار بالا &
متوسط (۵٪) &
بالا \\

\textbf{بیکاری} &
بالا (۱۵-۲۰٪) &
بالا (۱۵٪ — شوک) &
متوسط (۸٪) &
بسیار بالا (۲۵-۳۰٪ — نماند) &
بالا (۱۵٪+) &
بسیار بالا &
بسیار بالا (۵۰٪+) &
بالا (۱۵-۱۸٪ جوانان ۳۵٪) &
بالا \\

\altrow
\textbf{نابرابری (جینی)} &
۰.۳۴ → ۰.۳۲ (بهبود) &
۰.۲۷ → ۰.۳۴ (بدتر) &
۰.۵۶ → ۰.۴۷ (بهبود کند) &
۰.۵۹ → ۰.۶۳ (بدتر!) &
۰.۳۶ → ۰.۳۹ (بدتر) &
--- &
--- (بدتر) &
۰.۳۶ → ۰.۴۰ (بدتر) &
--- \\

\textbf{فقر} &
کاهش تدریجی &
افزایش موقت سپس کاهش شدید &
کاهش قابل‌توجه (۴۵٪→۱۲٪) &
کاهش (۵۰٪→۲۵٪) اما ناکافی &
کاهش تدریجی &
بسیار بالا (ماند) &
افزایش شدید &
بدون تغییر محسوس &
بسیار بالا (ماند) \\

\altrow
\textbf{نقش تحریم/رفع آن} &
-- (تحریم نبود) &
-- &
$\sim$ (تحریم محدود — رفع) &
\cmark (رفع تحریم = محرک رشد) &
\cmark (\lr{IMF} مشروط) &
-- &
\cmark (برداشته شد — اثر محدود) &
-- &
\cmark (تحریم‌ها باقی ماند) \\

\textbf{کمک بین‌المللی مالی} &
\cmark (صندوق‌های \lr{EC}) &
\cmark (\lr{PHARE}: \$۱.۵B+) &
$\sim$ (محدود) &
\cmark (\lr{EU + US}: \$۱B+) &
\cmark (\lr{IMF}: \$۴۳B) &
\cmark ($\sim$\$۵B) &
\cmark ($\sim$\$۶۰B+) &
\cmark ($\sim$\$۱B) &
$\sim$ (محدود) \\

\altrow
\textbf{ارزیابی کلی اقتصادی} &
\cellgreen{موفق — عضو \lr{EU}} &
\cellgreen{موفق — معجزهٔ لهستان} &
\cellgreen{بسیار موفق — ببر آمریکای لاتین} &
\cellorange{مختلط — رشد اما نابرابری} &
\cellorange{بازیابی تدریجی} &
\cellorange{وابسته به کمک} &
\cellred{فاجعه} &
\cellred{ضعیف — عامل عقب‌گرد} &
\cellorange{رشد صوری بدون توسعه} \\

\textbf{درس اقتصادی برای ایران} &
مشوق عضویت = تسریع‌کننده &
شوک‌درمانی هزینهٔ اجتماعی دارد &
ثبات ماکرو + رشد = ضامن دموکراسی &
رشد بدون عدالت = ناپایداری &
بستهٔ \lr{IMF} با شرط = دوسویه &
وابستگی بلندمدت خطرناک &
بازسازی بدون برنامه = هدررفت &
اقتصاد ضعیف = سرخوردگی = بازگشت &
رشد بدون آزادی ≠ توسعه \\

\end{longtable}
\end{landscape}

\begin{lessonlearned}
\textbf{فرمول اقتصادی گذار موفق (از مقایسهٔ ۹ نمونه):}
\begin{enumerate}[nosep]
    \item \textbf{تثبیت فوری} (ماه ۱-۶): جلوگیری از فروپاشی ارزی و بانکی (درس اندونزی ۱۹۹۸)
    \item \textbf{رفع تحریم سریع} (ماه ۱-۳): آفریقای جنوبی نشان داد رفع تحریم یکی از مؤثرترین محرک‌هاست
    \item \textbf{بستهٔ کمک هدفمند} (ماه ۳-۱۲): نه مثل عراق (پول‌پاشی بدون نظارت) بلکه مثل لهستان (\lr{PHARE} مشروط)
    \item \textbf{اصلاحات ساختاری تدریجی} (سال ۱-۵): مدل شیلی (ثبات + آزادسازی تدریجی) نه شوک لهستانی
    \item \textbf{شبکهٔ ایمنی اجتماعی}: در همهٔ نمونه‌هایی که اقشار آسیب‌پذیر فراموش شدند، مردم از دموکراسی سرخورده شدند
\end{enumerate}
\end{lessonlearned}

\sectiondivider

%═══════════════════════════════════════════════════════════
\section{جدول نهم: نتایج سیاسی و شاخص‌های فعلی}
\label{app:a:table9}
%═══════════════════════════════════════════════════════════

\begin{landscape}
\pagestyle{empty}
\bigtablefontsize

\begin{longtable}{
  >{\raggedleft\arraybackslash}p{2.1cm}|
  >{\raggedleft\arraybackslash}p{1.7cm}
  >{\raggedleft\arraybackslash}p{1.7cm}
  >{\raggedleft\arraybackslash}p{1.7cm}
  >{\raggedleft\arraybackslash}p{1.7cm}
  >{\raggedleft\arraybackslash}p{1.7cm}
  >{\raggedleft\arraybackslash}p{1.7cm}
  >{\raggedleft\arraybackslash}p{1.7cm}
  >{\raggedleft\arraybackslash}p{1.7cm}
  >{\raggedleft\arraybackslash}p{1.7cm}
}
\caption{نتایج سیاسی و شاخص‌های فعلی (۲۰۲۳-۲۰۲۴)}
\label{tab:app-a-results} \\

\toprule
\headerrow
\rot{\textbf{شاخص}} &
\rot{\textbf{اسپانیا}} &
\rot{\textbf{لهستان}} &
\rot{\textbf{شیلی}} &
\rot{\textbf{آفریقای جنوبی}} &
\rot{\textbf{اندونزی}} &
\rot{\textbf{تیمور شرقی}} &
\rot{\textbf{عراق}} &
\rot{\textbf{تونس}} &
\rot{\textbf{میانمار}} \\
\midrule
\endfirsthead

\multicolumn{10}{c}{\small\textit{ادامهٔ جدول \ref{tab:app-a-results}: نتایج سیاسی}} \\
\toprule
\headerrow
\rot{\textbf{شاخص}} &
\rot{\textbf{اسپانیا}} &
\rot{\textbf{لهستان}} &
\rot{\textbf{شیلی}} &
\rot{\textbf{آفریقای جنوبی}} &
\rot{\textbf{اندونزی}} &
\rot{\textbf{تیمور شرقی}} &
\rot{\textbf{عراق}} &
\rot{\textbf{تونس}} &
\rot{\textbf{میانمار}} \\
\midrule
\endhead

\bottomrule
\endlastfoot

\textbf{تعداد انتقال مسالمت‌آمیز قدرت} &
\cellgreen{۸+} &
\cellgreen{۶+} &
\cellgreen{۶+} &
\cellgreen{۴} &
\cellgreen{۴} &
\cellorange{۳} &
\cellorange{۲ (با بحران)} &
\cellred{۱ (سپس کودتا)} &
\cellred{۱ (سپس کودتا)} \\

\altrow
\textbf{وضعیت \lr{Freedom House} (۲۰۲۴)} &
\cellgreen{آزاد (۱.۰)} &
\cellgreen{آزاد (۲.۰)} &
\cellgreen{آزاد (۱.۰)} &
\cellgreen{آزاد (۲.۰)} &
\cellorange{نیمه‌آزاد (۳.۰)} &
\cellorange{نیمه‌آزاد (۳.۰)} &
\cellred{غیرآزاد (۵.۵)} &
\cellred{غیرآزاد (۳.۵)} &
\cellred{غیرآزاد (۶.۵)} \\

\textbf{وضعیت آزادی مطبوعات (\lr{RSF} ۲۰۲۴)} &
\cellgreen{رتبهٔ ۳۶} &
\cellorange{رتبهٔ ۴۷} &
\cellgreen{رتبهٔ ۵۲} &
\cellorange{رتبهٔ ۵۵} &
\cellorange{رتبهٔ ۱۱۱} &
\cellorange{رتبهٔ ۱۰۵} &
\cellred{رتبهٔ ۱۶۹} &
\cellred{رتبهٔ ۱۱۸} &
\cellred{رتبهٔ ۱۷۱} \\

\altrow
\textbf{وضعیت \lr{TI CPI} (فساد) ۲۰۲۳} &
\cellgreen{۶۰} &
\cellorange{۵۴} &
\cellgreen{۶۷} &
\cellorange{۴۱} &
\cellorange{۳۴} &
\cellorange{۴۲} &
\cellred{۲۳} &
\cellorange{۴۰} &
\cellred{۲۰} \\

\textbf{درصد زنان در پارلمان} &
\cellgreen{۴۴٪} &
\cellorange{۲۹٪} &
\cellgreen{۳۵٪} &
\cellorange{۲۸٪} &
\cellorange{۲۲٪} &
\cellgreen{۳۸٪} &
\cellorange{۲۶٪} &
\cellorange{۲۶٪} &
--- \\

\altrow
\textbf{کودتا یا بازگشت اقتدارگرایی} &
\cmark نافرجام (۲۳-F ۱۹۸۱) &
$\sim$ (افت دموکراتیک \lr{PiS} ۲۰۱۵-۲۰۲۳) &
\xmark &
\xmark (اما فساد \lr{ANC}) &
\xmark (اما افت اخیر) &
$\sim$ (بحران ۲۰۰۶) &
$\sim$ (بی‌ثباتی مزمن) &
\cmark (سعید ۲۰۲۱) &
\cmark (کودتای ۲۰۲۱) \\

\textbf{وضعیت فعلی نظام سیاسی} &
\cellgreen{دموکراسی تحکیم‌یافتهٔ پارلمانی} &
\cellgreen{دموکراسی بازیابی‌شده (پس از ۲۰۲۳)} &
\cellgreen{دموکراسی تحکیم‌یافته} &
\cellgreen{دموکراسی با چالش} &
\cellorange{دموکراسی انتخاباتی ناقص} &
\cellorange{دموکراسی شکننده} &
\cellred{اقتدارگرایی رقابتی} &
\cellred{اقتدارگرایی نوین} &
\cellred{خونتای نظامی + جنگ داخلی} \\

\altrow
\textbf{آیا «تحکیم» محقق شد؟} &
\cmark (قطعی) &
\cmark (با نوسان) &
\cmark (قطعی) &
\cmark (با چالش) &
$\sim$ (ناقص) &
$\sim$ (شکننده) &
\xmark &
\xmark (بازگشت) &
\xmark (بازگشت) \\

\end{longtable}
\end{landscape}

\sectiondivider

%═══════════════════════════════════════════════════════════
\section{جدول دهم: نگاشت سناریوها به نمونه‌ها}
\label{app:a:table10}
%═══════════════════════════════════════════════════════════

\begin{table}[htbp]
\centering
\caption{نگاشت سناریوهای گذار ایران (فصل ۴) به نمونه‌های تاریخی}
\label{tab:app-a-mapping}
\begin{tabularx}{\textwidth}{>{\raggedleft\arraybackslash}p{3cm}
                             >{\raggedleft\arraybackslash}p{3cm}
                             >{\raggedleft\arraybackslash}X
                             >{\centering\arraybackslash}p{2cm}}
\toprule
\headerrow سناریوی ایران & نمونهٔ نزدیک‌ترین & درس اصلی & نتیجهٔ نمونه \\
\midrule
A: فروپاشی ناگهانی & اندونزی ۱۹۹۸ + لیبی ۲۰۱۱ & سرعت + خلأ = خطر شدید؛ نیاز به آمادگی قبلی & \statuswarn مختلط \\
\altrow B: مذاکره‌ای (مطلوب) & آفریقای جنوبی ۱۹۹۰ + اسپانیا ۱۹۷۵ & صبر + فراگیری + آشتی = بهترین نتیجه & \statusok موفق \\
C: انقلاب مردمی & تونس ۲۰۱۱ + اندونزی ۱۹۹۸ & سرعت ← نهادسازی سریع ضروری؛ خطر بازگشت & \statuswarn مختلط \\
\altrow D: تحول از درون & اسپانیا ۱۹۷۵ + میانمار ۲۰۱۰ & اصلاح‌طلب واقعی لازم (خوان‌کارلوس ≠ تاتمادو) & \statuswarn خطرناک \\
E: مداخلهٔ نظامی (رد) & عراق ۲۰۰۳ & \emphred{به هیچ وجه تکرار نشود} & \statusbad فاجعه \\
\altrow F: بحران ممتد & ونزوئلا + لبنان & خستگی = فرسایش ← فرصت از دست رفته & \statusbad بن‌بست \\
\bottomrule
\end{tabularx}
\end{table}

\begin{keypoint}
\textbf{مهم‌ترین یافتهٔ نگاشت:} سناریوی B (مذاکره‌ای) تنها سناریویی است که نمونه‌های تاریخی آن (آفریقای جنوبی، اسپانیا، لهستان) همگی به دموکراسی تحکیم‌یافته رسیده‌اند. سایر سناریوها نتایج مختلط یا منفی دارند. این تأیید مجدد اولویت مدل مذاکره‌ای است — هرچند آمادگی برای سناریوهای دیگر ضروری است (\seeChapter{ch:scenarios}).
\end{keypoint}

\sectiondivider

%═══════════════════════════════════════════════════════════
\section{نمودار حبابی: هزینه در مقابل نتیجه}
\label{app:a:bubble}
%═══════════════════════════════════════════════════════════

\begin{figure}[htbp]
\centering
\begin{tikzpicture}
\begin{axis}[
    width=14cm,
    height=10cm,
    xlabel={هزینهٔ بین‌المللی (میلیارد دلار)},
    ylabel={شاخص دموکراسی \lr{V-Dem} (۲۰۲۳)},
    xmin=-2, xmax=70,
    ymin=0, ymax=1,
    xtick={0,10,20,30,40,50,60},
    ytick={0,0.2,0.4,0.6,0.8,1.0},
    x tick label style={font=\footnotesize},
    y tick label style={font=\footnotesize},
    xlabel style={font=\small},
    ylabel style={font=\small},
    legend pos=outer north east,
    legend style={font=\tiny, draw=none, fill=none},
    grid=major,
    grid style={gray!20},
    clip=false,
]

% حباب‌ها: (هزینه بین‌المللی میلیارد$, V-Dem Score, اندازه=جمعیت/۱۰M)
% اسپانیا
\addplot[
  only marks, mark=*, mark size=5pt,
  color=MainBlue, fill=MainBlue!40
] coordinates {(2, 0.81)};
\addlegendentry{اسپانیا}

% لهستان
\addplot[
  only marks, mark=*, mark size=5pt,
  color=MainBlue!70, fill=MainBlue!25
] coordinates {(5, 0.55)};
\addlegendentry{لهستان}

% شیلی
\addplot[
  only marks, mark=*, mark size=3pt,
  color=MainGreen, fill=MainGreen!40
] coordinates {(1, 0.79)};
\addlegendentry{شیلی}

% آفریقای جنوبی
\addplot[
  only marks, mark=*, mark size=5.5pt,
  color=MainGreen!70, fill=MainGreen!25
] coordinates {(3, 0.72)};
\addlegendentry{آفریقای جنوبی}

% اندونزی
\addplot[
  only marks, mark=*, mark size=8pt,
  color=MainOrange, fill=MainOrange!30
] coordinates {(8, 0.52)};
\addlegendentry{اندونزی}

% تیمور شرقی
\addplot[
  only marks, mark=*, mark size=2pt,
  color=MainPurple, fill=MainPurple!30
] coordinates {(5, 0.58)};
\addlegendentry{تیمور شرقی}

% عراق
\addplot[
  only marks, mark=*, mark size=5pt,
  color=MainRed, fill=MainRed!30
] coordinates {(60, 0.25)};
\addlegendentry{عراق}

% تونس
\addplot[
  only marks, mark=*, mark size=3pt,
  color=MainYellow!80!black, fill=MainYellow!30
] coordinates {(1.5, 0.18)};
\addlegendentry{تونس}

% میانمار
\addplot[
  only marks, mark=*, mark size=6pt,
  color=MainRed!70, fill=MainRed!15
] coordinates {(2, 0.08)};
\addlegendentry{میانمار}

% ایران (پیش‌بینی — ستاره)
\addplot[
  only marks, mark=star, mark size=7pt,
  color=MainPurple, fill=MainPurple!50,
  line width=1.5pt
] coordinates {(4, 0.65)};
\addlegendentry{ایران (هدف مدل ۶)}

% خط هدف
\draw[dashed, MainGreen!60, thick] (axis cs:0,0.6) -- (axis cs:70,0.6);
\node[font=\tiny, MainGreen!80!black, anchor=west] at (axis cs:45,0.62) {آستانهٔ دموکراسی تحکیم‌یافته};

% ناحیهٔ «بازده بالا»
\fill[MainGreen!8] (axis cs:0,0.6) rectangle (axis cs:10,1);
\node[font=\tiny, MainGreen!60!black, rotate=90] at (axis cs:0.8,0.8) {بازده بالا / هزینهٔ پایین};

% ناحیهٔ «فاجعه»
\fill[MainRed!8] (axis cs:40,0) rectangle (axis cs:70,0.4);
\node[font=\tiny, MainRed!60!black] at (axis cs:55,0.15) {هزینهٔ بالا / شکست};

\end{axis}
\end{tikzpicture}
\caption{نمودار حبابی: هزینهٔ بین‌المللی در برابر نتیجهٔ دموکراتیک (اندازهٔ حباب $\propto$ جمعیت)}
\label{fig:app-a-bubble}
\end{figure}

\begin{keypoint}
\textbf{یافتهٔ کلیدی نمودار حبابی:} هیچ رابطهٔ مستقیمی بین \textbf{حجم هزینهٔ بین‌المللی} و \textbf{نتیجهٔ دموکراتیک} وجود ندارد. عراق با بیش از ۶۰ میلیارد دلار هزینهٔ بین‌المللی، بدترین نتیجه را داشت؛ شیلی با کمتر از ۱ میلیارد، بهترین. آنچه تعیین‌کننده است: \textbf{مالکیت ملی}، \textbf{مسیر مذاکره‌ای}، و \textbf{طراحی هوشمند نظارت}. بودجهٔ $۲.۵$-$۵$ میلیارد دلاری پیشنهادی برای ایران (\seeChapter{ch:budget}) در «ناحیهٔ بازده بالا» قرار می‌گیرد.
\end{keypoint}

\sectiondivider

%═══════════════════════════════════════════════════════════
\section{خلاصهٔ مقایسه‌ای: کارت امتیاز نُه نمونه}
\label{app:a:scorecard}
%═══════════════════════════════════════════════════════════

\begin{table}[htbp]
\centering
\caption{کارت امتیاز مقایسه‌ای: ارزیابی کلی نُه نمونه در ۸ بُعد}
\label{tab:app-a-scorecard}
\bigtablefontsize
\begin{tabularx}{\textwidth}{
  >{\raggedleft\arraybackslash}p{2.2cm}
  >{\centering\arraybackslash}p{1.1cm}
  >{\centering\arraybackslash}p{1.1cm}
  >{\centering\arraybackslash}p{1.1cm}
  >{\centering\arraybackslash}p{1.1cm}
  >{\centering\arraybackslash}p{1.1cm}
  >{\centering\arraybackslash}p{1.1cm}
  >{\centering\arraybackslash}p{1.1cm}
  >{\centering\arraybackslash}p{1.1cm}
  >{\centering\arraybackslash}p{1.3cm}
}
\toprule
\headerrow
\textbf{بُعد} &
\rot{\textbf{اسپانیا}} &
\rot{\textbf{لهستان}} &
\rot{\textbf{شیلی}} &
\rot{\textbf{آفریقا}} &
\rot{\textbf{اندونزی}} &
\rot{\textbf{تیمور}} &
\rot{\textbf{عراق}} &
\rot{\textbf{تونس}} &
\rot{\textbf{میانمار}} \\
\midrule

مسالمت‌آمیز بودن &
\starrating{4} &
\starrating{5} &
\starrating{4} &
\starrating{3} &
\starrating{3} &
\starrating{2} &
\starrating{1} &
\starrating{4} &
\starrating{3} \\

\altrow
توافق سیاسی &
\starrating{5} &
\starrating{5} &
\starrating{4} &
\starrating{5} &
\starrating{3} &
\starrating{3} &
\starrating{1} &
\starrating{4} &
\starrating{2} \\

عدالت انتقالی &
\starrating{1} &
\starrating{2} &
\starrating{4} &
\starrating{5} &
\starrating{1} &
\starrating{3} &
\starrating{1} &
\starrating{4} &
\starrating{0} \\

\altrow
اصلاح امنیتی &
\starrating{3} &
\starrating{4} &
\starrating{3} &
\starrating{4} &
\starrating{3} &
\starrating{3} &
\starrating{1} &
\starrating{3} &
\starrating{0} \\

قانون اساسی &
\starrating{5} &
\starrating{3} &
\starrating{3} &
\starrating{5} &
\starrating{3} &
\starrating{4} &
\starrating{2} &
\starrating{5} &
\starrating{1} \\

\altrow
فراگیری &
\starrating{3} &
\starrating{2} &
\starrating{2} &
\starrating{5} &
\starrating{2} &
\starrating{3} &
\starrating{1} &
\starrating{4} &
\starrating{2} \\

نتیجهٔ بلندمدت &
\starrating{5} &
\starrating{4} &
\starrating{5} &
\starrating{4} &
\starrating{3} &
\starrating{3} &
\starrating{1} &
\starrating{1} &
\starrating{1} \\

\altrow
انتقال به ایران &
\starrating{3} &
\starrating{3} &
\starrating{4} &
\starrating{5} &
\starrating{4} &
\starrating{3} &
\starrating{2} &
\starrating{3} &
\starrating{2} \\

\midrule
\headerrow
\textbf{مجموع (/۴۰)} &
\textbf{۲۹} &
\textbf{۲۸} &
\textbf{۲۹} &
\textbf{۳۶} &
\textbf{۲۲} &
\textbf{۲۴} &
\textbf{۱۰} &
\textbf{۲۸} &
\textbf{۱۱} \\

\bottomrule
\end{tabularx}
\end{table}

\begin{lessonlearned}
\textbf{آفریقای جنوبی با امتیاز ۳۶ از ۴۰}، جامع‌ترین الگوی قابل‌انتقال به ایران است. اسپانیا و شیلی (هر دو ۲۹) در بُعد نتیجهٔ بلندمدت برتر هستند اما در عدالت انتقالی (اسپانیا) یا فراگیری (شیلی) ضعیف‌تر. عراق (۱۰) و میانمار (۱۱) \textbf{ضد الگوهای اصلی} هستند. نکتهٔ حیاتی: حتی آفریقای جنوبی هم \textbf{مدل کامل نیست} — نابرابری اقتصادی و خشونت جنسیتی همچنان چالش‌های جدی هستند.
\end{lessonlearned}

\sectiondivider

%═══════════════════════════════════════════════════════════
\section{ده یافتهٔ کلان مقایسه‌ای}
\label{app:a:findings}
%═══════════════════════════════════════════════════════════

بر اساس تحلیل جامع نُه نمونه در ده بُعد، ده یافتهٔ کلان استخراج شده است:

\begin{enumerate}
\item \textbf{مسیر گذار تعیین‌کنندهٔ نتیجه است:} گذارهای \textbf{مذاکره‌ای} (اسپانیا، لهستان، آفریقای جنوبی) نتایج پایدارتری از گذارهای \textbf{یک‌جانبه} (مداخله، فروپاشی) داشته‌اند. احتمال موفقیت گذار مذاکره‌ای: $\sim$۷۵٪ در مقابل $\sim$۲۵٪ برای مداخله.

\item \textbf{اپوزیسیون سازمان‌یافته شرط لازم است:} در هر پنج نمونهٔ موفق، اپوزیسیون قبل از گذار \textbf{حداقل یک سازمان فراگیر} داشت (\lr{Solidarność}، \lr{ANC}، \lr{Concertación}). \emphorange{هشدار ایرانی:} اپوزیسیون فعلی ایران فاقد این ویژگی است.

\item \textbf{انحلال نیروهای امنیتی = فاجعه:} عراق این را اثبات کرد. مدل درست: \textbf{ادغام + تفکیک اقتصادی + نظارت مدنی تدریجی}. زمان مورد نیاز: ۵ تا ۱۵ سال.

\item \textbf{عدالت انتقالی بدون حقیقت، عدالت نیست:} مدل اسپانیایی (عفو بدون حقیقت) و مدل عراقی (انتقام بدون عدالت) هر دو شکست خوردند. مدل \lr{TRC} بهینه‌ترین است.

\item \textbf{قانون اساسی فراگیر سنگ بنای تحکیم است:} بهترین نتایج در کشورهایی حاصل شد که قانون اساسی از طریق \textbf{مجلس مؤسسان منتخب و فراگیر} نوشته شد (آفریقای جنوبی، تونس).

\item \textbf{مشوق اقتصادی بین‌المللی تسریع‌کننده است:} عضویت در \lr{EC/EU} (اسپانیا، لهستان) قوی‌ترین مشوق بود. \emphorange{چالش ایرانی:} معادل \lr{EU} برای ایران وجود ندارد؛ باید بسته‌ای ترکیبی (لغو تحریم + سرمایه‌گذاری + عضویت \lr{WTO}) طراحی شود.

\item \textbf{زنان و اقلیت‌ها: سهمیهٔ قانونی لازم است:} آفریقای جنوبی و تونس نشان دادند که بدون \textbf{سهمیهٔ قانونی}، مشارکت زنان و اقلیت‌ها در ساختارهای جدید محدود می‌ماند.

\item \textbf{نقش بین‌المللی مؤثر ≠ مدیریت مستقیم:} بهترین نتایج زمانی حاصل شد که نقش بین‌المللی \textbf{حمایتی-نظارتی} بود (مدل ۳-۴ از فصل ۳)، نه مدیریت مستقیم (مدل ۵). مدل ۶ پیشنهادی این درس را رعایت می‌کند.

\item \textbf{بُعد هسته‌ای ایران بی‌سابقه است:} تنها آفریقای جنوبی تجربهٔ خلع‌سلاح هسته‌ای داوطلبانه را داشت (۶ کلاهک، ۱۹۹۳). ایران با برنامه‌ای بسیار پیچیده‌تر، نیازمند \textbf{مکانیزم ویژه} است (\seeChapter{ch:guarantees}).

\item \textbf{هزینهٔ پیشگیری بسیار کمتر از هزینهٔ شکست است:} مقایسهٔ شیلی ($<$\$۱B) با عراق ($>$\$۶۰B بین‌المللی + \$۲T آمریکا) نشان می‌دهد که \textbf{سرمایه‌گذاری در مدل درست}، هزارها برابر ارزان‌تر از \textbf{مدیریت شکست} است.
\end{enumerate}

\sectiondivider

%═══════════════════════════════════════════════════════════
\section{نمودار تکمیلی: عوامل موفقیت و شکست}
\label{app:a:success-factors}
%═══════════════════════════════════════════════════════════

\begin{figure}[htbp]
\centering
\begin{tikzpicture}[
  node distance=0.6cm,
  factor/.style={
    draw, rounded corners=3pt, minimum width=6cm,
    minimum height=0.7cm, font=\small, align=center
  },
  success/.style={factor, fill=MainGreen!15, draw=MainGreen!60, text=DarkGray},
  failure/.style={factor, fill=MainRed!15, draw=MainRed!60, text=DarkGray},
  title/.style={font=\bfseries\small, anchor=south}
]

% عنوان‌ها
\node[title, MainGreen] at (-4, 6.5) {عوامل موفقیت (الگوها)};
\node[title, MainRed] at (4, 6.5) {عوامل شکست (ضد الگوها)};

% خط جداکننده
\draw[gray!40, thick, dashed] (0,-0.5) -- (0,6.5);

% عوامل موفقیت (چپ)
\node[success] at (-4, 5.8) {۱. گذار مذاکره‌ای (اسپانیا، آفریقای جنوبی)};
\node[success] at (-4, 5.0) {۲. اپوزیسیون سازمان‌یافته (لهستان، شیلی)};
\node[success] at (-4, 4.2) {۳. ادغام تدریجی نیروها (آفریقای جنوبی)};
\node[success] at (-4, 3.4) {۴. کمیسیون حقیقت (\lr{TRC})};
\node[success] at (-4, 2.6) {۵. قانون اساسی فراگیر (آفریقای جنوبی، تونس)};
\node[success] at (-4, 1.8) {۶. مشوق اقتصادی خارجی (اسپانیا/\lr{EC})};
\node[success] at (-4, 1.0) {۷. سهمیهٔ زنان/اقلیت‌ها (آفریقای جنوبی)};
\node[success] at (-4, 0.2) {۸. نظارت بین‌المللی حمایتی (مدل ۳-۴)};

% عوامل شکست (راست)
\node[failure] at (4, 5.8) {۱. مداخلهٔ نظامی خارجی (عراق)};
\node[failure] at (4, 5.0) {۲. انحلال کامل ارتش (عراق)};
\node[failure] at (4, 4.2) {۳. اجتثاث بدون عدالت (عراق)};
\node[failure] at (4, 3.4) {۴. عفو بدون حقیقت (اسپانیا)};
\node[failure] at (4, 2.6) {۵. گشایش صوری (میانمار)};
\node[failure] at (4, 1.8) {۶. بدون \lr{SSR} واقعی (میانمار)};
\node[failure] at (4, 1.0) {۷. فقدان اپوزیسیون منسجم (عراق)};
\node[failure] at (4, 0.2) {۸. مدیریت مستقیم خارجی (عراق، تیمور)};

\end{tikzpicture}
\caption{مقایسهٔ بصری ۸ عامل موفقیت و ۸ عامل شکست از نُه نمونهٔ تاریخی}
\label{fig:app-a-factors}
\end{figure}

\sectiondivider

%═══════════════════════════════════════════════════════════
\section{جمع‌بندی پیوست}
\label{app:a:conclusion}
%═══════════════════════════════════════════════════════════

\begin{chaptersummary}
جمع‌بندی پیوست الف — مقایسهٔ جامع نُه نمونه:

\begin{enumerate}[nosep]
\item نُه نمونه در ده بُعد و بیش از ۶۰ شاخص مقایسه شدند.
\item \textbf{آفریقای جنوبی} (امتیاز ۳۶/۴۰) جامع‌ترین الگوی قابل‌انتقال به ایران است.
\item \textbf{عراق} و \textbf{میانمار} مهم‌ترین ضد الگوها هستند.
\item مسیر مذاکره‌ای، اپوزیسیون سازمان‌یافته، و اصلاح تدریجی امنیتی سه عامل کلیدی موفقیت‌اند.
\item مداخلهٔ نظامی، انحلال ارتش، و اجتثاث افراطی سه عامل کلیدی شکست‌اند.
\item مدل ۶ (ترکیبی-تطبیقی) از بهترین عناصر هر نمونه طراحی شده است.
\item بُعد هسته‌ای و ساختار سپاه، ایران را منحصربه‌فرد اما نه استثنا می‌سازد.
\item رابطهٔ معناداری بین حجم هزینهٔ بین‌المللی و نتیجهٔ دموکراتیک وجود ندارد؛ \textbf{طراحی هوشمند} تعیین‌کننده است.
\item هر سناریوی گذار ایران (فصل ۴) نمونهٔ تاریخی مرتبط خود را دارد (جدول ۱۰).
\item جزئیات هر نمونه در پیوست‌های ب تا ح آمده است.
\end{enumerate}

\vspace{0.3cm}
\textit{برای مطالعهٔ تفصیلی هر نمونه:}
\begin{itemize}[nosep]
\item آفریقای جنوبی: \seeChapter{app:south-africa}
\item شیلی: \seeChapter{app:chile}
\item تونس: \seeChapter{app:tunisia}
\item لهستان و اروپای شرقی: \seeChapter{app:poland}
\item عراق (ضد الگو): \seeChapter{app:iraq}
\item میانمار (گذار ناتمام): \seeChapter{app:myanmar}
\item تیمور شرقی: \seeChapter{app:timor}
\end{itemize}
\end{chaptersummary}

\chapterend

%══════════════════════════════════════════════════════════════
% پایان پیوست الف
%══════════════════════════════════════════════════════════════