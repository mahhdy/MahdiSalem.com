%══════════════════════════════════════════════════════════════
% پیوست ج: مطالعه موردی عراق — نمونهٔ منفی
% فایل: appendices/app-f-iraq.tex
% حجم هدف: ۸-۱۰ صفحه
%══════════════════════════════════════════════════════════════

\chapter{مطالعهٔ موردی: عراق — نمونهٔ منفی (۲۰۰۳-۲۰۱۰)}
\label{app:iraq}

\begin{executivesummary}
عراق \textbf{مهم‌ترین ضد الگو} (\lr{Counter-Model}) در این کتاب است. مداخلهٔ نظامی آمریکا در مارس ۲۰۰۳ رژیم \person{صدام حسین}{\lr{Saddam Hussein}} را سرنگون کرد اما به‌جای دموکراسی، \textbf{جنگ داخلی فرقه‌ای}، \textbf{فروپاشی دولت}، و \textbf{ظهور داعش} را به ارمغان آورد. هزینهٔ انسانی: بیش از ۲۰۰,۰۰۰ کشته. هزینهٔ مالی: بیش از ۲ تریلیون دلار (برای آمریکا). هر تصمیم اشتباهی که در گذار ممکن است گرفته شود، در عراق گرفته شد: مداخلهٔ نظامی بدون برنامه، انحلال ارتش، اجتثاث بدون عدالت، بی‌توجهی به بافت فرقه‌ای-قومی، و تحمیل مدل خارجی. این پیوست هفت «اشتباه مهلک» عراق را تحلیل و \textbf{ضد درس‌آموخته‌ها} (\lr{Anti-Lessons}) را برای ایران استخراج می‌کند.
\end{executivesummary}

%═══════════════════════════════════════════════════════════
\section{زمینه و بافت تاریخی}
\label{app:iraq:context}
%═══════════════════════════════════════════════════════════

\subsection{رژیم بعث: ساختار و ویژگی‌ها}

\begin{table}[htbp]
\centering
\caption{مشخصات عراق در آستانهٔ سقوط (۲۰۰۳)}
\label{tab:app-iraq-profile}
\begin{tabularx}{\textwidth}{>{\raggedleft\arraybackslash}p{4.5cm} >{\raggedleft\arraybackslash}X}
\toprule
\headerrow \textbf{شاخص} & \textbf{مقدار} \\
\midrule
جمعیت & ۲۶ میلیون نفر \\
\altrow مساحت & ۴۳۸,۰۰۰ \lr{km²} \\
ترکیب قومی-مذهبی & شیعه ۶۰٪ + سنی ۲۰٪ + کرد ۱۵٪ + سایر ۵٪ \\
\altrow \lr{GDP per capita} & $\sim$\$۹۰۰ (پس از ۱۲ سال تحریم) \\
طول عمر رژیم بعث & ۳۵ سال (۱۹۶۸-۲۰۰۳) \\
\altrow اندازهٔ نیروهای مسلح & $\sim$۴۰۰,۰۰۰ (ارتش + حرس جمهوری + فدائیان) \\
سلاح‌های کشتارجمعی & ادعای آمریکا (نادرست) — برنامهٔ هسته‌ای متوقف \\
\altrow تحریم‌های بین‌المللی & جامع از ۱۹۹۰ (قطعنامهٔ ۶۶۱) \\
اپوزیسیون & تبعیدی + پراکنده (شورای حاکمیت عراق) \\
\altrow وضعیت جامعهٔ مدنی & سرکوب‌شدهٔ کامل \\
\bottomrule
\end{tabularx}
\end{table}

\subsection{وجوه مشابهت عراق و ایران}

\begin{table}[htbp]
\centering
\caption{مقایسهٔ ساختاری عراق (۲۰۰۳) و ایران (وضع فعلی)}
\label{tab:app-iraq-iran-comparison}
\begin{tabularx}{\textwidth}{>{\raggedleft\arraybackslash}p{3cm} >{\raggedleft\arraybackslash}X >{\raggedleft\arraybackslash}X >{\centering\arraybackslash}p{1.5cm}}
\toprule
\headerrow \textbf{بُعد} & \textbf{عراق ۲۰۰۳} & \textbf{ایران فعلی} & \textbf{مشابهت} \\
\midrule
رژیم & اقتدارگرای تمامیت‌خواه & اقتدارگرای ایدئولوژیک & \rating{3} \\
\altrow ایدئولوژی & بعثیسم/ناسیونالیسم عربی & اسلام سیاسی/ولایت فقیه & \rating{2} \\
تنوع فرقه‌ای-قومی & بسیار بالا (شیعه/سنی/کرد) & بالا (فارس/ترک/کرد/بلوچ/عرب) & \rating{4} \\
\altrow نفت & ذخایر عظیم (ردهٔ ۵ جهان) & ذخایر عظیم (ردهٔ ۴ جهان) & \rating{5} \\
نیروهای مسلح موازی & حرس جمهوری + فدائیان صدام & سپاه پاسداران + بسیج & \rating{5} \\
\altrow تحریم‌ها & ۱۲ سال تحریم شدید & ۴۰+ سال تحریم متنوع & \rating{4} \\
دیاسپورا & بزرگ و سیاسی & بسیار بزرگ و فعال & \rating{4} \\
\altrow اپوزیسیون سازمان‌یافته & ضعیف و تبعیدی & ضعیف و پراکنده & \rating{4} \\
بُعد هسته‌ای & ادعایی (نادرست) & واقعی (غنی‌سازی ۶۰٪+) & \rating{3} \\
\altrow رابطه با همسایگان & متخاصم (جنگ ایران-عراق، کویت) & متخاصم با برخی + نفوذ نیابتی & \rating{4} \\
\midrule
\headerrow \multicolumn{3}{l}{\textbf{میانگین مشابهت}} & \textbf{\rating{4}} \\
\bottomrule
\end{tabularx}
\end{table}

\begin{casestudy}
\textbf{تفاوت حیاتی:} با وجود مشابهت‌های ساختاری بالا (۴ از ۵)، یک تفاوت بنیادین وجود دارد: عراق از طریق \textbf{مداخلهٔ نظامی خارجی} وارد گذار شد و ایران (در مدل ۶ پیشنهادی) نباید. مشابهت بالای ایران و عراق دقیقاً دلیل اهمیت مطالعهٔ این ضد الگوست: هر اشتباهی که در عراق شد، \textbf{در ایران با شدت بیشتری} تکرار خواهد شد (جمعیت ۳ برابر، ابعاد هسته‌ای، ژئوپلیتیک حساس‌تر).
\end{casestudy}

\sectiondivider

%═══════════════════════════════════════════════════════════
\section{هفت اشتباه مهلک: تشریح و ضد درس‌آموخته‌ها}
\label{app:iraq:mistakes}
%═══════════════════════════════════════════════════════════

\subsection{اشتباه اول: مداخلهٔ نظامی خارجی}

\begin{warningbox}
\textbf{اشتباه مهلک ۱: حملهٔ نظامی به‌جای حمایت از گذار داخلی}

\begin{itemize}[nosep]
\item \textbf{واقعیت:} آمریکا و ائتلاف «داوطلبانه» (\lr{Coalition of the Willing}) در ۲۰ مارس ۲۰۰۳ با ۱۷۷,۰۰۰ نیرو به عراق حمله کردند.
\item \textbf{بهانه:} سلاح‌های کشتارجمعی (که وجود نداشت) + ارتباط با القاعده (که نبود).
\item \textbf{بدون مجوز شورای امنیت:} قطعنامهٔ ۱۴۴۱ مجوز جنگ نمی‌داد — مداخله غیرقانونی.
\item \textbf{نتیجه:} سقوط رژیم در ۳ هفته → اما ۱۷ سال اشغال و خشونت.
\item \textbf{هزینهٔ انسانی:} ۲۰۰,۰۰۰+ کشتهٔ عراقی + ۴,۵۰۰ سرباز آمریکایی + ۵ میلیون آواره.
\item \textbf{هزینهٔ مالی:} $>$\$۲ تریلیون (برای آمریکا تنها).
\end{itemize}

\textbf{ضد درس‌آموختهٔ ایرانی:} مداخلهٔ نظامی خارجی در ایران = سناریوی \lr{E} = \textbf{مطلقاً مردود}. ایران ۳ برابر جمعیت عراق، ۳ برابر مساحت، کوهستانی‌تر، مسلح‌تر (سپاه + هسته‌ای)، و با ناسیونالیسم قوی‌تر. هزینهٔ احتمالی: $>$\$۱۰ تریلیون + $>$۵۰۰,۰۰۰ کشته (\seeChapter{ch:scenarios}).
\end{warningbox}

\subsection{اشتباه دوم: انحلال ارتش (فرمان شمارهٔ ۲)}

\begin{table}[htbp]
\centering
\caption{فرمان شمارهٔ ۲ \lr{CPA}: انحلال نیروهای مسلح عراق}
\label{tab:app-iraq-order2}
\begin{tabularx}{\textwidth}{>{\raggedleft\arraybackslash}p{4cm} >{\raggedleft\arraybackslash}X}
\toprule
\headerrow \textbf{بُعد} & \textbf{جزئیات} \\
\midrule
صادرکننده & \person{ال.پل برمر}{\lr{L. Paul Bremer III}} — حاکم \lr{CPA} \\
\altrow تاریخ & ۲۳ مه ۲۰۰۳ (۶ هفته پس از سقوط بغداد) \\
محتوا & انحلال کامل: ارتش + حرس جمهوری + وزارت دفاع + وزارت اطلاعات + تمام ساختارهای امنیتی \\
\altrow تعداد متأثران & $\sim$۴۰۰,۰۰۰ نظامی + $\sim$۳۵۰,۰۰۰ کارمند وزارتخانه‌های منحله = $\sim$۷۵۰,۰۰۰ بیکار مسلح \\
هشدارهای نادیده‌گرفته‌شده & ژنرال \lr{Garner} (پیشینیان برمر)، \lr{CIA}، وزارت خارجه — همه مخالف بودند \\
\altrow نتیجهٔ فوری & ۷۵۰,۰۰۰ مرد مسلح و خشمگین بدون درآمد و هویت \\
نتیجهٔ بلندمدت & شورشیگری سنی → القاعدهٔ عراق → \textbf{داعش} \\
\bottomrule
\end{tabularx}
\end{table}

\begin{keypoint}
\textbf{ضد درس‌آموختهٔ ایرانی:} سپاه پاسداران ایران ($\sim$۱۹۰,۰۰۰) + بسیج ($\sim$۶۰۰,۰۰۰ فعال) + ارتش ($\sim$۴۲۰,۰۰۰) = بیش از ۱.۲ میلیون نفر مسلح. انحلال = ۱.۲ میلیون بیکار مسلح. بعد از عراق، هیچ تحلیلگر جدی انحلال را توصیه نمی‌کند. مدل درست: \textbf{ادغام} (آفریقای جنوبی) + \textbf{تفکیک اقتصادی} (اندونزی) + \textbf{بازنشستگی افتخاری} + \textbf{نظارت مدنی تدریجی} (\seeChapter{ch:guarantees}).
\end{keypoint}

\subsection{اشتباه سوم: اجتثاث بعث بدون عدالت}

\begin{table}[htbp]
\centering
\caption{مقایسهٔ اجتثاث بعث عراق با مدل‌های جایگزین}
\label{tab:app-iraq-debaath}
\begin{tabularx}{\textwidth}{>{\raggedleft\arraybackslash}p{2.5cm} >{\raggedleft\arraybackslash}X >{\centering\arraybackslash}p{2cm}}
\toprule
\headerrow \textbf{مدل} & \textbf{توضیح} & \textbf{نتیجه} \\
\midrule
عراق: اجتثاث افراطی & حذف همهٔ اعضای ردهٔ ۱-۴ بعث ($\sim$۸۵,۰۰۰ نفر) از کار — بدون تفکیک عامل/عضو عادی & \statusbad فاجعه \\
\altrow آلمان: نازی‌زدایی & ابتدا افراطی (متفقین) → سپس تعدیل (دولت آلمان) & \statuswarn متوسط \\
چک: لوستراسیون هدفمند & فقط مقامات بالا + بررسی فردی & \statusok خوب \\
\altrow آفریقای جنوبی: ادغام + TRC & نه پاکسازی بلکه اعتراف + عفو مشروط & \statusok خوب \\
\textbf{پیشنهاد ایران} & \textbf{لوستراسیون هوشمند:} بررسی فردی + تفکیک عامل/عضو عادی + بازنشستگی افتخاری & -- \\
\bottomrule
\end{tabularx}
\end{table}

\begin{lessonlearned}
\textbf{اجتثاث بعث:} فرمان شمارهٔ ۱ برمر (۱۶ مه ۲۰۰۳) حدود ۸۵,۰۰۰ بعثی را از مشاغل دولتی اخراج کرد — \textbf{بدون تفکیک} بین جنایتکاران واقعی و معلمان/پزشکانی که برای داشتن شغل مجبور به عضویت بودند. نتیجه: سرمایهٔ انسانی کشور نابود شد + هزاران نفر خشمگین و بیکار به شورشیان پیوستند. \emphred{ضد درس‌آموختهٔ ایرانی:} در ایران، میلیون‌ها نفر عضو بسیج، سپاه، یا وابسته به نهادهای حکومتی هستند. پاکسازی کور = فاجعه. مدل درست: \textbf{لوستراسیون هوشمند} (بررسی فردی: ردهٔ ۱ محاکمه + ردهٔ ۲-۳ بررسی + ردهٔ ۴+ عفو مشروط).
\end{lessonlearned}

\subsection{اشتباه چهارم: فقدان برنامهٔ پس از سقوط}

\begin{enumerate}[nosep]
\item \textbf{طرح وزارت خارجه (\lr{Future of Iraq Project}):} ۱۳ جلد، ۲,۵۰۰ صفحه — \textbf{نادیده گرفته شد} توسط پنتاگون
\item \textbf{طرح ارتش (\lr{Army War College}):} ۶۰۰ صفحه هشدار — نادیده گرفته شد
\item \textbf{فرض اشتباه رامسفلد:} «مردم عراق ما را با گل استقبال می‌کنند» — نکردند
\item \textbf{فرض اشتباه چلبی:} «تبعیدیان آماده‌اند کشور را اداره کنند» — نبودند
\item \textbf{نتیجه:} بی‌نظمی کامل (غارت)، فقدان خدمات عمومی، خلأ قدرت
\end{enumerate}

\begin{warningbox}
\textbf{ضد درس‌آموختهٔ ایرانی:} \textbf{فاز ۰ (پیش‌گذار)} مهم‌ترین فاز است. بدون برنامهٔ دقیق برای ۷۲ ساعت اول، ماه اول، و سال اول — حتی بهترین نیت‌ها به فاجعه منجر می‌شوند. عراق نشان داد که \textbf{سرنگونی} آسان‌ترین بخش کار است؛ \textbf{ساختن} بسیار دشوارتر. برنامهٔ ایران باید شامل: نقشهٔ خدمات ضروری، تأمین آب/برق/غذا، جلوگیری از غارت، حفظ آرشیوها، مدیریت مرزها، و زنجیرهٔ فرماندهی موقت باشد (\seeChapter{ch:timeline}).
\end{warningbox}

\subsection{اشتباه پنجم: تحمیل مدل خارجی (فدرالیسم فرقه‌ای)}

\begin{enumerate}[nosep]
\item \textbf{نظام فرقه‌ای (\lr{Muhasasa}):} قدرت بر اساس فرقه (شیعه/سنی/کرد) تقسیم شد — نه بر اساس شهروندی
\item \textbf{فدرالیسم قومی:} منطقهٔ کردستان عملاً مستقل شد
\item \textbf{قانون اساسی ۲۰۰۵:} تحت فشار زمانی و با مشارکت محدود سنی‌ها نوشته شد — \textbf{بمب ساعتی}
\item \textbf{نتیجه:} هر انتخاباتی به جنگ فرقه‌ای تبدیل شد + سنی‌ها احساس حذف کردند → داعش
\end{enumerate}

\begin{keypoint}
\textbf{ضد درس‌آموختهٔ ایرانی:} تقسیم قدرت بر اساس \textbf{قومیت/مذهب} (مدل لبنان/عراق) = فرمول جنگ داخلی. مدل درست برای ایران: \textbf{شهروندی فراگیر} + حقوق فرهنگی-زبانی اقوام + تمرکززدایی اداری (نه قومی) + سهمیهٔ تنوع (نه سهمیهٔ فرقه‌ای). قانون اساسی آفریقای جنوبی (۱۱ زبان اما شهروندی واحد) الگوی بهتری است (\seeChapter{ch:guarantees}).
\end{keypoint}

\subsection{اشتباه ششم: بی‌توجهی به امنیت و خلأ قدرت}

\begin{table}[htbp]
\centering
\caption{تبعات خلأ امنیتی در عراق}
\label{tab:app-iraq-security}
\begin{tabularx}{\textwidth}{>{\centering\arraybackslash}p{2cm} >{\raggedleft\arraybackslash}X >{\centering\arraybackslash}p{2.5cm}}
\toprule
\headerrow \textbf{دوره} & \textbf{بحران امنیتی} & \textbf{تلفات} \\
\midrule
۲۰۰۳ & غارت گسترده + ناامنی + حمله به زیرساخت‌ها & $\sim$۷,۰۰۰ \\
\altrow ۲۰۰۴ & شورش فلوجه + شورش مقتدی صدر & $\sim$۱۶,۰۰۰ \\
۲۰۰۵ & ترور + انفجار + تخریب مرقد سامرا & $\sim$۲۰,۰۰۰ \\
\altrow ۲۰۰۶-۲۰۰۷ & \textbf{جنگ داخلی فرقه‌ای} (اوج خشونت) & $\sim$۵۵,۰۰۰ \\
۲۰۰۸-۲۰۱۱ & افزایش نیرو (\lr{Surge}) + کاهش نسبی & $\sim$۲۵,۰۰۰ \\
\altrow ۲۰۱۴-۲۰۱۷ & \textbf{داعش}: اشغال موصل + نسل‌کشی ایزدی‌ها & $\sim$۵۰,۰۰۰+ \\
\bottomrule
\end{tabularx}
\end{table}

\subsection{اشتباه هفتم: بی‌اعتنایی به فرهنگ و بافت محلی}

\begin{enumerate}[nosep]
\item \textbf{ناآگاهی فرهنگی:} مقامات \lr{CPA} زبان عربی نمی‌دانستند و تاریخ عراق را نمی‌شناختند
\item \textbf{تبعیدیان قطع‌ارتباط:} رهبران تبعیدی (چلبی، علاوی) سال‌ها خارج بودند و مشروعیت نداشتند
\item \textbf{تحمیل الگوی غربی:} تلاش برای ایجاد «دموکراسی جفرسونی» در جامعهٔ قبیله‌ای-فرقه‌ای
\item \textbf{بی‌احترامی:} ابوغریب + بازرسی خانه‌ها + عملیات‌های شبانه = تحقیر مردم
\item \textbf{نتیجه:} «آزادی‌بخشان» به «اشغالگران» تبدیل شدند
\end{enumerate}

\begin{recommendation}
\textbf{ضد درس‌آموختهٔ کلان ایرانی:} \textbf{مالکیت ملی غیرقابل‌مذاکره} است. مدل ۶ بر اصل «ایرانیان عامل اصلی» استوار است. هرگونه نظارت بین‌المللی باید: ۱) به دعوت ایرانیان باشد؛ ۲) به زبان و فرهنگ ایرانی حساس باشد؛ ۳) کارشناسان ایرانی (داخل + دیاسپورا) نقش محوری داشته باشند؛ ۴) تصمیم‌های کلیدی با ایرانیان باشد نه خارجی‌ها (\seeChapter{ch:approaches}).
\end{recommendation}

\sectiondivider

%═══════════════════════════════════════════════════════════
\section{ماتریس اشتباهات عراق و ضد درس‌آموخته‌های ایرانی}
\label{app:iraq:matrix}
%═══════════════════════════════════════════════════════════

\begin{table}[htbp]
\centering
\caption{ماتریس هفت اشتباه مهلک عراق و ضد درس‌آموخته‌های ایرانی}
\label{tab:app-iraq-lessons}
\begin{tabularx}{\textwidth}{
  >{\centering\arraybackslash}p{0.7cm}
  >{\raggedleft\arraybackslash}p{2.5cm}
  >{\raggedleft\arraybackslash}X
  >{\raggedleft\arraybackslash}p{3.5cm}
}
\toprule
\headerrow \textbf{\#} & \textbf{اشتباه عراق} & \textbf{نتیجه} & \textbf{ضد درس‌آموختهٔ ایرانی} \\
\midrule
۱ & مداخلهٔ نظامی خارجی & جنگ + ۲۰۰K کشته + \$۲T & \cellgreen{سناریوی E مطلقاً مردود} \\
\altrow
۲ & انحلال ارتش & ۷۵۰K بیکار مسلح → داعش & \cellgreen{ادغام (نه انحلال) سپاه} \\
۳ & اجتثاث بدون تفکیک & سرمایهٔ انسانی نابود شد & \cellgreen{لوستراسیون هوشمند فردی} \\
\altrow
۴ & فقدان برنامهٔ پسا-سقوط & غارت + خلأ قدرت + هرج & \cellgreen{فاز ۰ دقیق + ۷۲ ساعت اول} \\
۵ & فدرالیسم فرقه‌ای & جنگ فرقه‌ای & \cellgreen{شهروندی فراگیر + تمرکززدایی اداری} \\
\altrow
۶ & خلأ امنیتی & ترور + میلیشیا + داعش & \cellgreen{حفظ نظم: فوریت ۷۲ ساعته} \\
۷ & تحمیل مدل خارجی & «اشغالگر» نه «آزادی‌بخش» & \cellgreen{مالکیت ملی ایرانی} \\
\bottomrule
\end{tabularx}
\end{table}

\sectiondivider

%═══════════════════════════════════════════════════════════
\section{نمودار: مسیر فاجعه‌بار عراق}
\label{app:iraq:diagram}
%═══════════════════════════════════════════════════════════

\begin{figure}[htbp]
\centering
\begin{tikzpicture}[
  node distance=0.6cm,
  phase/.style={
    draw, rounded corners=5pt, minimum width=3cm,
    minimum height=1.2cm, font=\small, align=center,
    thick
  },
  bad/.style={phase, fill=MainRed!15, draw=MainRed!60},
  worse/.style={phase, fill=MainRed!30, draw=MainRed!80},
  arrow/.style={->, very thick, >=stealth, MainRed},
  mistake/.style={
    draw=MainRed, fill=MainRed!5, rounded corners=2pt,
    font=\tiny, align=center, text width=2.5cm
  }
]

% مسیر فاجعه‌بار
\node[bad] (invasion) at (0,0) {حملهٔ نظامی\\مارس ۲۰۰۳};
\node[bad] (fall) at (4,0) {سقوط بغداد\\آوریل ۲۰۰۳};
\node[worse] (orders) at (8,0) {فرمان ۱ و ۲\\مه ۲۰۰۳};
\node[worse] (chaos) at (12,0) {هرج‌ومرج\\۲۰۰۳-۲۰۰۴};

\node[worse] (insurgency) at (0,-3) {شورشیگری\\۲۰۰۴-۲۰۰۵};
\node[worse] (civilwar) at (4,-3) {جنگ داخلی\\۲۰۰۶-۲۰۰۷};
\node[bad] (surge) at (8,-3) {افزایش نیرو\\۲۰۰۷-۲۰۰۸};
\node[worse] (isis) at (12,-3) {\textbf{داعش}\\۲۰۱۴};

\draw[arrow] (invasion) -- (fall);
\draw[arrow] (fall) -- (orders);
\draw[arrow] (orders) -- (chaos);
\draw[arrow] (chaos) -- (insurgency);
\draw[arrow] (insurgency) -- (civilwar);
\draw[arrow] (civilwar) -- (surge);
\draw[arrow] (surge) -- (isis);

% اشتباهات
\node[mistake] at (0,1.5) {اشتباه ۱:\\مداخلهٔ نظامی};
\node[mistake] at (8,1.5) {اشتباه ۲+۳:\\انحلال+اجتثاث};
\node[mistake] at (12,1.5) {اشتباه ۴:\\بدون برنامه};
\node[mistake] at (0,-4.5) {اشتباه ۶:\\خلأ امنیتی};
\node[mistake] at (4,-4.5) {اشتباه ۵:\\فرقه‌گرایی};
\node[mistake] at (12,-4.5) {اشتباه ۷:\\تحمیل خارجی};

% فلش‌های اشتباه
\draw[MainRed!40, dashed, ->] (0,1.1) -- (invasion.north);
\draw[MainRed!40, dashed, ->] (8,1.1) -- (orders.north);
\draw[MainRed!40, dashed, ->] (12,1.1) -- (chaos.north);
\draw[MainRed!40, dashed, ->] (0,-4.1) -- (insurgency.south);
\draw[MainRed!40, dashed, ->] (4,-4.1) -- (civilwar.south);
\draw[MainRed!40, dashed, ->] (12,-4.1) -- (isis.south);

% هزینه‌ها
\node[font=\small\bfseries, MainRed] at (6,-6) {مجموع: ۲۰۰,۰۰۰+ کشته | \$۲+ تریلیون | ۵ میلیون آواره | ظهور داعش};

\end{tikzpicture}
\caption{مسیر فاجعه‌بار عراق: از مداخله تا داعش}
\label{fig:app-iraq-path}
\end{figure}

\sectiondivider

%═══════════════════════════════════════════════════════════
\section{هزینه-فایدهٔ مقایسه‌ای: عراق vs مدل ۶}
\label{app:iraq:cost-benefit}
%═══════════════════════════════════════════════════════════

\begin{table}[htbp]
\centering
\caption{مقایسهٔ هزینه-فایده: مدل عراقی vs مدل ۶ برای ایران}
\label{tab:app-iraq-cost}
\begin{tabularx}{\textwidth}{>{\raggedleft\arraybackslash}p{3.5cm} >{\centering\arraybackslash}X >{\centering\arraybackslash}X}
\toprule
\headerrow \textbf{شاخص} & \textbf{مدل عراقی (سناریوی E)} & \textbf{مدل ۶ پیشنهادی (سناریوی B)} \\
\midrule
هزینهٔ مالی بین‌المللی & $>$\$۶۰B (+ \$۲T آمریکا) & \$۲.۵-۵B \\
\altrow هزینهٔ انسانی & ۲۰۰,۰۰۰+ کشته & حداقلی (هدف: $<$۱,۰۰۰) \\
مدت & ۷+ سال (هنوز ناتمام) & ۵-۱۰ سال \\
\altrow نتیجهٔ دموکراتیک & \lr{V-Dem}: ۰.۲۵ & هدف: $>$۰.۶۵ \\
مشروعیت داخلی & \statusbad بسیار پایین & \statusok بالا (مالکیت ملی) \\
\altrow مشروعیت بین‌المللی & \statusbad بدون قطعنامه & \statusok با قطعنامه \\
ریسک جنگ داخلی & \riskhigh محقق شد & \risklow مدیریت‌شده \\
\altrow ریسک تروریسم & \riskhigh داعش ظهور کرد & \risklow کنترل‌شده \\
اثر منطقه‌ای & \statusbad بی‌ثباتی گسترده & \statusok ثبات‌بخشی \\
\bottomrule
\end{tabularx}
\end{table}

\sectiondivider

%═══════════════════════════════════════════════════════════
\section{جمع‌بندی پیوست}
\label{app:iraq:conclusion}
%═══════════════════════════════════════════════════════════

\begin{chaptersummary}
جمع‌بندی پیوست ج — عراق: نمونهٔ منفی:

\begin{enumerate}[nosep]
\item عراق \textbf{مهم‌ترین ضد الگوی} این کتاب است: هر اشتباهی که در گذار ممکن بود، رخ داد.
\item \textbf{هفت اشتباه مهلک:} مداخلهٔ نظامی، انحلال ارتش، اجتثاث افراطی، فقدان برنامه، فدرالیسم فرقه‌ای، خلأ امنیتی، تحمیل مدل خارجی.
\item \textbf{هزینه:} ۲۰۰,۰۰۰+ کشته + \$۲+ تریلیون + ۵M آواره + داعش — در مقابل هیچ نتیجهٔ دموکراتیک پایدار.
\item مشابهت ساختاری ایران و عراق بالاست (\rating{4})، بنابراین هر اشتباه عراقی در ایران \textbf{با شدت بیشتر} تکرار می‌شود.
\item \textbf{سناریوی E (مداخلهٔ نظامی) مطلقاً مردود} است — عراق اثبات کرد.
\item مدل ۶ دقیقاً بر مبنای \textbf{اجتناب از اشتباهات عراق} طراحی شده: مالکیت ملی، ادغام (نه انحلال)، لوستراسیون هوشمند (نه اجتثاث)، برنامهٔ دقیق پیش‌گذار، شهروندی فراگیر (نه فرقه‌گرایی).
\item \textbf{فاز ۰ (پیش‌گذار)} مهم‌ترین درس عراق است: بدون آمادگی، حتی بهترین نیت‌ها به فاجعه می‌انجامد.
\item عراق نشان داد که \textbf{سرنگونی آسان‌ترین بخش} کار است — ساختن بسیار دشوارتر.
\end{enumerate}

\vspace{0.3cm}
\textit{مطالعهٔ تکمیلی:}
\begin{itemize}[nosep]
\item مقایسهٔ جامع ۹ نمونه: \seeChapter{app:comparison}
\item سناریوی مداخلهٔ نظامی (رد): \seeChapter{ch:scenarios}
\item اصلاح بخش امنیتی: \seeChapter{ch:guarantees}
\item آسیب‌شناسی و ریسک‌ها: \seeChapter{ch:risks}
\item میانمار (ضد الگوی دیگر): \seeChapter{app:myanmar}
\end{itemize}
\end{chaptersummary}

\chapterend

%══════════════════════════════════════════════════════════════
% پایان پیوست ج
%══════════════════════════════════════════════════════════════