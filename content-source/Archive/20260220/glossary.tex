% ╔══════════════════════════════════════════════════════════════════╗
% ║  واژه‌نامه — تعاریف اصلی                                       ║
% ╚══════════════════════════════════════════════════════════════════╝

% ---- مفاهیم نظری گذار ----

\newglossaryentry{democratic-transition}{
    name={گذار دموکراتیک},
    description={\lr{Democratic Transition} — 
    فرایند تغییر نظام سیاسی از اقتدارگرایی به دموکراسی، 
    شامل مراحل آزادسازی، دموکراتیزاسیون و تحکیم},
    sort={گذار}
}

\newglossaryentry{consolidation}{
    name={تحکیم دموکراتیک},
    description={\lr{Democratic Consolidation} — 
    مرحله‌ای که در آن دموکراسی «تنها بازی ممکن» 
    (\lr{the only game in town}) می‌شود و بازگشت 
    به اقتدارگرایی بعید است},
    sort={تحکیم}
}

\newglossaryentry{backsliding}{
    name={بازگشت اقتدارگرایانه},
    description={\lr{Authoritarian Backsliding/Reversal} — 
    فرایند تضعیف تدریجی یا ناگهانی نهادها و ارزش‌های 
    دموکراتیک و بازگشت به حکمرانی اقتدارگرایانه},
    sort={بازگشت}
}

\newglossaryentry{transitional-justice}{
    name={عدالت انتقالی},
    description={\lr{Transitional Justice} — 
    مجموعه‌ی اقدامات قضایی و غیرقضایی برای مواجهه 
    با میراث نقض حقوق بشر شامل: محاکمه، 
    کمیسیون حقیقت، غرامت و اصلاح نهادی},
    sort={عدالت}
}

\newglossaryentry{ssr}{
    name={اصلاح بخش امنیتی},
    description={\lr{Security Sector Reform (SSR)} — 
    بازسازی نهادهای نظامی، انتظامی و اطلاعاتی 
    برای تبعیت از حکمرانی دموکراتیک و 
    نظارت غیرنظامی},
    sort={اصلاح}
}

\newglossaryentry{ddr}{
    name={خلع سلاح، بسیج‌زدایی و بازادغام},
    description={\lr{Disarmament, Demobilization \& 
    Reintegration (DDR)} — فرایند سه‌مرحله‌ای 
    جمع‌آوری سلاح، انحلال ساختارهای نظامی و 
    بازگرداندن نیروها به زندگی مدنی},
    sort={خلع}
}

\newglossaryentry{r2p}{
    name={مسئولیت حمایت},
    description={\lr{Responsibility to Protect (R2P)} — 
    اصل بین‌المللی مبنی بر مسئولیت جامعه‌ی جهانی 
    برای حمایت از مردم در برابر نسل‌کشی، جرایم جنگی، 
    پاکسازی قومی و جرایم علیه بشریت},
    sort={مسئولیت}
}

\newglossaryentry{spoiler}{
    name={اسپویلر},
    description={\lr{Spoiler} — بازیگری که از فرایند 
    صلح یا گذار منتفع نمی‌شود و فعالانه تلاش 
    می‌کند آن را تخریب کند},
    sort={اسپویلر}
}

\newglossaryentry{election-monitoring}{
    name={نظارت انتخاباتی},
    description={\lr{Election Monitoring/Observation} — 
    حضور سازمان‌یافته‌ی ناظران ملی یا بین‌المللی 
    در مراحل مختلف فرایند انتخابات برای ارزیابی 
    آزاد و منصفانه بودن آن},
    sort={نظارت انتخاباتی}
}

\newglossaryentry{trc}{
    name={کمیسیون حقیقت و آشتی},
    description={\lr{Truth and Reconciliation Commission 
    (TRC)} — نهاد رسمی موقت برای بررسی نقض‌های 
    گسترده‌ی حقوق بشر، ثبت شهادت قربانیان و 
    ارائه‌ی توصیه برای جلوگیری از تکرار},
    sort={کمیسیون حقیقت}
}

\newglossaryentry{lustration}{
    name={تطهیر},
    description={\lr{Lustration/Vetting} — 
    فرایند بررسی پیشینه‌ی کارکنان دولتی و 
    حذف افراد دخیل در نقض حقوق بشر از 
    مناصب عمومی},
    sort={تطهیر}
}

\newglossaryentry{soma}{
    name={توافق‌نامه‌ی وضعیت مأموریت},
    description={\lr{Status of Mission Agreement (SOMA)} — 
    توافقی حقوقی بین سازمان بین‌المللی و دولت 
    میزبان درباره‌ی وضعیت حقوقی، مصونیت‌ها و 
    امتیازات اعضای مأموریت},
    sort={توافقنامه}
}

\newglossaryentry{contact-group}{
    name={گروه تماس},
    description={\lr{Contact Group} — 
    گروهی از دولت‌های کلیدی و نهادهای بین‌المللی 
    که برای هماهنگی دیپلماسی و فشار مشترک 
    درباره‌ی یک بحران خاص تشکیل می‌شود},
    sort={گروه تماس}
}

\newglossaryentry{srsg}{
    name={نماینده‌ی ویژه‌ی دبیرکل},
    description={\lr{Special Representative of the 
    Secretary-General (SRSG)} — 
    مقام ارشدی که توسط دبیرکل سازمان ملل 
    برای مدیریت یک مأموریت خاص منصوب می‌شود},
    sort={نماینده}
}
\newglossaryentry{roadmap}{
    name={نقشه راه},
    description={Roadmap — طرح عملیاتی مرحله‌بندی‌شده برای رسیدن از وضعیت فعلی به هدف مطلوب}
}

\newglossaryentry{exit-strategy}{
    name={استراتژی خروج},
    description={Exit Strategy — برنامه از پیش تعریف‌شده برای پایان دادن به یک مأموریت بین‌المللی}
}

\newglossaryentry{benchmarks}{
    name={معیارهای سنجش},
    description={Benchmarks — شاخص‌های کمّی و کیفی برای ارزیابی پیشرفت به سمت اهداف تعریف‌شده}
}

\newglossaryentry{national-charter}{
    name={منشور ملی},
    description={National Charter — سند توافقی میان گروه‌های سیاسی درباره اصول بنیادین گذار}
}
% واژه‌های جدید فصل ۷
\newglossaryentry{authoritarian-reversal}{
    name={بازگشت اقتدارگرایی},
    description={Authoritarian Reversal — بازگشت به حکومت اقتدارگرا پس از دوره‌ای از گذار یا دموکراسی ناقص}
}

\newglossaryentry{early-warning}{
    name={هشدار زودهنگام},
    description={Early Warning System — نظامی برای شناسایی نشانه‌های بحران قبل از وقوع آن}
}

\newglossaryentry{transition-capture}{
    name={مصادره گذار},
    description={Transition Capture — قبضه کردن فرایند گذار توسط یک گروه سیاسی خاص}
}

\newglossaryentry{international-fatigue}{
    name={خستگی بین‌المللی},
    description={International Fatigue — کاهش تدریجی توجه و منابع جامعه بین‌المللی پس از دوره اولیه}
}

\newglossaryentry{resilience}{
    name={تاب‌آوری},
    description={Resilience — توانایی یک سیستم برای جذب شوک و بازگشت به حالت عادی یا انطباق}
}

\newglossaryentry{oligarchy}{
    name={الیگارشی},
    description={Oligarchy — حکومت گروهی کوچک از نخبگان ثروتمند یا قدرتمند}
}
% واژه‌های جدید فصل ۸
\newglossaryentry{unmoit}{
    name={\lr{UNMOIT}},
    description={United Nations Mission for Oversight of Iran's Transition — مأموریت سازمان ملل متحد برای نظارت بر گذار ایران (نام پیشنهادی)}
}
\newglossaryentry{truth-commission}{
    name={کمیسیون حقیقت‌یابی},
    description={Truth and Reconciliation Commission — نهاد غیرقضایی برای بررسی نقض حقوق بشر و ایجاد آشتی ملی}
}

\newglossaryentry{trust-fund}{
    name={صندوق امانی},
    description={Trust Fund — نهاد مالی شفاف برای مدیریت کمک‌های بین‌المللی و دارایی‌های ملی}
}

\newglossaryentry{cert}{
    name={\lr{CERT}},
    description={Computer Emergency Response Team — تیم واکنش سریع به رخدادهای سایبری}
}

\newglossaryentry{interim-laws}{
    name={قوانین موقت},
    description={Interim Laws — قوانین انتقالی که تا تصویب قانون اساسی جدید حاکم‌اند}
}

\newglossaryentry{rome-statute}{
    name={اساسنامه رم},
    description={Rome Statute — اساسنامه دادگاه کیفری بین‌المللی (۱۹۹۸) برای محاکمه جنایات بین‌المللی}
}
% واژه‌های جدید فصل ۹
\newglossaryentry{raci}{
    name={\lr{RACI}},
    description={Responsible, Accountable, Consulted, Informed — ماتریس تعیین مسئولیت برای فعالیت‌ها و بازیگران}
}

\newglossaryentry{gantt}{
    name={نمودار گانت},
    description={Gantt Chart — نمودار میله‌ای زمان‌بندی پروژه که فعالیت‌ها را بر محور زمان نمایش می‌دهد}
}

\newglossaryentry{pdca}{
    name={\lr{PDCA}},
    description={Plan-Do-Check-Act — چرخه مدیریت کیفیت و بهبود مستمر دمینگ}
}

\newglossaryentry{kpi}{
    name={\lr{KPI}},
    description={Key Performance Indicator — شاخص کلیدی عملکرد برای سنجش پیشرفت}
}

\newglossaryentry{disinformation}{
    name={اطلاعات نادرست},
    description={Disinformation — انتشار عمدی اطلاعات غلط با هدف فریب یا ایجاد اختلال}
}

\newglossaryentry{constituent-assembly}{
    name={مجلس مؤسسان},
    description={Constituent Assembly — مجلس منتخب ویژه تدوین قانون اساسی جدید}
}

\newglossaryentry{institutional-dependency}{
    name={وابستگی نهادی},
    description={Institutional Dependency — وضعیتی که نهادهای محلی بدون حمایت خارجی توان عملکرد ندارند}
}
% واژه‌های جدید فصل ۱۰
\newglossaryentry{assessed-budget}{
    name={بودجه ارزیابی‌شده},
    description={Assessed Budget — بودجه‌ای که از محل حق عضویت اجباری کشورهای عضو سازمان ملل تأمین می‌شود}
}

\newglossaryentry{multi-donor-trust}{
    name={صندوق امانی چندجانبه},
    description={Multi-Donor Trust Fund (MDTF) — مکانیزم مالی مشترک برای دریافت و مدیریت کمک‌های چند کشور}
}

\newglossaryentry{iati}{
    name={\lr{IATI}},
    description={International Aid Transparency Initiative — ابتکار بین‌المللی شفافیت کمک‌ها، استانداردی برای گزارش‌دهی مالی کمک‌های توسعه‌ای}
}

\newglossaryentry{sigir}{
    name={\lr{SIGIR}},
    description={Special Inspector General for Iraq Reconstruction — بازرس ویژه بازسازی عراق، نهاد حسابرسی مستقل آمریکایی}
}

\newglossaryentry{donors-conference}{
    name={کنفرانس کمک‌دهندگان},
    description={Donors Conference/Pledging Conference — نشست بین‌المللی برای جمع‌آوری تعهدات مالی کشورها و نهادها}
}

\newglossaryentry{whistleblower}{
    name={افشاگر},
    description={Whistleblower — فردی که تخلف یا فساد در سازمان خود را گزارش می‌دهد}
}
% --- واژه‌های جدید پیوست الف ---
\newglossaryentry{pacted-transition}{
  name={گذار مذاکره‌ای},
  description={\lr{Pacted Transition} — گذاری که طی توافق بین نخبگان رژیم و اپوزیسیون صورت می‌گیرد}
}
\newglossaryentry{codesa}{
  name={کودسا},
  description={\lr{Convention for a Democratic South Africa (CODESA)} — مذاکرات چندجانبهٔ ۱۹۹۱-۱۹۹۲ آفریقای جنوبی}
}
\newglossaryentry{moncloa}{
  name={پاکت‌های مونکلوا},
  description={\lr{Moncloa Pacts (1977)} — توافق‌نامه‌های اقتصادی-سیاسی بین احزاب اسپانیا در آستانهٔ گذار}
}
\newglossaryentry{cavr}{
  name={کاور},
  description={\lr{CAVR (Comissão de Acolhimento, Verdade e Reconciliação)} — کمیسیون حقیقت و آشتی تیمور شرقی ۲۰۰۲-۲۰۰۵}
}
\newglossaryentry{ivd}{
  name={آی‌وی‌دی},
  description={\lr{Instance Vérité et Dignité (IVD)} — نهاد حقیقت و کرامت تونس ۲۰۱۴-۲۰۱۹}
}
\newglossaryentry{de-baathification}{
  name={اجتثاث بعث},
  description={\lr{De-Ba'athification} — برنامهٔ پاکسازی اعضای حزب بعث از نهادهای دولتی عراق پس از ۲۰۰۳}
}
\newglossaryentry{untaet}{
  name={آنتت},
  description={\lr{UNTAET (UN Transitional Administration in East Timor)} — مأموریت مدیریت انتقالی سازمان ملل در تیمور شرقی ۱۹۹۹-۲۰۰۲}
}
\newglossaryentry{interfet}{
  name={اینترفت},
  description={\lr{INTERFET (International Force East Timor)} — نیروی چندملیتی به رهبری استرالیا در تیمور شرقی ۱۹۹۹}
}
\newglossaryentry{tatmadaw}{
  name={تاتمادو},
  description={\lr{Tatmadaw} — نیروهای مسلح میانمار، عامل کودتای ۲۰۲۱}
}
\newglossaryentry{tni}{
  name={تی‌ان‌آی},
  description={\lr{TNI (Tentara Nasional Indonesia)} — نیروهای مسلح اندونزی با سابقهٔ امپراتوری اقتصادی مشابه سپاه}
}
% --- واژه‌های جدید پیوست ب ---
\newglossaryentry{apartheid}{
  name={آپارتاید},
  description={\lr{Apartheid} — نظام جداسازی نژادی حاکم بر آفریقای جنوبی (۱۹۴۸-۱۹۹۴) مبتنی بر تفوق نژاد سفید}
}
\newglossaryentry{sunset-clauses}{
  name={بندهای غروب},
  description={\lr{Sunset Clauses} — تضمین‌های موقت (معمولاً ۵ ساله) برای کاهش ترس نخبگان رژیم پیشین و تسهیل مذاکره. پیشنهاد جو اسلوو (۱۹۹۲) در آفریقای جنوبی}
}
\newglossaryentry{ubuntu}{
  name={اوبونتو},
  description={\lr{Ubuntu} — فلسفهٔ آفریقایی به معنای «من هستم چون ما هستیم»؛ مبنای فلسفی \lr{TRC} و آشتی ملی}
}
\newglossaryentry{sandf}{
  name={سندف},
  description={\lr{SANDF (South African National Defence Force)} — نیروی دفاعی ملی آفریقای جنوبی پس از ادغام ۷ نیروی مسلح (۱۹۹۴)}
}
\newglossaryentry{unomsa}{
  name={آنومسا},
  description={\lr{UNOMSA (UN Observer Mission in South Africa)} — مأموریت ناظران سازمان ملل در آفریقای جنوبی (۱۹۹۲-۱۹۹۴) با ۲,۱۲۰ ناظر}
}
\newglossaryentry{sufficient-consensus}{
  name={اجماع کافی},
  description={\lr{Sufficient Consensus} — اصل مذاکراتی \lr{CODESA}: نه اتفاق آرا بلکه توافق اکثریت قریب به اتفاق طرف‌ها}
}
\newglossaryentry{anc}{
  name={ای‌ان‌سی},
  description={\lr{ANC (African National Congress)} — کنگرهٔ ملی آفریقا، قدیمی‌ترین حزب آزادی‌بخش آفریقا (تأسیس ۱۹۱۲)، حزب حاکم ۱۹۹۴-۲۰۲۴}
}
\newglossaryentry{bill-of-rights}{
  name={منشور حقوق},
  description={\lr{Bill of Rights} — فصل دوم قانون اساسی آفریقای جنوبی (۱۹۹۶)؛ از پیشرفته‌ترین اسناد حقوقی جهان شامل حقوق مدنی، سیاسی، اجتماعی-اقتصادی و محیط‌زیستی}
}
% --- واژه‌های جدید پیوست پ ---
\newglossaryentry{plebiscite}{
  name={پلبیسیت},
  description={\lr{Plebiscite} — رفراندوم عمومی؛ در شیلی: رأی‌گیری ۱۹۸۸ دربارهٔ ادامهٔ حکومت پینوشه (نتیجه: «نه» ۵۶٪)}
}
\newglossaryentry{concertacion}{
  name={کنسرتاسیون},
  description={\lr{Concertación de Partidos por la Democracia} — ائتلاف فراگیر ۱۶ حزب اپوزیسیون شیلی (۱۹۸۸-۲۰۱۳) که کمپین «نه» را هدایت کرد}
}
\newglossaryentry{incremental-justice}{
  name={عدالت تدریجی},
  description={\lr{Incremental Justice} — مدل شیلیایی عدالت انتقالی: پیشروی گام‌به‌گام از حقیقت‌یابی به غرامت و سپس محاکمه در طول دهه‌ها}
}
\newglossaryentry{institutional-locks}{
  name={قفل‌های نهادی},
  description={\lr{Institutional Locks/Authoritarian Enclaves} — سازوکارهای نهادی که رژیم پیشین قبل از ترک قدرت ایجاد می‌کند تا قدرت خود را حفظ کند (مثال: سناتورهای منصوب پینوشه)}
}
\newglossaryentry{pvt}{
  name={شمارش موازی آرا},
  description={\lr{Parallel Vote Tabulation (PVT)} — سیستم شمارش مستقل آرا توسط ناظران مدنی؛ اولین بار در شیلی ۱۹۸۸ به‌طور سیستماتیک استفاده شد}
}
\newglossaryentry{universal-jurisdiction}{
  name={صلاحیت جهانی},
  description={\lr{Universal Jurisdiction} — اصل حقوقی که به دادگاه‌های هر کشور اجازه می‌دهد جنایات بین‌المللی را صرف‌نظر از محل وقوع محاکمه کنند؛ مبنای بازداشت پینوشه در لندن (۱۹۹۸)}
}
\newglossaryentry{chicago-boys}{
  name={پسران شیکاگو},
  description={\lr{Chicago Boys} — اقتصاددانان شیلیایی تحصیل‌کردهٔ دانشگاه شیکاگو که سیاست‌های نئولیبرال پینوشه را طراحی کردند}
}
\newglossaryentry{rettig}{
  name={کمیسیون رتیگ},
  description={\lr{Rettig Commission (1991)} — اولین کمیسیون حقیقت شیلی به ریاست رائول رتیگ؛ ۳,۱۹۷ قربانی کشته/ناپدید شناسایی کرد}
}
\newglossaryentry{valech}{
  name={کمیسیون والش},
  description={\lr{Valech Commission (2003-2004)} — دومین کمیسیون حقیقت شیلی متمرکز بر شکنجه؛ ۲۸,۴۵۹ قربانی شناسایی کرد}
}
% --- واژه‌های جدید پیوست ت ---
\newglossaryentry{arab-spring}{
  name={بهار عربی},
  description={\lr{Arab Spring} — موج انقلابی ۲۰۱۰-۲۰۱۲ در خاورمیانه و شمال آفریقا آغازشده از تونس}
}
\newglossaryentry{quartet}{
  name={چهارگانهٔ گفت‌وگوی ملی},
  description={\lr{Tunisian National Dialogue Quartet} — ائتلاف چهار سازمان مدنی تونس (\lr{UGTT, UTICA, LTDH, ONAT}) که بحران ۲۰۱۳ را حل کردند؛ برندهٔ نوبل صلح ۲۰۱۵}
}
\newglossaryentry{ugtt}{
  name={یو‌جی‌تی‌تی},
  description={\lr{UGTT (Union Générale Tunisienne du Travail)} — اتحادیهٔ عمومی کارگران تونس با ۷۰۰,۰۰۰ عضو؛ نقش کلیدی در انقلاب ۲۰۱۱ و گفت‌وگوی ملی ۲۰۱۳}
}
\newglossaryentry{ennahda}{
  name={النهضه},
  description={\lr{Ennahda (حرکة النهضة)} — حزب اسلام‌گرای میانه‌رو تونس به رهبری راشد غنوشی؛ اولین حزب اسلامی که داوطلبانه قدرت را واگذار کرد}
}
\newglossaryentry{etat-civil}{
  name={دولت مدنی},
  description={\lr{État civil / Civil State} — مفهوم کلیدی مادهٔ ۲ قانون اساسی تونس ۲۰۱۴: دولتی نه دینی و نه نظامی، مبتنی بر شهروندی و قانون}
}
\newglossaryentry{parite}{
  name={نمایندگی برابر},
  description={\lr{Parité / Gender Parity} — اصل برابری جنسیتی در نمایندگی سیاسی؛ مادهٔ ۴۶ قانون اساسی تونس ۲۰۱۴}
}
\newglossaryentry{democratic-fatigue}{
  name={خستگی دموکراتیک},
  description={\lr{Democratic Fatigue} — پدیدهٔ سرخوردگی مردم از بی‌ثباتی و ناکارآمدی دموکراسی نوپا که زمینهٔ بازگشت اقتدارگرایی را فراهم می‌کند}
}
\newglossaryentry{democratic-backsliding}{
  name={بازگشت اقتدارگرایی},
  description={\lr{Democratic Backsliding / Autocratization} — فرآیند تدریجی تضعیف نهادهای دموکراتیک و بازگشت به حکومت فردی؛ نمونه: تونس ۲۰۲۱، مجارستان، ترکیه}
}
% --- واژه‌های جدید پیوست ث ---
\newglossaryentry{solidarity}{
  name={همبستگی},
  description={\lr{Solidarność (Solidarity)} — اتحادیهٔ کارگری مستقل لهستان (تأسیس ۱۹۸۰) با ۱۰ میلیون عضو؛ بزرگ‌ترین جنبش اجتماعی غیرخشونت‌آمیز قرن بیستم}
}
\newglossaryentry{round-table}{
  name={مذاکرات میزگرد},
  description={\lr{Round Table Talks} — مذاکرات ۵۹ روزهٔ لهستان در سال ۱۹۸۸ بین دولت و اتحادیهٔ همبستگی}
}
\newglossaryentry{ipn}{
  name={آی‌پی‌ان},
  description={\lr{IPN (Instytut Pamięci Narodowej)} — مؤسسهٔ حافظهٔ ملی لهستان (تأسیس ۱۹۹۸) مسئول حفظ آرشیو پلیس مخفی، تحقیق تاریخی و تعقیب جنایات دورهٔ کمونیسم}
}
\newglossaryentry{shock-therapy}{
  name={شوک‌تراپی},
  description={\lr{Shock Therapy} — برنامهٔ اصلاحات اقتصادی سریع و همزمان (آزادسازی قیمت‌ها + خصوصی‌سازی + ریاضت مالی)؛ طراحی بالتسروویچ در لهستان ۱۹۹۰}
}
\newglossaryentry{eu-conditionality}{
  name={مشروطیت اروپایی},
  description={\lr{EU Conditionality} — مجموعه شرایط سیاسی و اقتصادی که کشورهای متقاضی عضویت \lr{EU} باید رعایت کنند (معیارهای کپنهاگ ۱۹۹۳)؛ قوی‌ترین ابزار تحکیم دموکراسی در اروپای شرقی}
}
\newglossaryentry{copenhagen-criteria}{
  name={معیارهای کپنهاگ},
  description={\lr{Copenhagen Criteria (1993)} — سه معیار عضویت \lr{EU}: ۱) ثبات نهادهای دموکراتیک، ۲) اقتصاد بازار کارآمد، ۳) ظرفیت اجرای قوانین \lr{EU}}
}
\newglossaryentry{velvet-revolution}{
  name={انقلاب مخملی},
  description={\lr{Velvet Revolution} — گذار مسالمت‌آمیز چکسلواکی (نوامبر ۱۹۸۹) طی ۱۰ روز؛ به رهبری واتسلاو هاول}
}
\newglossaryentry{transition-losers}{
  name={شکست‌خوردگان گذار},
  description={\lr{Transition Losers} — گروه‌های اجتماعی که از فرآیند گذار (خصوصی‌سازی، آزادسازی) آسیب دیدند و مستعد جذب پوپولیسم هستند}
}
\newglossaryentry{domino-effect}{
  name={اثر دومینو},
  description={\lr{Domino Effect} — پدیدهٔ سرایت گذار دموکراتیک از یک کشور به همسایگان؛ نمونه: اروپای شرقی ۱۹۸۹ و بهار عربی ۲۰۱۱}
}
% --- واژه‌های جدید پیوست ج ---
\newglossaryentry{cpa}{
  name={سی‌پی‌اِی},
  description={\lr{CPA (Coalition Provisional Authority)} — ادارهٔ موقت ائتلاف در عراق (۲۰۰۳-۲۰۰۴) به ریاست پل برمر؛ صادرکنندهٔ فرمان‌های فاجعه‌بار انحلال ارتش و اجتثاث بعث}
}
\newglossaryentry{muhasasa}{
  name={محاصصه},
  description={\lr{Muhasasa (Sectarian Power-Sharing)} — نظام تقسیم قدرت بر اساس فرقه (شیعه/سنی/کرد) در عراق پس از ۲۰۰۳؛ عامل اصلی فرقه‌گرایی سیاسی}
}
\newglossaryentry{order-one}{
  name={فرمان شمارهٔ ۱},
  description={\lr{CPA Order Number 1: De-Ba'athification (16 May 2003)} — فرمان اجتثاث بعث: اخراج $\sim$۸۵,۰۰۰ بعثی از مشاغل دولتی بدون تفکیک}
}
\newglossaryentry{order-two}{
  name={فرمان شمارهٔ ۲},
  description={\lr{CPA Order Number 2: Dissolution of Entities (23 May 2003)} — فرمان انحلال کامل نیروهای مسلح عراق ($\sim$۷۵۰,۰۰۰ نفر بیکار مسلح)}
}
\newglossaryentry{smart-lustration}{
  name={لوستراسیون هوشمند},
  description={\lr{Smart Lustration / Targeted Vetting} — بررسی فردی (نه جمعی) سوابق مقامات رژیم پیشین با تفکیک عاملان اصلی از اعضای عادی؛ جایگزین پیشنهادی اجتثاث افراطی}
}
\newglossaryentry{counter-model}{
  name={ضد الگو},
  description={\lr{Counter-Model / Anti-Model} — نمونه‌ای که به‌دلیل شکست، درس‌آموخته‌های «چه نباید کرد» ارائه می‌دهد (مثال اصلی: عراق ۲۰۰۳)}
}
% --- واژه‌های جدید پیوست چ ---
\newglossaryentry{tatmadaw-full}{
  name={تاتمادو},
  description={\lr{Tatmadaw (တပ်မတော်)} — نیروهای مسلح میانمار؛ حکومت مستقیم ۱۹۶۲-۲۰۱۱ و کودتای ۲۰۲۱؛ دارای امپراتوری اقتصادی \lr{MEHL} و \lr{MEC}}
}
\newglossaryentry{mehl}{
  name={ام‌ای‌اچ‌ال},
  description={\lr{MEHL (Myanmar Economic Holdings Limited)} — هلدینگ اقتصادی تاتمادو؛ بزرگ‌ترین شرکت میانمار با فعالیت در یشم، بانکداری، مخابرات و مستغلات}
}
\newglossaryentry{nld}{
  name={ان‌ال‌دی},
  description={\lr{NLD (National League for Democracy)} — لیگ ملی برای دموکراسی میانمار به رهبری آنگ سان سوچی؛ برندهٔ انتخابات ۲۰۱۵ و ۲۰۲۰}
}
\newglossaryentry{nug}{
  name={ان‌یو‌جی},
  description={\lr{NUG (National Unity Government)} — دولت وحدت ملی میانمار؛ دولت موازی تشکیل‌شده پس از کودتای ۲۰۲۱ از نمایندگان \lr{NLD} و گروه‌های قومی}
}
\newglossaryentry{cdm}{
  name={سی‌دی‌ام},
  description={\lr{CDM (Civil Disobedience Movement)} — جنبش نافرمانی مدنی میانمار پس از کودتای ۲۰۲۱؛ صدها هزار کارمند دولت اعتصاب کردند}
}
\newglossaryentry{incomplete-transition}{
  name={گذار ناتمام},
  description={\lr{Incomplete/Stalled Transition} — فرآیند گذار دموکراتیک که به‌دلیل فقدان اصلاحات ساختاری یا مقاومت نیروهای قدیم، قبل از تحکیم متوقف یا معکوس می‌شود}
}
% --- واژه‌های جدید پیوست ح ---
\newglossaryentry{unamet}{
  name={یونامت},
  description={\lr{UNAMET (UN Mission in East Timor)} — مأموریت سازمان ملل برای سازماندهی رفراندوم استقلال تیمور شرقی (ژوئن-اکتبر ۱۹۹۹)}
}
\newglossaryentry{untaet-full}{
  name={آنتت},
  description={\lr{UNTAET (UN Transitional Administration in East Timor)} — مأموریت مدیریت انتقالی سازمان ملل (۱۹۹۹-۲۰۰۲) با حاکمیت کامل بر تیمور شرقی؛ جامع‌ترین مأموریت \lr{UN}}
}
\newglossaryentry{crp}{
  name={فرآیند آشتی جامعه‌محور},
  description={\lr{CRP (Community Reconciliation Process)} — مکانیزم عدالت انتقالی محلی در تیمور شرقی: اعتراف علنی عاملان جرایم سبک + عذرخواهی + خدمت اجتماعی + پذیرش مجدد}
}
\newglossaryentry{chega}{
  name={شِگا},
  description={\lr{Chega! (Enough!)} — عنوان گزارش نهایی کمیسیون حقیقت تیمور شرقی (\lr{CAVR})؛ ۵ جلد، ۲,۵۰۰+ صفحه، مستندسازی ۱۰۲,۸۰۰ کشته}
}
\newglossaryentry{fretilin}{
  name={فرتیلین},
  description={\lr{FRETILIN (Frente Revolucionária de Timor-Leste Independente)} — جبههٔ انقلابی استقلال تیمور شرقی؛ بزرگ‌ترین حزب سیاسی و رهبر مقاومت}
}
\newglossaryentry{vieira-de-mello}{
  name={ویئیرا دملو},
  description={\lr{Sérgio Vieira de Mello (1948-2003)} — دیپلمات برزیلی، نمایندهٔ ویژهٔ دبیرکل سازمان ملل و حاکم موقت تیمور شرقی؛ در بمب‌گذاری دفتر سازمان ملل در بغداد کشته شد}
}