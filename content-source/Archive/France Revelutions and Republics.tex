% ═══════════════════════════════════════════════════════════════════════════════
%                    تاریخ تحولات فرانسه: انقلاب‌ها و جمهوری‌ها
%                         فصل ۰ و ۱ - نسخه تصحیح‌شده
% ═══════════════════════════════════════════════════════════════════════════════

\documentclass[12pt,a4paper]{book}

% ──────────────────────────────────────────────────────────────────────────────
%                              پکیج‌های پایه
% ──────────────────────────────────────────────────────────────────────────────
\usepackage{amsmath,amssymb}
\usepackage{geometry}
\geometry{top=2.5cm, bottom=2.5cm, left=2cm, right=2.5cm, headheight=15pt}
\usepackage{graphicx}
\usepackage{array}
\usepackage{booktabs}
\usepackage{longtable}
\usepackage{multirow}
\usepackage{colortbl}
\usepackage{xcolor}
\usepackage{tikz}
\usetikzlibrary{shapes.geometric, arrows.meta, positioning, calc, backgrounds, 
	fit, decorations.pathmorphing, shadows, patterns}
\usepackage{pgfplots}
\pgfplotsset{compat=1.18}
\usepackage{tcolorbox}
\tcbuselibrary{skins,breakable}
\usepackage{enumitem}
\usepackage{pdflscape}
\usepackage{fancyhdr}
\usepackage{setspace}
\usepackage{titlesec}
\usepackage{float}
\usepackage{pdfpages}
\usepackage{hyperref}

% ──────────────────────────────────────────────────────────────────────────────
%                              تعریف رنگ‌ها
% ──────────────────────────────────────────────────────────────────────────────
\definecolor{bleurepublique}{RGB}{0, 35, 149}
\definecolor{rougerevolution}{RGB}{237, 41, 57}
\definecolor{orroyal}{RGB}{255, 215, 0}
\definecolor{vertnapoleon}{RGB}{0, 100, 0}
\definecolor{violetempire}{RGB}{128, 0, 128}
\definecolor{fondclair}{RGB}{255, 253, 240}
\definecolor{gris}{RGB}{128, 128, 128}
\definecolor{grisclair}{RGB}{245, 245, 245}
\definecolor{noirsombre}{RGB}{30, 30, 30}

% رنگ‌های کمکی از پیش ترکیب‌شده (برای جلوگیری از خطای TikZ)
\definecolor{bleulight}{RGB}{230, 235, 250}
\definecolor{rougelight}{RGB}{253, 235, 237}
\definecolor{vertlight}{RGB}{235, 250, 235}
\definecolor{violetlight}{RGB}{245, 235, 250}
\definecolor{orroyallight}{RGB}{255, 250, 230}
\definecolor{grislight}{RGB}{248, 248, 248}

\definecolor{bleumid}{RGB}{180, 195, 230}
\definecolor{rougemid}{RGB}{245, 180, 185}
\definecolor{vertmid}{RGB}{180, 220, 180}
\definecolor{violetmid}{RGB}{210, 180, 220}
\definecolor{orroyalmid}{RGB}{255, 240, 180}

\definecolor{orroyaldark}{RGB}{200, 170, 0}

% ──────────────────────────────────────────────────────────────────────────────
%                              فونت و زبان فارسی
% ──────────────────────────────────────────────────────────────────────────────
\usepackage{fontspec}
% اگر فونت در سیستم نصب است:
\setmainfont{Vazirmatn}

\usepackage{xepersian}
\settextfont{Vazirmatn}
\setdigitfont{Vazirmatn}

% ──────────────────────────────────────────────────────────────────────────────
%                              hyperref settings
% ──────────────────────────────────────────────────────────────────────────────
\hypersetup{
	colorlinks=true,
	linkcolor=bleurepublique,
	urlcolor=bleurepublique,
	citecolor=vertnapoleon
}

% ──────────────────────────────────────────────────────────────────────────────
%                              تعریف کادرها
% ──────────────────────────────────────────────────────────────────────────────

% کادر خلاصه فصل (آبی)
\newtcolorbox{kholasebox}[1][]{
	enhanced,
	breakable,
	colback=bleulight,
	colframe=bleurepublique,
	coltitle=white,
	fonttitle=\bfseries\large,
	title={#1},
	boxrule=2pt,
	arc=4pt,
	left=10pt,
	right=10pt,
	top=10pt,
	bottom=10pt,
	drop shadow={opacity=0.3}
}

% کادر نقل قول (طلایی)
\newtcolorbox{naghlbox}[1][]{
	enhanced,
	breakable,
	colback=orroyallight,
	colframe=orroyaldark,
	coltitle=black,
	fonttitle=\bfseries,
	title={#1},
	boxrule=1.5pt,
	arc=3pt,
	borderline west={4pt}{0pt}{orroyal},
	left=15pt,
	right=10pt,
	top=8pt,
	bottom=8pt
}

% کادر الگو/درس (سبز)
\newtcolorbox{olgoobox}[1][]{
	enhanced,
	breakable,
	colback=vertlight,
	colframe=vertnapoleon,
	coltitle=white,
	fonttitle=\bfseries,
	title={#1},
	boxrule=1.5pt,
	arc=4pt,
	left=10pt,
	right=10pt,
	top=8pt,
	bottom=8pt,
	before upper={\parindent15pt}
}

% کادر انقلاب/بحران (قرمز)
\newtcolorbox{enghelabbox}[1][]{
	enhanced,
	breakable,
	colback=rougelight,
	colframe=rougerevolution,
	coltitle=white,
	fonttitle=\bfseries,
	title={#1},
	boxrule=2pt,
	arc=4pt,
	left=10pt,
	right=10pt,
	top=8pt,
	bottom=8pt
}

% کادر امپراتوری (بنفش)
\newtcolorbox{empirebox}[1][]{
	enhanced,
	breakable,
	colback=violetlight,
	colframe=violetempire,
	coltitle=white,
	fonttitle=\bfseries,
	title={#1},
	boxrule=1.5pt,
	arc=4pt,
	left=10pt,
	right=10pt,
	top=8pt,
	bottom=8pt
}

% کادر نکته
\newtcolorbox{noktebox}[1][]{
	enhanced,
	colback=grisclair,
	colframe=gris,
	fonttitle=\bfseries,
	title={#1},
	boxrule=1pt,
	arc=3pt,
	left=8pt,
	right=8pt
}

% ──────────────────────────────────────────────────────────────────────────────
%                         تنظیمات صفحه‌آرایی
% ──────────────────────────────────────────────────────────────────────────────
\pagestyle{fancy}
\fancyhf{}
\fancyhead[RO]{\leftmark}
\fancyhead[LE]{\rightmark}
\fancyfoot[C]{\thepage}
\renewcommand{\headrulewidth}{1pt}
\renewcommand{\footrulewidth}{0.5pt}

\setstretch{1.5}

\titleformat{\chapter}[display]
{\normalfont\huge\bfseries\color{bleurepublique}}
{\chaptertitlename\ \thechapter}{20pt}{\Huge}

\titleformat{\section}
{\normalfont\Large\bfseries\color{bleurepublique}}
{\thesection}{1em}{}

\titleformat{\subsection}
{\normalfont\large\bfseries\color{bleurepublique}}
{\thesubsection}{1em}{}

% ══════════════════════════════════════════════════════════════════════════════
%                              شروع سند
% ══════════════════════════════════════════════════════════════════════════════
\begin{document}
	
	% ──────────────────────────────────────────────────────────────────────────────
	%                              صفحه عنوان
	% ──────────────────────────────────────────────────────────────────────────────
	\begin{titlepage}
		\pagecolor{fondclair}
		\begin{tikzpicture}[remember picture, overlay]
			% پس‌زمینه با پرچم فرانسه
			\fill[bleurepublique] (current page.north west) rectangle 
			([xshift=3cm]current page.south west);
			\fill[rougerevolution] (current page.north east) rectangle 
			([xshift=-3cm]current page.south east);
			
			% کادر مرکزی
			\node[
			rectangle,
			draw=orroyal,
			line width=3pt,
			fill=white,
			minimum width=12cm,
			minimum height=18cm,
			drop shadow={opacity=0.4}
			] at (current page.center) {};
			
			% عنوان اصلی
			\node[
			text=bleurepublique,
			font=\Huge\bfseries,
			text width=10cm,
			align=center
			] at ([yshift=4cm]current page.center) {
				تاریخ تحولات فرانسه
			};
			
			% زیرعنوان
			\node[
			text=rougerevolution,
			font=\LARGE\bfseries,
			text width=10cm,
			align=center
			] at ([yshift=2cm]current page.center) {
				انقلاب‌ها و جمهوری‌ها
			};
			
			% محدوده زمانی
			\node[
			text=gris,
			font=\Large,
			text width=10cm,
			align=center
			] at ([yshift=0.5cm]current page.center) {
				۱۷۸۹ — ۲۰۲۴
			};
			
			% خط تزئینی
			\draw[orroyal, line width=2pt] 
			([yshift=-0.5cm, xshift=-4cm]current page.center) -- 
			([yshift=-0.5cm, xshift=4cm]current page.center);
			
			% توضیح
			\node[
			text=noirsombre,
			font=\normalsize,
			text width=9cm,
			align=center
			] at ([yshift=-2cm]current page.center) {
				پژوهش چندوجهی در تاریخ سیاسی، اجتماعی، اقتصادی و فرهنگی\\[5pt]
				با تأکید بر نقش طبقات، ذهنیت‌ها و فلسفه سیاسی
			};
			
			% نماد جمهوری
			\node[
			text=bleurepublique,
			font=\Huge
			] at ([yshift=-5cm]current page.center) {
				\textbf{RF}
			};
			
			% تاریخ
			\node[
			text=gris,
			font=\small
			] at ([yshift=-7cm]current page.center) {
				۱۴۰۳
			};
		\end{tikzpicture}
	\end{titlepage}
	\nopagecolor
	
	% ──────────────────────────────────────────────────────────────────────────────
	%                              فهرست مطالب
	% ──────────────────────────────────────────────────────────────────────────────
	\tableofcontents
	\newpage
	
	% ██████████████████████████████████████████████████████████████████████████████
	%
	%                    فصل ۰: خلاصه مدیریتی و نقشه‌های مفهومی
	%
	% ██████████████████████████████████████████████████████████████████████████████
	
	\chapter*{فصل صفر: خلاصه مدیریتی}
	\addcontentsline{toc}{chapter}{فصل صفر: خلاصه مدیریتی}
	
	\begin{kholasebox}[خلاصه فصل صفر]
		این فصل نمای کلی از ۲۳۵ سال تحولات فرانسه را ارائه می‌دهد. فرانسه از ۱۷۸۹ تا ۲۰۲۴ شاهد \textbf{سه انقلاب بزرگ}، \textbf{دو امپراتوری}، \textbf{دو بازگشت سلطنتی}، \textbf{پنج جمهوری}، و \textbf{یک دوره اشغال} بوده است. این تحولات فرانسه را به «آزمایشگاه انقلاب‌ها» تبدیل کرده و الگوهای مهمی برای فهم تغییرات سیاسی ارائه می‌دهد.
	\end{kholasebox}
	
	\section*{۰.۱ نقشه کلان تحولات فرانسه}
	
	% ──────────────────────────────────────────────────────────────────────────────
	%                    خط زمانی کلان ۱۷۸۹-۲۰۲۴
	% ──────────────────────────────────────────────────────────────────────────────
	
%	\begin{landscape}
	\begin{figure}[H]
		\centering
		\begin{tikzpicture}[
			scale=1,
			transform shape,
			every node/.style={font=\small}
			]
			% خط اصلی زمان
			\draw[line width=2pt, gris] (0,0) -- (16,0);
			
			% دوره‌های مختلف با رنگ‌بندی
			\fill[orroyalmid] (0,0.3) rectangle (0.5,0.6);
			\fill[rougemid] (0.5,0.3) rectangle (2,0.6);
			\fill[violetmid] (2,0.3) rectangle (3.5,0.6);
			\fill[orroyalmid] (3.5,0.3) rectangle (4.5,0.6);
			\fill[orroyallight] (4.5,0.3) rectangle (5.8,0.6);
			\fill[bleumid] (5.8,0.3) rectangle (6.3,0.6);
			\fill[violetlight] (6.3,0.3) rectangle (7.8,0.6);
			\fill[bleurepublique!70] (7.8,0.3) rectangle (11.5,0.6);
			\fill[gris!70] (11.5,0.3) rectangle (12,0.6);
			\fill[bleumid] (12,0.3) rectangle (13,0.6);
			\fill[bleurepublique] (13,0.3) rectangle (16,0.6);
			
			% نشانگرهای تاریخ
			\foreach \x/\year in {0.5/۱۷۸۹, 2/۱۷۹۹, 3.5/۱۸۱۵, 4.5/۱۸۳۰, 
				5.8/۱۸۴۸, 6.3/۱۸۵۲, 7.8/۱۸۷۰, 11.5/۱۹۴۰,
				12/۱۹۴۴, 13/۱۹۵۸, 16/۲۰۲۴}{
				\draw[black, line width=1pt] (\x,-0.2) -- (\x,0.2);
				\node[below, text=black] at (\x,-0.3) {\tiny\year};
			}
			
			% برچسب‌های دوره‌ها
			\node[above, text=rougerevolution, font=\tiny\bfseries] at (1.25,0.7) {انقلاب};
			\node[above, text=violetempire, font=\tiny\bfseries] at (2.75,0.7) {ناپلئون};
			\node[above, text=bleurepublique, font=\tiny\bfseries] at (9.65,0.7) {جمهوری سوم};
			\node[above, text=bleurepublique, font=\tiny\bfseries] at (14.5,0.7) {جمهوری پنجم};
			
			% راهنما
			\node[right, font=\footnotesize] at (0,-1.5) {
				\begin{tabular}{r@{\hspace{3pt}}l@{\hspace{10pt}}r@{\hspace{3pt}}l}
					\textcolor{rougerevolution}{$\blacksquare$} & انقلاب &
					\textcolor{bleurepublique}{$\blacksquare$} & جمهوری \\
					\textcolor{violetempire}{$\blacksquare$} & امپراتوری &
					\textcolor{orroyal}{$\blacksquare$} & سلطنت \\
				\end{tabular}
			};
		\end{tikzpicture}
		\caption{خط زمانی کلان تحولات فرانسه (۱۷۸۹-۲۰۲۴)}
	\end{figure}
%	\end{landscape}
	\section*{۰.۲ آمار کلیدی}
	
	\begin{table}[H]
		\centering
		\caption{آمار کلیدی تحولات سیاسی فرانسه (۱۷۸۹-۲۰۲۴)}
		\begin{tabular}{|>{\columncolor{bleulight}}r|c|c|}
			\hline
			\rowcolor{bleumid}
			\textbf{شاخص} & \textbf{تعداد} & \textbf{توضیح} \\
			\hline
			تعداد جمهوری‌ها & ۵ & اول تا پنجم \\
			\hline
			\rowcolor{grisclair}
			تعداد انقلاب‌های بزرگ & ۳ & ۱۷۸۹، ۱۸۳۰، ۱۸۴۸ \\
			\hline
			تعداد امپراتوری‌ها & ۲ & ناپلئون اول و سوم \\
			\hline
			\rowcolor{grisclair}
			تعداد قانون اساسی & ۱۵+ & از ۱۷۹۱ تا ۱۹۵۸ \\
			\hline
			میانگین عمر رژیم‌ها & ۱۵ سال & قبل از جمهوری سوم \\
			\hline
			\rowcolor{grisclair}
			طولانی‌ترین جمهوری & ۷۰ سال & جمهوری سوم \\
			\hline
			کوتاه‌ترین جمهوری & ۴ سال & جمهوری دوم \\
			\hline
		\end{tabular}
	\end{table}
	
	\section*{۰.۳ نقشه مفهومی: چرا فرانسه آزمایشگاه انقلاب‌ها شد؟}
	
	\begin{figure}[H]
		\centering
		\begin{tikzpicture}[
			node distance=2cm,
			mainnode/.style={
				rectangle,
				draw=bleurepublique,
				line width=2pt,
				fill=bleulight,
				text=black,
				minimum width=3.5cm,
				minimum height=1cm,
				align=center,
				font=\small\bfseries
			},
			rougenode/.style={
				rectangle,
				draw=rougerevolution,
				line width=1pt,
				fill=rougelight,
				text=black,
				minimum width=3cm,
				minimum height=0.8cm,
				align=center,
				font=\footnotesize
			},
			vertnode/.style={
				rectangle,
				draw=vertnapoleon,
				line width=1pt,
				fill=vertlight,
				text=black,
				minimum width=3cm,
				minimum height=0.8cm,
				align=center,
				font=\footnotesize
			},
			violetnode/.style={
				rectangle,
				draw=violetempire,
				line width=1pt,
				fill=violetlight,
				text=black,
				minimum width=3cm,
				minimum height=0.8cm,
				align=center,
				font=\footnotesize
			},
			orroyalnode/.style={
				rectangle,
				draw=orroyaldark,
				line width=1pt,
				fill=orroyallight,
				text=black,
				minimum width=3cm,
				minimum height=0.8cm,
				align=center,
				font=\footnotesize
			},
			smallrouge/.style={
				rectangle,
				draw=rougerevolution,
				line width=1pt,
				fill=rougelight,
				text=black,
				minimum width=2.5cm,
				minimum height=0.6cm,
				align=center,
				font=\scriptsize
			},
			smallvert/.style={
				rectangle,
				draw=vertnapoleon,
				line width=1pt,
				fill=vertlight,
				text=black,
				minimum width=2.5cm,
				minimum height=0.6cm,
				align=center,
				font=\scriptsize
			},
			smallviolet/.style={
				rectangle,
				draw=violetempire,
				line width=1pt,
				fill=violetlight,
				text=black,
				minimum width=2.5cm,
				minimum height=0.6cm,
				align=center,
				font=\scriptsize
			},
			smallorroyal/.style={
				rectangle,
				draw=orroyaldark,
				line width=1pt,
				fill=orroyallight,
				text=black,
				minimum width=2.5cm,
				minimum height=0.6cm,
				align=center,
				font=\scriptsize
			},
			arrow/.style={
				->,
				>=Stealth,
				line width=1pt,
				gris
			}
			]
			% گره مرکزی
			\node[mainnode] (center) {
				\begin{tabular}{c}
					فرانسه\\
					آزمایشگاه انقلاب‌ها
				\end{tabular}
			};
			
			% گره‌های اصلی
			\node[rougenode, above left=2cm and 1cm of center] (ideol) {
				\begin{tabular}{c}
					عامل ایدئولوژیک\\
					روشنگری و انتقاد
				\end{tabular}
			};
			
			\node[vertnode, above right=2cm and 1cm of center] (econ) {
				\begin{tabular}{c}
					عامل اقتصادی\\
					نابرابری ساختاری
				\end{tabular}
			};
			
			\node[violetnode, below left=2cm and 1cm of center] (social) {
				\begin{tabular}{c}
					عامل اجتماعی\\
					تعارض طبقاتی
				\end{tabular}
			};
			
			\node[orroyalnode, below right=2cm and 1cm of center] (polit) {
				\begin{tabular}{c}
					عامل سیاسی\\
					تمرکز و استبداد
				\end{tabular}
			};
			
			% گره‌های فرعی
			\node[smallrouge, left=1.5cm of ideol] (ideol1) {
				\begin{tabular}{c}
					ولتر، روسو\\
					مونتسکیو
				\end{tabular}
			};
			
			\node[smallvert, right=1.5cm of econ] (econ1) {
				\begin{tabular}{c}
					مالیات ناعادلانه\\
					بحران مالی
				\end{tabular}
			};
			
			\node[smallviolet, left=1.5cm of social] (social1) {
				\begin{tabular}{c}
					بورژوازی صاعد\\
					اشراف مقاوم
				\end{tabular}
			};
			
			\node[smallorroyal, right=1.5cm of polit] (polit1) {
				\begin{tabular}{c}
					سلطنت مطلقه\\
					ضعف نهادها
				\end{tabular}
			};
			
			% پیکان‌ها
			\draw[arrow] (ideol) -- (center);
			\draw[arrow] (econ) -- (center);
			\draw[arrow] (social) -- (center);
			\draw[arrow] (polit) -- (center);
			\draw[arrow] (ideol1) -- (ideol);
			\draw[arrow] (econ1) -- (econ);
			\draw[arrow] (social1) -- (social);
			\draw[arrow] (polit1) -- (polit);
			
		\end{tikzpicture}
		\caption{نقشه مفهومی: عوامل انقلابی بودن فرانسه}
	\end{figure}
	
	\section*{۰.۴ چرخه‌های تحول}
	
	\begin{figure}[H]
		\centering
		\begin{tikzpicture}[
			scale=1.2,
			rougecycle/.style={
				rectangle,
				draw=rougerevolution,
				line width=2pt,
				fill=rougelight,
				text=black,
				minimum width=2.5cm,
				minimum height=1cm,
				align=center,
				font=\small\bfseries
			},
			violetcycle/.style={
				rectangle,
				draw=violetempire,
				line width=2pt,
				fill=violetlight,
				text=black,
				minimum width=2.5cm,
				minimum height=1cm,
				align=center,
				font=\small\bfseries
			},
			bleucycle/.style={
				rectangle,
				draw=bleurepublique,
				line width=2pt,
				fill=bleulight,
				text=black,
				minimum width=2.5cm,
				minimum height=1cm,
				align=center,
				font=\small\bfseries
			},
			vertcycle/.style={
				rectangle,
				draw=vertnapoleon,
				line width=2pt,
				fill=vertlight,
				text=black,
				minimum width=2.5cm,
				minimum height=1cm,
				align=center,
				font=\small\bfseries
			}
			]
			% مراحل چرخه
			\node[rougecycle] (crisis) at (0,3) {
				\begin{tabular}{c}
					۱. بحران\\
					اقتصادی-سیاسی
				\end{tabular}
			};
			
			\node[rougecycle] (revolt) at (4,3) {
				\begin{tabular}{c}
					۲. شورش\\
					مردمی
				\end{tabular}
			};
			
			\node[violetcycle] (radical) at (6,0) {
				\begin{tabular}{c}
					۳. رادیکال‌شدن\\
					و ترور
				\end{tabular}
			};
			
			\node[violetcycle] (reaction) at (4,-3) {
				\begin{tabular}{c}
					۴. واکنش\\
					و کودتا
				\end{tabular}
			};
			
			\node[bleucycle] (strong) at (0,-3) {
				\begin{tabular}{c}
					۵. دولت قوی\\
					امپراتوری/استبداد
				\end{tabular}
			};
			
			\node[vertcycle] (stabil) at (-4,0) {
				\begin{tabular}{c}
					۶. تثبیت\\
					موقت
				\end{tabular}
			};
			
			% پیکان‌ها
			\draw[->, >=Stealth, line width=2pt, gris] (crisis) -- (revolt);
			\draw[->, >=Stealth, line width=2pt, gris] (revolt) -- (radical);
			\draw[->, >=Stealth, line width=2pt, gris] (radical) -- (reaction);
			\draw[->, >=Stealth, line width=2pt, gris] (reaction) -- (strong);
			\draw[->, >=Stealth, line width=2pt, gris] (strong) -- (stabil);
			\draw[->, >=Stealth, line width=2pt, rougerevolution, dashed] 
			(stabil) -- (crisis) node[midway, left, font=\scriptsize] {بازگشت};
			
		\end{tikzpicture}
		\caption{چرخه تکرارشونده تحولات فرانسه}
	\end{figure}
	
	\section*{۰.۵ پنج جمهوری: مقایسه اجمالی}
	
	\begin{table}[H]
		\centering
		\small
		\caption{مقایسه پنج جمهوری فرانسه}
		\begin{tabular}{|>{\bfseries}r|c|c|c|c|}
			\hline
			\rowcolor{bleumid}
			\textbf{جمهوری} & \textbf{دوره} & \textbf{طول عمر} & \textbf{پایان} & \textbf{ویژگی اصلی} \\
			\hline
			اول & ۱۷۹۲-۱۸۰۴ & ۱۲ سال & کودتا & رادیکالیسم \\
			\hline
			\rowcolor{grisclair}
			دوم & ۱۸۴۸-۱۸۵۲ & ۴ سال & کودتا & ناپایداری \\
			\hline
			سوم & ۱۸۷۰-۱۹۴۰ & ۷۰ سال & شکست نظامی & پارلمانتاریسم \\
			\hline
			\rowcolor{grisclair}
			چهارم & ۱۹۴۶-۱۹۵۸ & ۱۲ سال & بحران الجزایر & ضعف اجرایی \\
			\hline
			پنجم & ۱۹۵۸-اکنون & ۶۶+ سال & ادامه دارد & ریاست‌محوری \\
			\hline
		\end{tabular}
	\end{table}
	
	\section*{۰.۶ الگوهای کلیدی}
	
	\begin{olgoobox}[الگوهای اصلی قابل استخراج]
		\begin{enumerate}[nosep]
			\item \textbf{الگوی بحران مالی}: هر انقلاب بزرگ با بحران مالی آغاز شد
			\item \textbf{الگوی طبقه میانی}: بورژوازی موتور تحولات بود
			\item \textbf{الگوی رادیکال‌شدن}: انقلاب‌ها به چپ می‌لغزند، سپس واکنش می‌آید
			\item \textbf{الگوی ناپلئونی}: بحران → مرد قوی → تمرکز قدرت
			\item \textbf{الگوی پاریس‌محوری}: پایتخت تعیین‌کننده سرنوشت کشور است
			\item \textbf{الگوی نهادسازی تدریجی}: جمهوری‌ها از تجربه‌های قبلی آموختند
		\end{enumerate}
	\end{olgoobox}
	
	\section*{۰.۷ نقش فلسفه سیاسی: نگاه اجمالی}
	
	\begin{naghlbox}[نقل قول کلیدی]
		\textit{«ایده‌ها، بدون شرایط اجتماعی، قدرت تغییر ندارند؛ اما شرایط اجتماعی بدون ایده‌ها، جهت ندارند.»}
		\begin{flushright}
			— الکسی دو توکویل
		\end{flushright}
	\end{naghlbox}
	
	فلسفه سیاسی در تحولات فرانسه نقشی دوگانه داشت:
	
	\begin{itemize}[nosep]
		\item \textbf{در سطح نخبگان}: ایده‌های روشنگری چارچوب مفهومی انقلاب را فراهم کرد
		\item \textbf{در سطح عوام}: شعارهای ساده‌شده («آزادی، برابری، برادری») بسیج‌کننده بود
		\item \textbf{الاهیات سیاسی}: سکولاریسم فرانسوی (لائیسیته) خود محصول تعارض با کلیسا بود
	\end{itemize}
	
	% ══════════════════════════════════════════════════════════════════════════════
	\newpage
	% ══════════════════════════════════════════════════════════════════════════════
	
	% ██████████████████████████████████████████████████████████████████████████████
	%
	%                    فصل ۱: مقدمه و چارچوب نظری
	%
	% ██████████████████████████████████████████████████████████████████████████████
	
	\chapter{مقدمه و چارچوب نظری}
	
	\begin{kholasebox}[خلاصه فصل اول]
		این فصل چارچوب نظری تحلیل تحولات فرانسه را ارائه می‌دهد. ما از رویکرد چندتخصصی استفاده می‌کنیم که تاریخ سیاسی، جامعه‌شناسی، روان‌شناسی اجتماعی، اقتصاد سیاسی و فلسفه سیاسی را ترکیب می‌کند. پرسش محوری این است: چرا فرانسه، برخلاف انگلستان، مسیر تدریجی نرفت و به «آزمایشگاه انقلاب‌ها» تبدیل شد؟
	\end{kholasebox}
	
	\section{طرح مسئله: معمای فرانسوی}
	
	فرانسه در تاریخ جهان جایگاه منحصربه‌فردی دارد. این کشور از ۱۷۸۹ تا امروز پانزده قانون اساسی، پنج جمهوری، دو امپراتوری، دو بار بازگشت سلطنت، و یک دوره اشغال را تجربه کرده است. این در حالی است که انگلستان بدون انقلاب خشونت‌آمیز، از سلطنت مطلقه به دموکراسی پارلمانی گذار کرد.
	
	\begin{enghelabbox}[پرسش محوری پژوهش]
		\textbf{چرا فرانسه نتوانست مسیر تدریجی را بپیماید؟}
		
		آیا این ناشی از ساختار اجتماعی، فرهنگ سیاسی، ایدئولوژی، یا ترکیبی از همه بود؟
	\end{enghelabbox}
	
	\subsection{فرانسه در مقایسه با همسایگان}
	
	\begin{table}[H]
		\centering
		\caption{مقایسه مسیرهای گذار به دموکراسی در اروپای غربی}
		\begin{tabular}{|>{\bfseries}r|c|c|c|}
			\hline
			\rowcolor{bleumid}
			\textbf{کشور} & \textbf{نوع گذار} & \textbf{تعداد انقلاب} & \textbf{تثبیت دموکراسی} \\
			\hline
			انگلستان & تدریجی & ۱ (۱۶۸۸) & ۱۸۳۲-۱۹۲۸ \\
			\hline
			\rowcolor{grisclair}
			فرانسه & انقلابی & ۳+ & ۱۸۷۰-۱۹۵۸ \\
			\hline
			آلمان & از بالا & ۱ (ناکام ۱۸۴۸) & ۱۹۴۹ \\
			\hline
			\rowcolor{grisclair}
			هلند & تدریجی & ۰ & قرن ۱۹ \\
			\hline
		\end{tabular}
	\end{table}
	
	\section{چارچوب نظری: رویکرد چندوجهی}
	
	برای فهم تحولات فرانسه، ما از پنج لنز تحلیلی استفاده می‌کنیم:
	
	\begin{figure}[H]
		\centering
		\begin{tikzpicture}[
			scale=0.95,
			transform shape,
			mainbox/.style={
				rectangle,
				draw=bleurepublique,
				line width=2pt,
				fill=bleulight,
				text=black,
				minimum width=4cm,
				minimum height=1.5cm,
				align=center,
				font=\small\bfseries,
				rounded corners=5pt
			},
			rougebox/.style={
				rectangle,
				draw=rougerevolution,
				line width=1.5pt,
				fill=rougelight,
				text=black,
				minimum width=3.5cm,
				minimum height=2cm,
				align=center,
				font=\scriptsize,
				rounded corners=3pt
			},
			vertbox/.style={
				rectangle,
				draw=vertnapoleon,
				line width=1.5pt,
				fill=vertlight,
				text=black,
				minimum width=3.5cm,
				minimum height=2cm,
				align=center,
				font=\scriptsize,
				rounded corners=3pt
			},
			violetbox/.style={
				rectangle,
				draw=violetempire,
				line width=1.5pt,
				fill=violetlight,
				text=black,
				minimum width=3.5cm,
				minimum height=2cm,
				align=center,
				font=\scriptsize,
				rounded corners=3pt
			},
			orroyalbox/.style={
				rectangle,
				draw=orroyaldark,
				line width=1.5pt,
				fill=orroyallight,
				text=black,
				minimum width=3.5cm,
				minimum height=2cm,
				align=center,
				font=\scriptsize,
				rounded corners=3pt
			},
			bleubox/.style={
				rectangle,
				draw=bleurepublique,
				line width=1.5pt,
				fill=bleulight,
				text=black,
				minimum width=3.5cm,
				minimum height=2cm,
				align=center,
				font=\scriptsize,
				rounded corners=3pt
			}
			]
			% مرکز
			\node[mainbox] (center) at (0,0) {
				\begin{tabular}{c}
					تحلیل چندوجهی\\
					تحولات فرانسه
				\end{tabular}
			};
			
			% پنج لنز تحلیلی
			\node[rougebox] (hist) at (-6,2.5) {
				\begin{tabular}{c}
					\textbf{۱. تاریخ سیاسی}\\[3pt]
					رویدادها، بازیگران\\
					تصمیمات، تصادفات
				\end{tabular}
			};
			
			\node[vertbox] (soc) at (-3,4) {
				\begin{tabular}{c}
					\textbf{۲. جامعه‌شناسی}\\[3pt]
					طبقات، گروه‌ها\\
					بسیج، سازمان‌دهی
				\end{tabular}
			};
			
			\node[violetbox] (psy) at (3,4) {
				\begin{tabular}{c}
					\textbf{۳. روان‌شناسی اجتماعی}\\[3pt]
					ذهنیت‌ها، ترس‌ها\\
					امیدها، نمادها
				\end{tabular}
			};
			
			\node[orroyalbox] (econ) at (6,2.5) {
				\begin{tabular}{c}
					\textbf{۴. اقتصاد سیاسی}\\[3pt]
					توزیع ثروت\\
					مالیات، بحران‌ها
				\end{tabular}
			};
			
			\node[bleubox] (phil) at (0,-3.5) {
				\begin{tabular}{c}
					\textbf{۵. فلسفه سیاسی}\\[3pt]
					ایده‌ها، مشروعیت\\
					عدالت، آزادی
				\end{tabular}
			};
			
			% اتصالات
			\draw[->, >=Stealth, line width=1.5pt, rougerevolution] (hist) -- (center);
			\draw[->, >=Stealth, line width=1.5pt, vertnapoleon] (soc) -- (center);
			\draw[->, >=Stealth, line width=1.5pt, violetempire] (psy) -- (center);
			\draw[->, >=Stealth, line width=1.5pt, orroyaldark] (econ) -- (center);
			\draw[->, >=Stealth, line width=1.5pt, bleurepublique] (phil) -- (center);
			
		\end{tikzpicture}
		\caption{چارچوب نظری پنج‌وجهی تحلیل تحولات فرانسه}
	\end{figure}
	
	\subsection{لنز اول: تاریخ سیاسی (رویدادمحور)}
	
	این لنز بر رویدادها، تصمیمات، و بازیگران تمرکز دارد. تاریخ‌نگاران سنتی مانند \textbf{آلفونس اولار} و \textbf{آلبر ماتیه} از این رویکرد استفاده کردند.
	
	\begin{noktebox}[پرسش‌های کلیدی]
		\begin{itemize}[nosep]
			\item چه کسی چه تصمیمی گرفت؟
			\item رویدادها به چه ترتیبی رخ دادند؟
			\item تصادفات چه نقشی داشتند؟
		\end{itemize}
	\end{noktebox}
	
	\subsection{لنز دوم: جامعه‌شناسی سیاسی (ساختارمحور)}
	
	این لنز بر طبقات اجتماعی و روابط قدرت تمرکز دارد. \textbf{ثدا اسکاچپول} در کتاب «دولت‌ها و انقلاب‌های اجتماعی» این رویکرد را به کار برد.
	
	\begin{table}[H]
		\centering
		\caption{طبقات اجتماعی و نقش آنها در انقلاب}
		\begin{tabular}{|>{\bfseries}r|c|c|}
			\hline
			\rowcolor{bleumid}
			\textbf{طبقه} & \textbf{خواست اصلی} & \textbf{روش عمل} \\
			\hline
			اشراف & حفظ امتیازات & مقاومت، مهاجرت \\
			\hline
			\rowcolor{grisclair}
			بورژوازی & قدرت سیاسی & قانون‌گذاری، سازش \\
			\hline
			طبقه متوسط & شایسته‌سالاری & رادیکالیسم \\
			\hline
			\rowcolor{grisclair}
			صنعتگران & نان و کار & شورش، خشونت \\
			\hline
			دهقانان & زمین، نان & قیام محلی \\
			\hline
		\end{tabular}
	\end{table}
	
	\subsection{لنز سوم: روان‌شناسی اجتماعی (ذهنیت‌ها)}
	
	مکتب \textbf{آنال} فرانسه، به ویژه \textbf{ژرژ لوفور} و \textbf{میشل ووول}، بر «ذهنیت‌ها» (mentalités) تأکید کردند: ترس‌ها، امیدها، شایعات، و نمادهای جمعی.
	
	\begin{naghlbox}[وحشت بزرگ ۱۷۸۹]
		«در تابستان ۱۷۸۹، شایعه‌ای در سراسر فرانسه پیچید که راهزنان استخدام‌شده توسط اشراف، برای سوزاندن محصول می‌آیند. این "وحشت بزرگ" نشان داد که انقلاب‌ها فقط با واقعیت‌ها نه، بلکه با تصورات هم شکل می‌گیرند.»
		\begin{flushright}
			— ژرژ لوفور، وحشت بزرگ
		\end{flushright}
	\end{naghlbox}
	
	\subsection{لنز چهارم: اقتصاد سیاسی}
	
	این لنز بر توزیع منابع و تعارضات اقتصادی تمرکز دارد. \textbf{ارنست لابروس} نشان داد که بحران‌های اقتصادی با موج‌های انقلابی همبستگی دارند.
	
	\begin{figure}[H]
		\centering
		\begin{tikzpicture}
			\begin{axis}[
				width=13cm,
				height=6cm,
				xlabel={سال},
				ylabel={شاخص قیمت نان (۱۷۸۰=۱۰۰)},
				xmin=1780, xmax=1800,
				ymin=80, ymax=220,
				xtick={1780,1785,1788,1789,1792,1795,1800},
				ytick={80,100,120,140,160,180,200,220},
				grid=both,
				grid style={gris!30},
				legend pos=north west,
				legend style={font=\small}
				]
				% خط قیمت نان
				\addplot[
				rougerevolution,
				line width=2pt,
				mark=*,
				mark options={fill=rougerevolution}
				] coordinates {
					(1780,100) (1782,95) (1784,98) (1786,105) 
					(1788,180) (1789,200) (1790,120) (1792,130)
					(1793,150) (1794,160) (1795,210) (1796,140)
					(1798,110) (1800,100)
				};
				\addlegendentry{قیمت نان}
				
			\end{axis}
			
			% نقاط انقلابی - خارج از axis
			\node[
			circle,
			draw=bleurepublique,
			fill=bleulight,
			minimum size=8pt
			] at (5.2,4.8) {};
			\node[above right, text=bleurepublique, font=\scriptsize] at (5.2,4.8) {انقلاب ۱۷۸۹};
			
			\node[
			circle,
			draw=violetempire,
			fill=violetlight,
			minimum size=8pt
			] at (7.5,3.2) {};
			\node[above, text=violetempire, font=\scriptsize] at (7.5,3.4) {دوره ترور};
			
		\end{tikzpicture}
		\caption{همبستگی قیمت نان و رویدادهای انقلابی (۱۷۸۰-۱۸۰۰)}
	\end{figure}
	
	\subsection{لنز پنجم: فلسفه سیاسی}
	
	این لنز کلیدی‌ترین لنز برای فهم «چرایی» انقلاب است. ایده‌ها فقط انقلاب را توجیه نکردند، بلکه \textbf{جهت} دادند.
	
	\section{فلسفه سیاسی و انقلاب: تحلیل عمیق}
	
	\begin{kholasebox}[پرسش کلیدی این بخش]
		فلسفه سیاسی چه نقشی در انقلاب‌های فرانسه داشت؟ آیا ایده‌ها علت انقلاب بودند یا فقط توجیه‌کننده؟ میزان آگاهی نخبگان و عوام از این ایده‌ها چقدر بود؟
	\end{kholasebox}
	
	\subsection{سه سنت فلسفی مؤثر}
	
	\begin{figure}[H]
		\centering
		\begin{tikzpicture}[
			scale=0.95,
			bleutrad/.style={
				rectangle,
				draw=bleurepublique,
				line width=2pt,
				fill=bleulight,
				text=black,
				minimum width=4.5cm,
				minimum height=3cm,
				align=center,
				font=\small,
				rounded corners=5pt
			},
			rougetrad/.style={
				rectangle,
				draw=rougerevolution,
				line width=2pt,
				fill=rougelight,
				text=black,
				minimum width=4.5cm,
				minimum height=3cm,
				align=center,
				font=\small,
				rounded corners=5pt
			},
			verttrad/.style={
				rectangle,
				draw=vertnapoleon,
				line width=2pt,
				fill=vertlight,
				text=black,
				minimum width=4.5cm,
				minimum height=3cm,
				align=center,
				font=\small,
				rounded corners=5pt
			},
			thinkerbleu/.style={
				rectangle,
				draw=bleurepublique,
				line width=1pt,
				fill=bleulight,
				text=black,
				minimum width=3cm,
				minimum height=0.7cm,
				align=center,
				font=\scriptsize
			},
			thinkerrouge/.style={
				rectangle,
				draw=rougerevolution,
				line width=1pt,
				fill=rougelight,
				text=black,
				minimum width=3cm,
				minimum height=0.7cm,
				align=center,
				font=\scriptsize
			},
			thinkervert/.style={
				rectangle,
				draw=vertnapoleon,
				line width=1pt,
				fill=vertlight,
				text=black,
				minimum width=3cm,
				minimum height=0.7cm,
				align=center,
				font=\scriptsize
			}
			]
			% سه سنت اصلی
			\node[bleutrad] (liberal) at (-5,0) {
				\begin{tabular}{c}
					\textbf{لیبرالیسم}\\[5pt]
					آزادی فردی\\
					قرارداد اجتماعی\\
					حقوق طبیعی\\
					تفکیک قوا
				\end{tabular}
			};
			
			\node[rougetrad] (republican) at (0,0) {
				\begin{tabular}{c}
					\textbf{جمهوری‌خواهی}\\[5pt]
					اراده عمومی\\
					فضیلت مدنی\\
					مشارکت سیاسی\\
					برابری
				\end{tabular}
			};
			
			\node[verttrad] (radical) at (5,0) {
				\begin{tabular}{c}
					\textbf{رادیکالیسم}\\[5pt]
					برابری اقتصادی\\
					دموکراسی مستقیم\\
					بازتوزیع ثروت\\
					سوسیالیسم اولیه
				\end{tabular}
			};
			
			% متفکران هر سنت
			\node[thinkerbleu] at (-5,2.5) {مونتسکیو، لاک};
			\node[thinkerbleu] at (-5,3.3) {ولتر، کندیاک};
			
			\node[thinkerrouge] at (0,2.5) {روسو، مابلی};
\node[thinkerrouge] at (0,3.3) {روبسپیر، سن‌ژوست};

\node[thinkervert] at (5,2.5) {بابوف، مورلی};
\node[thinkervert] at (5,3.3) {مارا، ابر};

% پیکان‌های اثرگذاری
\draw[->, >=Stealth, line width=1.5pt, gris, dashed] 
(liberal) -- (republican) node[midway, above, font=\tiny] {تأثیر};
\draw[->, >=Stealth, line width=1.5pt, gris, dashed] 
(republican) -- (radical) node[midway, above, font=\tiny] {رادیکال‌تر};

\end{tikzpicture}
\caption{سه سنت فلسفه سیاسی مؤثر بر انقلاب فرانسه}
\end{figure}

\subsubsection{۱. سنت لیبرالی: مونتسکیو و ولتر}

\begin{table}[H]
\centering
\caption{مقایسه آرای مونتسکیو و ولتر}
\begin{tabular}{|>{\bfseries}r|p{5cm}|p{5cm}|}
\hline
\rowcolor{bleumid}
\textbf{موضوع} & \textbf{مونتسکیو (۱۶۸۹-۱۷۵۵)} & \textbf{ولتر (۱۶۹۴-۱۷۷۸)} \\
\hline
آزادی & تفکیک قوا، توازن نهادی & آزادی بیان، تساهل دینی \\
\hline
\rowcolor{grisclair}
حکومت مطلوب & سلطنت مشروطه (انگلیسی) & استبداد روشنفکرانه \\
\hline
دین & نقد ملایم کلیسا & ضدیت شدید با کلیسا \\
\hline
\rowcolor{grisclair}
روش & تحلیل ساختاری، تاریخی & طنز، انتقاد اجتماعی \\
\hline
تأثیر & قانون اساسی‌گرایی & سکولاریسم \\
\hline
\end{tabular}
\end{table}

\begin{naghlbox}[مونتسکیو درباره تفکیک قوا]
«وقتی در یک شخص یا یک نهاد، قدرت قانون‌گذاری با قدرت اجرایی متحد شود، آزادی وجود ندارد؛ زیرا می‌توان ترسید که همان پادشاه یا همان سنا، قوانین ظالمانه وضع کند و ظالمانه اجرا نماید.»
\begin{flushright}
— مونتسکیو، روح‌القوانین (۱۷۴۸)
\end{flushright}
\end{naghlbox}

\subsubsection{۲. سنت جمهوری‌خواهانه: ژان-ژاک روسو}

روسو مهم‌ترین فیلسوف انقلاب فرانسه است. او مفهوم \textbf{«اراده عمومی»} (volonté générale) را مطرح کرد که بعدها توسط ژاکوبن‌ها رادیکالیزه شد.

\begin{enghelabbox}[مفاهیم کلیدی روسو]
\begin{description}[nosep]
\item[قرارداد اجتماعی:] مردم حاکمیت را به «کل» می‌سپارند، نه به فرد یا گروه
\item[اراده عمومی:] خیر عمومی که با رأی اکثریت قابل کشف است
\item[آزادی واقعی:] اطاعت از قانونی که خود وضع کرده‌ایم
\item[فضیلت مدنی:] شهروندان باید منافع جمعی را بر فردی ترجیح دهند
\end{description}
\end{enghelabbox}

\begin{naghlbox}[روسو درباره آزادی]
«انسان آزاد زاده می‌شود، اما همه‌جا در زنجیر است. کسی که خود را ارباب دیگران می‌پندارد، خود برده‌تر از آنهاست. این تغییر چگونه رخ داد؟ نمی‌دانم. چه چیزی می‌تواند آن را مشروع سازد؟ گمان می‌کنم می‌توانم به این پرسش پاسخ دهم.»
\begin{flushright}
— ژان-ژاک روسو، قرارداد اجتماعی (۱۷۶۲)
\end{flushright}
\end{naghlbox}

\textbf{تناقض روسو}: روسو هم الهام‌بخش آزادی‌خواهان بود و هم توجیه‌گر استبداد. مفهوم «اراده عمومی» می‌توانست به این معنا تفسیر شود که اقلیت مخالف، «دشمن مردم» است و باید «مجبور به آزادی» شود. روبسپیر این تفسیر را به کار بست.

\subsubsection{۳. سنت رادیکال: از مابلی تا بابوف}

\begin{figure}[H]
\centering
\begin{tikzpicture}[
scale=0.9,
transform shape,
thinker/.style={
	rectangle,
	draw=vertnapoleon,
	line width=1.5pt,
	fill=vertlight,
	text=black,
	minimum width=3cm,
	minimum height=1.2cm,
	align=center,
	font=\small
},
idea/.style={
	rectangle,
	draw=vertnapoleon,
	line width=1pt,
	fill=white,
	text=black,
	minimum width=4cm,
	minimum height=0.8cm,
	align=center,
	font=\scriptsize
}
]
% متفکران
\node[thinker] (morelly) at (0,0) {
	\begin{tabular}{c}
		\textbf{مورلی}\\
		{\scriptsize (قرن ۱۸)}
	\end{tabular}
};

\node[thinker] (mably) at (4,0) {
	\begin{tabular}{c}
		\textbf{مابلی}\\
		{\scriptsize (۱۷۰۹-۱۷۸۵)}
	\end{tabular}
};

\node[thinker] (marat) at (8,0) {
	\begin{tabular}{c}
		\textbf{مارا}\\
		{\scriptsize (۱۷۴۳-۱۷۹۳)}
	\end{tabular}
};

\node[thinker] (babeuf) at (12,0) {
	\begin{tabular}{c}
		\textbf{بابوف}\\
		{\scriptsize (۱۷۶۰-۱۷۹۷)}
	\end{tabular}
};

% ایده‌ها
\node[idea] at (0,-1.8) {مالکیت اشتراکی};
\node[idea] at (4,-1.8) {برابری اقتصادی};
\node[idea] at (8,-1.8) {دیکتاتوری انقلابی};
\node[idea] at (12,-1.8) {کمونیسم اولیه};

% پیکان‌ها
\draw[->, >=Stealth, line width=1.5pt, vertnapoleon] (morelly) -- (mably);
\draw[->, >=Stealth, line width=1.5pt, vertnapoleon] (mably) -- (marat);
\draw[->, >=Stealth, line width=1.5pt, vertnapoleon] (marat) -- (babeuf);

% برچسب
\node[below, font=\footnotesize, text=gris] at (6,-3) {
	خط سیر رادیکال‌شدن: از نقد مالکیت تا توطئه برابرها
};

\end{tikzpicture}
\caption{تکامل اندیشه رادیکال در قرن هجدهم فرانسه}
\end{figure}

\begin{table}[H]
\centering
\caption{مقایسه سه سطح رادیکالیسم}
\begin{tabular}{|>{\bfseries}r|p{4cm}|p{4cm}|p{4cm}|}
\hline
\rowcolor{vertlight}
\textbf{سطح} & \textbf{لیبرال} & \textbf{جمهوری‌خواه} & \textbf{رادیکال} \\
\hline
برابری & برابری حقوقی & برابری سیاسی & برابری اقتصادی \\
\hline
\rowcolor{grisclair}
مالکیت & حق مقدس & حق مشروط & باید محدود شود \\
\hline
دولت & حداقلی & فعال در تربیت & بازتوزیع‌کننده \\
\hline
\rowcolor{grisclair}
روش & اصلاح تدریجی & انقلاب سیاسی & انقلاب اجتماعی \\
\hline
\end{tabular}
\end{table}

\subsection{میزان آگاهی از فلسفه سیاسی: نخبگان و عوام}

یکی از پرسش‌های کلیدی این است: آیا مردم عادی واقعاً ایده‌های روشنگری را می‌شناختند؟

\begin{olgoobox}[یافته کلیدی: دو سطح آگاهی]
تحقیقات تاریخی (به‌ویژه کارهای \textbf{روژه شارتیه} و \textbf{رابرت دارنتن}) نشان می‌دهد که آگاهی از فلسفه سیاسی در دو سطح متفاوت عمل می‌کرد:

\begin{enumerate}[nosep]
\item \textbf{سطح نخبگان}: آشنایی مستقیم با متون
\item \textbf{سطح عوام}: دریافت غیرمستقیم از طریق واسطه‌ها
\end{enumerate}
\end{olgoobox}

\begin{figure}[H]
\centering
\begin{tikzpicture}[
scale=0.95,
transform shape,
violetlevel/.style={
	rectangle,
	draw=violetempire,
	line width=1.5pt,
	fill=violetlight,
	text=black,
	minimum width=11cm,
	minimum height=1.5cm,
	align=center,
	font=\small
},
bleulevel/.style={
	rectangle,
	draw=bleurepublique,
	line width=1.5pt,
	fill=bleulight,
	text=black,
	minimum width=11cm,
	minimum height=1.5cm,
	align=center,
	font=\small
},
vertlevel/.style={
	rectangle,
	draw=vertnapoleon,
	line width=1.5pt,
	fill=vertlight,
	text=black,
	minimum width=11cm,
	minimum height=1.5cm,
	align=center,
	font=\small
},
rougelevel/.style={
	rectangle,
	draw=rougerevolution,
	line width=1.5pt,
	fill=rougelight,
	text=black,
	minimum width=11cm,
	minimum height=1.5cm,
	align=center,
	font=\small
},
orroyallevel/.style={
	rectangle,
	draw=orroyaldark,
	line width=1.5pt,
	fill=orroyallight,
	text=black,
	minimum width=11cm,
	minimum height=1.5cm,
	align=center,
	font=\small
},
arrow/.style={
	->,
	>=Stealth,
	line width=2pt,
	gris
}
]
% سطوح جامعه
\node[violetlevel] (philosophers) at (0,6) {
	\begin{tabular}{c}
		\textbf{فیلسوفان} (روسو، ولتر، مونتسکیو)\\
		{\footnotesize تولید ایده‌های اصلی — مخاطب: نخبگان باسواد}
	\end{tabular}
};

\node[bleulevel] (salons) at (0,4) {
	\begin{tabular}{c}
		\textbf{سالن‌ها، آکادمی‌ها، لژهای ماسونی}\\
		{\footnotesize بحث و انتشار — مخاطب: اشراف روشنفکر و بورژوازی بالا}
	\end{tabular}
};

\node[vertlevel] (press) at (0,2) {
	\begin{tabular}{c}
		\textbf{مطبوعات، جزوات، تئاتر}\\
		{\footnotesize ساده‌سازی و عمومی‌سازی — مخاطب: طبقه متوسط شهری}
	\end{tabular}
};

\node[rougelevel] (oral) at (0,0) {
	\begin{tabular}{c}
		\textbf{فرهنگ شفاهی: کافه، میخانه، بازار}\\
		{\footnotesize شایعه، ترانه، شعار — مخاطب: توده مردم}
	\end{tabular}
};

\node[orroyallevel] (rural) at (0,-2) {
	\begin{tabular}{c}
		\textbf{روستا: کشیش، معلم، سرباز بازگشته}\\
		{\footnotesize انتقال کند و گزینشی — مخاطب: دهقانان}
	\end{tabular}
};

% پیکان‌ها
\draw[arrow] (philosophers) -- (salons);
\draw[arrow] (salons) -- (press);
\draw[arrow] (press) -- (oral);
\draw[arrow] (oral) -- (rural);

% برچسب زمانی
\node[right, font=\scriptsize, text=gris] at (6,6) {دهه ۱۷۵۰};
\node[right, font=\scriptsize, text=gris] at (6,4) {دهه ۱۷۶۰-۷۰};
\node[right, font=\scriptsize, text=gris] at (6,2) {دهه ۱۷۸۰};
\node[right, font=\scriptsize, text=gris] at (6,0) {۱۷۸۸-۸۹};
\node[right, font=\scriptsize, text=gris] at (6,-2) {۱۷۸۹-۹۳};

\end{tikzpicture}
\caption{مسیر انتشار ایده‌های روشنگری در لایه‌های جامعه فرانسه}
\end{figure}

\subsubsection{آگاهی نخبگان}

\begin{table}[H]
\centering
\caption{میزان آشنایی گروه‌های نخبه با متون روشنگری}
\begin{tabular}{|>{\bfseries}r|c|c|c|}
\hline
\rowcolor{bleumid}
\textbf{گروه} & \textbf{دایره‌المعارف} & \textbf{قرارداد اجتماعی} & \textbf{روح‌القوانین} \\
\hline
اشراف دربار & متوسط & کم & بالا \\
\hline
\rowcolor{grisclair}
پارلمان‌ها & بالا & متوسط & بسیار بالا \\
\hline
روحانیون بالا & کم & بسیار کم & متوسط \\
\hline
\rowcolor{grisclair}
بورژوازی بالا & بالا & بالا & بالا \\
\hline
وکلا و پزشکان & بسیار بالا & بسیار بالا & بسیار بالا \\
\hline
\end{tabular}
\end{table}

\subsubsection{آگاهی عوام: شواهد و محدودیت‌ها}

\begin{noktebox}[شواهد آگاهی مردم عادی]
\begin{itemize}[nosep]
\item \textbf{دفاتر شکایات (Cahiers de doléances)}: ۶۰,۰۰۰ دفتر از سراسر فرانسه که خواست‌های مردم را نشان می‌دهد
\item \textbf{شعارها و ترانه‌ها}: «آزادی، برابری، برادری» ساده‌سازی ایده‌های پیچیده بود
\item \textbf{نمادها}: درخت آزادی، کلاه فریژی، سه‌رنگ
\item \textbf{محدودیت}: اکثریت دهقانان بی‌سواد بودند (۵۰-۷۰٪)
\end{itemize}
\end{noktebox}

\begin{naghlbox}[تحلیل روژه شارتیه]
«انقلاب فرانسه را نباید "کاربرد" ایده‌های روشنگری دانست. روشنگری انقلاب را نساخت؛ انقلاب بود که روشنگری را بازتفسیر کرد و به خود اختصاص داد. رابطه علّی ساده میان کتاب‌ها و باریکادها وجود ندارد.»
\begin{flushright}
— روژه شارتیه، ریشه‌های فرهنگی انقلاب فرانسه (۱۹۹۰)
\end{flushright}
\end{naghlbox}

\subsection{الاهیات سیاسی: نقش دین در تحولات}

علاوه بر فلسفه سکولار، \textbf{الاهیات سیاسی} نیز نقش مهمی داشت — هم به عنوان هدف انتقاد و هم به عنوان منبع مشروعیت.

\begin{enghelabbox}[تعارض دین و انقلاب]
رابطه انقلاب فرانسه با دین، یکی از پیچیده‌ترین ابعاد آن است:
\begin{itemize}[nosep]
\item کلیسای کاتولیک بزرگ‌ترین مالک زمین بود
\item کلیسا از نظام موجود دفاع می‌کرد
\item انقلاب به ضدیت با کلیسا کشیده شد
\item «لائیسیته» فرانسوی از دل این تعارض زاده شد
\end{itemize}
\end{enghelabbox}

\begin{table}[H]
\centering
\caption{طیف مواضع الاهیاتی-سیاسی در آستانه انقلاب}
\begin{tabular}{|>{\bfseries}r|p{3.5cm}|p{4cm}|p{3.5cm}|}
\hline
\rowcolor{orroyallight}
\textbf{موضع} & \textbf{نظریه} & \textbf{حامیان} & \textbf{پیامد سیاسی} \\
\hline
حق الهی شاهان & قدرت شاه از خداست & بوسوئه، کلیسای رسمی & حفظ وضع موجود \\
\hline
\rowcolor{grisclair}
گالیکانیسم & استقلال کلیسای فرانسه از پاپ & بخشی از روحانیون & اصلاحات محدود \\
\hline
ژانسنیسم & اصلاح دینی، مخالف یسوعیان & پارلمان‌ها، برخی روحانیون & مقاومت در برابر استبداد \\
\hline
\rowcolor{grisclair}
دئیسم & خدای ساعت‌ساز، دین طبیعی & ولتر، روشنفکران & تساهل، ضد کلیسا \\
\hline
آتئیسم & انکار خدا & دیدرو، دولباخ & سکولاریسم رادیکال \\
\hline
\end{tabular}
\end{table}

\subsubsection{ژانسنیسم: پل میان دین و انقلاب}

\begin{olgoobox}[نقش پنهان ژانسنیسم]
\textbf{ژانسنیسم}، جنبشی اصلاح‌طلب درون کاتولیسیسم، نقشی غیرمنتظره در آماده‌سازی انقلاب داشت:

\begin{enumerate}[nosep]
\item مخالفت با اقتدارگرایی یسوعیان و پاپ
\item تأکید بر وجدان فردی در برابر سلسله‌مراتب
\item حمایت پارلمان‌ها از ژانسنیست‌ها → تقویت نهادهای مقاوم
\item ایجاد «فرهنگ مقاومت» در برابر قدرت مطلقه
\end{enumerate}

دیل ون کلی در کتاب \textbf{«ریشه‌های دینی انقلاب فرانسه»} (۱۹۹۶) این نقش را مستند کرده است.
\end{olgoobox}

\subsubsection{دین مدنی: جایگزین انقلابی}

انقلاب فرانسه تلاش کرد \textbf{دین مدنی} جدیدی بسازد:

\begin{table}[H]
\centering
\caption{عناصر دین مدنی انقلابی}
\begin{tabular}{|>{\bfseries}r|c|c|}
\hline
\rowcolor{rougelight}
\textbf{عنصر} & \textbf{مصداق انقلابی} & \textbf{معادل مسیحی} \\
\hline
خدا & موجود برتر / عقل & خدای مسیحی \\
\hline
\rowcolor{grisclair}
کتاب مقدس & اعلامیه حقوق بشر & انجیل \\
\hline
قدیسان & شهدای انقلاب (مارا) & قدیسان مسیحی \\
\hline
\rowcolor{grisclair}
اعیاد & جشن فدراسیون، جشن موجود برتر & کریسمس، عید پاک \\
\hline
معبد & پانتئون & کلیسا \\
\hline
\rowcolor{grisclair}
تقویم & تقویم انقلابی & تقویم گریگوری \\
\hline
\end{tabular}
\end{table}

\begin{naghlbox}[روبسپیر درباره دین مدنی]
«آتئیسم اشرافی است. ایده موجود برتر که بر طبیعت نظارت دارد و از بی‌گناهی ستم‌دیده دفاع می‌کند، کاملاً مردمی است.»
\begin{flushright}
— ماکسیمیلیان روبسپیر، خطابه در مجلس (۷ مه ۱۷۹۴)
\end{flushright}
\end{naghlbox}

\section{روش‌شناسی: تاریخ از بالا و از پایین}

این پژوهش از دو رویکرد مکمل استفاده می‌کند:

\begin{figure}[H]
\centering
\begin{tikzpicture}[
scale=0.95,
bleuapproach/.style={
	rectangle,
	draw=bleurepublique,
	line width=2pt,
	fill=bleulight,
	text=black,
	minimum width=6cm,
	minimum height=4cm,
	align=center,
	font=\small,
	rounded corners=5pt
},
rougeapproach/.style={
	rectangle,
	draw=rougerevolution,
	line width=2pt,
	fill=rougelight,
	text=black,
	minimum width=6cm,
	minimum height=4cm,
	align=center,
	font=\small,
	rounded corners=5pt
}
]
% دو رویکرد
\node[bleuapproach] (top) at (-4,0) {
	\begin{tabular}{c}
		\textbf{\large تاریخ از بالا}\\[8pt]
		نخبگان، دولتمردان\\
		تصمیمات، قوانین\\
		دیپلماسی، جنگ\\[5pt]
		منابع: اسناد رسمی\\
		مذاکرات مجلس\\
		خاطرات رجال
	\end{tabular}
};

\node[rougeapproach] (bottom) at (4,0) {
	\begin{tabular}{c}
		\textbf{\large تاریخ از پایین}\\[8pt]
		توده‌ها، زنان، فقرا\\
		زندگی روزمره\\
		شورش، اعتراض\\[5pt]
		منابع: پرونده‌های پلیس\\
		دفاتر شکایات\\
		فرهنگ عامه
	\end{tabular}
};

% پیکان دوطرفه
\draw[<->, >=Stealth, line width=2pt, vertnapoleon] 
(top) -- (bottom) 
node[midway, above, font=\small\bfseries, text=vertnapoleon] {تعامل};

% گره میانی
\node[
ellipse,
draw=vertnapoleon,
line width=1.5pt,
fill=vertlight,
minimum width=3cm,
minimum height=1.5cm,
font=\small\bfseries
] at (0,-3) {
	\begin{tabular}{c}
		تاریخ جامع\\
		(این پژوهش)
	\end{tabular}
};

\draw[->, >=Stealth, line width=1.5pt, gris] (top) -- (0,-2);
\draw[->, >=Stealth, line width=1.5pt, gris] (bottom) -- (0,-2);

\end{tikzpicture}
\caption{ترکیب دو رویکرد تاریخ‌نگاری}
\end{figure}

\subsection{تاریخ از بالا: چهره‌ها و تصمیمات}

\begin{table}[H]
\centering
\caption{بازیگران کلیدی هر دوره و تصمیمات سرنوشت‌ساز}
\small
\begin{tabular}{|c|>{\bfseries}r|p{4.5cm}|p{3.5cm}|}
\hline
\rowcolor{bleumid}
\textbf{دوره} & \textbf{بازیگر کلیدی} & \textbf{تصمیم سرنوشت‌ساز} & \textbf{پیامد} \\
\hline
۱۷۸۹ & لویی شانزدهم & تعلیق مجلس طبقات & شورش پاریس \\
\hline
\rowcolor{grisclair}
۱۷۹۲ & ژیروندن‌ها & اعلام جنگ به اتریش & رادیکال‌شدن \\
\hline
۱۷۹۳ & روبسپیر & دولت انقلابی موقت & ترور \\
\hline
\rowcolor{grisclair}
۱۷۹۹ & ناپلئون & کودتای برومر & پایان جمهوری اول \\
\hline
۱۸۱۵ & متحدین & بازگرداندن بوربون‌ها & بازگشت سلطنت \\
\hline
\rowcolor{grisclair}
۱۸۴۸ & لویی فیلیپ & سرکوب ضیافت‌ها & انقلاب فوریه \\
\hline
۱۸۷۰ & ناپلئون سوم & جنگ با پروس & سقوط امپراتوری \\
\hline
\rowcolor{grisclair}
۱۹۴۰ & پتن & متارکه با آلمان & ویشی \\
\hline
۱۹۵۸ & دوگل & قانون اساسی جدید & جمهوری پنجم \\
\hline
\end{tabular}
\end{table}

\subsection{تاریخ از پایین: توده‌ها در صحنه}

\begin{enghelabbox}[صداهای فراموش‌شده]
تاریخ سنتی بر نخبگان تمرکز داشت. اما انقلاب‌ها را توده‌ها ساختند:
\begin{itemize}[nosep]
\item \textbf{سانکولوت‌ها}: کارگران و صنعتگران پاریس که موتور رادیکالیسم بودند
\item \textbf{زنان}: راهپیمایی به ورسای (اکتبر ۱۷۸۹) را زنان رهبری کردند
\item \textbf{دهقانان}: «وحشت بزرگ» و سوزاندن قلعه‌های اشراف
\item \textbf{سربازان}: ارتش توده‌ای انقلاب، متفاوت از ارتش حرفه‌ای سلطنتی
\end{itemize}
\end{enghelabbox}

\begin{naghlbox}[جورج رود، مورخ «تاریخ از پایین»]
«ما نباید فقط بپرسیم رهبران چه کردند، بلکه باید بپرسیم: چه کسی در خیابان بود؟ از کجا آمده بود؟ چه می‌خواست؟ و چرا شورش کرد نه قبل‌تر و نه بعدتر؟»
\begin{flushright}
— جورج رود، جمعیت در انقلاب فرانسه (۱۹۵۹)
\end{flushright}
\end{naghlbox}

\section{تحلیل ذهنیت‌ها (Mentalités)}

مکتب آنال فرانسه مفهوم \textbf{ذهنیت} را وارد تاریخ‌نگاری کرد: ساختارهای ذهنی جمعی که به کندی تغییر می‌کنند.

\begin{figure}[H]
\centering
\begin{tikzpicture}[
scale=0.9,
orroyalment/.style={
	rectangle,
	draw=orroyaldark,
	line width=2pt,
	fill=orroyallight,
	text=black,
	minimum width=5cm,
	minimum height=3.5cm,
	align=center,
	font=\small,
	rounded corners=5pt
},
rougement/.style={
	rectangle,
	draw=rougerevolution,
	line width=2pt,
	fill=rougelight,
	text=black,
	minimum width=5cm,
	minimum height=3.5cm,
	align=center,
	font=\small,
	rounded corners=5pt
}
]
% ذهنیت سنتی
\node[orroyalment] (trad) at (-4,0) {
	\begin{tabular}{c}
		\textbf{\large ذهنیت سنتی}\\[8pt]
		سلسله‌مراتب طبیعی است\\
		شاه پدر ملت است\\
		تغییر خطرناک است\\
		دین تضمین نظم است\\[5pt]
		{\scriptsize (روستا، کلیسا، اشراف)}
	\end{tabular}
};

% ذهنیت انقلابی
\node[rougement] (rev) at (4,0) {
	\begin{tabular}{c}
		\textbf{\large ذهنیت انقلابی}\\[8pt]
		همه برابرند\\
		مردم حاکم‌اند\\
		تغییر ضروری است\\
		عقل راهنماست\\[5pt]
		{\scriptsize (شهر، روشنفکران، بورژوازی)}
	\end{tabular}
};

% تعارض
\draw[<->, >=Stealth, line width=3pt, black!60, 
decorate, decoration={zigzag, segment length=8pt, amplitude=3pt}] 
(trad) -- (rev)
node[midway, above, font=\bfseries] {تعارض};

% نتیجه
\node[
rectangle,
draw=violetempire,
line width=1.5pt,
fill=violetlight,
minimum width=10cm,
minimum height=1.2cm,
font=\small,
rounded corners=3pt
] at (0,-3.5) {
	\begin{tabular}{c}
		این تعارض ذهنیت‌ها تا امروز در فرانسه ادامه دارد\\
		(راست/چپ، لائیک/مذهبی، پاریس/استان‌ها)
	\end{tabular}
};

\end{tikzpicture}
\caption{تعارض ذهنیت‌های سنتی و انقلابی}
\end{figure}

\subsection{مؤلفه‌های ذهنیت انقلابی فرانسوی}

\begin{table}[H]
\centering
\caption{عناصر سازنده ذهنیت انقلابی فرانسوی}
\begin{tabular}{|>{\bfseries}r|p{5cm}|p{5cm}|}
\hline
\rowcolor{rougelight}
\textbf{مؤلفه} & \textbf{محتوا} & \textbf{تجلی} \\
\hline
تقدس انقلاب & انقلاب رویداد مقدس است & جشن‌های ملی، تقویم جدید \\
\hline
\rowcolor{grisclair}
ماهیت جهانی & انقلاب پیام جهانی دارد & صدور انقلاب، ناسیونالیسم \\
\hline
دوگانه‌انگاری & خیر مطلق در برابر شر مطلق & شکار «دشمنان مردم» \\
\hline
\rowcolor{grisclair}
اراده‌گرایی & اراده مردم همه‌چیز را ممکن می‌کند & مهندسی اجتماعی \\
\hline
پاریس‌محوری & پایتخت قلب انقلاب است & تحقیر استان‌ها \\
\hline
\rowcolor{grisclair}
خشونت مشروع & خشونت انقلابی عادلانه است & گیوتین، کشتارها \\
\hline
\end{tabular}
\end{table}

\section{ساختار کتاب}

\begin{figure}[H]
\centering
\begin{tikzpicture}[
scale=0.85,
transform shape,
bleuch/.style={
	rectangle,
	draw=bleurepublique,
	line width=1.5pt,
	fill=bleulight,
	text=black,
	minimum width=3.2cm,
	minimum height=1cm,
	align=center,
	font=\scriptsize,
	rounded corners=3pt
},
rougech/.style={
	rectangle,
	draw=rougerevolution,
	line width=1.5pt,
	fill=rougelight,
	text=black,
	minimum width=3.2cm,
	minimum height=1cm,
	align=center,
	font=\scriptsize,
	rounded corners=3pt
},
violetch/.style={
	rectangle,
	draw=violetempire,
	line width=1.5pt,
	fill=violetlight,
	text=black,
	minimum width=3.2cm,
	minimum height=1cm,
	align=center,
	font=\scriptsize,
	rounded corners=3pt
},
orroyalch/.style={
	rectangle,
	draw=orroyaldark,
	line width=1.5pt,
	fill=orroyallight,
	text=black,
	minimum width=3.2cm,
	minimum height=1cm,
	align=center,
	font=\scriptsize,
	rounded corners=3pt
},
vertch/.style={
	rectangle,
	draw=vertnapoleon,
	line width=1.5pt,
	fill=vertlight,
	text=black,
	minimum width=3.2cm,
	minimum height=1cm,
	align=center,
	font=\scriptsize,
	rounded corners=3pt
},
grisch/.style={
	rectangle,
	draw=gris,
	line width=1.5pt,
	fill=grisclair,
	text=black,
	minimum width=3.2cm,
	minimum height=1cm,
	align=center,
	font=\scriptsize,
	rounded corners=3pt
}
]
% ردیف بالا: مقدماتی
\node[bleuch] (ch0) at (0,4) {
	\begin{tabular}{c}\textbf{فصل ۰}\\خلاصه مدیریتی\end{tabular}
};
\node[bleuch] (ch1) at (3.5,4) {
	\begin{tabular}{c}\textbf{فصل ۱}\\چارچوب نظری\end{tabular}
};

% ردیف دوم: رژیم قدیم و انقلاب
\node[orroyalch] (ch2) at (0,2) {
	\begin{tabular}{c}\textbf{فصل ۲}\\رژیم قدیم\end{tabular}
};
\node[rougech] (ch3) at (3.5,2) {
	\begin{tabular}{c}\textbf{فصل ۳}\\انقلاب کبیر\end{tabular}
};
\node[violetch] (ch4) at (7,2) {
	\begin{tabular}{c}\textbf{فصل ۴}\\دوره ناپلئونی\end{tabular}
};

% ردیف سوم: قرن نوزدهم
\node[orroyalch] (ch5) at (0,0) {
	\begin{tabular}{c}\textbf{فصل ۵}\\قرن ناآرام\\۱۸۱۵-۱۸۷۰\end{tabular}
};
\node[bleuch] (ch6) at (3.5,0) {
	\begin{tabular}{c}\textbf{فصل ۶}\\جمهوری سوم\\۱۸۷۰-۱۹۴۰\end{tabular}
};

% ردیف چهارم: قرن بیستم
\node[grisch] (ch7) at (0,-2) {
	\begin{tabular}{c}\textbf{فصل ۷}\\ویشی و چهارم\\۱۹۴۰-۱۹۵۸\end{tabular}
};
\node[bleuch] (ch8) at (3.5,-2) {
	\begin{tabular}{c}\textbf{فصل ۸}\\جمهوری پنجم\\۱۹۵۸-۲۰۲۴\end{tabular}
};

% ردیف پنجم: تحلیل
\node[vertch] (ch9) at (0,-4) {
	\begin{tabular}{c}\textbf{فصل ۹}\\تحلیل تطبیقی\end{tabular}
};
\node[vertch] (ch10) at (3.5,-4) {
	\begin{tabular}{c}\textbf{فصل ۱۰}\\نتیجه و الگوها\end{tabular}
};

% پیکان‌ها
\draw[->, >=Stealth, line width=1pt, gris] (ch0) -- (ch1);
\draw[->, >=Stealth, line width=1pt, gris] (ch1) -- (ch2);
\draw[->, >=Stealth, line width=1pt, gris] (ch2) -- (ch3);
\draw[->, >=Stealth, line width=1pt, gris] (ch3) -- (ch4);
\draw[->, >=Stealth, line width=1pt, gris] (ch4) -- (ch5);
\draw[->, >=Stealth, line width=1pt, gris] (ch5) -- (ch6);
\draw[->, >=Stealth, line width=1pt, gris] (ch6) -- (ch7);
\draw[->, >=Stealth, line width=1pt, gris] (ch7) -- (ch8);
\draw[->, >=Stealth, line width=1pt, gris] (ch8) -- (ch9);
\draw[->, >=Stealth, line width=1pt, gris] (ch9) -- (ch10);

% راهنما
\node[right, font=\scriptsize] at (8,-1) {
	\begin{tabular}{r@{\hspace{3pt}}l}
		\textcolor{bleurepublique}{$\blacksquare$} & چارچوب/جمهوری \\
		\textcolor{rougerevolution}{$\blacksquare$} & انقلاب \\
		\textcolor{violetempire}{$\blacksquare$} & امپراتوری \\
		\textcolor{orroyaldark}{$\blacksquare$} & سلطنت \\
		\textcolor{vertnapoleon}{$\blacksquare$} & تحلیل \\
	\end{tabular}
};

\end{tikzpicture}
\caption{ساختار فصول کتاب}
\end{figure}

\section{پرسش‌های راهنما برای هر فصل}

\begin{table}[H]
\centering
\small
\caption{پرسش‌های کلیدی هر فصل}
\begin{tabular}{|c|p{10cm}|}
\hline
\rowcolor{bleumid}
\textbf{فصل} & \textbf{پرسش‌های کلیدی} \\
\hline
۲ & چرا رژیم قدیم نتوانست اصلاح شود؟ نقش بحران مالی چه بود؟ \\
\hline
\rowcolor{grisclair}
۳ & چرا انقلاب رادیکال شد؟ ترور اجتناب‌ناپذیر بود؟ \\
\hline
۴ & ناپلئون انقلاب را نجات داد یا خیانت کرد؟ \\
\hline
\rowcolor{grisclair}
۵ & چرا فرانسه نتوانست ثبات یابد؟ نقش طبقه کارگر جدید چه بود؟ \\
\hline
۶ & چگونه جمهوری سوم ۷۰ سال دوام آورد؟ نقش لائیسیته چه بود؟ \\
\hline
\rowcolor{grisclair}
۷ & ویشی انحراف بود یا ادامه یک سنت؟ \\
\hline
۸ & دوگل چگونه بحران جمهوری را حل کرد؟ آیا پنجم پایدار است؟ \\
\hline
\rowcolor{grisclair}
۹ & چرا مسیر فرانسه با انگلستان و آلمان متفاوت شد؟ \\
\hline
۱۰ & چه الگوهایی قابل تعمیم است؟ درس‌ها برای امروز چیست؟ \\
\hline
\end{tabular}
\end{table}

\section{منابع فصل اول}

\subsection{منابع اولیه}

\begin{enumerate}[nosep]
\item روسو، ژان-ژاک. قرارداد اجتماعی. ۱۷۶۲.
\item مونتسکیو، شارل دو. روح‌القوانین. ۱۷۴۸.
\item ولتر. رساله‌های فلسفی. ۱۷۳۴.
\item دیدرو و دالامبر. دایره‌المعارف. ۱۷۵۱-۱۷۷۲.
\end{enumerate}

\subsection{منابع ثانویه}

\begin{enumerate}[nosep]
\item Chartier, Roger. \textit{The Cultural Origins of the French Revolution}. Duke UP, 1991.
\item Darnton, Robert. \textit{The Forbidden Best-Sellers of Pre-Revolutionary France}. Norton, 1995.
\item Furet, François. \textit{Interpreting the French Revolution}. Cambridge UP, 1981.
\item Hunt, Lynn. \textit{Politics, Culture, and Class in the French Revolution}. UC Press, 1984.
\item Lefebvre, Georges. \textit{The Great Fear of 1789}. Pantheon, 1973.
\item Rudé, George. \textit{The Crowd in the French Revolution}. Oxford UP, 1959.
\item Skocpol, Theda. \textit{States and Social Revolutions}. Cambridge UP, 1979.
\item Tackett, Timothy. \textit{The Coming of the Terror}. Harvard UP, 2015.
\item Van Kley, Dale. \textit{The Religious Origins of the French Revolution}. Yale UP, 1996.
\item Vovelle, Michel. \textit{The Revolution Against the Church}. Ohio State UP, 1991.
\end{enumerate}

\begin{olgoobox}[جمع‌بندی فصل اول]
در این فصل چارچوب نظری پژوهش را ترسیم کردیم:

\begin{itemize}[nosep]
\item \textbf{پرسش محوری}: چرا فرانسه آزمایشگاه انقلاب‌ها شد؟
\item \textbf{رویکرد}: تحلیل چندوجهی (تاریخی، جامعه‌شناختی، روان‌شناختی، اقتصادی، فلسفی)
\item \textbf{فلسفه سیاسی}: سه سنت لیبرال، جمهوری‌خواه، و رادیکال
\item \textbf{الاهیات سیاسی}: از حق الهی شاهان تا دین مدنی انقلابی
\item \textbf{ذهنیت‌ها}: تعارض ذهنیت سنتی و انقلابی
\end{itemize}

\textbf{در فصل بعد}: به رژیم قدیم می‌پردازیم و ریشه‌های ساختاری انقلاب را بررسی می‌کنیم.
\end{olgoobox}

% ══════════════════════════════════════════════════════════════════════════════
%                    پایان فصل ۱
% ══════════════════════════════════════════════════════════════════════════════

\end{document}