% ═══════════════════════════════════════════════════════════════════════════════
%                    تاریخ تحولات فرانسه - فصل [X]
% ═══════════════════════════════════════════════════════════════════════════════

\documentclass[12pt,a4paper]{book}

% ─────────────────────────── پکیج‌ها ───────────────────────────
\usepackage{amsmath,amssymb}
\usepackage{geometry}
\geometry{top=2.5cm, bottom=2.5cm, left=2cm, right=2.5cm, headheight=15pt}
\usepackage{graphicx}
\usepackage{array,booktabs,longtable,multirow,colortbl}
\usepackage{xcolor}
\usepackage{tikz}
\usetikzlibrary{shapes.geometric, arrows.meta, positioning, calc, backgrounds, 
	fit, decorations.pathmorphing, shadows, patterns}
\usepackage{pgfplots}
\pgfplotsset{compat=1.18}
\usepackage{tcolorbox}
\tcbuselibrary{skins,breakable}
\usepackage{enumitem}
\usepackage{fancyhdr}
\usepackage{setspace}
\usepackage{titlesec}
\usepackage{float}
\usepackage{pdfpages}
\usepackage{pdflscape}  % برای صفحات landscape
\usepackage{hyperref}

% ─────────────────────────── رنگ‌ها ───────────────────────────
\definecolor{bleurepublique}{RGB}{0, 35, 149}
\definecolor{rougerevolution}{RGB}{237, 41, 57}
\definecolor{orroyal}{RGB}{255, 215, 0}
\definecolor{vertnapoleon}{RGB}{0, 100, 0}
\definecolor{violetempire}{RGB}{128, 0, 128}
\definecolor{fondclair}{RGB}{255, 253, 240}
\definecolor{gris}{RGB}{128, 128, 128}
\definecolor{grisclair}{RGB}{245, 245, 245}
\definecolor{noirsombre}{RGB}{30, 30, 30}

% رنگ‌های کمکی
\definecolor{bleulight}{RGB}{230, 235, 250}
\definecolor{rougelight}{RGB}{253, 235, 237}
\definecolor{vertlight}{RGB}{235, 250, 235}
\definecolor{violetlight}{RGB}{245, 235, 250}
\definecolor{orroyallight}{RGB}{255, 250, 230}
\definecolor{grislight}{RGB}{248, 248, 248}
\definecolor{bleumid}{RGB}{180, 195, 230}
\definecolor{rougemid}{RGB}{245, 180, 185}
\definecolor{vertmid}{RGB}{180, 220, 180}
\definecolor{violetmid}{RGB}{210, 180, 220}
\definecolor{orroyalmid}{RGB}{255, 240, 180}
\definecolor{orroyaldark}{RGB}{200, 170, 0}

% ─────────────────────────── فونت فارسی ───────────────────────────
\usepackage{fontspec}
\setmainfont{Vazirmatn}
\usepackage{xepersian}
\settextfont{Vazirmatn}
\setdigitfont{Vazirmatn}

% ─────────────────────────── هایپرلینک ───────────────────────────
\hypersetup{
	colorlinks=true,
	linkcolor=bleurepublique,
	urlcolor=bleurepublique,
	citecolor=vertnapoleon
}

% ─────────────────────────── کادرها ───────────────────────────
\newtcolorbox{kholasebox}[1][]{enhanced,breakable,colback=bleulight,
	colframe=bleurepublique,coltitle=white,fonttitle=\bfseries\large,
	title={#1},boxrule=2pt,arc=4pt,left=10pt,right=10pt,top=10pt,bottom=10pt,
	drop shadow={opacity=0.3}}

\newtcolorbox{naghlbox}[1][]{enhanced,breakable,colback=orroyallight,
	colframe=orroyaldark,coltitle=black,fonttitle=\bfseries,title={#1},
	boxrule=1.5pt,arc=3pt,borderline west={4pt}{0pt}{orroyal},
	left=15pt,right=10pt,top=8pt,bottom=8pt}

\newtcolorbox{olgoobox}[1][]{enhanced,breakable,colback=vertlight,
	colframe=vertnapoleon,coltitle=white,fonttitle=\bfseries,title={#1},
	boxrule=1.5pt,arc=4pt,left=10pt,right=10pt,top=8pt,bottom=8pt,
	before upper={\parindent15pt}}

\newtcolorbox{enghelabbox}[1][]{enhanced,breakable,colback=rougelight,
	colframe=rougerevolution,coltitle=white,fonttitle=\bfseries,title={#1},
	boxrule=2pt,arc=4pt,left=10pt,right=10pt,top=8pt,bottom=8pt}

\newtcolorbox{empirebox}[1][]{enhanced,breakable,colback=violetlight,
	colframe=violetempire,coltitle=white,fonttitle=\bfseries,title={#1},
	boxrule=1.5pt,arc=4pt,left=10pt,right=10pt,top=8pt,bottom=8pt}

\newtcolorbox{noktebox}[1][]{enhanced,colback=grisclair,colframe=gris,
	fonttitle=\bfseries,title={#1},boxrule=1pt,arc=3pt,left=8pt,right=8pt}

% ─────────────────────────── صفحه‌آرایی ───────────────────────────
\pagestyle{fancy}
\fancyhf{}
\fancyhead[RO]{\leftmark}
\fancyhead[LE]{\rightmark}
\fancyfoot[C]{\thepage}
\renewcommand{\headrulewidth}{1pt}
\renewcommand{\footrulewidth}{0.5pt}
\setstretch{1.5}

\titleformat{\chapter}[display]
{\normalfont\huge\bfseries\color{bleurepublique}}
{\chaptertitlename\ \thechapter}{20pt}{\Huge}
\titleformat{\section}
{\normalfont\Large\bfseries\color{bleurepublique}}{\thesection}{1em}{}
\titleformat{\subsection}
{\normalfont\large\bfseries\color{bleurepublique}}{\thesubsection}{1em}{}

% ═══════════════════════════════════════════════════════════════════════════════
\begin{document}
% ██████████████████████████████████████████████████████████████████████████████
%
%                    فصل ۳: انقلاب کبیر (۱۷۸۹-۱۷۹۹)
%
% ██████████████████████████████████████████████████████████████████████████████

\chapter{انقلاب کبیر (۱۷۸۹-۱۷۹۹)}

\begin{kholasebox}[خلاصه فصل سوم]
	این فصل به یک دهه پرتلاطم می‌پردازد که چهره فرانسه و جهان را تغییر داد. از گشایش مجلس طبقات (مه ۱۷۸۹) تا کودتای ناپلئون (نوامبر ۱۷۹۹)، فرانسه از سلطنت مطلقه به سلطنت مشروطه، سپس به جمهوری، ترور، و سرانجام به دیکتاتوری گذر کرد. ما می‌پرسیم: چرا انقلاب رادیکال شد؟ آیا ترور اجتناب‌ناپذیر بود؟ و چرا انقلاب «فرزندان خود را خورد»؟
\end{kholasebox}

\section{دوره‌بندی انقلاب}

\begin{figure}[H]
	\centering
	\begin{tikzpicture}[
		scale=0.85,
		transform shape
		]
		% خط زمان اصلی
		\draw[line width=3pt, gris] (0,0) -- (16,0);
		
		% دوره‌ها با رنگ
		\fill[bleulight] (0,0.4) rectangle (4,0.8);
		\node[above, font=\tiny\bfseries, text=bleurepublique] at (2,0.9) {لیبرال};
		
		\fill[orroyallight] (4,0.4) rectangle (7,0.8);
		\node[above, font=\tiny\bfseries, text=orroyaldark] at (5.5,0.9) {مشروطه};
		
		\fill[rougelight] (7,0.4) rectangle (10,0.8);
		\node[above, font=\tiny\bfseries, text=rougerevolution] at (8.5,0.9) {جمهوری رادیکال};
		
		\fill[rougemid] (10,0.4) rectangle (12,0.8);
		\node[above, font=\tiny\bfseries, text=rougerevolution] at (11,0.9) {ترور};
		
		\fill[violetlight] (12,0.4) rectangle (16,0.8);
		\node[above, font=\tiny\bfseries, text=violetempire] at (14,0.9) {ترمیدور و دایرکتوار};
		
		% سال‌ها و رویدادها
		\foreach \x/\label in {
			0/{مه ۸۹},
			4/{اکتبر ۸۹},
			7/{اوت ۹۲},
			10/{ژوئن ۹۳},
			12/{ژوئیه ۹۴},
			16/{نوامبر ۹۹}
		} {
			\draw[black, line width=1.5pt] (\x,-0.3) -- (\x,0.3);
			\node[below, font=\tiny] at (\x,-0.4) {\label};
		}
		
		% رویدادهای کلیدی
		\node[rectangle, draw=bleurepublique, fill=bleulight, 
		minimum width=1.8cm, minimum height=0.6cm, font=\tiny, align=center]
		at (2,-1.5) {سوگند\\ژودوپوم};
		\draw[->, >=Stealth, bleurepublique] (2,-1.1) -- (2,-0.4);
		
		\node[rectangle, draw=rougerevolution, fill=rougelight,
		minimum width=1.8cm, minimum height=0.6cm, font=\tiny, align=center]
		at (2.5,2) {سقوط\\باستیل};
		\draw[->, >=Stealth, rougerevolution] (2.5,1.6) -- (2.5,0.9);
		
		\node[rectangle, draw=orroyaldark, fill=orroyallight,
		minimum width=1.8cm, minimum height=0.6cm, font=\tiny, align=center]
		at (5.5,-1.5) {قانون اساسی\\۱۷۹۱};
		\draw[->, >=Stealth, orroyaldark] (5.5,-1.1) -- (5.5,-0.4);
		
		\node[rectangle, draw=rougerevolution, fill=rougelight,
		minimum width=1.8cm, minimum height=0.6cm, font=\tiny, align=center]
		at (8.5,2) {اعدام\\لویی ۱۶};
		\draw[->, >=Stealth, rougerevolution] (8.5,1.6) -- (8.5,0.9);
		
		\node[rectangle, draw=rougerevolution, fill=rougemid,
		minimum width=1.8cm, minimum height=0.6cm, font=\tiny, align=center]
		at (11,-1.5) {قانون\\مظنونین};
		\draw[->, >=Stealth, rougerevolution] (11,-1.1) -- (11,-0.4);
		
		\node[rectangle, draw=violetempire, fill=violetlight,
		minimum width=1.8cm, minimum height=0.6cm, font=\tiny, align=center]
		at (14,2) {کودتای\\برومر};
		\draw[->, >=Stealth, violetempire] (14,1.6) -- (14,0.9);
		
	\end{tikzpicture}
	\caption{دوره‌بندی انقلاب فرانسه (۱۷۸۹-۱۷۹۹)}
\end{figure}

\section{مجلس طبقات و تولد انقلاب (مه-ژوئیه ۱۷۸۹)}

\subsection{گشایش مجلس طبقات: ۵ مه ۱۷۸۹}

\begin{table}[H]
	\centering
	\caption{ترکیب مجلس طبقات عمومی ۱۷۸۹}
	\begin{tabular}{|>{\bfseries}r|c|c|c|c|}
		\hline
		\rowcolor{bleumid}
		\textbf{طبقه} & \textbf{نمایندگان} & \textbf{جمعیت نمایندگی} & \textbf{رأی سنتی} & \textbf{خواست} \\
		\hline
		روحانیون & ۲۹۱ & ۱۳۰,۰۰۰ & ۱ رأی & حفظ امتیازات \\
		\hline
		\rowcolor{grisclair}
		اشراف & ۲۷۰ & ۴۰۰,۰۰۰ & ۱ رأی & حفظ امتیازات \\
		\hline
		طبقه سوم & ۵۷۸ & ۲۷,۰۰۰,۰۰۰ & ۱ رأی & رأی سرانه \\
		\hline
	\end{tabular}
\end{table}

\begin{enghelabbox}[بن‌بست اولیه]
	از همان روز اول، بحران شروع شد:
	\begin{itemize}[nosep]
		\item \textbf{شاه}: درباره روش رأی‌گیری سکوت کرد
		\item \textbf{طبقات ممتاز}: خواستار رأی به تفکیک طبقه
		\item \textbf{طبقه سوم}: خواستار رأی سرانه
		\item \textbf{نتیجه}: بن‌بست شش‌هفته‌ای
	\end{itemize}
\end{enghelabbox}

\subsection{تشکیل مجلس ملی: ۱۷ ژوئن ۱۷۸۹}

\begin{naghlbox}[اعلام مجلس ملی]
	«مجلس [طبقه سوم]... اعلام می‌کند که نام تنها مناسب آن، مجلس ملی است... تفسیر و بیان اراده عمومی تنها به این مجلس تعلق دارد.»
	\begin{flushright}
		— قطعنامه مجلس ملی، ۱۷ ژوئن ۱۷۸۹
	\end{flushright}
\end{naghlbox}

این اقدام یک‌جانبه، \textbf{اولین عمل انقلابی} بود: طبقه سوم خود را نماینده کل ملت اعلام کرد.

\subsection{سوگند ژودوپوم: ۲۰ ژوئن ۱۷۸۹}

\begin{figure}[H]
	\centering
	\begin{tikzpicture}[
		scale=0.9,
		transform shape
		]
		% کادر مرکزی
		\node[rectangle, draw=bleurepublique, line width=3pt, fill=bleulight,
		minimum width=12cm, minimum height=4cm, rounded corners=10pt] 
		at (0,0) {};
		
		% متن سوگند
		\node[text width=10cm, align=center, font=\small] at (0,0.5) {
			\textbf{«ما سوگند یاد می‌کنیم که هرگز از هم جدا نشویم}\\[5pt]
			\textbf{و هر جا که شرایط اقتضا کند گرد هم آییم،}\\[5pt]
			\textbf{تا زمانی که قانون اساسی مملکت تدوین و}\\[5pt]
			\textbf{بر پایه‌های مستحکم استوار گردد.»}
		};
		
		% تاریخ
		\node[font=\footnotesize, text=gris] at (0,-1.3) {
			سالن ژودوپوم (بازی توپ)، ورسای — ۲۰ ژوئن ۱۷۸۹
		};
		
		% اهمیت
		\node[rectangle, draw=rougerevolution, fill=rougelight,
		minimum width=10cm, minimum height=1cm, rounded corners=3pt]
		at (0,-3) {
			\textbf{اهمیت}: نمایندگان متعهد شدند تا تدوین قانون اساسی، منحل نشوند
		};
		
	\end{tikzpicture}
	\caption{سوگند ژودوپوم — نقطه عطف انقلاب}
\end{figure}

\subsection{تسلیم شاه: ۲۷ ژوئن ۱۷۸۹}

شاه ابتدا مقاومت کرد، اما سرانجام تسلیم شد و به روحانیون و اشراف دستور داد به مجلس ملی بپیوندند. نکر گفته بود: «تنها نیروی نظامی می‌تواند کار را تمام کند» — اما شاه توان یا اراده استفاده از آن را نداشت.

\section{سقوط باستیل: ۱۴ ژوئیه ۱۷۸۹}

\subsection{زمینه‌ها}

\begin{table}[H]
	\centering
	\caption{عوامل شورش ۱۴ ژوئیه}
	\begin{tabular}{|>{\bfseries}r|p{8cm}|}
		\hline
		\rowcolor{rougelight}
		\textbf{عامل} & \textbf{توضیح} \\
		\hline
		برکناری نکر & ۱۱ ژوئیه — نکر محبوب برکنار شد \\
		\hline
		\rowcolor{grisclair}
		تجمع نظامی & ۳۰,۰۰۰ سرباز در اطراف پاریس \\
		\hline
		قیمت نان & در بالاترین سطح قرن \\
		\hline
		\rowcolor{grisclair}
		شایعات & «ارتش می‌آید تا پاریس را قتل‌عام کند» \\
		\hline
		سازمان‌دهی & تشکیل «کمیته دائمی» در شهرداری \\
		\hline
	\end{tabular}
\end{table}

\subsection{روز سرنوشت‌ساز}

\begin{figure}[H]
	\centering
	\begin{tikzpicture}[
		scale=0.9,
		event/.style={
			rectangle, draw=rougerevolution, line width=1.5pt, fill=rougelight,
			text=black, minimum width=3.5cm, minimum height=1.5cm,
			align=center, font=\scriptsize, rounded corners=3pt
		}
		]
		% خط زمان روز
		\draw[line width=2pt, gris] (0,0) -- (14,0);
		
		% ساعات
		\foreach \x/\hour in {0/صبح, 3.5/۱۰, 7/۱۳:۳۰, 10.5/۱۷, 14/شب} {
			\draw[black, line width=1pt] (\x,-0.2) -- (\x,0.2);
			\node[below, font=\tiny] at (\x,-0.3) {\hour};
		}
		
		% رویدادها
		\node[event] at (1.5,2) {
			\begin{tabular}{c}
				غارت\\
				زرادخانه اَنوالید\\
				(۳۰,۰۰۰ تفنگ)
			\end{tabular}
		};
		\draw[->, >=Stealth, rougerevolution] (1.5,1.2) -- (1.5,0.3);
		
		\node[event] at (5,2) {
			\begin{tabular}{c}
				محاصره\\
				باستیل\\
				مذاکره ناکام
			\end{tabular}
		};
		\draw[->, >=Stealth, rougerevolution] (5,1.2) -- (5,0.3);
		
		\node[event] at (8.5,2) {
			\begin{tabular}{c}
				حمله نهایی\\
				تسلیم فرمانده\\
				(دو لونه)
			\end{tabular}
		};
		\draw[->, >=Stealth, rougerevolution] (8.5,1.2) -- (8.5,0.3);
		
		\node[event] at (12,2) {
			\begin{tabular}{c}
				قتل فرمانده\\
				سر بر نیزه\\
				جشن خیابانی
			\end{tabular}
		};
		\draw[->, >=Stealth, rougerevolution] (12,1.2) -- (12,0.3);
		
		% آمار
		\node[rectangle, draw=gris, fill=grisclair, 
		minimum width=8cm, minimum height=1cm, font=\scriptsize]
		at (7,-2) {
			کشته‌ها: ۹۸ مهاجم، ۱ مدافع | زندانیان آزادشده: فقط ۷ نفر!
		};
		
	\end{tikzpicture}
	\caption{وقایع ۱۴ ژوئیه ۱۷۸۹}
\end{figure}

\begin{naghlbox}[واکنش لویی شانزدهم]
	وقتی دوک دو لیانکور خبر سقوط باستیل را به شاه داد:
	
	\textbf{شاه}: «آیا این شورش است؟»
	
	\textbf{دوک}: «نه، اعلیحضرت، این انقلاب است.»
	\begin{flushright}
		— روایت سنتی، ۱۵ ژوئیه ۱۷۸۹
	\end{flushright}
\end{naghlbox}

\begin{olgoobox}[اهمیت نمادین باستیل]
	باستیل فقط یک زندان قدیمی با ۷ زندانی بود. اما نمادی بود از:
	\begin{itemize}[nosep]
		\item \textbf{استبداد}: زندان بدون محاکمه (lettres de cachet)
		\item \textbf{خودسری}: قدرت مطلقه شاه
		\item \textbf{ظلم}: قرن‌ها سرکوب
	\end{itemize}
	
	سقوط باستیل نشان داد که \textbf{مردم می‌توانند نمادهای قدرت را سرنگون کنند}.
\end{olgoobox}

\section{وحشت بزرگ و شب ۴ اوت ۱۷۸۹}

\subsection{وحشت بزرگ (Grande Peur)}

\begin{figure}[H]
	\centering
	\begin{tikzpicture}[
		scale=0.9,
		phase/.style={
			rectangle, draw=#1, line width=1.5pt, fill=#1,
			text=black, minimum width=4.5cm, minimum height=2cm,
			align=center, font=\small, rounded corners=5pt
		}
		]
		% مراحل
		\node[phase=rougelight] (rumor) at (0,0) {
			\begin{tabular}{c}
				\textbf{۱. شایعه}\\[3pt]
				«راهزنان می‌آیند»\\
				«اشراف مزدور استخدام کرده‌اند»
			\end{tabular}
		};
		
		\node[phase=rougemid] (panic) at (5.5,0) {
			\begin{tabular}{c}
				\textbf{۲. وحشت}\\[3pt]
				فرار، مسلح شدن\\
				هجوم به قلعه‌ها
			\end{tabular}
		};
		
		\node[phase=vertlight] (action) at (11,0) {
			\begin{tabular}{c}
				\textbf{۳. اقدام}\\[3pt]
				سوزاندن اسناد فئودالی\\
				تصرف زمین‌ها
			\end{tabular}
		};
		
		% پیکان‌ها
		\draw[->, >=Stealth, line width=2pt, gris] (rumor) -- (panic);
		\draw[->, >=Stealth, line width=2pt, gris] (panic) -- (action);
		
		% توضیح
		\node[rectangle, draw=bleurepublique, fill=bleulight,
		minimum width=12cm, minimum height=1.2cm, font=\small, rounded corners=3pt]
		at (5.5,-2.5) {
			\textbf{نتیجه}: انقلاب از پاریس به روستاها گسترش یافت
		};
		
	\end{tikzpicture}
	\caption{مراحل «وحشت بزرگ» (اواخر ژوئیه ۱۷۸۹)}
\end{figure}

\subsection{شب ۴ اوت: پایان فئودالیسم}

\begin{enghelabbox}[شب معجزه]
	در شب ۴ اوت ۱۷۸۹، نمایندگان اشراف و روحانیون در یک جلسه پرهیجان، داوطلبانه از امتیازات خود گذشتند:
	\begin{itemize}[nosep]
		\item لغو سرواژ (رعیتی)
		\item لغو حقوق فئودالی
		\item لغو معافیت‌های مالیاتی
		\item لغو انحصار شکار
		\item لغو دادگاه‌های اشرافی
		\item لغو فروش مناصب
	\end{itemize}
	
	\textbf{واقعیت}: برخی حقوق «لغو» شدند، برخی باید «بازخرید» می‌شدند — تمایزی که بعداً مشکل‌ساز شد.
\end{enghelabbox}

\begin{naghlbox}[اعلامیه مجلس]
	«مجلس ملی رژیم فئودالی را کاملاً نابود می‌کند.»
	\begin{flushright}
		— ماده اول مصوبه ۴ اوت ۱۷۸۹
	\end{flushright}
\end{naghlbox}

\section{اعلامیه حقوق بشر و شهروند: ۲۶ اوت ۱۷۸۹}

\begin{figure}[H]
	\centering
	\begin{tikzpicture}[
		scale=0.95
		]
		% کادر اصلی
		\node[rectangle, draw=bleurepublique, line width=3pt, fill=bleulight,
		minimum width=14cm, minimum height=10cm, rounded corners=10pt]
		at (0,0) {};
		
		% عنوان
		\node[font=\large\bfseries, text=bleurepublique] at (0,4) {
			اعلامیه حقوق بشر و شهروند
		};
		\node[font=\small, text=gris] at (0,3.4) {۲۶ اوت ۱۷۸۹};
		
		% مواد کلیدی
		\node[text width=12cm, align=right, font=\small] at (0,1) {
			\textbf{ماده ۱}: انسان‌ها آزاد و برابر در حقوق زاده می‌شوند و می‌مانند.\\[5pt]
			\textbf{ماده ۲}: حقوق طبیعی: آزادی، مالکیت، امنیت، مقاومت در برابر ظلم.\\[5pt]
			\textbf{ماده ۳}: منبع هر حاکمیتی ذاتاً در ملت است.\\[5pt]
			\textbf{ماده ۶}: قانون بیان اراده عمومی است. همه شهروندان حق دارند در وضع آن شرکت کنند.\\[5pt]
			\textbf{ماده ۱۱}: آزادی بیان اندیشه و عقاید از گران‌بهاترین حقوق انسان است.\\[5pt]
			\textbf{ماده ۱۷}: مالکیت حقی مقدس و تخطی‌ناپذیر است.
		};
		
		% منابع فکری
		\node[rectangle, draw=vertnapoleon, fill=vertlight,
		minimum width=5cm, minimum height=1.5cm, font=\scriptsize, 
		rounded corners=3pt]
		at (-4,-3) {
			\begin{tabular}{c}
				\textbf{منابع}\\
				لاک، روسو، مونتسکیو\\
				اعلامیه استقلال آمریکا
			\end{tabular}
		};
		
		% تناقضات
		\node[rectangle, draw=rougerevolution, fill=rougelight,
		minimum width=5cm, minimum height=1.5cm, font=\scriptsize,
		rounded corners=3pt]
		at (4,-3) {
			\begin{tabular}{c}
				\textbf{سکوت‌ها}\\
				زنان، بردگان\\
				فقرا (رأی سنسیتر)
			\end{tabular}
		};
		
	\end{tikzpicture}
	\caption{اعلامیه حقوق بشر و شهروند ۱۷۸۹}
\end{figure}

\begin{olgoobox}[اهمیت تاریخی اعلامیه]
	اعلامیه ۱۷۸۹ یکی از مهم‌ترین اسناد تاریخ بشر است:
	\begin{enumerate}[nosep]
		\item \textbf{جهان‌شمولی}: نه فقط حقوق فرانسوی‌ها، بلکه «بشر»
		\item \textbf{طبیعی بودن حقوق}: حقوق پیش از دولت وجود دارند
		\item \textbf{ملت به‌جای شاه}: منبع حاکمیت تغییر کرد
		\item \textbf{الگو برای آینده}: اعلامیه جهانی حقوق بشر ۱۹۴۸ از آن الهام گرفت
	\end{enumerate}
\end{olgoobox}

\section{راهپیمایی به ورسای: ۵-۶ اکتبر ۱۷۸۹}

\begin{table}[H]
	\centering
	\caption{راهپیمایی زنان به ورسای}
	\begin{tabular}{|>{\bfseries}r|p{9cm}|}
		\hline
		\rowcolor{rougelight}
		\textbf{مرحله} & \textbf{رویداد} \\
		\hline
		علت & کمبود نان در پاریس، خشم از ضیافت افسران \\
		\hline
		\rowcolor{grisclair}
		صبح ۵ اکتبر & ۷۰۰۰ زن از پاریس به سوی ورسای حرکت کردند \\
		\hline
		عصر ۵ اکتبر & رسیدن به ورسای، مذاکره با شاه، تأیید اعلامیه حقوق \\
		\hline
		\rowcolor{grisclair}
		شب ۵-۶ اکتبر & گارد ملی لافایت رسید، جمعیت شب را ماند \\
		\hline
		صبح ۶ اکتبر & هجوم به کاخ، کشته شدن نگهبانان، تهدید ملکه \\
		\hline
		\rowcolor{grisclair}
		نتیجه & خانواده سلطنتی مجبور به بازگشت به پاریس شد \\
		\hline
	\end{tabular}
\end{table}

\begin{naghlbox}[فریاد جمعیت]
	«نانوا، زن نانوا، و شاگرد نانوا!» — جمعیت شاه، ملکه و ولیعهد را به پاریس می‌برد.
	\begin{flushright}
		— شعار زنان، ۶ اکتبر ۱۷۸۹
	\end{flushright}
\end{naghlbox}

\begin{olgoobox}[نقش زنان در انقلاب]
	راهپیمایی اکتبر نشان داد که زنان بازیگران فعال انقلاب بودند:
	\begin{itemize}[nosep]
		\item زنان بازار و محله‌ها سازمان‌دهندگان اصلی بودند
		\item خواست اصلی آنها «نان» بود — نه حقوق انتزاعی
		\item آنها شاه را «اسیر» مردم کردند
		\item اما بعداً از حقوق سیاسی محروم ماندند
	\end{itemize}
\end{olgoobox}

\section{دوره سلطنت مشروطه (۱۷۸۹-۱۷۹۲)}

\subsection{اصلاحات مجلس مؤسسان}

\begin{table}[H]
	\centering
	\caption{اصلاحات اصلی مجلس مؤسسان (۱۷۸۹-۱۷۹۱)}
	\small
	\begin{tabular}{|>{\bfseries}r|p{4cm}|p{4cm}|p{3cm}|}
		\hline
		\rowcolor{bleumid}
		\textbf{حوزه} & \textbf{قدیم} & \textbf{جدید} & \textbf{ارزیابی} \\
		\hline
		اداری & استان‌های نامنظم & ۸۳ دپارتمان مساوی & موفق، پایدار \\
		\hline
		\rowcolor{grisclair}
		قضایی & دادگاه‌های موروثی & قضات انتخابی، هیئت منصفه & تا حدی موفق \\
		\hline
		مالی & مالیات‌های متعدد و ناعادلانه & مالیات یکسان بر دارایی & ناکام در اجرا \\
		\hline
		\rowcolor{grisclair}
		اقتصادی & اصناف و انحصارات & آزادی کسب‌وکار (قانون لوشاپلیه) & دوپهلو \\
		\hline
		کلیسایی & کلیسای مستقل & «نظام‌نامه مدنی روحانیون» & فاجعه‌بار \\
		\hline
	\end{tabular}
\end{table}

\subsection{نظام‌نامه مدنی روحانیون: اشتباه بزرگ}

\begin{enghelabbox}[ریشه جنگ داخلی]
	نظام‌نامه مدنی روحانیون (ژوئیه ۱۷۹۰) کلیسا را دولتی کرد:
	\begin{itemize}[nosep]
		\item انتخاب اسقف‌ها توسط مردم (نه پاپ)
		\item حقوق روحانیون از دولت
		\item سوگند وفاداری به قانون اساسی
	\end{itemize}
	
	\textbf{نتیجه}:
	\begin{itemize}[nosep]
		\item نیمی از روحانیون سوگند خوردند («سوگندخورده»)
		\item نیمی امتناع کردند («سرپیچ»)
		\item پاپ نظام‌نامه را محکوم کرد
		\item فرانسه به دو اردوگاه تقسیم شد
	\end{itemize}
\end{enghelabbox}

\subsection{قانون اساسی ۱۷۹۱}

\begin{figure}[H]
	\centering
	\begin{tikzpicture}[
		scale=0.9,
		power/.style={
			rectangle, draw=#1, line width=2pt, fill=#1,
			text=black, minimum width=4cm, minimum height=2.5cm,
			align=center, font=\small, rounded corners=5pt
		}
		]
		% عنوان
		\node[font=\large\bfseries, text=bleurepublique] at (0,5) {
			قانون اساسی ۱۷۹۱: تفکیک قوا
		};
		
		% سه قوه
		\node[power=bleulight] (legis) at (-5,2) {
			\begin{tabular}{c}
				\textbf{قوه مقننه}\\[5pt]
				مجلس قانون‌گذار\\
				۷۴۵ نماینده\\
				انتخاب غیرمستقیم
			\end{tabular}
		};
		
		\node[power=orroyallight] (exec) at (0,2) {
			\begin{tabular}{c}
				\textbf{قوه مجریه}\\[5pt]
				شاه\\
				وزرای منصوب\\
				حق وتوی تعلیقی
			\end{tabular}
		};
		
		\node[power=violetlight] (judi) at (5,2) {
			\begin{tabular}{c}
				\textbf{قوه قضائیه}\\[5pt]
				قضات انتخابی\\
				هیئت منصفه\\
				استقلال از شاه
			\end{tabular}
		};
		
		% رأی‌دهندگان
		\node[rectangle, draw=vertnapoleon, fill=vertlight,
		minimum width=10cm, minimum height=1.5cm, font=\small, rounded corners=3pt]
		at (0,-1.5) {
			\begin{tabular}{c}
				\textbf{شهروندان «فعال»}: مردان بالای ۲۵ سال که مالیات معادل ۳ روز کار می‌دهند\\
				(۴.۳ میلیون از ۲۸ میلیون — حدود ۱۵٪)
			\end{tabular}
		};
		
		% مشکل
		\node[rectangle, draw=rougerevolution, fill=rougelight,
		minimum width=10cm, minimum height=1cm, font=\small, rounded corners=3pt]
		at (0,-3.5) {
			\textbf{مشکل}: شاه با قانون اساسی موافق نبود + رأی محدود ← بی‌ثباتی
		};
		
	\end{tikzpicture}
	\caption{ساختار قانون اساسی ۱۷۹۱}
\end{figure}

\subsection{فرار به وارن: ۲۰-۲۱ ژوئن ۱۷۹۱}

\begin{table}[H]
	\centering
	\caption{فرار و دستگیری شاه}
	\begin{tabular}{|>{\bfseries}r|p{9cm}|}
		\hline
		\rowcolor{orroyallight}
		\textbf{مرحله} & \textbf{رویداد} \\
		\hline
		نقشه & فرار به مرز شرقی، پیوستن به ارتش اتریش \\
		\hline
		\rowcolor{grisclair}
		شب ۲۰ ژوئن & خانواده سلطنتی مبدل از تویلری خارج شد \\
		\hline
		۲۱ ژوئن & شناسایی شاه در وارن توسط رئیس پست \\
		\hline
		\rowcolor{grisclair}
		بازگشت & بازگرداندن شاه به پاریس در میان سکوت جمعیت \\
		\hline
		پیامد & شاه «تعلیق» شد، اما مجلس او را بازگرداند \\
		\hline
	\end{tabular}
\end{table}

\begin{enghelabbox}[نقطه بی‌بازگشت]
	فرار به وارن، تصویر «شاه-پدر» را نابود کرد:
	\begin{itemize}[nosep]
		\item شاه می‌خواست با دشمن خارجی علیه ملت متحد شود
		\item او در نامه‌ای انقلاب را محکوم کرده بود
		\item اعتماد مردم به شاه از بین رفت
		\item جمهوری‌خواهی از ایده حاشیه‌ای به جریان اصلی تبدیل شد
	\end{itemize}
\end{enghelabbox}

\subsection{کشتار شان‌دو‌مارس: ۱۷ ژوئیه ۱۷۹۱}

\begin{naghlbox}[اولین خون]
	جمهوری‌خواهان در میدان شان‌دو‌مارس طوماری برای برکناری شاه امضا می‌کردند. گارد ملی لافایت به روی جمعیت آتش گشود: ۵۰ کشته.
	
	\textbf{اهمیت}: اولین شکاف میان انقلابیون میانه‌رو و رادیکال.
\end{naghlbox}

\section{جنگ و سقوط سلطنت (۱۷۹۲)}

\subsection{اعلام جنگ: ۲۰ آوریل ۱۷۹۲}

\begin{figure}[H]
	\centering
	\begin{tikzpicture}[
		scale=0.9,
		group/.style={
			rectangle, draw=#1, line width=1.5pt, fill=#1,
			text=black, minimum width=5cm, minimum height=2.5cm,
			align=center, font=\small, rounded corners=5pt
		}
		]
		% عنوان
		\node[font=\large\bfseries, text=bleurepublique] at (0,5) {
			چرا همه خواستار جنگ بودند؟
		};
		
		% گروه‌ها
		\node[group=bleulight] (giron) at (-5,2) {
			\begin{tabular}{c}
				\textbf{ژیروندن‌ها}\\[5pt]
				«جنگ انقلاب را\\
				گسترش می‌دهد»\\
				«آزادی برای همه ملت‌ها»
			\end{tabular}
		};
		
		\node[group=orroyallight] (king) at (0,2) {
			\begin{tabular}{c}
				\textbf{شاه و دربار}\\[5pt]
				«شکست فرانسه\\
				سلطنت را باز می‌گرداند»\\
				(محاسبه اشتباه)
			\end{tabular}
		};
		
		\node[group=rougelight] (jacobin) at (5,2) {
			\begin{tabular}{c}
				\textbf{ژاکوبن‌های چپ}\\[5pt]
				«جنگ خائنان را\\
				افشا می‌کند»\\
				(روبسپیر مخالف بود)
			\end{tabular}
		};
		
		% نتیجه
		\node[rectangle, draw=rougerevolution, fill=rougelight,
		minimum width=12cm, minimum height=1.5cm, font=\small, rounded corners=5pt]
		at (0,-1) {
			\textbf{نتیجه}: جنگ با اتریش و پروس — آغاز ۲۳ سال جنگ در اروپا
		};
		
	\end{tikzpicture}
	\caption{ائتلاف عجیب طرفداران جنگ}
\end{figure}

\subsection{شکست‌های نظامی و رادیکال‌شدن}

\begin{table}[H]
	\centering
	\caption{مارپیچ رادیکال‌شدن بهار-تابستان ۱۷۹۲}
	\begin{tabular}{|c|p{5cm}|p{5cm}|}
		\hline
		\rowcolor{rougelight}
		\textbf{تاریخ} & \textbf{رویداد} & \textbf{پیامد} \\
		\hline
		آوریل & شکست‌های اولیه در جبهه & شایعه خیانت \\
		\hline
		\rowcolor{grisclair}
		ژوئن & شاه وزرای ژیروندن را برکنار کرد & خشم مردم \\
		\hline
		۲۰ ژوئن & هجوم جمعیت به تویلری & شاه تحقیر شد \\
		\hline
		\rowcolor{grisclair}
		۱۱ ژوئیه & اعلام «میهن در خطر» & بسیج عمومی \\
		\hline
		۲۵ ژوئیه & بیانیه برانشوایگ & تهدید نابودی پاریس \\
		\hline
		\rowcolor{grisclair}
		اوت & فدره‌های مارسی به پاریس & نیروی نظامی انقلابی \\
		\hline
	\end{tabular}
\end{table}

\begin{enghelabbox}[بیانیه برانشوایگ: اشتباه مهلک]
	فرمانده پروسی تهدید کرد که اگر به شاه آسیبی برسد، پاریس را ویران خواهد کرد.
	
	\textbf{نتیجه}: به‌جای ترساندن پاریسی‌ها، آنها را خشمگین و متحد کرد. شاه «عامل دشمن» به نظر رسید.
\end{enghelabbox}

\subsection{۱۰ اوت ۱۷۹۲: سقوط سلطنت}

\begin{figure}[H]
	\centering
	\begin{tikzpicture}[
		scale=0.85,
		transform shape,
		event/.style={
			rectangle, draw=rougerevolution, line width=2pt, fill=rougelight,
			text=black, minimum width=4cm, minimum height=1.5cm,
			align=center, font=\small, rounded corners=5pt
		}
		]
		% خط زمان
		\draw[line width=3pt, gris] (0,0) -- (14,0);
		
		% ساعات
		\node[below, font=\scriptsize] at (0,-0.3) {شب قبل};
		\node[below, font=\scriptsize] at (3.5,-0.3) {سحر};
		\node[below, font=\scriptsize] at (7,-0.3) {صبح};
		\node[below, font=\scriptsize] at (10.5,-0.3) {ظهر};
		\node[below, font=\scriptsize] at (14,-0.3) {عصر};
		
		% رویدادها
		\node[event] at (1.5,2.5) {
			\begin{tabular}{c}
				تشکیل کمون\\
				شورشی پاریس
			\end{tabular}
		};
		\draw[->, >=Stealth, rougerevolution, line width=1.5pt] (1.5,1.7) -- (1.5,0.3);
		
		\node[event] at (5,2.5) {
			\begin{tabular}{c}
				حرکت به سوی\\
				تویلری
			\end{tabular}
		};
		\draw[->, >=Stealth, rougerevolution, line width=1.5pt] (5,1.7) -- (5,0.3);
		
		\node[event] at (8.5,2.5) {
			\begin{tabular}{c}
				نبرد خونین\\
				۶۰۰ سوئیسی کشته
			\end{tabular}
		};
		\draw[->, >=Stealth, rougerevolution, line width=1.5pt] (8.5,1.7) -- (8.5,0.3);
		
		\node[event] at (12.5,2.5) {
			\begin{tabular}{c}
				شاه «تعلیق»\\
				به تامپل منتقل شد
			\end{tabular}
		};
		\draw[->, >=Stealth, rougerevolution, line width=1.5pt] (12.5,1.7) -- (12.5,0.3);
		
		% نتیجه
		\node[rectangle, draw=bleurepublique, fill=bleulight,
		minimum width=12cm, minimum height=1.2cm, font=\small\bfseries, rounded corners=5pt]
		at (7,-2) {
			پایان سلطنت — فراخوان کنوانسیون ملی با رأی همگانی مردان
		};
		
	\end{tikzpicture}
	\caption{سقوط سلطنت: ۱۰ اوت ۱۷۹۲}
\end{figure}

\subsection{کشتار سپتامبر ۱۷۹۲}

\begin{enghelabbox}[خشونت توده‌ای]
	از ۲ تا ۶ سپتامبر، جمعیت‌ها به زندان‌های پاریس هجوم بردند و ۱۱۰۰ تا ۱۴۰۰ زندانی را کشتند:
	\begin{itemize}[nosep]
		\item بسیاری روحانیون «سرپیچ» بودند
		\item برخی جنایتکاران عادی
		\item شایعه: «زندانیان می‌خواهند قیام کنند»
	\end{itemize}
	
	\textbf{اهمیت}: خشونت «از پایین» که رهبران نتوانستند (یا نخواستند) متوقف کنند.
\end{enghelabbox}

\section{جمهوری اول و کنوانسیون (۱۷۹۲-۱۷۹۵)}

\subsection{اعلام جمهوری: ۲۱ سپتامبر ۱۷۹۲}

\begin{naghlbox}[اعلام جمهوری]
	«کنوانسیون ملی اعلام می‌کند که سلطنت در فرانسه ملغی است.»
	
	«سال اول جمهوری فرانسه» از ۲۲ سپتامبر ۱۷۹۲ آغاز شد.
	\begin{flushright}
		— مصوبه کنوانسیون، ۲۱-۲۲ سپتامبر ۱۷۹۲
	\end{flushright}
\end{naghlbox}

\subsection{جناح‌بندی کنوانسیون}

\begin{figure}[H]
	\centering
	\begin{tikzpicture}[
		scale=0.9,
		faction/.style={
			rectangle, draw=#1, line width=2pt, fill=#1,
			text=black, minimum width=5cm, minimum height=4cm,
			align=center, font=\small, rounded corners=5pt
		}
		]
		% عنوان
		\node[font=\large\bfseries, text=bleurepublique] at (0,5.5) {
			جناح‌های کنوانسیون ملی
		};
		
		% سه جناح
		\node[faction=bleulight] (giron) at (-5.5,1.5) {
			\begin{tabular}{c}
				\textbf{ژیروندن‌ها}\\
				(راست)\\[5pt]
				{\scriptsize بورژوازی استانی}\\
				{\scriptsize فدرالیسم}\\
				{\scriptsize اقتصاد آزاد}\\
				{\scriptsize علیه پاریس}\\[3pt]
				{\tiny بریسو، ورنیو، رولان}
			\end{tabular}
		};
		
		\node[faction=grisclair] (plain) at (0,1.5) {
			\begin{tabular}{c}
				\textbf{دشت/باتلاق}\\
				(مرکز)\\[5pt]
				{\scriptsize اکثریت ساکت}\\
				{\scriptsize بدون ایدئولوژی ثابت}\\
				{\scriptsize رأی تعیین‌کننده}\\
				{\scriptsize بقا اولویت اول}\\[3pt]
				{\tiny سی‌یس، بارر}
			\end{tabular}
		};
		
		\node[faction=rougelight] (mont) at (5.5,1.5) {
			\begin{tabular}{c}
				\textbf{مونتانیارها/کوه}\\
				(چپ)\\[5pt]
				{\scriptsize ژاکوبن‌ها}\\
				{\scriptsize تمرکزگرا}\\
				{\scriptsize کنترل قیمت}\\
				{\scriptsize اتحاد با سانکولوت‌ها}\\[3pt]
				{\tiny روبسپیر، دانتون، مارا}
			\end{tabular}
		};
		
		% تعداد
		\node[font=\scriptsize, text=bleurepublique] at (-5.5,-1.2) {۱۶۰ نماینده};
		\node[font=\scriptsize, text=gris] at (0,-1.2) {۴۰۰ نماینده};
		\node[font=\scriptsize, text=rougerevolution] at (5.5,-1.2) {۱۴۰ نماینده};
		
		% کشمکش
		\draw[<->, >=Stealth, line width=2pt, black!60,
		decorate, decoration={zigzag, segment length=6pt, amplitude=2pt}]
		(giron.east) -- (mont.west);
		
		\node[font=\scriptsize\bfseries] at (0,-2.5) {
			کشمکش اصلی: ژیروندن‌ها در برابر مونتانیارها (۱۷۹۲-۱۷۹۳)
		};
		
	\end{tikzpicture}
	\caption{جناح‌بندی در کنوانسیون ملی}
\end{figure}

\subsection{محاکمه و اعدام لویی شانزدهم}

\begin{table}[H]
	\centering
	\caption{محاکمه لویی شانزدهم (دسامبر ۱۷۹۲ - ژانویه ۱۷۹۳)}
	\begin{tabular}{|>{\bfseries}r|p{5cm}|p{5cm}|}
		\hline
		\rowcolor{orroyallight}
		\textbf{مرحله} & \textbf{رویداد} & \textbf{رأی} \\
		\hline
		اتهام & خیانت، توطئه با دشمن & — \\
		\hline
		\rowcolor{grisclair}
		پرسش ۱ & آیا شاه مجرم است؟ & ۶۹۳ آری (تقریباً اتفاق آرا) \\
		\hline
		پرسش ۲ & آیا رأی مردم لازم است؟ & ۴۲۴ نه، ۲۸۷ آری \\
		\hline
		\rowcolor{grisclair}
		پرسش ۳ & مجازات چیست؟ & ۳۶۱ اعدام فوری، ۳۶۰ غیر \\
		\hline
		پرسش ۴ & آیا تأخیر در اجرا؟ & ۳۸۰ نه \\
		\hline
	\end{tabular}
\end{table}

\begin{naghlbox}[رأی سرنوشت‌ساز]
	یک رأی تفاوت. فیلیپ اگالیته (پسرعموی شاه) هم به اعدام رأی داد.
	
	دانتون: «پادشاهان اروپا ما را به چالش می‌کشند. ما دستکش یک سر شاه را به پایشان می‌اندازیم.»
\end{naghlbox}

\begin{figure}[H]
	\centering
	\begin{tikzpicture}
		% کادر
		\node[rectangle, draw=rougerevolution, line width=3pt, fill=rougelight,
		minimum width=12cm, minimum height=3cm, rounded corners=10pt]
		at (0,0) {};
		
		% متن
		\node[font=\large\bfseries, text=rougerevolution] at (0,0.5) {
			۲۱ ژانویه ۱۷۹۳
		};
		\node[font=\normalsize] at (0,-0.3) {
			لویی شانزدهم در میدان انقلاب (کنکورد امروزی) گردن زده شد
		};
		\node[font=\small, text=gris] at (0,-1) {
			آخرین سخنان: «من بی‌گناه از همه اتهامات می‌میرم...»
		};
		
	\end{tikzpicture}
	\caption{اعدام لویی شانزدهم}
\end{figure}

\begin{olgoobox}[اهمیت اعدام شاه]
	اعدام لویی شانزدهم نقطه بی‌بازگشت بود:
	\begin{enumerate}[nosep]
		\item \textbf{نمادین}: پایان هزار سال سلطنت «مقدس»
		\item \textbf{داخلی}: همه نمایندگان در خون شاه شریک شدند — بازگشتی نبود
		\item \textbf{خارجی}: ائتلاف اول علیه فرانسه تشکیل شد (انگلستان، اسپانیا، هلند...)
		\item \textbf{ایدئولوژیک}: جمهوری دیگر انتخاب نبود، ضرورت بود
	\end{enumerate}
\end{olgoobox}

\section{بحران ۱۷۹۳: جمهوری در محاصره}

% صفحه افقی برای نقشه بحران
\begin{landscape}
	\begin{figure}[p]
		\centering
		\begin{tikzpicture}[
			scale=0.8,
			transform shape,
			threat/.style={
				rectangle, draw=rougerevolution, line width=2pt, fill=rougelight,
				text=black, minimum width=4cm, minimum height=2cm,
				align=center, font=\small, rounded corners=5pt
			},
			center/.style={
				ellipse, draw=bleurepublique, line width=3pt, fill=bleulight,
				text=black, minimum width=5cm, minimum height=3cm,
				align=center, font=\bfseries
			}
			]
			% عنوان
			\node[font=\Large\bfseries, text=rougerevolution] at (0,8) {
				جمهوری در محاصره: بهار ۱۷۹۳
			};
			
			% مرکز
			\node[center] (paris) at (0,0) {
				پاریس\\
				کنوانسیون
			};
			
			% تهدیدات خارجی
\node[threat] (north) at (0,5) {
	\begin{tabular}{c}
		\textbf{شمال}\\
		اتریش، پروس\\
		هلند
	\end{tabular}
};

\node[threat] (east) at (7,2) {
	\begin{tabular}{c}
		\textbf{شرق}\\
		ارتش‌های امپراتوری\\
		تهدید آلزاس
	\end{tabular}
};

\node[threat] (south) at (5,-4) {
	\begin{tabular}{c}
		\textbf{جنوب}\\
		اسپانیا\\
		پیمونت-ساردنی
	\end{tabular}
};

\node[threat] (west) at (-7,2) {
	\begin{tabular}{c}
		\textbf{غرب}\\
		انگلستان\\
		محاصره دریایی
	\end{tabular}
};

% تهدیدات داخلی
\node[rectangle, draw=violetempire, line width=2pt, fill=violetlight,
text=black, minimum width=4cm, minimum height=2cm,
align=center, font=\small, rounded corners=5pt] 
(vendee) at (-6,-3) {
	\begin{tabular}{c}
		\textbf{واندé}\\
		شورش دهقانی\\
		سلطنت‌طلب-کاتولیک
	\end{tabular}
};

\node[rectangle, draw=orroyaldark, line width=2pt, fill=orroyallight,
text=black, minimum width=4cm, minimum height=2cm,
align=center, font=\small, rounded corners=5pt]
(federalist) at (0,-5) {
	\begin{tabular}{c}
		\textbf{شورش فدرالیست}\\
		لیون، مارسی، بوردو\\
		ژیروندن‌های فراری
	\end{tabular}
};

\node[rectangle, draw=rougerevolution, line width=2pt, fill=rougemid,
text=black, minimum width=4cm, minimum height=2cm,
align=center, font=\small, rounded corners=5pt]
(toulon) at (6,-2) {
	\begin{tabular}{c}
		\textbf{تولون}\\
		تسلیم به انگلیس\\
		ناوگان از دست رفت
	\end{tabular}
};

% پیکان‌های تهدید
\draw[->, >=Stealth, line width=2pt, rougerevolution] (north) -- (paris);
\draw[->, >=Stealth, line width=2pt, rougerevolution] (east) -- (paris);
\draw[->, >=Stealth, line width=2pt, rougerevolution] (south) -- (paris);
\draw[->, >=Stealth, line width=2pt, rougerevolution] (west) -- (paris);
\draw[->, >=Stealth, line width=2pt, violetempire] (vendee) -- (paris);
\draw[->, >=Stealth, line width=2pt, orroyaldark] (federalist) -- (paris);
\draw[->, >=Stealth, line width=2pt, rougerevolution] (toulon) -- (paris);

% آمار
\node[rectangle, draw=gris, fill=grisclair,
minimum width=8cm, minimum height=1.5cm, font=\small]
at (0,-8) {
	\begin{tabular}{c}
		۶۰ دپارتمان از ۸۳ در شورش یا اشغال\\
		ارتش در حال فروپاشی | خزانه خالی | قحطی
	\end{tabular}
};

\end{tikzpicture}
\caption{جمهوری در محاصره: بحران بهار ۱۷۹۳}
\end{figure}
\end{landscape}

\subsection{شورش واندé: جنگ داخلی}

\begin{enghelabbox}[ریشه‌های شورش واندé]
در مارس ۱۷۹۳، دهقانان غرب فرانسه علیه جمهوری شوریدند:

\textbf{علل فوری}:
\begin{itemize}[nosep]
\item سربازگیری اجباری (۳۰۰,۰۰۰ نفر)
\item کشیشان «سرپیچ» محبوب محلی
\end{itemize}

\textbf{علل عمیق‌تر}:
\begin{itemize}[nosep]
\item انقلاب سود چندانی برای دهقانان نداشت
\item وفاداری به کلیسا و شاه
\item تنفر از بورژوازی شهری
\end{itemize}

\textbf{نتیجه}: جنگ داخلی وحشیانه با ۲۰۰,۰۰۰-۳۰۰,۰۰۰ کشته (تخمین‌ها متفاوت است).
\end{enghelabbox}

\subsection{سقوط ژیروندن‌ها: ۲ ژوئن ۱۷۹۳}

\begin{table}[H]
\centering
\caption{سقوط ژیروندن‌ها}
\begin{tabular}{|>{\bfseries}r|p{9cm}|}
\hline
\rowcolor{rougelight}
\textbf{مرحله} & \textbf{رویداد} \\
\hline
۳۱ مه & قیام سانکولوت‌ها، محاصره کنوانسیون \\
\hline
\rowcolor{grisclair}
۱ ژوئن & کنوانسیون مقاومت کرد \\
\hline
۲ ژوئن & ۸۰,۰۰۰ گارد ملی کنوانسیون را محاصره کردند \\
\hline
\rowcolor{grisclair}
نتیجه & ۲۹ نماینده ژیروندن بازداشت شدند \\
\hline
پیامد & مونتانیارها قدرت را در دست گرفتند \\
\hline
\end{tabular}
\end{table}

\begin{naghlbox}[ایرو درباره سقوط ژیروندن‌ها]
«انقلاب مانند ساتورن فرزندان خود را می‌خورد.»
\begin{flushright}
— پیر ویکتورنین ایرو، نماینده ژیروندن (قبل از اعدام)
\end{flushright}
\end{naghlbox}

\section{دوره ترور (ژوئن ۱۷۹۳ - ژوئیه ۱۷۹۴)}

\subsection{ساختار حکومت انقلابی}

\begin{figure}[H]
\centering
\begin{tikzpicture}[
scale=0.9,
organ/.style={
rectangle, draw=#1, line width=2pt, fill=#1,
text=black, minimum width=5cm, minimum height=2cm,
align=center, font=\small, rounded corners=5pt
}
]
% عنوان
\node[font=\large\bfseries, text=rougerevolution] at (0,6) {
ساختار حکومت انقلابی
};

% کنوانسیون
\node[organ=bleulight] (conv) at (0,4) {
\begin{tabular}{c}
	\textbf{کنوانسیون ملی}\\
	{\scriptsize قوه مقننه، حاکمیت ملی}
\end{tabular}
};

% کمیته‌ها
\node[organ=rougelight] (csp) at (-4,1.5) {
\begin{tabular}{c}
	\textbf{کمیته نجات عمومی}\\
	{\scriptsize ۱۲ عضو، قدرت اجرایی}\\
	{\scriptsize روبسپیر، سن‌ژوست، کارنو}
\end{tabular}
};

\node[organ=violetlight] (csg) at (4,1.5) {
\begin{tabular}{c}
	\textbf{کمیته امنیت عمومی}\\
	{\scriptsize پلیس سیاسی}\\
	{\scriptsize بازداشت مظنونین}
\end{tabular}
};

% دادگاه
\node[organ=rougemid] (trib) at (0,-1) {
\begin{tabular}{c}
	\textbf{دادگاه انقلابی}\\
	{\scriptsize محاکمه سریع، بدون استیناف}\\
	{\scriptsize فقط دو حکم: تبرئه یا مرگ}
\end{tabular}
};

% نمایندگان مأمور
\node[organ=vertlight] (repr) at (0,-3.5) {
\begin{tabular}{c}
	\textbf{نمایندگان مأمور}\\
	{\scriptsize اعزام به ارتش‌ها و دپارتمان‌ها}\\
	{\scriptsize قدرت نامحدود محلی}
\end{tabular}
};

% پیکان‌ها
\draw[->, >=Stealth, line width=1.5pt, gris] (conv) -- (csp);
\draw[->, >=Stealth, line width=1.5pt, gris] (conv) -- (csg);
\draw[->, >=Stealth, line width=1.5pt, gris] (csp) -- (trib);
\draw[->, >=Stealth, line width=1.5pt, gris] (csg) -- (trib);
\draw[->, >=Stealth, line width=1.5pt, gris] (csp) -- (repr);

\end{tikzpicture}
\caption{ساختار حکومت انقلابی دوره ترور}
\end{figure}

\subsection{قانون مظنونین: ۱۷ سپتامبر ۱۷۹۳}

\begin{enghelabbox}[تعریف «مظنون»]
قانون مظنونین هر کسی را که مشمول یکی از این دسته‌ها بود، قابل بازداشت اعلام کرد:
\begin{enumerate}[nosep]
\item کسانی که «با رفتار، روابط، گفتار یا نوشته‌ها» خود را دشمن آزادی نشان داده‌اند
\item اشراف مهاجر و خویشاوندانشان
\item کارمندان برکنارشده
\item کسانی که گواهی «شهروند خوب» ندارند
\end{enumerate}

\textbf{نتیجه}: حدود ۵۰۰,۰۰۰ بازداشت در سراسر فرانسه
\end{enghelabbox}

\subsection{اقتصاد جنگی: کنترل قیمت‌ها}

\begin{table}[H]
\centering
\caption{سیاست‌های اقتصادی دوره ترور}
\begin{tabular}{|>{\bfseries}r|p{5cm}|p{4.5cm}|}
\hline
\rowcolor{vertlight}
\textbf{سیاست} & \textbf{محتوا} & \textbf{نتیجه} \\
\hline
ماکزیمم عمومی & سقف قیمت برای ۳۹ کالای ضروری & کاهش موقت تورم \\
\hline
\rowcolor{grisclair}
ماکزیمم دستمزد & سقف دستمزد کارگران & نارضایتی سانکولوت‌ها \\
\hline
بسیج اقتصادی & مصادره، کنترل تولید & افزایش تولید جنگی \\
\hline
\rowcolor{grisclair}
اسینیا & پول کاغذی & تورم شدید پس از ترور \\
\hline
\end{tabular}
\end{table}

\subsection{«جمهوری فضیلت»: پروژه روبسپیر}

\begin{figure}[H]
\centering
\begin{tikzpicture}[
scale=0.9,
element/.style={
rectangle, draw=bleurepublique, line width=1.5pt, fill=bleulight,
text=black, minimum width=4cm, minimum height=1.5cm,
align=center, font=\small, rounded corners=3pt
}
]
% عنوان
\node[font=\large\bfseries, text=bleurepublique] at (0,5) {
پروژه بازسازی جامعه
};

% مرکز
\node[ellipse, draw=rougerevolution, line width=2pt, fill=rougelight,
minimum width=4cm, minimum height=2cm, font=\bfseries]
(center) at (0,2) {
\begin{tabular}{c}
	جمهوری\\
	فضیلت
\end{tabular}
};

% عناصر
\node[element] (cal) at (-5,2) {
\begin{tabular}{c}
	تقویم انقلابی\\
	{\scriptsize ۱۲ ماه ۳۰ روزه}
\end{tabular}
};

\node[element] (cult) at (5,2) {
\begin{tabular}{c}
	پرستش موجود برتر\\
	{\scriptsize جایگزین مسیحیت}
\end{tabular}
};

\node[element] (edu) at (-5,-0.5) {
\begin{tabular}{c}
	آموزش جمهوری‌خواهانه\\
	{\scriptsize تربیت شهروند}
\end{tabular}
};

\node[element] (name) at (5,-0.5) {
\begin{tabular}{c}
	تغییر نام‌ها\\
	{\scriptsize خیابان، شهر، افراد}
\end{tabular}
};

\node[element] (dress) at (0,-1) {
\begin{tabular}{c}
	لباس جمهوری‌خواهانه\\
	{\scriptsize بدون تجملات اشرافی}
\end{tabular}
};

% خطوط
\draw[bleurepublique, line width=1pt] (cal) -- (center);
\draw[bleurepublique, line width=1pt] (cult) -- (center);
\draw[bleurepublique, line width=1pt] (edu) -- (center);
\draw[bleurepublique, line width=1pt] (name) -- (center);
\draw[bleurepublique, line width=1pt] (dress) -- (center);

\end{tikzpicture}
\caption{پروژه «جمهوری فضیلت» روبسپیر}
\end{figure}

\begin{naghlbox}[روبسپیر درباره ترور و فضیلت]
«اگر انگیزه حکومت مردمی در زمان صلح، فضیلت است، انگیزه‌اش در زمان انقلاب هم فضیلت است و هم ترور: فضیلت، بدون آن دموکراسی ناتوان است؛ ترور، بدون آن فضیلت عاجز است. ترور چیزی نیست جز عدالت فوری، سخت‌گیر، انعطاف‌ناپذیر.»
\begin{flushright}
— ماکسیمیلیان روبسپیر، ۵ فوریه ۱۷۹۴
\end{flushright}
\end{naghlbox}

\subsection{آمار ترور}

\begin{figure}[H]
\centering
\begin{tikzpicture}
\begin{axis}[
width=14cm,
height=8cm,
ybar,
bar width=15pt,
ylabel={تعداد اعدام‌ها},
xlabel={ماه (۱۷۹۳-۱۷۹۴)},
ymin=0, ymax=1400,
xtick=data,
xticklabels={
	مارس,آوریل,مه,ژوئن,ژوئیه,اوت,سپت,اکت,نوو,دسا,
	ژان,فور,مارس,آور,مه,ژوئن,ژوئیه
},
x tick label style={rotate=45, anchor=east, font=\tiny},
nodes near coords style={font=\tiny},
grid=major,
grid style={gris!30}
]
\addplot[fill=rougerevolution!70, draw=rougerevolution] coordinates {
	(1,20) (2,35) (3,40) (4,50) (5,60) (6,70) (7,80)
	(8,180) (9,200) (10,170) (11,150) (12,200)
	(13,350) (14,500) (15,700) (16,800) (17,1300)
};
\end{axis}

% نشانگر قانون پرریال
\draw[->, >=Stealth, line width=2pt, rougerevolution] (11.5,6.5) -- (11.5,5.5);
\node[above, font=\scriptsize\bfseries, text=rougerevolution] at (11.5,6.6) {قانون ۲۲ پرریال};

\end{tikzpicture}
\caption{روند اعدام‌ها در دوره ترور (مارس ۱۷۹۳ - ژوئیه ۱۷۹۴)}
\end{figure}

\begin{table}[H]
\centering
\caption{آمار ترور: چه کسانی اعدام شدند؟}
\begin{tabular}{|>{\bfseries}r|c|c|}
\hline
\rowcolor{rougelight}
\textbf{گروه اجتماعی} & \textbf{درصد اعدام‌شدگان} & \textbf{درصد جمعیت} \\
\hline
اشراف & ۸.۵٪ & ۱.۵٪ \\
\hline
\rowcolor{grisclair}
روحانیون & ۶.۵٪ & ۰.۵٪ \\
\hline
بورژوازی & ۲۵٪ & ۸٪ \\
\hline
\rowcolor{grisclair}
دهقانان & ۲۸٪ & ۸۵٪ \\
\hline
کارگران و صنعتگران & ۳۲٪ & ۵٪ \\
\hline
\end{tabular}
\end{table}

\begin{olgoobox}[نکته کلیدی: ترور طبقاتی نبود]
برخلاف تصور رایج، اکثر قربانیان ترور اشراف و روحانیون نبودند. اکثریت از طبقات پایین بودند — به‌ویژه در مناطق شورشی مانند واندé و لیون. ترور بیشتر «سیاسی» بود تا «طبقاتی».
\end{olgoobox}

\subsection{قربانیان برجسته}

\begin{table}[H]
\centering
\small
\caption{قربانیان برجسته ترور}
\begin{tabular}{|>{\bfseries}r|c|p{6cm}|}
\hline
\rowcolor{rougelight}
\textbf{نام} & \textbf{تاریخ اعدام} & \textbf{جرم} \\
\hline
ماری آنتوانت & ۱۶ اکتبر ۱۷۹۳ & خیانت، توطئه \\
\hline
\rowcolor{grisclair}
نمایندگان ژیروندن (۲۱ نفر) & ۳۱ اکتبر ۱۷۹۳ & فدرالیسم، توطئه \\
\hline
فیلیپ اگالیته & ۶ نوامبر ۱۷۹۳ & توطئه سلطنتی! \\
\hline
\rowcolor{grisclair}
مادام رولان & ۸ نوامبر ۱۷۹۳ & توطئه ژیروندن \\
\hline
دانتون و یارانش & ۵ آوریل ۱۷۹۴ & فساد، خیانت \\
\hline
\rowcolor{grisclair}
ابر و «خشمگینان» & ۲۴ مارس ۱۷۹۴ & افراط‌گرایی \\
\hline
لاووازیه (شیمیدان) & ۸ مه ۱۷۹۴ & سابقه مالیاتی \\
\hline
\end{tabular}
\end{table}

\begin{naghlbox}[دانتون در پای گیوتین]
«نشان دادن سر مرا به مردم فراموش نکن. چنین سری به‌این‌زودی دیده نخواهد شد.»
\begin{flushright}
— ژرژ ژاک دانتون، ۵ آوریل ۱۷۹۴
\end{flushright}
\end{naghlbox}

\subsection{ماکسیمیلیان روبسپیر: چهره ترور}

\begin{figure}[H]
\centering
\begin{tikzpicture}[
scale=0.95,
trait/.style={
rectangle, draw=#1, line width=1.5pt, fill=#1,
text=black, minimum width=4.5cm, minimum height=1.2cm,
align=center, font=\small, rounded corners=3pt
}
]
% کادر مرکزی
\node[ellipse, draw=rougerevolution, line width=3pt, fill=rougelight,
minimum width=4cm, minimum height=2.5cm, font=\bfseries]
(center) at (0,0) {
\begin{tabular}{c}
	روبسپیر\\
	{\scriptsize (۱۷۵۸-۱۷۹۴)}
\end{tabular}
};

% ویژگی‌ها
\node[trait=bleulight] at (-5.5,2) {زاهد و پرهیزکار};
\node[trait=bleulight] at (-5.5,0.5) {وکیل، خطیب ماهر};
\node[trait=bleulight] at (-5.5,-1) {«تباهی‌ناپذیر»};

\node[trait=rougelight] at (5.5,2) {ایمان به فضیلت مطلق};
\node[trait=rougelight] at (5.5,0.5) {عدم تساهل با مخالفان};
\node[trait=rougelight] at (5.5,-1) {توجیه خشونت با اخلاق};

% برچسب‌ها
\node[font=\footnotesize\bfseries, text=bleurepublique] at (-5.5,3) {ویژگی‌های شخصی};
\node[font=\footnotesize\bfseries, text=rougerevolution] at (5.5,3) {ویژگی‌های سیاسی};

% نقل قول
\node[rectangle, draw=gris, fill=grisclair,
minimum width=10cm, minimum height=1.5cm, font=\small, rounded corners=3pt]
at (0,-3.5) {
\begin{tabular}{c}
	«هیچ‌کس انقلاب را بی‌آنکه شر باشد، دوست نمی‌دارد؛\\
	و هیچ‌کس آن را بی‌آنکه فضیلت‌مند باشد، خدمت نمی‌کند.»
\end{tabular}
};

\end{tikzpicture}
\caption{ماکسیمیلیان روبسپیر: پرتره روان‌شناختی}
\end{figure}

\subsection{جشن موجود برتر: ۸ ژوئن ۱۷۹۴}

\begin{enghelabbox}[دین مدنی روبسپیر]
روبسپیر معتقد بود جمهوری به «دین» نیاز دارد — نه مسیحیت، نه آتئیسم، بلکه «پرستش موجود برتر»:
\begin{itemize}[nosep]
\item اعتقاد به خدایی که عدالت را پاداش می‌دهد
\item اعتقاد به جاودانگی روح
\item وظایف اخلاقی شهروندی
\end{itemize}

جشن بزرگی برگزار شد که روبسپیر خود آن را رهبری کرد. این اقدام بسیاری را ترساند: آیا می‌خواهد دیکتاتور شود؟
\end{enghelabbox}

\section{سقوط روبسپیر: ۹ ترمیدور (۲۷ ژوئیه ۱۷۹۴)}

\subsection{چرا روبسپیر سقوط کرد؟}

\begin{figure}[H]
\centering
\begin{tikzpicture}[
scale=0.9,
cause/.style={
rectangle, draw=rougerevolution, line width=1.5pt, fill=rougelight,
text=black, minimum width=4.5cm, minimum height=1.5cm,
align=center, font=\small, rounded corners=3pt
}
]
% مرکز
\node[ellipse, draw=bleurepublique, line width=3pt, fill=bleulight,
minimum width=4cm, minimum height=2cm, font=\bfseries]
(fall) at (0,0) {
\begin{tabular}{c}
	سقوط\\
	روبسپیر
\end{tabular}
};

% علل
\node[cause] (c1) at (-5,3) {
\begin{tabular}{c}
	قانون ۲۲ پرریال\\
	{\scriptsize ترور بی‌حد و حصر}
\end{tabular}
};

\node[cause] (c2) at (0,4) {
\begin{tabular}{c}
	ترس نمایندگان\\
	{\scriptsize همه احساس خطر می‌کردند}
\end{tabular}
};

\node[cause] (c3) at (5,3) {
\begin{tabular}{c}
	پیروزی نظامی\\
	{\scriptsize فلوروس — دیگر بحران نیست}
\end{tabular}
};

\node[cause] (c4) at (-5,-3) {
\begin{tabular}{c}
	انزوای روبسپیر\\
	{\scriptsize دو هفته غیبت از کمیته}
\end{tabular}
};

\node[cause] (c5) at (0,-4) {
\begin{tabular}{c}
	ائتلاف مخالفان\\
	{\scriptsize «دشت» + چپ + راست}
\end{tabular}
};

\node[cause] (c6) at (5,-3) {
\begin{tabular}{c}
	خستگی از ترور\\
	{\scriptsize حتی کمیته‌ها}
\end{tabular}
};

% پیکان‌ها
\draw[->, >=Stealth, line width=1.5pt, gris] (c1) -- (fall);
\draw[->, >=Stealth, line width=1.5pt, gris] (c2) -- (fall);
\draw[->, >=Stealth, line width=1.5pt, gris] (c3) -- (fall);
\draw[->, >=Stealth, line width=1.5pt, gris] (c4) -- (fall);
\draw[->, >=Stealth, line width=1.5pt, gris] (c5) -- (fall);
\draw[->, >=Stealth, line width=1.5pt, gris] (c6) -- (fall);

\end{tikzpicture}
\caption{علل سقوط روبسپیر}
\end{figure}

\subsection{روزهای سرنوشت‌ساز}

\begin{table}[H]
\centering
\caption{کودتای ترمیدور}
\begin{tabular}{|>{\bfseries}r|p{9cm}|}
\hline
\rowcolor{violetlight}
\textbf{تاریخ} & \textbf{رویداد} \\
\hline
۸ ترمیدور (۲۶ ژوئیه) & روبسپیر در کنوانسیون سخنرانی کرد، از «توطئه» گفت اما نام نبرد \\
\hline
\rowcolor{grisclair}
شب ۸ ترمیدور & ائتلاف مخالفان شکل گرفت \\
\hline
۹ ترمیدور صبح & حمله به روبسپیر در کنوانسیون، فریاد «مرگ بر ظالم!» \\
\hline
\rowcolor{grisclair}
۹ ترمیدور عصر & روبسپیر و یارانش بازداشت، سپس آزاد شدند \\
\hline
۹ ترمیدور شب & کمون پاریس به حمایت برخاست، اما ناکام ماند \\
\hline
\rowcolor{grisclair}
۱۰ ترمیدور (۲۸ ژوئیه) & روبسپیر، سن‌ژوست و ۲۰ نفر دیگر اعدام شدند \\
\hline
\end{tabular}
\end{table}

\begin{naghlbox}[واپسین لحظات]
شب ۹ ترمیدور، گلوله‌ای (تیر خودش یا دیگری؟) فک روبسپیر را خرد کرد. فردا صبح، بدون محاکمه، گردن زده شد.

جمعیت فریاد می‌زد: «کثافت! به جهنم برو!»
\end{naghlbox}

\section{واکنش ترمیدوری و دایرکتوار (۱۷۹۴-۱۷۹۹)}

\subsection{دوره ترمیدوری: پایان رادیکالیسم}

\begin{table}[H]
\centering
\caption{تغییرات پس از ترمیدور}
\begin{tabular}{|>{\bfseries}r|p{5cm}|p{5cm}|}
\hline
\rowcolor{violetlight}
\textbf{حوزه} & \textbf{دوره ترور} & \textbf{پس از ترمیدور} \\
\hline
سیاست & کمیته نجات عمومی قدرت‌مدار & کنوانسیون قدرت را پس گرفت \\
\hline
\rowcolor{grisclair}
ترور & دادگاه انقلابی فعال & دادگاه بسته شد \\
\hline
اقتصاد & کنترل قیمت (ماکزیمم) & آزادسازی، تورم شدید \\
\hline
\rowcolor{grisclair}
ژاکوبن‌ها & قدرتمند & کلوپ بسته شد \\
\hline
سانکولوت‌ها & متحد حکومت & سرکوب شدند \\
\hline
\end{tabular}
\end{table}

\subsection{«ترور سفید»}

\begin{enghelabbox}[انتقام‌جویی]
پس از سقوط روبسپیر، موج انتقام‌جویی آغاز شد:
\begin{itemize}[nosep]
\item حمله به ژاکوبن‌ها و سانکولوت‌ها
\item آزادی زندانیان سیاسی
\item بازگشت روحانیون «سرپیچ»
\item «جوانان طلایی» (ژونس دوره) در خیابان‌ها
\item قتل‌های انتقامی در جنوب
\end{itemize}
\end{enghelabbox}

\subsection{شورش‌های ۱۷۹۵}

\begin{table}[H]
\centering
\caption{شورش‌های سال سوم (۱۷۹۵)}
\begin{tabular}{|>{\bfseries}r|c|p{4cm}|p{4cm}|}
\hline
\rowcolor{rougelight}
\textbf{شورش} & \textbf{تاریخ} & \textbf{خواست} & \textbf{نتیجه} \\
\hline
ژرمینال & ۱ آوریل & نان، قانون اساسی ۱۷۹۳ & سرکوب \\
\hline
\rowcolor{grisclair}
پرریال & ۲۰ مه & نان، آزادی زندانیان & سرکوب خشن \\
\hline
واندمیر & ۵ اکتبر & سلطنت‌طلبان & سرکوب توسط ناپلئون \\
\hline
\end{tabular}
\end{table}

\subsection{قانون اساسی سال سوم (۱۷۹۵) و دایرکتوار}

\begin{figure}[H]
\centering
\begin{tikzpicture}[
scale=0.9,
branch/.style={
rectangle, draw=#1, line width=2pt, fill=#1,
text=black, minimum width=5cm, minimum height=2cm,
align=center, font=\small, rounded corners=5pt
}
]
% عنوان
\node[font=\large\bfseries, text=violetempire] at (0,5) {
ساختار دایرکتوار (۱۷۹۵-۱۷۹۹)
};

% قوه مقننه
\node[branch=bleulight] (anc) at (-4,2.5) {
\begin{tabular}{c}
	\textbf{شورای پیران}\\
	{\scriptsize ۲۵۰ عضو، بالای ۴۰ سال}\\
	{\scriptsize تأیید یا رد قوانین}
\end{tabular}
};

\node[branch=bleulight] (cinq) at (4,2.5) {
\begin{tabular}{c}
	\textbf{شورای پانصد}\\
	{\scriptsize ۵۰۰ عضو، بالای ۳۰ سال}\\
	{\scriptsize پیشنهاد قوانین}
\end{tabular}
};

% قوه مجریه
\node[branch=violetlight] (dir) at (0,0) {
\begin{tabular}{c}
	\textbf{دایرکتوار}\\
	{\scriptsize ۵ مدیر، انتخاب توسط پیران}\\
	{\scriptsize هر سال یکی تعویض می‌شد}
\end{tabular}
};

% رأی‌دهندگان
\node[branch=vertlight] (vote) at (0,-2.5) {
\begin{tabular}{c}
	\textbf{رأی‌دهندگان}\\
	{\scriptsize مردان مالیات‌ده (حدود ۵ میلیون)}\\
	{\scriptsize انتخاب غیرمستقیم دو مرحله‌ای}
\end{tabular}
};

% پیکان‌ها
\draw[->, >=Stealth, line width=1.5pt, gris] (vote) -- (cinq);
\draw[->, >=Stealth, line width=1.5pt, gris] (vote) -- (anc);
\draw[->, >=Stealth, line width=1.5pt, gris] (anc) -- (dir);

% مشکل
\node[rectangle, draw=rougerevolution, fill=rougelight,
minimum width=12cm, minimum height=1cm, font=\small, rounded corners=3pt]
at (0,-4.5) {
\textbf{مشکل}: نظام ناپایدار — کودتاهای پی‌در‌پی برای حفظ قدرت
};

\end{tikzpicture}
\caption{ساختار نظام دایرکتوار}
\end{figure}

\subsection{بحران‌های دایرکتوار}

\begin{table}[H]
\centering
\caption{کودتاهای دوره دایرکتوار}
\begin{tabular}{|c|>{\bfseries}r|p{4cm}|p{4cm}|}
\hline
\rowcolor{violetlight}
\textbf{تاریخ} & \textbf{نام} & \textbf{علیه چه کسی} & \textbf{نتیجه} \\
\hline
۱۷۹۷ & فروکتیدور & سلطنت‌طلبان & پیروزی دایرکتوار \\
\hline
\rowcolor{grisclair}
۱۷۹۸ & فلورئال & ژاکوبن‌ها & پیروزی دایرکتوار \\
\hline
۱۷۹۹ & پرریال & مدیران میانه‌رو & سی‌یس به قدرت رسید \\
\hline
\rowcolor{grisclair}
۱۷۹۹ & برومر & دایرکتوار & ناپلئون به قدرت رسید \\
\hline
\end{tabular}
\end{table}

\begin{olgoobox}[چرا دایرکتوار شکست خورد؟]
\begin{enumerate}[nosep]
\item \textbf{بی‌ثباتی ذاتی}: هر سال انتخابات، هر سال بحران
\item \textbf{فساد}: مدیران و سیاستمداران فاسد
\item \textbf{بحران اقتصادی}: تورم، ورشکستگی دولت
\item \textbf{جنگ مداوم}: وابستگی به ارتش
\item \textbf{بی‌مشروعیتی}: نه راست راضی، نه چپ
\item \textbf{ظهور ژنرال‌ها}: ناپلئون در ایتالیا و مصر
\end{enumerate}
\end{olgoobox}

\section{کودتای برومر: پایان انقلاب؟}

\subsection{۱۸-۱۹ برومر سال هشتم (۹-۱۰ نوامبر ۱۷۹۹)}

\begin{figure}[H]
\centering
\begin{tikzpicture}[
scale=0.85,
transform shape,
event/.style={
rectangle, draw=violetempire, line width=2pt, fill=violetlight,
text=black, minimum width=4cm, minimum height=1.5cm,
align=center, font=\small, rounded corners=5pt
}
]
% خط زمان
\draw[line width=3pt, gris] (0,0) -- (14,0);

% دو روز
\node[below, font=\small\bfseries] at (3.5,-0.5) {۱۸ برومر};
\node[below, font=\small\bfseries] at (10.5,-0.5) {۱۹ برومر};

% رویدادها
\node[event] at (2,2.5) {
\begin{tabular}{c}
	شایعه توطئه\\
	انتقال مجلس به سن‌کلو
\end{tabular}
};
\draw[->, >=Stealth, violetempire, line width=1.5pt] (2,1.7) -- (2,0.3);

\node[event] at (6,2.5) {
\begin{tabular}{c}
	ناپلئون در شورای پیران\\
	موافقت با استعفای مدیران
\end{tabular}
};
\draw[->, >=Stealth, violetempire, line width=1.5pt] (6,1.7) -- (6,0.3);

\node[event] at (10,2.5) {
\begin{tabular}{c}
	ناپلئون در شورای پانصد\\
	مقاومت، فریاد «خارج کن!»
\end{tabular}
};
\draw[->, >=Stealth, violetempire, line width=1.5pt] (10,1.7) -- (10,0.3);

\node[event] at (13,2.5) {
\begin{tabular}{c}
	مداخله نظامی\\
	پراکنده کردن نمایندگان
\end{tabular}
};
\draw[->, >=Stealth, violetempire, line width=1.5pt] (13,1.7) -- (13,0.3);

% نتیجه
\node[rectangle, draw=bleurepublique, fill=bleulight,
minimum width=12cm, minimum height=1.5cm, font=\small\bfseries, rounded corners=5pt]
at (7,-2.5) {
\begin{tabular}{c}
	نتیجه: کنسولا — سه کنسول با ناپلئون به‌عنوان «کنسول اول»\\
	پایان جمهوری اول به معنای واقعی
\end{tabular}
};

\end{tikzpicture}
\caption{کودتای برومر (۹-۱۰ نوامبر ۱۷۹۹)}
\end{figure}

\begin{naghlbox}[ناپلئون پس از کودتا]
«شهروندان! انقلاب به اصولی که آن را آغاز کرده بود، بازگشت. انقلاب پایان یافت.»
\begin{flushright}
— اعلامیه کنسول‌ها، ۱۵ دسامبر ۱۷۹۹
\end{flushright}
\end{naghlbox}

\section{ارزیابی: انقلاب چه کرد؟}

\begin{figure}[H]
\centering
\begin{tikzpicture}[
scale=0.9,
achieve/.style={
rectangle, draw=vertnapoleon, line width=1.5pt, fill=vertlight,
text=black, minimum width=5.5cm, minimum height=1.2cm,
align=center, font=\small, rounded corners=3pt
},
fail/.style={
rectangle, draw=rougerevolution, line width=1.5pt, fill=rougelight,
text=black, minimum width=5.5cm, minimum height=1.2cm,
align=center, font=\small, rounded corners=3pt
}
]
% عناوین
\node[font=\large\bfseries, text=vertnapoleon] at (-4,5) {دستاوردها};
\node[font=\large\bfseries, text=rougerevolution] at (4,5) {شکست‌ها/هزینه‌ها};

% دستاوردها
\node[achieve] at (-4,3.8) {پایان فئودالیسم و امتیازات};
\node[achieve] at (-4,2.4) {برابری حقوقی (مردان)};
\node[achieve] at (-4,1) {اعلامیه حقوق بشر};
\node[achieve] at (-4,-0.4) {نظام اداری مدرن (دپارتمان‌ها)};
\node[achieve] at (-4,-1.8) {نظام متریک، آزادی اقتصادی};
\node[achieve] at (-4,-3.2) {سکولاریسم (لائیسیته)};

% شکست‌ها
\node[fail] at (4,3.8) {ترور و خشونت گسترده};
\node[fail] at (4,2.4) {عدم ثبات سیاسی};
\node[fail] at (4,1) {محرومیت زنان از حقوق};
\node[fail] at (4,-0.4) {جنگ‌های خانمان‌سوز};
\node[fail] at (4,-1.8) {پایان به دیکتاتوری};
\node[fail] at (4,-3.2) {شکاف‌های ماندگار (راست/چپ، لائیک/مذهبی)};

% خط جداکننده
\draw[gris, line width=2pt, dashed] (0,4.5) -- (0,-4);

\end{tikzpicture}
\caption{ترازنامه انقلاب فرانسه}
\end{figure}

\begin{olgoobox}[الگوی کلیدی: چرخه انقلابی]
انقلاب فرانسه الگویی را نشان داد که بارها تکرار شد:

\begin{enumerate}[nosep]
\item \textbf{بحران}: اقتصادی، سیاسی، مشروعیت
\item \textbf{فروپاشی}: نظام قدیم ناتوان از پاسخ
\item \textbf{امید}: اصلاح‌طلبان قدرت را می‌گیرند
\item \textbf{رادیکالیزاسیون}: بحران + جنگ ← افراطی‌ها
\item \textbf{ترور}: سرکوب «دشمنان»
\item \textbf{واکنش}: خستگی از خشونت
\item \textbf{مرد قوی}: ژنرالی که نظم را باز می‌گرداند
\end{enumerate}

این الگو در ۱۸۳۰، ۱۸۴۸، و حتی در انقلاب‌های دیگر (روسیه، چین) تکرار شد.
\end{olgoobox}






\section{منابع فصل سوم}

\subsection{منابع اولیه}

\begin{enumerate}[nosep]
\item اعلامیه حقوق بشر و شهروند. ۱۷۸۹.
\item روبسپیر، ماکسیمیلیان. سخنرانی‌ها و نوشته‌ها.
\item مذاکرات کنوانسیون ملی. آرشیو پارلمانی.
\item مارا، ژان-پل. دوست مردم (روزنامه).
\end{enumerate}

\subsection{منابع ثانویه}

\begin{enumerate}[nosep]
\item Furet, François and Ozouf, Mona. \textit{A Critical Dictionary of the French Revolution}. Harvard UP, 1989.
\item Lefebvre, Georges. \textit{The French Revolution}. 2 vols. Columbia UP, 1962-64.
\item Schama, Simon. \textit{Citizens: A Chronicle of the French Revolution}. Knopf, 1989.
\item Soboul, Albert. \textit{The Sans-Culottes}. Princeton UP, 1980.
\item Tackett, Timothy. \textit{The Coming of the Terror in the French Revolution}. Harvard UP, 2015.
\item McPhee, Peter. \textit{Liberty or Death: The French Revolution}. Yale UP, 2016.
\item Israel, Jonathan. \textit{Revolutionary Ideas}. Princeton UP, 2014.
\item Ozouf, Mona. \textit{Festivals and the French Revolution}. Harvard UP, 1988.
\item Palmer, R.R. \textit{Twelve Who Ruled}. Princeton UP, 1941.
\item Hampson, Norman. \textit{The Terror in the French Revolution}. Historical Association, 1981.
\end{enumerate}

\begin{kholasebox}[جمع‌بندی فصل سوم]
در این فصل دیدیم که:

\begin{itemize}[nosep]
\item \textbf{۱۷۸۹}: از مجلس طبقات به سقوط باستیل و اعلامیه حقوق
\item \textbf{۱۷۸۹-۱۷۹۱}: تلاش برای سلطنت مشروطه — شکست به دلیل بی‌اعتمادی متقابل
\item \textbf{۱۷۹۲}: جنگ، سقوط سلطنت، تأسیس جمهوری
\item \textbf{۱۷۹۳}: بحران چندجانبه، ترور به‌عنوان «ضرورت»
\item \textbf{۱۷۹۴}: اوج ترور، سپس سقوط روبسپیر
\item \textbf{۱۷۹۵-۱۷۹۹}: دایرکتوار ناپایدار
\item \textbf{۱۷۹۹}: کودتای ناپلئون — پایان انقلاب؟
\end{itemize}

\textbf{پرسش باقی‌مانده}: آیا ناپلئون انقلاب را تثبیت کرد یا به آن خیانت کرد؟

\textbf{در فصل بعد}: دوره ناپلئونی (۱۷۹۹-۱۸۱۵)
\end{kholasebox}

% ══════════════════════════════════════════════════════════════════════════════
%                    پایان فصل ۳
% ══════════════════════════════════════════════════════════════════════════════			
					
					
					
					
					
					
\end{document}