% ╔══════════════════════════════════════════════════════════════════╗
% ║  فصل ۳: رویکردها، ساختارها و محدوده‌های نظارت                  ║
% ║  تحلیل مقایسه‌ای شش مدل + مدل ترکیبی پیشنهادی                 ║
% ║  ** قلب تحلیلی کتاب **                                         ║
% ╚══════════════════════════════════════════════════════════════════╝

% ---- صفحه‌ی آغازین فصل ----
\chapteropening{۳}
    {رویکردها، ساختارها و محدوده‌های نظارت}
    {MainGreen}
    {بدترین کار این است که مردم را وادار کنیم 
    چرخ را دوباره اختراع کنند، در حالی که تجربه‌ی 
    دیگران روی میز است. اما بدتر از آن، کپی 
    کورکورانه‌ی تجربه‌ی دیگران بدون درک 
    زمینه‌ی خودمان است.}
    {کوفی عنان، دبیرکل سابق سازمان ملل}

\chapter{رویکردها، ساختارها و محدوده‌های نظارت: 
تحلیل مقایسه‌ای}
\label{ch:approaches}
\minitoc

% ---- خلاصه‌ی اجرایی فصل ----
\begin{executivesummary}
این فصل \emphgreen{قلب تحلیلی کتاب} است. 
در اینجا شش مدل متمایز نظارت بین‌المللی بر 
گذار شناسایی، تشریح و مقایسه می‌شوند — 
از «نظارت انتخاباتی محدود» تا 
«مدیریت بین‌المللی مستقیم». هر مدل از نظر 
قوت‌ها، ضعف‌ها، نیروی انسانی، هزینه، 
تناسب با ایران و درس‌های تاریخی تحلیل 
می‌شود. سپس \emphgreen{مدل ششم (ترکیبی-تطبیقی)} 
به‌عنوان مدل پیشنهادی این کتاب معرفی می‌گردد: 
مدلی فازبندی‌شده که عناصر بهترین مدل‌ها را 
با ویژگی‌های خاص ایران ترکیب می‌کند.
\end{executivesummary}

% ============================================================
\section{طیف نظارت: از حداقل تا حداکثر}
\label{sec:monitoring-spectrum}
% ============================================================

پیش از بررسی تفصیلی هر مدل، مهم است که 
جایگاه آن‌ها را در یک طیف کلی ببینیم:

\begin{figure}[htbp]
    \centering
    \begin{tikzpicture}[
        scale=0.95,
        model/.style={
            draw, rounded corners=3pt,
            minimum height=1.8cm, minimum width=2.6cm,
            align=center, font=\footnotesize\bfseries,
            drop shadow={shadow xshift=0.5mm, 
            shadow yshift=-0.5mm, opacity=0.3}
        }
    ]
    
    % فلش طیف
    \draw[ultra thick, -{Stealth[length=4mm]}, 
        gray!60] (-1, -1.5) -- (15, -1.5);
    
    % برچسب‌های طیف
    \node[font=\small\bfseries, MainGreen] 
        at (0, -2.2) {حداقلی};
    \node[font=\small\bfseries, MainRed] 
        at (14, -2.2) {حداکثری};
    
    \node[font=\tiny, MediumGray] 
        at (0, -2.7) {تهاجم کم به حاکمیت};
    \node[font=\tiny, MediumGray] 
        at (14, -2.7) {تهاجم زیاد به حاکمیت};
    
    % مدل‌ها
    \node[model, fill=GreenBG, draw=MainGreen] 
        (m1) at (0.5, 0.5) {
        مدل ۱\\[2pt]
        \footnotesize نظارت\\انتخاباتی\\محدود
    };
    
    \node[model, fill=GreenBG, draw=MainGreen!70] 
        (m2) at (3.3, 0.5) {
        مدل ۲\\[2pt]
        \footnotesize نظارت\\مشورتی\\و فنی
    };
    
    \node[model, fill=BlueBG, draw=MainBlue] 
        (m3) at (6.1, 0.5) {
        مدل ۳\\[2pt]
        \footnotesize نظارت\\ساختاری\\و نهادی
    };
    
    \node[model, fill=OrangeBG, draw=MainOrange] 
        (m4) at (8.9, 0.5) {
        مدل ۴\\[2pt]
        \footnotesize نظارت\\اجرایی\\و تضمینی
    };
    
    \node[model, fill=RedBG, draw=MainRed] 
        (m5) at (11.7, 0.5) {
        مدل ۵\\[2pt]
        \footnotesize مدیریت\\بین‌المللی\\مستقیم
    };
    
    % مدل ۶ پیشنهادی
    \node[model, fill=PurpleBG, draw=MainPurple, 
        line width=1.5pt] 
        (m6) at (7.5, 3.2) {
        مدل ۶\\[2pt]
        \footnotesize ترکیبی\\تطبیقی\\(پیشنهادی)
    };
    
    % فلش‌های مدل ۶
    \draw[MainPurple, thick, dashed, ->] 
        (m6.south west) -- (m2.north east);
    \draw[MainPurple, thick, dashed, ->] 
        (m6.south) -- (m3.north);
    \draw[MainPurple, thick, dashed, ->] 
        (m6.south east) -- (m4.north west);
    
    % برچسب مدل ۶
    \node[font=\tiny\itshape, MainPurple, 
        text width=4cm, align=center] 
        at (7.5, 4.3) {
        ترکیب عناصر مدل‌های ۲، ۳ و ۴\\
        با فازبندی مرحله‌ای
    };
    
    \end{tikzpicture}
    \caption{طیف مدل‌های نظارت بین‌المللی 
    و جایگاه مدل ترکیبی پیشنهادی}
    \label{fig:monitoring-spectrum}
\end{figure}

\sectiondivider

% ============================================================
\section{مدل ۱: نظارت انتخاباتی محدود}
\label{sec:model1}
% ============================================================

\begin{definitionbox}{نظارت انتخاباتی محدود 
(\lr{Election-Only Monitoring})}
حضور ناظران بین‌المللی صرفاً در دوره‌ی 
انتخابات (معمولاً ۱ تا ۳ ماه) برای مشاهده‌ی 
فرایند رأی‌گیری، شمارش آرا و اعلام نتایج. 
بدون دخالت در طراحی نهادها، اصلاح قوانین 
یا نظارت بلندمدت.
\end{definitionbox}

\subsection{محدوده و مکانیزم‌ها}

\begin{itemize}[itemsep=4pt]
    \item \textbf{پیش از رأی‌گیری:} 
    بررسی فهرست رأی‌دهندگان، ثبت‌نام نامزدها، 
    دسترسی به رسانه، فضای مبارزاتی
    \item \textbf{روز رأی‌گیری:} 
    حضور در شعب، مشاهده‌ی فرایند، 
    گزارش تخلفات
    \item \textbf{پس از رأی‌گیری:} 
    نظارت بر شمارش، تأیید نتایج، 
    گزارش نهایی
\end{itemize}

\subsection{نهادهای مجری معمول}

\begin{table}[htbp]
    \centering
    \caption{نهادهای اصلی نظارت انتخاباتی 
    بین‌المللی}
    \label{tab:election-monitors}
    \tablefontsize
    \begin{tabularx}{\textwidth}{
        L{2.5cm} X C{1.5cm} C{2cm}
    }
        \toprule
        \headerrow
        \textbf{نهاد} & 
        \textbf{تخصص و سابقه} & 
        \textbf{تعداد مأموریت} &
        \textbf{اعتبار} \\
        \midrule
        
        \lr{OSCE/ODIHR} &
        معتبرترین نهاد نظارت انتخاباتی اروپا، 
        استانداردهای دقیق &
        ۴۰۰+ &
        \starrating{5} \\
        \altrow
        
        \lr{EU EOM} &
        مأموریت‌های نظارت انتخاباتی اتحادیه 
        اروپا در خارج از اروپا &
        ۱۵۰+ &
        \starrating{4} \\
        
        \lr{Carter Center} &
        نظارت انتخاباتی مستقل، 
        ۳۹ کشور &
        ۱۱۰+ &
        \starrating{5} \\
        \altrow
        
        \lr{Commonwealth} &
        نظارت در کشورهای مشترک‌المنافع &
        ۱۵۰+ &
        \starrating{3} \\
        
        \lr{African Union} &
        نظارت در آفریقا &
        ۱۰۰+ &
        \starrating{3} \\
        
        \bottomrule
    \end{tabularx}
\end{table}

\subsection{تحلیل قوت‌ها و ضعف‌ها}

\begin{table}[htbp]
    \centering
    \caption{تحلیل \lr{SWOT} مدل ۱: 
    نظارت انتخاباتی محدود}
    \label{tab:model1-swot}
    \begin{tabularx}{\textwidth}{
        C{0.5cm} X X
    }
        \toprule
        \headerrow
        & \textbf{مثبت} & \textbf{منفی} \\
        \midrule
        
        \rotatebox{90}{\footnotesize\bfseries درونی} &
        \cellgreen{
        \textbf{قوت‌ها:}\\
        • کم‌هزینه (\$۵-۲۰M)\\
        • کم‌تنش با حاکمیت\\
        • تجربه‌ی فراوان جهانی\\
        • استقرار سریع\\
        • استانداردهای شفاف
        } &
        \cellred{
        \textbf{ضعف‌ها:}\\
        • سطحی و ناکافی\\
        • فاقد نظارت ساختاری\\
        • قابل دور زدن\\
        • فقط روز انتخابات\\
        • بدون پی‌گیری
        } \\
        \altrow
        
        \rotatebox{90}{\footnotesize\bfseries بیرونی} &
        \cellgreen{
        \textbf{فرصت‌ها:}\\
        • نقطه‌ی شروع خوب\\
        • مشروعیت‌بخشی اولیه\\
        • پذیرش عمومی بالا
        } &
        \cellred{
        \textbf{تهدیدها:}\\
        • مشروعیت‌بخشی کاذب\\
        • توهم دموکراسی\\
        • مصادره توسط اقتدارگرایان
        } \\
        
        \bottomrule
    \end{tabularx}
\end{table}

\subsection{نیروی انسانی و هزینه}

\begin{table}[htbp]
    \centering
    \caption{نیازمندی‌های عملیاتی مدل ۱}
    \label{tab:model1-operations}
    \begin{tabularx}{\textwidth}{L{3.5cm} X}
        \toprule
        \headerrow
        \textbf{عنصر} & \textbf{تخمین} \\
        \midrule
        ناظران بلندمدت (\lr{LTO}) & 
        ۵۰-۱۰۰ نفر (۲ ماه قبل) \\
        \altrow
        ناظران کوتاه‌مدت (\lr{STO}) & 
        ۵۰۰-۲,۰۰۰ نفر (۱ هفته) \\
        تیم هسته‌ای (\lr{Core Team}) & 
        ۱۵-۳۰ نفر \\
        \altrow
        مدت استقرار & ۱-۳ ماه \\
        بودجه‌ی تخمینی & 
        \$۵-۲۰ میلیون \\
        \altrow
        نهادهای پشتیبان & 
        \lr{OSCE, EU, Carter Center} \\
        \bottomrule
    \end{tabularx}
\end{table}

\subsection{نمونه‌ی تاریخی و درس‌های آموخته}

\begin{casestudy}{انتخابات میانمار ۲۰۱۰}
در ۲۰۱۰ ارتش میانمار انتخاباتی برگزار کرد 
که ظاهراً «چندحزبی» بود. ناظران بین‌المللی 
محدودی حضور داشتند. حزب حامی ارتش با 
اکثریت قاطع «پیروز» شد. جامعه‌ی بین‌المللی 
نتوانست فرایند را رد کند زیرا نظارت کافی 
نداشت. نتیجه: مشروعیت‌بخشی به نظام نظامی 
برای ۵ سال دیگر.

\vspace{4pt}
\textbf{درس برای ایران:} 
\emphred{نظارت انتخاباتی بدون نظارت ساختاری، 
ابزار مشروعیت‌بخشی به اقتدارگرایی است.} 
اگر در ایران فقط به نظارت روز انتخابات 
بسنده شود — بدون نظارت بر فیلترینگ نامزدها، 
آزادی رسانه و دسترسی اپوزیسیون — نتیجه 
تکرار تجربه‌ی میانمار خواهد بود.
\end{casestudy}

\begin{keypoint}
\textbf{حکم نهایی درباره‌ی مدل ۱ برای ایران:}
\emphred{ناکافی و خطرناک} اگر به‌تنهایی 
به‌کار رود. می‌تواند \emphgreen{بخشی از 
یک مدل ترکیبی} باشد (فاز انتخاباتی)، 
اما هرگز نباید تمام برنامه‌ی نظارت باشد.
\end{keypoint}

\sectiondivider

% ============================================================
\section{مدل ۲: نظارت مشورتی و فنی}
\label{sec:model2}
% ============================================================

\begin{definitionbox}{نظارت مشورتی و فنی 
(\lr{Advisory \& Technical Monitoring})}
ارائه‌ی مشاوره‌ی تخصصی و کمک فنی 
به نهادهای دوره‌ی گذار در حوزه‌های مختلف 
(طراحی قانون انتخابات، ظرفیت‌سازی نهادی، 
آموزش قضات و...) بدون قدرت اجرایی یا 
حق وتو.
\end{definitionbox}

\subsection{محدوده و مکانیزم‌ها}

\begin{itemize}[itemsep=4pt]
    \item \textbf{مشاوره‌ی حقوقی:} 
    کمک به تدوین قانون اساسی، قانون انتخابات، 
    قانون احزاب و قانون رسانه
    \item \textbf{ظرفیت‌سازی نهادی:} 
    آموزش کارکنان کمیسیون انتخابات، قضات، 
    پلیس و کارمندان دولتی
    \item \textbf{انتقال دانش:} 
    به‌اشتراک‌گذاری تجربه‌ی کشورهای دیگر
    \item \textbf{ارزیابی و گزارش:} 
    تهیه‌ی گزارش‌های فنی درباره‌ی وضعیت 
    نهادها و پیشنهاد اصلاحات
\end{itemize}

\subsection{نهادهای مجری معمول}

\begin{table}[htbp]
    \centering
    \caption{نهادهای اصلی نظارت مشورتی}
    \label{tab:advisory-bodies}
    \tablefontsize
    \begin{tabularx}{\textwidth}{
        L{2.5cm} X L{3cm}
    }
        \toprule
        \headerrow
        \textbf{نهاد} & 
        \textbf{تخصص} & 
        \textbf{نمونه‌ی فعالیت} \\
        \midrule
        
        \lr{UNDP} &
        ظرفیت‌سازی حکمرانی، 
        توسعه‌ی نهادی، 
        مدیریت انتخابات &
        ۱۷۰ کشور \\
        \altrow
        
        \lr{International IDEA} &
        طراحی نظام انتخاباتی، 
        دموکراسی‌سنجی &
        تونس، مصر، نپال \\
        
        \lr{Venice Commission}\newline
        (شورای اروپا) &
        حقوق اساسی، 
        قانون اساسی تطبیقی &
        ۶۰+ کشور \\
        \altrow
        
        \lr{IFES} &
        زیرساخت انتخاباتی، 
        فناوری رأی‌گیری &
        ۱۴۵ کشور \\
        
        \lr{ICTJ} &
        عدالت انتقالی، 
        کمیسیون حقیقت &
        ۳۰+ کشور \\
        
        \bottomrule
    \end{tabularx}
\end{table}

\subsection{تحلیل قوت‌ها و ضعف‌ها}

\begin{table}[htbp]
    \centering
    \caption{تحلیل \lr{SWOT} مدل ۲: 
    نظارت مشورتی و فنی}
    \label{tab:model2-swot}
    \begin{tabularx}{\textwidth}{
        C{0.5cm} X X
    }
        \toprule
        \headerrow
        & \textbf{مثبت} & \textbf{منفی} \\
        \midrule
        
        \rotatebox{90}{\footnotesize\bfseries درونی} &
        \cellgreen{
        \textbf{قوت‌ها:}\\
        • ظرفیت‌سازی واقعی\\
        • انتقال دانش بین‌المللی\\
        • پذیرش بالای داخلی\\
        • حفظ مالکیت ملی\\
        • هزینه‌ی متوسط
        } &
        \cellred{
        \textbf{ضعف‌ها:}\\
        • فاقد قدرت اجرایی\\
        • وابسته به حسن نیت\\
        • توصیه‌ها قابل نادیده‌گیری\\
        • سرعت کم\\
        • بدون ضمانت اجرا
        } \\
        \altrow
        
        \rotatebox{90}{\footnotesize\bfseries بیرونی} &
        \cellgreen{
        \textbf{فرصت‌ها:}\\
        • ساخت ظرفیت بلندمدت\\
        • ایجاد شبکه‌ی متخصصان\\
        • مقبولیت بین‌المللی
        } &
        \cellred{
        \textbf{تهدیدها:}\\
        • نادیده‌گرفتن توسط بازیگران\\
        قدرتمند داخلی\\
        • سوءاستفاده‌ی نمایشی\\
        • فرسایش در طول زمان
        } \\
        
        \bottomrule
    \end{tabularx}
\end{table}

\subsection{نمونه‌ی تاریخی: تونس ۲۰۱۱-۲۰۱۴}

\begin{casestudy}{نظارت مشورتی در تونس}
پس از انقلاب یاسمین (۲۰۱۱)، تونس از 
مشاوره‌ی فنی گسترده‌ی نهادهای بین‌المللی 
بهره برد: \lr{UNDP} در طراحی سیستم انتخاباتی، 
\lr{Venice Commission} در تدوین قانون اساسی، 
\lr{International IDEA} در ظرفیت‌سازی احزاب. 
نتیجه: قانون اساسی ۲۰۱۴ یکی از مترقی‌ترین 
قوانین اساسی جهان عرب شد.

\vspace{4pt}
\textbf{اما...} بدون ضمانت اجرایی، رئیس‌جمهور 
قیس سعید در ۲۰۲۱ قانون اساسی را تعلیق کرد، 
مجلس را منحل نمود و عملاً به اقتدارگرایی 
بازگشت. ناظران مشورتی هیچ ابزاری برای 
جلوگیری نداشتند.

\vspace{4pt}
\textbf{درس برای ایران:}
\emphblue{مشاوره بدون ضمانت اجرا، در برابر 
بازیگران فرصت‌طلب آسیب‌پذیر است.} مدل 
مشورتی باید با مکانیزم‌های تضمینی ترکیب شود.
\end{casestudy}

\sectiondivider

% ============================================================
\section{مدل ۳: نظارت ساختاری و نهادی}
\label{sec:model3}
% ============================================================

\begin{definitionbox}{نظارت ساختاری و نهادی 
(\lr{Structural \& Institutional Oversight})}
نظارت بلندمدت و عمیق بر فرایند اصلاح 
ساختار نهادهای کلیدی کشور — از قانون اساسی 
و قوه‌ی قضاییه تا بخش امنیتی و نظام اقتصادی — 
با معیارهای مشخص (\lr{benchmarks}) 
و مکانیزم‌های مشوق/تنبیه.
\end{definitionbox}

\subsection{محدوده و مکانیزم‌ها}

\begin{itemize}[itemsep=4pt]
    \item \textbf{نظارت بر قانون اساسی:} 
    بررسی انطباق قانون اساسی جدید با 
    استانداردهای دموکراتیک بین‌المللی
    \item \textbf{اصلاح بخش امنیتی (\lr{SSR}):}
    نظارت بر بازسازی ارتش، پلیس و 
    نهادهای اطلاعاتی
    \item \textbf{اصلاح قضایی:} 
    استقلال قوه‌ی قضاییه، آموزش قضات، 
    بازنگری قوانین
    \item \textbf{معیارسنجی (\lr{Benchmarking}):}
    تعیین شاخص‌های مشخص و اندازه‌گیری 
    دوره‌ای پیشرفت
    \item \textbf{مشوق/تنبیه (\lr{Conditionality}):}
    پیوند پیشرفت در اصلاحات با مزایای 
    بین‌المللی (رفع تحریم، کمک مالی، عضویت)
\end{itemize}

\subsection{نمونه‌ی تاریخی: 
اروپای شرقی و مسیر الحاق به اتحادیه‌ی اروپا}

\begin{casestudy}{نظارت ساختاری اتحادیه‌ی اروپا 
بر اروپای شرقی (۱۹۹۰-۲۰۰۴)}
پس از فروپاشی شوروی، کشورهای اروپای شرقی 
(لهستان، چک، مجارستان، استونی و...) برای 
الحاق به اتحادیه‌ی اروپا باید «معیارهای 
کپنهاگ» را برآورده می‌کردند:

\begin{enumerate}[itemsep=2pt, font=\small]
    \item ثبات نهادهای تضمین‌کننده‌ی دموکراسی
    \item حاکمیت قانون
    \item حقوق بشر و حمایت از اقلیت‌ها
    \item اقتصاد بازار کارآمد
    \item ظرفیت پذیرش تعهدات عضویت
\end{enumerate}

\lr{EU} هر سال «گزارش پیشرفت» هر کشور را 
منتشر می‌کرد. پیشرفت → نزدیکی به عضویت؛ 
عقب‌گرد → تعویق. 
\emphgreen{این مشوق (عضویت در EU) 
قوی‌ترین اهرم نظارت ساختاری در تاریخ بود.}

\vspace{4pt}
\textbf{نتیجه:} ۱۰ کشور اروپای شرقی در 
۲۰۰۴ و ۲۰۰۷ به EU پیوستند. 
اکثر آن‌ها دموکراسی‌های پایدار شدند 
(البته مجارستان و لهستان اخیراً پس‌رفت داشته‌اند).

\vspace{4pt}
\textbf{درس برای ایران:}
\emphblue{مشوق خارجی (عضویت در EU) 
کلید موفقیت این مدل بود. ایران چنین مشوقی 
ندارد.} باید جایگزین‌هایی طراحی شود: 
رفع تحریم‌ها، بسته‌ی حمایت اقتصادی، 
و عضویت در سازمان‌های بین‌المللی.
\end{casestudy}

\subsection{تحلیل مقایسه‌ای: قوت‌ها، ضعف‌ها 
و نیازمندی‌ها}

\begin{table}[htbp]
    \centering
    \caption{مشخصات عملیاتی مدل ۳}
    \label{tab:model3-specs}
    \begin{tabularx}{\textwidth}{L{3.5cm} X}
        \toprule
        \headerrow
        \textbf{عنصر} & \textbf{تخمین/توضیح} \\
        \midrule
        نیروی انسانی & 
        ۵۰۰-۲,۰۰۰ کارشناس بخشی \\
        \altrow
        مدت زمان & ۳-۷ سال (حداقل) \\
        بودجه‌ی تخمینی & 
        \$۵۰۰M - \$۲B \\
        \altrow
        پیش‌نیاز اصلی & 
        اجماع داخلی + مشوق خارجی قوی \\
        قوت اصلی & 
        عمق بالا، اصلاح واقعی نهادها \\
        \altrow
        ضعف اصلی & 
        زمان‌بر، نیاز به مشوق 
        (ایران مشوق EU ندارد) \\
        تناسب با ایران & 
        \cellblue{\textbf{بالا — اما نیاز به 
        طراحی مشوق جایگزین}} \\
        \bottomrule
    \end{tabularx}
\end{table}

\sectiondivider

% ============================================================
\section{مدل ۴: نظارت اجرایی و تضمینی}
\label{sec:model4}
% ============================================================

\begin{definitionbox}{نظارت اجرایی و تضمینی 
(\lr{Executive \& Guarantee-Based Oversight})}
حضور بین‌المللی با قدرت اجرایی محدود 
در حوزه‌های خاص (مانند امنیت، مالیه یا 
انتخابات) و مکانیزم‌های تضمین‌کننده 
برای جلوگیری از بازگشت اقتدارگرایانه. 
شامل حق وتو یا تأیید بر تصمیمات 
کلیدی در دوره‌ی معین.
\end{definitionbox}

\subsection{محدوده و مکانیزم‌ها}

\begin{itemize}[itemsep=4pt]
    \item \textbf{قدرت اجرایی محدود:} 
    مثلاً تأیید فرمانده‌ی ارتش جدید، 
    نظارت بر بودجه‌ی دفاعی، 
    تأیید قضات ارشد
    
    \item \textbf{مکانیزم قفل‌کننده 
    (\lr{Locking Mechanism}):}
    ضمانت‌هایی که تغییر آن‌ها 
    نیاز به تأیید بین‌المللی دارد 
    (مثلاً تغییر قانون اساسی)
    
    \item \textbf{نیروهای بین‌المللی:} 
    حضور نظامی/انتظامی 
    بین‌المللی برای تضمین امنیت 
    (مشابه \lr{KFOR} در کوزوو)
    
    \item \textbf{نظارت مالی:}
    مدیریت مشترک درآمد نفت 
    یا صندوق امانی بین‌المللی
\end{itemize}

\subsection{نمونه‌های تاریخی}

\begin{table}[htbp]
    \centering
    \caption{نمونه‌های نظارت اجرایی در تاریخ}
    \label{tab:model4-examples}
    \tablefontsize
    \begin{tabularx}{\textwidth}{
        L{2cm} C{1.5cm} X C{1.5cm}
    }
        \toprule
        \headerrow
        \textbf{مورد} & 
        \textbf{نهاد} & 
        \textbf{محدوده‌ی اختیار} &
        \textbf{نتیجه} \\
        \midrule
        
        تیمور شرقی &
        \lr{UNTAET} &
        مدیریت کامل اجرایی → انتقال 
        تدریجی به دولت ملی &
        \cellgreen{نسبتاً موفق} \\
        \altrow
        
        کوزوو &
        \lr{UNMIK} &
        مدیریت اجرایی + \lr{KFOR} 
        (نیروی نظامی) &
        \cellorange{ناتمام} \\
        
        بوسنی &
        \lr{OHR} &
        نماینده‌ی عالی با قدرت 
        «بان» (حکم اجرایی) &
        \cellorange{وابستگی} \\
        \altrow
        
        کامبوج &
        \lr{UNTAC} &
        مدیریت مشترک + نظارت 
        انتخاباتی &
        \cellgreen{موفق اولیه} \\
        
        \bottomrule
    \end{tabularx}
\end{table}

\begin{lessonlearned}
\textbf{از تجربه‌ی تیمور شرقی (\lr{UNTAET}, 
۱۹۹۹-۲۰۰۲):}
\lr{UNTAET} موفق‌ترین نمونه‌ی نظارت اجرایی 
محسوب می‌شود. دلایل موفقیت:
\begin{enumerate}[itemsep=2pt, font=\small]
    \item جمعیت کم (۱ میلیون) → قابل مدیریت
    \item حمایت قاطع شورای امنیت
    \item همکاری مردم محلی
    \item \textbf{زمان‌بندی شفاف خروج}
\end{enumerate}

\textbf{اما:} تیمور ۱ میلیون نفر جمعیت داشت. 
ایران ۸۵ میلیون. مقیاس‌پذیری مدل \lr{UNTAET} 
برای ایران یک چالش بنیادین است. 
همچنین ایران — بر خلاف تیمور — 
یک قدرت منطقه‌ای با غرور ملی قوی است. 
\emphblue{نظارت اجرایی بدون زمان‌بندی 
خروج شفاف، به «اشغال» تبدیل می‌شود.}
\end{lessonlearned}

\begin{warningbox}
\textbf{هشدار درباره‌ی مدل ۴ برای ایران:}
هرگونه حضور «اجرایی» بین‌المللی در ایران 
با حساسیت شدید ملی‌گرایانه مواجه خواهد شد. 
ایرانیان حافظه‌ی تاریخی عمیقی از مداخله‌ی 
خارجی دارند (کودتای ۲۸ مرداد ۱۳۳۲، 
قرارداد ۱۹۱۹ وثوق‌الدوله). \emphred{هر 
عنصر اجرایی باید با دقت فوق‌العاده، شفافیت 
کامل و زمان‌بندی مشخص طراحی شود.}
\end{warningbox}

\sectiondivider

% ============================================================
\section{مدل ۵: مدیریت بین‌المللی مستقیم}
\label{sec:model5}
% ============================================================

\begin{definitionbox}{مدیریت بین‌المللی مستقیم 
(\lr{Full International Administration})}
کنترل کامل یا شبه‌کامل حکمرانی توسط 
نهاد بین‌المللی (معمولاً سازمان ملل یا 
ائتلاف بین‌المللی) برای مدت معین. 
تمام تصمیمات اجرایی، قضایی و تقنینی 
توسط مقام بین‌المللی اتخاذ می‌شود.
\end{definitionbox}

\begin{warningbox}
\textbf{موضع قاطع این کتاب:}
\emphred{مدل ۵ برای ایران نه ممکن است، 
نه مطلوب و نه مشروع.} بررسی این مدل 
صرفاً به‌منظور کامل بودن تحلیل و 
نشان دادن خطرات آن انجام می‌شود.
\end{warningbox}

\subsection{نمونه‌ی تاریخی: عراق (CPA, ۲۰۰۳-۲۰۰۴)}

\begin{casestudy}{مدیریت مستقیم آمریکا در عراق}
پس از سقوط صدام حسین (آوریل ۲۰۰۳)، 
\org{مرجع موقت ائتلاف}%
{Coalition Provisional Authority, CPA} 
به ریاست \person{پل برمر}{L. Paul Bremer} 
مدیریت کامل عراق را بر عهده گرفت.

\vspace{4pt}
\textbf{اقدامات فاجعه‌بار:}
\begin{enumerate}[itemsep=2pt, font=\small]
    \item \textbf{دی‌بعثی‌سازی:} اخراج 
    ۵۰۰,۰۰۰ نفر → بیکاری و خشم
    \item \textbf{انحلال ارتش:} ۳۵۰,۰۰۰ 
    سرباز مسلح بیکار → شورش مسلحانه
    \item \textbf{تصمیم‌گیری بدون مشورت:} 
    ناآشنایی با فرهنگ و جامعه‌ی عراق
    \item \textbf{فساد گسترده:} میلیاردها 
    دلار هدر رفت
\end{enumerate}

\vspace{4pt}
\textbf{نتیجه:} جنگ داخلی، ظهور داعش، 
بیش از ۵۰۰,۰۰۰ کشته، ۲+ تریلیون 
دلار هزینه، و عراقی که هنوز بی‌ثبات است.

\vspace{4pt}
\textbf{درس برای ایران:}
\emphred{مدیریت مستقیم بین‌المللی بر 
کشوری با ۸۵ میلیون جمعیت، تمدن ۳۰۰۰ 
ساله و هویت ملی قوی، فاجعه‌بار خواهد بود. 
این گزینه باید از ابتدا از روی میز 
برداشته شود.}
\end{casestudy}

\sectiondivider

% ============================================================
\section{مدل ۶ (پیشنهادی): مدل ترکیبی-تطبیقی ایران}
\label{sec:model6}
% ============================================================

\begin{keypoint}
\textbf{ایده‌ی محوری مدل ۶:}
هیچ مدل واحدی جواب نمی‌دهد. 
\emphpurple{مدل ترکیبی-تطبیقی} عناصر 
بهترین مدل‌ها (۲، ۳ و ۴) را ترکیب می‌کند 
و آن‌ها را در سه فاز زمانی سازمان‌دهی 
می‌کند — از نظارت سنگین‌تر در آغاز 
به نظارت سبک‌تر در انتها.
\end{keypoint}

\subsection{فازبندی مدل ترکیبی}

\begin{figure}[htbp]
    \centering
    \begin{tikzpicture}[
        phase/.style={
            draw, rounded corners=4pt,
            minimum height=3cm, minimum width=4.2cm,
            align=center, font=\small,
            drop shadow={opacity=0.2}
        }
    ]
    
    % فاز ۱
    \node[phase, fill=RedBG, draw=MainRed] 
        (p1) at (0, 0) {
        \textbf{\large فاز ۱}\\[4pt]
        \textbf{تثبیت}\\[2pt]
        ماه ۱ تا ۶\\[4pt]
        {\footnotesize 
        نظارت اجرایی (مدل ۴)\\
        بر امنیت + حقوق بشر\\
        + انتخابات اولیه}
    };
    
    % فاز ۲
    \node[phase, fill=OrangeBG, draw=MainOrange] 
        (p2) at (5.5, 0) {
        \textbf{\large فاز ۲}\\[4pt]
        \textbf{نهادسازی}\\[2pt]
        ماه ۶ تا ۲۴\\[4pt]
        {\footnotesize 
        نظارت ساختاری (مدل ۳)\\
        بر قانون اساسی + SSR\\
        + عدالت انتقالی}
    };
    
    % فاز ۳
    \node[phase, fill=GreenBG, draw=MainGreen] 
        (p3) at (11, 0) {
        \textbf{\large فاز ۳}\\[4pt]
        \textbf{تحکیم}\\[2pt]
        ماه ۲۴ تا ۶۰\\[4pt]
        {\footnotesize 
        نظارت مشورتی (مدل ۲)\\
        بر عملکرد نهادها\\
        + کاهش تدریجی حضور}
    };
    
    % فلش‌ها
    \draw[-{Stealth}, ultra thick, MainOrange] 
        (p1) -- (p2) 
        node[midway, above, font=\tiny\bfseries] 
        {ارزیابی};
    \draw[-{Stealth}, ultra thick, MainGreen] 
        (p2) -- (p3) 
        node[midway, above, font=\tiny\bfseries] 
        {ارزیابی};
    
    % شدت نظارت
    \draw[ultra thick, MainRed!70, -{Stealth}] 
        (0, -2.5) -- (11, -2.5)
        node[midway, below, font=\small\bfseries] 
        {کاهش تدریجی عمق نظارت};
    
    % مالکیت ملی
    \draw[ultra thick, MainGreen!70, -{Stealth}] 
        (0, -3.3) -- (11, -3.3)
        node[midway, below, font=\small\bfseries] 
        {افزایش تدریجی مالکیت ملی};
    
    \end{tikzpicture}
    \caption{فازبندی مدل ترکیبی-تطبیقی 
    پیشنهادی برای ایران}
    \label{fig:hybrid-model}
\end{figure}

\subsection{جزئیات هر فاز}

\subsubsection{فاز ۱: تثبیت (ماه ۱ تا ۶)}

\begin{table}[htbp]
    \centering
    \caption{جزئیات عملیاتی فاز ۱ (تثبیت)}
    \label{tab:phase1-details}
    \tablefontsize
    \begin{tabularx}{\textwidth}{
        L{3cm} X
    }
        \toprule
        \headerrow
        \textbf{عنصر} & \textbf{توضیح} \\
        \midrule
        
        هدف اصلی &
        جلوگیری از خلأ امنیتی، ثبت نقض 
        حقوق بشر، آماده‌سازی انتخابات اولیه \\
        \altrow
        
        مدل غالب &
        مدل ۴ (نظارت اجرایی) + عناصر مدل ۱ 
        (نظارت انتخاباتی) \\
        
        اختیارات اجرایی &
        تأیید فرماندهان امنیتی، نظارت بر بودجه‌ی 
        دفاعی، حفاظت از زیرساخت‌های حیاتی 
        (نفت، هسته‌ای) \\
        \altrow
        
        نیروی انسانی &
        ۳,۰۰۰-۵,۰۰۰ بین‌المللی + 
        ۱۰,۰۰۰+ ناظر ایرانی \\
        
        بودجه‌ی تخمینی &
        \$۵۰۰M - \$۱B \\
        \altrow
        
        اقدامات کلیدی &
        آزادی زندانیان سیاسی، بازگشایی 
        رسانه‌ها، اعلام تقویم انتخاباتی، 
        ایجاد کمیسیون انتخابات مستقل \\
        
        نقطه‌ی عطف خروج &
        برگزاری موفق انتخابات اولیه 
        (رفراندوم یا مجلس مؤسسان) \\
        
        \bottomrule
    \end{tabularx}
\end{table}

\subsubsection{فاز ۲: نهادسازی (ماه ۶ تا ۲۴)}

\begin{table}[htbp]
    \centering
    \caption{جزئیات عملیاتی فاز ۲ (نهادسازی)}
    \label{tab:phase2-details}
    \tablefontsize
    \begin{tabularx}{\textwidth}{
        L{3cm} X
    }
        \toprule
        \headerrow
        \textbf{عنصر} & \textbf{توضیح} \\
        \midrule
        
        هدف اصلی &
        تدوین قانون اساسی، اصلاح بخش امنیتی، 
        عدالت انتقالی، انتخابات پارلمانی \\
        \altrow
        
        مدل غالب &
        مدل ۳ (نظارت ساختاری) + عناصر مدل ۲ 
        (مشاوره‌ی فنی) \\
        
        اختیارات &
        مشاوره با قدرت نفوذ: توصیه‌های 
        \lr{Venice Commission} + معیارسنجی + 
        مشوق‌های مالی (رفع تحریم مرحله‌ای) \\
        \altrow
        
        نیروی انسانی &
        ۱,۵۰۰-۳,۰۰۰ کارشناس بخشی \\
        
        بودجه‌ی تخمینی &
        \$۸۰۰M - \$۱.۵B \\
        \altrow
        
        اقدامات کلیدی &
        رفراندوم قانون اساسی، تأسیس دادگاه 
        قانون اساسی، آغاز \lr{SSR}، 
        تأسیس کمیسیون حقیقت و آشتی \\
        
        نقطه‌ی عطف خروج &
        برگزاری انتخابات پارلمانی و 
        ریاست‌جمهوری آزاد \\
        
        \bottomrule
    \end{tabularx}
\end{table}

\subsubsection{فاز ۳: تحکیم (ماه ۲۴ تا ۶۰)}

\begin{table}[htbp]
    \centering
    \caption{جزئیات عملیاتی فاز ۳ (تحکیم)}
    \label{tab:phase3-details}
    \tablefontsize
    \begin{tabularx}{\textwidth}{
        L{3cm} X
    }
        \toprule
        \headerrow
        \textbf{عنصر} & \textbf{توضیح} \\
        \midrule
        
        هدف اصلی &
        نهادینه‌سازی دموکراسی، 
        انتقال کامل مسئولیت به نهادهای ملی، 
        کاهش تدریجی حضور بین‌المللی \\
        \altrow
        
        مدل غالب &
        مدل ۲ (نظارت مشورتی) + 
        ارزیابی‌های دوره‌ای مستقل \\
        
        اختیارات &
        مشاوره‌ی درخواست‌محور، ارزیابی 
        سالانه، گزارش‌دهی بین‌المللی \\
        \altrow
        
        نیروی انسانی &
        ۲۰۰-۵۰۰ مشاور بلندمدت \\
        
        بودجه‌ی تخمینی &
        \$۲۰۰-۵۰۰M (کل فاز) \\
        \altrow
        
        اقدامات کلیدی &
        دومین دوره‌ی انتخابات (اولین انتقال 
        مسالمت‌آمیز قدرت)، تکمیل \lr{SSR}، 
        جمع‌بندی عدالت انتقالی \\
        
        نقطه‌ی عطف خروج &
        اولین انتقال مسالمت‌آمیز قدرت + 
        شاخص‌های تحکیم محقق شده \\
        
        \bottomrule
    \end{tabularx}
\end{table}

\subsection{مکانیزم بازخورد و تنظیم مسیر}

\begin{recommendation}
\textbf{اصل انطباق‌پذیری:}
مدل ترکیبی نباید خشک و از‌پیش‌تعیین‌شده 
اجرا شود. در پایان هر فاز (و حتی درون هر فاز) 
باید ارزیابی مستقل انجام شود و بر اساس 
شرایط واقعی تنظیمات لازم اعمال گردد. 
سه سناریوی ممکن در هر نقطه‌ی ارزیابی:

\begin{enumerate}[itemsep=3pt]
    \item \textcolor{MainGreen}{\textbf{پیشرفت مطلوب}} 
    → انتقال به فاز بعد طبق برنامه
    \item \textcolor{MainOrange}{\textbf{پیشرفت جزئی}} 
    → تمدید فاز فعلی با اصلاحات
    \item \textcolor{MainRed}{\textbf{بحران یا بازگشت}} 
    → تقویت عناصر اجرایی و بازگشت 
    موقت به فاز قبل
\end{enumerate}
\end{recommendation}

\begin{figure}[htbp]
    \centering
    \begin{tikzpicture}[
        node distance=2cm,
        decision/.style={
            diamond, draw=MainPurple, fill=PurpleBG,
            text width=2.2cm, align=center, 
            inner sep=1pt, font=\tiny\bfseries,
            aspect=1.5
        },
        process/.style={
            rectangle, draw, rounded corners=3pt,
            minimum height=0.8cm, minimum width=2.5cm,
            align=center, font=\tiny\bfseries
        },
        good/.style={process, fill=GreenBG, draw=MainGreen},
        mid/.style={process, fill=OrangeBG, draw=MainOrange},
        bad/.style={process, fill=RedBG, draw=MainRed},
        arr/.style={-{Stealth[length=2mm]}, thick}
    ]
    
    % شروع
    \node[process, fill=BlueBG, draw=MainBlue] 
        (start) {پایان فاز N};
    
    % ارزیابی
    \node[decision, below=1.5cm of start] 
        (eval) {ارزیابی\\مستقل};
    
    % سه مسیر
    \node[good, below left=1.5cm and 2cm of eval] 
        (next) {انتقال به\\فاز N+1};
    \node[mid, below=1.5cm of eval] 
        (extend) {تمدید فاز N\\با اصلاحات};
    \node[bad, below right=1.5cm and 2cm of eval] 
        (back) {بازگشت به\\فاز N-1};
    
    % فلش‌ها
    \draw[arr, MainGreen] (eval) -- (next)
        node[midway, above left, font=\tiny] 
        {شاخص‌ها محقق};
    \draw[arr, MainOrange] (eval) -- (extend)
        node[midway, right, font=\tiny] 
        {پیشرفت جزئی};
    \draw[arr, MainRed] (eval) -- (back)
        node[midway, above right, font=\tiny] 
        {بحران/بازگشت};
    
    \draw[arr] (start) -- (eval);
    
    \end{tikzpicture}
    \caption{مکانیزم بازخورد و تنظیم مسیر 
    در مدل ترکیبی-تطبیقی}
    \label{fig:feedback-mechanism}
\end{figure}

\subsection{اصول بنیادین مدل ترکیبی}

مدل ششم بر هفت اصل بنیادین استوار است 
که باید در تمام فازها رعایت شوند:

\begin{table}[htbp]
    \centering
    \caption{هفت اصل بنیادین مدل ترکیبی-تطبیقی}
    \label{tab:model6-principles}
    \begin{tabularx}{\textwidth}{
        C{0.6cm} L{3cm} X
    }
        \toprule
        \headerrow
        \textbf{\#} & 
        \textbf{اصل} & 
        \textbf{معنای عملی} \\
        \midrule
        
        ۱ &
        \textbf{مالکیت ملی}
        \newline\lr{\tiny National Ownership} &
        ایرانیان تصمیم‌گیرنده‌ی نهایی هستند. 
        نقش بین‌المللی: همراهی و نظارت، 
        نه مدیریت \\
        \altrow
        
        ۲ &
        \textbf{فازبندی تدریجی}
        \newline\lr{\tiny Phased Approach} &
        شروع با نظارت عمیق‌تر، کاهش تدریجی 
        تا استقلال کامل \\
        
        ۳ &
        \textbf{انطباق‌پذیری}
        \newline\lr{\tiny Adaptability} &
        مدل نباید خشک باشد؛ بازنگری مستمر 
        بر اساس شرایط واقعی \\
        \altrow
        
        ۴ &
        \textbf{فراگیری}
        \newline\lr{\tiny Inclusivity} &
        همه‌ی گروه‌ها (زنان، اقوام، جوانان، 
        مذهبیون، سکولارها) باید نمایندگی شوند \\
        
        ۵ &
        \textbf{شفافیت}
        \newline\lr{\tiny Transparency} &
        تمام تصمیمات، بودجه‌ها و ارزیابی‌ها 
        باید عمومی باشند \\
        \altrow
        
        ۶ &
        \textbf{پاسخگویی دوسویه}
        \newline\lr{\tiny Mutual Accountability} &
        هم نهادهای ایرانی و هم نهادهای 
        بین‌المللی باید پاسخگو باشند \\
        
        ۷ &
        \textbf{خروج شفاف}
        \newline\lr{\tiny Clear Exit} &
        زمان‌بندی و شرایط خروج مأموریت 
        بین‌المللی از ابتدا تعریف شود \\
        
        \bottomrule
    \end{tabularx}
\end{table}

\sectiondivider

% ============================================================
\section{جدول مقایسه‌ای کلان شش مدل}
\label{sec:grand-comparison}
% ============================================================

جدول زیر خلاصه‌ی مقایسه‌ای شش مدل را 
ارائه می‌دهد. به دلیل گستردگی، این جدول 
در صفحه‌ی افقی چاپ شده است.

% ---- جدول بزرگ در صفحه‌ی افقی ----
\begin{landscape}
\begin{table}[htbp]
    \centering
    \caption{جدول مقایسه‌ای کلان شش مدل 
    نظارت بین‌المللی}
    \label{tab:grand-comparison}
    \bigtablefontsize
    \setlength{\tabcolsep}{4pt}
    \begin{tabularx}{\linewidth}{
        L{2.5cm}
        C{2cm} C{2cm} C{2cm} 
        C{2cm} C{2cm} C{2.5cm}
    }
        \toprule
        \headerrow
        \textbf{معیار} & 
        \rotsmall{\textbf{مدل ۱: انتخاباتی}} &
        \rotsmall{\textbf{مدل ۲: مشورتی}} &
        \rotsmall{\textbf{مدل ۳: ساختاری}} &
        \rotsmall{\textbf{مدل ۴: اجرایی}} &
        \rotsmall{\textbf{مدل ۵: مدیریت مستقیم}} &
        \rotsmall{\textbf{مدل ۶: ترکیبی (پیشنهادی)}} \\
        \midrule
        
        عمق نظارت & 
        \rating{1} & \rating{2} & \rating{3} & 
        \rating{4} & \rating{5} & \rating{4} \\
        \altrow
        
        پذیرش داخلی & 
        \rating{5} & \rating{4} & \rating{3} & 
        \rating{2} & \rating{1} & \rating{3} \\
        
        اثربخشی واقعی & 
        \rating{1} & \rating{2} & \rating{4} & 
        \rating{4} & \rating{3} & \rating{5} \\
        \altrow
        
        ریسک شکست & 
        \cellred{\rating{5}} & 
        \cellorange{\rating{4}} & 
        \cellgreen{\rating{2}} & 
        \cellorange{\rating{3}} & 
        \cellred{\rating{5}} & 
        \cellgreen{\rating{2}} \\
        
        هزینه & 
        \$۵-۲۰M & \$۵۰-۲۰۰M & 
        \$۵۰۰M-۲B & \$۱-۳B & 
        \$۳-۱۰B+ & \$۲.۵-۵B \\
        \altrow
        
        نیروی انسانی & 
        ۲۰۰-۲K & ۵۰-۱۵۰ & 
        ۵۰۰-۲K & ۲-۱۰K & 
        ۱۰-۵۰K & ۶-۱۲K \\
        
        مدت زمان & 
        ۱-۳ ماه & ۱-۳ سال & 
        ۳-۷ سال & ۲-۵ سال & 
        ۳-۱۰+ سال & ۵ سال (فازی) \\
        \altrow
        
        نمونه‌ی موفق & 
        غنا ۲۰۱۲ & تونس ۲۰۱۱ & 
        لهستان ۱۹۹۰ & تیمور ۱۹۹۹ & 
        --- & (پیشنهادی) \\
        
        نمونه‌ی شکست & 
        میانمار ۲۰۱۰ & تونس ۲۰۲۱ & 
        مجارستان ۲۰۱۰+ & کوزوو (وابستگی) & 
        عراق ۲۰۰۳ & --- \\
        \altrow
        
        \textbf{تناسب با ایران} & 
        \cellred{ناکافی} & 
        \cellorange{جزئی} & 
        \cellblue{بالا (بخشی)} & 
        \cellorange{پرریسک} & 
        \cellred{نامناسب} & 
        \cellcolor{PurpleBG}\textbf{بهینه} \\
        
        \bottomrule
    \end{tabularx}
    
    \vspace{6pt}
    {\footnotesize
    \rating{1} = خیلی کم \hspace{0.5cm}
    \rating{2} = کم \hspace{0.5cm}
    \rating{3} = متوسط \hspace{0.5cm}
    \rating{4} = بالا \hspace{0.5cm}
    \rating{5} = خیلی بالا
    }
\end{table}
\end{landscape}

\sectiondivider

% ============================================================
\section{نقش نظارت شهروندی و مکمل‌های 
غیررسمی}
\label{sec:citizen-monitoring}
% ============================================================

در کنار نظارت رسمی بین‌المللی، نظارت 
شهروندی و مکمل‌های غیررسمی نقش حیاتی 
دارند — به‌ویژه در کشوری مانند ایران که 
جامعه‌ی مدنی فعال و جمعیت جوان و 
آشنا به فناوری دارد.

\subsection{انواع نظارت شهروندی}

\begin{table}[htbp]
    \centering
    \caption{انواع نظارت شهروندی و 
    ابزارهای مرتبط}
    \label{tab:citizen-monitoring}
    \tablefontsize
    \begin{tabularx}{\textwidth}{
        L{2.5cm} X L{3cm}
    }
        \toprule
        \headerrow
        \textbf{نوع} & 
        \textbf{توضیح} & 
        \textbf{ابزار/نمونه} \\
        \midrule
        
        ناظران داخلی انتخابات &
        شهروندان آموزش‌دیده که در شعب 
        رأی‌گیری حضور دارند &
        \lr{ISFED} (گرجستان)، 
        \lr{NAMFREL} (فیلیپین) \\
        \altrow
        
        روزنامه‌نگاری شهروندی &
        مستندسازی رویدادها توسط 
        شهروندان عادی &
        تلگرام، توییتر، 
        \lr{YouTube} \\
        
        پلتفرم‌های گزارش‌دهی &
        سیستم‌های آنلاین گزارش 
        تخلفات و نقض حقوق &
        \lr{Ushahidi}، 
        \lr{Electionwatch} \\
        \altrow
        
        نظارت اجتماعی &
        پایش عملکرد نهادهای عمومی 
        توسط سازمان‌های مردم‌نهاد &
        \lr{Transparency Intl.}\\
        
        نظارت دیجیتال &
        تحلیل داده‌های باز، تصاویر 
        ماهواره‌ای، هوش مصنوعی &
        \lr{Bellingcat}، 
        \lr{Planet Labs} \\
        
        \bottomrule
    \end{tabularx}
\end{table}

\begin{operationalnote}
\textbf{برای ایران:}
با توجه به سابقه‌ی فعالیت مدنی ایرانیان 
(حتی تحت سرکوب) و مهارت فناورانه‌ی 
نسل جوان، نظارت شهروندی می‌تواند 
مکمل بسیار قوی‌ای برای نظارت رسمی 
بین‌المللی باشد. \emphgreen{باید از همین 
الان برنامه‌ی آموزش ناظران شهروندی 
ایرانی (داخل و دیاسپورا) آغاز شود.}
\end{operationalnote}

\sectiondivider

% ============================================================
\section{جمع‌بندی فصل}
\label{sec:ch3-summary}
% ============================================================

\begin{chaptersummary}

\textbf{آنچه در این فصل آموختیم:}

\begin{enumerate}[
    label=\textcolor{DarkGray}{\bfseries\arabic*.},
    itemsep=4pt
]
    \item \textbf{شش مدل متمایز} نظارت بین‌المللی 
    وجود دارد — از حداقلی (انتخاباتی) تا 
    حداکثری (مدیریت مستقیم).
    
    \item \textbf{مدل ۱ (انتخاباتی)} برای ایران 
    \emphred{ناکافی} است — نظارت بدون عمق 
    ساختاری، ابزار مشروعیت‌بخشی کاذب می‌شود.
    
    \item \textbf{مدل ۲ (مشورتی)} مفید اما 
    \emphred{ناکافی} است — بدون ضمانت اجرا، 
    تجربه‌ی تونس تکرار می‌شود.
    
    \item \textbf{مدل ۳ (ساختاری)} مناسب‌ترین 
    مدل منفرد است اما نیاز به \emphblue{مشوق 
    خارجی قوی} دارد که معادل عضویت EU 
    برای ایران طراحی شود.
    
    \item \textbf{مدل ۴ (اجرایی)} عناصر ضروری 
    دارد اما \emphred{حساسیت ملی‌گرایانه} و 
    \emphred{مقیاس‌ناپذیری} چالش‌های اصلی‌اند.
    
    \item \textbf{مدل ۵ (مدیریت مستقیم)} باید 
    \emphred{قاطعانه رد شود} — تجربه‌ی عراق 
    کافی است.
    
    \item \textbf{مدل ۶ (ترکیبی-تطبیقی)} — 
    پیشنهاد این کتاب — عناصر مدل‌های ۲، ۳ 
    و ۴ را در سه فاز زمانی ترکیب می‌کند: 
    تثبیت → نهادسازی → تحکیم.
    
    \item \textbf{هفت اصل بنیادین} مدل ترکیبی: 
    مالکیت ملی، فازبندی، انطباق‌پذیری، 
    فراگیری، شفافیت، پاسخگویی دوسویه 
    و خروج شفاف.
    
    \item \textbf{نظارت شهروندی} مکمل حیاتی 
    نظارت رسمی است و باید از الان 
    برنامه‌ریزی شود.
\end{enumerate}

\vspace{6pt}
\begin{center}
    \textcolor{MainOrange}{
        \faArrowLeft\hspace{8pt}
        \textbf{فصل بعد: سناریوهای گذار و 
        مدل‌های نظارتی متناظر}
        \hspace{8pt}\faArrowLeft
    }
\end{center}

\end{chaptersummary}

\chapterend