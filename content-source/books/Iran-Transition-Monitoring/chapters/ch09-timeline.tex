% ═══════════════════════════════════════════════════════════════════════════════
% فصل ۹: زمان‌بندی، تیم‌سازی و ساختارسازی
% فایل: chapters/ch09-timeline.tex
% رنگ فصل: زرد (MainYellow)
% ═══════════════════════════════════════════════════════════════════════════════

\chapteropening{۹}{زمان‌بندی، تیم‌سازی و ساختارسازی}{MainYellow}{%
برنامه‌ریزی بدون اجرا، رؤیاپردازی است. اجرا بدون برنامه‌ریزی، کابوس.%
}{پیتر دراکر}

\chapter{زمان‌بندی، تیم‌سازی و ساختارسازی}
\label{ch:timeline}

\minitoc

% ─────────────────────────────────────────────────────────────────────────────
% خلاصه اجرایی
% ─────────────────────────────────────────────────────────────────────────────

\begin{executivesummary}
نظارت بین‌المللی بر گذار ایران یک عملیات چندفازی است که از مرحله پیش‌گذار (۶-۱۲ ماه قبل از سقوط رژیم) آغاز و تا ۱۰ سال پس از آن ادامه می‌یابد. این فصل زمان‌بندی تفصیلی پنج فاز، ساختار تیم‌ها در هر فاز، مکانیزم هماهنگی، و ابزارهای مدیریت پروژه را ارائه می‌دهد. تأکید اصلی بر \emph{آمادگی پیش‌گذار} (که اغلب نادیده گرفته می‌شود)، \emph{سرعت عمل در ۷۲ ساعت اول}، و \emph{انتقال تدریجی مسئولیت} به نهادهای ایرانی است. ماتریس \lr{RACI} و نمودار گانت عملیاتی، ابزارهای اجرایی این فصل‌اند.
\end{executivesummary}

\section{درآمد: زمان به‌عنوان متغیر حیاتی}
\label{sec:timeline-intro}

در فرایند گذار دموکراتیک، زمان‌بندی صرفاً یک ابزار مدیریتی نیست؛ \emph{متغیر استراتژیک} است. اقدام زودهنگام یا دیرهنگام، هر دو می‌توانند فاجعه‌بار باشند.

\begin{keypoint}
\textbf{پارادوکس زمان در گذار}: 
\begin{itemize}[nosep]
    \item خیلی سریع = بی‌ثباتی، اشتباهات جبران‌ناپذیر
    \item خیلی آهسته = از دست رفتن فرصت، خستگی مردم، بازگشت اقتدارگرایی
    \item نقطه بهینه = «به‌اندازه کافی سریع برای حفظ momentum، به‌اندازه کافی محتاط برای جلوگیری از خطا»
\end{itemize}
\end{keypoint}

\begin{casestudy}{مصر در مقابل تونس: اهمیت زمان‌بندی}
مصر پس از سقوط مبارک، به فشار خیابان و نظامیان، انتخابات سریع برگزار کرد (کمتر از ۱ سال) — قبل از آنکه نهادها و احزاب آماده شوند. نتیجه: پیروزی اخوان‌المسلمین، واکنش نظامیان، کودتا. تونس ۲.۵ سال صبر کرد، قانون اساسی توافقی نوشت، و سپس انتخابات برگزار کرد. نتیجه: گذار نسبتاً موفق (هرچند شکننده).
\end{casestudy}

\sectiondivider

% ═══════════════════════════════════════════════════════════════════════════════
\section{نمای کلی فازبندی}
\label{sec:phasing-overview}
% ═══════════════════════════════════════════════════════════════════════════════

\subsection{پنج فاز عملیات}
\label{subsec:five-phases}

\begin{figure}[htbp]
\centering
\begin{tikzpicture}[
    node distance=0.3cm,
    every node/.style={font=\small},
    phase/.style={rectangle, rounded corners=5pt, minimum width=3cm, minimum height=1.8cm, align=center, text=white, font=\small\bfseries},
    arrow/.style={-{Stealth[length=3mm]}, very thick, DarkGray},
    label/.style={font=\footnotesize, align=center, text=DarkGray}
]
    % Phases
    \node[phase, fill=MediumGray] (p0) {فاز ۰\\پیش‌گذار\\{\footnotesize ۶-۱۲ ماه قبل}};
    \node[phase, fill=MainRed, right=0.5cm of p0] (p1) {فاز ۱\\تثبیت\\{\footnotesize ماه ۱-۶}};
    \node[phase, fill=MainOrange, right=0.5cm of p1] (p2) {فاز ۲\\نهادسازی\\{\footnotesize ماه ۶-۲۴}};
    \node[phase, fill=MainGreen, right=0.5cm of p2] (p3) {فاز ۳\\تحکیم\\{\footnotesize ماه ۲۴-۶۰}};
    \node[phase, fill=MainBlue, right=0.5cm of p3] (p4) {فاز ۴\\پی‌گیری\\{\footnotesize ماه ۶۰-۱۲۰}};
    
    % Arrows
    \draw[arrow] (p0) -- (p1);
    \draw[arrow] (p1) -- (p2);
    \draw[arrow] (p2) -- (p3);
    \draw[arrow] (p3) -- (p4);
    
    % Labels below
    \node[label, below=0.4cm of p0] {آمادگی\\برنامه‌ریزی};
    \node[label, below=0.4cm of p1] {امنیت\\نظم اولیه};
    \node[label, below=0.4cm of p2] {قانون اساسی\\انتخابات};
    \node[label, below=0.4cm of p3] {دموکراسی\\نهادها};
    \node[label, below=0.4cm of p4] {نظارت\\مشاوره};
    
    % Personnel indicator
    \node[label, above=0.4cm of p0] {۵۰-۲۰۰};
    \node[label, above=0.4cm of p1] {۳-۵K};
    \node[label, above=0.4cm of p2] {۶-۱۲K};
    \node[label, above=0.4cm of p3] {۱.۵-۳K};
    \node[label, above=0.4cm of p4] {۲۰۰-۵۰۰};
    \node[font=\footnotesize, anchor=south] at ($(p2)+(0,1.5)$) {\textbf{کارکنان بین‌المللی}};
    
\end{tikzpicture}
\caption{نمای کلی پنج فاز عملیات نظارت بین‌المللی}
\label{fig:five-phases}
\end{figure}

\begin{table}[htbp]
\centering
\caption{خلاصه ویژگی‌های هر فاز}
\label{tab:phase-summary}
\begin{tabularx}{\textwidth}{>{\raggedleft\arraybackslash}p{1.5cm}
                             >{\raggedleft\arraybackslash}p{2.5cm}
                             >{\raggedleft\arraybackslash}X
                             >{\raggedleft\arraybackslash}p{2cm}
                             >{\raggedleft\arraybackslash}p{2.5cm}}
\toprule
\headerrow فاز & بازه زمانی & هدف اصلی & مدل غالب & شاخص پایان \\
\midrule
۰ & ۶-۱۲ ماه قبل & آمادگی، برنامه‌ریزی، شبکه‌سازی & --- & شروع گذار \\
\altrow ۱ & ماه ۱-۶ & تثبیت، امنیت، نظم اولیه & مدل ۴ & امنیت نسبی \\
۲ & ماه ۶-۲۴ & قانون اساسی، انتخابات، نهادها & مدل ۳+۴ & انتخابات آزاد \\
\altrow ۳ & ماه ۲۴-۶۰ & تحکیم دموکراسی، انتقال کامل & مدل ۲+۳ & دولت منتخب پایدار \\
۴ & ماه ۶۰-۱۲۰ & پی‌گیری، مشاوره، ارزیابی & مدل ۲ & خروج کامل \\
\bottomrule
\end{tabularx}
\end{table}

\sectiondivider

% ═══════════════════════════════════════════════════════════════════════════════
\section{فاز ۰: پیش‌گذار (۶-۱۲ ماه قبل)}
\label{sec:phase-zero}
% ═══════════════════════════════════════════════════════════════════════════════

این فاز، مهم‌ترین و اغلب نادیده‌گرفته‌ترین مرحله است. آمادگی قبل از بحران، تفاوت میان موفقیت و فاجعه است.

\begin{warningbox}
در اکثر گذارهای تاریخی، جامعه بین‌المللی \emph{غافلگیر} شده است. فروپاشی شوروی (۱۹۹۱)، سقوط بن‌علی (۲۰۱۱)، سقوط بشار اسد — هیچ‌کدام با آمادگی کافی بین‌المللی همراه نبود. این سند دقیقاً برای پر کردن این خلأ نوشته شده است.
\end{warningbox}

\subsection{اقدامات فاز ۰}
\label{subsec:phase0-actions}

\begin{table}[htbp]
\centering
\caption{اقدامات تفصیلی فاز ۰ (پیش‌گذار)}
\label{tab:phase0-actions}
\begin{tabularx}{\textwidth}{>{\raggedleft\arraybackslash}p{0.5cm}
                             >{\raggedleft\arraybackslash}p{3cm}
                             >{\raggedleft\arraybackslash}X
                             >{\raggedleft\arraybackslash}p{2.5cm}
                             >{\raggedleft\arraybackslash}p{2cm}}
\toprule
\headerrow \# & اقدام & شرح & مسئول & زمان \\
\midrule
۱ & تشکیل گروه برنامه‌ریزی & تیم ۵۰-۱۰۰ نفره در نیویورک/ژنو & \lr{DPPA/UN} & ماه ۱-۳ \\
\altrow ۲ & تحلیل سناریو & بروزرسانی سناریوهای فصل ۴ بر اساس اطلاعات جدید & گروه برنامه‌ریزی & ماه ۱-۶ \\
۳ & شناسایی \lr{SRSG} بالقوه & فهرست ۵-۱۰ نامزد، مذاکرات محرمانه & دبیرکل & ماه ۳-۶ \\
\altrow ۴ & پیش‌نویس قطعنامه & نسخه‌های مختلف برای سناریوهای مختلف & \lr{P3+} & ماه ۳-۹ \\
۵ & شبکه‌سازی با اپوزیسیون & ارتباط غیررسمی با همه گروه‌ها & \lr{DPPA} + دولت‌ها & مستمر \\
\altrow ۶ & نقشه‌برداری ظرفیت & شناسایی ایرانیان متخصص (داخل و دیاسپورا) & \lr{UNDP} & ماه ۱-۶ \\
۷ & پیش‌موقعیت‌یابی & ذخیره تجهیزات در کشورهای همسایه & \lr{DPKO/DOS} & ماه ۶-۱۲ \\
\altrow ۸ & آموزش پیشینی & آماده‌سازی ناظران از فهرست آماده‌باش & \lr{UN/EU} & ماه ۶-۱۲ \\
۹ & طرح ارتباطات عمومی & پیام‌ها، مخاطبان، رسانه‌ها & تیم ارتباطات & ماه ۹-۱۲ \\
\altrow ۱۰ & طرح بودجه اولیه & برآورد فاز ۱ و ۲، شناسایی حامیان & \lr{OCHA/WB} & ماه ۶-۱۲ \\
\bottomrule
\end{tabularx}
\end{table}

\begin{lessonlearned}{تیمور شرقی: ارزش آمادگی}
در تیمور شرقی، سازمان ملل ماه‌ها قبل از رفراندوم ۱۹۹۹ برنامه‌ریزی کرده بود. وقتی خشونت پس از رفراندوم آغاز شد، \lr{INTERFET} (نیروی چندملیتی به رهبری استرالیا) ظرف ۱۰ روز مستقر شد. بدون این آمادگی، فاجعه بسیار بزرگ‌تر بود.
\end{lessonlearned}

\sectiondivider

% ═══════════════════════════════════════════════════════════════════════════════
\section{فاز ۱: تثبیت (ماه ۱-۶)}
\label{sec:phase-one}
% ═══════════════════════════════════════════════════════════════════════════════

\subsection{۷۲ ساعت اول: لحظه سرنوشت‌ساز}
\label{subsec:first-72}

\begin{warningbox}
۷۲ ساعت اول پس از سقوط رژیم، پنجره فرصتی است که آینده گذار را رقم می‌زند. تصمیمات این ساعات، بازگشت‌ناپذیرند.
\end{warningbox}

\begin{table}[htbp]
\centering
\caption{اقدامات حیاتی ۷۲ ساعت اول}
\label{tab:first-72}
\begin{tabularx}{\textwidth}{>{\raggedleft\arraybackslash}p{1.5cm}
                             >{\raggedleft\arraybackslash}p{3cm}
                             >{\raggedleft\arraybackslash}X
                             >{\raggedleft\arraybackslash}p{2.5cm}}
\toprule
\headerrow ساعت & اقدام & شرح & مسئول \\
\midrule
۰-۶ & ارزیابی وضعیت & تماس با همه طرف‌ها، ارزیابی امنیتی & \lr{DPPA} \\
\altrow ۶-۱۲ & بیانیه دبیرکل & حمایت از گذار مسالمت‌آمیز، دعوت به خویشتنداری & دبیرکل \\
۱۲-۲۴ & نشست شورای امنیت & بحث اولیه، بیانیه ریاست & شورای امنیت \\
\altrow ۲۴-۴۸ & اعزام تیم ارزیابی & تیم ۲۰-۳۰ نفره به تهران & \lr{DPPA/OCHA} \\
۴۸-۷۲ & تماس با دولت موقت & شناسایی طرف‌های مذاکره، ارزیابی نیازها & تیم ارزیابی \\
\altrow ۷۲+ & پیش‌نویس قطعنامه & ارائه متن به شورای امنیت & \lr{P3+} \\
\bottomrule
\end{tabularx}
\end{table}

\subsection{اقدامات ماه اول}
\label{subsec:month-one}

\begin{enumerate}[nosep]
    \item \textbf{انتخاب و اعزام \lr{SRSG}} (از فهرست آماده فاز ۰)
    \item \textbf{تصویب قطعنامه شورای امنیت} (ایجاد \lr{UNMOIT})
    \item \textbf{مذاکره \lr{SOMA}} با دولت انتقالی/موقت
    \item \textbf{استقرار تیم پیشتاز} (۵۰۰-۱,۰۰۰ نفر)
    \item \textbf{ایجاد دفتر مرکزی} در تهران
    \item \textbf{شروع پایش حقوق بشر} (فوری)
    \item \textbf{برقراری ارتباط با نیروهای امنیتی} (ارتش، پلیس)
    \item \textbf{ارزیابی وضعیت اقتصادی} و نیازهای فوری
\end{enumerate}

\subsection{اقدامات ماه ۲-۶}
\label{subsec:months-2-6}

\begin{table}[htbp]
\centering
\caption{اقدامات کلیدی فاز ۱ (ماه ۲-۶)}
\label{tab:phase1-actions}
\begin{tabularx}{\textwidth}{>{\raggedleft\arraybackslash}p{2.5cm}
                             >{\raggedleft\arraybackslash}X
                             >{\raggedleft\arraybackslash}p{2.5cm}
                             >{\raggedleft\arraybackslash}p{2cm}}
\toprule
\headerrow حوزه & اقدام‌های کلیدی & مسئول & زمان \\
\midrule
امنیت & نظارت بر نیروهای مسلح، شروع \lr{DDR}، کنترل مرزها & معاون امنیتی & ماه ۱-۶ \\
\altrow سیاسی & تشکیل شورای مشورتی، گفتگوی ملی، زمان‌بندی انتخابات & معاون سیاسی & ماه ۲-۶ \\
حقوقی & تصویب قوانین موقت، منشور حقوق بنیادین & \lr{SRSG}+شورا & ماه ۲-۴ \\
\altrow حقوق بشر & مستندسازی نقض‌ها، حفظ اسناد، محافظت از شهود & \lr{OHCHR} & ماه ۱-۶ \\
اقتصاد & رفع تحریم‌ها، تثبیت ارز، کمک بشردوستانه & \lr{IMF/WB} & ماه ۱-۶ \\
\altrow نهادسازی & ایجاد کمیسیون انتخابات، شروع ثبت رأی‌دهندگان & \lr{UNDP/IFES} & ماه ۳-۶ \\
رسانه & رفع فیلترینگ، صدور مجوز رسانه، نظارت بر تعادل & کمیسیون رسانه & ماه ۱-۶ \\
\altrow استقرار & ایجاد ۳۱ دفتر استانی، استخدام ۳-۵K بین‌المللی & \lr{UNMOIT} اداری & ماه ۲-۶ \\
\bottomrule
\end{tabularx}
\end{table}

\sectiondivider

% ═══════════════════════════════════════════════════════════════════════════════
\section{فاز ۲: نهادسازی (ماه ۶-۲۴)}
\label{sec:phase-two}
% ═══════════════════════════════════════════════════════════════════════════════

فاز ۲ قلب عملیات است. در این فاز، نهادهای دموکراتیک ساخته می‌شوند و اولین انتخابات آزاد برگزار می‌شود.

\subsection{قانون اساسی}
\label{subsec:constitution}

\begin{table}[htbp]
\centering
\caption{فرایند تدوین قانون اساسی}
\label{tab:constitution-process}
\begin{tabularx}{\textwidth}{>{\raggedleft\arraybackslash}p{2.5cm}
                             >{\raggedleft\arraybackslash}X
                             >{\raggedleft\arraybackslash}p{2.5cm}
                             >{\raggedleft\arraybackslash}p{2cm}}
\toprule
\headerrow مرحله & شرح & مشارکت‌کنندگان & بازه زمانی \\
\midrule
مشاوره عمومی & جمع‌آوری نظرات مردم (آنلاین و حضوری) & مردم، \lr{NGO}ها & ماه ۶-۱۰ \\
\altrow انتخاب مجلس مؤسسان & انتخابات برای ۲۰۰-۳۰۰ نماینده & کمیسیون انتخابات & ماه ۱۰-۱۲ \\
تدوین پیش‌نویس & نگارش با کمک مشاوران بین‌المللی & مجلس مؤسسان & ماه ۱۲-۱۸ \\
\altrow بازخورد عمومی & انتشار پیش‌نویس، جمع‌آوری نظرات & مردم، احزاب & ماه ۱۸-۲۰ \\
بازنگری و اصلاح & اصلاح بر اساس بازخورد & مجلس مؤسسان & ماه ۲۰-۲۲ \\
\altrow رفراندوم & تصویب عمومی قانون اساسی & مردم & ماه ۲۲-۲۴ \\
\bottomrule
\end{tabularx}
\end{table}

\begin{keypoint}
\textbf{دو رویکرد به قانون اساسی:}
\begin{itemize}[nosep]
    \item \textbf{اول انتخابات، بعد قانون اساسی} (مدل عراق): سریع‌تر اما ریسک مصادره
    \item \textbf{اول قانون اساسی، بعد انتخابات} (مدل تونس): کندتر اما پایدارتر
\end{itemize}
\textbf{توصیه}: مدل تونس، با تعدیل — یعنی مجلس مؤسسان منتخب ابتدا قانون اساسی بنویسد، سپس انتخابات پارلمانی و ریاست‌جمهوری برگزار شود.
\end{keypoint}

\subsection{انتخابات}
\label{subsec:elections}

\begin{table}[htbp]
\centering
\caption{تقویم انتخاباتی پیشنهادی}
\label{tab:election-calendar}
\begin{tabularx}{\textwidth}{>{\raggedleft\arraybackslash}p{3cm}
                             >{\centering\arraybackslash}p{2.5cm}
                             >{\raggedleft\arraybackslash}X
                             >{\raggedleft\arraybackslash}p{2.5cm}}
\toprule
\headerrow انتخابات & ماه (تقریبی) & هدف & نظام انتخاباتی \\
\midrule
مجلس مؤسسان & ماه ۱۰-۱۲ & تدوین قانون اساسی & تناسبی \\
\altrow رفراندوم قانون اساسی & ماه ۲۲-۲۴ & تصویب مردمی & اکثریت ساده \\
پارلمان & ماه ۲۶-۳۰ & قوه مقننه & تناسبی مختلط \\
\altrow ریاست‌جمهوری & ماه ۲۸-۳۲ & قوه مجریه & دومرحله‌ای \\
شوراهای محلی & ماه ۳۰-۳۶ & حکمرانی محلی & تناسبی \\
\bottomrule
\end{tabularx}
\end{table}

\begin{lessonlearned}{آفریقای جنوبی: صبر استراتژیک}
آفریقای جنوبی ۴ سال (۱۹۹۰-۱۹۹۴) صبر کرد تا اولین انتخابات دموکراتیک برگزار شود. این زمان صرف مذاکره، ساخت اعتماد، و آماده‌سازی نهادها شد. اولین انتخابات ۱۹۹۴ یکی از موفق‌ترین انتخابات گذار در تاریخ بود: مشارکت ۸۶٪، پذیرش نتایج توسط همه طرف‌ها.
\end{lessonlearned}

\sectiondivider

% ═══════════════════════════════════════════════════════════════════════════════
\section{فاز ۳: تحکیم (ماه ۲۴-۶۰)}
\label{sec:phase-three}
% ═══════════════════════════════════════════════════════════════════════════════

پس از انتخابات و تشکیل دولت منتخب، فاز تحکیم آغاز می‌شود. نظارت بین‌المللی تدریجاً کاهش و مالکیت ملی افزایش می‌یابد.

\subsection{اقدامات کلیدی فاز ۳}
\label{subsec:phase3-actions}

\begin{table}[htbp]
\centering
\caption{اقدامات فاز ۳ (تحکیم)}
\label{tab:phase3-actions}
\begin{tabularx}{\textwidth}{>{\raggedleft\arraybackslash}p{3cm}
                             >{\raggedleft\arraybackslash}X
                             >{\raggedleft\arraybackslash}p{2.5cm}}
\toprule
\headerrow حوزه & اقدامات & تغییر نسبت به فاز ۲ \\
\midrule
نظارت انتخاباتی & نظارت بر انتخابات دوم، آموزش ناظران داخلی & کاهش ۵۰٪ \\
\altrow حقوق بشر & ادامه پایش، کمیسیون حقیقت فعال & ثابت \\
اصلاح امنیتی & ادامه \lr{DDR}، آموزش پلیس، نظارت بر ارتش & کاهش ۳۰٪ \\
\altrow قضایی & دادگاه ویژه فعال، اصلاح قوه قضاییه & ثابت \\
اقتصادی & خصوصی‌سازی نظارت‌شده، جذب سرمایه‌گذاری & مشاوره‌ای \\
\altrow ظرفیت‌سازی & آموزش کارکنان دولت، انتقال مهارت & افزایش \\
رسانه & حمایت از رسانه مستقل، آموزش روزنامه‌نگاران & کاهش ۴۰٪ \\
\altrow کاهش حضور & بسته شدن دفاتر استانی، کاهش کارکنان بین‌المللی & ۶-۱۲K → ۱.۵-۳K \\
\bottomrule
\end{tabularx}
\end{table}

\subsection{معیارهای تحکیم}
\label{subsec:consolidation-criteria}

\begin{recommendation}
دموکراسی ایران «تحکیم‌یافته» تلقی می‌شود اگر:
\begin{enumerate}[nosep]
    \item حداقل ۲ انتخابات آزاد و عادلانه برگزار شده باشد
    \item حداقل ۱ انتقال مسالمت‌آمیز قدرت رخ داده باشد
    \item نظامیان مداخله‌ای در سیاست نداشته باشند
    \item آزادی مطبوعات در سطح «نسبتاً آزاد» باشد
    \item استقلال قوه قضاییه تضمین شده باشد
    \item نرخ اعتماد عمومی به دموکراسی بالای ۵۰٪ باشد
    \item حقوق اقلیت‌ها رعایت شود
\end{enumerate}
\end{recommendation}

\sectiondivider

% ═══════════════════════════════════════════════════════════════════════════════
\section{فاز ۴: پی‌گیری و خروج (ماه ۶۰-۱۲۰)}
\label{sec:phase-four}
% ═══════════════════════════════════════════════════════════════════════════════

\subsection{استراتژی خروج}
\label{subsec:exit-strategy}

\begin{keypoint}
\textbf{سه سناریوی خروج:}
\begin{enumerate}[nosep]
    \item \textbf{خروج موفق}: معیارهای تحکیم محقق شده → خروج تدریجی با ۱-۲ سال هم‌پوشانی
    \item \textbf{خروج تعدیل‌شده}: برخی معیارها محقق → کاهش اما نه خروج کامل → تمدید ۲-۳ ساله
    \item \textbf{بازنگری اساسی}: عقب‌گرد جدی → بازنگری کل استراتژی → ممکن است نیاز به افزایش
\end{enumerate}
\end{keypoint}

\begin{table}[htbp]
\centering
\caption{مراحل خروج تدریجی}
\label{tab:exit-steps}
\begin{tabularx}{\textwidth}{>{\raggedleft\arraybackslash}p{2.5cm}
                             >{\raggedleft\arraybackslash}X
                             >{\raggedleft\arraybackslash}p{2cm}
                             >{\raggedleft\arraybackslash}p{2.5cm}}
\toprule
\headerrow مرحله & اقدام & ماه & باقیمانده \\
\midrule
تبدیل مأموریت & تغییر \lr{UNMOIT} به دفتر سیاسی \lr{UN} & ۶۰ & ۵۰۰ نفر \\
\altrow بسته شدن دفاتر & ادغام ۳۱ استان در ۵-۷ منطقه & ۶۰-۷۲ & ۳۰۰ نفر \\
کاهش تخصصی & پایان مأموریت انتخاباتی و امنیتی & ۷۲-۸۴ & ۲۰۰ نفر \\
\altrow مشاوره محض & فقط مشاوران ارشد و ارزیابان & ۸۴-۱۰۸ & ۵۰-۱۰۰ نفر \\
خروج نهایی & بسته شدن مأموریت، گزارش نهایی & ۱۰۸-۱۲۰ & ۰ \\
\bottomrule
\end{tabularx}
\end{table}

\begin{warningbox}
خروج زودهنگام به‌اندازه ماندن بیش‌ازحد خطرناک است. خروج آمریکا از عراق (۲۰۱۱) و افغانستان (۲۰۲۱) نشان داد که خروج بدون تحکیم واقعی، می‌تواند همه دستاوردها را نابود کند. خروج باید بر اساس شاخص‌ها باشد، نه تقویم سیاسی.
\end{warningbox}

\sectiondivider

% ═══════════════════════════════════════════════════════════════════════════════
\section{ساختار هماهنگی}
\label{sec:coordination-structure}
% ═══════════════════════════════════════════════════════════════════════════════

\subsection{سلسله‌مراتب هماهنگی}
\label{subsec:coordination-hierarchy}

\begin{figure}[htbp]
\centering
\begin{tikzpicture}[
    node distance=1cm,
    every node/.style={font=\small, align=center},
    global/.style={rectangle, rounded corners, draw=MainPurple, fill=LightPurple, minimum width=4.5cm, minimum height=0.9cm},
    strategic/.style={rectangle, rounded corners, draw=MainBlue, fill=LightBlue, minimum width=4cm, minimum height=0.9cm},
    operational/.style={rectangle, rounded corners, draw=MainGreen, fill=LightGreen, minimum width=3.5cm, minimum height=0.9cm},
    tactical/.style={rectangle, rounded corners, draw=MainOrange, fill=LightOrange, minimum width=3cm, minimum height=0.9cm},
    arrow/.style={-{Stealth[length=2.5mm]}, thick}
]
    % Global level
    \node[global] (sc) {شورای امنیت \lr{UN}};
    \node[global, right=2.5cm of sc] (cg) {گروه تماس بین‌المللی\\{\footnotesize \lr{P5} + \lr{EU} + ژاپن + ترکیه + ...}};
    
    % Strategic level
    \node[strategic, below=1.2cm of sc] (srsg) {\lr{SRSG}\\ستاد مرکزی تهران};
    \node[strategic, right=2.5cm of srsg] (advisory) {شورای مشورتی ایرانی\\{\footnotesize ۲۰-۳۰ نماینده}};
    
    % Operational level
    \node[operational, below left=1.2cm and 1.5cm of srsg] (pol) {بخش سیاسی};
    \node[operational, below=1.2cm of srsg] (ops) {بخش عملیاتی};
    \node[operational, below right=1.2cm and 1.5cm of srsg] (hr) {بخش حقوق بشر};
    
    % Tactical level
    \node[tactical, below=1.2cm of pol] (prov1) {دفاتر استانی\\منطقه ۱-۳};
    \node[tactical, below=1.2cm of ops] (prov2) {دفاتر استانی\\منطقه ۴-۵};
    \node[tactical, below=1.2cm of hr] (prov3) {دفاتر استانی\\منطقه ۶-۷};
    
    % Arrows
    \draw[arrow] (sc) -- (srsg);
    \draw[arrow, dashed] (cg) -- (srsg);
    \draw[arrow, dashed] (advisory) -- (srsg);
    \draw[arrow] (srsg) -- (pol);
    \draw[arrow] (srsg) -- (ops);
    \draw[arrow] (srsg) -- (hr);
    \draw[arrow] (pol) -- (prov1);
    \draw[arrow] (ops) -- (prov2);
    \draw[arrow] (hr) -- (prov3);
    
    % Labels
    \node[font=\footnotesize\bfseries, MainPurple, anchor=east] at (-3.5,0) {سطح جهانی};
    \node[font=\footnotesize\bfseries, MainBlue, anchor=east] at (-3.5,-2.2) {سطح استراتژیک};
    \node[font=\footnotesize\bfseries, MainGreen, anchor=east] at (-3.5,-4.4) {سطح عملیاتی};
    \node[font=\footnotesize\bfseries, MainOrange, anchor=east] at (-3.5,-6.6) {سطح تاکتیکی};
    
\end{tikzpicture}
\caption{سلسله‌مراتب هماهنگی نظارت بین‌المللی}
\label{fig:coordination-hierarchy}
\end{figure}

\subsection{گروه تماس بین‌المللی}
\label{subsec:contact-group}

\begin{table}[htbp]
\centering
\caption{ترکیب و وظایف گروه تماس بین‌المللی}
\label{tab:contact-group}
\begin{tabularx}{\textwidth}{>{\raggedleft\arraybackslash}p{3cm}
                             >{\raggedleft\arraybackslash}X}
\toprule
\headerrow ویژگی & شرح \\
\midrule
اعضا & \lr{P5} + آلمان + ژاپن + ترکیه + عربستان + هند + \lr{EU} + اتحادیه آفریقا \\
\altrow سطح نمایندگی & وزیر خارجه یا معاون \\
دبیرخانه & دفتر \lr{SRSG} \\
\altrow فرکانس نشست & ماهانه (عادی)، فوری (بحران) \\
وظایف & هماهنگی سیاسی، تضمین منابع، جلوگیری از مداخله یکجانبه \\
\altrow تصمیم‌گیری & اجماعی (نه رأی‌گیری) \\
\bottomrule
\end{tabularx}
\end{table}

\subsection{شورای مشورتی ایرانی}
\label{subsec:advisory-council}

\begin{table}[htbp]
\centering
\caption{ترکیب شورای مشورتی ایرانی}
\label{tab:advisory-council}
\begin{tabularx}{\textwidth}{>{\raggedleft\arraybackslash}p{3.5cm}
                             >{\centering\arraybackslash}p{2cm}
                             >{\raggedleft\arraybackslash}X}
\toprule
\headerrow دسته & تعداد & معیار انتخاب \\
\midrule
احزاب/جنبش‌ها سیاسی & ۶-۸ & نمایندگان همه طیف‌ها \\
\altrow جامعه مدنی & ۴-۶ & \lr{NGO}های معتبر، حقوق بشری \\
زنان & ۳-۴ & فعالان حقوق زنان \\
\altrow اقوام & ۴-۶ & نمایندگان اقوام اصلی \\
اقلیت‌های مذهبی & ۲-۳ & سنّی، بهایی، مسیحی، یهودی، زرتشتی \\
\altrow جوانان & ۲-۳ & زیر ۳۵ سال \\
دیاسپورا & ۲-۳ & نمایندگان جوامع مهاجر \\
\altrow حقوقدانان & ۲-۳ & متخصصان حقوق اساسی \\
\midrule
\textbf{مجموع} & \textbf{۲۵-۳۶} & \\
\bottomrule
\end{tabularx}
\end{table}

\begin{casestudy}{عراق: شورای حکومتی ناکارآمد}
شورای حکومتی عراق (۲۰۰۳-۲۰۰۴) که توسط \lr{CPA} منصوب شد، فاقد مشروعیت مردمی بود و ترکیب آن بر اساس سهمیه‌بندی قومی-مذهبی بود. این الگو به «محاصصه» (تقسیم طایفه‌ای قدرت) دامن زد و تا امروز عراق از آن رنج می‌برد. شورای مشورتی ایران باید \emph{مشاوره‌ای} باشد، نه \emph{حکومتی}، و ترکیب آن بر اساس تخصص و اعتبار باشد نه صرفاً نمایندگی قومی.
\end{casestudy}

\sectiondivider

% ═══════════════════════════════════════════════════════════════════════════════
\section{تیم‌سازی و ساختار منابع انسانی}
\label{sec:team-building}
% ═══════════════════════════════════════════════════════════════════════════════

\subsection{ساختار رهبری}
\label{subsec:leadership-structure}

\begin{table}[htbp]
\centering
\caption{ساختار رهبری \lr{UNMOIT}}
\label{tab:leadership}
\begin{tabularx}{\textwidth}{>{\raggedleft\arraybackslash}p{3.5cm}
                             >{\raggedleft\arraybackslash}p{2.5cm}
                             >{\raggedleft\arraybackslash}X}
\toprule
\headerrow مقام & سطح \lr{UN} & مسئولیت اصلی \\
\midrule
\lr{SRSG} & \lr{USG} & رهبری کل مأموریت \\
\altrow معاون سیاسی & \lr{ASG} & مذاکرات، احزاب، گفتگوی ملی \\
معاون عملیاتی & \lr{ASG} & لجستیک، امور اداری، مالی \\
\altrow معاون حقوق بشر & \lr{D-2} & پایش، مستندسازی، عدالت انتقالی \\
معاون امنیتی & \lr{D-2} & اصلاح بخش امنیتی، \lr{DDR} \\
\altrow مشاور ارشد انتخابات & \lr{D-2} & کمیسیون انتخابات، فنی \\
سخنگو & \lr{D-1} & ارتباطات عمومی، رسانه \\
\bottomrule
\end{tabularx}
\end{table}

\subsection{دفاتر استانی}
\label{subsec:provincial-offices}

\begin{table}[htbp]
\centering
\caption{ساختار دفاتر استانی}
\label{tab:provincial-offices}
\begin{tabularx}{\textwidth}{>{\raggedleft\arraybackslash}p{3cm}
                             >{\raggedleft\arraybackslash}X}
\toprule
\headerrow عنصر & شرح \\
\midrule
تعداد & ۳۱ دفتر (یکی در هر استان) + ۵-۷ دفتر منطقه‌ای \\
\altrow رئیس & کارمند بین‌المللی \lr{P-5/D-1} با معاون ایرانی \\
کارکنان (فاز ۲) & ۵۰-۲۰۰ نفر بسته به اندازه استان \\
\altrow وظایف & پایش محلی، ارتباط با جوامع، گزارش‌دهی، نظارت انتخاباتی \\
اولویت استقرار & ابتدا استان‌های حساس (مرزی، قومی، پرجمعیت) \\
\bottomrule
\end{tabularx}
\end{table}

\begin{recommendation}
\textbf{اولویت‌بندی استقرار دفاتر استانی:}
\begin{enumerate}[nosep]
    \item \textbf{فوری (هفته ۱-۴)}: تهران، اصفهان، مشهد (خراسان رضوی)، تبریز (آذربایجان شرقی)، اهواز (خوزستان)
    \item \textbf{اولویت بالا (ماه ۱-۲)}: کردستان، سیستان‌وبلوچستان، کرمانشاه، فارس (شیراز)، هرمزگان
    \item \textbf{اولویت متوسط (ماه ۲-۴)}: سایر استان‌های مرزی و پرجمعیت
    \item \textbf{تکمیلی (ماه ۴-۶)}: استان‌های کوچک‌تر مرکزی
\end{enumerate}
\end{recommendation}

\sectiondivider

% ═══════════════════════════════════════════════════════════════════════════════
\section{ماتریس \lr{RACI}}
\label{sec:raci-matrix}
% ═══════════════════════════════════════════════════════════════════════════════

\bilingual{ماتریس مسئولیت}{RACI Matrix} ابزاری برای تعیین نقش هر بازیگر در هر فعالیت است:
\begin{itemize}[nosep]
    \item \textbf{\lr{R} = مسئول اجرا} (\lr{Responsible})
    \item \textbf{\lr{A} = پاسخگو} (\lr{Accountable})
    \item \textbf{\lr{C} = مشاور} (\lr{Consulted})
    \item \textbf{\lr{I} = اطلاع‌یافته} (\lr{Informed})
\end{itemize}

\begin{landscape}
\begin{table}[htbp]
\centering
\bigtablefontsize
\caption{ماتریس \lr{RACI} فعالیت‌های کلیدی}
\label{tab:raci-matrix}
\begin{tabularx}{\linewidth}{>{\raggedleft\arraybackslash}p{3cm}
                             >{\centering\arraybackslash}p{1.2cm}
                             >{\centering\arraybackslash}p{1.5cm}
                             >{\centering\arraybackslash}p{1.5cm}
                             >{\centering\arraybackslash}p{1.5cm}
                             >{\centering\arraybackslash}p{1.2cm}
                             >{\centering\arraybackslash}p{1.2cm}
                             >{\centering\arraybackslash}p{1.2cm}
                             >{\centering\arraybackslash}p{1.5cm}
                             >{\centering\arraybackslash}p{1.5cm}}
\toprule
\headerrow فعالیت & \rot{\lr{SC}} & \rot{\lr{SRSG}} & \rot{دولت انتقالی} & \rot{شورای مشورتی} & \rot{کمیسیون انتخابات} & \rot{\lr{OHCHR}} & \rot{\lr{IMF/WB}} & \rot{جامعه مدنی} & \rot{گروه تماس} \\
\midrule
قطعنامه & \lr{A,R} & \lr{C} & \lr{C} & \lr{I} & \lr{I} & \lr{C} & \lr{I} & \lr{I} & \lr{C} \\
\altrow انتخاب \lr{SRSG} & \lr{A} & --- & \lr{C} & \lr{I} & \lr{I} & \lr{I} & \lr{I} & \lr{I} & \lr{C} \\
امنیت عمومی & \lr{I} & \lr{A} & \lr{R} & \lr{C} & \lr{I} & \lr{C} & \lr{I} & \lr{I} & \lr{I} \\
\altrow اصلاح سپاه & \lr{A} & \lr{R} & \lr{R} & \lr{C} & \lr{I} & \lr{C} & \lr{I} & \lr{I} & \lr{C} \\
برگزاری انتخابات & \lr{I} & \lr{A} & \lr{C} & \lr{C} & \lr{R} & \lr{I} & \lr{I} & \lr{C} & \lr{I} \\
\altrow قانون اساسی & \lr{I} & \lr{C} & \lr{A,R} & \lr{C} & \lr{I} & \lr{C} & \lr{I} & \lr{C} & \lr{I} \\
حقوق بشر & \lr{I} & \lr{A} & \lr{C} & \lr{C} & \lr{I} & \lr{R} & \lr{I} & \lr{C} & \lr{I} \\
\altrow اقتصاد & \lr{I} & \lr{C} & \lr{A,R} & \lr{C} & \lr{I} & \lr{I} & \lr{R} & \lr{I} & \lr{C} \\
رفع تحریم & \lr{A,R} & \lr{C} & \lr{C} & \lr{I} & \lr{I} & \lr{I} & \lr{C} & \lr{I} & \lr{R} \\
\altrow عدالت انتقالی & \lr{I} & \lr{C} & \lr{A,R} & \lr{C} & \lr{I} & \lr{R} & \lr{I} & \lr{C} & \lr{I} \\
\bottomrule
\end{tabularx}
\end{table}
\end{landscape}

\sectiondivider

% ═══════════════════════════════════════════════════════════════════════════════
\section{نمودار گانت عملیاتی}
\label{sec:gantt-chart}
% ═══════════════════════════════════════════════════════════════════════════════

\begin{figure}[htbp]
\centering
\begin{ganttchart}[
    x unit=0.4cm,
    y unit chart=0.6cm,
    y unit title=0.8cm,
    title height=1,
    bar height=0.5,
    group peaks height=0.3,
    title label font=\footnotesize\bfseries,
    bar label font=\footnotesize,
    group label font=\footnotesize\bfseries,
    milestone label font=\footnotesize,
    bar/.append style={fill=MainYellow!60},
    group/.append style={fill=MainBlue!60},
    milestone/.append style={fill=MainRed},
    vgrid={*{11}{draw=LightGray, thin}},
    hgrid={draw=VeryLightGray},
    today=0,
    today rule/.style={draw=MainRed, ultra thick, dashed},
    today label=گذار
]{-6}{54}
    % Titles
    \gantttitlelist{-6,...,0,...,54}{1} \\
    
    % Phase 0
    \ganttgroup{فاز ۰: پیش‌گذار}{-6}{0} \\
    \ganttbar{تشکیل تیم برنامه‌ریزی}{-6}{-3} \\
    \ganttbar{شناسایی \lr{SRSG}}{-6}{-1} \\
    \ganttbar{پیش‌نویس قطعنامه}{-4}{0} \\
    \ganttbar{پیش‌موقعیت‌یابی}{-3}{0} \\
    
    % Phase 1
    \ganttgroup{فاز ۱: تثبیت}{1}{6} \\
    \ganttbar{استقرار اولیه}{1}{3} \\
    \ganttbar{امنیت و نظم}{1}{6} \\
    \ganttbar{قوانین موقت}{2}{5} \\
    \ganttmilestone{تصویب قطعنامه}{1} \\
    
    % Phase 2
    \ganttgroup{فاز ۲: نهادسازی}{7}{24} \\
    \ganttbar{مشاوره عمومی قانون اساسی}{7}{12} \\
    \ganttbar{مجلس مؤسسان}{10}{12} \\
    \ganttbar{تدوین قانون اساسی}{12}{20} \\
    \ganttbar{رفراندوم}{22}{24} \\
    \ganttbar{انتخابات پارلمان}{26}{30} \\
    \ganttmilestone{رفراندوم قانون اساسی}{24} \\
    
    % Phase 3
    \ganttgroup{فاز ۳: تحکیم}{25}{48} \\
    \ganttbar{تشکیل دولت منتخب}{30}{33} \\
    \ganttbar{انتقال تدریجی مسئولیت}{30}{48} \\
    \ganttbar{انتخابات دوم}{42}{48} \\
    \ganttmilestone{انتقال قدرت}{33} \\
    
    % Phase 4
    \ganttgroup{فاز ۴: خروج}{49}{54} \\
    \ganttbar{کاهش حضور}{49}{54} \\
    \ganttmilestone{خروج کامل}{54}
    
\end{ganttchart}
\caption{نمودار گانت کلان عملیات نظارت (ماه -۶ تا ۵۴)}
\label{fig:gantt-chart}
\end{figure}

\begin{lessonlearned}{بوسنی: طولانی‌ترین مأموریت}
مأموریت بین‌المللی بوسنی (\lr{OHR}) از ۱۹۹۵ آغاز شد و تا ۲۰۲۴ (۲۹ سال!) ادامه داشت. دلیل: شروع بدون معیارهای خروج مشخص + ساختار سیاسی ناکارآمد دیتون + بن‌بست قومی. درس ایران: \textbf{معیارهای خروج باید از روز اول تعریف شوند} و هر فاز بر اساس شاخص‌های عینی (نه تقویم سیاسی) پایان یابد.
\end{lessonlearned}

\sectiondivider

% ═══════════════════════════════════════════════════════════════════════════════
\section{مکانیزم بازخورد و اصلاح مسیر}
\label{sec:feedback-mechanism}
% ═══════════════════════════════════════════════════════════════════════════════

هیچ طرحی بدون تغییر اجرا نمی‌شود. مکانیزم بازخورد تضمین می‌کند که اشتباهات شناسایی و اصلاح شوند.

\subsection{سطوح ارزیابی}
\label{subsec:evaluation-levels}

\begin{table}[htbp]
\centering
\caption{نظام ارزیابی پنج‌سطحی}
\label{tab:evaluation-levels}
\begin{tabularx}{\textwidth}{>{\raggedleft\arraybackslash}p{0.5cm}
                             >{\raggedleft\arraybackslash}p{2.5cm}
                             >{\centering\arraybackslash}p{2cm}
                             >{\raggedleft\arraybackslash}X
                             >{\raggedleft\arraybackslash}p{2.5cm}}
\toprule
\headerrow \# & نوع ارزیابی & فرکانس & شرح & مسئول \\
\midrule
۱ & پایش روزانه & روزانه & شاخص‌های هشدار زودهنگام (\seeChapter{ch:risks}) & تیم پایش \\
\altrow ۲ & گزارش ماهانه & ماهانه & گزارش \lr{SRSG} به دبیرکل و شورای امنیت & دفتر \lr{SRSG} \\
۳ & بازنگری فصلی & هر ۳ ماه & بررسی پیشرفت شاخص‌ها، تعدیل برنامه & تیم برنامه‌ریزی \\
\altrow ۴ & ارزیابی مستقل & هر ۶ ماه & تیم خارجی مستقل از \lr{UNMOIT} & \lr{OIOS/UN} \\
۵ & بازنگری استراتژیک & سالانه & بازنگری کل استراتژی، تصمیم ادامه/تعدیل/خروج & شورای امنیت \\
\bottomrule
\end{tabularx}
\end{table}

\subsection{نمودار چرخه بازخورد}
\label{subsec:feedback-cycle}

\begin{figure}[htbp]
\centering
\begin{tikzpicture}[
    node distance=2cm,
    every node/.style={font=\small, align=center},
    step/.style={circle, draw=MainYellow, fill=LightYellow, minimum size=2.2cm, thick},
    decision/.style={diamond, draw=MainRed, fill=LightRed, minimum width=2.5cm, minimum height=2cm, thick, aspect=1.5},
    arrow/.style={-{Stealth[length=3mm]}, thick, MainYellow!80!black}
]
    % Cycle nodes
    \node[step] (plan) {برنامه‌ریزی\\{\footnotesize \lr{Plan}}};
    \node[step, right=2.5cm of plan] (exec) {اجرا\\{\footnotesize \lr{Do}}};
    \node[step, below=2.5cm of exec] (check) {ارزیابی\\{\footnotesize \lr{Check}}};
    \node[decision, below=2.5cm of plan] (decide) {تصمیم\\{\footnotesize \lr{Act}}};
    
    % Arrows
    \draw[arrow] (plan) -- node[above, font=\footnotesize] {استقرار} (exec);
    \draw[arrow] (exec) -- node[right, font=\footnotesize] {داده‌ها} (check);
    \draw[arrow] (check) -- node[below, font=\footnotesize] {تحلیل} (decide);
    
    % Decision outcomes
    \draw[arrow, MainGreen] (decide) -- node[left, font=\footnotesize, MainGreen] {ادامه/بهبود} (plan);
    \draw[arrow, MainOrange, dashed] (decide) to[bend left=50] node[right=0.3cm, font=\footnotesize, MainOrange] {تعدیل جدی} (plan);
    \draw[arrow, MainRed, dotted] (decide) -- ++(0,-1.5) node[below, font=\footnotesize, MainRed] {بازنگری اساسی};
    
    % Center label
    \node[font=\bfseries] at ($(plan)!0.5!(check)$) {\lr{PDCA}};
    
\end{tikzpicture}
\caption{چرخه بازخورد مداوم (مدل \lr{PDCA})}
\label{fig:feedback-cycle}
\end{figure}

\begin{keypoint}
\textbf{سه نتیجه ممکن هر ارزیابی:}
\begin{enumerate}[nosep]
    \item \textbf{\textcolor{MainGreen}{سبز — ادامه}}: شاخص‌ها مثبت، برنامه طبق زمان‌بندی → ادامه با بهبودهای جزئی
    \item \textbf{\textcolor{MainOrange}{نارنجی — تعدیل}}: برخی شاخص‌ها منفی، تأخیر → تعدیل برنامه، تمدید فاز، تغییر اولویت
    \item \textbf{\textcolor{MainRed}{قرمز — بازنگری}}: عقب‌گرد جدی، بحران → بازنگری کل استراتژی، ممکن است نیاز به قطعنامه جدید
\end{enumerate}
\end{keypoint}

\subsection{شاخص‌های کلیدی عملکرد (\lr{KPI})}
\label{subsec:kpis}

\begin{table}[htbp]
\centering
\caption{شاخص‌های کلیدی عملکرد در هر فاز}
\label{tab:kpis}
\begin{tabularx}{\textwidth}{>{\raggedleft\arraybackslash}p{3cm}
                             >{\raggedleft\arraybackslash}X
                             >{\raggedleft\arraybackslash}p{2.5cm}
                             >{\raggedleft\arraybackslash}p{2cm}}
\toprule
\headerrow شاخص & تعریف عملیاتی & هدف & فاز مرتبط \\
\midrule
امنیت عمومی & تعداد حوادث امنیتی در ماه & کاهش ۸۰٪ & فاز ۱ \\
\altrow آزادی رسانه & امتیاز \lr{RSF} آزادی مطبوعات & بالای ۵۰ & فاز ۱-۲ \\
ثبت رأی‌دهندگان & درصد واجدین ثبت‌نام‌شده & بالای ۸۰٪ & فاز ۲ \\
\altrow مشارکت انتخاباتی & درصد رأی‌دهندگان & بالای ۶۰٪ & فاز ۲ \\
اعتماد عمومی & نظرسنجی اعتماد به نهادها & بالای ۵۰٪ & فاز ۲-۳ \\
\altrow حقوق بشر & تعداد موارد نقض مستند & کاهش مستمر & همه فازها \\
برابری جنسیتی & درصد زنان در نهادها & بالای ۳۰٪ & فاز ۲-۳ \\
\altrow فساد & امتیاز \lr{TI CPI} & بالای ۴۰ & فاز ۲-۳ \\
رشد اقتصادی & نرخ رشد \lr{GDP} & مثبت & فاز ۱-۳ \\
\altrow انتقال مسئولیت & درصد فعالیت‌ها با مدیریت ایرانی & بالای ۹۰٪ & فاز ۳ \\
\bottomrule
\end{tabularx}
\end{table}

\sectiondivider

% ═══════════════════════════════════════════════════════════════════════════════
\section{مدیریت انتقال}
\label{sec:transition-management}
% ═══════════════════════════════════════════════════════════════════════════════

\subsection{اصل بنیادین: از حضور به مشارکت به مالکیت}
\label{subsec:presence-to-ownership}

\begin{figure}[htbp]
\centering
\begin{tikzpicture}[
    font=\small
]
    % Axes
    \draw[-{Stealth}, thick] (0,0) -- (12,0) node[below] {زمان (ماه)};
    \draw[-{Stealth}, thick] (0,0) -- (0,6) node[above, rotate=90, anchor=south] {سهم مسئولیت (\%)};
    
    % Grid
    \foreach \y in {1,...,5} {
        \draw[VeryLightGray] (0,\y) -- (11.5,\y);
    }
    
    % Y labels
    \node[anchor=east, font=\footnotesize] at (0,1) {۲۰};
    \node[anchor=east, font=\footnotesize] at (0,2) {۴۰};
    \node[anchor=east, font=\footnotesize] at (0,3) {۶۰};
    \node[anchor=east, font=\footnotesize] at (0,4) {۸۰};
    \node[anchor=east, font=\footnotesize] at (0,5) {۱۰۰};
    
    % X labels
    \node[anchor=north, font=\footnotesize] at (1,0) {۶};
    \node[anchor=north, font=\footnotesize] at (3,0) {۱۲};
    \node[anchor=north, font=\footnotesize] at (5,0) {۲۴};
    \node[anchor=north, font=\footnotesize] at (7,0) {۳۶};
    \node[anchor=north, font=\footnotesize] at (9,0) {۴۸};
    \node[anchor=north, font=\footnotesize] at (11,0) {۶۰};
    
    % International line (decreasing)
    \draw[MainBlue, very thick] plot[smooth] coordinates {(0,4.5) (1,4) (3,3.5) (5,2.5) (7,1.5) (9,1) (11,0.3)};
    \node[MainBlue, anchor=west, font=\footnotesize] at (11.2,0.3) {بین‌المللی};
    
    % Iranian line (increasing)
    \draw[MainGreen, very thick] plot[smooth] coordinates {(0,0.5) (1,1) (3,1.5) (5,2.5) (7,3.5) (9,4) (11,4.7)};
    \node[MainGreen, anchor=west, font=\footnotesize] at (11.2,4.7) {ایرانی};
    
    % Crossover point
    \fill[MainRed] (5,2.5) circle (3pt);
    \node[MainRed, anchor=south, font=\footnotesize\bfseries] at (5,2.7) {نقطه تقاطع (ماه ۲۴)};
    
    % Phase labels
    \draw[dashed, DarkGray] (1,0) -- (1,5.5);
    \draw[dashed, DarkGray] (5,0) -- (5,5.5);
    \draw[dashed, DarkGray] (9,0) -- (9,5.5);
    
    \node[font=\footnotesize\bfseries, DarkGray] at (0.5,5.8) {فاز ۱};
    \node[font=\footnotesize\bfseries, DarkGray] at (3,5.8) {فاز ۲};
    \node[font=\footnotesize\bfseries, DarkGray] at (7,5.8) {فاز ۳};
    \node[font=\footnotesize\bfseries, DarkGray] at (10,5.8) {فاز ۴};
    
\end{tikzpicture}
\caption{منحنی انتقال مسئولیت از بین‌المللی به ایرانی}
\label{fig:transition-curve}
\end{figure}

\begin{warningbox}
\textbf{خطر «وابستگی نهادی»}: اگر نهادهای ایرانی به حضور بین‌المللی عادت کنند و ظرفیت مستقل نسازند، خروج ناظران به خلأ می‌انجامد. تجربه افغانستان نشان داد که ۲۰ سال حضور بدون انتقال واقعی مسئولیت، یک‌شبه فرو می‌ریزد. \textbf{انتقال باید از روز اول آغاز شود}، نه فقط در فاز خروج.
\end{warningbox}

\subsection{ابزارهای انتقال مسئولیت}
\label{subsec:transfer-tools}

\begin{table}[htbp]
\centering
\caption{ابزارهای انتقال مسئولیت}
\label{tab:transfer-tools}
\begin{tabularx}{\textwidth}{>{\raggedleft\arraybackslash}p{3cm}
                             >{\raggedleft\arraybackslash}X
                             >{\raggedleft\arraybackslash}p{2cm}
                             >{\raggedleft\arraybackslash}p{2.5cm}}
\toprule
\headerrow ابزار & شرح & فاز شروع & شاخص موفقیت \\
\midrule
آموزش حین خدمت & هر بین‌المللی یک همتای ایرانی آموزش می‌دهد & فاز ۱ & ۹۰٪ پست‌ها ایرانی \\
\altrow مدیریت سایه‌ای & ایرانی ابتدا ناظر، سپس مدیر مشترک، سپس مستقل & فاز ۱ & عملکرد مستقل \\
بورسیه و اعزام & اعزام ایرانیان به مأموریت‌های \lr{UN} در کشورهای دیگر & فاز ۲ & ۵۰۰+ بورسیه \\
\altrow مستندسازی دانش & ثبت رویه‌ها، درس‌آموخته‌ها، راهنماها به فارسی & فاز ۱ & کتابخانه کامل \\
آزمون آمادگی & ارزیابی دوره‌ای توانایی نهاد ایرانی برای مدیریت مستقل & فاز ۲ & قبولی ۸۰٪+ \\
\altrow خروج آزمایشی & ناظران ۱ ماه غایب، عملکرد ارزیابی می‌شود & فاز ۳ & بدون افت عملکرد \\
\bottomrule
\end{tabularx}
\end{table}

\begin{recommendation}
\textbf{قاعده ۳۰-۵۰-۸۰}:
\begin{itemize}[nosep]
    \item \textbf{فاز ۱}: حداقل ۳۰٪ مدیریت توسط ایرانیان
    \item \textbf{فاز ۲}: حداقل ۵۰٪ مدیریت توسط ایرانیان (نقطه تقاطع)
    \item \textbf{فاز ۳}: حداقل ۸۰٪ مدیریت توسط ایرانیان
    \item \textbf{فاز ۴}: ۱۰۰٪ مدیریت ایرانی (بین‌المللی‌ها فقط مشاور)
\end{itemize}
\end{recommendation}

\sectiondivider

% ═══════════════════════════════════════════════════════════════════════════════
\section{مدیریت ارتباطات و اطلاع‌رسانی}
\label{sec:communications-management}
% ═══════════════════════════════════════════════════════════════════════════════

\subsection{استراتژی ارتباطات}
\label{subsec:comms-strategy}

\begin{table}[htbp]
\centering
\caption{مخاطبان و پیام‌های کلیدی}
\label{tab:comms-audiences}
\begin{tabularx}{\textwidth}{>{\raggedleft\arraybackslash}p{2.5cm}
                             >{\raggedleft\arraybackslash}X
                             >{\raggedleft\arraybackslash}p{3cm}}
\toprule
\headerrow مخاطب & پیام کلیدی & کانال اصلی \\
\midrule
مردم ایران & «این گذار متعلق به شماست؛ ما کمک می‌کنیم» & تلویزیون، رادیو، شبکه‌های اجتماعی \\
\altrow نخبگان سیاسی & «فراگیری و مشارکت، نه حذف» & جلسات مستقیم، بیانیه‌ها \\
نیروهای امنیتی & «جایی برای شما در آینده هست — اگر به قانون احترام بگذارید» & کانال‌های خاص، میانجیان \\
\altrow جامعه بین‌المللی & «سرمایه‌گذاری در ثبات ایران، سرمایه‌گذاری در امنیت جهانی» & رسانه‌های بین‌المللی، کنفرانس‌ها \\
دیاسپورا & «تخصص شما ارزشمند است — با مسئولیت مشارکت کنید» & رسانه‌های فارسی‌زبان خارج \\
\altrow مخالفان گذار & «فرایند عادلانه است؛ صدای شما هم شنیده می‌شود» & گفتگوی مستقیم، رسانه \\
\bottomrule
\end{tabularx}
\end{table}

\begin{casestudy}{تونس: قدرت شفافیت}
یکی از عوامل موفقیت نسبی گذار تونس، شفافیت فرایند قانون‌نویسی بود. جلسات مجلس مؤسسان زنده پخش می‌شد. مردم پیش‌نویس قانون اساسی را آنلاین می‌خواندند و نظر می‌دادند. این شفافیت اعتماد عمومی را بالا برد و مقاومت اسلام‌گرایان افراطی را خنثی کرد.
\end{casestudy}

\subsection{مقابله با اطلاعات نادرست}
\label{subsec:counter-disinfo}

\begin{warningbox}
در عصر شبکه‌های اجتماعی، \bilingual{اطلاعات نادرست}{Disinformation} بزرگ‌ترین تهدید اطلاعاتی است. عناصر رژیم قبلی، قدرت‌های خارجی (روسیه، چین)، و گروه‌های افراطی همگی انگیزه و ظرفیت انتشار اطلاعات نادرست دارند:
\begin{itemize}[nosep]
    \item «گذار توطئه خارجی است»
    \item «ناظران جاسوس‌اند»
    \item «انتخابات تقلبی بود»
    \item «فلان قوم می‌خواهد جدا شود»
\end{itemize}
\end{warningbox}

\textbf{راهکارهای مقابله:}
\begin{enumerate}[nosep]
    \item تیم واکنش سریع رسانه‌ای (پاسخ در کمتر از ۱ ساعت)
    \item همکاری با پلتفرم‌های اجتماعی برای حذف محتوای مخرب
    \item شبکه «بررسی واقعیت» (\lr{Fact-Checking}) فارسی‌زبان
    \item آموزش سواد رسانه‌ای عمومی
    \item شفافیت حداکثری خود مأموریت (بهترین پادزهر شایعه)
\end{enumerate}

\sectiondivider

% ═══════════════════════════════════════════════════════════════════════════════
\section{خلاصه زمانی یکپارچه}
\label{sec:integrated-timeline}
% ═══════════════════════════════════════════════════════════════════════════════

\begin{table}[htbp]
\centering
\caption{خلاصه نقاط عطف کلیدی}
\label{tab:milestones-summary}
\begin{tabularx}{\textwidth}{>{\centering\arraybackslash}p{2cm}
                             >{\raggedleft\arraybackslash}p{4cm}
                             >{\raggedleft\arraybackslash}X}
\toprule
\headerrow ماه & نقطه عطف & شاخص تحقق \\
\midrule
ماه ۰ & \emphred{سقوط رژیم / آغاز گذار} & رویداد سیاسی \\
\altrow ساعت ۷۲ & اعزام تیم ارزیابی & تیم در تهران \\
هفته ۲ & تصویب قطعنامه شورای امنیت & متن مصوب \\
\altrow ماه ۱ & استقرار \lr{SRSG} و تیم پیشتاز & دفتر فعال \\
ماه ۳ & ۳۱ دفتر استانی فعال & پوشش سراسری \\
\altrow ماه ۶ & پایان فاز تثبیت & امنیت نسبی، قوانین موقت \\
ماه ۱۲ & انتخابات مجلس مؤسسان & مجلس تشکیل‌شده \\
\altrow ماه ۲۴ & رفراندوم قانون اساسی & تصویب مردمی \\
ماه ۳۰ & انتخابات پارلمان & پارلمان منتخب \\
\altrow ماه ۳۳ & انتقال قدرت به دولت منتخب & مراسم تحلیف \\
ماه ۴۸ & انتخابات دوم (محلی/پارلمانی) & انتقال مسالمت‌آمیز \\
\altrow ماه ۶۰ & تبدیل مأموریت به دفتر سیاسی & کاهش ۸۰٪ حضور \\
ماه ۱۲۰ & \emphgreen{خروج کامل} & پایان مأموریت \\
\bottomrule
\end{tabularx}
\end{table}

\sectiondivider

% ═══════════════════════════════════════════════════════════════════════════════
% جمع‌بندی فصل
% ═══════════════════════════════════════════════════════════════════════════════

\begin{chaptersummary}
یافته‌های کلیدی این فصل:

\begin{enumerate}
    \item \textbf{فاز ۰ (پیش‌گذار) مهم‌ترین فاز است}: آمادگی قبل از بحران تفاوت میان موفقیت و فاجعه است. تشکیل تیم برنامه‌ریزی، شناسایی \lr{SRSG}، و پیش‌موقعیت‌یابی باید \emph{الان} آغاز شود.
    
    \item \textbf{۷۲ ساعت اول سرنوشت‌ساز است}: اعزام تیم ارزیابی، بیانیه دبیرکل، و نشست شورای امنیت باید در سه روز اول اتفاق بیفتد.
    
    \item \textbf{مدل «اول قانون اساسی، بعد انتخابات»}: تجربه تونس نشان داد که صبر برای تدوین قانون اساسی توافقی، از انتخابات عجولانه (مدل مصر/عراق) بهتر است.
    
    \item \textbf{خروج بر اساس شاخص، نه تقویم}: معیارهای خروج باید از روز اول تعریف و پایش شوند. خروج زودهنگام به‌اندازه ماندن بیش‌ازحد خطرناک است.
    
    \item \textbf{قاعده ۳۰-۵۰-۸۰}: سهم مدیریت ایرانی باید از ۳۰٪ در فاز ۱ به ۱۰۰٪ در فاز ۴ برسد. انتقال از روز اول آغاز می‌شود.
    
    \item \textbf{ماتریس \lr{RACI} ابهام نقش‌ها را رفع می‌کند}: تعیین دقیق مسئول اجرا، پاسخگو، مشاور، و اطلاع‌یافته برای هر فعالیت.
    
    \item \textbf{مکانیزم بازخورد پنج‌سطحی}: از پایش روزانه تا بازنگری استراتژیک سالانه، با شاخص‌های عینی.
    
    \item \textbf{ارتباطات استراتژیک حیاتی است}: پیام‌های متناسب با هر مخاطب، شفافیت حداکثری، و مقابله فعال با اطلاعات نادرست.
\end{enumerate}

\vspace{0.5cm}
\textit{در فصل بعد (\ref{ch:budget})، بودجه‌بندی تفصیلی، منابع مالی، و مکانیزم شفافیت مالی بررسی خواهد شد.}
\end{chaptersummary}

\chapterend