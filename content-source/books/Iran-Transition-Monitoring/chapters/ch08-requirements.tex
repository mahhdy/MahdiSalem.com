% ═══════════════════════════════════════════════════════════════════════════════
% فصل ۸: نیازمندی‌ها — انسانی، نهادی، فنی، حقوقی
% فایل: chapters/ch08-requirements.tex
% رنگ فصل: زرد (MainYellow)
% ═══════════════════════════════════════════════════════════════════════════════

\chapteropening{۸}{نیازمندی‌ها: انسانی، نهادی، فنی، حقوقی}{MainYellow}{%
بدون ابزار مناسب، حتی بهترین معمار هم نمی‌تواند خانه‌ای بسازد. نظارت مؤثر نیازمند منابع مؤثر است.%
}{ضرب‌المثل}

\chapter{نیازمندی‌ها: انسانی، نهادی، فنی، حقوقی}
\label{ch:requirements}

\minitoc

% ─────────────────────────────────────────────────────────────────────────────
% خلاصه اجرایی
% ─────────────────────────────────────────────────────────────────────────────

\begin{executivesummary}
اجرای مؤثر نظارت بین‌المللی بر گذار ایران نیازمند چهار دسته منابع است: \emph{نیروی انسانی} (۶,۰۰۰-۱۲,۰۰۰ بین‌المللی و ۲۰,۰۰۰-۵۰,۰۰۰ ایرانی در اوج استقرار)، \emph{ساختارهای نهادی} (از دفتر نماینده ویژه تا کمیسیون‌های تخصصی)، \emph{زیرساخت‌های فنی} (ارتباطات، سامانه‌های اطلاعاتی، امنیت سایبری)، و \emph{چارچوب حقوقی} (قطعنامه‌ها، توافق‌نامه‌ها، قوانین موقت). این فصل برای هر دسته، جزئیات، استانداردها، و برآورد کمّی ارائه می‌دهد. تأکید اصلی بر تعادل میان ظرفیت بین‌المللی و مالکیت ملی، و انتقال تدریجی مسئولیت به نهادهای ایرانی است.
\end{executivesummary}

\section{درآمد: از طراحی تا اجرا}
\label{sec:req-intro}

فصول پیشین چارچوب مفهومی، سناریوها، بازیگران، تضمین‌ها و ریسک‌ها را بررسی کردند. اکنون به پرسش عملیاتی می‌رسیم: \emph{برای اجرای این طرح، دقیقاً چه چیزهایی لازم است؟}

\begin{keypoint}
تفاوت میان طرح‌های موفق و ناموفق، اغلب نه در طراحی بلکه در اجراست. طراحی بدون منابع کافی، آرزوپردازی است. منابع بدون طراحی، هدررفت. این فصل پل میان این دو را می‌سازد.
\end{keypoint}

نیازمندی‌ها در چهار حوزه دسته‌بندی می‌شوند:

\begin{enumerate}[nosep]
    \item \textbf{انسانی}: چه کسانی، چند نفر، با چه تخصص‌هایی
    \item \textbf{نهادی}: چه ساختارها و سازمان‌هایی باید ایجاد شوند
    \item \textbf{فنی}: چه زیرساخت‌ها و ابزارهایی لازم است
    \item \textbf{حقوقی}: چه چارچوب قانونی و توافق‌نامه‌هایی ضروری است
\end{enumerate}

\sectiondivider

% ═══════════════════════════════════════════════════════════════════════════════
\section{نیازمندی‌های انسانی}
\label{sec:human-requirements}
% ═══════════════════════════════════════════════════════════════════════════════

\subsection{برآورد کلی نیروی انسانی}
\label{subsec:hr-overview}

بر اساس تجربه مأموریت‌های مشابه و با در نظر گرفتن ویژگی‌های ایران (جمعیت ۸۵ میلیون، مساحت ۱.۶ میلیون کیلومتر مربع، ۳۱ استان)، برآورد نیروی انسانی به شرح زیر است:

\begin{table}[htbp]
\centering
\caption{برآورد کلی نیروی انسانی بر حسب فاز}
\label{tab:hr-overview}
\begin{tabularx}{\textwidth}{>{\raggedleft\arraybackslash}p{3cm}
                             >{\centering\arraybackslash}p{2.5cm}
                             >{\centering\arraybackslash}p{2.5cm}
                             >{\centering\arraybackslash}p{2.5cm}
                             >{\centering\arraybackslash}X}
\toprule
\headerrow دسته & فاز ۱ (۱-۶ ماه) & فاز ۲ (۶-۲۴ ماه) & فاز ۳ (۲۴-۶۰ ماه) & توضیح \\
\midrule
کارکنان بین‌المللی & ۳,۰۰۰-۵,۰۰۰ & ۶,۰۰۰-۱۲,۰۰۰ & ۱,۵۰۰-۳,۰۰۰ & اوج در فاز ۲ (انتخابات) \\
\altrow کارکنان ایرانی (مستقیم) & ۵,۰۰۰-۱۰,۰۰۰ & ۲۰,۰۰۰-۵۰,۰۰۰ & ۱۰,۰۰۰-۲۰,۰۰۰ & ناظران، مترجمان، پشتیبان \\
ناظران انتخاباتی موقت & --- & ۵۰,۰۰۰-۱۰۰,۰۰۰ & --- & فقط روز انتخابات \\
\altrow مشاوران کوتاه‌مدت & ۵۰۰-۱,۰۰۰ & ۱,۰۰۰-۲,۰۰۰ & ۵۰۰-۱,۰۰۰ & مأموریت‌های خاص \\
\midrule
\textbf{مجموع (بدون موقت)} & \textbf{۸,۵۰۰-۱۶,۰۰۰} & \textbf{۲۷,۰۰۰-۶۴,۰۰۰} & \textbf{۱۲,۰۰۰-۲۴,۰۰۰} & \\
\bottomrule
\end{tabularx}
\end{table}

\begin{lessonlearned}{تیمور شرقی: نسبت جمعیت به ناظر}
در \lr{UNTAET} تیمور شرقی (جمعیت ۸۰۰,۰۰۰)، حدود ۱۱,۰۰۰ کارمند بین‌المللی و محلی حضور داشتند — نسبت ۱ به ۷۳. برای ایران با همین نسبت، بیش از ۱ میلیون نفر لازم بود که نه ممکن است و نه مطلوب. نسبت واقع‌بینانه برای ایران: ۱ به ۱,۵۰۰-۳,۰۰۰ (با تأکید بر ظرفیت‌سازی محلی).
\end{lessonlearned}

\subsection{تفکیک تخصصی نیروی انسانی بین‌المللی}
\label{subsec:hr-specialization}

\begin{table}[htbp]
\centering
\caption{تفکیک تخصصی کارکنان بین‌المللی (فاز ۲ — اوج)}
\label{tab:hr-specialization}
\begin{tabularx}{\textwidth}{>{\raggedleft\arraybackslash}p{0.5cm}
                             >{\raggedleft\arraybackslash}p{3.5cm}
                             >{\centering\arraybackslash}p{2cm}
                             >{\centering\arraybackslash}p{1.5cm}
                             >{\raggedleft\arraybackslash}X}
\toprule
\headerrow \# & حوزه تخصصی & تعداد & درصد & وظایف اصلی \\
\midrule
۱ & نظارت انتخاباتی & ۲,۵۰۰-۵,۰۰۰ & ۴۰٪ & نظارت بر ثبت‌نام، رأی‌گیری، شمارش \\
\altrow ۲ & حقوق بشر & ۸۰۰-۱,۵۰۰ & ۱۲٪ & پایش، مستندسازی، گزارش‌دهی \\
۳ & اصلاح بخش امنیتی & ۵۰۰-۱,۰۰۰ & ۸٪ & مشاوره به ارتش/پلیس، نظارت بر \lr{DDR} \\
\altrow ۴ & حاکمیت قانون/قضایی & ۴۰۰-۸۰۰ & ۶٪ & اصلاح قضایی، عدالت انتقالی \\
۵ & اقتصادی/مالی & ۳۰۰-۶۰۰ & ۵٪ & نظارت مالی، ضدفساد، بانکداری \\
\altrow ۶ & رسانه/ارتباطات & ۲۰۰-۴۰۰ & ۳٪ & نظارت رسانه، آموزش روزنامه‌نگاران \\
۷ & جامعه مدنی/جنسیت & ۲۰۰-۴۰۰ & ۳٪ & توانمندسازی \lr{NGO}ها، زنان، جوانان \\
\altrow ۸ & اداری/مالی/لجستیک & ۸۰۰-۱,۵۰۰ & ۱۲٪ & پشتیبانی عملیاتی \\
۹ & امنیت/حفاظت & ۴۰۰-۸۰۰ & ۶٪ & امنیت کارکنان و تأسیسات \\
\altrow ۱۰ & ارتباطات/\lr{IT} & ۲۰۰-۴۰۰ & ۳٪ & زیرساخت فنی، امنیت سایبری \\
۱۱ & هماهنگی/سیاست‌گذاری & ۲۰۰-۴۰۰ & ۳٪ & دفتر \lr{SRSG}، تحلیل، برنامه‌ریزی \\
\midrule
& \textbf{مجموع} & \textbf{۶,۵۰۰-۱۲,۸۰۰} & \textbf{۱۰۰٪} & \\
\bottomrule
\end{tabularx}
\end{table}

\subsection{معیارهای استخدام کارکنان بین‌المللی}
\label{subsec:hr-criteria}

\begin{recommendation}
معیارهای کلیدی برای انتخاب کارکنان بین‌المللی:
\begin{enumerate}[nosep]
    \item \textbf{تجربه مرتبط}: حداقل ۵ سال در حوزه تخصصی، ترجیحاً در کشورهای مسلمان یا خاورمیانه
    \item \textbf{مهارت زبانی}: فارسی (ترجیحی)، انگلیسی (الزامی)، عربی یا ترکی (مفید)
    \item \textbf{حساسیت فرهنگی}: آموزش اجباری پیش از استقرار + ارزیابی
    \item \textbf{سلامت و آمادگی}: توانایی کار در شرایط سخت
    \item \textbf{تعهد اخلاقی}: پیشینه پاک، امضای کد رفتاری
    \item \textbf{تنوع}: حداقل ۴۰٪ زن، حداکثر ۱۵٪ از هر کشور واحد
\end{enumerate}
\end{recommendation}

\subsection{نیروی انسانی ایرانی}
\label{subsec:iranian-staff}

نیروی انسانی ایرانی ستون فقرات عملیات است و بدون آن، هیچ نظارتی امکان‌پذیر نیست.

\begin{table}[htbp]
\centering
\caption{دسته‌بندی نیروی انسانی ایرانی}
\label{tab:iranian-staff}
\begin{tabularx}{\textwidth}{>{\raggedleft\arraybackslash}p{3cm}
                             >{\centering\arraybackslash}p{2.5cm}
                             >{\raggedleft\arraybackslash}X
                             >{\raggedleft\arraybackslash}p{3cm}}
\toprule
\headerrow دسته & تعداد (فاز ۲) & وظایف & منبع جذب \\
\midrule
مترجمان & ۲,۰۰۰-۴,۰۰۰ & ترجمه همزمان و کتبی & دانشگاه‌ها، دیاسپورا \\
\altrow ناظران محلی (دائم) & ۵,۰۰۰-۱۰,۰۰۰ & پایش روزانه، گزارش‌دهی & جامعه مدنی، دانشجویان \\
ناظران انتخابات (موقت) & ۵۰,۰۰۰-۱۰۰,۰۰۰ & نظارت روز رأی‌گیری & داوطلب عمومی \\
\altrow کارشناسان فنی & ۱,۰۰۰-۲,۰۰۰ & حقوقی، مالی، \lr{IT} & متخصصان داخل و دیاسپورا \\
پشتیبانی اداری & ۳,۰۰۰-۶,۰۰۰ & دفتری، لجستیک، رانندگی & بازار کار محلی \\
\altrow امنیت محلی & ۲,۰۰۰-۴,۰۰۰ & حفاظت تأسیسات & نیروهای امنیتی اصلاح‌شده \\
رابطان اجتماعی & ۱,۰۰۰-۲,۰۰۰ & ارتباط با جوامع محلی & فعالان مدنی، معتمدان \\
\midrule
\textbf{مجموع} & \textbf{۶۴,۰۰۰-۱۲۸,۰۰۰} & & \\
\bottomrule
\end{tabularx}
\end{table}

\begin{warningbox}
\textbf{چالش غربالگری}: چگونه می‌توان اطمینان یافت که کارکنان ایرانی وابسته به نظام قبلی نیستند؟ غربالگری افراطی (مثل بعث‌زدایی عراق) فاجعه‌بار است، اما بدون غربالگری، نفوذ محتمل است.

\textbf{راه‌حل پیشنهادی}: 
\begin{itemize}[nosep]
    \item غربالگری فقط برای پست‌های حساس
    \item تمرکز بر رفتار آینده نه سوابق صرف
    \item دوره آزمایشی با نظارت
    \item مکانیزم گزارش‌دهی و اخراج سریع
\end{itemize}
\end{warningbox}

\subsection{نقش دیاسپورا در تأمین نیروی انسانی}
\label{subsec:diaspora-hr}

دیاسپورای ایرانی (۴-۵ میلیون نفر) منبع ارزشمندی از نیروی متخصص دوزبانه است.

\begin{table}[htbp]
\centering
\caption{ظرفیت‌ها و محدودیت‌های نیروی دیاسپورا}
\label{tab:diaspora-hr}
\begin{tabularx}{\textwidth}{>{\centering\arraybackslash}p{0.8cm}
                             >{\raggedleft\arraybackslash}X
                             >{\raggedleft\arraybackslash}X}
\toprule
\headerrow & مزایا & محدودیت‌ها/ریسک‌ها \\
\midrule
\cmark & تسلط به فارسی و زبان‌های بین‌المللی & فاصله از واقعیت روزمره داخل \\
\altrow \cmark & آشنایی با فرهنگ ایرانی و بین‌المللی & ممکن است وابستگی جناحی داشته باشند \\
\cmark & تحصیلات و تجربه در نهادهای معتبر & انتظارات حقوقی بالاتر \\
\altrow \cmark & انگیزه بالا برای خدمت & ممکن است تنش با نیروی داخلی ایجاد کنند \\
\cmark & شبکه‌های بین‌المللی & تعهد بلندمدت به ماندن نامطمئن \\
\bottomrule
\end{tabularx}
\end{table}

\begin{recommendation}
سیاست پیشنهادی برای دیاسپورا:
\begin{itemize}[nosep]
    \item سقف ۲۰-۳۰٪ از کارکنان ایرانی از دیاسپورا
    \item ترجیح برای پست‌های فنی و مشاوره‌ای (نه اجرایی ارشد)
    \item تعهد حداقل ۲ ساله برای پست‌های کلیدی
    \item مکانیزم انتقال دانش به نیروی داخلی
\end{itemize}
\end{recommendation}

\sectiondivider

% ═══════════════════════════════════════════════════════════════════════════════
\section{نیازمندی‌های نهادی}
\label{sec:institutional-requirements}
% ═══════════════════════════════════════════════════════════════════════════════

ساختارهای نهادی چارچوبی فراهم می‌آورند که در آن افراد می‌توانند مؤثر عمل کنند.

\subsection{ساختار کلان نهادی}
\label{subsec:institutional-structure}

\begin{figure}[htbp]
\centering
\begin{tikzpicture}[
    node distance=1.2cm,
    every node/.style={font=\small, align=center},
    top/.style={rectangle, rounded corners, draw=MainPurple, fill=LightPurple, minimum width=4cm, minimum height=1cm},
    mid/.style={rectangle, rounded corners, draw=MainBlue, fill=LightBlue, minimum width=3.5cm, minimum height=0.9cm},
    low/.style={rectangle, rounded corners, draw=MainGreen, fill=LightGreen, minimum width=3cm, minimum height=0.8cm},
    support/.style={rectangle, rounded corners, draw=MainOrange, fill=LightOrange, minimum width=2.5cm, minimum height=0.8cm},
    arrow/.style={-{Stealth[length=2.5mm]}, thick}
]
    % Top level
    \node[top] (sc) {شورای امنیت \lr{UN}};
    \node[top, right=2cm of sc] (sg) {دبیرکل \lr{UN}};
    
    % SRSG level
    \node[mid, below=1.5cm of sg] (srsg) {\lr{SRSG}\\نماینده ویژه دبیرکل};
    
    % Deputy level
    \node[low, below left=1.5cm and 2cm of srsg] (d1) {معاون سیاسی};
    \node[low, below=1.5cm of srsg] (d2) {معاون عملیاتی};
    \node[low, below right=1.5cm and 2cm of srsg] (d3) {معاون حقوق بشر};
    
    % Pillars
    \node[support, below=1.2cm of d1] (p1) {انتخابات};
    \node[support, left=0.3cm of p1] (p2) {امنیت};
    \node[support, below=1.2cm of d2] (p3) {اقتصاد};
    \node[support, right=0.3cm of p3] (p4) {قضایی};
    \node[support, below=1.2cm of d3] (p5) {رسانه};
    \node[support, right=0.3cm of p5] (p6) {مدنی};
    
    % Arrows
    \draw[arrow] (sc) -- (sg);
    \draw[arrow] (sg) -- (srsg);
    \draw[arrow] (srsg) -- (d1);
    \draw[arrow] (srsg) -- (d2);
    \draw[arrow] (srsg) -- (d3);
    \draw[arrow] (d1) -- (p1);
    \draw[arrow] (d1) -- (p2);
    \draw[arrow] (d2) -- (p3);
    \draw[arrow] (d2) -- (p4);
    \draw[arrow] (d3) -- (p5);
    \draw[arrow] (d3) -- (p6);
    
    % Side boxes
    \node[mid, right=3cm of srsg] (contact) {گروه تماس\\بین‌المللی};
    \node[mid, below=0.5cm of contact] (consult) {شورای مشورتی\\ایرانی};
    
    \draw[arrow, dashed] (srsg) -- (contact);
    \draw[arrow, dashed] (srsg) -- (consult);
    
\end{tikzpicture}
\caption{ساختار کلان نهادی نظارت بین‌المللی}
\label{fig:institutional-structure}
\end{figure}

\subsection{نهادهای اصلی مورد نیاز}
\label{subsec:key-institutions}

\subsubsection{دفتر نماینده ویژه دبیرکل (\lr{SRSG})}

\begin{table}[htbp]
\centering
\caption{مشخصات دفتر \lr{SRSG}}
\label{tab:srsg-office}
\begin{tabularx}{\textwidth}{>{\raggedleft\arraybackslash}p{3.5cm}
                             >{\raggedleft\arraybackslash}X}
\toprule
\headerrow ویژگی & شرح \\
\midrule
عنوان رسمی & \lr{United Nations Mission for Oversight of Iran's Transition (UNMOIT)} \\
\altrow رئیس & نماینده ویژه دبیرکل (\lr{SRSG}) در سطح معاون دبیرکل \\
معاونان & ۳-۵ معاون (سیاسی، عملیاتی، حقوق بشر، هماهنگی، اداری) \\
\altrow کارکنان دفتر مرکزی & ۳۰۰-۵۰۰ نفر \\
دفاتر منطقه‌ای & ۳۱ دفتر استانی + ۵-۷ دفتر منطقه‌ای \\
\altrow محل استقرار مرکزی & تهران (ترجیحاً در ساختمان‌های موجود، نه پایگاه ایزوله) \\
مدت مأموریت اولیه & ۲ سال با امکان تمدید \\
\altrow گزارش‌دهی & دبیرکل → شورای امنیت (ماهانه) \\
\bottomrule
\end{tabularx}
\end{table}

\begin{casestudy}{انتخاب \lr{SRSG}: درس‌های گذشته}
انتخاب \lr{SRSG} تأثیر عمیقی بر موفقیت مأموریت دارد:
\begin{itemize}[nosep]
    \item \textbf{موفق}: \person{سرجیو ویرا دملو}{Sergio Vieira de Mello} در تیمور شرقی — دیپلمات باتجربه، کاریزماتیک، محترم
    \item \textbf{ناموفق}: \person{پل برمر}{Paul Bremer} در عراق — بدون تجربه منطقه، تصمیمات یکجانبه
\end{itemize}
\textbf{معیارهای پیشنهادی}: تجربه خاورمیانه/جهان اسلام، زبان (ترجیحاً فارسی یا عربی)، اعتبار در میان همه طرف‌ها، تعهد به مالکیت ملی، سابقه موفق در مأموریت‌های مشابه.
\end{casestudy}

\subsubsection{کمیسیون مستقل انتخابات}

\begin{table}[htbp]
\centering
\caption{ساختار کمیسیون انتخابات}
\label{tab:election-commission}
\begin{tabularx}{\textwidth}{>{\raggedleft\arraybackslash}p{3cm}
                             >{\raggedleft\arraybackslash}X}
\toprule
\headerrow عنصر & شرح \\
\midrule
ترکیب & ۷-۱۱ عضو: اکثریت ایرانی + ۲-۳ بین‌المللی (بدون حق رأی) \\
\altrow انتخاب اعضا & پیشنهاد شورای مشورتی، تأیید مجلس موقت/انتقالی \\
استقلال & بودجه مستقل، مصونیت از فشار سیاسی \\
\altrow وظایف & ثبت رأی‌دهندگان، تأیید نامزدها، اجرای انتخابات، رسیدگی به شکایات \\
کارکنان & ۵,۰۰۰-۱۰,۰۰۰ دائم + ۱۰۰,۰۰۰+ موقت (روز انتخابات) \\
\altrow نظارت بین‌المللی & مشاوران \lr{UNDP}/\lr{IFES} در همه سطوح \\
\bottomrule
\end{tabularx}
\end{table}

\subsubsection{کمیسیون حقیقت‌یابی و آشتی}

\begin{keypoint}
کمیسیون حقیقت نه جایگزین دادگاه است و نه مانع آن. وظیفه‌اش کشف حقیقت، شنیدن صدای قربانیان، و ایجاد روایت مشترک ملی است. تجربه آفریقای جنوبی نشان داد که عفو مشروط (در ازای اعتراف کامل) می‌تواند به آشتی کمک کند.
\end{keypoint}

\begin{table}[htbp]
\centering
\caption{ساختار کمیسیون حقیقت‌یابی}
\label{tab:truth-commission}
\begin{tabularx}{\textwidth}{>{\raggedleft\arraybackslash}p{3cm}
                             >{\raggedleft\arraybackslash}X}
\toprule
\headerrow عنصر & شرح \\
\midrule
ترکیب & ۱۵-۲۱ عضو (کاملاً ایرانی) + هیئت مشاوران بین‌المللی \\
\altrow معیار انتخاب & اعتبار اخلاقی، استقلال از همه جناح‌ها، تخصص (حقوقدان، روانشناس، مورخ، فعال حقوق بشر) \\
محدوده زمانی & بررسی نقض حقوق بشر از ۱۳۵۷ تا پایان رژیم \\
\altrow مدت فعالیت & ۳-۵ سال \\
اختیارات & احضار شهود، دسترسی به اسناد، توصیه عفو مشروط \\
\altrow خروجی & گزارش نهایی عمومی + توصیه‌ها برای جبران و اصلاحات \\
\bottomrule
\end{tabularx}
\end{table}

\subsubsection{دادگاه ویژه برای جنایات سنگین}

\begin{warningbox}
دادگاه‌های ویژه می‌توانند به عدالت یا انتقام بینجامند. تفاوت در طراحی است:
\begin{itemize}[nosep]
    \item \textbf{موفق}: دادگاه‌های رواندا و سیرالئون — استانداردهای بین‌المللی، وکیل مدافع، شفافیت
    \item \textbf{ناموفق}: دادگاه‌های انقلابی ایران ۱۳۵۷ — بدون وکیل، محاکمات چنددقیقه‌ای، اعدام فوری
\end{itemize}
\end{warningbox}

\begin{table}[htbp]
\centering
\caption{ساختار دادگاه ویژه}
\label{tab:special-tribunal}
\begin{tabularx}{\textwidth}{>{\raggedleft\arraybackslash}p{3cm}
                             >{\raggedleft\arraybackslash}X}
\toprule
\headerrow عنصر & شرح \\
\midrule
صلاحیت & جنایات علیه بشریت، شکنجه سیستماتیک، کشتار جمعی (۱۳۶۷ و...) \\
\altrow ترکیب قضات & ترکیبی: ۳ ایرانی + ۲ بین‌المللی در هر شعبه \\
دادستان & ایرانی با معاون بین‌المللی \\
\altrow استانداردها & مطابق اساسنامه رم (حقوق متهم، وکیل، استیناف) \\
تعداد پرونده & تمرکز بر ۱۰۰-۵۰۰ مسئول اصلی (نه هزاران نفر) \\
\altrow مجازات‌ها & حبس (بدون اعدام — مطابق استانداردهای بین‌المللی) \\
\bottomrule
\end{tabularx}
\end{table}

\subsubsection{صندوق امانی بین‌المللی}

\begin{table}[htbp]
\centering
\caption{ساختار صندوق امانی}
\label{tab:trust-fund}
\begin{tabularx}{\textwidth}{>{\raggedleft\arraybackslash}p{3cm}
                             >{\raggedleft\arraybackslash}X}
\toprule
\headerrow عنصر & شرح \\
\midrule
هدف & مدیریت شفاف کمک‌های بین‌المللی و دارایی‌های آزادشده ایران \\
\altrow مدیریت & هیئت امنای مشترک (ایرانی-بین‌المللی) \\
حسابرس & شرکت حسابرسی بین‌المللی مستقل \\
\altrow شفافیت & انتشار ماهانه گزارش مالی عمومی \\
حوزه‌های هزینه & بازسازی، انتخابات، عدالت انتقالی، خدمات عمومی \\
\altrow کنترل & هیچ برداشتی بدون تأیید مشترک ایرانی-بین‌المللی \\
\bottomrule
\end{tabularx}
\end{table}

\subsubsection{سایر نهادهای ضروری}

\begin{itemize}[nosep]
    \item \textbf{شورای مشورتی ایرانی}: نمایندگان همه طیف‌ها، مشورت با \lr{SRSG}
    \item \textbf{کمیسیون رسانه}: تنظیم‌گری، صدور مجوز، نظارت بر تعادل
    \item \textbf{کمیسیون ضدفساد}: پیشگیری، تحقیق، آموزش
    \item \textbf{نهاد حقوق اقوام}: تضمین حقوق زبانی-فرهنگی
    \item \textbf{دفتر جبران خسارت قربانیان}: مالی و نمادین
\end{itemize}

\sectiondivider

% ═══════════════════════════════════════════════════════════════════════════════
\section{نیازمندی‌های فنی}
\label{sec:technical-requirements}
% ═══════════════════════════════════════════════════════════════════════════════

\subsection{زیرساخت ارتباطات}
\label{subsec:communications}

\begin{table}[htbp]
\centering
\caption{نیازمندی‌های ارتباطی}
\label{tab:comms-requirements}
\begin{tabularx}{\textwidth}{>{\raggedleft\arraybackslash}p{3cm}
                             >{\raggedleft\arraybackslash}X
                             >{\raggedleft\arraybackslash}p{3cm}}
\toprule
\headerrow حوزه & نیاز & اولویت \\
\midrule
اینترنت & رفع فیلترینگ، پهنای باند کافی & فوری (روز ۱) \\
\altrow تلفن همراه & پوشش سراسری، امنیت مکالمات & فوری \\
شبکه داخلی \lr{UN} & \lr{VPN} امن، ویدئوکنفرانس & هفته اول \\
\altrow رادیو & ارتباط اضطراری، مناطق دوردست & ماه اول \\
ماهواره & پشتیبان اینترنت، مناطق بدون پوشش & ماه اول \\
\bottomrule
\end{tabularx}
\end{table}

\begin{lessonlearned}{افغانستان: اهمیت ارتباطات}
در انتخابات ۲۰۰۴ افغانستان، قطع ارتباط با صدها ایستگاه رأی‌گیری در مناطق دوردست، امکان تقلب را فراهم کرد و اعتماد به نتایج را تضعیف کرد. سیستم ارتباط چندلایه (اینترنت + تلفن + رادیو + ماهواره) ضروری است.
\end{lessonlearned}

\subsection{سامانه‌های اطلاعاتی}
\label{subsec:information-systems}

\begin{table}[htbp]
\centering
\caption{سامانه‌های اطلاعاتی مورد نیاز}
\label{tab:info-systems}
\begin{tabularx}{\textwidth}{>{\raggedleft\arraybackslash}p{3.5cm}
                             >{\raggedleft\arraybackslash}X
                             >{\raggedleft\arraybackslash}p{2.5cm}}
\toprule
\headerrow سامانه & کارکرد & فاز استقرار \\
\midrule
ثبت رأی‌دهندگان & پایگاه داده ۶۰+ میلیون واجد شرایط & فاز ۱ \\
\altrow مدیریت انتخابات & برنامه‌ریزی، توزیع مواد، گزارش‌دهی & فاز ۱-۲ \\
مستندسازی حقوق بشر & ثبت شهادت‌ها، شواهد، پرونده‌ها & فاز ۱ \\
\altrow هشدار زودهنگام & پایش شاخص‌ها، تحلیل ریسک & فاز ۱ \\
مدیریت مالی & حسابداری، شفافیت، گزارش‌دهی & فاز ۱ \\
\altrow مدیریت منابع انسانی & استخدام، آموزش، ارزیابی & فاز ۱ \\
گزارش‌دهی عمومی & اطلاع‌رسانی به مردم و رسانه‌ها & فاز ۱ \\
\bottomrule
\end{tabularx}
\end{table}

\subsection{امنیت سایبری}
\label{subsec:cybersecurity}

\begin{warningbox}
ایران دارای ظرفیت سایبری تهاجمی قابل‌توجه است (گروه‌هایی مثل \lr{APT33}, \lr{APT34}). در صورت گذار، عناصر مقاوم ممکن است از حملات سایبری برای اختلال استفاده کنند:
\begin{itemize}[nosep]
    \item حمله به سامانه انتخابات
    \item نشت اطلاعات محرمانه شهود
    \item اختلال در ارتباطات
    \item انتشار اطلاعات جعلی
\end{itemize}
\end{warningbox}

\textbf{راهکارهای امنیت سایبری:}
\begin{enumerate}[nosep]
    \item سامانه‌های حیاتی آفلاین یا ایزوله
    \item رمزنگاری سرتاسری همه ارتباطات
    \item پشتیبان‌گیری چندلایه
    \item تیم واکنش سریع سایبری (\lr{CERT})
    \item همکاری با شرکت‌های امنیتی معتبر
    \item آموزش امنیت سایبری به همه کارکنان
\end{enumerate}

\subsection{فناوری انتخابات}
\label{subsec:election-tech}

\begin{keypoint}
انتخاب میان رأی‌گیری کاغذی و الکترونیکی تصمیمی استراتژیک است. در شرایط گذار با اعتماد پایین، \textbf{رأی کاغذی با شمارش علنی} معمولاً امن‌تر و قابل‌اعتمادتر است.
\end{keypoint}

\begin{table}[htbp]
\centering
\caption{مقایسه گزینه‌های فناوری انتخابات}
\label{tab:election-tech}
\begin{tabularx}{\textwidth}{>{\raggedleft\arraybackslash}p{2.5cm}
                             >{\raggedleft\arraybackslash}X
                             >{\raggedleft\arraybackslash}X
                             >{\centering\arraybackslash}p{2cm}}
\toprule
\headerrow گزینه & مزایا & معایب & توصیه \\
\midrule
کاملاً کاغذی & شفافیت، سادگی، قابل شمارش مجدد & کُند، خطای انسانی & \cmark توصیه \\
\altrow الکترونیک + کاغذ & سرعت، کاهش خطا، پشتیبان کاغذی & هزینه، پیچیدگی & قابل بررسی \\
کاملاً الکترونیک & سرعت بالا & ریسک هک، عدم قابلیت بازشماری & \xmark رد \\
\altrow رأی اینترنتی & دسترسی آسان & ریسک امنیتی بالا & \xmark رد \\
\bottomrule
\end{tabularx}
\end{table}

\sectiondivider

% ═══════════════════════════════════════════════════════════════════════════════
\section{نیازمندی‌های حقوقی}
\label{sec:legal-requirements}
% ═══════════════════════════════════════════════════════════════════════════════

\subsection{چارچوب بین‌المللی}
\label{subsec:international-legal}

\subsubsection{قطعنامه شورای امنیت}

\begin{table}[htbp]
\centering
\caption{عناصر کلیدی قطعنامه پیشنهادی شورای امنیت}
\label{tab:unsc-resolution}
\begin{tabularx}{\textwidth}{>{\raggedleft\arraybackslash}p{3cm}
                             >{\raggedleft\arraybackslash}X}
\toprule
\headerrow عنصر & محتوا \\
\midrule
فصل & ترجیحاً فصل ۶ (توافقی)، در صورت نیاز فصل ۷ (الزامی) \\
\altrow مأموریت & ایجاد \lr{UNMOIT}، تعیین اختیارات، مدت، گزارش‌دهی \\
اصول & تأکید بر مالکیت ملی، تمامیت ارضی، عدم مداخله \\
\altrow تحریم‌ها & تعلیق یا لغو تحریم‌های قبلی مشروط به پیشرفت \\
تعهدات کشورها & همکاری، عدم مداخله یکجانبه، کمک مالی \\
\altrow مکانیزم بازنگری & بررسی هر ۶ ماه، امکان تمدید یا تعدیل \\
\bottomrule
\end{tabularx}
\end{table}

\subsubsection{توافق‌نامه وضعیت مأموریت (\lr{SOMA})}

\bilingual{توافق‌نامه وضعیت مأموریت}{Status of Mission Agreement — SOMA} میان \lr{UN} و دولت انتقالی ایران، چارچوب حقوقی حضور بین‌المللی را تعیین می‌کند.

\begin{table}[htbp]
\centering
\caption{عناصر کلیدی \lr{SOMA}}
\label{tab:soma}
\begin{tabularx}{\textwidth}{>{\raggedleft\arraybackslash}p{3.5cm}
                             >{\raggedleft\arraybackslash}X}
\toprule
\headerrow موضوع & محتوا \\
\midrule
مصونیت & مصونیت کارکردی (نه مطلق) برای کارکنان، لغو برای جرایم سنگین \\
\altrow تردد & آزادی حرکت در سراسر کشور بدون مجوز قبلی \\
ارتباطات & آزادی ارتباطات رمزنگاری‌شده \\
\altrow مالیات و گمرک & معافیت برای تجهیزات و واردات مأموریت \\
تأسیسات & دسترسی به ساختمان‌ها، امنیت تأسیسات \\
\altrow حل اختلاف & کمیته مشترک + داوری بین‌المللی \\
\bottomrule
\end{tabularx}
\end{table}

\subsection{چارچوب داخلی موقت}
\label{subsec:domestic-legal}

در دوره گذار، قبل از تصویب قانون اساسی جدید، قوانین موقت ضروری‌اند:

\begin{table}[htbp]
\centering
\caption{قوانین موقت ضروری}
\label{tab:interim-laws}
\begin{tabularx}{\textwidth}{>{\raggedleft\arraybackslash}p{3.5cm}
                             >{\raggedleft\arraybackslash}X
                             >{\raggedleft\arraybackslash}p{2.5cm}}
\toprule
\headerrow قانون & محتوای کلیدی & مرجع تصویب \\
\midrule
قانون انتخابات موقت & نظام انتخاباتی، حق رأی، ثبت‌نام، شکایات & شورای انتقالی \\
\altrow قانون احزاب موقت & آزادی تشکیل، ثبت، تأمین مالی & شورای انتقالی \\
قانون رسانه موقت & آزادی مطبوعات، صدور مجوز، ضدانحصار & شورای انتقالی \\
\altrow قانون تجمعات & آزادی تجمع، محدودیت‌های امنیتی معقول & شورای انتقالی \\
منشور حقوق بنیادین & حقوق اساسی غیرقابل تعلیق & اعلامیه \lr{SRSG} \\
\altrow قانون عدالت انتقالی & کمیسیون حقیقت، دادگاه ویژه، جبران & شورای انتقالی \\
\bottomrule
\end{tabularx}
\end{table}

\begin{casestudy}{عراق: خلأ قانونی}
یکی از اشتباهات بزرگ \lr{CPA} در عراق، انحلال قوانین موجود بدون جایگزین بود. این خلأ به هرج‌ومرج، غارت، و تضعیف حاکمیت قانون انجامید. در ایران باید از «قانون‌زدایی» افراطی اجتناب شود. قوانین موجود که با حقوق بشر سازگارند، موقتاً حفظ شوند.
\end{casestudy}

\subsection{الحاق به معاهدات بین‌المللی}
\label{subsec:treaties}

\begin{recommendation}
الحاق فوری (ماه‌های اول):
\begin{itemize}[nosep]
    \item پروتکل اختیاری میثاق حقوق مدنی-سیاسی
    \item کنوانسیون علیه شکنجه (پروتکل اختیاری)
    \item اساسنامه رم دادگاه کیفری بین‌المللی
    \item کنوانسیون رفع تبعیض علیه زنان (\lr{CEDAW}) بدون شرط
\end{itemize}

الحاق میان‌مدت (سال اول):
\begin{itemize}[nosep]
    \item کنوانسیون ضد فساد سازمان ملل
    \item کنوانسیون‌های \lr{ILO} (کار اجباری، آزادی انجمن)
    \item کنوانسیون حقوق کودک (پروتکل‌های اختیاری)
\end{itemize}
\end{recommendation}

\sectiondivider

% ═══════════════════════════════════════════════════════════════════════════════
\section{ماتریس یکپارچه نیازمندی‌ها}
\label{sec:requirements-matrix}
% ═══════════════════════════════════════════════════════════════════════════════

جدول زیر نیازمندی‌های کلیدی را با فاز، اولویت، و مسئول خلاصه می‌کند:

\begin{landscape}
\begin{table}[htbp]
\centering
\bigtablefontsize
\caption{ماتریس یکپارچه نیازمندی‌های کلیدی}
\label{tab:requirements-matrix}
\begin{tabularx}{\linewidth}{>{\raggedleft\arraybackslash}p{1cm}
                             >{\raggedleft\arraybackslash}p{3cm}
                             >{\centering\arraybackslash}p{1.5cm}
                             >{\centering\arraybackslash}p{1.5cm}
                             >{\raggedleft\arraybackslash}X
                             >{\raggedleft\arraybackslash}p{2.5cm}
                             >{\raggedleft\arraybackslash}p{2.5cm}}
\toprule
\headerrow \# & نیازمندی & حوزه & فاز & شرح مختصر & مسئول اصلی & پیش‌نیاز \\
\midrule
۱ & \lr{SRSG} و دفتر مرکزی & نهادی & ۱ & انتخاب و استقرار نماینده ویژه & دبیرکل \lr{UN} & قطعنامه \lr{SC} \\
\altrow ۲ & قطعنامه شورای امنیت & حقوقی & ۰ & چارچوب حقوقی مأموریت & \lr{P5} & اجماع \\
۳ & \lr{SOMA} & حقوقی & ۱ & توافق با دولت انتقالی & \lr{UN} + دولت & دولت انتقالی \\
\altrow ۴ & کمیسیون انتخابات & نهادی & ۱ & تأسیس و تجهیز & دولت + \lr{UNDP} & قانون موقت \\
۵ & سامانه ثبت رأی‌دهندگان & فنی & ۱ & پایگاه داده ۶۰M+ & کمیسیون + \lr{IFES} & زیرساخت \lr{IT} \\
\altrow ۶ & ۶-۱۲K کارمند بین‌المللی & انسانی & ۱-۲ & استخدام و استقرار & \lr{UN HR} & بودجه \\
۷ & ۲۰-۵۰K کارمند ایرانی & انسانی & ۱-۲ & استخدام و آموزش & \lr{UNMOIT} + محلی & غربالگری \\
\altrow ۸ & کمیسیون حقیقت & نهادی & ۲ & تأسیس و شروع کار & مجلس موقت & قانون عدالت انتقالی \\
۹ & دادگاه ویژه & نهادی & ۲ & تأسیس و شروع محاکمات & دولت + \lr{UN} & تحقیقات اولیه \\
\altrow ۱۰ & صندوق امانی & نهادی & ۱ & مدیریت کمک‌ها و دارایی‌ها & \lr{UN} + دولت & توافق حامیان \\
۱۱ & زیرساخت ارتباطات & فنی & ۱ & اینترنت، تلفن، رادیو & وزارت \lr{ICT} + \lr{UN} & رفع فیلترینگ \\
\altrow ۱۲ & امنیت سایبری & فنی & ۱ & تیم و پروتکل‌ها & \lr{UNMOIT IT} & متخصصان \\
۱۳ & قوانین موقت & حقوقی & ۱ & انتخابات، احزاب، رسانه & شورای انتقالی & مشروعیت \\
\altrow ۱۴ & ۳۱ دفتر استانی & نهادی & ۱-۲ & حضور سراسری & \lr{UNMOIT} & کارمند + تأسیسات \\
\bottomrule
\end{tabularx}
\end{table}
\end{landscape}

\sectiondivider

% ═══════════════════════════════════════════════════════════════════════════════
% جمع‌بندی فصل
% ═══════════════════════════════════════════════════════════════════════════════

\begin{chaptersummary}
یافته‌های کلیدی این فصل:

\begin{enumerate}
    \item \textbf{نیروی انسانی عظیم اما تدریجی}: اوج نیاز در فاز ۲ (۲۷,۰۰۰-۶۴,۰۰۰ نفر) است، اما استقرار باید تدریجی و با آموزش کافی باشد. اتکای بیش‌ازحد به نیروی بین‌المللی نه ممکن است و نه مطلوب.
    
    \item \textbf{ظرفیت ایرانی ستون فقرات است}: ۸۰-۹۰٪ نیروی انسانی باید ایرانی باشد. دیاسپورا منبع ارزشمند اما باید با دقت مدیریت شود (سقف ۲۰-۳۰٪).
    
    \item \textbf{هشت نهاد کلیدی باید ایجاد شوند}: دفتر \lr{SRSG}، کمیسیون انتخابات، کمیسیون حقیقت، دادگاه ویژه، صندوق امانی، کمیسیون رسانه، کمیسیون ضدفساد، و نهاد حقوق اقوام.
    
    \item \textbf{زیرساخت فنی حیاتی‌تر از آنچه به نظر می‌رسد}: رفع فیلترینگ اینترنت در روز اول، سامانه ثبت رأی‌دهندگان، و امنیت سایبری از اولویت‌های بالا هستند.
    
    \item \textbf{انتخاب فناوری انتخابات}: رأی کاغذی با شمارش علنی در مرحله اول، امن‌ترین و قابل‌اعتمادترین گزینه است.
    
    \item \textbf{چارچوب حقوقی سه‌لایه}: قطعنامه شورای امنیت (بین‌المللی) + \lr{SOMA} (دوجانبه) + قوانین موقت (داخلی) باید همزمان آماده شوند.
    
    \item \textbf{از خلأ قانونی اجتناب شود}: قوانین موجود که با حقوق بشر سازگارند، تا تصویب قانون جدید حفظ شوند (درس عراق).
    
    \item \textbf{الحاق به معاهدات بین‌المللی}: اساسنامه رم، \lr{CEDAW}، پروتکل شکنجه و سایر معاهدات حقوق بشری باید در اولویت باشند.
\end{enumerate}

\vspace{0.5cm}
\textit{در فصل بعد (\ref{ch:timeline})، زمان‌بندی تفصیلی، ساختار تیم‌ها و مکانیزم هماهنگی بررسی خواهد شد.}
\end{chaptersummary}

\chapterend