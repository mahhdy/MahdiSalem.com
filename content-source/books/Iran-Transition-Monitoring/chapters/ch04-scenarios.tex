% ╔══════════════════════════════════════════════════════════════════╗
% ║  فصل ۴: سناریوهای گذار و مدل‌های نظارتی متناظر               ║
% ║  شش سناریو + ماتریس تطبیق + درخت تصمیم                       ║
% ╚══════════════════════════════════════════════════════════════════╝

% ---- صفحه‌ی آغازین فصل ----
\chapteropening{۴}
    {سناریوهای گذار و مدل‌های نظارتی متناظر}
    {MainOrange}
    {آینده قابل پیش‌بینی نیست، 
    اما می‌توان برای آن آماده شد. 
    بهترین استراتژی آن است که برای 
    چندین آینده‌ی ممکن هم‌زمان برنامه داشته باشیم.}
    {پیتر شوارتز، بنیان‌گذار سناریوپردازی مدرن}

\chapter{سناریوهای گذار و مدل‌های نظارتی متناظر}
\label{ch:scenarios}
\minitoc

% ---- خلاصه‌ی اجرایی ----
\begin{executivesummary}
هر سناریوی گذار، مدل نظارتی متفاوتی 
می‌طلبد. این فصل \emphorange{شش سناریوی 
محتمل} برای تغییر سیاسی در ایران را تحلیل 
می‌کند و برای هر یک مدل نظارتی مناسب، 
زمان واکنش لازم، ریسک‌های اصلی و 
نمونه‌های تاریخی مشابه را مشخص می‌سازد. 
هدف آن است که بازیگران داخلی و بین‌المللی 
\emphorange{برای هر آینده‌ی ممکن آماده باشند} 
و در لحظه‌ی بحران مجبور به تصمیم‌گیری 
ارتجالی نشوند. سناریوی مطلوب 
(گذار مذاکره‌ای) و سناریوی محتمل‌تر 
(فروپاشی یا انقلاب مردمی) تمایز داده شده 
و مدل ترکیبی-تطبیقی فصل ۳ برای هر 
سناریو تنظیم شده است.
\end{executivesummary}

% ============================================================
\section{چرا سناریوسازی؟ روش‌شناسی و 
محدودیت‌ها}
\label{sec:why-scenarios}
% ============================================================

\begin{definitionbox}{سناریوسازی (\lr{Scenario Planning})}
روشی ساختاریافته برای تصور چندین آینده‌ی 
ممکن و طراحی استراتژی‌های انطباقی برای 
هر یک. سناریوها پیش‌بینی نیستند — ابزار 
آمادگی هستند.
\end{definitionbox}

سناریوسازی برای موضوع این کتاب به سه 
دلیل ضروری است:

\begin{enumerate}[
    label=\textcolor{MainOrange}{\bfseries\arabic*.},
    itemsep=6pt
]
    \item \textbf{عدم قطعیت بالا:} 
    هیچ‌کس نمی‌داند تغییر در ایران دقیقاً 
    کِی و چگونه رخ خواهد داد. ممکن است 
    فردا باشد یا یک دهه‌ی دیگر.
    
    \item \textbf{تفاوت بنیادین مدل‌ها:} 
    همان‌طور که فصل ۳ نشان داد 
    (\seeChapter{ch:approaches})، 
    هر نوع گذار مدل نظارتی متفاوتی 
    می‌طلبد. آمادگی برای فقط یک سناریو 
    خطرناک است.
    
    \item \textbf{زمان واکنش محدود:} 
    در لحظه‌ی فروپاشی یا انقلاب، 
    فرصت طراحی از صفر وجود ندارد. 
    باید برنامه‌های آماده‌باش 
    (\lr{Contingency Plans}) 
    از پیش تدوین شده باشند.
\end{enumerate}

\begin{warningbox}
\textbf{محدودیت‌های سناریوسازی:}
\begin{itemize}[itemsep=2pt]
    \item سناریوها \emphred{پیش‌بینی نیستند} — 
    واقعیت احتمالاً ترکیبی از چند سناریو 
    خواهد بود
    \item ممکن است سناریویی رخ دهد که 
    اصلاً تصور نکرده‌ایم 
    (\lr{Black Swan})
    \item سناریوها ایستا نیستند — ممکن 
    است از یک سناریو به سناریوی دیگر 
    جابه‌جایی رخ دهد
    \item احتمالات ذکرشده تخمینی و ذهنی 
    هستند، نه محاسبات ریاضی
\end{itemize}
\end{warningbox}

\subsection{متغیرهای کلیدی سناریوسازی}

شش سناریوی این فصل بر مبنای تعامل 
چهار متغیر کلیدی ساخته شده‌اند:

\begin{figure}[htbp]
    \centering
    \begin{tikzpicture}[
        var/.style={
            draw=MainOrange, fill=OrangeBG,
            rounded corners=3pt,
            minimum height=1.3cm, minimum width=3.5cm,
            align=center, font=\small\bfseries
        },
        center/.style={
            draw=MainPurple, fill=PurpleBG,
            circle, minimum size=2cm,
            align=center, font=\footnotesize\bfseries
        },
        conn/.style={thick, MainOrange!60}
    ]
    
    \node[center] (c) {نوع\\گذار};
    
    \node[var] (v1) at (90:3.5cm) 
        {سرعت تغییر\\[2pt]
        \footnotesize تدریجی ← → ناگهانی};
    \node[var] (v2) at (0:4cm) 
        {نقش عامل خارجی\\[2pt]
        \footnotesize حداقلی ← → حداکثری};
    \node[var] (v3) at (270:3.5cm) 
        {سطح خشونت\\[2pt]
        \footnotesize مسالمت‌آمیز ← → خشن};
    \node[var] (v4) at (180:4cm) 
        {نقش نظام قدیم\\[2pt]
        \footnotesize حذف ← → مشارکت};
    
    \draw[conn] (c) -- (v1);
    \draw[conn] (c) -- (v2);
    \draw[conn] (c) -- (v3);
    \draw[conn] (c) -- (v4);
    
    \end{tikzpicture}
    \caption{چهار متغیر کلیدی تعیین‌کننده‌ی 
    نوع سناریوی گذار}
    \label{fig:scenario-variables}
\end{figure}

\sectiondivider

% ============================================================
\section{سناریوی A: فروپاشی ناگهانی نظام}
\label{sec:scenario-a}
% ============================================================

\begin{scenariobox}{فروپاشی ناگهانی 
(\lr{Sudden Collapse})}

\begin{tabularx}{\textwidth}{L{3cm} X}
    \textbf{احتمال:} & متوسط (۲۰-۳۰٪) \\
    \textbf{سرعت:} & بسیار بالا (روزها تا هفته‌ها) \\
    \textbf{خشونت:} & متوسط تا بالا \\
    \textbf{نقش نظام قدیم:} & حذف/فرار \\
    \textbf{عامل خارجی:} & واکنشی (نه آغازگر) \\
    \textbf{محرک احتمالی:} & بحران جانشینی رهبری، 
    شورش نظامی، فروپاشی اقتصادی ناگهانی، 
    اعتصاب عمومی فلج‌کننده \\
    \textbf{نمونه‌ی مشابه:} & فروپاشی شوروی ۱۹۹۱، 
    سقوط بن‌علی تونس ۲۰۱۱، 
    سقوط چائوشسکو رومانی ۱۹۸۹ \\
    \textbf{زمان واکنش نظارتی:} & ۴۸-۷۲ ساعت
\end{tabularx}

\end{scenariobox}

\subsection{ویژگی‌ها و پویایی}

در این سناریو، نظام بدون مذاکره و بدون 
برنامه‌ی انتقالی فرو می‌پاشد. ممکن است 
رهبر فوت کند و جانشینی بحرانی شود، 
یا بخشی از نیروهای مسلح از اطاعت سر 
باز زنند، یا شورشی شهری کنترل‌ناپذیر 
شود. ویژگی‌های اصلی:

\begin{itemize}[itemsep=4pt]
    \item \textbf{خلأ قدرت فوری:} 
    هیچ نهاد مشروعی برای مدیریت 
    انتقال وجود ندارد
    \item \textbf{ریسک بالای خشونت:} 
    نیروهای امنیتی ممکن است تجزیه 
    شوند — بخشی مقاومت و بخشی فرار
    \item \textbf{رقابت بر سر قدرت:} 
    گروه‌های مختلف (سپاه، اپوزیسیون، 
    گروه‌های قومی) هم‌زمان ادعای 
    قدرت می‌کنند
    \item \textbf{بحران انسانی:} 
    مهاجرت، کمبود مواد غذایی و دارو، 
    قطع خدمات
    \item \textbf{ریسک دخالت خارجی:} 
    همسایگان و قدرت‌های بزرگ وسوسه‌ی 
    مداخله پیدا می‌کنند
\end{itemize}

\subsection{مدل نظارتی متناسب}

\begin{table}[htbp]
    \centering
    \caption{مدل نظارتی برای سناریوی 
    فروپاشی ناگهانی}
    \label{tab:scenario-a-model}
    \tablefontsize
    \begin{tabularx}{\textwidth}{L{3cm} X}
        \toprule
        \headerrow
        \textbf{عنصر} & \textbf{توضیح} \\
        \midrule
        
        مدل غالب &
        مدل ۴ (اجرایی) تقویت‌شده + 
        عناصر امنیتی مدل ۵ در 
        هفته‌های نخست \\
        \altrow
        
        اولویت فوری &
        جلوگیری از خلأ امنیتی، 
        حفاظت از زیرساخت‌ها 
        (نفت، هسته‌ای، آب)، 
        مدیریت بحران انسانی \\
        
        نیروی انسانی فوری &
        تیم پیشرو (\lr{Advance Team}) 
        ۵۰-۱۰۰ نفر در ۴۸ ساعت، 
        سپس ۵,۰۰۰+ در ۲ هفته \\
        \altrow
        
        مکانیزم تصمیم‌گیری &
        جلسه‌ی اضطراری شورای امنیت 
        → قطعنامه‌ی فوری → 
        انتصاب \lr{SRSG} \\
        
        چالش اصلی &
        سرعت: آیا جامعه‌ی بین‌المللی 
        می‌تواند به اندازه‌ی کافی سریع 
        واکنش نشان دهد؟ \\
        \altrow
        
        پیش‌نیاز حیاتی &
        \emphred{برنامه‌ی آماده‌باش از الان 
        باید تدوین شده باشد} \\
        
        \bottomrule
    \end{tabularx}
\end{table}

\begin{lessonlearned}
\textbf{از تجربه‌ی رومانی (۱۹۸۹):}
سقوط چائوشسکو ظرف ۱۰ روز رخ داد. 
جامعه‌ی بین‌المللی هیچ برنامه‌ای نداشت. 
نتیجه: عناصر نظام قدیم (به رهبری ایلیسکو) 
قدرت را تصاحب کردند و «انقلاب دزدیده شد». 
دموکراسی واقعی سال‌ها به تأخیر افتاد.

\vspace{4pt}
\emphblue{درس: سرعت واکنش حیاتی است. 
بدون برنامه‌ی آماده‌باش از پیش، فروپاشی 
ناگهانی به مصادره‌ی گذار توسط فرصت‌طلبان 
منجر می‌شود.}
\end{lessonlearned}

\begin{keypoint}
\textbf{برنامه‌ی آماده‌باش (\lr{Contingency Plan}) 
باید شامل موارد زیر باشد:}
\begin{enumerate}[itemsep=2pt, font=\small]
    \item فهرست تماس اضطراری 
    (چه کسی با چه کسی تماس می‌گیرد)
    \item متن پیش‌نویس قطعنامه‌ی 
    شورای امنیت
    \item فهرست ناظران آماده‌باش 
    (\lr{Stand-by Roster})
    \item پروتکل امنیتی برای 
    حفاظت از تأسیسات هسته‌ای
    \item کانال‌های ارتباطی با 
    نیروهای داخلی قابل اعتماد
    \item برنامه‌ی رسانه‌ای بحران 
    (مقابله با اطلاعات نادرست)
\end{enumerate}
\end{keypoint}

\sectiondivider

% ============================================================
\section{سناریوی B: گذار مذاکره‌ای}
\label{sec:scenario-b}
% ============================================================

\begin{scenariobox}{گذار مذاکره‌ای 
(\lr{Negotiated Transition})}

\begin{tabularx}{\textwidth}{L{3cm} X}
    \textbf{احتمال:} & پایین تا متوسط (۱۰-۲۰٪) \\
    \textbf{سرعت:} & تدریجی (ماه‌ها تا سال‌ها) \\
    \textbf{خشونت:} & حداقلی \\
    \textbf{نقش نظام قدیم:} & مشارکت فعال \\
    \textbf{عامل خارجی:} & تسهیل‌گر/میانجی \\
    \textbf{محرک احتمالی:} & شکاف عمیق درون نظام 
    بین تندروها و نرم‌روها، فشار اقتصادی 
    غیرقابل تحمل، تهدید خارجی مشترک \\
    \textbf{نمونه‌ی مشابه:} & آفریقای جنوبی ۱۹۹۰-۱۹۹۴، 
    لهستان ۱۹۸۹ (میز گرد)، 
    اسپانیا ۱۹۷۵-۱۹۸۲ \\
    \textbf{زمان واکنش نظارتی:} & هفته‌ها تا ماه‌ها 
    (فرصت برنامه‌ریزی وجود دارد)
\end{tabularx}

\end{scenariobox}

\subsection{ویژگی‌ها و پویایی}

\begin{pullquote}
گذار مذاکره‌ای \emphorange{مطلوب‌ترین} سناریو 
است — کمترین هزینه‌ی انسانی، بالاترین شانس 
موفقیت بلندمدت، بیشترین فرصت برای طراحی 
نظارت. اما \emphorange{محتمل‌ترین} نیست.
\end{pullquote}

در این سناریو، بخشی از نخبگان نظام 
(\lr{softliners}) به این نتیجه می‌رسند 
که ادامه‌ی وضع موجود ناممکن یا 
بسیار پرهزینه است و حاضر می‌شوند 
با اپوزیسیون مذاکره کنند. 
شرایط لازم:

\begin{enumerate}[
    label=\textcolor{MainOrange}{\bfseries\alph*)},
    itemsep=4pt
]
    \item \textbf{شکاف درون نظام:} 
    تندروها (\lr{hardliners}) ضعیف 
    شده باشند یا درون خود تقسیم شوند — 
    مثلاً بحران جانشینی رهبری
    
    \item \textbf{اپوزیسیون سازمان‌یافته:} 
    طرف مقابل مذاکره باید وجود داشته 
    باشد — نه پراکنده و بی‌سازمان
    
    \item \textbf{اعتماد حداقلی:} 
    طرفین باید حداقلی از اعتماد 
    (یا تضمین خارجی جایگزین اعتماد) 
    داشته باشند
    
    \item \textbf{فشار کافی:} 
    فشار مردمی + اقتصادی + بین‌المللی 
    به حدی باشد که مذاکره را 
    جذاب‌تر از سرکوب کند
    
    \item \textbf{میانجی قابل اعتماد:} 
    شخص یا نهادی که هر دو طرف 
    بپذیرند (سازمان ملل؟ یک کشور 
    بی‌طرف؟ یک شخصیت مورد احترام؟)
\end{enumerate}

\subsection{مدل نظارتی متناسب}

\begin{table}[htbp]
    \centering
    \caption{مدل نظارتی برای سناریوی 
    گذار مذاکره‌ای}
    \label{tab:scenario-b-model}
    \tablefontsize
    \begin{tabularx}{\textwidth}{L{3cm} X}
        \toprule
        \headerrow
        \textbf{عنصر} & \textbf{توضیح} \\
        \midrule
        
        مدل غالب &
        مدل ۶ (ترکیبی-تطبیقی) استاندارد — 
        بهترین تناسب \\
        \altrow
        
        نقش بین‌المللی &
        تسهیل‌گری مذاکرات + تضمین 
        اجرای توافقات + نظارت ساختاری \\
        
        اولویت &
        طراحی قواعد بازی، تأمین امنیت 
        مذاکره‌کنندگان، نظارت بر 
        رعایت تعهدات طرفین \\
        \altrow
        
        زمان‌بندی &
        مرحله‌ی مذاکره: ۳-۱۲ ماه، 
        سپس ورود به فاز ۱ مدل ۶ \\
        
        نقش ویژه &
        \lr{SRSG} یا میانجی ویژه 
        به‌عنوان تسهیل‌گر مذاکرات \\
        \altrow
        
        مزیت اصلی &
        \cellgreen{فرصت برنامه‌ریزی: 
        می‌توان نظارت را از قبل 
        طراحی و مستقر کرد} \\
        
        \bottomrule
    \end{tabularx}
\end{table}

\begin{casestudy}{میز گرد لهستان (۱۹۸۹)}
در فوریه-آوریل ۱۹۸۹، دولت کمونیستی 
لهستان و اتحادیه‌ی همبستگی 
(\lr{Solidarity}) به رهبری 
\person{لخ والنسا}{Lech Walesa} 
بر سر میز مذاکره نشستند. نتیجه: 
انتخابات نیمه‌آزاد ژوئن ۱۹۸۹ 
که همبستگی ۹۹ از ۱۰۰ کرسی 
سنا را بُرد.

\vspace{4pt}
\textbf{عوامل موفقیت:}
\begin{enumerate}[itemsep=2pt, font=\small]
    \item فشار اقتصادی غیرقابل تحمل 
    بر رژیم
    \item حمایت پاپ ژان پل دوم 
    (مشروعیت مذهبی-ملی)
    \item اپوزیسیون سازمان‌یافته و 
    دارای رهبر شناخته‌شده
    \item تغییر موضع شوروی 
    (گورباچف دخالت نکرد)
    \item تضمین‌های امنیتی برای 
    نخبگان رژیم قدیم
\end{enumerate}

\vspace{4pt}
\textbf{درس برای ایران:}
\emphblue{گذار مذاکره‌ای در لهستان ممکن شد 
زیرا هم نظام ضعیف شده بود و هم اپوزیسیون 
قوی و متحد بود. در ایران، شرط دوم هنوز 
محقق نشده — اتحاد و سازمان‌دهی اپوزیسیون 
پیش‌شرط حیاتی این سناریو است.}
\end{casestudy}

\sectiondivider

% ============================================================
\section{سناریوی C: انقلاب مردمی}
\label{sec:scenario-c}
% ============================================================

\begin{scenariobox}{انقلاب مردمی 
(\lr{Popular Revolution})}

\begin{tabularx}{\textwidth}{L{3cm} X}
    \textbf{احتمال:} & متوسط (۲۰-۳۰٪) \\
    \textbf{سرعت:} & متوسط (هفته‌ها تا ماه‌ها) \\
    \textbf{خشونت:} & متوسط (بستگی به واکنش رژیم) \\
    \textbf{نقش نظام قدیم:} & سرنگونی/حذف \\
    \textbf{عامل خارجی:} & حمایتی (نه آغازگر) \\
    \textbf{محرک احتمالی:} & اعتراضات گسترده 
    (مشابه ۱۴۰۱ اما بزرگ‌تر و پایدارتر)، 
    شکاف در نیروهای امنیتی \\
    \textbf{نمونه‌ی مشابه:} & تونس ۲۰۱۱ (مثبت)، 
    مصر ۲۰۱۱ (ابتدا مثبت سپس شکست)، 
    لیبی ۲۰۱۱ (منفی) \\
    \textbf{زمان واکنش نظارتی:} & ۱-۴ هفته
\end{tabularx}

\end{scenariobox}

\subsection{ویژگی‌ها و پویایی}

این سناریو نزدیک‌ترین به تجربه‌ی اخیر 
ایران است — خیزش «زن، زندگی، آزادی» 
(۱۴۰۱) الگوی اولیه‌ی آن را نشان داد. 
تفاوت: این بار اعتراضات باید به حدی 
گسترده و پایدار باشند که نظام توان 
سرکوب نداشته باشد.

\begin{table}[htbp]
    \centering
    \caption{شرایط موفقیت و شکست 
    انقلاب مردمی}
    \label{tab:revolution-conditions}
    \begin{tabularx}{\textwidth}{
        C{0.5cm} X X
    }
        \toprule
        \headerrow
        & \textbf{\textcolor{MainGreen}{شرایط موفقیت}} & 
        \textbf{\textcolor{MainRed}{عوامل شکست}} \\
        \midrule
        
        ۱ &
        \cellgreen{مشارکت فراطبقاتی و 
        فراقومی — نه فقط طبقه متوسط} &
        \cellred{محدود ماندن به یک شهر 
        یا یک طبقه} \\
        \altrow
        
        ۲ &
        \cellgreen{حفظ خصلت مسالمت‌آمیز 
        (قدرت اخلاقی)} &
        \cellred{خشونت‌ورزی معترضان → 
        مشروعیت‌بخشی به سرکوب} \\
        
        ۳ &
        \cellgreen{شکاف در نیروهای امنیتی 
        — بخشی از سپاه/ارتش به مردم 
        بپیوندد} &
        \cellred{وحدت نیروهای سرکوبگر} \\
        \altrow
        
        ۴ &
        \cellgreen{رهبری شبکه‌ای 
        (غیرمتمرکز اما هماهنگ)} &
        \cellred{فقدان رهبری یا 
        رهبری تک‌صدایی} \\
        
        ۵ &
        \cellgreen{حمایت بین‌المللی سریع 
        و قاطع} &
        \cellred{سکوت یا تردید 
        جامعه بین‌المللی} \\
        \altrow
        
        ۶ &
        \cellgreen{اعتصاب عمومی 
        (فلج اقتصادی نظام)} &
        \cellred{ادامه‌ی چرخه‌ی اقتصادی 
        → تاب‌آوری نظام} \\
        
        \bottomrule
    \end{tabularx}
\end{table}

\subsection{مدل نظارتی متناسب}

در این سناریو، نظارت بین‌المللی باید 
در دو مرحله عمل کند:

\begin{enumerate}[
    label=\textcolor{MainOrange}{\bfseries 
    مرحله‌ی \arabic*:},
    itemsep=8pt
]
    \item \textbf{حین انقلاب (هفته‌ها):} 
    نظارت حقوق بشری فعال — مستندسازی 
    سرکوب، فشار دیپلماتیک بر رژیم، 
    حفاظت از شهروندان 
    (گزارشگر ویژه + \lr{OHCHR})
    
    \item \textbf{پس از سرنگونی:} 
    ورود فوری به فاز ۱ مدل ۶ 
    (مشابه سناریوی A اما با 
    مشروعیت مردمی بالاتر)
\end{enumerate}

\begin{lessonlearned}
\textbf{مقایسه‌ی تونس و مصر (۲۰۱۱):}

هر دو انقلاب مردمی بودند. اما نتایج 
بسیار متفاوت:

\begin{tabularx}{\textwidth}{L{3cm} X X}
    & \textbf{تونس} & \textbf{مصر} \\
    \midrule
    ارتش & بی‌طرف ماند & 
    «محافظ» شد سپس کودتا کرد \\
    اپوزیسیون & اجماع حداقلی & 
    تضاد شدید (اخوان vs سکولار) \\
    قانون اساسی & مشارکتی و مترقی & 
    یک‌جانبه و فرقه‌ای \\
    نظارت بین‌المللی & مشورتی فعال & 
    حداقلی و منفعل \\
    نتیجه & دموکراسی (تا ۲۰۲۱) & 
    کودتای ۲۰۱۳ \\
\end{tabularx}

\vspace{4pt}
\emphblue{درس: انقلاب مردمی فقط نیمی از 
کار است. نیمه‌ی دوم — نهادسازی — تعیین‌کننده 
است. و اینجاست که نظارت بین‌المللی 
تفاوت می‌سازد.}
\end{lessonlearned}

\sectiondivider

% ============================================================
\section{سناریوی D: تحول از درون نظام}
\label{sec:scenario-d}
% ============================================================

\begin{scenariobox}{تحول از درون نظام 
(\lr{Intra-Regime Reform})}

\begin{tabularx}{\textwidth}{L{3cm} X}
    \textbf{احتمال:} & پایین (۵-۱۰٪) \\
    \textbf{سرعت:} & بسیار تدریجی (سال‌ها) \\
    \textbf{خشونت:} & حداقلی \\
    \textbf{نقش نظام قدیم:} & هدایت‌کننده \\
    \textbf{عامل خارجی:} & فشار تدریجی \\
    \textbf{محرک احتمالی:} & به قدرت رسیدن 
    یک فرد اصلاح‌طلب واقعی پس از 
    مرگ رهبر، محاسبه‌ی عقلایی 
    بخشی از نظام \\
    \textbf{نمونه‌ی مشابه:} & میانمار ۲۰۱۰-۲۰۱۵ 
    (شکست‌خورده)، اتحاد شوروی 
    (گلاسنوست/پرسترویکا — ناخواسته 
    به فروپاشی انجامید) \\
    \textbf{زمان واکنش نظارتی:} & ماه‌ها تا سال‌ها
\end{tabularx}

\end{scenariobox}

\subsection{چرا این سناریو بعید است}

\begin{warningbox}
در ایران، بر خلاف میانمار یا شوروی، 
ساختار نظام به‌گونه‌ای طراحی شده که 
اصلاحات بنیادین را \emphred{ساختاراً مسدود} 
می‌کند:
\begin{itemize}[itemsep=2pt]
    \item شورای نگهبان وتوی مطلق 
    بر هرگونه تغییر قانونی دارد
    \item رهبر فراتر از قانون اساسی 
    قرار دارد
    \item سپاه منافع اقتصادی عظیمی 
    در حفظ وضع موجود دارد
    \item تجربه‌ی شکست اصلاحات 
    خاتمی (۱۳۷۶-۱۳۸۴) نشان داد 
    که «اصلاح از درون» در این 
    ساختار ناممکن است
\end{itemize}
\end{warningbox}

\subsection{اگر رخ دهد: مدل نظارتی}

\begin{table}[htbp]
    \centering
    \caption{مدل نظارتی برای سناریوی 
    تحول درونی}
    \label{tab:scenario-d-model}
    \begin{tabularx}{\textwidth}{L{3cm} X}
        \toprule
        \headerrow
        \textbf{عنصر} & \textbf{توضیح} \\
        \midrule
        مدل غالب & مدل ۲ (مشورتی) با 
        فشار برای حرکت به مدل ۳ \\
        \altrow
        نقش بین‌المللی & فشار دیپلماتیک + 
        مشاوره فنی + مشوق اقتصادی \\
        خطر اصلی & 
        \cellred{گذار نمایشی: اصلاحات 
        سطحی بدون تغییر واقعی قدرت} \\
        \altrow
        معیار سنجش & آیا اصلاحات واقعی‌اند؟ 
        آزادی زندانیان، رسانه آزاد، 
        انتخابات بدون فیلتر \\
        نمونه هشداردهنده & 
        \cellred{میانمار: ارتش «دموکراسی» 
        داد و ۵ سال بعد پس گرفت (کودتای ۲۰۲۱)} \\
        \bottomrule
    \end{tabularx}
\end{table}

\sectiondivider

% ============================================================
\section{سناریوی E: دخالت نظامی خارجی}
\label{sec:scenario-e}
% ============================================================

\begin{scenariobox}{دخالت نظامی خارجی 
(\lr{Foreign Military Intervention})}

\begin{tabularx}{\textwidth}{L{3cm} X}
    \textbf{احتمال:} & بسیار پایین (< ۵٪) \\
    \textbf{سرعت:} & متغیر \\
    \textbf{خشونت:} & بسیار بالا \\
    \textbf{نقش نظام قدیم:} & مقاومت مسلحانه \\
    \textbf{عامل خارجی:} & آغازگر و مجری \\
    \textbf{نمونه‌ی مشابه:} & عراق ۲۰۰۳، 
    افغانستان ۲۰۰۱، لیبی ۲۰۱۱ — 
    \emphred{همه شکست‌خورده} \\
    \textbf{مدل نظارتی:} & مدل ۵ اجباری
\end{tabularx}

\end{scenariobox}

\begin{warningbox}
\textbf{موضع قاطع این کتاب:}

\emphred{دخالت نظامی خارجی در ایران باید 
به هر قیمت اجتناب شود.} دلایل:

\begin{enumerate}[itemsep=3pt, font=\small]
    \item \textbf{ابعاد نظامی:} ایران ۱.۶ 
    میلیون کیلومتر مربع وسعت و ۸۵ 
    میلیون جمعیت دارد — مقایسه شود 
    با عراق (۲۵M) و افغانستان (۳۰M) 
    که هر دو شکست خوردند
    
    \item \textbf{مقاومت ملی:} حتی مخالفان 
    رژیم در برابر مداخله‌ی نظامی 
    خارجی مقاومت خواهند کرد — 
    ناسیونالیسم ایرانی قوی‌ترین 
    نیروی متحدکننده است
    
    \item \textbf{فاجعه‌ی انسانی:} جنگ در 
    ایران می‌تواند میلیون‌ها آواره و 
    صدها هزار کشته ایجاد کند
    
    \item \textbf{بی‌ثباتی جهانی:} 
    قیمت نفت چندبرابر، بحران 
    مهاجرت، گسترش جنگ به منطقه
    
    \item \textbf{تأسیسات هسته‌ای:} 
    خطر آلودگی رادیواکتیو
    
    \item \textbf{مشروعیت صفر:} 
    هر نظامی که پس از دخالت نظامی 
    ایجاد شود، مشروعیت مردمی 
    نخواهد داشت
\end{enumerate}
\end{warningbox}

\sectiondivider

% ============================================================
\section{سناریوی F: بحران ممتد و بی‌ثباتی مزمن}
\label{sec:scenario-f}
% ============================================================

\begin{scenariobox}{بحران ممتد 
(\lr{Prolonged Crisis / Chronic Instability})}

\begin{tabularx}{\textwidth}{L{3cm} X}
    \textbf{احتمال:} & متوسط تا بالا (۲۵-۳۵٪) \\
    \textbf{سرعت:} & بسیار کند / رکود \\
    \textbf{خشونت:} & پراکنده و مزمن \\
    \textbf{نقش نظام قدیم:} & تضعیف‌شده اما 
    حذف‌نشده \\
    \textbf{عامل خارجی:} & خسته و منفعل \\
    \textbf{محرک:} & نه نظام توان سرکوب 
    کامل دارد و نه مردم توان سرنگونی — 
    بن‌بست \\
    \textbf{نمونه‌ی مشابه:} & ونزوئلا (۲۰۱۹-حال)، 
    لبنان (۲۰۱۹-حال)، سودان (۲۰۱۹-۲۰۲۳) \\
    \textbf{زمان واکنش نظارتی:} & نامحدود (مزمن)
\end{tabularx}

\end{scenariobox}

\subsection{چرا این سناریو محتمل‌ترین است}

\begin{keypoint}
بسیاری از تحلیلگران معتقدند که 
\emphorange{محتمل‌ترین آینده‌ی کوتاه‌مدت 
ایران نه سقوط ناگهانی و نه اصلاحات، 
بلکه ادامه‌ی وضعیت بحرانی فعلی} است — 
اعتراضات دوره‌ای، سرکوب، فرسایش تدریجی 
نظام و جامعه، مهاجرت نخبگان و 
زوال اقتصادی.
\end{keypoint}

\begin{table}[htbp]
    \centering
    \caption{ویژگی‌ها و پیامدهای بحران ممتد}
    \label{tab:prolonged-crisis}
    \begin{tabularx}{\textwidth}{L{3cm} X}
        \toprule
        \headerrow
        \textbf{بُعد} & \textbf{توضیح} \\
        \midrule
        اقتصادی & فقر فزاینده، تورم مزمن، 
        فرار سرمایه، اقتصاد زیرزمینی \\
        \altrow
        اجتماعی & مهاجرت نخبگان (\lr{brain drain})، 
        افسردگی اجتماعی، افزایش اعتیاد 
        و جرم \\
        سیاسی & بن‌بست: نظام ناتوان از 
        اصلاح و مردم ناتوان از تغییر \\
        \altrow
        حقوق بشری & سرکوب مزمن، 
        اعدام‌های پراکنده، 
        زندانیان سیاسی \\
        منطقه‌ای & ادامه‌ی ماجراجویی 
        منطقه‌ای برای انحراف افکار عمومی \\
        \altrow
        بین‌المللی & خستگی (\lr{fatigue}) — 
        جهان ایران را فراموش می‌کند \\
        \bottomrule
    \end{tabularx}
\end{table}

\subsection{نقش نظارت بین‌المللی در بحران ممتد}

\begin{recommendation}
حتی اگر گذار فوری رخ ندهد، 
جامعه‌ی بین‌المللی وظایفی دارد:
\begin{enumerate}[itemsep=3pt]
    \item \textbf{نظارت حقوق بشری مستمر:} 
    گزارشگر ویژه + مستندسازی 
    (برای عدالت انتقالی آینده)
    \item \textbf{حمایت از جامعه مدنی:} 
    تأمین مالی رسانه‌های مستقل و 
    سازمان‌های حقوق بشری
    \item \textbf{آماده‌سازی:} 
    تدوین برنامه آماده‌باش، 
    آموزش ناظران ایرانی، 
    شبکه‌سازی
    \item \textbf{فشار دیپلماتیک:} 
    حفظ تحریم‌های هدفمند، 
    محاکمه بین‌المللی ناقضان 
    حقوق بشر
    \item \textbf{مقابله با خستگی:} 
    ایران را در دستور کار 
    بین‌المللی نگه داشتن
\end{enumerate}
\end{recommendation}

\sectiondivider

% ============================================================
\section{ماتریس جامع سناریو-مدل}
\label{sec:scenario-matrix}
% ============================================================

\begin{landscape}
\begin{table}[htbp]
    \centering
    \caption{ماتریس جامع سناریوها و 
    مشخصات نظارتی متناظر}
    \label{tab:scenario-matrix}
    \bigtablefontsize
    \setlength{\tabcolsep}{3pt}
    \begin{tabularx}{\linewidth}{
        L{2.2cm}
        C{2.2cm} C{2.2cm} C{2.2cm} 
        C{2.2cm} C{2.2cm} C{2.2cm}
    }
        \toprule
        \headerrow
        \textbf{معیار} & 
        \rotsmall{\textbf{A: فروپاشی}} &
        \rotsmall{\textbf{B: مذاکره}} &
        \rotsmall{\textbf{C: انقلاب}} &
        \rotsmall{\textbf{D: اصلاح درونی}} &
        \rotsmall{\textbf{E: مداخله خارجی}} &
        \rotsmall{\textbf{F: بحران ممتد}} \\
        \midrule
        
        احتمال & ۲۰-۳۰٪ & ۱۰-۲۰٪ & 
        ۲۰-۳۰٪ & ۵-۱۰٪ & <۵٪ & ۲۵-۳۵٪ \\
        \altrow
        
        مطلوبیت & \cellorange{متوسط} & 
        \cellgreen{بالا} & \cellorange{متوسط} & 
        \cellorange{متوسط} & \cellred{خیلی کم} & 
        \cellred{کم} \\
        
        فوریت نظارت & 
        \rating{5} & \rating{3} & \rating{4} & 
        \rating{2} & \rating{5} & \rating{3} \\
        \altrow
        
        پیچیدگی & 
        \rating{5} & \rating{3} & \rating{4} & 
        \rating{2} & \rating{5} & \rating{4} \\
        
        شانس موفقیت & 
        \rating{2} & \rating{4} & \rating{3} & 
        \rating{2} & \rating{1} & \rating{2} \\
        \altrow
        
        مدل غالب & ۴+۵ → ۶ & ۶ استاندارد & 
        ۴ → ۶ & ۲ → ۳ & ۵ (اجباری) & ۲+HR \\
        
        زمان واکنش & ۴۸ ساعت & هفته‌ها & 
        ۱-۴ هفته & ماه‌ها & فوری & مزمن \\
        \altrow
        
        هزینه نظارت & \$\$\$\$ & \$\$\$ & 
        \$\$\$\$ & \$\$ & \$\$\$\$\$ & \$\$ \\
        
        نیروی انسانی & ۵-۱۰K & ۶-۱۲K & 
        ۵-۱۰K & ۱۰۰-۵۰۰ & ۱۰-۵۰K & ۵۰-۲۰۰ \\
        \altrow
        
        خطر اصلی & سوریه‌ای شدن & 
        مصادره & رادیکالیزم & 
        نمایشی بودن & فاجعه & فرسایش \\
        
        پیش‌نیاز & آماده‌باش & اپوزیسیون متحد & 
        شبکه مدنی & شکاف درون نظام & 
        — & صبر استراتژیک \\
        
        \bottomrule
    \end{tabularx}
\end{table}
\end{landscape}

\sectiondivider

% ============================================================
\section{درخت تصمیم: انتخاب مدل نظارتی 
بر اساس سناریو}
\label{sec:decision-tree}
% ============================================================

\begin{figure}[htbp]
    \centering
    \begin{tikzpicture}[
        decision/.style={
            diamond, draw=MainOrange, fill=OrangeBG,
            text width=2.4cm, align=center,
            inner sep=2pt, font=\tiny\bfseries,
            aspect=1.8
        },
        outcome/.style={
            rectangle, draw, rounded corners=3pt,
            text width=2.8cm, align=center,
            font=\tiny\bfseries, minimum height=0.9cm
        },
        good/.style={outcome, fill=GreenBG, draw=MainGreen},
        mid/.style={outcome, fill=OrangeBG, draw=MainOrange},
        bad/.style={outcome, fill=RedBG, draw=MainRed},
        arr/.style={-{Stealth[length=2mm]}, thick}
    ]
    
    % ریشه
    \node[decision] (root) at (0,0) 
        {نوع تغییر\\چیست؟};
    
    % سطح ۱
    \node[decision] (sudden) at (-5,-3) 
        {خلأ امنیتی\\وجود دارد؟};
    \node[decision] (negot) at (0,-3) 
        {سپاه در\\مذاکره شرکت\\می‌کند؟};
    \node[decision] (reform) at (5,-3) 
        {اصلاحات\\واقعی است؟};
    
    % سطح ۲
    \node[bad] (m45) at (-7,-6) 
        {مدل ۴+۵\\تثبیت فوری};
    \node[mid] (m4) at (-3,-6) 
        {مدل ۶\\فاز ۱ تقویت‌شده};
    \node[good] (m6std) at (-0.5,-6) 
        {مدل ۶\\استاندارد};
    \node[mid] (m6plus) at (2.5,-6) 
        {مدل ۶\\فاز ۱ تقویت‌شده};
    \node[good] (m23) at (4,-6) 
        {مدل ۲→۳\\تدریجی};
    \node[bad] (m2hr) at (7,-6) 
        {مدل ۲+HR\\فشار};
    
    % فلش‌های سطح ۰→۱
    \draw[arr] (root) -- (sudden) 
        node[midway, above left, font=\tiny] 
        {ناگهانی/انقلابی};
    \draw[arr] (root) -- (negot)
        node[midway, right, font=\tiny] 
        {مذاکره‌ای};
    \draw[arr] (root) -- (reform)
        node[midway, above right, font=\tiny] 
        {تدریجی/درونی};
    
    % فلش‌های سطح ۱→۲
    \draw[arr, MainRed] (sudden) -- (m45)
        node[midway, left, font=\tiny] {بله};
    \draw[arr, MainGreen] (sudden) -- (m4)
        node[midway, right, font=\tiny] {خیر};
    \draw[arr, MainGreen] (negot) -- (m6std)
        node[midway, left, font=\tiny] {بله};
    \draw[arr, MainOrange] (negot) -- (m6plus)
        node[midway, right, font=\tiny] {خیر};
    \draw[arr, MainGreen] (reform) -- (m23)
        node[midway, left, font=\tiny] {بله};
    \draw[arr, MainRed] (reform) -- (m2hr)
        node[midway, right, font=\tiny] {خیر};
    
    \end{tikzpicture}
    \caption{درخت تصمیم ساده‌شده: 
    انتخاب مدل نظارتی بر اساس 
    سناریوی گذار}
    \label{fig:decision-tree}
\end{figure}

\begin{operationalnote}
این درخت تصمیم ابزاری \emphgreen{ساده‌شده} 
برای تصمیم‌گیری سریع در لحظه‌ی بحران است. 
در واقعیت، تصمیم‌گیری نیاز به تحلیل 
عمیق‌تر و مشورت گسترده دارد. اما داشتن 
چنین ابزاری از تصمیم‌گیری ارتجالی 
بهتر است.
\end{operationalnote}

\sectiondivider

% ============================================================
\section{جمع‌بندی فصل}
\label{sec:ch4-summary}
% ============================================================

\begin{chaptersummary}

\textbf{آنچه در این فصل آموختیم:}

\begin{enumerate}[
    label=\textcolor{DarkGray}{\bfseries\arabic*.},
    itemsep=4pt
]
    \item \textbf{شش سناریوی محتمل} 
    برای گذار شناسایی شد: فروپاشی 
    ناگهانی، مذاکره، انقلاب مردمی، 
    اصلاح درونی، مداخله خارجی 
    و بحران ممتد.
    
    \item \textbf{مطلوب‌ترین سناریو} 
    گذار مذاکره‌ای است (کمترین هزینه 
    انسانی) اما \textbf{محتمل‌ترین} 
    ادامه‌ی بحران ممتد یا فروپاشی/ 
    انقلاب است.
    
    \item \textbf{هر سناریو مدل نظارتی 
    متفاوتی} می‌طلبد — از نظارت 
    اجرایی فوری (فروپاشی) تا 
    نظارت مشورتی صبورانه 
    (بحران ممتد).
    
    \item \textbf{مداخله‌ی نظامی خارجی} 
    باید قاطعانه رد شود — تمام 
    نمونه‌های تاریخی شکست‌خورده‌اند.
    
    \item \textbf{برنامه‌ی آماده‌باش} 
    حیاتی‌ترین توصیه‌ی این فصل 
    است: منتظر نمانید — از الان 
    برای هر سناریو آماده شوید.
    
    \item \textbf{واقعیت احتمالاً ترکیبی} 
    از چند سناریو خواهد بود — 
    مدل ترکیبی-تطبیقی (فصل ۳) 
    دقیقاً برای همین انعطاف‌پذیری 
    طراحی شده است.
    
    \item \textbf{درخت تصمیم} ابزاری 
    ساده اما مفید برای واکنش سریع 
    در لحظه‌ی بحران ارائه شد — 
    باید از پیش تمرین و بازبینی شود.
    
    \item \textbf{حتی در سناریوی بحران ممتد} 
    (محتمل‌ترین)، جامعه‌ی بین‌المللی 
    وظایف فعالی دارد: نظارت حقوق بشری، 
    حمایت از جامعه‌ی مدنی، آماده‌سازی 
    و مقابله با خستگی بین‌المللی.
\end{enumerate}

\vspace{6pt}
\begin{center}
    \textcolor{MainOrange}{
        \faArrowLeft\hspace{8pt}
        \textbf{فصل بعد: نهادها، بازیگران، 
        سازمان‌ها و نقش هر یک}
        \hspace{8pt}\faArrowLeft
    }
\end{center}

\end{chaptersummary}

\chapterend