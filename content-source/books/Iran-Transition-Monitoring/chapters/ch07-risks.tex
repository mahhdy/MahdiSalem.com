% ═══════════════════════════════════════════════════════════════════════════════
% فصل ۷: آسیب‌شناسی، ریسک‌ها و چالش‌های پیش‌رو
% فایل: chapters/ch07-risks.tex
% رنگ فصل: قرمز (MainRed)
% ═══════════════════════════════════════════════════════════════════════════════

\chapteropening{۷}{آسیب‌شناسی، ریسک‌ها و چالش‌های پیش‌رو}{MainRed}{%
هیچ طرحی برای نبرد از اولین برخورد با دشمن جان سالم به در نمی‌برد؛ اما طرح نداشتن به معنای شکست حتمی است.%
}{هلموت فون مولتکه}

\chapter{آسیب‌شناسی، ریسک‌ها و چالش‌های پیش‌رو}
\label{ch:risks}

\minitoc

% ─────────────────────────────────────────────────────────────────────────────
% خلاصه اجرایی
% ─────────────────────────────────────────────────────────────────────────────

\begin{executivesummary}
نظارت بین‌المللی بر گذار دموکراتیک، علی‌رغم ضرورت و فواید بالقوه، با ریسک‌ها و آسیب‌های جدی مواجه است که عدم شناسایی و مدیریت آن‌ها می‌تواند کل فرایند را به شکست بکشاند. این فصل به تحلیل انتقادی شش دسته ریسک می‌پردازد: \emph{آسیب‌شناسی مفهومی} (ادراک نئواستعماری، خستگی بین‌المللی)، \emph{ریسک‌های امنیتی} (بازگشت اقتدارگرایی، تجزیه، خشونت)، \emph{ریسک‌های سیاسی} (مصادره گذار، پوپولیسم)، \emph{ریسک‌های اقتصادی} (فروپاشی، الیگارشی)، \emph{ریسک‌های اجتماعی} (انتقام‌جویی، تضاد دیاسپورا-داخل)، و \emph{ریسک‌های خود نظارت} (ناکارآمدی، فساد، سوگیری). برای هر ریسک، نمونه‌های تاریخی، احتمال وقوع در ایران، و راهکارهای پیشگیری ارائه می‌شود. نقشه حرارتی ریسک و ماتریس پاسخ در پایان فصل، ابزار تصمیم‌گیری عملیاتی را فراهم می‌آورد.
\end{executivesummary}

\section{درآمد: چرا آسیب‌شناسی ضروری است؟}
\label{sec:risks-intro}

تاریخ گذارهای دموکراتیک مملو از شکست‌ها، عقب‌گردها و فاجعه‌های انسانی است. از هر سه تلاش گذار، تقریباً یکی به دموکراسی تحکیم‌یافته می‌رسد، یکی به اقتدارگرایی بازمی‌گردد، و یکی در منطقه خاکستری بین این دو معلق می‌ماند.\footnote{\متن‌لاتین{Diamond, Larry. "Facing Up to the Democratic Recession." \emph{Journal of Democracy} 26, no. 1 (2015): 141-155.}} نظارت بین‌المللی که قرار است این ریسک‌ها را کاهش دهد، خود می‌تواند منشأ ریسک‌های جدید باشد.

\begin{keypoint}
شناسایی پیشینی ریسک‌ها و طراحی مکانیزم‌های پاسخ، تفاوت میان گذار موفق و فاجعه است. تجربه عراق، لیبی، و میانمار نشان می‌دهد که حتی نیت خوب و منابع فراوان، بدون مدیریت ریسک، به شکست می‌انجامد.
\end{keypoint}

این فصل با رویکردی انتقادی و واقع‌بینانه، نه برای منصرف کردن از نظارت بین‌المللی، بلکه برای \emph{طراحی هوشمندتر} آن نگاشته شده است. هر ریسکی که شناسایی و برنامه‌ریزی شود، ریسکی است که می‌توان مدیریتش کرد.

\sectiondivider

% ═══════════════════════════════════════════════════════════════════════════════
\section{آسیب‌شناسی مفهومی نظارت بین‌المللی}
\label{sec:conceptual-pathology}
% ═══════════════════════════════════════════════════════════════════════════════

پیش از بررسی ریسک‌های عینی، باید به نقدهای بنیادین نظارت بین‌المللی پرداخت که در ادبیات آکادمیک و گفتمان‌های ضداستعماری مطرح‌اند.

\subsection{نقد نئواستعماری و «مهندسی دموکراسی»}
\label{subsec:neocolonial-critique}

منتقدان پسااستعماری معتقدند نظارت بین‌المللی اغلب پوششی برای تحمیل مدل‌های غربی حکمرانی است.\footnote{\متن‌لاتین{Paris, Roland. "International Peacebuilding and the 'Mission Civilisatrice'." \emph{Review of International Studies} 28, no. 4 (2002): 637-656.}} مفاهیمی چون \bilingual{ظرفیت‌سازی}{Capacity Building} یا \bilingual{حکمرانی خوب}{Good Governance} می‌توانند ابزار سلطه فرهنگی تعبیر شوند.

\begin{warningbox}
در جامعه ایرانی با حافظه تاریخی مداخلات خارجی (کودتای ۱۳۳۲، قرارداد ۱۹۱۹، حمایت از صدام)، حساسیت به هرگونه «قیمومیت» بین‌المللی بسیار بالاست. حتی کمک‌های صادقانه می‌توانند با مقاومت روبرو شوند.
\end{warningbox}

\textbf{راهکارهای پیشگیری:}
\begin{itemize}[nosep]
    \item تأکید مکرر بر \emph{مالکیت ملی} (ایرانیان تصمیم‌گیرنده، بین‌المللی‌ها تسهیل‌گر)
    \item اجتناب از زبان «نجات‌بخشانه» در ارتباطات عمومی
    \item حضور پررنگ چهره‌های بین‌المللی از جنوب جهانی (نه فقط غرب)
    \item شفافیت کامل درباره منافع کشورهای حامی
    \item مکانیزم شکایت و اعتراض برای شهروندان ایرانی
\end{itemize}

\subsection{خستگی بین‌المللی و کاهش تعهد}
\label{subsec:international-fatigue}

\bilingual{خستگی بین‌المللی}{International Fatigue} پدیده‌ای مستند است که طی آن توجه و منابع جامعه جهانی پس از ماه‌های اولیه کاهش می‌یابد.\footnote{\متن‌لاتین{Autesserre, Séverine. \emph{Peaceland: Conflict Resolution and the Everyday Politics of International Intervention}. Cambridge University Press, 2014.}}

\begin{casestudy}{افغانستان: از «ملت‌سازی» تا فراموشی}
تعهد اولیه جامعه بین‌المللی به افغانستان پس از ۲۰۰۱ با شعار «دیگر هرگز تنها نخواهید بود» آغاز شد. اما تدریجاً توجه به عراق، سپس سوریه، و سپس اوکراین معطوف شد. بودجه کمک‌ها از ۱۵.۷ میلیارد دلار در ۲۰۱۱ به ۴.۲ میلیارد در ۲۰۲۰ کاهش یافت. نتیجه: سقوط ۲۰۲۱ و بازگشت طالبان.
\end{casestudy}

\textbf{چرخه توجه بین‌المللی:}

\begin{figure}[htbp]
\centering
\begin{tikzpicture}[
    node distance=2.5cm,
    every node/.style={font=\small},
    phase/.style={rectangle, rounded corners, draw=MainRed, fill=LightRed, minimum width=2.8cm, minimum height=1cm, align=center},
    arrow/.style={-{Stealth[length=3mm]}, thick, MainRed}
]
    \node[phase] (crisis) {بحران اولیه\\توجه حداکثری};
    \node[phase, left=of crisis] (commit) {تعهد منابع\\کنفرانس کمک‌ها};
    \node[phase, below=of commit] (routine) {روتین شدن\\کاهش پوشش رسانه‌ای};
    \node[phase, below=of crisis] (fatigue) {خستگی\\کاهش بودجه};
    \node[phase, below right=1.5cm and -1cm of fatigue] (exit) {استراتژی خروج\\«پیروزی» اعلامی};
    
    \draw[arrow] (crisis) -- (commit);
    \draw[arrow] (commit) -- (routine);
    \draw[arrow] (routine) -- (fatigue);
    \draw[arrow] (fatigue) -- (exit);
    \draw[arrow, dashed, DarkGray] (exit) to[bend right=40] node[right, font=\footnotesize] {بحران جدید} (crisis);
\end{tikzpicture}
\caption{چرخه توجه بین‌المللی و خطر خستگی}
\label{fig:attention-cycle}
\end{figure}

\textbf{راهکارهای پیشگیری:}
\begin{itemize}[nosep]
    \item تعهدات چندساله الزام‌آور (نه سالانه)
    \item پیوند به منافع ملی کشورهای حامی (امنیت انرژی، مهاجرت، تروریسم)
    \item تنوع‌بخشی به منابع مالی (نه وابستگی به یک حامی)
    \item ایجاد ذی‌نفعان داخلی در کشورهای حامی (شرکت‌ها، دیاسپورا)
\end{itemize}

\subsection{رقابت ناظران و تضاد منافع}
\label{subsec:competing-monitors}

هنگامی که چندین نهاد بین‌المللی همزمان وارد می‌شوند، رقابت بر سر حوزه نفوذ، منابع، و اعتبار می‌تواند به ناهماهنگی و حتی تضاد بینجامد.

\begin{lessonlearned}{بوسنی: سه پا در یک کفش}
در بوسنی پس از جنگ، \lr{OHR}، \lr{OSCE}، \lr{EU}، \lr{NATO}، و \lr{UNDP} هر یک مأموریت‌های همپوشان داشتند. تعارض میان \lr{OHR} و \lr{EU} بر سر اولویت‌های اصلاحات، فرایند الحاق به اتحادیه اروپا را سال‌ها به تأخیر انداخت.
\end{lessonlearned}

\subsection{مشروعیت‌بخشی کاذب}
\label{subsec:false-legitimization}

نظارت بین‌المللی می‌تواند ناخواسته به فرایندهای غیردموکراتیک مشروعیت ببخشد.

\begin{casestudy}{میانمار ۲۰۱۰: انتخابات تأییدشده، دموکراسی کاذب}
انتخابات ۲۰۱۰ میانمار با حضور برخی ناظران بین‌المللی برگزار شد. اعلام «پیشرفت» توسط برخی نهادها، فشار بین‌المللی برای اصلاحات واقعی را کاهش داد. نتیجه: دموکراسی ناقصی که در ۲۰۲۱ با کودتا فروپاشید.
\end{casestudy}

\textbf{راهکارهای پیشگیری:}
\begin{itemize}[nosep]
    \item معیارهای صریح و علنی برای «موفقیت» از ابتدا
    \item گزارش‌دهی صادقانه حتی اگر منفی باشد
    \item اجتناب از فشار سیاسی برای اعلام «پیروزی» زودهنگام
    \item مکانیزم نظارت بر ناظران (ارزیابی مستقل)
\end{itemize}

% ═══════════════════════════════════════════════════════════════════════════════
\section{ریسک‌های امنیتی}
\label{sec:security-risks}
% ═══════════════════════════════════════════════════════════════════════════════

ریسک‌های امنیتی حیاتی‌ترین تهدیدات برای گذار موفق‌اند و می‌توانند تمام دستاوردها را یک‌شبه نابود کنند.

\subsection{بازگشت اقتدارگرایی}
\label{subsec:authoritarian-reversal}

\bilingual{بازگشت اقتدارگرایی}{Authoritarian Reversal} یا \bilingual{موج معکوس}{Reverse Wave} پدیده‌ای رایج است که طی آن نیروهای قدیم یا جدید، دستاوردهای دموکراتیک را به عقب می‌رانند.

\begin{warningbox}
مصر ۲۰۱۳ نمونه کلاسیک است: انتخابات آزاد ← رئیس‌جمهور منتخب ← کودتای نظامی ← سرکوب شدیدتر از قبل. این سناریو در ایران با حضور سپاه پاسداران، محتمل‌ترین و خطرناک‌ترین ریسک است.
\end{warningbox}

\begin{table}[htbp]
\centering
\caption{نمونه‌های بازگشت اقتدارگرایی پس از گذار}
\label{tab:reversal-examples}
\begin{tabularx}{\textwidth}{>{\raggedleft\arraybackslash}p{2cm} 
                             >{\raggedleft\arraybackslash}p{2cm}
                             >{\raggedleft\arraybackslash}X
                             >{\raggedleft\arraybackslash}X}
\toprule
\headerrow کشور & سال بازگشت & مکانیزم & پیامد \\
\midrule
مصر & ۲۰۱۳ & کودتای نظامی با حمایت مردمی & دیکتاتوری نظامی \\
\altrow تایلند & ۲۰۱۴ & کودتای نظامی & حکومت نظامی ممتد \\
ترکیه & ۲۰۱۶+ & تمرکز قدرت پس از کودتای نافرجام & اقتدارگرایی رقابتی \\
\altrow میانمار & ۲۰۲۱ & کودتای نظامی & جنگ داخلی \\
ونزوئلا & ۲۰۰۰+ & تضعیف تدریجی نهادها & اقتدارگرایی پوپولیستی \\
\bottomrule
\end{tabularx}
\end{table}

\textbf{عوامل افزایش‌دهنده ریسک در ایران:}
\begin{enumerate}[nosep]
    \item قدرت اقتصادی-نظامی سپاه پاسداران
    \item شبکه اطلاعاتی گسترده
    \item امکان بسیج بخشی از جامعه با روایت «هرج‌ومرج»
    \item حمایت احتمالی بازیگران خارجی (روسیه، چین)
    \item ضعف احتمالی نهادهای دموکراتیک نوپا
\end{enumerate}

\textbf{راهکارهای پیشگیری:}
\begin{enumerate}[nosep]
    \item اصلاح ساختاری نیروهای مسلح در فاز اول (\seeChapter{ch:guarantees})
    \item ممنوعیت قانون اساسی از دخالت نظامیان در سیاست
    \item نظارت بین‌المللی بر بخش امنیتی (نه فقط انتخابات)
    \item تقویت سریع نهادهای مدنی به‌عنوان موازنه
    \item ضمانت‌های بین‌المللی علیه کودتا (تهدید به تحریم فوری)
\end{enumerate}

\subsection{تجزیه و جنگ داخلی}
\label{subsec:fragmentation}

\begin{keypoint}
ایران با تنوع قومی-زبانی (فارس ۶۱٪، آذری ۱۶٪، کرد ۱۰٪، لر ۶٪، عرب ۲٪، بلوچ ۲٪، ترکمن ۲٪، و دیگران)، در صورت مدیریت نادرست گذار، مستعد بحران‌های قومی است.
\end{keypoint}

\begin{casestudy}{یوگسلاوی: از فدراسیون تا جنگ}
یوگسلاوی پس از مرگ تیتو و فروپاشی کمونیسم، نتوانست گذار مسالمت‌آمیز داشته باشد. رقابت رهبران قومی، ضعف نهادهای مشترک، و دخالت خارجی ناهماهنگ به جنگ‌های خونین ۱۹۹۱-۲۰۰۱ با بیش از ۱۴۰,۰۰۰ کشته و ۴ میلیون آواره انجامید.
\end{casestudy}

\textbf{تفاوت‌های ایران با یوگسلاوی:}
\begin{itemize}[nosep]
    \item[\cmark] هویت ملی ایرانی قوی‌تر از هویت یوگسلاوی
    \item[\cmark] سابقه طولانی همزیستی (بر خلاف ترکیب مصنوعی یوگسلاوی)
    \item[\cmark] نبود مرزهای داخلی فدرالی که تجزیه را تسهیل کند
    \item[\xmark] سابقه سرکوب اقوام در چهار دهه اخیر
    \item[\xmark] مرزهای طولانی با کشورهای بی‌ثبات
    \item[\xmark] امکان حمایت خارجی از جنبش‌های تجزیه‌طلب
\end{itemize}

\textbf{راهکارهای پیشگیری:}
\begin{enumerate}[nosep]
    \item تعهد صریح همه بازیگران به تمامیت ارضی
    \item طراحی ساختار نامتمرکز اما نه فدرال
    \item تضمین حقوق فرهنگی-زبانی اقوام
    \item مشارکت نمایندگان اقوام در دولت انتقالی
    \item نظارت بین‌المللی بر مناطق حساس مرزی
\end{enumerate}

\subsection{خشونت فرقه‌ای و انتقام‌جویی}
\label{subsec:sectarian-violence}

\begin{lessonlearned}{عراق: حمام خون فرقه‌ای}
سیاست «بعث‌زدایی» افراطی پس از ۲۰۰۳، همراه با انحلال ارتش، میلیون‌ها سنّی را به حاشیه راند. نتیجه: جنگ داخلی ۲۰۰۶-۲۰۰۸ با دهها هزار کشته، و سپس ظهور داعش. نظارت بین‌المللی نه تنها جلوی این فاجعه را نگرفت، بلکه آمریکا به‌عنوان «ناظر» خود عامل اصلی آن بود.
\end{lessonlearned}

\textbf{عوامل ریسک خشونت در ایران:}
\begin{itemize}[nosep]
    \item انباشت کینه‌های ۴۵ ساله
    \item گستردگی نیروهای امنیتی و بسیجی در محلات
    \item سلاح‌های پراکنده در صورت فروپاشی نظم
    \item امکان تحریک خارجی (داعش، القاعده)
\end{itemize}

\subsection{تروریسم و ناامنی}
\label{subsec:terrorism}

گذار می‌تواند خلأ امنیتی ایجاد کند که گروه‌های تروریستی از آن بهره‌برداری کنند.

\begin{table}[htbp]
\centering
\caption{تهدیدات تروریستی بالقوه در فرایند گذار ایران}
\label{tab:terror-threats}
\begin{tabularx}{\textwidth}{>{\raggedleft\arraybackslash}p{3cm}
                             >{\raggedleft\arraybackslash}p{2cm}
                             >{\raggedleft\arraybackslash}X
                             >{\raggedleft\arraybackslash}X}
\toprule
\headerrow تهدید & احتمال & مکانیزم & پیامد بالقوه \\
\midrule
داعش/القاعده & متوسط & نفوذ از مرزهای شرقی و غربی & حملات در مناطق سنّی‌نشین \\
\altrow گروه‌های شیعه افراطی & متوسط-بالا & مقاومت علیه «تغییر ضداسلامی» & ترور شخصیت‌های اصلاح‌طلب \\
جریان‌های قومی مسلح & متوسط & بهره‌برداری از خلأ امنیتی & ناامنی در مناطق مرزی \\
\altrow عناصر سپاه زیرزمینی & بالا & ضدانقلاب مسلح & خرابکاری و ترور \\
\bottomrule
\end{tabularx}
\end{table}

\sectiondivider

% ═══════════════════════════════════════════════════════════════════════════════
\section{ریسک‌های سیاسی}
\label{sec:political-risks}
% ═══════════════════════════════════════════════════════════════════════════════

\subsection{مصادره گذار توسط یک جناح}
\label{subsec:capture}

\bilingual{مصادره گذار}{Transition Capture} زمانی رخ می‌دهد که یک گروه سیاسی، به نام «انقلاب» یا «دموکراسی»، قدرت را قبضه می‌کند.

\begin{warningbox}
در تاریخ ایران، انقلاب ۱۳۵۷ نمونه کلاسیک مصادره است: جنبش گسترده با خواسته‌های متنوع، توسط یک جریان خاص مصادره شد. خطر تکرار این الگو جدی است.
\end{warningbox}

\textbf{سناریوهای مصادره:}
\begin{enumerate}[nosep]
    \item \textbf{مصادره نظامی}: ارتش یا سپاه «انتظامی» به نام «ثبات» قدرت را می‌گیرد
    \item \textbf{مصادره ایدئولوژیک}: یک جریان سیاسی (چپ یا راست افراطی) رقبا را حذف می‌کند
    \item \textbf{مصادره قومی}: یک قوم غالب، دیگران را به حاشیه می‌راند
    \item \textbf{مصادره الیگارشیک}: نخبگان اقتصادی، دموکراسی صوری ایجاد می‌کنند
\end{enumerate}

\textbf{راهکارهای پیشگیری:}
\begin{itemize}[nosep]
    \item ائتلاف گذار با نمایندگان واقعی همه جریان‌ها
    \item قانون انتخابات تناسبی (نه اکثریتی)
    \item تضمین‌های حقوق اقلیت در قانون اساسی
    \item نظارت بین‌المللی بر فراگیری فرایند (نه فقط انتخابات)
\end{itemize}

\subsection{بن‌بست سیاسی و فلج نهادی}
\label{subsec:deadlock}

\begin{casestudy}{بلژیک و عراق: دولت‌های بدون دولت}
بلژیک در ۲۰۱۰-۲۰۱۱ برای ۵۴۱ روز بدون دولت بود. عراق پس از انتخابات ۲۰۱۰ نُه ماه طول کشید تا دولت تشکیل شود. در ایران با چنددستگی اپوزیسیون، این سناریو محتمل است و می‌تواند مردم را از دموکراسی سرخورده کند.
\end{casestudy}

\subsection{پوپولیسم و عوام‌فریبی}
\label{subsec:populism}

گذارهای دموکراتیک زمین حاصلخیزی برای پوپولیست‌هاست که با وعده‌های ساده‌انگارانه، رأی‌دهندگان سرخورده را جذب می‌کنند.

\begin{lessonlearned}{ونزوئلا: از دموکراسی تا چاوزیسم}
ونزوئلا در دهه ۱۹۹۰ یکی از باثبات‌ترین دموکراسی‌های آمریکای لاتین بود. فساد نخبگان و نابرابری، زمینه را برای ظهور چاوز فراهم کرد که با وعده «انقلاب» بر سر کار آمد و تدریجاً نهادهای دموکراتیک را تخریب کرد. ناظران بین‌المللی تا دیر متوجه روند نشدند.
\end{lessonlearned}

\sectiondivider

% ═══════════════════════════════════════════════════════════════════════════════
\section{ریسک‌های اقتصادی}
\label{sec:economic-risks}
% ═══════════════════════════════════════════════════════════════════════════════

\subsection{فروپاشی اقتصادی}
\label{subsec:economic-collapse}

گذار سیاسی اغلب با بحران اقتصادی همراه است: کاهش سرمایه‌گذاری، فرار سرمایه، تورم، و بیکاری.

\begin{keypoint}
اقتصاد ایران پیش از گذار در وضعیت بحرانی است: تورم ۴۰-۵۰٪، بیکاری رسمی ۱۰٪ (واقعی ۲۵-۳۰٪)، ارزش ریال در ۱۰ سال ۹۰٪ کاهش، ذخایر ارزی محدود. گذار می‌تواند این بحران را تشدید یا تخفیف دهد.
\end{keypoint}

\textbf{سناریوی بدبینانه:}
\begin{itemize}[nosep]
    \item فرار سرمایه گسترده در هفته‌های اول
    \item توقف صادرات نفت به دلیل بی‌ثباتی
    \item هجوم به بانک‌ها و فروپاشی سیستم مالی
    \item کاهش ۳۰-۵۰٪ تولید ناخالص داخلی (مشابه عراق پس از ۲۰۰۳)
\end{itemize}

\textbf{سناریوی خوش‌بینانه:}
\begin{itemize}[nosep]
    \item رفع سریع تحریم‌ها و آزادسازی دارایی‌های بلوکه‌شده (۱۰۰+ میلیارد دلار)
    \item بسته کمک بین‌المللی فوری (مشابه طرح مارشال)
    \item بازگشت اعتماد و سرمایه‌گذاری
    \item رشد ۵-۱۰٪ سالانه در سال‌های اول
\end{itemize}

\subsection{غارت دارایی‌ها و الیگارشی}
\label{subsec:looting-oligarchy}

\begin{casestudy}{روسیه دهه ۱۹۹۰: از کمونیسم تا الیگارشی}
خصوصی‌سازی سریع در روسیه، با نظارت ناکافی بین‌المللی، به انتقال دارایی‌های عمومی به گروهی کوچک از «الیگارش‌ها» انجامید. ثروت ۷ نفر از ثروتمندترین روس‌ها برابر نیمی از کل جمعیت شد. نارضایتی عمومی، زمینه‌ساز ظهور پوتین بود.
\end{casestudy}

\textbf{ریسک ویژه ایران:}
\begin{itemize}[nosep]
    \item امپراتوری اقتصادی سپاه (۲۰-۴۰٪ اقتصاد)
    \item بنیادها و نهادهای شبه‌دولتی
    \item شبکه‌های رانت دهه‌ها قدیمی
    \item فقدان شفافیت مالکیت واقعی
\end{itemize}

\begin{recommendation}
قبل از هرگونه خصوصی‌سازی، باید:
\begin{enumerate}[nosep]
    \item حسابرسی کامل از دارایی‌های دولتی و شبه‌دولتی
    \item شناسایی مالکیت واقعی شرکت‌ها (نه صوری)
    \item ایجاد صندوق امانی بین‌المللی برای دارایی‌های استراتژیک
    \item قوانین ضدانحصار و ضدفساد قبل از خصوصی‌سازی
\end{enumerate}
\end{recommendation}

\sectiondivider

% ═══════════════════════════════════════════════════════════════════════════════
\section{ریسک‌های اجتماعی}
\label{sec:social-risks}
% ═══════════════════════════════════════════════════════════════════════════════

\subsection{انتقام‌جویی و عدالت انتقامی}
\label{subsec:revenge}

تمایل طبیعی قربانیان سرکوب به انتقام می‌تواند چرخه خشونت ایجاد کند.

\begin{warningbox}
تجربه عراق نشان داد که «بعث‌زدایی» افراطی می‌تواند میلیون‌ها نفر را به دشمن گذار تبدیل کند. در ایران با میلیون‌ها عضو و وابسته به نهادهای حکومتی (سپاه، بسیج، نهادهای مذهبی)، سیاست انتقامی فاجعه‌بار خواهد بود.
\end{warningbox}

\textbf{تعادل دشوار:}
\begin{itemize}[nosep]
    \item عدالت برای قربانیان ضروری است
    \item انتقام جمعی چرخه خشونت ایجاد می‌کند
    \item بخشش بدون پاسخگویی، بی‌عدالتی است
    \item عفو عمومی، مصونیت برای جنایتکاران است
\end{itemize}

\textbf{راهکار پیشنهادی:}
\begin{enumerate}[nosep]
    \item تمرکز بر رهبران و آمران (نه مجریان رده‌پایین)
    \item کمیسیون حقیقت با فرصت اعتراف و کاهش مجازات
    \item جبران خسارت مادی و نمادین برای قربانیان
    \item منع از مناصب حساس (نه مجازات کیفری) برای همکاران رده‌میانی
\end{enumerate}

\subsection{تضاد دیاسپورا و داخل}
\label{subsec:diaspora-conflict}

\begin{keypoint}
دیاسپورای ایرانی (۴-۵ میلیون نفر) دارای سرمایه، تخصص، و شبکه‌های بین‌المللی است، اما ممکن است با واقعیت‌های داخل فاصله داشته باشد. تنش میان «آن‌ها که ماندند» و «آن‌ها که رفتند» می‌تواند به شکاف سیاسی بدل شود.
\end{keypoint}

\begin{table}[htbp]
\centering
\caption{منابع تنش دیاسپورا-داخل}
\label{tab:diaspora-tension}
\begin{tabularx}{\textwidth}{>{\raggedleft\arraybackslash}p{3.5cm}
                             >{\raggedleft\arraybackslash}X
                             >{\raggedleft\arraybackslash}X}
\toprule
\headerrow موضوع & دیدگاه دیاسپورا (احتمالی) & دیدگاه داخل (احتمالی) \\
\midrule
سرعت تغییرات & تغییرات رادیکال و سریع & محتاط‌تر، ترس از بی‌ثباتی \\
\altrow عدالت انتقالی & مجازات شدید عاملان & مصالحه و آرامش \\
رابطه با غرب & استقبال از حمایت غربی & نگرانی از وابستگی \\
\altrow اولویت‌های اقتصادی & آزادسازی سریع، جذب سرمایه & حمایت از صنایع داخلی \\
هویت ملی & تأکید بر هویت پیشااسلامی & تنوع بیشتر در تعریف هویت \\
\bottomrule
\end{tabularx}
\end{table}

\textbf{راهکارهای پیشگیری:}
\begin{itemize}[nosep]
    \item سهمیه معقول (نه غالب) برای دیاسپورا در نهادهای انتقالی
    \item مشوق بازگشت تدریجی، نه هجوم
    \item گفتگوی ساختاریافته دیاسپورا-داخل از همین حالا
    \item نظارت بر جریان سرمایه دیاسپورا برای جلوگیری از «استعمار اقتصادی»
\end{itemize}

\subsection{بحران هویت و خلأ ارزشی}
\label{subsec:identity-crisis}

فروپاشی ایدئولوژی حاکم می‌تواند به خلأ ارزشی بینجامد.

\begin{casestudy}{روسیه پس از شوروی: از کمونیسم تا نیهیلیسم}
فروپاشی شوروی نه تنها یک نظام سیاسی، بلکه یک جهان‌بینی را فرو ریخت. نرخ خودکشی، اعتیاد، و جرم در دهه ۱۹۹۰ به‌شدت افزایش یافت. امید به زندگی مردان از ۶۴ به ۵۷ سال کاهش یافت. این خلأ، زمینه‌ساز بازگشت به ناسیونالیسم اقتدارگرای پوتین شد.
\end{casestudy}

\sectiondivider

% ═══════════════════════════════════════════════════════════════════════════════
\section{ریسک‌های ناشی از خود نظارت بین‌المللی}
\label{sec:monitoring-risks}
% ═══════════════════════════════════════════════════════════════════════════════

نظارت بین‌المللی نه تنها ممکن است ریسک‌ها را کاهش ندهد، بلکه می‌تواند خود منشأ ریسک‌های جدید باشد.

\subsection{ناکافی بودن و نمادین شدن}
\label{subsec:insufficient}

\begin{table}[htbp]
\centering
\caption{طیف نظارت: از ناکافی تا بیش‌ازحد}
\label{tab:monitoring-spectrum}
\begin{tabularx}{\textwidth}{>{\raggedleft\arraybackslash}p{2.5cm}
                             >{\raggedleft\arraybackslash}X
                             >{\raggedleft\arraybackslash}X}
\toprule
\headerrow سطح & مشکل & نمونه \\
\midrule
ناکافی & مشروعیت‌بخشی به فرایند معیوب & میانمار ۲۰۱۰ \\
\altrow متوسط ضعیف & عدم توانایی جلوگیری از نقض & افغانستان ۲۰۱۴ \\
متوسط مناسب & تأثیر مثبت با محدودیت & تونس ۲۰۱۱-۲۰۱۴ \\
\altrow قوی & تأثیر قابل‌توجه & تیمور شرقی ۱۹۹۹ \\
بیش‌ازحد & تضعیف مالکیت ملی & عراق ۲۰۰۳ \\
\bottomrule
\end{tabularx}
\end{table}

\subsection{ناآشنایی فرهنگی و زبانی}
\label{subsec:cultural-ignorance}

\begin{warningbox}
بسیاری از ناظران بین‌المللی با زبان، فرهنگ، و پیچیدگی‌های ایران آشنا نیستند. این می‌تواند به:
\begin{itemize}[nosep]
    \item سوءتفاهم‌های ارتباطی
    \item تفسیر نادرست رویدادها
    \item نادیده گرفتن نشانه‌های هشدار
    \item توصیه‌های نامناسب
\end{itemize}
بینجامد.
\end{warningbox}

\textbf{راهکارهای پیشگیری:}
\begin{enumerate}[nosep]
    \item آموزش اجباری زبان و فرهنگ برای همه ناظران
    \item تیم‌های مختلط ایرانی-بین‌المللی
    \item مشاوران فرهنگی در همه سطوح
    \item فروتنی نهادی: اعتراف به محدودیت‌های دانش
\end{enumerate}

\subsection{فساد و سوءاستفاده}
\label{subsec:corruption}

\begin{casestudy}{بوسنی و کوزوو: فساد صلح‌بانان}
گزارش‌های متعدد از فساد، قاچاق، و حتی بهره‌برداری جنسی توسط کارکنان بین‌المللی در بوسنی و کوزوو منتشر شده است. در یک مورد، کارکنان \lr{DynCorp} (پیمانکار آمریکایی) در قاچاق زنان برای فحشا دست داشتند.
\end{casestudy}

\textbf{راهکارهای پیشگیری:}
\begin{itemize}[nosep]
    \item غربالگری دقیق کارکنان بین‌المللی
    \item کدهای رفتاری الزام‌آور با ضمانت اجرا
    \item مکانیزم شکایت امن برای شهروندان محلی
    \item نظارت بر ناظران توسط نهاد مستقل
    \item عدم مصونیت برای جرایم سنگین
\end{itemize}

\subsection{سوگیری و جانبداری}
\label{subsec:bias}

نهادهای بین‌المللی ممکن است به دلایل ژئوپلیتیکی، مالی، یا ایدئولوژیک، به نفع یک جناح داخلی سوگیری داشته باشند.

\textbf{منابع سوگیری:}
\begin{enumerate}[nosep]
    \item فشار کشورهای تأمین‌کننده بودجه
    \item ارتباطات پیشین با برخی گروه‌های اپوزیسیون
    \item ترجیحات ایدئولوژیک کارکنان
    \item تأثیر لابی‌های دیاسپورا
\end{enumerate}

\begin{recommendation}
برای کاهش سوگیری:
\begin{enumerate}[nosep]
    \item تنوع در منابع مالی (هیچ کشوری بیش از ۲۵٪)
    \item تنوع در ترکیب کارکنان (ملیتی، جنسیتی، تخصصی)
    \item ارزیابی مستقل دوره‌ای
    \item مکانیزم اعتراض برای طرف‌های معترض
\end{enumerate}
\end{recommendation}

\sectiondivider

% ═══════════════════════════════════════════════════════════════════════════════
\section{نقشه حرارتی ریسک}
\label{sec:risk-heatmap}
% ═══════════════════════════════════════════════════════════════════════════════

نمودار زیر ماتریس ریسک را بر اساس دو محور \emph{احتمال وقوع} و \emph{شدت پیامد} نشان می‌دهد:

\begin{figure}[htbp]
\centering
\begin{tikzpicture}[
    font=\footnotesize,
    cell/.style={minimum width=2.5cm, minimum height=1.2cm, align=center},
]
    % Grid
    \draw[step=2.5cm, DarkGray, thin] (0,0) grid (10,6);
    
    % Axis labels
    \node[rotate=90, anchor=center] at (-0.8,3) {\textbf{شدت پیامد}};
    \node[anchor=center] at (5,-0.5) {\textbf{احتمال وقوع}};
    
    % Y-axis labels
    \node[anchor=east] at (0,0.6) {پایین};
    \node[anchor=east] at (0,1.8) {};
    \node[anchor=east] at (0,3) {متوسط};
    \node[anchor=east] at (0,4.2) {};
    \node[anchor=east] at (0,5.4) {بالا};
    
    % X-axis labels  
    \node[anchor=north] at (1.25,0) {پایین};
    \node[anchor=north] at (3.75,0) {متوسط};
    \node[anchor=north] at (6.25,0) {متوسط-بالا};
    \node[anchor=north] at (8.75,0) {بالا};
    
    % Color zones (background)
    \fill[green!20] (0,0) rectangle (2.5,2);
    \fill[yellow!30] (2.5,0) rectangle (5,2);
    \fill[yellow!30] (0,2) rectangle (2.5,4);
    \fill[orange!30] (5,0) rectangle (7.5,2);
    \fill[orange!30] (2.5,2) rectangle (5,4);
    \fill[orange!30] (0,4) rectangle (2.5,6);
    \fill[red!30] (7.5,0) rectangle (10,2);
    \fill[red!30] (5,2) rectangle (7.5,4);
    \fill[red!30] (2.5,4) rectangle (5,6);
    \fill[red!40] (7.5,2) rectangle (10,4);
    \fill[red!40] (5,4) rectangle (7.5,6);
    \fill[red!50] (7.5,4) rectangle (10,6);
    
    % Risk items
    \node[cell, fill=white, draw=MainRed, thick, rounded corners] at (8.75,5.4) {\textbf{بازگشت اقتدار}\\(سپاه)};
    \node[cell, fill=white, draw=MainRed, thick, rounded corners] at (6.25,5.4) {\textbf{فروپاشی}\\اقتصادی};
    \node[cell, fill=white, draw=MainOrange, thick, rounded corners] at (6.25,3) {\textbf{خشونت}\\فرقه‌ای};
    \node[cell, fill=white, draw=MainOrange, thick, rounded corners] at (3.75,5.4) {\textbf{تجزیه}\\ارضی};
    \node[cell, fill=white, draw=MainOrange, thick, rounded corners] at (8.75,3) {\textbf{مصادره}\\گذار};
    \node[cell, fill=white, draw=MainYellow, thick, rounded corners] at (6.25,1.2) {\textbf{خستگی}\\بین‌المللی};
    \node[cell, fill=white, draw=MainYellow, thick, rounded corners] at (3.75,3) {\textbf{پوپولیسم}};
    \node[cell, fill=white, draw=MainGreen, thick, rounded corners] at (1.25,3) {\textbf{مداخله}\\نظامی};
    
    % Legend
    \node[anchor=west] at (11,5) {\colorbox{red!50}{\ \ } بحرانی};
    \node[anchor=west] at (11,4) {\colorbox{red!30}{\ \ } بالا};
    \node[anchor=west] at (11,3) {\colorbox{orange!30}{\ \ } متوسط};
    \node[anchor=west] at (11,2) {\colorbox{yellow!30}{\ \ } پایین-متوسط};
    \node[anchor=west] at (11,1) {\colorbox{green!20}{\ \ } پایین};
    
\end{tikzpicture}
\caption{نقشه حرارتی ریسک‌های گذار ایران}
\label{fig:risk-heatmap}
\end{figure}

\sectiondivider

% ═══════════════════════════════════════════════════════════════════════════════
\section{ماتریس پاسخ به ریسک}
\label{sec:risk-response-matrix}
% ═══════════════════════════════════════════════════════════════════════════════

جدول زیر برای هر ریسک اصلی، راهکارهای پیشگیری، کشف زودهنگام، و پاسخ را خلاصه می‌کند:

\begin{landscape}
\begin{table}[htbp]
\centering
\bigtablefontsize
\caption{ماتریس جامع پاسخ به ریسک}
\label{tab:risk-response-matrix}
\begin{tabularx}{\linewidth}{>{\raggedleft\arraybackslash}p{2cm}
                             >{\raggedleft\arraybackslash}p{1.5cm}
                             >{\raggedleft\arraybackslash}p{1.5cm}
                             >{\raggedleft\arraybackslash}X
                             >{\raggedleft\arraybackslash}X
                             >{\raggedleft\arraybackslash}X
                             >{\raggedleft\arraybackslash}p{2cm}}
\toprule
\headerrow ریسک & احتمال & شدت & پیشگیری & کشف زودهنگام & پاسخ & مسئول اصلی \\
\midrule
بازگشت اقتدارگرایی & \riskhigh & \riskhigh & اصلاح سپاه، ممنوعیت قانونی، نظارت امنیتی & پایش جابجایی نظامی، اطلاعات انسانی & فشار فوری بین‌المللی، تحریم، انزوا & شورای امنیت \\
\altrow تجزیه ارضی & \riskmedium & \riskhigh & فدرالیسم نامتقارن، حقوق اقوام، مشارکت & پایش تنش‌های قومی، رسانه‌های محلی & میانجیگری فوری، مذاکره، حفظ صلح & \lr{SRSG} \\
خشونت فرقه‌ای & \riskmedium & \riskhigh & عدالت انتقالی متوازن، گفتگوی ملی & گزارش خشونت، نشانه‌های تحریک & نیروی حفظ صلح، میانجیگری، عدالت & \lr{OHCHR} \\
\altrow فروپاشی اقتصادی & \riskmedium & \riskhigh & رفع تحریم سریع، بسته کمک، صندوق امانی & شاخص‌های اقتصادی، نرخ ارز & تزریق نقدینگی، کمک غذایی فوری & \lr{IMF/WB} \\
مصادره گذار & \riskhigh & \riskmedium & ائتلاف فراگیر، قانون تناسبی، نظارت بر فراگیری & پایش تمرکز قدرت، شکایات احزاب & فشار دیپلماتیک، مشروط‌سازی کمک & \lr{EU/UN} \\
\altrow پوپولیسم & \riskmedium & \riskmedium & آموزش شهروندی، رسانه مستقل، نهادهای قوی & نظرسنجی، تحلیل گفتمان & افشاگری، حمایت از رسانه، آموزش & جامعه مدنی \\
خستگی بین‌المللی & \riskmedium & \riskmedium & تعهدات چندساله، پیوند به منافع ملی & کاهش بودجه، کاهش پوشش رسانه‌ای & کمپین یادآوری، لابی دیاسپورا & دولت انتقالی \\
\altrow الیگارشی & \riskmedium & \riskmedium & شفافیت مالکیت، ضدانحصار، خصوصی‌سازی تدریجی & تمرکز ثروت، معاملات مشکوک & تحقیق، مصادره، اصلاح قوانین & \lr{TI/WB} \\
ناآشنایی فرهنگی & \riskhigh & \risklow & آموزش اجباری، تیم مختلط، مشاوران محلی & شکایات، سوءتفاهم‌ها & بازآموزی، جایگزینی & \lr{SRSG} \\
\altrow فساد ناظران & \risklow & \riskmedium & غربالگری، کد رفتاری، نظارت بر ناظر & گزارش‌های افشاگر، شکایات & اخراج، پیگرد، جبران & بازرسی \lr{UN} \\
\bottomrule
\end{tabularx}
\end{table}
\end{landscape}

\sectiondivider

% ═══════════════════════════════════════════════════════════════════════════════
\section{نظام هشدار زودهنگام}
\label{sec:early-warning}
% ═══════════════════════════════════════════════════════════════════════════════

برای مدیریت مؤثر ریسک، نظام \bilingual{هشدار زودهنگام}{Early Warning System} ضروری است. این نظام باید شاخص‌های پیشرو را پایش کند و قبل از بحران، هشدار دهد.

\subsection{شاخص‌های پیشرو برای هر دسته ریسک}
\label{subsec:leading-indicators}

\begin{table}[htbp]
\centering
\caption{شاخص‌های هشدار زودهنگام}
\label{tab:early-warning-indicators}
\begin{tabularx}{\textwidth}{>{\raggedleft\arraybackslash}p{2.5cm}
                             >{\raggedleft\arraybackslash}X
                             >{\raggedleft\arraybackslash}p{2.5cm}
                             >{\raggedleft\arraybackslash}p{2cm}}
\toprule
\headerrow دسته ریسک & شاخص‌های کلیدی & منبع داده & فرکانس پایش \\
\midrule
امنیتی & جابجایی نیروها، خرید سلاح، بیانیه‌های تهدیدآمیز، ترور & اطلاعاتی، رسانه، \lr{OSINT} & روزانه \\
\altrow سیاسی & نظرسنجی اعتماد، شکایات انتخاباتی، خشونت سیاسی & نظرسنجی، گزارش احزاب & هفتگی \\
اقتصادی & نرخ ارز، تورم، بیکاری، صف نان/بنزین & بانک مرکزی، میدانی & روزانه \\
\altrow اجتماعی & تنش‌های قومی، خشونت محلی، مهاجرت داخلی & گزارش‌های محلی، \lr{UNHCR} & هفتگی \\
نظارتی & شکایات از ناظران، تأخیر در تصمیم‌گیری & سامانه شکایت & ماهانه \\
\bottomrule
\end{tabularx}
\end{table}

\subsection{ساختار نظام هشدار}
\label{subsec:warning-structure}

\begin{figure}[htbp]
\centering
\begin{tikzpicture}[
    node distance=1.5cm,
    every node/.style={font=\small},
    source/.style={rectangle, rounded corners, draw=MainBlue, fill=LightBlue, minimum width=2cm, minimum height=0.8cm, align=center},
    process/.style={rectangle, rounded corners, draw=MainOrange, fill=LightOrange, minimum width=2.5cm, minimum height=0.8cm, align=center},
    output/.style={rectangle, rounded corners, draw=MainRed, fill=LightRed, minimum width=2cm, minimum height=0.8cm, align=center},
    arrow/.style={-{Stealth[length=2.5mm]}, thick}
]
    % Sources
    \node[source] (s1) {منابع اطلاعاتی};
    \node[source, below=0.5cm of s1] (s2) {گزارش‌های میدانی};
    \node[source, below=0.5cm of s2] (s3) {رسانه‌ها و \lr{OSINT}};
    \node[source, below=0.5cm of s3] (s4) {نظرسنجی‌ها};
    \node[source, below=0.5cm of s4] (s5) {شکایات شهروندی};
    
    % Processing
    \node[process, right=2cm of s3] (collect) {جمع‌آوری\\و تأیید};
    \node[process, right=1.5cm of collect] (analyze) {تحلیل\\و امتیازدهی};
    \node[process, right=1.5cm of analyze] (decide) {تصمیم‌گیری\\و اولویت‌بندی};
    
    % Outputs
    \node[output, right=2cm of decide] (green) {\textcolor{MainGreen}{\faCheckCircle} عادی};
    \node[output, above=0.3cm of green] (yellow) {\textcolor{MainYellow}{\faExclamationCircle} هشدار};
    \node[output, above=0.3cm of yellow] (red) {\textcolor{MainRed}{\faExclamationTriangle} بحران};
    
    % Arrows
    \draw[arrow] (s1.east) -- ++(0.5,0) |- (collect.west);
    \draw[arrow] (s2.east) -- ++(0.5,0) |- (collect.west);
    \draw[arrow] (s3.east) -- (collect.west);
    \draw[arrow] (s4.east) -- ++(0.5,0) |- (collect.west);
    \draw[arrow] (s5.east) -- ++(0.5,0) |- (collect.west);
    
    \draw[arrow] (collect) -- (analyze);
    \draw[arrow] (analyze) -- (decide);
    
    \draw[arrow] (decide.east) -- ++(0.5,0) |- (green.west);
    \draw[arrow] (decide.east) -- ++(0.5,0) |- (yellow.west);
    \draw[arrow] (decide.east) -- ++(0.5,0) |- (red.west);
    
\end{tikzpicture}
\caption{ساختار نظام هشدار زودهنگام}
\label{fig:early-warning-structure}
\end{figure}

\begin{keypoint}
نظام هشدار زودهنگام باید:
\begin{itemize}[nosep]
    \item \textbf{چندمنبعی} باشد (اتکا به یک منبع خطرناک است)
    \item \textbf{محلی‌محور} باشد (تهران‌محوری کافی نیست)
    \item \textbf{سریع} باشد (تأخیر = فاجعه)
    \item \textbf{مستقل} باشد (تحت فشار سیاسی نباشد)
    \item \textbf{عملیاتی} باشد (هشدار بدون پاسخ بی‌فایده است)
\end{itemize}
\end{keypoint}

\sectiondivider

% ═══════════════════════════════════════════════════════════════════════════════
\section{سطوح پاسخ به بحران}
\label{sec:crisis-response-levels}
% ═══════════════════════════════════════════════════════════════════════════════

برای هر سطح هشدار، پروتکل پاسخ از پیش تعریف‌شده باید وجود داشته باشد:

\begin{table}[htbp]
\centering
\caption{پروتکل پاسخ به سطوح مختلف بحران}
\label{tab:crisis-protocol}
\begin{tabularx}{\textwidth}{>{\raggedleft\arraybackslash}p{2cm}
                             >{\raggedleft\arraybackslash}p{2cm}
                             >{\raggedleft\arraybackslash}X
                             >{\raggedleft\arraybackslash}X
                             >{\raggedleft\arraybackslash}p{2.5cm}}
\toprule
\headerrow سطح & رنگ & شرایط فعال‌سازی & اقدامات & تصمیم‌گیر \\
\midrule
۱ & \cellgreen{سبز} & شاخص‌ها در محدوده عادی & پایش روتین، گزارش ماهانه & تیم پایش \\
\altrow ۲ & \cellorange{زرد} & یک شاخص در آستانه هشدار & افزایش فرکانس پایش، تحلیل علت & رئیس بخش \\
۳ & \cellorange{نارنجی} & چند شاخص در آستانه یا یک شاخص بحرانی & فعال‌سازی تیم بحران، گزارش فوری & معاون \lr{SRSG} \\
\altrow ۴ & \cellred{قرمز} & بحران فعال یا قریب‌الوقوع & جلسه اضطراری، اقدام میدانی، اطلاع به \lr{SC} & \lr{SRSG} \\
۵ & \cellred{قرمز تیره} & فاجعه گسترده & درخواست مداخله، تخلیه، بازنگری کل مأموریت & دبیرکل/\lr{SC} \\
\bottomrule
\end{tabularx}
\end{table}

\begin{warningbox}
مهم‌ترین درس از شکست‌های گذشته (رواندا ۱۹۹۴، سربرنیتسا ۱۹۹۵): هشدار بدون پاسخ بی‌فایده است. در هر دو مورد، هشدارهای کافی وجود داشت اما اراده سیاسی برای اقدام نبود. ساختار نظارت باید شامل تعهد قبلی به اقدام در صورت هشدار باشد.
\end{warningbox}

\sectiondivider

% ═══════════════════════════════════════════════════════════════════════════════
\section{راهکارهای کلان کاهش ریسک}
\label{sec:macro-risk-mitigation}
% ═══════════════════════════════════════════════════════════════════════════════

فراتر از پاسخ به ریسک‌های خاص، راهکارهای کلانی وجود دارد که مجموعه ریسک‌ها را کاهش می‌دهد:

\subsection{تقویت تاب‌آوری سیستمی}
\label{subsec:systemic-resilience}

\bilingual{تاب‌آوری}{Resilience} توانایی سیستم برای جذب شوک و بازیابی است. گذار تاب‌آور گذاری است که یک بحران آن را به عقب نمی‌راند.

\textbf{عناصر تاب‌آوری:}
\begin{enumerate}[nosep]
    \item \textbf{تنوع}: عدم اتکا به یک نهاد، یک رهبر، یک منبع مالی
    \item \textbf{افزونگی}: پشتیبان برای هر عنصر حیاتی
    \item \textbf{مدولاریت}: توانایی عملکرد بخش‌ها به‌صورت مستقل در بحران
    \item \textbf{انطباق‌پذیری}: ظرفیت تغییر سریع استراتژی
    \item \textbf{یادگیری}: مکانیزم درس‌آموزی از اشتباهات
\end{enumerate}

\subsection{ایجاد ذی‌نفعان متعدد در موفقیت}
\label{subsec:multiple-stakeholders}

هرچه افراد و گروه‌های بیشتری منفعت در موفقیت گذار داشته باشند، شکست دادن آن سخت‌تر است.

\begin{recommendation}
استراتژی «ذی‌نفع‌سازی گسترده»:
\begin{itemize}[nosep]
    \item توزیع فرصت‌های شغلی در مناطق مختلف
    \item سهام‌دار کردن کارگران در صنایع خصوصی‌شده
    \item مشارکت محلی در پروژه‌های بازسازی
    \item پیوند منافع اقتصادی طبقه متوسط به ثبات دموکراتیک
    \item ایجاد انگیزه برای بازگشت دیاسپورا
\end{itemize}
\end{recommendation}

\subsection{حفظ انعطاف استراتژیک}
\label{subsec:strategic-flexibility}

\begin{lessonlearned}{تونس: انعطاف نجات‌بخش}
وقتی پیش‌نویس اول قانون اساسی تونس با اعتراض روبرو شد، به‌جای اصرار، فرایند بازنگری شد. این انعطاف که برخی آن را «ضعف» می‌خواندند، در نهایت به اجماع گسترده‌تر انجامید. در مقابل، اصرار مرسی در مصر بر قانون اساسی اسلام‌گرایانه، زمینه‌ساز کودتا شد.
\end{lessonlearned}

\textbf{اصول انعطاف استراتژیک:}
\begin{itemize}[nosep]
    \item اهداف غیرقابل مذاکره (دموکراسی، حقوق بشر) ثابت
    \item روش‌ها و زمان‌بندی قابل تعدیل
    \item بازنگری دوره‌ای استراتژی (هر ۶ ماه)
    \item پذیرش اشتباه و اصلاح مسیر
\end{itemize}

\sectiondivider

% ═══════════════════════════════════════════════════════════════════════════════
\section{تحلیل سناریوی بدترین حالت}
\label{sec:worst-case-scenario}
% ═══════════════════════════════════════════════════════════════════════════════

برنامه‌ریزی برای بدترین حالت، نه نشانه بدبینی، بلکه الزام حرفه‌ای است.

\subsection{سناریوی فاجعه: «سوریه‌ای شدن»}
\label{subsec:syria-scenario}

\begin{warningbox}
\textbf{سناریوی فاجعه:} گذار با فروپاشی ناگهانی آغاز می‌شود ← خلأ امنیتی ← سپاه در برخی مناطق مقاومت می‌کند ← گروه‌های قومی مسلح می‌شوند ← مداخله کشورهای منطقه (عربستان، ترکیه، اسرائیل) ← جنگ داخلی تمام‌عیار ← میلیون‌ها آواره ← ظهور گروه‌های تروریستی ← فروپاشی کامل دولت ← بحران بشردوستانه عظیم.

\textbf{احتمال:} پایین (۵-۱۰٪) اما نه صفر
\textbf{پیامد:} فاجعه‌بار برای ایران، منطقه و جهان
\end{warningbox}

\textbf{عوامل جلوگیری:}
\begin{enumerate}[nosep]
    \item اجماع بین‌المللی قوی قبل از گذار
    \item آمادگی قبلی نیروهای جایگزین امنیتی
    \item توافق با بخشی از سپاه (شکستن یکپارچگی)
    \item مداخله سریع بشردوستانه و امنیتی
    \item جلوگیری از تسلیح گروه‌های غیردولتی
\end{enumerate}

\subsection{طرح اضطراری}
\label{subsec:contingency-plan}

\begin{table}[htbp]
\centering
\caption{اقدامات اضطراری در سناریوی فاجعه}
\label{tab:contingency-actions}
\begin{tabularx}{\textwidth}{>{\raggedleft\arraybackslash}p{2.5cm}
                             >{\raggedleft\arraybackslash}X
                             >{\raggedleft\arraybackslash}p{3cm}
                             >{\raggedleft\arraybackslash}p{2.5cm}}
\toprule
\headerrow بازه زمانی & اقدام & مسئول & پیش‌نیاز \\
\midrule
ساعات اول & تماس با همه طرف‌ها برای آتش‌بس & دبیرکل \lr{UN} & کانال ارتباطی از قبل \\
\altrow ۲۴-۷۲ ساعت & نشست اضطراری شورای امنیت & اعضای \lr{P5} & پیش‌نویس قطعنامه آماده \\
هفته اول & استقرار ناظران غیرمسلح در مناطق امن & \lr{DPPA} & تیم آماده‌باش \\
\altrow هفته ۱-۴ & کریدور بشردوستانه، کمک غذایی & \lr{OCHA}, \lr{WFP} & پیش‌موقعیت‌یابی کمک‌ها \\
ماه ۱-۳ & استقرار نیروی حفظ صلح (در صورت تصویب) & \lr{DPKO} & آمادگی کشورهای مشارکت‌کننده \\
\bottomrule
\end{tabularx}
\end{table}

\sectiondivider

% ═══════════════════════════════════════════════════════════════════════════════
% جمع‌بندی فصل
% ═══════════════════════════════════════════════════════════════════════════════

\begin{chaptersummary}
یافته‌های کلیدی این فصل:

\begin{enumerate}
    \item \textbf{ریسک‌ها اجتناب‌ناپذیرند}: هر گذاری با ریسک همراه است؛ هدف مدیریت است نه حذف.
    
    \item \textbf{بازگشت اقتدارگرایی جدی‌ترین تهدید است}: با حضور سپاه پاسداران، این ریسک در ایران بالاتر از میانگین است و نیاز به توجه ویژه دارد.
    
    \item \textbf{ریسک تجزیه قابل مدیریت است}: برخلاف نگرانی‌های رایج، ایران یوگسلاوی نیست و با سیاست‌های درست، می‌توان وحدت ملی را حفظ کرد.
    
    \item \textbf{اقتصاد می‌تواند ناجی یا قاتل گذار باشد}: رفع سریع تحریم‌ها و بسته کمک بین‌المللی حیاتی است.
    
    \item \textbf{نظارت بین‌المللی خود ریسک‌هایی دارد}: از ناکافی بودن تا فساد و سوگیری که باید با طراحی هوشمند کنترل شوند.
    
    \item \textbf{نظام هشدار زودهنگام ضروری است}: با شاخص‌های تعریف‌شده، منابع متنوع، و پروتکل پاسخ.
    
    \item \textbf{انعطاف استراتژیک کلید بقاست}: اصرار بر طرح‌های از پیش تعیین‌شده می‌تواند فاجعه‌بار باشد.
    
    \item \textbf{طرح اضطراری باید از قبل آماده باشد}: «امیدوار به بهترین، آماده برای بدترین».
\end{enumerate}

\vspace{0.5cm}
\textit{در فصل بعد (\ref{ch:requirements})، نیازمندی‌های انسانی، نهادی، فنی و حقوقی برای اجرای نظارت بین‌المللی مؤثر بررسی خواهد شد.}
\end{chaptersummary}

\chapterend