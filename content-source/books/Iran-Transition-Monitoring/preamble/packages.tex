% ╔══════════════════════════════════════════════════════════════════╗
% ║  نظارت بین‌المللی بر گذار دموکراتیک ایران                      ║
% ║  ابعاد، انتخاب‌ها و انتظارات                                    ║
% ║  نویسنده: مهدی سالم                                             ║
% ║  تاریخ آغاز: ژوئن ۲۰۲۵                                         ║
% ║  موتور حروف‌چینی: XeLaTeX                                       ║
% ║  پریامبل جامع — نسخه ۱.۰                                       ║
% ╚══════════════════════════════════════════════════════════════════╝
% ╔══════════════════════════════════════════════════════════════════╗
% ║  پریامبل اصلاح‌شده — نسخه ۱.۱                                   ║
% ╚══════════════════════════════════════════════════════════════════╝

% ============================================================
% بخش ۱: هندسه‌ی صفحه
% ============================================================
\usepackage[
a4paper,
top=2.5cm,
bottom=2.5cm,
inner=2.8cm,
outer=2.2cm,
headheight=15pt,
headsep=12pt,
footskip=30pt,
marginparwidth=1.5cm,
marginparsep=3mm,
includehead,
includefoot
]{geometry}

\usepackage{changepage}
\usepackage{afterpage}

% ============================================================
% بخش ۲: فونت و یونیکد (قبل از xepersian)
% ============================================================
\usepackage{fontspec}
\usepackage{xunicode}
\usepackage{xltxtra}

% ============================================================
% بخش ۳: ریاضی
% ============================================================
\usepackage{amsmath}
\usepackage{amssymb}
\usepackage{amsfonts}
\usepackage{mathtools}
\usepackage{unicode-math}          % ✅ اضافه شد برای \setmathfont

% ============================================================
% بخش ۴: رنگ‌ها
% ============================================================
\usepackage[
dvipsnames,
svgnames,
x11names,
table
]{xcolor}

% ---- رنگ‌های اصلی ----
\definecolor{MainPurple}{HTML}{6A0DAD}
\definecolor{LightPurple}{HTML}{E8D5F5}
\definecolor{DarkPurple}{HTML}{4A0078}
\definecolor{PurpleBG}{HTML}{F3E5F5}

\definecolor{MainBlue}{HTML}{1565C0}
\definecolor{LightBlue}{HTML}{BBDEFB}
\definecolor{DarkBlue}{HTML}{0D47A1}
\definecolor{BlueBG}{HTML}{E3F2FD}

\definecolor{MainGreen}{HTML}{2E7D32}
\definecolor{LightGreen}{HTML}{C8E6C9}
\definecolor{DarkGreen}{HTML}{1B5E20}
\definecolor{GreenBG}{HTML}{E8F5E9}

\definecolor{MainOrange}{HTML}{E65100}
\definecolor{LightOrange}{HTML}{FFE0B2}
\definecolor{DarkOrange}{HTML}{BF360C}
\definecolor{OrangeBG}{HTML}{FFF3E0}

\definecolor{MainRed}{HTML}{B71C1C}
\definecolor{LightRed}{HTML}{FFCDD2}
\definecolor{DarkRed}{HTML}{7F0000}
\definecolor{RedBG}{HTML}{FFEBEE}

\definecolor{MainYellow}{HTML}{F9A825}
\definecolor{LightYellow}{HTML}{FFF9C4}
\definecolor{DarkYellow}{HTML}{F57F17}
\definecolor{YellowBG}{HTML}{FFFDE7}

\definecolor{DarkGray}{HTML}{424242}
\definecolor{MediumGray}{HTML}{757575}
\definecolor{LightGray}{HTML}{E0E0E0}
\definecolor{VeryLightGray}{HTML}{F5F5F5}

\definecolor{LinkColor}{HTML}{1976D2}
\definecolor{CiteColor}{HTML}{6A1B9A}
\definecolor{URLColor}{HTML}{E65100}
\definecolor{CodeBG}{HTML}{263238}
\definecolor{CodeText}{HTML}{EEFFFF}
\definecolor{HighlightGold}{HTML}{FFD54F}
\definecolor{TableHeaderBG}{HTML}{E8EAF6}   % بنفش روشن برای هدر
\definecolor{TableAltRow}{HTML}{F8F8F8}     % خاکستری خیلی روشن برای

% ============================================================
% بخش ۵: جداول
% ============================================================
\usepackage{booktabs}
\usepackage{tabularx}
\usepackage{longtable}
\usepackage{ltablex}
\keepXColumns
\usepackage{multirow}
\usepackage{multicol}
\usepackage{array}
\usepackage{makecell}
\usepackage{colortbl}
\usepackage{threeparttable}
\usepackage{siunitx}

% انواع ستون سفارشی
\newcolumntype{L}[1]{>{\raggedleft\arraybackslash}p{#1}}
\newcolumntype{R}[1]{>{\raggedright\arraybackslash}p{#1}}
\newcolumntype{C}[1]{>{\centering\arraybackslash}p{#1}}
\newcolumntype{Y}{>{\centering\arraybackslash}X}

\setlength{\tabcolsep}{6pt}
\renewcommand{\arraystretch}{1.3}

% ============================================================
% بخش ۶: تصاویر
% ============================================================
\usepackage{graphicx}
\usepackage[export]{adjustbox}
\usepackage{float}
\usepackage[section]{placeins}     % ✅ فقط یک بار با آپشن
\usepackage{subcaption}
\usepackage{caption}
\usepackage{rotating}
\usepackage{pdflscape}
\usepackage{pdfpages}

\graphicspath{{figures/}{figures/diagrams/}{figures/charts/}}

\captionsetup{
	font={small},
	labelfont={bf, color=DarkGray},
	textfont={color=DarkGray},
	justification=centering,
	skip=8pt
}
\captionsetup[figure]{name=شکل}
\captionsetup[table]{name=جدول}

% ============================================================
% بخش ۷: TikZ — نسخهٔ سبک‌شده
% ============================================================
\usepackage{tikz}

\usetikzlibrary{
	arrows.meta,
	backgrounds,
	calc,
	chains,
	decorations.pathreplacing,
	decorations.markings,
	fit,
	intersections,
	matrix,
	patterns,
	positioning,
	quotes,
	shapes.geometric,
	shapes.multipart,
	shapes.symbols,
	trees,
	shadows
}
% ✅ استایل‌های TikZ که در فصول استفاده شده
\tikzset{
	% استایل‌های عمومی
	    % استایل برای نودهای فارسی
	persian/.style={
		execute at begin node={\beginR},
		execute at end node={\endR}
	},
	% استایل برای نودهای فارسی با align
	persian box/.style={
		persian,
		align=right,
		text width=#1
	},
	persian box/.default=10cm,	
	center/.style={
		rectangle,
		rounded corners=3pt,
		draw=MainPurple,
		fill=PurpleBG,
		minimum width=2cm,
		minimum height=1cm,
		align=center,
		font=\small
	},
	conn/.style={
		->,
		>=Stealth,
		thick,
		MainBlue
	},
	arr/.style={
		->,
		>=Stealth,
		thick,
		MainPurple
	},
	arrow/.style={
		->,
		>=Stealth,
		thick,
		MainBlue
	},
	box/.style={
		rectangle,
		rounded corners,
		draw=MainPurple,
		fill=PurpleBG,
		minimum width=2.5cm,
		minimum height=0.8cm,
		align=center,
		font=\small
	},
	process/.style={
		rectangle,
		rounded corners,
		draw=MainBlue,
		fill=BlueBG,
		minimum width=2cm,
		minimum height=1cm,
		align=center,
		font=\small
	},
	decision/.style={
		diamond,
		draw=MainOrange,
		fill=OrangeBG,
		minimum width=1.5cm,
		minimum height=1cm,
		align=center,
		font=\small,
		aspect=2
	},
	source/.style={
		rectangle,
		rounded corners,
		draw=MediumGray,
		fill=VeryLightGray,
		minimum width=2.5cm,
		minimum height=0.6cm,
		align=center,
		font=\footnotesize
	},
	phase/.style={
		rectangle,
		rounded corners=5pt,
		draw=MainPurple,
		fill=PurpleBG,
		minimum width=2cm,
		minimum height=1cm,
		align=center,
		font=\small
	}
}
\usepackage{pgfplots}
\pgfplotsset{compat=1.18}

\usepgfplotslibrary{
	colorbrewer,
	fillbetween,
	groupplots,
	statistics
}

\usepackage{pgf-pie}
\usepackage{forest}

% ============================================================
% بخش ۸: کادرها
% ============================================================
\usepackage[
most,
skins,
breakable
]{tcolorbox}

% ============================================================
% بخش ۹: آیکون‌ها — رفع تداخل
% ============================================================
\usepackage{fontawesome5}
\usepackage{pifont}

% ⚠️ قبل از marvosym، نمادهای تداخلی را ذخیره می‌کنیم
\usepackage{savesym}
\usepackage{wasysym}
\savesymbol{Cross}                 % ✅ ذخیره \Cross از wasysym
\savesymbol{Square}
\usepackage{marvosym}              % حالا marvosym بارگذاری می‌شود

% ============================================================
% بخش ۱۰: فهرست‌ها
% ============================================================
\usepackage{enumitem}
\setlist{noitemsep, topsep=4pt}
\setlist[itemize,1]{label=\textcolor{MainPurple}{■}}
\setlist[itemize,2]{label=\textcolor{MainBlue}{▪}}
\setlist[enumerate,1]{label=\textcolor{MainPurple}{\arabic*.}}

% ============================================================
% بخش ۱۱: عنوان‌بندی
% ============================================================
\usepackage{titlesec}
\usepackage{titletoc}
\usepackage{minitoc}
% ============================================================
% بخش ۱۲: سرصفحه — فقط fancyhdr
% ============================================================
\usepackage{fancyhdr}
\usepackage{lastpage}
% ⚠️ scrlayer-scrpage حذف شد — با fancyhdr تداخل دارد

% ============================================================
% بخش ۱۳: لینک‌ها و ارجاع
% ============================================================
\PassOptionsToPackage{hyphens}{url}  % ✅ قبل از hyperref

\usepackage[
colorlinks=true,
linkcolor=MainPurple,
citecolor=CiteColor,
urlcolor=URLColor,
bookmarks=true,
bookmarksnumbered=true,
pdfencoding=unicode,
breaklinks=true,
pdfauthor={مهدی سالم},
pdftitle={نظارت بین‌المللی بر گذار دموکراتیک ایران}
]{hyperref}

\usepackage[nameinlink, capitalise]{cleveref}
\usepackage{bookmark}

% ============================================================
% بخش ۱۴: فهرست منابع
% ============================================================
\usepackage[
backend=biber,
style=apa,
sorting=nyt,
maxcitenames=3,
maxbibnames=99
]{biblatex}

\addbibresource{references.bib}

% ============================================================
% بخش ۱۵: واژه‌نامه و نمایه
% ============================================================
\usepackage[
toc,
section=chapter,
style=long3col,
nonumberlist
]{glossaries}
\usepackage{glossaries-extra}
\usepackage{makeidx}
\makeindex
\makeglossaries

% ============================================================
% بخش ۱۶: پانویس
% ============================================================
\usepackage[perpage, bottom, hang]{footmisc}
\setlength{\footnotesep}{8pt}

% ============================================================
% بخش ۱۷: فاصله‌گذاری
% ============================================================
\usepackage{setspace}
\onehalfspacing

\usepackage{parskip}
\setlength{\parindent}{0pt}
\setlength{\parskip}{6pt plus 2pt minus 1pt}

\usepackage{ragged2e}

% ============================================================
% بخش ۱۸: ابزارهای متنوع
% ============================================================
\usepackage{xspace}
\usepackage{xparse}
\usepackage{etoolbox}
\usepackage{calc}
\usepackage{comment}
\usepackage{soul}

% ============================================================
% بخش ۱۹: نقل‌قول و متن
% ============================================================
\usepackage{epigraph}
\usepackage{csquotes}

% ============================================================
% بخش ۲۰: ضمائم
% ============================================================
\usepackage[toc, page]{appendix}

% ============================================================
% بخش ۲۱: تنظیمات سرریز
% ============================================================
\sloppy
\tolerance=1000
\emergencystretch=3em
\hfuzz=2pt
\vfuzz=2pt

% ⚠️ overfullrule را قبل از نسخه نهایی حذف کنید
% \overfullrule=5pt

% ============================================================
% پایان پریامبل — xepersian در main.tex بارگذاری شود
% ============================================================