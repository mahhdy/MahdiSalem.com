% ╔══════════════════════════════════════════════════════════════════╗
% ║  نظارت بین‌المللی بر گذار دموکراتیک ایران                      ║
% ║  ابعاد، انتخاب‌ها و انتظارات                                    ║
% ║  مهدی سالم — ۲۰۲۵                                              ║
% ╚══════════════════════════════════════════════════════════════════╝

\documentclass[
    a4paper,
    12pt,
    twoside,
    openright,
    final              % تغییر به draft برای نمایش سرریزها
]{book}
% بارگذاری بسته‌ها و تنظیمات
% ╔══════════════════════════════════════════════════════════════════╗
% ║  نظارت بین‌المللی بر گذار دموکراتیک ایران                      ║
% ║  ابعاد، انتخاب‌ها و انتظارات                                    ║
% ║  نویسنده: مهدی سالم                                             ║
% ║  تاریخ آغاز: ژوئن ۲۰۲۵                                         ║
% ║  موتور حروف‌چینی: XeLaTeX                                       ║
% ║  پریامبل جامع — نسخه ۱.۰                                       ║
% ╚══════════════════════════════════════════════════════════════════╝
% ╔══════════════════════════════════════════════════════════════════╗
% ║  پریامبل اصلاح‌شده — نسخه ۱.۱                                   ║
% ╚══════════════════════════════════════════════════════════════════╝

% ============================================================
% بخش ۱: هندسه‌ی صفحه
% ============================================================
\usepackage[
a4paper,
top=2.5cm,
bottom=2.5cm,
inner=2.8cm,
outer=2.2cm,
headheight=15pt,
headsep=12pt,
footskip=30pt,
marginparwidth=1.5cm,
marginparsep=3mm,
includehead,
includefoot
]{geometry}

\usepackage{changepage}
\usepackage{afterpage}

% ============================================================
% بخش ۲: فونت و یونیکد (قبل از xepersian)
% ============================================================
\usepackage{fontspec}
\usepackage{xunicode}
\usepackage{xltxtra}

% ============================================================
% بخش ۳: ریاضی
% ============================================================
\usepackage{amsmath}
\usepackage{amssymb}
\usepackage{amsfonts}
\usepackage{mathtools}
\usepackage{unicode-math}          % ✅ اضافه شد برای \setmathfont

% ============================================================
% بخش ۴: رنگ‌ها
% ============================================================
\usepackage[
dvipsnames,
svgnames,
x11names,
table
]{xcolor}

% ---- رنگ‌های اصلی ----
\definecolor{MainPurple}{HTML}{6A0DAD}
\definecolor{LightPurple}{HTML}{E8D5F5}
\definecolor{DarkPurple}{HTML}{4A0078}
\definecolor{PurpleBG}{HTML}{F3E5F5}

\definecolor{MainBlue}{HTML}{1565C0}
\definecolor{LightBlue}{HTML}{BBDEFB}
\definecolor{DarkBlue}{HTML}{0D47A1}
\definecolor{BlueBG}{HTML}{E3F2FD}

\definecolor{MainGreen}{HTML}{2E7D32}
\definecolor{LightGreen}{HTML}{C8E6C9}
\definecolor{DarkGreen}{HTML}{1B5E20}
\definecolor{GreenBG}{HTML}{E8F5E9}

\definecolor{MainOrange}{HTML}{E65100}
\definecolor{LightOrange}{HTML}{FFE0B2}
\definecolor{DarkOrange}{HTML}{BF360C}
\definecolor{OrangeBG}{HTML}{FFF3E0}

\definecolor{MainRed}{HTML}{B71C1C}
\definecolor{LightRed}{HTML}{FFCDD2}
\definecolor{DarkRed}{HTML}{7F0000}
\definecolor{RedBG}{HTML}{FFEBEE}

\definecolor{MainYellow}{HTML}{F9A825}
\definecolor{LightYellow}{HTML}{FFF9C4}
\definecolor{DarkYellow}{HTML}{F57F17}
\definecolor{YellowBG}{HTML}{FFFDE7}

\definecolor{DarkGray}{HTML}{424242}
\definecolor{MediumGray}{HTML}{757575}
\definecolor{LightGray}{HTML}{E0E0E0}
\definecolor{VeryLightGray}{HTML}{F5F5F5}

\definecolor{LinkColor}{HTML}{1976D2}
\definecolor{CiteColor}{HTML}{6A1B9A}
\definecolor{URLColor}{HTML}{E65100}
\definecolor{CodeBG}{HTML}{263238}
\definecolor{CodeText}{HTML}{EEFFFF}
\definecolor{HighlightGold}{HTML}{FFD54F}
\definecolor{TableHeaderBG}{HTML}{E8EAF6}   % بنفش روشن برای هدر
\definecolor{TableAltRow}{HTML}{F8F8F8}     % خاکستری خیلی روشن برای

% ============================================================
% بخش ۵: جداول
% ============================================================
\usepackage{booktabs}
\usepackage{tabularx}
\usepackage{longtable}
\usepackage{ltablex}
\keepXColumns
\usepackage{multirow}
\usepackage{multicol}
\usepackage{array}
\usepackage{makecell}
\usepackage{colortbl}
\usepackage{threeparttable}
\usepackage{siunitx}

% انواع ستون سفارشی
\newcolumntype{L}[1]{>{\raggedleft\arraybackslash}p{#1}}
\newcolumntype{R}[1]{>{\raggedright\arraybackslash}p{#1}}
\newcolumntype{C}[1]{>{\centering\arraybackslash}p{#1}}
\newcolumntype{Y}{>{\centering\arraybackslash}X}

\setlength{\tabcolsep}{6pt}
\renewcommand{\arraystretch}{1.3}

% ============================================================
% بخش ۶: تصاویر
% ============================================================
\usepackage{graphicx}
\usepackage[export]{adjustbox}
\usepackage{float}
\usepackage[section]{placeins}     % ✅ فقط یک بار با آپشن
\usepackage{subcaption}
\usepackage{caption}
\usepackage{rotating}
\usepackage{pdflscape}
\usepackage{pdfpages}

\graphicspath{{figures/}{figures/diagrams/}{figures/charts/}}

\captionsetup{
	font={small},
	labelfont={bf, color=DarkGray},
	textfont={color=DarkGray},
	justification=centering,
	skip=8pt
}
\captionsetup[figure]{name=شکل}
\captionsetup[table]{name=جدول}

% ============================================================
% بخش ۷: TikZ — نسخهٔ سبک‌شده
% ============================================================
\usepackage{tikz}

\usetikzlibrary{
	arrows.meta,
	backgrounds,
	calc,
	chains,
	decorations.pathreplacing,
	decorations.markings,
	fit,
	intersections,
	matrix,
	patterns,
	positioning,
	quotes,
	shapes.geometric,
	shapes.multipart,
	shapes.symbols,
	trees,
	shadows
}
% ✅ استایل‌های TikZ که در فصول استفاده شده
\tikzset{
	% استایل‌های عمومی
	    % استایل برای نودهای فارسی
	persian/.style={
		execute at begin node={\beginR},
		execute at end node={\endR}
	},
	% استایل برای نودهای فارسی با align
	persian box/.style={
		persian,
		align=right,
		text width=#1
	},
	persian box/.default=10cm,	
	center/.style={
		rectangle,
		rounded corners=3pt,
		draw=MainPurple,
		fill=PurpleBG,
		minimum width=2cm,
		minimum height=1cm,
		align=center,
		font=\small
	},
	conn/.style={
		->,
		>=Stealth,
		thick,
		MainBlue
	},
	arr/.style={
		->,
		>=Stealth,
		thick,
		MainPurple
	},
	arrow/.style={
		->,
		>=Stealth,
		thick,
		MainBlue
	},
	box/.style={
		rectangle,
		rounded corners,
		draw=MainPurple,
		fill=PurpleBG,
		minimum width=2.5cm,
		minimum height=0.8cm,
		align=center,
		font=\small
	},
	process/.style={
		rectangle,
		rounded corners,
		draw=MainBlue,
		fill=BlueBG,
		minimum width=2cm,
		minimum height=1cm,
		align=center,
		font=\small
	},
	decision/.style={
		diamond,
		draw=MainOrange,
		fill=OrangeBG,
		minimum width=1.5cm,
		minimum height=1cm,
		align=center,
		font=\small,
		aspect=2
	},
	source/.style={
		rectangle,
		rounded corners,
		draw=MediumGray,
		fill=VeryLightGray,
		minimum width=2.5cm,
		minimum height=0.6cm,
		align=center,
		font=\footnotesize
	},
	phase/.style={
		rectangle,
		rounded corners=5pt,
		draw=MainPurple,
		fill=PurpleBG,
		minimum width=2cm,
		minimum height=1cm,
		align=center,
		font=\small
	}
}
\usepackage{pgfplots}
\pgfplotsset{compat=1.18}

\usepgfplotslibrary{
	colorbrewer,
	fillbetween,
	groupplots,
	statistics
}

\usepackage{pgf-pie}
\usepackage{forest}

% ============================================================
% بخش ۸: کادرها
% ============================================================
\usepackage[
most,
skins,
breakable
]{tcolorbox}

% ============================================================
% بخش ۹: آیکون‌ها — رفع تداخل
% ============================================================
\usepackage{fontawesome5}
\usepackage{pifont}

% ⚠️ قبل از marvosym، نمادهای تداخلی را ذخیره می‌کنیم
\usepackage{savesym}
\usepackage{wasysym}
\savesymbol{Cross}                 % ✅ ذخیره \Cross از wasysym
\savesymbol{Square}
\usepackage{marvosym}              % حالا marvosym بارگذاری می‌شود

% ============================================================
% بخش ۱۰: فهرست‌ها
% ============================================================
\usepackage{enumitem}
\setlist{noitemsep, topsep=4pt}
\setlist[itemize,1]{label=\textcolor{MainPurple}{■}}
\setlist[itemize,2]{label=\textcolor{MainBlue}{▪}}
\setlist[enumerate,1]{label=\textcolor{MainPurple}{\arabic*.}}

% ============================================================
% بخش ۱۱: عنوان‌بندی
% ============================================================
\usepackage{titlesec}
\usepackage{titletoc}
\usepackage{minitoc}
% ============================================================
% بخش ۱۲: سرصفحه — فقط fancyhdr
% ============================================================
\usepackage{fancyhdr}
\usepackage{lastpage}
% ⚠️ scrlayer-scrpage حذف شد — با fancyhdr تداخل دارد

% ============================================================
% بخش ۱۳: لینک‌ها و ارجاع
% ============================================================
\PassOptionsToPackage{hyphens}{url}  % ✅ قبل از hyperref

\usepackage[
colorlinks=true,
linkcolor=MainPurple,
citecolor=CiteColor,
urlcolor=URLColor,
bookmarks=true,
bookmarksnumbered=true,
pdfencoding=unicode,
breaklinks=true,
pdfauthor={مهدی سالم},
pdftitle={نظارت بین‌المللی بر گذار دموکراتیک ایران}
]{hyperref}

\usepackage[nameinlink, capitalise]{cleveref}
\usepackage{bookmark}

% ============================================================
% بخش ۱۴: فهرست منابع
% ============================================================
\usepackage[
backend=biber,
style=apa,
sorting=nyt,
maxcitenames=3,
maxbibnames=99
]{biblatex}

\addbibresource{references.bib}

% ============================================================
% بخش ۱۵: واژه‌نامه و نمایه
% ============================================================
\usepackage[
toc,
section=chapter,
style=long3col,
nonumberlist
]{glossaries}
\usepackage{glossaries-extra}
\usepackage{makeidx}
\makeindex
\makeglossaries

% ============================================================
% بخش ۱۶: پانویس
% ============================================================
\usepackage[perpage, bottom, hang]{footmisc}
\setlength{\footnotesep}{8pt}

% ============================================================
% بخش ۱۷: فاصله‌گذاری
% ============================================================
\usepackage{setspace}
\onehalfspacing

\usepackage{parskip}
\setlength{\parindent}{0pt}
\setlength{\parskip}{6pt plus 2pt minus 1pt}

\usepackage{ragged2e}

% ============================================================
% بخش ۱۸: ابزارهای متنوع
% ============================================================
\usepackage{xspace}
\usepackage{xparse}
\usepackage{etoolbox}
\usepackage{calc}
\usepackage{comment}
\usepackage{soul}

% ============================================================
% بخش ۱۹: نقل‌قول و متن
% ============================================================
\usepackage{epigraph}
\usepackage{csquotes}

% ============================================================
% بخش ۲۰: ضمائم
% ============================================================
\usepackage[toc, page]{appendix}

% ============================================================
% بخش ۲۱: تنظیمات سرریز
% ============================================================
\sloppy
\tolerance=1000
\emergencystretch=3em
\hfuzz=2pt
\vfuzz=2pt

% ⚠️ overfullrule را قبل از نسخه نهایی حذف کنید
% \overfullrule=5pt

% ============================================================
% پایان پریامبل — xepersian در main.tex بارگذاری شود
% ============================================================
% ============================================================
% بارگذاری xepersian — آخرین بسته
% ============================================================
\usepackage{xepersian}
% مرحله ۳: فونت‌ها (بعد از xepersian)
% ═══════════════════════════════════════════════════════
% ============================================================
% تنظیمات فونت — پس از xepersian
% ============================================================
% ╔══════════════════════════════════════════════════════════════════╗
% ║  تنظیمات فونت — پس از xepersian                                 ║
% ╚══════════════════════════════════════════════════════════════════╝

% ═══════════════════════════════════════════════════════
% فونت فارسی اصلی
% ═══════════════════════════════════════════════════════

% گزینه ۱: فونت سیستمی (توصیه‌شده)
\settextfont{XB Zar}

% گزینه ۲: فونت از پوشه
% \settextfont[
%     Scale=1.0,
%     Extension=.ttf,
%     Path=fonts/,
%     BoldFont=XB Zar Bold,
%     ItalicFont=XB Zar Italic
% ]{XB Zar}

% گزینه ۳: فونت‌های جایگزین
% \settextfont{Vazir}
% \settextfont{B Nazanin}
% \settextfont{Iranian Sans}
% \settextfont{Sahel}

% ═══════════════════════════════════════════════════════
% فونت لاتین
% ═══════════════════════════════════════════════════════

% گزینه ۱: فونت سیستمی
\setlatintextfont{Times New Roman}

% گزینه ۲: فونت‌های جایگزین
% \setlatintextfont{Georgia}
% \setlatintextfont{Palatino Linotype}
% \setlatintextfont{Linux Libertine O}

% ═══════════════════════════════════════════════════════
% فونت اعداد
% ═══════════════════════════════════════════════════════
\setdigitfont{XB Zar}

% ═══════════════════════════════════════════════════════
% فونت عناوین (اختیاری)
% ═══════════════════════════════════════════════════════
% \defpersianfont\titlefont[Scale=1.2]{XB Titre}

% ═══════════════════════════════════════════════════════
% فونت کد (Monospace)
% ═══════════════════════════════════════════════════════
\setmonofont[Scale=0.85]{Consolas}
% جایگزین: Courier New, Fira Code, DejaVu Sans Mono|
% ✅ تعریف فونت آمار و ارقام بزرگ
\newfontfamily\statisticfont[Scale=2.5]{XB Zar}

% جایگزین اگر می‌خواهید فونت متفاوت:
% ═══════════════════════════════════════════════════════
% مرحله ۴: کادرها (بعد از xepersian)
% ═══════════════════════════════════════════════════════
% ╔══════════════════════════════════════════════════════════════════╗
% ║  تعریف کادرها و باکس‌های سند — نسخه ۱.۱                         ║
% ╚══════════════════════════════════════════════════════════════════╝

% ============================================================
% کادر ۱: خلاصه‌ی اجرایی فصل
% ============================================================
\newtcolorbox{executivesummary}{
	enhanced,
	breakable,
	colback=PurpleBG,
	colframe=MainPurple,
	coltitle=white,
	fonttitle=\large\bfseries,
	title={\faFile*[regular]\hspace{8pt}خلاصه‌ی اجرایی این فصل},
	arc=3mm,
	boxrule=1.2pt,
	left=8mm,
	right=8mm,
	top=6mm,
	bottom=6mm,
	toptitle=4mm,
	bottomtitle=4mm,
	before skip=15pt,
	after skip=20pt,
	fuzzy shadow={2mm}{-2mm}{0mm}{0.4mm}{black!30},
	watermark text={\faBookOpen},
	watermark opacity=0.05,
	watermark zoom=1.5
}

% ============================================================
% کادر ۲: نکته‌ی کلیدی
% ============================================================
\newtcolorbox{keypoint}[1][]{
	enhanced,
	breakable,
	colback=MainPurple!5,
	colframe=MainPurple,
	coltitle=white,
	fonttitle=\bfseries,
	title={\faKey\hspace{8pt}نکته‌ی کلیدی},
	arc=2mm,
	boxrule=0.8pt,
	left=6mm,
	right=6mm,
	before skip=10pt,
	after skip=10pt,
	#1
}

% ============================================================
% کادر ۳: درس آموخته
% ============================================================
\newtcolorbox{lessonlearned}[1][]{
	enhanced,
	breakable,
	colback=BlueBG,
	colframe=MainBlue,
	coltitle=white,
	fonttitle=\bfseries,
	title={\faGraduationCap\hspace{8pt}درس آموخته},
	arc=2mm,
	boxrule=0.8pt,
	left=6mm,
	right=6mm,
	borderline west={3pt}{0pt}{MainBlue},
	before skip=10pt,
	after skip=10pt,
	#1
}

% ============================================================
% کادر ۴: هشدار
% ============================================================
\newtcolorbox{warningbox}[1][]{
	enhanced,
	breakable,
	colback=RedBG,
	colframe=MainRed,
	coltitle=white,
	fonttitle=\bfseries,
	title={\faExclamationTriangle\hspace{8pt}هشدار},
	arc=2mm,
	boxrule=0.8pt,
	left=6mm,
	right=6mm,
	borderline west={3pt}{0pt}{MainRed},
	overlay unbroken and first={
		\node[
		anchor=north east,
		font=\fontsize{40}{40}\selectfont,
		text=MainRed!10
		] at ([xshift=-5mm, yshift=-5mm]frame.north east) {\faExclamationTriangle};
	},
	before skip=10pt,
	after skip=10pt,
	#1
}

% ============================================================
% کادر ۵: توصیه‌ی اجرایی
% ============================================================
\newtcolorbox{recommendation}[1][]{
	enhanced,
	breakable,
	colback=GreenBG,
	colframe=MainGreen,
	coltitle=white,
	fonttitle=\bfseries,
	title={\faCheckCircle\hspace{8pt}توصیه‌ی اجرایی},
	arc=2mm,
	boxrule=0.8pt,
	left=6mm,
	right=6mm,
	borderline west={3pt}{0pt}{MainGreen},
	before skip=10pt,
	after skip=10pt,
	#1
}

% ============================================================
% کادر ۶: سناریو
% ============================================================
\newtcolorbox{scenariobox}[2][]{
	enhanced,
	breakable,
	colback=OrangeBG,
	colframe=MainOrange,
	coltitle=white,
	fonttitle=\bfseries,
	title={\faRoute\hspace{8pt}سناریو: #2},
	arc=2mm,
	boxrule=0.8pt,
	left=6mm,
	right=6mm,
	before skip=10pt,
	after skip=10pt,
	#1
}

% ============================================================
% کادر ۷: تعریف
% ============================================================
\newtcolorbox{definitionbox}[2][]{
	enhanced,
	breakable,
	colback=BlueBG,
	colframe=MainBlue!70,
	coltitle=MainBlue!90!black,
	fonttitle=\bfseries,
	title={\faBookOpen\hspace{8pt}تعریف: #2},
	arc=2mm,
	boxrule=0.6pt,
	left=6mm,
	right=6mm,
	before skip=10pt,
	after skip=10pt,
	#1
}

% ============================================================
% کادر ۸: مطالعه‌ی موردی
% ============================================================
\newtcolorbox{casestudy}[2][]{
	enhanced,
	breakable,
	colback=VeryLightGray,
	colframe=MediumGray,
	coltitle=white,
	colbacktitle=DarkGray,
	fonttitle=\bfseries,
	title={\faGlobe\hspace{8pt}مطالعه‌ی موردی: #2},
	arc=2mm,
	boxrule=0.6pt,
	left=6mm,
	right=6mm,
	before skip=10pt,
	after skip=10pt,
	#1
}

% ============================================================
% کادر ۹: نقل‌قول برجسته
% ============================================================
\newtcolorbox{pullquote}[1][]{
	enhanced,
	breakable,
	colback=VeryLightGray!50,
	colframe=LightGray,
	boxrule=0pt,
	borderline west={4pt}{0pt}{MainPurple},
	left=12mm,
	right=8mm,
	top=4mm,
	bottom=4mm,
	before skip=12pt,
	after skip=12pt,
	fontupper=\large\color{DarkGray},
	% ✅ اصلاح: \itshape ممکن است با فارسی مشکل داشته باشد
	#1
}

% ============================================================
% کادر ۱۰: نکته‌ی عملیاتی
% ============================================================
\newtcolorbox{operationalnote}[1][]{
	enhanced,
	breakable,
	colback=YellowBG,
	colframe=MainYellow,
	coltitle=DarkYellow!80!black,
	fonttitle=\bfseries,
	title={\faCogs\hspace{8pt}نکته‌ی عملیاتی},
	arc=2mm,
	boxrule=0.8pt,
	left=6mm,
	right=6mm,
	before skip=10pt,
	after skip=10pt,
	#1
}

% ============================================================
% کادر ۱۱: مقایسه
% ============================================================
\newtcolorbox{comparisonbox}[2][]{
	enhanced,
	breakable,
	colback=GreenBG,
	colframe=MainGreen!70,
	coltitle=DarkGreen,
	fonttitle=\bfseries,
	title={\faBalanceScale\hspace{8pt}مقایسه: #2},
	arc=2mm,
	boxrule=0.6pt,
	left=6mm,
	right=6mm,
	before skip=10pt,
	after skip=10pt,
	#1
}

% ============================================================
% کادر ۱۲: پرسش تأملی
% ============================================================
\newtcolorbox{reflectionbox}[1][]{
	enhanced,
	breakable,
	colback=PurpleBG,
	colframe=MainPurple!50,
	boxrule=0.4pt,
	borderline south={2pt}{0pt}{MainPurple},
	left=10mm,
	right=6mm,
	top=4mm,
	bottom=4mm,
	before skip=10pt,
	after skip=10pt,
	overlay unbroken and first={
		\node[
		anchor=north west,
		font=\Large,
		text=MainPurple!40
		] at ([xshift=2mm, yshift=-2mm]frame.north west) {\faQuestion};
	},
	#1
}

% ============================================================
% کادر ۱۳: آمار و ارقام
% ============================================================
\newtcolorbox{statsbox}[1][]{
	enhanced,
	colback=white,
	colframe=MainBlue,
	boxrule=1.5pt,
	arc=0pt,
	left=14mm,
	right=6mm,
	top=6mm,
	bottom=6mm,
	before skip=10pt,
	after skip=10pt,
	overlay unbroken and first={
		\fill[MainBlue] 
		(frame.north west) rectangle 
		([xshift=5mm]frame.south west);
		\node[
		anchor=center,
		font=\Large,
		text=white,
		rotate=90
		] at ([xshift=2.5mm]frame.west) {\faChartBar};
	},
	#1
}

% ============================================================
% کادر ۱۴: خلاصه‌ی فصل
% ============================================================
\newtcolorbox{chaptersummary}{
	enhanced,
	breakable,
	colback=VeryLightGray,
	colframe=DarkGray,
	coltitle=white,
	colbacktitle=DarkGray,
	fonttitle=\large\bfseries,
	title={\faList\hspace{8pt}خلاصه‌ی فصل},
	% ✅ اصلاح: faListAlt → faList
	arc=3mm,
	boxrule=0.8pt,
	left=8mm,
	right=8mm,
	top=6mm,
	bottom=6mm,
	before skip=20pt,
	after skip=15pt,
    fuzzy shadow={1mm}{-1mm}{0mm}{0.3mm}{black!20}
}

% ============================================================
% کادر ۱۵: ارجاع به فصل دیگر
% ============================================================
\newtcolorbox{crossref}[1][]{
	enhanced,
	colback=LightPurple!30,
	colframe=MainPurple!30,
	boxrule=0.3pt,
	arc=1mm,
	left=10mm,
	right=4mm,
	top=2mm,
	bottom=2mm,
	before skip=6pt,
	after skip=6pt,
	fontupper=\small,
	overlay unbroken={
		\node[
		anchor=west,
		font=\small,
		text=MainPurple
		] at ([xshift=3mm]frame.west) {\faArrowRight};
		% ✅ اصلاح برای RTL: فلش به سمت راست
	},
	#1
}

% ============================================================
% کادرهای اضافی برای نیازهای خاص
% ============================================================

% کادر ۱۶: نقل‌قول رسمی (با منبع)
\newtcolorbox{formalquote}[2][]{
	enhanced,
	breakable,
	colback=VeryLightGray,
	colframe=VeryLightGray,
	boxrule=0pt,
	borderline west={3pt}{0pt}{MainPurple},
	left=12mm,
	right=8mm,
	top=6mm,
	bottom=6mm,
	before skip=12pt,
	after skip=12pt,
	overlay unbroken and last={
		\node[
		anchor=south east,
		font=\small\color{MediumGray}
		] at ([xshift=-4mm, yshift=4mm]frame.south east) {— #2};
	},
	#1
}

% کادر ۱۷: چک‌لیست
\newtcolorbox{checklistbox}[2][]{
	enhanced,
	breakable,
	colback=GreenBG!50,
	colframe=MainGreen!50,
	coltitle=DarkGreen,
	fonttitle=\bfseries,
	title={\faClipboardCheck\hspace{8pt}#2},
	arc=2mm,
	boxrule=0.5pt,
	left=6mm,
	right=6mm,
	before skip=10pt,
	after skip=10pt,
	#1
}

% کادر ۱۸: زمان‌بندی
\newtcolorbox{timelinebox}[2][]{
	enhanced,
	breakable,
	colback=BlueBG!50,
	colframe=MainBlue!50,
	coltitle=DarkBlue,
	fonttitle=\bfseries,
	title={\faClock\hspace{8pt}#2},
	arc=2mm,
	boxrule=0.5pt,
	left=6mm,
	right=6mm,
	before skip=10pt,
	after skip=10pt,
	#1
}
% ═══════════════════════════════════════════════════════
% مرحله ۵: دستورات (بعد از xepersian — نیاز به \lr)
% ═══════════════════════════════════════════════════════
% ╔══════════════════════════════════════════════════════════════════╗
% ║  دستورات سفارشی سند — نسخه ۱.۱                                  ║
% ║  ⚠️ این فایل باید پس از xepersian بارگذاری شود                  ║
% ╚══════════════════════════════════════════════════════════════════╝

% ============================================================
% دستورات متنی
% ============================================================

% --- دستورات متنی ---
\newcommand{\bilingual}[2]{#1 (\lr{#2})}
\newcommand{\termfn}[2]{#1\LTRfootnote{#2}}
\newcommand{\abbr}[3]{#1 (\lr{#2}; \lr{#3})}
\newcommand{\org}[2]{\textbf{#1} (\lr{\textit{#2}})}
\newcommand{\person}[2]{#1 (\lr{#2})}

% --- تأکید رنگی ---
\newcommand{\emphpurple}[1]{\textbf{\textcolor{MainPurple}{#1}}}
\newcommand{\emphblue}[1]{\textbf{\textcolor{MainBlue}{#1}}}
\newcommand{\emphgreen}[1]{\textbf{\textcolor{MainGreen}{#1}}}
\newcommand{\emphorange}[1]{\textbf{\textcolor{MainOrange}{#1}}}
\newcommand{\emphred}[1]{\textbf{\textcolor{MainRed}{#1}}}
\newcommand{\emphyellow}[1]{\textbf{\textcolor{DarkYellow}{#1}}}

% \newcommand{\statisticfont}{\fontsize{24}{28}\selectfont\bfseries}
% --- هایلایت ---
\newcommand{\highlight}[1]{\colorbox{HighlightGold!50}{#1}}

% ============================================================
% دستورات نمادی و امتیازدهی — اصلاح‌شده
% ============================================================

% --- ستاره‌های امتیاز (۱ تا ۵) ---
% ✅ بازنویسی کامل بدون \foreach و \numexpr مشکل‌دار
\newcommand{\starrating}[1]{%
	\ifcase#1\relax
	% 0 ستاره
	\textcolor{LightGray}{☆☆☆☆☆}%
	\or
	% 1 ستاره
	\textcolor{MainYellow}{★}\textcolor{LightGray}{☆☆☆☆}%
	\or
	% 2 ستاره
	\textcolor{MainYellow}{★★}\textcolor{LightGray}{☆☆☆}%
	\or
	% 3 ستاره
	\textcolor{MainYellow}{★★★}\textcolor{LightGray}{☆☆}%
	\or
	% 4 ستاره
	\textcolor{MainYellow}{★★★★}\textcolor{LightGray}{☆}%
	\or
	% 5 ستاره
	\textcolor{MainYellow}{★★★★★}%
	\fi
}

% --- دایره‌های امتیاز (۱ تا ۵) ---
\newcommand{\circlerating}[1]{%
	\ifcase#1\relax
	\textcolor{LightGray}{○○○○○}%
	\or
	\textcolor{MainBlue}{●}\textcolor{LightGray}{○○○○}%
	\or
	\textcolor{MainBlue}{●●}\textcolor{LightGray}{○○○}%
	\or
	\textcolor{MainBlue}{●●●}\textcolor{LightGray}{○○}%
	\or
	\textcolor{MainBlue}{●●●●}\textcolor{LightGray}{○}%
	\or
	\textcolor{MainBlue}{●●●●●}%
	\fi
}
% ✅ دستور rating که در فصول استفاده شده
\newcommand{\rating}[1]{%
	\ifcase#1\relax
	\textcolor{LightGray}{○○○○○}%
	\or
	\textcolor{MainPurple}{●}\textcolor{LightGray}{○○○○}%
	\or
	\textcolor{MainPurple}{●●}\textcolor{LightGray}{○○○}%
	\or
	\textcolor{MainPurple}{●●●}\textcolor{LightGray}{○○}%
	\or
	\textcolor{MainPurple}{●●●●}\textcolor{LightGray}{○}%
	\or
	\textcolor{MainPurple}{●●●●●}%
	\fi
}

% --- نشانگرهای وضعیت ---
\newcommand{\statusok}{\textcolor{MainGreen}{\faCheckCircle}}
\newcommand{\statuswarn}{\textcolor{MainOrange}{\faExclamationCircle}}
\newcommand{\statusbad}{\textcolor{MainRed}{\faTimesCircle}}
\newcommand{\statuspending}{\textcolor{MediumGray}{\faClock}}
\newcommand{\statusquestion}{\textcolor{MainBlue}{\faQuestionCircle}}

% --- تیک و ضربدر ---
\newcommand{\cmark}{\textcolor{MainGreen}{\ding{51}}}
\newcommand{\xmark}{\textcolor{MainRed}{\ding{55}}}
\newcommand{\qmark}{\textcolor{MainOrange}{\ding{73}}}

% --- سطح ریسک ---
\newcommand{\riskhigh}{\textcolor{MainRed}{\faArrowUp\ بالا}}
\newcommand{\riskmedium}{\textcolor{MainOrange}{\faMinus\ متوسط}}
\newcommand{\risklow}{\textcolor{MainGreen}{\faArrowDown\ پایین}}
% ✅ اصلاح: \faArrowLeft → \faMinus برای «متوسط»

% --- فلش‌های رنگی ---
\newcommand{\arrowup}{\textcolor{MainGreen}{\faArrowUp}}
\newcommand{\arrowdown}{\textcolor{MainRed}{\faArrowDown}}
\newcommand{\arrowflat}{\textcolor{MainOrange}{\faMinus}}

% --- نمادهای چراغ راهنما ---
\newcommand{\lightgreen}{\textcolor{MainGreen}{\faCircle}}
\newcommand{\lightyellow}{\textcolor{MainYellow}{\faCircle}}
\newcommand{\lightred}{\textcolor{MainRed}{\faCircle}}
\newcommand{\lightgray}{\textcolor{LightGray}{\faCircle}}

% ============================================================
% دستورات ساختاری
% ============================================================

% --- صفحه‌ی آغازین فصل ---
\newcommand{\chapteropening}[5]{%
	% #1 = شماره فصل (فارسی یا عربی)
	% #2 = عنوان فصل
	% #3 = رنگ فصل
	% #4 = نقل‌قول
	% #5 = منبع نقل‌قول
	\clearpage
	\thispagestyle{empty}
	\begin{tikzpicture}[remember picture, overlay]
		% پس‌زمینه
		\fill[#3!8] 
		(current page.south west) rectangle 
		(current page.north east);
		
		% نوار بالایی
		\fill[#3] 
		([yshift=-4cm]current page.north west) rectangle 
		(current page.north east);
		
		% شماره فصل بزرگ
		\node[
		anchor=north east,
		font=\fontsize{100}{100}\selectfont\bfseries,
		text=#3!20
		] at ([xshift=-1cm, yshift=-0.3cm]current page.north east) 
		{#1};
		
		% «فصل»
		\node[
		anchor=north west,
		font=\Large,
		text=white!90
		] at ([xshift=2cm, yshift=-1cm]current page.north west) 
		{فصل};
		
		% عنوان فصل
		\node[
		anchor=north west,
		font=\fontsize{24}{30}\selectfont\bfseries,
		text=white,
		text width=13cm,
		align=right
		] at ([xshift=2cm, yshift=-2cm]current page.north west) 
		{#2};
		
		% خط تزئینی
		\draw[white, line width=1.5pt] 
		([yshift=-4.2cm, xshift=2cm]current page.north west) -- 
		([yshift=-4.2cm, xshift=7cm]current page.north west);
		
		% نقل‌قول
		\node[
		anchor=center,
		text width=12cm,
		align=center,
		font=\large,
		text=#3!70!black
		] at ([yshift=0.5cm]current page.center) 
		{«#4»};
		
		% منبع نقل‌قول
		\node[
		anchor=center,
		font=\normalsize,
		text=#3!50!black
		] at ([yshift=-1.5cm]current page.center) 
		{— #5};
		
		% نوار پایینی
		\fill[#3!50] 
		(current page.south west) rectangle 
		([yshift=0.4cm]current page.south east);
		
	\end{tikzpicture}
	\clearpage
}

% --- نسخه ساده‌تر برای تست ---
\newcommand{\chapteropeningsimple}[2]{%
	% #1 = شماره فصل
	% #2 = عنوان فصل
	\clearpage
	\thispagestyle{empty}
	\vspace*{3cm}
	\begin{center}
		{\fontsize{60}{60}\selectfont\textcolor{MainPurple!30}{#1}}\\[1cm]
		{\huge\bfseries\textcolor{MainPurple}{#2}}
	\end{center}
	\vfill
	\clearpage
}

% --- خلاصه اجرایی سریع ---
\newcommand{\inlinesummary}[1]{%
	\begin{tcolorbox}[
		enhanced,
		colback=PurpleBG!50,
		colframe=MainPurple!30,
		boxrule=0.3pt,
		arc=1mm,
		left=4mm,
		right=4mm,
		top=2mm,
		bottom=2mm
		]
		\small\textcolor{DarkGray}{#1}
	\end{tcolorbox}
}

% --- جداکننده‌ی بخش‌ها ---
\newcommand{\sectiondivider}{%
	\begin{center}
		\vspace{6pt}
		\textcolor{LightGray}{\rule{0.35\textwidth}{0.4pt}}%
		\hspace{6pt}%
		\textcolor{MainPurple}{◆}%
		\hspace{6pt}%
		\textcolor{LightGray}{\rule{0.35\textwidth}{0.4pt}}
		\vspace{6pt}
	\end{center}
}

% --- پایان فصل ---
\newcommand{\chapterend}{%
	\vfill
	\begin{center}
		\textcolor{MainPurple}{\rule{2.5cm}{0.8pt}}\\[4pt]
		\textcolor{MainPurple}{\faBookOpen}\\[4pt]
		\textcolor{MainPurple}{\rule{2.5cm}{0.8pt}}
	\end{center}
	\clearpage
}

% ============================================================
% دستورات جدولی
% ============================================================

% --- فونت‌های جدول ---
\newcommand{\tablefontsize}{\small}
\newcommand{\bigtablefontsize}{\footnotesize}
\newcommand{\hugetablefontsize}{\scriptsize}

% --- چرخش عنوان ستون ---
\newcommand{\rot}[1]{\rotatebox{60}{\parbox{2.5cm}{\raggedleft #1}}}
\newcommand{\rotsmall}[1]{\rotatebox{60}{\parbox{2cm}{\raggedleft\footnotesize #1}}}

% --- سطرهای رنگی ---
\newcommand{\headerrow}{\rowcolor{TableHeaderBG}}
\newcommand{\altrow}{\rowcolor{TableAltRow}}

% --- سلول‌های رنگی ---
\newcommand{\cellgreen}[1]{\cellcolor{GreenBG}#1}
\newcommand{\cellred}[1]{\cellcolor{RedBG}#1}
\newcommand{\cellorange}[1]{\cellcolor{OrangeBG}#1}
\newcommand{\cellyellow}[1]{\cellcolor{YellowBG}#1}
\newcommand{\cellblue}[1]{\cellcolor{BlueBG}#1}
\newcommand{\cellpurple}[1]{\cellcolor{PurpleBG}#1}

% ============================================================
% دستورات ارجاعی
% ============================================================

% --- دستورات ارجاعی ---
\newcommand{\seeChapter}[1]{{\small\textcolor{MainPurple}{\faArrowLeft\ رجوع شود به فصل~\ref{#1}}}}
\newcommand{\seeAppendix}[1]{{\small\textcolor{MainPurple}{\faArrowLeft\ رجوع شود به پیوست~\ref{#1}}}}
\newcommand{\seeTable}[1]{{\small\textcolor{MainBlue}{← جدول~\ref{#1}}}}
\newcommand{\seeFigure}[1]{{\small\textcolor{MainBlue}{← شکل~\ref{#1}}}}
\newcommand{\seePage}[1]{{\small\textcolor{MediumGray}{(صفحه~\pageref{#1})}}}

% ============================================================
% دستورات عددی و تاریخی
% ============================================================

% --- عدد با جداکننده هزارگان ---
\newcommand{\numfa}[1]{%
	\persiandigits{#1}%
}

% --- درصد ---
\newcommand{\pct}[1]%
}

% --- سال میلادی ---
\newcommand{\yearce}[1]{%
	\lr{#1}%
}

% --- بازه زمانی ---
\newcommand{\daterange}[2]{%
	\lr{#1}–\lr{#2}%
}

% ============================================================
% ✅ minitoc جایگزین (اگر بسته minitoc بارگذاری نشده)
% ============================================================
% \newcommand{\minitoc}{}  % دستور خالی برای جلوگیری از خطا
%\newcommand{\dominitoc}{} % دستور خالی برای جلوگیری از خطا

% یا اگر می‌خواهید متن جایگزین نشان دهد:
% \newcommand{\minitoc}{\textit{فهرست مطالب این فصل در اینجا قرار می‌گیرد.}\par\vspace{1em}}

% ═══════════════════════════════════════════════════════
% مرحله ۶: صفحه‌آرایی (بعد از xepersian)
% ═══════════════════════════════════════════════════════
% ╔══════════════════════════════════════════════════════════════════╗
% ║  تنظیمات صفحه‌آرایی — نسخه ۱.۱                                  ║
% ║  ⚠️ این فایل باید پس از xepersian بارگذاری شود                  ║
% ╚══════════════════════════════════════════════════════════════════╝

% ============================================================
% سرصفحه و پاصفحه — تنظیم برای RTL
% ============================================================

\pagestyle{fancy}
\fancyhf{}

% ✅ تنظیمات RTL: در فارسی، صفحات فرد سمت چپ و زوج سمت راست هستند

% صفحات فرد (چپ در RTL)
\fancyhead[LO]{%
	\textcolor{MediumGray}{\small\rightmark}%
}
\fancyhead[RO]{%
	\textcolor{MainPurple}{\small\thepage}%
}

% صفحات زوج (راست در RTL)
\fancyhead[RE]{%
	\textcolor{MediumGray}{\small\leftmark}%
}
\fancyhead[LE]{%
	\textcolor{MainPurple}{\small\thepage}%
}

% خط سرصفحه
\renewcommand{\headrulewidth}{0.4pt}
\renewcommand{\headrule}{%
	{\color{MainPurple!30}\hrule width\headwidth height\headrulewidth}%
}

% پاصفحه
\fancyfoot[C]{%
	\vspace{2pt}%
	{\textcolor{LightGray}{\rule{3cm}{0.3pt}}}\\[-3pt]%
	{\textcolor{MediumGray}{\footnotesize نظارت بین‌المللی بر گذار دموکراتیک ایران}}%
}

% صفحات آغاز فصل
\fancypagestyle{plain}{%
	\fancyhf{}%
	\fancyfoot[C]{\textcolor{MainPurple}{\thepage}}%
	\renewcommand{\headrulewidth}{0pt}%
}

% صفحات خالی
\fancypagestyle{empty}{%
	\fancyhf{}%
	\renewcommand{\headrulewidth}{0pt}%
	\renewcommand{\footrulewidth}{0pt}%
}

% ============================================================
% تنظیم mark برای فصل و بخش
% ============================================================
\renewcommand{\chaptermark}[1]{%
	\markboth{فصل \thechapter: #1}{}%
}
\renewcommand{\sectionmark}[1]{%
	\markright{#1}%
}

% ============================================================
% عنوان‌بندی فصول — سازگار با xepersian
% ============================================================

% --- فصل ---
\titleformat{\chapter}[display]
{\normalfont\huge\bfseries}
{\textcolor{MainPurple!30}{\fontsize{60}{60}\selectfont\thechapter}}
{20pt}
{\color{MainPurple}\Huge}
[\vspace{-10pt}\textcolor{MainPurple!20}{\rule{\textwidth}{2pt}}]

\titlespacing*{\chapter}
{0pt}      % چپ
{20pt}     % قبل
{30pt}     % بعد

% --- بخش ---
\titleformat{\section}
{\normalfont\Large\bfseries\color{MainBlue}}
{\thesection}
{1em}
{}
[\vspace{-6pt}\textcolor{MainBlue!30}{\rule{\textwidth}{0.5pt}}]

\titlespacing*{\section}
{0pt}{20pt}{12pt}

% --- زیربخش ---
\titleformat{\subsection}
{\normalfont\large\bfseries\color{MainGreen!80!black}}
{\thesubsection}
{1em}
{}

\titlespacing*{\subsection}
{0pt}{16pt}{8pt}

% --- زیرزیربخش ---
\titleformat{\subsubsection}
{\normalfont\normalsize\bfseries\color{MainOrange!80!black}}
{\thesubsubsection}
{1em}
{}

\titlespacing*{\subsubsection}
{0pt}{12pt}{6pt}

% --- پاراگراف ---
\titleformat{\paragraph}[runin]
{\normalfont\normalsize\bfseries\color{DarkGray}}
{\theparagraph}
{1em}
{}
[.]

\titlespacing*{\paragraph}
{0pt}{10pt}{8pt}

% ============================================================
% فهرست مطالب
% ============================================================

\setcounter{secnumdepth}{3}
\setcounter{tocdepth}{2}

% ============================================================
% Epigraph
% ============================================================
\setlength{\epigraphwidth}{0.65\textwidth}
\setlength{\epigraphrule}{0pt}
\renewcommand{\epigraphflush}{center}
\renewcommand{\sourceflush}{center}

% ============================================================
% پاورقی — شماره‌گذاری در هر صفحه
% تنظیمات شکل و جدول
% ============================================================

% شماره‌گذاری پیوسته (نه بر اساس فصل)
% \counterwithout{figure}{chapter}
% \counterwithout{table}{chapter}

% یا شماره‌گذاری بر اساس فصل (پیش‌فرض)
% اگر می‌خواهید ۱.۱، ۱.۲ و... باشد، همین‌طور بگذارید

% ============================================================
% فاصله‌ی شناورها
% ============================================================
\setlength{\textfloatsep}{20pt plus 4pt minus 4pt}
\setlength{\floatsep}{16pt plus 4pt minus 4pt}
\setlength{\intextsep}{16pt plus 4pt minus 4pt}

% ============================================================
% تنظیمات خاص برای PDF
% ============================================================

% عنوان سند در PDF viewer
\hypersetup{
	pdftitle={نظارت بین‌المللی بر گذار دموکراتیک ایران},
	pdfauthor={مهدی سالم},
	pdfsubject={گذار دموکراتیک},
	pdfkeywords={ایران، دموکراسی، گذار، نظارت}
}
% ============================================================
% عنوان‌های فارسی
% ============================================================
\renewcommand{\contentsname}{فهرست مطالب}
\renewcommand{\listfigurename}{فهرست شکل‌ها و نمودارها}
\renewcommand{\listtablename}{فهرست جداول}
\renewcommand{\bibname}{فهرست منابع و مراجع}
\renewcommand{\appendixname}{پیوست}
\renewcommand{\indexname}{نمایه}
\renewcommand{\figurename}{شکل}
\renewcommand{\tablename}{جدول}
\renewcommand{\chaptername}{فصل}
\renewcommand{\partname}{بخش}

% ============================================================
% واژه‌نامه
% ============================================================
% ╔══════════════════════════════════════════════════════════════════╗
% ║  واژه‌نامه — تعاریف اصلی                                       ║
% ╚══════════════════════════════════════════════════════════════════╝

% ---- مفاهیم نظری گذار ----

\newglossaryentry{democratic-transition}{
    name={گذار دموکراتیک},
    description={\lr{Democratic Transition} — 
    فرایند تغییر نظام سیاسی از اقتدارگرایی به دموکراسی، 
    شامل مراحل آزادسازی، دموکراتیزاسیون و تحکیم},
    sort={گذار}
}

\newglossaryentry{consolidation}{
    name={تحکیم دموکراتیک},
    description={\lr{Democratic Consolidation} — 
    مرحله‌ای که در آن دموکراسی «تنها بازی ممکن» 
    (\lr{the only game in town}) می‌شود و بازگشت 
    به اقتدارگرایی بعید است},
    sort={تحکیم}
}

\newglossaryentry{backsliding}{
    name={بازگشت اقتدارگرایانه},
    description={\lr{Authoritarian Backsliding/Reversal} — 
    فرایند تضعیف تدریجی یا ناگهانی نهادها و ارزش‌های 
    دموکراتیک و بازگشت به حکمرانی اقتدارگرایانه},
    sort={بازگشت}
}

\newglossaryentry{transitional-justice}{
    name={عدالت انتقالی},
    description={\lr{Transitional Justice} — 
    مجموعه‌ی اقدامات قضایی و غیرقضایی برای مواجهه 
    با میراث نقض حقوق بشر شامل: محاکمه، 
    کمیسیون حقیقت، غرامت و اصلاح نهادی},
    sort={عدالت}
}

\newglossaryentry{ssr}{
    name={اصلاح بخش امنیتی},
    description={\lr{Security Sector Reform (SSR)} — 
    بازسازی نهادهای نظامی، انتظامی و اطلاعاتی 
    برای تبعیت از حکمرانی دموکراتیک و 
    نظارت غیرنظامی},
    sort={اصلاح}
}

\newglossaryentry{ddr}{
    name={خلع سلاح، بسیج‌زدایی و بازادغام},
    description={\lr{Disarmament, Demobilization \& 
    Reintegration (DDR)} — فرایند سه‌مرحله‌ای 
    جمع‌آوری سلاح، انحلال ساختارهای نظامی و 
    بازگرداندن نیروها به زندگی مدنی},
    sort={خلع}
}

\newglossaryentry{r2p}{
    name={مسئولیت حمایت},
    description={\lr{Responsibility to Protect (R2P)} — 
    اصل بین‌المللی مبنی بر مسئولیت جامعه‌ی جهانی 
    برای حمایت از مردم در برابر نسل‌کشی، جرایم جنگی، 
    پاکسازی قومی و جرایم علیه بشریت},
    sort={مسئولیت}
}

\newglossaryentry{spoiler}{
    name={اسپویلر},
    description={\lr{Spoiler} — بازیگری که از فرایند 
    صلح یا گذار منتفع نمی‌شود و فعالانه تلاش 
    می‌کند آن را تخریب کند},
    sort={اسپویلر}
}

\newglossaryentry{election-monitoring}{
    name={نظارت انتخاباتی},
    description={\lr{Election Monitoring/Observation} — 
    حضور سازمان‌یافته‌ی ناظران ملی یا بین‌المللی 
    در مراحل مختلف فرایند انتخابات برای ارزیابی 
    آزاد و منصفانه بودن آن},
    sort={نظارت انتخاباتی}
}

\newglossaryentry{trc}{
    name={کمیسیون حقیقت و آشتی},
    description={\lr{Truth and Reconciliation Commission 
    (TRC)} — نهاد رسمی موقت برای بررسی نقض‌های 
    گسترده‌ی حقوق بشر، ثبت شهادت قربانیان و 
    ارائه‌ی توصیه برای جلوگیری از تکرار},
    sort={کمیسیون حقیقت}
}

\newglossaryentry{lustration}{
    name={تطهیر},
    description={\lr{Lustration/Vetting} — 
    فرایند بررسی پیشینه‌ی کارکنان دولتی و 
    حذف افراد دخیل در نقض حقوق بشر از 
    مناصب عمومی},
    sort={تطهیر}
}

\newglossaryentry{soma}{
    name={توافق‌نامه‌ی وضعیت مأموریت},
    description={\lr{Status of Mission Agreement (SOMA)} — 
    توافقی حقوقی بین سازمان بین‌المللی و دولت 
    میزبان درباره‌ی وضعیت حقوقی، مصونیت‌ها و 
    امتیازات اعضای مأموریت},
    sort={توافقنامه}
}

\newglossaryentry{contact-group}{
    name={گروه تماس},
    description={\lr{Contact Group} — 
    گروهی از دولت‌های کلیدی و نهادهای بین‌المللی 
    که برای هماهنگی دیپلماسی و فشار مشترک 
    درباره‌ی یک بحران خاص تشکیل می‌شود},
    sort={گروه تماس}
}

\newglossaryentry{srsg}{
    name={نماینده‌ی ویژه‌ی دبیرکل},
    description={\lr{Special Representative of the 
    Secretary-General (SRSG)} — 
    مقام ارشدی که توسط دبیرکل سازمان ملل 
    برای مدیریت یک مأموریت خاص منصوب می‌شود},
    sort={نماینده}
}
\newglossaryentry{roadmap}{
    name={نقشه راه},
    description={Roadmap — طرح عملیاتی مرحله‌بندی‌شده برای رسیدن از وضعیت فعلی به هدف مطلوب}
}

\newglossaryentry{exit-strategy}{
    name={استراتژی خروج},
    description={Exit Strategy — برنامه از پیش تعریف‌شده برای پایان دادن به یک مأموریت بین‌المللی}
}

\newglossaryentry{benchmarks}{
    name={معیارهای سنجش},
    description={Benchmarks — شاخص‌های کمّی و کیفی برای ارزیابی پیشرفت به سمت اهداف تعریف‌شده}
}

\newglossaryentry{national-charter}{
    name={منشور ملی},
    description={National Charter — سند توافقی میان گروه‌های سیاسی درباره اصول بنیادین گذار}
}
% واژه‌های جدید فصل ۷
\newglossaryentry{authoritarian-reversal}{
    name={بازگشت اقتدارگرایی},
    description={Authoritarian Reversal — بازگشت به حکومت اقتدارگرا پس از دوره‌ای از گذار یا دموکراسی ناقص}
}

\newglossaryentry{early-warning}{
    name={هشدار زودهنگام},
    description={Early Warning System — نظامی برای شناسایی نشانه‌های بحران قبل از وقوع آن}
}

\newglossaryentry{transition-capture}{
    name={مصادره گذار},
    description={Transition Capture — قبضه کردن فرایند گذار توسط یک گروه سیاسی خاص}
}

\newglossaryentry{international-fatigue}{
    name={خستگی بین‌المللی},
    description={International Fatigue — کاهش تدریجی توجه و منابع جامعه بین‌المللی پس از دوره اولیه}
}

\newglossaryentry{resilience}{
    name={تاب‌آوری},
    description={Resilience — توانایی یک سیستم برای جذب شوک و بازگشت به حالت عادی یا انطباق}
}

\newglossaryentry{oligarchy}{
    name={الیگارشی},
    description={Oligarchy — حکومت گروهی کوچک از نخبگان ثروتمند یا قدرتمند}
}
% واژه‌های جدید فصل ۸
\newglossaryentry{unmoit}{
    name={\lr{UNMOIT}},
    description={United Nations Mission for Oversight of Iran's Transition — مأموریت سازمان ملل متحد برای نظارت بر گذار ایران (نام پیشنهادی)}
}
\newglossaryentry{truth-commission}{
    name={کمیسیون حقیقت‌یابی},
    description={Truth and Reconciliation Commission — نهاد غیرقضایی برای بررسی نقض حقوق بشر و ایجاد آشتی ملی}
}

\newglossaryentry{trust-fund}{
    name={صندوق امانی},
    description={Trust Fund — نهاد مالی شفاف برای مدیریت کمک‌های بین‌المللی و دارایی‌های ملی}
}

\newglossaryentry{cert}{
    name={\lr{CERT}},
    description={Computer Emergency Response Team — تیم واکنش سریع به رخدادهای سایبری}
}

\newglossaryentry{interim-laws}{
    name={قوانین موقت},
    description={Interim Laws — قوانین انتقالی که تا تصویب قانون اساسی جدید حاکم‌اند}
}

\newglossaryentry{rome-statute}{
    name={اساسنامه رم},
    description={Rome Statute — اساسنامه دادگاه کیفری بین‌المللی (۱۹۹۸) برای محاکمه جنایات بین‌المللی}
}
% واژه‌های جدید فصل ۹
\newglossaryentry{raci}{
    name={\lr{RACI}},
    description={Responsible, Accountable, Consulted, Informed — ماتریس تعیین مسئولیت برای فعالیت‌ها و بازیگران}
}

\newglossaryentry{gantt}{
    name={نمودار گانت},
    description={Gantt Chart — نمودار میله‌ای زمان‌بندی پروژه که فعالیت‌ها را بر محور زمان نمایش می‌دهد}
}

\newglossaryentry{pdca}{
    name={\lr{PDCA}},
    description={Plan-Do-Check-Act — چرخه مدیریت کیفیت و بهبود مستمر دمینگ}
}

\newglossaryentry{kpi}{
    name={\lr{KPI}},
    description={Key Performance Indicator — شاخص کلیدی عملکرد برای سنجش پیشرفت}
}

\newglossaryentry{disinformation}{
    name={اطلاعات نادرست},
    description={Disinformation — انتشار عمدی اطلاعات غلط با هدف فریب یا ایجاد اختلال}
}

\newglossaryentry{constituent-assembly}{
    name={مجلس مؤسسان},
    description={Constituent Assembly — مجلس منتخب ویژه تدوین قانون اساسی جدید}
}

\newglossaryentry{institutional-dependency}{
    name={وابستگی نهادی},
    description={Institutional Dependency — وضعیتی که نهادهای محلی بدون حمایت خارجی توان عملکرد ندارند}
}
% واژه‌های جدید فصل ۱۰
\newglossaryentry{assessed-budget}{
    name={بودجه ارزیابی‌شده},
    description={Assessed Budget — بودجه‌ای که از محل حق عضویت اجباری کشورهای عضو سازمان ملل تأمین می‌شود}
}

\newglossaryentry{multi-donor-trust}{
    name={صندوق امانی چندجانبه},
    description={Multi-Donor Trust Fund (MDTF) — مکانیزم مالی مشترک برای دریافت و مدیریت کمک‌های چند کشور}
}

\newglossaryentry{iati}{
    name={\lr{IATI}},
    description={International Aid Transparency Initiative — ابتکار بین‌المللی شفافیت کمک‌ها، استانداردی برای گزارش‌دهی مالی کمک‌های توسعه‌ای}
}

\newglossaryentry{sigir}{
    name={\lr{SIGIR}},
    description={Special Inspector General for Iraq Reconstruction — بازرس ویژه بازسازی عراق، نهاد حسابرسی مستقل آمریکایی}
}

\newglossaryentry{donors-conference}{
    name={کنفرانس کمک‌دهندگان},
    description={Donors Conference/Pledging Conference — نشست بین‌المللی برای جمع‌آوری تعهدات مالی کشورها و نهادها}
}

\newglossaryentry{whistleblower}{
    name={افشاگر},
    description={Whistleblower — فردی که تخلف یا فساد در سازمان خود را گزارش می‌دهد}
}
% --- واژه‌های جدید پیوست الف ---
\newglossaryentry{pacted-transition}{
  name={گذار مذاکره‌ای},
  description={\lr{Pacted Transition} — گذاری که طی توافق بین نخبگان رژیم و اپوزیسیون صورت می‌گیرد}
}
\newglossaryentry{codesa}{
  name={کودسا},
  description={\lr{Convention for a Democratic South Africa (CODESA)} — مذاکرات چندجانبهٔ ۱۹۹۱-۱۹۹۲ آفریقای جنوبی}
}
\newglossaryentry{moncloa}{
  name={پاکت‌های مونکلوا},
  description={\lr{Moncloa Pacts (1977)} — توافق‌نامه‌های اقتصادی-سیاسی بین احزاب اسپانیا در آستانهٔ گذار}
}
\newglossaryentry{cavr}{
  name={کاور},
  description={\lr{CAVR (Comissão de Acolhimento, Verdade e Reconciliação)} — کمیسیون حقیقت و آشتی تیمور شرقی ۲۰۰۲-۲۰۰۵}
}
\newglossaryentry{ivd}{
  name={آی‌وی‌دی},
  description={\lr{Instance Vérité et Dignité (IVD)} — نهاد حقیقت و کرامت تونس ۲۰۱۴-۲۰۱۹}
}
\newglossaryentry{de-baathification}{
  name={اجتثاث بعث},
  description={\lr{De-Ba'athification} — برنامهٔ پاکسازی اعضای حزب بعث از نهادهای دولتی عراق پس از ۲۰۰۳}
}
\newglossaryentry{untaet}{
  name={آنتت},
  description={\lr{UNTAET (UN Transitional Administration in East Timor)} — مأموریت مدیریت انتقالی سازمان ملل در تیمور شرقی ۱۹۹۹-۲۰۰۲}
}
\newglossaryentry{interfet}{
  name={اینترفت},
  description={\lr{INTERFET (International Force East Timor)} — نیروی چندملیتی به رهبری استرالیا در تیمور شرقی ۱۹۹۹}
}
\newglossaryentry{tatmadaw}{
  name={تاتمادو},
  description={\lr{Tatmadaw} — نیروهای مسلح میانمار، عامل کودتای ۲۰۲۱}
}
\newglossaryentry{tni}{
  name={تی‌ان‌آی},
  description={\lr{TNI (Tentara Nasional Indonesia)} — نیروهای مسلح اندونزی با سابقهٔ امپراتوری اقتصادی مشابه سپاه}
}
% --- واژه‌های جدید پیوست ب ---
\newglossaryentry{apartheid}{
  name={آپارتاید},
  description={\lr{Apartheid} — نظام جداسازی نژادی حاکم بر آفریقای جنوبی (۱۹۴۸-۱۹۹۴) مبتنی بر تفوق نژاد سفید}
}
\newglossaryentry{sunset-clauses}{
  name={بندهای غروب},
  description={\lr{Sunset Clauses} — تضمین‌های موقت (معمولاً ۵ ساله) برای کاهش ترس نخبگان رژیم پیشین و تسهیل مذاکره. پیشنهاد جو اسلوو (۱۹۹۲) در آفریقای جنوبی}
}
\newglossaryentry{ubuntu}{
  name={اوبونتو},
  description={\lr{Ubuntu} — فلسفهٔ آفریقایی به معنای «من هستم چون ما هستیم»؛ مبنای فلسفی \lr{TRC} و آشتی ملی}
}
\newglossaryentry{sandf}{
  name={سندف},
  description={\lr{SANDF (South African National Defence Force)} — نیروی دفاعی ملی آفریقای جنوبی پس از ادغام ۷ نیروی مسلح (۱۹۹۴)}
}
\newglossaryentry{unomsa}{
  name={آنومسا},
  description={\lr{UNOMSA (UN Observer Mission in South Africa)} — مأموریت ناظران سازمان ملل در آفریقای جنوبی (۱۹۹۲-۱۹۹۴) با ۲,۱۲۰ ناظر}
}
\newglossaryentry{sufficient-consensus}{
  name={اجماع کافی},
  description={\lr{Sufficient Consensus} — اصل مذاکراتی \lr{CODESA}: نه اتفاق آرا بلکه توافق اکثریت قریب به اتفاق طرف‌ها}
}
\newglossaryentry{anc}{
  name={ای‌ان‌سی},
  description={\lr{ANC (African National Congress)} — کنگرهٔ ملی آفریقا، قدیمی‌ترین حزب آزادی‌بخش آفریقا (تأسیس ۱۹۱۲)، حزب حاکم ۱۹۹۴-۲۰۲۴}
}
\newglossaryentry{bill-of-rights}{
  name={منشور حقوق},
  description={\lr{Bill of Rights} — فصل دوم قانون اساسی آفریقای جنوبی (۱۹۹۶)؛ از پیشرفته‌ترین اسناد حقوقی جهان شامل حقوق مدنی، سیاسی، اجتماعی-اقتصادی و محیط‌زیستی}
}
% --- واژه‌های جدید پیوست پ ---
\newglossaryentry{plebiscite}{
  name={پلبیسیت},
  description={\lr{Plebiscite} — رفراندوم عمومی؛ در شیلی: رأی‌گیری ۱۹۸۸ دربارهٔ ادامهٔ حکومت پینوشه (نتیجه: «نه» ۵۶٪)}
}
\newglossaryentry{concertacion}{
  name={کنسرتاسیون},
  description={\lr{Concertación de Partidos por la Democracia} — ائتلاف فراگیر ۱۶ حزب اپوزیسیون شیلی (۱۹۸۸-۲۰۱۳) که کمپین «نه» را هدایت کرد}
}
\newglossaryentry{incremental-justice}{
  name={عدالت تدریجی},
  description={\lr{Incremental Justice} — مدل شیلیایی عدالت انتقالی: پیشروی گام‌به‌گام از حقیقت‌یابی به غرامت و سپس محاکمه در طول دهه‌ها}
}
\newglossaryentry{institutional-locks}{
  name={قفل‌های نهادی},
  description={\lr{Institutional Locks/Authoritarian Enclaves} — سازوکارهای نهادی که رژیم پیشین قبل از ترک قدرت ایجاد می‌کند تا قدرت خود را حفظ کند (مثال: سناتورهای منصوب پینوشه)}
}
\newglossaryentry{pvt}{
  name={شمارش موازی آرا},
  description={\lr{Parallel Vote Tabulation (PVT)} — سیستم شمارش مستقل آرا توسط ناظران مدنی؛ اولین بار در شیلی ۱۹۸۸ به‌طور سیستماتیک استفاده شد}
}
\newglossaryentry{universal-jurisdiction}{
  name={صلاحیت جهانی},
  description={\lr{Universal Jurisdiction} — اصل حقوقی که به دادگاه‌های هر کشور اجازه می‌دهد جنایات بین‌المللی را صرف‌نظر از محل وقوع محاکمه کنند؛ مبنای بازداشت پینوشه در لندن (۱۹۹۸)}
}
\newglossaryentry{chicago-boys}{
  name={پسران شیکاگو},
  description={\lr{Chicago Boys} — اقتصاددانان شیلیایی تحصیل‌کردهٔ دانشگاه شیکاگو که سیاست‌های نئولیبرال پینوشه را طراحی کردند}
}
\newglossaryentry{rettig}{
  name={کمیسیون رتیگ},
  description={\lr{Rettig Commission (1991)} — اولین کمیسیون حقیقت شیلی به ریاست رائول رتیگ؛ ۳,۱۹۷ قربانی کشته/ناپدید شناسایی کرد}
}
\newglossaryentry{valech}{
  name={کمیسیون والش},
  description={\lr{Valech Commission (2003-2004)} — دومین کمیسیون حقیقت شیلی متمرکز بر شکنجه؛ ۲۸,۴۵۹ قربانی شناسایی کرد}
}
% --- واژه‌های جدید پیوست ت ---
\newglossaryentry{arab-spring}{
  name={بهار عربی},
  description={\lr{Arab Spring} — موج انقلابی ۲۰۱۰-۲۰۱۲ در خاورمیانه و شمال آفریقا آغازشده از تونس}
}
\newglossaryentry{quartet}{
  name={چهارگانهٔ گفت‌وگوی ملی},
  description={\lr{Tunisian National Dialogue Quartet} — ائتلاف چهار سازمان مدنی تونس (\lr{UGTT, UTICA, LTDH, ONAT}) که بحران ۲۰۱۳ را حل کردند؛ برندهٔ نوبل صلح ۲۰۱۵}
}
\newglossaryentry{ugtt}{
  name={یو‌جی‌تی‌تی},
  description={\lr{UGTT (Union Générale Tunisienne du Travail)} — اتحادیهٔ عمومی کارگران تونس با ۷۰۰,۰۰۰ عضو؛ نقش کلیدی در انقلاب ۲۰۱۱ و گفت‌وگوی ملی ۲۰۱۳}
}
\newglossaryentry{ennahda}{
  name={النهضه},
  description={\lr{Ennahda (حرکة النهضة)} — حزب اسلام‌گرای میانه‌رو تونس به رهبری راشد غنوشی؛ اولین حزب اسلامی که داوطلبانه قدرت را واگذار کرد}
}
\newglossaryentry{etat-civil}{
  name={دولت مدنی},
  description={\lr{État civil / Civil State} — مفهوم کلیدی مادهٔ ۲ قانون اساسی تونس ۲۰۱۴: دولتی نه دینی و نه نظامی، مبتنی بر شهروندی و قانون}
}
\newglossaryentry{parite}{
  name={نمایندگی برابر},
  description={\lr{Parité / Gender Parity} — اصل برابری جنسیتی در نمایندگی سیاسی؛ مادهٔ ۴۶ قانون اساسی تونس ۲۰۱۴}
}
\newglossaryentry{democratic-fatigue}{
  name={خستگی دموکراتیک},
  description={\lr{Democratic Fatigue} — پدیدهٔ سرخوردگی مردم از بی‌ثباتی و ناکارآمدی دموکراسی نوپا که زمینهٔ بازگشت اقتدارگرایی را فراهم می‌کند}
}
\newglossaryentry{democratic-backsliding}{
  name={بازگشت اقتدارگرایی},
  description={\lr{Democratic Backsliding / Autocratization} — فرآیند تدریجی تضعیف نهادهای دموکراتیک و بازگشت به حکومت فردی؛ نمونه: تونس ۲۰۲۱، مجارستان، ترکیه}
}
% --- واژه‌های جدید پیوست ث ---
\newglossaryentry{solidarity}{
  name={همبستگی},
  description={\lr{Solidarność (Solidarity)} — اتحادیهٔ کارگری مستقل لهستان (تأسیس ۱۹۸۰) با ۱۰ میلیون عضو؛ بزرگ‌ترین جنبش اجتماعی غیرخشونت‌آمیز قرن بیستم}
}
\newglossaryentry{round-table}{
  name={مذاکرات میزگرد},
  description={\lr{Round Table Talks} — مذاکرات ۵۹ روزهٔ لهستان در سال ۱۹۸۸ بین دولت و اتحادیهٔ همبستگی}
}
\newglossaryentry{ipn}{
  name={آی‌پی‌ان},
  description={\lr{IPN (Instytut Pamięci Narodowej)} — مؤسسهٔ حافظهٔ ملی لهستان (تأسیس ۱۹۹۸) مسئول حفظ آرشیو پلیس مخفی، تحقیق تاریخی و تعقیب جنایات دورهٔ کمونیسم}
}
\newglossaryentry{shock-therapy}{
  name={شوک‌تراپی},
  description={\lr{Shock Therapy} — برنامهٔ اصلاحات اقتصادی سریع و همزمان (آزادسازی قیمت‌ها + خصوصی‌سازی + ریاضت مالی)؛ طراحی بالتسروویچ در لهستان ۱۹۹۰}
}
\newglossaryentry{eu-conditionality}{
  name={مشروطیت اروپایی},
  description={\lr{EU Conditionality} — مجموعه شرایط سیاسی و اقتصادی که کشورهای متقاضی عضویت \lr{EU} باید رعایت کنند (معیارهای کپنهاگ ۱۹۹۳)؛ قوی‌ترین ابزار تحکیم دموکراسی در اروپای شرقی}
}
\newglossaryentry{copenhagen-criteria}{
  name={معیارهای کپنهاگ},
  description={\lr{Copenhagen Criteria (1993)} — سه معیار عضویت \lr{EU}: ۱) ثبات نهادهای دموکراتیک، ۲) اقتصاد بازار کارآمد، ۳) ظرفیت اجرای قوانین \lr{EU}}
}
\newglossaryentry{velvet-revolution}{
  name={انقلاب مخملی},
  description={\lr{Velvet Revolution} — گذار مسالمت‌آمیز چکسلواکی (نوامبر ۱۹۸۹) طی ۱۰ روز؛ به رهبری واتسلاو هاول}
}
\newglossaryentry{transition-losers}{
  name={شکست‌خوردگان گذار},
  description={\lr{Transition Losers} — گروه‌های اجتماعی که از فرآیند گذار (خصوصی‌سازی، آزادسازی) آسیب دیدند و مستعد جذب پوپولیسم هستند}
}
\newglossaryentry{domino-effect}{
  name={اثر دومینو},
  description={\lr{Domino Effect} — پدیدهٔ سرایت گذار دموکراتیک از یک کشور به همسایگان؛ نمونه: اروپای شرقی ۱۹۸۹ و بهار عربی ۲۰۱۱}
}
% --- واژه‌های جدید پیوست ج ---
\newglossaryentry{cpa}{
  name={سی‌پی‌اِی},
  description={\lr{CPA (Coalition Provisional Authority)} — ادارهٔ موقت ائتلاف در عراق (۲۰۰۳-۲۰۰۴) به ریاست پل برمر؛ صادرکنندهٔ فرمان‌های فاجعه‌بار انحلال ارتش و اجتثاث بعث}
}
\newglossaryentry{muhasasa}{
  name={محاصصه},
  description={\lr{Muhasasa (Sectarian Power-Sharing)} — نظام تقسیم قدرت بر اساس فرقه (شیعه/سنی/کرد) در عراق پس از ۲۰۰۳؛ عامل اصلی فرقه‌گرایی سیاسی}
}
\newglossaryentry{order-one}{
  name={فرمان شمارهٔ ۱},
  description={\lr{CPA Order Number 1: De-Ba'athification (16 May 2003)} — فرمان اجتثاث بعث: اخراج $\sim$۸۵,۰۰۰ بعثی از مشاغل دولتی بدون تفکیک}
}
\newglossaryentry{order-two}{
  name={فرمان شمارهٔ ۲},
  description={\lr{CPA Order Number 2: Dissolution of Entities (23 May 2003)} — فرمان انحلال کامل نیروهای مسلح عراق ($\sim$۷۵۰,۰۰۰ نفر بیکار مسلح)}
}
\newglossaryentry{smart-lustration}{
  name={لوستراسیون هوشمند},
  description={\lr{Smart Lustration / Targeted Vetting} — بررسی فردی (نه جمعی) سوابق مقامات رژیم پیشین با تفکیک عاملان اصلی از اعضای عادی؛ جایگزین پیشنهادی اجتثاث افراطی}
}
\newglossaryentry{counter-model}{
  name={ضد الگو},
  description={\lr{Counter-Model / Anti-Model} — نمونه‌ای که به‌دلیل شکست، درس‌آموخته‌های «چه نباید کرد» ارائه می‌دهد (مثال اصلی: عراق ۲۰۰۳)}
}
% --- واژه‌های جدید پیوست چ ---
\newglossaryentry{tatmadaw-full}{
  name={تاتمادو},
  description={\lr{Tatmadaw (တပ်မတော်)} — نیروهای مسلح میانمار؛ حکومت مستقیم ۱۹۶۲-۲۰۱۱ و کودتای ۲۰۲۱؛ دارای امپراتوری اقتصادی \lr{MEHL} و \lr{MEC}}
}
\newglossaryentry{mehl}{
  name={ام‌ای‌اچ‌ال},
  description={\lr{MEHL (Myanmar Economic Holdings Limited)} — هلدینگ اقتصادی تاتمادو؛ بزرگ‌ترین شرکت میانمار با فعالیت در یشم، بانکداری، مخابرات و مستغلات}
}
\newglossaryentry{nld}{
  name={ان‌ال‌دی},
  description={\lr{NLD (National League for Democracy)} — لیگ ملی برای دموکراسی میانمار به رهبری آنگ سان سوچی؛ برندهٔ انتخابات ۲۰۱۵ و ۲۰۲۰}
}
\newglossaryentry{nug}{
  name={ان‌یو‌جی},
  description={\lr{NUG (National Unity Government)} — دولت وحدت ملی میانمار؛ دولت موازی تشکیل‌شده پس از کودتای ۲۰۲۱ از نمایندگان \lr{NLD} و گروه‌های قومی}
}
\newglossaryentry{cdm}{
  name={سی‌دی‌ام},
  description={\lr{CDM (Civil Disobedience Movement)} — جنبش نافرمانی مدنی میانمار پس از کودتای ۲۰۲۱؛ صدها هزار کارمند دولت اعتصاب کردند}
}
\newglossaryentry{incomplete-transition}{
  name={گذار ناتمام},
  description={\lr{Incomplete/Stalled Transition} — فرآیند گذار دموکراتیک که به‌دلیل فقدان اصلاحات ساختاری یا مقاومت نیروهای قدیم، قبل از تحکیم متوقف یا معکوس می‌شود}
}
% --- واژه‌های جدید پیوست ح ---
\newglossaryentry{unamet}{
  name={یونامت},
  description={\lr{UNAMET (UN Mission in East Timor)} — مأموریت سازمان ملل برای سازماندهی رفراندوم استقلال تیمور شرقی (ژوئن-اکتبر ۱۹۹۹)}
}
\newglossaryentry{untaet-full}{
  name={آنتت},
  description={\lr{UNTAET (UN Transitional Administration in East Timor)} — مأموریت مدیریت انتقالی سازمان ملل (۱۹۹۹-۲۰۰۲) با حاکمیت کامل بر تیمور شرقی؛ جامع‌ترین مأموریت \lr{UN}}
}
\newglossaryentry{crp}{
  name={فرآیند آشتی جامعه‌محور},
  description={\lr{CRP (Community Reconciliation Process)} — مکانیزم عدالت انتقالی محلی در تیمور شرقی: اعتراف علنی عاملان جرایم سبک + عذرخواهی + خدمت اجتماعی + پذیرش مجدد}
}
\newglossaryentry{chega}{
  name={شِگا},
  description={\lr{Chega! (Enough!)} — عنوان گزارش نهایی کمیسیون حقیقت تیمور شرقی (\lr{CAVR})؛ ۵ جلد، ۲,۵۰۰+ صفحه، مستندسازی ۱۰۲,۸۰۰ کشته}
}
\newglossaryentry{fretilin}{
  name={فرتیلین},
  description={\lr{FRETILIN (Frente Revolucionária de Timor-Leste Independente)} — جبههٔ انقلابی استقلال تیمور شرقی؛ بزرگ‌ترین حزب سیاسی و رهبر مقاومت}
}
\newglossaryentry{vieira-de-mello}{
  name={ویئیرا دملو},
  description={\lr{Sérgio Vieira de Mello (1948-2003)} — دیپلمات برزیلی، نمایندهٔ ویژهٔ دبیرکل سازمان ملل و حاکم موقت تیمور شرقی؛ در بمب‌گذاری دفتر سازمان ملل در بغداد کشته شد}
}

% ============================================================
% شروع سند
% ============================================================
\begin{document}

% ---- صفحات ابتدایی ----
\frontmatter
% ╔══════════════════════════════════════════════════════════════════╗
% ║  صفحه‌ی عنوان                                                   ║
% ╚══════════════════════════════════════════════════════════════════╝

\begin{titlepage}
\begin{tikzpicture}[remember picture, overlay]

    % پس‌زمینه
    \fill[white] (current page.south west) rectangle (current page.north east);
    
    % نوار بنفش بالا
    \fill[MainPurple] 
        (current page.north west) rectangle 
        ([yshift=-6cm]current page.north east);
    
    % نوار نازک طلایی
    \fill[HighlightGold] 
        ([yshift=-6cm]current page.north west) rectangle 
        ([yshift=-6.3cm]current page.north east);
    
    % عنوان اصلی
    \node[
        anchor=center,
        font=\fontsize{26}{32}\selectfont\bfseries,
        text=white,
        text width=15cm,
        align=center
    ] at ([yshift=-2.5cm]current page.north) {
            \rl{نظارت بین‌المللی بر گذار دموکراتیک ایران}

    };
    
    % زیرعنوان
    \node[
        anchor=center,
        font=\fontsize{18}{22}\selectfont,
        text=white!85,
        text width=14cm,
        align=center
    ] at ([yshift=-4.5cm]current page.north) {
            \rl{ابعاد، انتخاب‌ها و انتظارات}

    };
    
    % خط تزئینی وسط
    \draw[MainPurple!30, line width=0.5pt] 
        ([yshift=2cm, xshift=3cm]current page.center) -- 
        ([yshift=2cm, xshift=-3cm]current page.center);
    \node[text=MainPurple!50] at ([yshift=2cm]current page.center) {◆};
    
    % عنوان انگلیسی
    \node[
        anchor=center,
        font=\large,
        text=MainPurple!70
    ] at ([yshift=0.5cm]current page.center) {
        \lr{International Monitoring of Iran's Democratic Transition}
    };
    
    \node[
        anchor=center,
        font=\normalsize,
        text=MainPurple!50
    ] at ([yshift=-0.5cm]current page.center) {
        \lr{Dimensions, Choices, and Expectations}
    };
    
    % نام نویسنده
    \node[
        anchor=center,
        font=\LARGE\bfseries,
        text=DarkGray
    ] at ([yshift=-2.5cm]current page.center) {
            \rl{مهدی سالم}
    };
    
    \node[
        anchor=center,
        font=\large,
        text=MediumGray
    ] at ([yshift=-3.5cm]current page.center) {
        \lr{Mehdi Salem}
    };
    
    % تاریخ
    \node[
        anchor=center,
        font=\large,
        text=MediumGray
    ] at ([yshift=-5cm]current page.center) {
            \rl{زمستان ۱۴۰۴ — فوریه ۲۰۲۶}

    };
    
    % نوار پایین
    \fill[MainPurple] 
        (current page.south west) rectangle 
        ([yshift=1.5cm]current page.south east);
    
    \fill[HighlightGold] 
        ([yshift=1.5cm]current page.south west) rectangle 
        ([yshift=1.8cm]current page.south east);
    
    % متن پایین
    \node[
        anchor=center,
        font=\small,
        text=white!80
    ] at ([yshift=0.75cm]current page.south) {
        \rl{نسخه‌ی اول — برای بازبینی و نقد}
    };

\end{tikzpicture}
\end{titlepage}

% ---- صفحه‌ی حقوقی (پشت جلد) ----
\clearpage
\thispagestyle{empty}
\vspace*{\fill}
\begin{center}
    \textcolor{MediumGray}{\small
        \textbf{نظارت بین‌المللی بر گذار دموکراتیک ایران: 
        ابعاد، انتخاب‌ها و انتظارات}\\[6pt]
        نویسنده: مهدی سالم\\[6pt]
        نسخه‌ی اول — ژوئن ۲۰۲۵\\[12pt]
        \rule{4cm}{0.3pt}\\[12pt]
        این سند برای استفاده‌ی آزاد در خدمت 
        گذار دموکراتیک ایران تدوین شده است.\\
        بازتولید با ذکر منبع مجاز است.\\[12pt]
        حروف‌چینی با \lr{XeLaTeX} و \lr{XePersian}
    }
\end{center}
\vspace*{\fill}
\clearpage
% ╔══════════════════════════════════════════════════════════════════╗
% ║  تقدیم‌نامه                                                     ║
% ╚══════════════════════════════════════════════════════════════════╝

\clearpage
\thispagestyle{empty}
\vspace*{5cm}

\begin{center}
    \textcolor{MainPurple}{\rule{3cm}{0.5pt}}
\end{center}

\vspace{1cm}

\begin{pullquote}
    تقدیم به همه‌ی کسانی که 
    برای آزادی و دموکراسی در ایران 
    جان باختند، زندان کشیدند، 
    و با تمام وجود ایستادگی کردند.
    
    \vspace{8pt}
    
    و تقدیم به نسلی که خواهد ساخت 
    آنچه نسل‌های پیشین آرزویش را داشتند.
\end{pullquote}

\vspace{2cm}

\begin{center}
    \textcolor{MainPurple}{\rule{3cm}{0.5pt}}
\end{center}

\vfill

\begin{center}
    \textcolor{MediumGray}{\small
        \lr{«}زن، زندگی، آزادی\lr{»}\\[4pt]
        \lr{«}مرد، میهن، آبادی\lr{»}
    }
\end{center}

\clearpage
% ╔══════════════════════════════════════════════════════════════════╗
% ║  فصل ۰: پیش‌گفتار و چکیده‌ی اجرایی                              ║
% ╚══════════════════════════════════════════════════════════════════╝

% ---- صفحه‌ی آغازین فصل ----
\chapteropening{۰}
{\rl{پیش‌گفتار و چکیده‌ی اجرایی}}
    {MainPurple}
    {\rl{دموکراسی چیزی نیست که یک‌بار به دست آید و 
    برای همیشه بماند؛ باید هر روز از نو تولدش داد.}}
    {\rl{واتسلاو هاول، رئیس‌جمهور چکسلواکی و جمهوری چک}}

\chapter*{پیش‌گفتار و چکیده‌ی اجرایی}
\addcontentsline{toc}{chapter}{\rl{پیش‌گفتار و چکیده‌ی اجرایی}}
\label{ch:preface}
\minitoc

% ============================================================
\section*{چرا این کتاب؟ چرا الان؟}
\addcontentsline{toc}{section}{چرا این کتاب؟ چرا الان؟}
% ============================================================

ایران در آستانه‌ی یکی از مهم‌ترین لحظات تاریخ معاصر خود ایستاده 
است. نظامی که بیش از چهار دهه بر این کشور حکومت کرده، با 
بحران‌های هم‌زمان مشروعیت، کارآمدی و جانشینی دست‌وپنجه نرم 
می‌کند. جنبش‌های اعتراضی پی‌درپی — از جنبش سبز ۱۳۸۸ تا 
قیام آبان ۱۳۹۸ و خیزش 
\bilingual{زن، زندگی، آزادی}{Woman, Life, Freedom} 
در ۱۴۰۱ — نشان داده‌اند که جامعه‌ی ایرانی خواهان تغییر بنیادین 
است. پرسش دیگر «آیا تغییر خواهد آمد؟» نیست، بلکه «چگونه؟»، 
«چه وقت؟» و «با چه پیامدهایی؟» است.

\begin{keypoint}
تجربه‌ی تاریخی نشان می‌دهد که لحظه‌ی سقوط یک نظام اقتدارگرا 
تنها آغاز راه است. آنچه \emphpurple{پس از سقوط} اتفاق می‌افتد — 
فرایند گذار، نهادسازی و تحکیم دموکراسی — به‌مراتب دشوارتر، 
پیچیده‌تر و تعیین‌کننده‌تر از خودِ لحظه‌ی تغییر است.
\end{keypoint}

این کتاب بر یک بُعد حیاتی و اغلب نادیده‌گرفته‌شده‌ی فرایند 
گذار تمرکز دارد: 
\emphpurple{نظارت بین‌المللی}. 
نظارتی که اگر درست طراحی و اجرا شود، می‌تواند تفاوت میان یک 
گذار موفق (مانند آفریقای جنوبی) و یک فاجعه (مانند عراق یا لیبی) 
باشد.

تدوین این سند بر یک باور بنیادین استوار است: 
\emphpurple{ایرانیان باید مالک و هدایت‌کننده‌ی فرایند گذار باشند}، 
اما جامعه‌ی بین‌المللی نیز مسئولیت و ظرفیت دارد که از این فرایند 
پشتیبانی کند — نه مدیریت آن، بلکه \emph{همراهی} و 
\emph{نظارت} بر آن.

\sectiondivider

% ============================================================
\section*{این کتاب برای چه کسی نوشته شده؟}
\addcontentsline{toc}{section}{مخاطبان سند}
% ============================================================

این سند برای طیف متنوعی از مخاطبان تدوین شده و تلاش کرده 
زبانی پیدا کند که برای همه‌ی آن‌ها قابل فهم و مفید باشد:

\begin{table}[htbp]
    \centering
    \caption{مخاطبان سند و نحوه‌ی بهره‌برداری}
    \label{tab:audiences}
    \tablefontsize
    \begin{tabularx}{\textwidth}{
        L{3cm} X L{4cm}
    }
        \toprule
        \headerrow
        \textbf{مخاطب} & 
        \textbf{نیاز اصلی} & 
        \textbf{فصول کلیدی} \\
        \midrule
        نیروهای سیاسی اپوزیسیون &
        درک مدل‌های ممکن نظارت و آمادگی برای تعامل با نهادهای 
        بین‌المللی &
        فصول ۳، ۴، ۹، ۱۱ \\
        \altrow
        فعالان جامعه‌ی مدنی &
        شناخت حقوق و ابزارهای نظارتی و نقش خود در فرایند &
        فصول ۵، ۶، ۸ \\
        سیاست‌گذاران بین‌المللی &
        چارچوب تصمیم‌گیری و تخمین هزینه‌ها و ریسک‌ها &
        فصول ۳، ۷، ۱۰ \\
        \altrow
        نهادهای بین‌المللی (\lr{UN, EU}) &
        راهنمای عملیاتی و سناریوهای احتمالی &
        فصول ۴، ۸، ۹ \\
        پژوهشگران و آکادمیا &
        مبانی نظری و تحلیل تطبیقی &
        فصول ۱، ۲، ۳ + پیوست‌ها \\
        \altrow
        دیاسپورای ایرانی &
        درک نقش خود و فرصت‌های مشارکت &
        فصول ۵، ۶، ۱۱ \\
        رسانه‌ها &
        تحلیل جامع و قابل استناد &
        فصل ۰ (چکیده) + فصل ۱۲ \\
        \bottomrule
    \end{tabularx}
\end{table}

% ============================================================
\section*{روش‌شناسی}
\addcontentsline{toc}{section}{روش‌شناسی}
% ============================================================

این سند بر مبنای چهار ستون روش‌شناختی تدوین شده است:

\begin{enumerate}[label=\textcolor{MainPurple}{\arabic*.}]
    \item \textbf{مرور ادبیات نظری:} 
    بررسی جامع مکاتب فکری مطالعات گذار دموکراتیک 
    (\lr{Transitology})، از 
    \person{ساموئل هانتینگتون}{Samuel Huntington} و 
    \person{خوان لینتز}{Juan Linz} تا 
    \person{لری دایموند}{Larry Diamond} و 
    \person{توماس کاروترز}{Thomas Carothers}.
    
    \item \textbf{تحلیل تطبیقی:} 
    مطالعه‌ی عمیق ۹ نمونه‌ی تاریخی گذار 
    (آفریقای جنوبی، شیلی، لهستان، تونس، تیمور شرقی، 
    کوزوو، عراق، میانمار و لیبی) 
    و استخراج درس‌های آموخته برای ایران.
    
    \item \textbf{تحلیل نهادی:} 
    بررسی ساختار، ظرفیت و سوابق نهادهای بین‌المللی 
    (سازمان ملل، اتحادیه‌ی اروپا، سازمان‌های غیردولتی و...) 
    در زمینه‌ی نظارت بر گذار.
    
    \item \textbf{سناریوسازی:} 
    طراحی شش سناریوی محتمل گذار در ایران و تطبیق 
    مدل نظارتی مناسب با هر سناریو.
\end{enumerate}

\begin{warningbox}
\textbf{یک تذکر مهم روش‌شناختی:} 
هیچ مدل نظارتی واحدی وجود ندارد که مستقیماً برای ایران قابل 
استفاده باشد. ایران کشوری است با ویژگی‌های منحصربه‌فرد 
(جمعیت ۸۵ میلیونی، ژئوپلیتیک پیچیده، برنامه‌ی هسته‌ای، 
تنوع قومی-مذهبی، ارتش ایدئولوژیک) که هرگونه الگوبرداری 
مکانیکی از نمونه‌های دیگر را محکوم به شکست می‌کند. 
این سند نه یک نسخه‌ی آماده، بلکه یک \emphred{چارچوب 
تحلیلی} برای تصمیم‌گیری آگاهانه ارائه می‌دهد.
\end{warningbox}

\sectiondivider

% ============================================================
\section*{نقشه‌ی خوانش سند}
\addcontentsline{toc}{section}{نقشه‌ی خوانش سند}
% ============================================================

این کتاب به‌گونه‌ای طراحی شده که هم به‌صورت خطی (از ابتدا تا 
انتها) و هم به‌صورت گزینشی (مراجعه به فصول مورد نیاز) قابل 
خوانش باشد. نمودار زیر نقشه‌ی ارتباط فصول را نشان می‌دهد:

\begin{figure}[htbp]
    \centering
    \begin{tikzpicture}[
        node distance=1.2cm and 2.5cm,
        box/.style={
            draw, rounded corners=3pt,
            minimum height=0.9cm, minimum width=3.8cm,
            font=\footnotesize\bfseries, align=center
        },
        purplebox/.style={box, fill=PurpleBG, draw=MainPurple},
        bluebox/.style={box, fill=BlueBG, draw=MainBlue},
        greenbox/.style={box, fill=GreenBG, draw=MainGreen},
        orangebox/.style={box, fill=OrangeBG, draw=MainOrange},
        redbox/.style={box, fill=RedBG, draw=MainRed},
        yellowbox/.style={box, fill=YellowBG, draw=MainYellow},
        arr/.style={-{Stealth[length=2.5mm]}, thick, gray!70}
    ]
    
    % ستون چپ: مبانی
    \node[bluebox] (ch1) {ف۱: مبانی نظری};
    \node[bluebox, below=of ch1] (ch2) {ف۲: چرا ایران؟};
    
    % ستون وسط: تحلیل
    \node[greenbox, right=of ch1] (ch3) {ف۳: رویکردها};
    \node[orangebox, below=of ch3] (ch4) {ف۴: سناریوها};
    \node[orangebox, below=of ch4] (ch5) {ف۵: بازیگران};
    
    % ستون راست: عملیاتی
    \node[greenbox, right=of ch3] (ch6) {ف۶: تضمین‌ها};
    \node[redbox, below=of ch6] (ch7) {ف۷: ریسک‌ها};
    \node[yellowbox, below=of ch7] (ch8) {ف۸: نیازمندی‌ها};
    
    % پایین
    \node[yellowbox, below=3.5cm of ch5] (ch9) {ف۹: زمان‌بندی};
    \node[yellowbox, left=of ch9] (ch10) {ف۱۰: بودجه};
    \node[purplebox, right=of ch9] (ch11) {ف۱۱: نقشه‌ی راه};
    
    % فلش‌ها
    \draw[arr] (ch1) -- (ch3);
    \draw[arr] (ch2) -- (ch4);
    \draw[arr] (ch1) -- (ch2);
    \draw[arr] (ch3) -- (ch4);
    \draw[arr] (ch3) -- (ch6);
    \draw[arr] (ch4) -- (ch5);
    \draw[arr] (ch5) -- (ch7);
    \draw[arr] (ch6) -- (ch7);
    \draw[arr] (ch7) -- (ch8);
    \draw[arr] (ch8) -- (ch9);
    \draw[arr] (ch8) -- (ch10);
    \draw[arr] (ch9) -- (ch11);
    \draw[arr] (ch10) -- (ch11);
    
    \end{tikzpicture}
    \caption{نقشه‌ی ارتباط فصول کتاب}
    \label{fig:chapter-map}
\end{figure}

\sectiondivider

% ============================================================
\section*{چکیده‌ی اجرایی}
\addcontentsline{toc}{section}{چکیده‌ی اجرایی}
% ============================================================

\begin{executivesummary}

\subsection*{یافته‌های کلیدی}

\begin{enumerate}[
    label=\textcolor{MainPurple}{\large\bfseries\arabic*.},
    leftmargin=2cm,
    itemsep=8pt
]
    \item \textbf{ایران یک مورد استاندارد نیست.}
    با جمعیت ۸۵ میلیونی، برنامه‌ی هسته‌ای، ارتش ایدئولوژیک 
    (\lr{IRGC})، شبکه‌ی نیابتی منطقه‌ای و تنوع قومی-مذهبی، 
    هیچ مدل نظارتی موجودی بدون تطبیق اساسی برای ایران کار 
    نخواهد کرد. (\seeChapter{ch:why-iran})
    
    \item \textbf{شش مدل نظارتی شناسایی شده‌اند.}
    از «نظارت انتخاباتی محدود» تا «مدیریت بین‌المللی مستقیم». 
    هیچ‌یک به‌تنهایی کافی نیست. مدل پیشنهادی این سند: 
    \emphpurple{مدل ترکیبی-تطبیقی} با فازبندی سه‌مرحله‌ای. 
    (\seeChapter{ch:approaches})
    
    \item \textbf{سناریوی گذار، مدل نظارت را تعیین می‌کند.}
    شش سناریو تحلیل شده‌اند. سناریوی مطلوب: «گذار مذاکره‌ای»؛ 
    سناریوی محتمل‌تر: «فروپاشی ناگهانی» یا «انقلاب مردمی». 
    برای هر سناریو مدل نظارتی متفاوتی لازم است. 
    (\seeChapter{ch:scenarios})
    
    \item \textbf{بزرگ‌ترین ریسک: بازگشت اقتدارگرایی.}
    تجربه‌ی مصر ۲۰۱۳ نشان داد که حتی انقلاب‌های پرشکوه 
    می‌توانند به بازگشت دیکتاتوری بینجامند. مدیریت نقش سپاه 
    پاسداران، حیاتی‌ترین چالش امنیتی گذار ایران است. 
    (\seeChapter{ch:risks})
    
    \item \textbf{هزینه‌ی تخمینی: ۳ تا ۵ میلیارد دلار در ۱۰ سال.}
    رقمی که در مقایسه با هزینه‌ی بی‌ثباتی منطقه‌ای 
    (جنگ عراق: ۲ تریلیون دلار) یک سرمایه‌گذاری بسیار 
    مقرون‌به‌صرفه است. (\seeChapter{ch:budget})
    
    \item \textbf{آمادگی باید از الان شروع شود.}
    منتظر سقوط نظام ماندن خطرناک است. شبکه‌سازی، 
    آموزش ناظران، طراحی برنامه‌ی آماده‌باش و ایجاد اجماع 
    بین‌المللی باید از همین امروز آغاز شود. 
    (\seeChapter{ch:roadmap})
\end{enumerate}

\subsection*{ده توصیه‌ی کلیدی}

\begin{table}[H]
    \centering
    \tablefontsize
    \begin{tabularx}{\textwidth}{
        C{0.8cm} X L{3cm}
    }
        \toprule
        \headerrow
        \textbf{\#} & \textbf{توصیه} & \textbf{مخاطب اصلی} \\
        \midrule
        ۱ & 
        \textbf{مالکیت ملی:} ایرانیان هدایت‌کننده‌ باشند، 
        نه موضوع نظارت &
        همه \\
        \altrow
        ۲ & 
        \textbf{آماده‌باش از الان:} برنامه‌ریزی منتظر 
        سقوط نماند &
        اپوزیسیون + بین‌الملل \\
        ۳ & 
        \textbf{مدل ترکیبی:} از هیچ مدل واحدی کپی نکنید &
        طراحان فرایند \\
        \altrow
        ۴ & 
        \textbf{فازبندی:} از نظارت سنگین شروع و تدریجاً 
        کاهش دهید &
        مأموریت بین‌المللی \\
        ۵ & 
        \textbf{فراگیری:} زنان، اقوام، جوانان و همه‌ی 
        طیف‌ها نمایندگی شوند &
        شورای مشورتی ملی \\
        \altrow
        ۶ & 
        \textbf{عدالت آشتی‌محور:} عدالت انتقالی 
        نه انتقام‌جویانه &
        کمیسیون حقیقت \\
        ۷ & 
        \textbf{اقتصاد فراموش نشود:} رفع تحریم + بسته‌ی 
        حمایتی = شرط لازم &
        قدرت‌های بزرگ \\
        \altrow
        ۸ & 
        \textbf{سپاه: مدیریت نه نابودی:} بازسازی حرفه‌ای 
        نه حذف کامل &
        طراحان \lr{SSR} \\
        ۹ & 
        \textbf{رسانه‌ی آزاد = اکسیژن:} از روز اول 
        تضمین شود &
        دولت موقت \\
        \altrow
        ۱۰ & 
        \textbf{خروج شفاف:} مأموریت بین‌المللی 
        نقطه‌ی پایان داشته باشد &
        شورای امنیت \\
        \bottomrule
    \end{tabularx}
\end{table}

\end{executivesummary}

\sectiondivider

% ============================================================
\section*{ساختار کتاب}
\addcontentsline{toc}{section}{ساختار کتاب}
% ============================================================

\begin{table}[htbp]
    \centering
    \caption{نقشه‌ی فصول کتاب}
    \label{tab:book-structure}
    \tablefontsize
    \begin{tabularx}{\textwidth}{
        C{0.7cm} C{0.4cm} X C{1.5cm}
    }
        \toprule
        \headerrow
        \textbf{فصل} & & \textbf{عنوان و محتوای اصلی} & 
        \textbf{صفحات} \\
        \midrule
        
        \cellcolor{PurpleBG} ۰ &
        \cellcolor{PurpleBG} 🟣 &
        \cellcolor{PurpleBG} پیش‌گفتار و چکیده‌ی اجرایی &
        \cellcolor{PurpleBG} ۵-۷ \\
        
        \cellcolor{BlueBG} ۱ &
        \cellcolor{BlueBG} 🔵 &
        \cellcolor{BlueBG} مبانی نظری: گذار دموکراتیک و 
        نظارت بین‌المللی &
        \cellcolor{BlueBG} ۱۲-۱۵ \\
        
        \cellcolor{BlueBG} ۲ &
        \cellcolor{BlueBG} 🔵 &
        \cellcolor{BlueBG} چرا ایران؟ ویژگی‌ها، پیچیدگی‌ها 
        و استثنائات &
        \cellcolor{BlueBG} ۱۰-۱۲ \\
        
        \cellcolor{GreenBG} ۳ &
        \cellcolor{GreenBG} 🟢 &
        \cellcolor{GreenBG} رویکردها و ساختارهای نظارت — 
        تحلیل مقایسه‌ای شش مدل &
        \cellcolor{GreenBG} ۱۸-۲۲ \\
        
        \cellcolor{OrangeBG} ۴ &
        \cellcolor{OrangeBG} 🟠 &
        \cellcolor{OrangeBG} سناریوهای گذار و مدل‌های نظارتی 
        متناظر &
        \cellcolor{OrangeBG} ۱۵-۱۸ \\
        
        \cellcolor{OrangeBG} ۵ &
        \cellcolor{OrangeBG} 🟠 &
        \cellcolor{OrangeBG} نهادها، بازیگران و نقش هر یک &
        \cellcolor{OrangeBG} ۱۸-۲۲ \\
        
        \cellcolor{GreenBG} ۶ &
        \cellcolor{GreenBG} 🟢 &
        \cellcolor{GreenBG} تضمین‌های موفقیت و پیش‌شرط‌های 
        ساختاری &
        \cellcolor{GreenBG} ۱۲-۱۵ \\
        
        \cellcolor{RedBG} ۷ &
        \cellcolor{RedBG} 🔴 &
        \cellcolor{RedBG} آسیب‌شناسی، ریسک‌ها و چالش‌ها &
        \cellcolor{RedBG} ۱۵-۱۸ \\
        
        \cellcolor{YellowBG} ۸ &
        \cellcolor{YellowBG} 🟡 &
        \cellcolor{YellowBG} نیازمندی‌ها: انسانی، نهادی، فنی، 
        حقوقی &
        \cellcolor{YellowBG} ۱۲-۱۵ \\
        
        \cellcolor{YellowBG} ۹ &
        \cellcolor{YellowBG} 🟡 &
        \cellcolor{YellowBG} زمان‌بندی، تیم‌سازی و 
        ساختارسازی &
        \cellcolor{YellowBG} ۱۰-۱۲ \\
        
        \cellcolor{YellowBG} ۱۰ &
        \cellcolor{YellowBG} 🟡 &
        \cellcolor{YellowBG} بودجه‌بندی و تأمین مالی &
        \cellcolor{YellowBG} ۸-۱۰ \\
        
        \cellcolor{PurpleBG} ۱۱ &
        \cellcolor{PurpleBG} 🟣 &
        \cellcolor{PurpleBG} نقشه‌ی راه اجرایی و توصیه‌های 
        نهایی &
        \cellcolor{PurpleBG} ۱۰-۱۲ \\
        
        \cellcolor{PurpleBG} ۱۲ &
        \cellcolor{PurpleBG} 🟣 &
        \cellcolor{PurpleBG} جمع‌بندی و کلام آخر &
        \cellcolor{PurpleBG} ۳-۵ \\
        
        \bottomrule
    \end{tabularx}
\end{table}

\begin{operationalnote}
\textbf{درباره‌ی پیوست‌ها:} 
علاوه بر ۱۳ فصل اصلی، ده پیوست تکمیلی ارائه شده شامل:
\begin{itemize}
    \item \textbf{پیوست الف:} جدول مقایسه‌ای جامع ۹ نمونه‌ی 
    تاریخی گذار
    \item \textbf{پیوست‌های ب تا ح:} هفت مطالعه‌ی موردی 
    تفصیلی (آفریقای جنوبی، شیلی، تونس، لهستان، عراق، 
    میانمار، تیمور شرقی)
    \item \textbf{پیوست خ:} واژه‌نامه‌ی تخصصی دوزبانه 
    (فارسی-انگلیسی)
    \item \textbf{پیوست د:} فهرست نهادها و سازمان‌های کلیدی
\end{itemize}
\end{operationalnote}

% ============================================================
\section*{سپاس‌گزاری}
\addcontentsline{toc}{section}{سپاس‌گزاری}
% ============================================================

\begin{pullquote}
این سند حاصل مطالعه‌ی تجربه‌ی ده‌ها کشور، صدها نهاد و 
هزاران انسانی است که در تاریخ معاصر برای ساختن جوامع آزاد 
و دموکراتیک تلاش کرده‌اند. از همه‌ی آن‌ها — موفق و ناموفق — 
آموخته‌ایم. سپاس ویژه از تمامی پژوهشگران، فعالان و 
سیاست‌گذارانی که دانش خود را در دسترس عموم قرار داده‌اند.
\end{pullquote}

\vspace{12pt}

\begin{flushright}
    \textbf{مهدی سالم}\\
    تابستان ۱۴۰۴ / ژوئن ۲۰۲۵
\end{flushright}

\chapterend

\dominitoc
\tableofcontents
\clearpage
\listoftables
\clearpage
\listoffigures

% ---- فصول اصلی ----
\mainmatter

% ╔══════════════════════════════════════════════════════════════════╗
% ║  فصل ۱: مبانی نظری و مفهومی                                    ║
% ║  گذار دموکراتیک و نظارت بین‌المللی                              ║
% ╚══════════════════════════════════════════════════════════════════╝

% ---- صفحه‌ی آغازین فصل ----
\chapteropening{۱}
    {مبانی نظری و مفهومی}
    {MainBlue}
    {دموکراسی بدترین شکل حکومت است، 
    به‌جز تمام شکل‌های دیگری که تاکنون آزموده شده‌اند.}
    {وینستون چرچیل}

\chapter{مبانی نظری و مفهومی: گذار دموکراتیک و نظارت بین‌المللی}
\label{ch:theoretical}
\minitoc

% ---- خلاصه‌ی اجرایی فصل ----
\begin{executivesummary}
این فصل سه وظیفه‌ی اصلی دارد: نخست، ایجاد 
\emphblue{زبان مشترک} از طریق تعریف دقیق مفاهیم کلیدی؛ 
دوم، مرور \emphblue{مکاتب فکری} مطالعات گذار دموکراتیک 
از دهه‌ی ۱۹۷۰ تا امروز؛ و سوم، تبیین 
\emphblue{مبانی نظری و حقوقی نظارت بین‌المللی} بر فرایند 
گذار. خواننده پس از مطالعه‌ی این فصل درکی روشن از 
«گذار دموکراتیک چیست؟»، «نظارت بین‌المللی چه معنایی 
دارد؟» و «چه چارچوب نظری‌ای برای تحلیل مورد ایران 
مناسب است؟» خواهد داشت.
\end{executivesummary}

% ============================================================
\section{گذار دموکراتیک: تعریف و مرزبندی مفهومی}
\label{sec:transition-definition}
% ============================================================

\begin{definitionbox}{گذار دموکراتیک}
\termfn{گذار دموکراتیک}{Democratic Transition} 
فاصله‌ی زمانی میان فروپاشی یا تضعیف یک نظام 
اقتدارگرا و استقرار یک نظام دموکراتیک است. این فرایند 
لزوماً خطی، یک‌سویه یا تضمین‌شده نیست و ممکن است 
به شکست، بازگشت یا رکود بینجامد.
\end{definitionbox}

برای درک دقیق‌تر، باید گذار دموکراتیک را از چند مفهوم 
مشابه اما متفاوت تمیز دهیم:

\begin{table}[htbp]
    \centering
    \caption{تمایز مفهومی: گذار دموکراتیک و مفاهیم مرتبط}
    \label{tab:concept-distinction}
    \tablefontsize
    \begin{tabularx}{\textwidth}{
        L{2.5cm} X X
    }
        \toprule
        \headerrow
        \textbf{مفهوم} & 
        \textbf{تعریف} & 
        \textbf{تفاوت با گذار دموکراتیک} \\
        \midrule
        
        \textbf{آزادسازی سیاسی}
        \newline\lr{\footnotesize Liberalization} &
        کاهش محدودیت‌های سیاسی و مدنی توسط نظام 
        اقتدارگرا بدون تغییر بنیادین ساختار قدرت &
        ممکن است بدون گذار واقعی رخ دهد؛ 
        ابزار نظام برای مدیریت فشار \\
        \altrow
        
        \textbf{دموکراتیزاسیون}
        \newline\lr{\footnotesize Democratization} &
        فرایند گسترده‌تر و بلندمدت‌تر ساختن نهادها و 
        فرهنگ دموکراتیک &
        گذار یک مقطع زمانی است؛ دموکراتیزاسیون 
        فرایندی مستمر \\
        
        \textbf{تحکیم}
        \newline\lr{\footnotesize Consolidation} &
        مرحله‌ای که دموکراسی نهادینه شده و بازگشت 
        به اقتدارگرایی بعید است &
        تحکیم پس از گذار می‌آید؛ بسیاری از کشورها 
        در گذار متوقف می‌شوند \\
        \altrow
        
        \textbf{تغییر رژیم}
        \newline\lr{\footnotesize Regime Change} &
        هرگونه تغییر بنیادین در ماهیت نظام سیاسی 
        (نه لزوماً به سمت دموکراسی) &
        تغییر رژیم می‌تواند از دیکتاتوری به 
        دیکتاتوری باشد \\
        
        \textbf{انقلاب}
        \newline\lr{\footnotesize Revolution} &
        تغییر سریع و بنیادین ساختار قدرت، اغلب 
        با خشونت یا بسیج توده‌ای &
        انقلاب یک \emph{مکانیزم} تغییر است؛ 
        گذار دموکراتیک یک \emph{مسیر} \\
        
        \bottomrule
    \end{tabularx}
\end{table}

\subsection{سه مرحله‌ی کلاسیک گذار}
\label{subsec:three-phases}

ادبیات کلاسیک 
(\person{اودانل و اشمیتر}{O'Donnell \& Schmitter}، ۱۹۸۶) 
گذار را به سه مرحله تقسیم می‌کند:

\begin{figure}[htbp]
    \centering
    \begin{tikzpicture}[
        phase/.style={
            draw, rounded corners=4pt,
            minimum height=2cm, minimum width=4cm,
            align=center, font=\small\bfseries,
            drop shadow={shadow xshift=1mm, shadow yshift=-1mm}
        },
        arr/.style={
            -{Stealth[length=3mm]}, ultra thick
        }
    ]
    
    % مراحل
    \node[phase, fill=LightBlue, draw=MainBlue] (lib) 
        {مرحله‌ی ۱\\[2pt]آزادسازی\\[2pt]
        \lr{\footnotesize Liberalization}};
    
    \node[phase, fill=LightGreen, draw=MainGreen, 
        right=2.5cm of lib] (trans) 
        {مرحله‌ی ۲\\[2pt]گذار\\[2pt]
        \lr{\footnotesize Transition}};
    
    \node[phase, fill=LightPurple, draw=MainPurple, 
        right=2.5cm of trans] (cons) 
        {مرحله‌ی ۳\\[2pt]تحکیم\\[2pt]
        \lr{\footnotesize Consolidation}};
    
    % فلش‌ها
    \draw[arr, MainBlue] (lib) -- (trans);
    \draw[arr, MainGreen] (trans) -- (cons);
    
    % فلش بازگشت
    \draw[arr, MainRed, dashed, bend right=40] 
        (trans.south) to 
        node[below, font=\footnotesize\color{MainRed}]
        {بازگشت اقتدارگرایانه} 
        (lib.south);
    
    \draw[arr, MainRed, dashed, bend right=30] 
        (cons.south east) to 
        node[below, font=\footnotesize\color{MainRed}, 
        text width=2.5cm, align=center]
        {پس‌رفت\\دموکراتیک} 
        ([yshift=-5mm]trans.south east);
    
    % برچسب‌های توضیحی
    \node[below=2.5cm of lib, text width=3.5cm, 
        align=center, font=\tiny\color{DarkGray}] {
        کاهش سرکوب\\
        آزادی نسبی مطبوعات\\
        آزادی زندانیان
    };
    
    \node[below=2.5cm of trans, text width=3.5cm, 
        align=center, font=\tiny\color{DarkGray}] {
        مذاکره یا فروپاشی\\
        انتخابات بنیادین\\
        قانون اساسی جدید
    };
    
    \node[below=2.5cm of cons, text width=3.5cm, 
        align=center, font=\tiny\color{DarkGray}] {
        نهادینه‌شدن قواعد\\
        فرهنگ دموکراتیک\\
        انتقال مسالمت‌آمیز قدرت
    };
    
    \end{tikzpicture}
    \caption{سه مرحله‌ی کلاسیک گذار دموکراتیک 
    و مسیرهای بازگشت}
    \label{fig:three-phases}
\end{figure}

\begin{lessonlearned}
\textbf{از تجربه‌ی مصر (۲۰۱۱-۲۰۱۳):}
مصر نشان داد که حتی پس از انقلاب پرشکوه و انتخابات آزاد 
(مرحله‌ی ۲)، بدون تحکیم نهادی (مرحله‌ی ۳) بازگشت 
به اقتدارگرایی نه‌تنها ممکن بلکه محتمل است. 
ارتش مصر در ژوئیه ۲۰۱۳ کودتا کرد و دموکراسی نوپا را 
در نطفه خفه کرد. دلیل اصلی: فقدان اجماع ملی بر 
قواعد بازی و نبود نظارت بین‌المللی مؤثر.
\end{lessonlearned}

% ============================================================
\section{مکاتب فکری مطالعات گذار}
\label{sec:schools-of-thought}
% ============================================================

مطالعات گذار دموکراتیک طی نیم‌قرن اخیر مسیر پرفراز 
و نشیبی را پیموده‌اند. در این بخش، مهم‌ترین مکاتب و 
نسل‌های فکری را مرور می‌کنیم.

\subsection{نسل اول: نظریه‌ی مدرنیزاسیون (دهه‌های ۱۹۵۰-۱۹۷۰)}
\label{subsec:modernization}

\person{سیمور مارتین لیپست}{Seymour Martin Lipset} 
در مقاله‌ی تأثیرگذار خود (۱۹۵۹) استدلال کرد که 
\emphblue{توسعه‌ی اقتصادی پیش‌شرط دموکراسی است}. 
بر اساس این دیدگاه، رشد اقتصادی منجر به گسترش طبقه‌ی 
متوسط، افزایش سطح تحصیلات و تقاضا برای مشارکت سیاسی 
می‌شود.

\person{ساموئل هانتینگتون}{Samuel Huntington} 
در کتاب بنیادین خود 
\emph{\lr{Political Order in Changing Societies}} (۱۹۶۸) 
هشدار داد که مدرنیزاسیون سریع بدون نهادسازی سیاسی 
به بی‌ثباتی و خشونت می‌انجامد — نکته‌ای که برای ایران 
بسیار آموزنده است.

\begin{table}[htbp]
    \centering
    \caption{خلاصه‌ی نظریه‌ی مدرنیزاسیون}
    \label{tab:modernization}
    \begin{tabularx}{\textwidth}{L{3cm} X}
        \toprule
        \headerrow
        \textbf{عنصر} & \textbf{توضیح} \\
        \midrule
        فرض بنیادین & 
        توسعه‌ی اقتصادی → طبقه‌ی متوسط → تقاضای 
        دموکراسی \\
        \altrow
        متفکران اصلی & 
        \lr{Lipset (1959)}, \lr{Rostow (1960)}, 
        \lr{Huntington (1968)} \\
        قوت‌ها & 
        همبستگی آماری قوی بین درآمد سرانه و دموکراسی؛ 
        توان پیش‌بینی بلندمدت \\
        \altrow
        ضعف‌ها & 
        نمی‌تواند «چرا الان؟» و «چگونه؟» را توضیح دهد؛ 
        موارد نقض فراوان (چین ثروتمند اما اقتدارگرا؛ 
        هند فقیر اما دموکراتیک) \\
        ربط به ایران & 
        ایران از نظر شاخص‌های مدرنیزاسیون (تحصیلات، 
        شهرنشینی، ارتباطات) آماده‌ی دموکراسی است، 
        اما ساختار سیاسی آن را مسدود کرده \\
        \bottomrule
    \end{tabularx}
\end{table}

\subsection{نسل دوم: مطالعات انتقال‌شناسی (دهه‌های ۱۹۸۰-۱۹۹۰)}
\label{subsec:transitology}

با موج سوم دموکراتیزاسیون 
(\person{هانتینگتون}{Huntington}، ۱۹۹۱) — 
سقوط دیکتاتوری‌ها در جنوب اروپا (پرتغال ۱۹۷۴، اسپانیا 
۱۹۷۵، یونان ۱۹۷۴)، آمریکای لاتین (دهه‌ی ۱۹۸۰) و 
اروپای شرقی (۱۹۸۹) — نسل جدیدی از مطالعات شکل 
گرفت که \termfn{انتقال‌شناسی}{Transitology} نام گرفت.

بنیان‌گذاران این مکتب 
\person{گیلرمو اودانل}{Guillermo O'Donnell} و 
\person{فیلیپه اشمیتر}{Philippe Schmitter} 
در اثر بنیادین خود 
\emph{\lr{Transitions from Authoritarian Rule}} (۱۹۸۶) 
چند اصل کلیدی مطرح کردند:

\begin{enumerate}[
    label=\textcolor{MainBlue}{\bfseries\arabic*.},
    itemsep=6pt
]
    \item \textbf{اصل عدم قطعیت 
    (\lr{Uncertainty}):} 
    نتیجه‌ی گذار از پیش معلوم نیست. بازیگران در شرایط 
    عدم اطمینان تصمیم می‌گیرند.
    
    \item \textbf{اصل عاملیت 
    (\lr{Agency over Structure}):} 
    انتخاب‌های نخبگان سیاسی — نه ساختارهای اقتصادی 
    یا اجتماعی — تعیین‌کننده‌ی نتیجه‌ی گذار هستند.
    
    \item \textbf{اصل پیمان‌سازی 
    (\lr{Pact-Making}):} 
    گذارهای موفق اغلب حاصل پیمان بین بخشی از نظام 
    قدیم (\lr{softliners}) و بخشی از اپوزیسیون 
    (\lr{moderates}) هستند.
    
    \item \textbf{اصل شکاف نخبگان 
    (\lr{Elite Splitting}):} 
    گذار زمانی ممکن می‌شود که بین تندروها 
    (\lr{hardliners}) و نرم‌روها 
    (\lr{softliners}) در درون نظام شکاف ایجاد شود.
\end{enumerate}

\person{خوان لینتز}{Juan Linz} و 
\person{آلفرد استپان}{Alfred Stepan} 
در \emph{\lr{Problems of Democratic Transition 
and Consolidation}} (۱۹۹۶) این تحلیل را عمیق‌تر 
کردند و پنج «عرصه‌ی تحکیم دموکراتیک» را 
شناسایی کردند:

\begin{figure}[htbp]
    \centering
    \begin{tikzpicture}[
        arena/.style={
            draw=MainBlue, fill=BlueBG, 
            rounded corners=3pt,
            minimum height=1.4cm, minimum width=3.5cm,
            align=center, font=\small\bfseries
        },
        center/.style={
            draw=MainPurple, fill=PurpleBG,
            circle, minimum size=2.5cm,
            align=center, font=\small\bfseries
        },
        conn/.style={thick, MainBlue!60}
    ]
    
    % مرکز
    \node[center] (dem) {تحکیم\\دموکراسی};
    
    % پنج عرصه
    \node[arena] (cs) at (90:4cm) 
        {جامعه‌ی مدنی\\
        \lr{\tiny Civil Society}};
    \node[arena] (ps) at (162:4cm) 
        {جامعه‌ی سیاسی\\
        \lr{\tiny Political Society}};
    \node[arena] (rl) at (234:4cm) 
        {حاکمیت قانون\\
        \lr{\tiny Rule of Law}};
    \node[arena] (sb) at (306:4cm) 
        {دستگاه دولتی\\
        \lr{\tiny State Bureaucracy}};
    \node[arena] (es) at (18:4cm) 
        {جامعه‌ی اقتصادی\\
        \lr{\tiny Economic Society}};
    
    % اتصالات
    \draw[conn] (dem) -- (cs);
    \draw[conn] (dem) -- (ps);
    \draw[conn] (dem) -- (rl);
    \draw[conn] (dem) -- (sb);
    \draw[conn] (dem) -- (es);
    
    % اتصالات بین عرصه‌ها
    \draw[conn, dashed, MainBlue!30] (cs) -- (ps);
    \draw[conn, dashed, MainBlue!30] (ps) -- (rl);
    \draw[conn, dashed, MainBlue!30] (rl) -- (sb);
    \draw[conn, dashed, MainBlue!30] (sb) -- (es);
    \draw[conn, dashed, MainBlue!30] (es) -- (cs);
    
    \end{tikzpicture}
    \caption{پنج عرصه‌ی تحکیم دموکراتیک 
    (لینتز و استپان، ۱۹۹۶)}
    \label{fig:five-arenas}
\end{figure}

\begin{keypoint}
مدل لینتز-استپان نشان می‌دهد که 
\emphblue{انتخابات آزاد به‌تنهایی دموکراسی نیست}. 
تحکیم واقعی مستلزم رشد هم‌زمان پنج عرصه است. 
نظارت بین‌المللی باید همه‌ی این عرصه‌ها را پوشش دهد، 
نه فقط انتخابات را.
\end{keypoint}

\subsection{نسل سوم: نقد و بازنگری (دهه‌ی ۲۰۰۰ به بعد)}
\label{subsec:critique}

از اوایل دهه‌ی ۲۰۰۰ موجی از نقد بر انتقال‌شناسی 
شکل گرفت. مهم‌ترین نقدها:

\subsubsection{نقد کاروترز: پایان پارادایم گذار}

\person{توماس کاروترز}{Thomas Carothers} 
در مقاله‌ی مشهور 
\emph{\lr{The End of the Transition Paradigm}} 
(۲۰۰۲) استدلال کرد که اکثر کشورهای به‌اصطلاح 
«در حال گذار» در واقع در یک 
\termfn{منطقه‌ی خاکستری}{Gray Zone} 
گیر کرده‌اند — نه کاملاً اقتدارگرا و نه واقعاً دموکراتیک.

\subsubsection{اقتدارگرایی رقابتی}

\person{استیون لویتسکی}{Steven Levitsky} و 
\person{لوکان وِی}{Lucan Way} 
مفهوم 
\termfn{اقتدارگرایی رقابتی}{Competitive Authoritarianism} 
را معرفی کردند (۲۰۱۰): نظام‌هایی که انتخابات برگزار 
می‌کنند اما زمین بازی کاملاً ناعادلانه است.

\begin{warningbox}
\textbf{ربط مستقیم به ایران:}
جمهوری اسلامی ایران نمونه‌ی بارز 
\emphred{اقتدارگرایی انتخاباتی} است — 
نظامی که ظاهر انتخاباتی دارد اما نامزدها 
پیش‌دستی توسط شورای نگهبان فیلتر می‌شوند، 
رسانه‌ها کنترل می‌شوند و اپوزیسیون واقعی 
اجازه‌ی فعالیت ندارد. هرگونه نظارت بین‌المللی 
باید از این واقعیت آغاز کند.
\end{warningbox}

\subsubsection{بازگشت اقتدارگرایانه}

\person{لری دایموند}{Larry Diamond} 
از مفهوم 
\termfn{رکود دموکراتیک}{Democratic Recession} 
(۲۰۱۵) سخن گفت: پس از سه دهه گسترش، 
دموکراسی در جهان عقب‌نشینی کرده است. مجارستان، 
ترکیه، روسیه و ونزوئلا نمونه‌هایی هستند.

\begin{table}[htbp]
    \centering
    \caption{خلاصه‌ی سه نسل مطالعات گذار دموکراتیک}
    \label{tab:three-generations}
    \tablefontsize
    \begin{tabularx}{\textwidth}{
        L{2cm} C{2cm} X X L{2.5cm}
    }
        \toprule
        \headerrow
        \textbf{نسل} & 
        \textbf{دوره} & 
        \textbf{پرسش محوری} & 
        \textbf{پاسخ} &
        \textbf{متفکران} \\
        \midrule
        اول: مدرنیزاسیون &
        ۵۰-۷۰ &
        \emph{چه شرایطی} دموکراسی را ممکن می‌کند؟ &
        توسعه‌ی اقتصادی و اجتماعی &
        \lr{Lipset, Huntington} \\
        \altrow
        دوم: انتقال‌شناسی &
        ۸۰-۹۰ &
        \emph{چگونه} گذار رخ می‌دهد و \emph{چه کسی} 
        آن را هدایت می‌کند؟ &
        انتخاب‌های نخبگان، پیمان‌سازی &
        \lr{O'Donnell, Schmitter, Linz, Stepan} \\
        سوم: نقد و بازنگری &
        ۲۰۰۰+ &
        \emph{چرا} بسیاری از گذارها شکست 
        می‌خورند؟ &
        منطقه‌ی خاکستری، اقتدارگرایی جدید، 
        عوامل ساختاری &
        \lr{Carothers, Levitsky, Diamond} \\
        \bottomrule
    \end{tabularx}
\end{table}

\sectiondivider

% ============================================================
\section{نظارت بین‌المللی: مفهوم، تکامل و انواع}
\label{sec:monitoring-concept}
% ============================================================

\begin{definitionbox}{نظارت بین‌المللی بر گذار}
حضور سازمان‌یافته‌ی نهادها و بازیگران بین‌المللی 
در فرایند گذار یک کشور با هدف \emphblue{مشاهده، 
ارزیابی، گزارش‌دهی و در برخی موارد تسهیل‌گری یا 
تضمین} احترام به اصول دموکراتیک، حقوق بشر و 
حاکمیت قانون.
\end{definitionbox}

\subsection{تکامل تاریخی نظارت بین‌المللی}
\label{subsec:monitoring-evolution}

نظارت بین‌المللی بر فرایندهای سیاسی داخلی 
کشورها پدیده‌ای نسبتاً جدید است:

\begin{figure}[htbp]
    \centering
    \begin{tikzpicture}[
        era/.style={
            draw, rounded corners=2pt,
            minimum height=1.2cm, minimum width=2.8cm,
            align=center, font=\tiny\bfseries
        },
        timeline/.style={ultra thick, gray!50}
    ]
    
    % خط زمان
    \draw[timeline, -{Stealth}] (0,0) -- (15,0);
    
    % علامت‌های زمانی
    \foreach \x/\y in {
        1/۱۹۴۵, 3/۱۹۶۰, 5/۱۹۷۵, 
        7/۱۹۸۹, 9/۲۰۰۰, 11/۲۰۱۱, 13/۲۰۲۰
    } {
        \draw[thick] (\x, 0.15) -- (\x, -0.15) 
            node[below, font=\tiny] {\y};
    }
    
    % دوره‌ها
    \node[era, fill=BlueBG, draw=MainBlue] at (2, 1.5) {
        تأسیس UN\\حاکمیت مطلق};
    
    \node[era, fill=BlueBG, draw=MainBlue] at (4, -1.5) {
        استعمارزدایی\\خودمختاری};
    
    \node[era, fill=GreenBG, draw=MainGreen] at (6, 1.5) {
        نظارت اولیه\\اروپای جنوبی};
    
    \node[era, fill=GreenBG, draw=MainGreen] at (8, -1.5) {
        پایان جنگ سرد\\موج سوم};
    
    \node[era, fill=OrangeBG, draw=MainOrange] at (10, 1.5) {
        نظارت حرفه‌ای\\مأموریت‌های UN};
    
    \node[era, fill=OrangeBG, draw=MainOrange] at (12, -1.5) {
        بهار عربی\\چالش‌های جدید};
    
    \node[era, fill=RedBG, draw=MainRed] at (14, 1.5) {
        رکود دموکراتیک\\بازنگری};
    
    \end{tikzpicture}
    \caption{تکامل تاریخی نظارت بین‌المللی بر گذار}
    \label{fig:monitoring-evolution}
\end{figure}

\subsection{انواع نظارت بین‌المللی}
\label{subsec:monitoring-types}

نظارت بین‌المللی طیف گسترده‌ای دارد. 
مهم است که این انواع را از هم تمیز دهیم 
زیرا هر یک ابزارها، نیروی انسانی و 
پیامدهای متفاوتی دارد:

\begin{table}[htbp]
    \centering
    \caption{انواع نظارت بین‌المللی بر گذار}
    \label{tab:monitoring-types}
    \tablefontsize
    \begin{tabularx}{\textwidth}{
        L{2.2cm} X C{2cm} C{1.8cm}
    }
        \toprule
        \headerrow
        \textbf{نوع نظارت} & 
        \textbf{توضیح} & 
        \textbf{مثال تاریخی} &
        \textbf{تهاجم به حاکمیت} \\
        \midrule
        
        \textbf{نظارت انتخاباتی}
        \newline\lr{\tiny Election Monitoring} &
        مشاهده و ارزیابی فرایند انتخابات از ثبت‌نام 
        تا شمارش آرا &
        ناظران \lr{OSCE} در اروپا &
        \cellgreen{پایین} \\
        \altrow
        
        \textbf{نظارت حقوق بشری}
        \newline\lr{\tiny HR Monitoring} &
        مستندسازی و گزارش‌دهی نقض حقوق بشر &
        \lr{OHCHR} در کلمبیا &
        \cellgreen{پایین-متوسط} \\
        
        \textbf{نظارت مشورتی}
        \newline\lr{\tiny Advisory Monitoring} &
        ارائه‌ی مشاوره‌ی فنی بدون قدرت اجرایی &
        \lr{Venice Commission} &
        \cellgreen{پایین} \\
        \altrow
        
        \textbf{نظارت ساختاری}
        \newline\lr{\tiny Structural Oversight} &
        نظارت بر اصلاح نهادها، قانون اساسی 
        و بخش امنیتی &
        \lr{EU} در اروپای شرقی &
        \cellorange{متوسط} \\
        
        \textbf{نظارت اجرایی}
        \newline\lr{\tiny Executive Oversight} &
        قدرت اجرایی محدود در برخی حوزه‌ها 
        (مثلاً امنیت یا مالیه) &
        \lr{UNTAET} تیمور شرقی &
        \cellorange{بالا} \\
        \altrow
        
        \textbf{مدیریت مستقیم}
        \newline\lr{\tiny Direct Administration} &
        کنترل کامل بین‌المللی بر حکمرانی &
        \lr{CPA} عراق &
        \cellred{بسیار بالا} \\
        
        \bottomrule
    \end{tabularx}
\end{table}

\begin{recommendation}
\textbf{برای ایران:}
هیچ‌یک از انواع فوق به‌تنهایی مناسب ایران نیست. 
مدل‌های ابتدایی (نظارت انتخاباتی صرف) ناکافی‌اند 
و مدل‌های انتهایی (مدیریت مستقیم) نه ممکن‌اند و 
نه مطلوب. فصل ۳ مدل ترکیبی-تطبیقی پیشنهادی 
را تشریح خواهد کرد. (\seeChapter{ch:approaches})
\end{recommendation}

\sectiondivider

% ============================================================
\section{مبانی حقوقی نظارت بین‌المللی}
\label{sec:legal-foundations}
% ============================================================

نظارت بین‌المللی در تنش دائمی با اصل 
\termfn{حاکمیت ملی}{National Sovereignty} 
قرار دارد. این بخش مبانی حقوقی‌ای را بررسی 
می‌کند که چنین نظارتی را مشروع و قانونی می‌سازد.

\subsection{حاکمیت ملی: مطلق یا مشروط؟}

تعریف کلاسیک حاکمیت (وستفالیایی، ۱۶۴۸) 
هرگونه مداخله‌ی خارجی در امور داخلی را 
ممنوع می‌داند. اما تحولات قرن بیستم و بیست‌ویکم 
این مفهوم را اصلاح کرده‌اند:

\begin{enumerate}[
    label=\textcolor{MainBlue}{\bfseries\alph*)},
    itemsep=6pt
]
    \item \textbf{اعلامیه‌ی جهانی حقوق بشر (۱۹۴۸):}
    حقوق بشر فراتر از مرزهای ملی است.
    
    \item \textbf{میثاقین بین‌المللی (۱۹۶۶):}
    حق تعیین سرنوشت و حق مشارکت سیاسی 
    جزو حقوق بنیادین انسان‌هاست.
    
    \item \textbf{مسئولیت حمایت 
    (\lr{R2P}، ۲۰۰۵):}
    اگر دولتی از حمایت شهروندانش در برابر 
    نسل‌کشی، جرایم جنگی، پاکسازی قومی و 
    جرایم علیه بشریت ناتوان یا ناخواسته باشد، 
    مسئولیت به جامعه‌ی بین‌المللی منتقل می‌شود.
    
    \item \textbf{رویه‌ی عملی شورای امنیت:}
    صدور قطعنامه‌های متعدد برای تأسیس 
    مأموریت‌های نظارتی (کامبوج، تیمور شرقی، 
    کوزوو، لیبی و...).
\end{enumerate}

\begin{table}[htbp]
    \centering
    \caption{تکامل مفهوم حاکمیت در حقوق بین‌الملل}
    \label{tab:sovereignty-evolution}
    \begin{tabularx}{\textwidth}{
        L{2.5cm} C{2cm} X
    }
        \toprule
        \headerrow
        \textbf{دوره} & 
        \textbf{مفهوم حاکمیت} & 
        \textbf{پیامد برای نظارت} \\
        \midrule
        
        وستفالیایی (۱۶۴۸+) &
        مطلق &
        هرگونه نظارت = مداخله \\
        \altrow
        
        پس از جنگ دوم (۱۹۴۵+) &
        مشروط به حقوق بشر &
        نظارت حقوق بشری مشروع \\
        
        پس از جنگ سرد (۱۹۸۹+) &
        مسئولیت‌محور &
        نظارت انتخاباتی و ساختاری \\
        \altrow
        
        قرن ۲۱ (۲۰۰۵+) &
        مسئولیت حمایت &
        نظارت جامع + مداخله در 
        موارد استثنایی \\
        
        \bottomrule
    \end{tabularx}
\end{table}

\subsection{حق مردم بر نظارت بین‌المللی}

مفهوم نوظهوری در حال شکل‌گیری است: 
\emphblue{حق مردم — نه دولت — بر درخواست 
نظارت بین‌المللی}. این مفهوم بر آن است که 
وقتی حکومتی مشروعیت مردمی ندارد، 
\emph{مردم} حق دارند از جامعه‌ی بین‌المللی 
بخواهند که بر فرایند تغییر نظارت کند.

\begin{reflectionbox}
\textbf{پرسش تأملی:}
آیا مردم ایران — که در خیزش‌های پی‌درپی خواست 
خود برای تغییر را اعلام کرده‌اند — حق دارند 
مستقیماً از جامعه‌ی بین‌المللی درخواست نظارت کنند، 
حتی اگر حکومت مخالف باشد؟ مبنای حقوقی چنین 
درخواستی چیست؟
\end{reflectionbox}

\sectiondivider

% ============================================================
\section{چارچوب تحلیلی این کتاب}
\label{sec:analytical-framework}
% ============================================================

با توجه به مرور مکاتب فکری و انواع نظارت، 
این کتاب از یک \emphblue{چارچوب تحلیلی سه‌بعدی} 
استفاده می‌کند که سه متغیر اصلی را در تعامل با 
یکدیگر بررسی می‌کند:

\begin{enumerate}[
    label=\textcolor{MainBlue}{\bfseries بُعد \arabic*:},
    itemsep=8pt
]
    \item \textbf{عمق نظارت 
    (\lr{Depth of Monitoring}):}
    از نظارت سطحی انتخاباتی تا مدیریت مستقیم 
    بین‌المللی — نظارت چقدر «عمیق» در ساختار 
    قدرت نفوذ می‌کند؟
    
    \item \textbf{گستره‌ی نظارت 
    (\lr{Scope of Monitoring}):}
    چه حوزه‌هایی تحت نظارت قرار می‌گیرند؟ 
    فقط انتخابات؟ یا امنیت، اقتصاد، حقوق بشر، 
    قانون اساسی و رسانه هم؟
    
    \item \textbf{مدت نظارت 
    (\lr{Duration of Monitoring}):}
    نظارت چقدر طول می‌کشد؟ سه ماه؟ سه سال؟ 
    ده سال؟ مدت زمان رابطه‌ی مستقیم با موفقیت 
    تحکیم دموکراسی دارد.
\end{enumerate}

\begin{figure}[htbp]
    \centering
    \begin{tikzpicture}[
        scale=1.2,
        axis/.style={-{Stealth}, thick, MainBlue},
        label/.style={font=\small\bfseries, MainBlue}
    ]
    
    % محورها
    \draw[axis] (0,0) -- (5.5,0) 
        node[right, label] {گستره};
    \draw[axis] (0,0) -- (0,5.5) 
        node[above, label] {عمق};
    \draw[axis] (0,0) -- (-3,-2.5) 
        node[below left, label] {مدت};
    
    % مقیاس عمق
    \foreach \y/\t in {
        1/انتخاباتی,
        2/مشورتی,
        3/ساختاری,
        4/اجرایی,
        5/مدیریت مستقیم
    } {
        \draw[gray!40] (-0.1,\y) -- (0.1,\y);
        \node[left, font=\tiny, text=MediumGray] 
            at (-0.2,\y) {\t};
    }
    
    % نقاط نمونه‌های تاریخی
    \filldraw[MainGreen, opacity=0.8] (3,2) circle (4pt) 
        node[right, font=\tiny, black] 
        {آفریقای جنوبی};
    \filldraw[MainGreen, opacity=0.8] (2,1.5) circle (4pt) 
        node[right, font=\tiny, black] 
        {شیلی};
    \filldraw[MainOrange, opacity=0.8] (4,3.5) circle (4pt) 
        node[right, font=\tiny, black] 
        {تیمور شرقی};
    \filldraw[MainRed, opacity=0.8] (5,5) circle (4pt) 
        node[right, font=\tiny, black] 
        {عراق (CPA)};
    \filldraw[MainBlue, opacity=0.8] (2,1) circle (4pt) 
        node[right, font=\tiny, black] 
        {تونس};
    
    % ناحیه‌ی پیشنهادی ایران
    \draw[MainPurple, ultra thick, dashed, 
        fill=MainPurple!10, opacity=0.5] 
        (2.5,2) -- (4.5,3) -- (4,4) -- (2,3) -- cycle;
    \node[font=\small\bfseries, MainPurple] 
        at (3.2,3) {ایران؟};
    
    \end{tikzpicture}
    \caption{چارچوب تحلیلی سه‌بعدی: 
    جایگاه نمونه‌های تاریخی و محدوده‌ی پیشنهادی 
    برای ایران}
    \label{fig:3d-framework}
\end{figure}

\subsection{ترکیب چارچوب تحلیلی با مکاتب فکری}

چارچوب سه‌بعدی فوق با سه بینش نظری ترکیب می‌شود:

\begin{table}[htbp]
    \centering
    \caption{ترکیب چارچوب تحلیلی با بینش‌های نظری}
    \label{tab:framework-theory}
    \begin{tabularx}{\textwidth}{
        L{2.5cm} X L{3.5cm}
    }
        \toprule
        \headerrow
        \textbf{بینش نظری} & 
        \textbf{کاربرد در این کتاب} & 
        \textbf{فصول مرتبط} \\
        \midrule
        
        مدرنیزاسیون &
        ارزیابی آمادگی ساختاری ایران: آیا جامعه‌ی 
        ایرانی از نظر تحصیلات، شهرنشینی و طبقه‌ی 
        متوسط «آماده» است؟ &
        فصل ۲ \\
        \altrow
        
        انتقال‌شناسی &
        تحلیل سناریوهای گذار: نقش نخبگان، 
        شکاف درون نظام، پیمان‌سازی &
        فصول ۴ و ۵ \\
        
        نقد و بازنگری &
        آسیب‌شناسی: ریسک‌های بازگشت 
        اقتدارگرایانه، منطقه‌ی خاکستری، 
        شکست‌های نظارت &
        فصل ۷ \\
        
        \bottomrule
    \end{tabularx}
\end{table}

\subsection{مدل تحلیلی DIME+H}

برای سازماندهی تحلیل‌ها در فصول آتی، 
از مدل \lr{DIME+H} استفاده می‌کنیم که 
پنج حوزه‌ی کلیدی گذار را پوشش می‌دهد:

\begin{figure}[htbp]
    \centering
    \begin{tikzpicture}[
        hex/.style={
            regular polygon, regular polygon sides=6,
            draw=#1, fill=#1!10,
            minimum size=2.5cm,
            font=\small\bfseries,
            align=center,
            inner sep=0pt
        }
    ]
    
    % شش‌ضلعی مرکزی
    \node[hex=MainPurple] (center) at (0,0) 
        {گذار\\دموکراتیک\\ایران};
    
    % پنج حوزه
    \node[hex=MainBlue] (D) at (90:3.2cm) 
        {D\\[2pt]دیپلماسی\\
        \lr{\tiny Diplomacy}};
    
    \node[hex=MainRed] (I) at (162:3.2cm) 
        {I\\[2pt]اطلاعات\\
        \lr{\tiny Information}};
    
    \node[hex=MainGreen] (M) at (234:3.2cm) 
        {M\\[2pt]نظامی\\
        \lr{\tiny Military}};
    
    \node[hex=MainOrange] (E) at (306:3.2cm) 
        {E\\[2pt]اقتصاد\\
        \lr{\tiny Economy}};
    
    \node[hex=DarkYellow] (H) at (18:3.2cm) 
        {H\\[2pt]انسانی\\
        \lr{\tiny Human}};
    
    % خطوط اتصال
    \foreach \n in {D,I,M,E,H} {
        \draw[gray!40, thick] (center) -- (\n);
    }
    
    \end{tikzpicture}
    \caption{مدل تحلیلی \lr{DIME+H} 
    برای بررسی ابعاد نظارت بر گذار}
    \label{fig:dimeh-model}
\end{figure}

\begin{operationalnote}
هر فصل از فصل ۳ به بعد، تحلیل‌های خود را 
بر مبنای این پنج حوزه سازمان‌دهی خواهد کرد 
تا خواننده بتواند ابعاد مختلف هر موضوع را 
به‌صورت نظام‌مند بررسی کند.
\end{operationalnote}

\sectiondivider

% ============================================================
\section{موج‌های دموکراتیزاسیون و جایگاه ایران}
\label{sec:waves}
% ============================================================

\person{ساموئل هانتینگتون}{Samuel Huntington} 
در کتاب \emph{\lr{The Third Wave}} (۱۹۹۱) 
سه موج بزرگ دموکراتیزاسیون را شناسایی کرد. 
برخی پژوهشگران از موج چهارم نیز سخن گفته‌اند:

\begin{table}[htbp]
    \centering
    \caption{موج‌های دموکراتیزاسیون و موج‌های 
    بازگشت اقتدارگرایانه}
    \label{tab:waves}
    \tablefontsize
    \begin{tabularx}{\textwidth}{
        C{1.2cm} C{2cm} X X C{1.5cm}
    }
        \toprule
        \headerrow
        \textbf{موج} & 
        \textbf{دوره} & 
        \textbf{نمونه‌ها} & 
        \textbf{موج بازگشت} &
        \textbf{ایران} \\
        \midrule
        
        اول &
        ۱۸۲۸-۱۹۲۶ &
        آمریکا، بریتانیا، فرانسه، 
        استرالیا، کانادا &
        ۱۹۲۲-۱۹۴۲: فاشیسم در اروپا &
        مشروطه ۱۲۸۵ (ناتمام) \\
        \altrow
        
        دوم &
        ۱۹۴۳-۱۹۶۲ &
        آلمان، ژاپن، ایتالیا (پس از جنگ)، 
        هند، اسرائیل &
        ۱۹۵۸-۱۹۷۵: کودتاها در 
        آفریقا و آمریکای لاتین &
        ملی‌شدن نفت ۱۳۳۲ → کودتای ۲۸ مرداد \\
        
        سوم &
        ۱۹۷۴-۱۹۹۵ &
        پرتغال، اسپانیا، آمریکای لاتین، 
        اروپای شرقی، آفریقای جنوبی &
        ۱۹۹۵-۲۰۱۰: روسیه، 
        اقتدارگرایی جدید &
        انقلاب ۱۳۵۷: ضددموکراتیک \\
        \altrow
        
        چهارم؟ &
        ۲۰۱۰-؟ &
        تونس، (بهار عربی)، 
        سودان ۲۰۱۹؟ &
        مصر ۲۰۱۳، لیبی، سوریه، 
        سودان ۲۰۲۳ &
        \cellcolor{PurpleBG}
        \textbf{۱۴۰۱-؟} \\
        
        \bottomrule
    \end{tabularx}
\end{table}

\begin{keypoint}
ایران در تمام موج‌های دموکراتیزاسیون حضور داشته 
اما هر بار ناکام مانده است: مشروطه (۱۲۸۵) → استبداد 
صغیر و سپس رضاخان؛ ملی‌شدن نفت (۱۳۳۲) → 
کودتای ۲۸ مرداد؛ انقلاب ۱۳۵۷ → جمهوری اسلامی. 
\emphblue{آیا خیزش «زن، زندگی، آزادی» آغازگر 
موفقیت ایران در موج چهارم خواهد بود؟} پاسخ تا حد 
زیادی به طراحی درست فرایند گذار و نظارت بر آن 
بستگی دارد.
\end{keypoint}

\sectiondivider

% ============================================================
\section{جمع‌بندی فصل و پل به فصل بعد}
\label{sec:ch1-summary}
% ============================================================

\begin{chaptersummary}

\textbf{آنچه در این فصل آموختیم:}

\begin{enumerate}[
    label=\textcolor{DarkGray}{\bfseries\arabic*.},
    itemsep=4pt
]
    \item \textbf{گذار دموکراتیک} فرایندی غیرخطی، 
    نامطمئن و چندبُعدی است — نه یک رویداد واحد.
    
    \item \textbf{سه نسل فکری} مطالعات گذار 
    وجود دارد: مدرنیزاسیون (چرا؟)، انتقال‌شناسی 
    (چگونه؟) و نقد (چرا شکست؟). هر سه برای 
    تحلیل ایران لازم‌اند.
    
    \item \textbf{نظارت بین‌المللی} طیف گسترده‌ای 
    دارد: از نظارت انتخاباتی ساده تا مدیریت مستقیم 
    بین‌المللی. انتخاب مدل مناسب حیاتی است.
    
    \item \textbf{مبانی حقوقی} نظارت تکامل یافته‌اند: 
    حاکمیت مطلق جای خود را به حاکمیت مشروط و 
    مسئولیت حمایت داده است.
    
    \item \textbf{چارچوب تحلیلی سه‌بعدی} 
    (عمق × گستره × مدت) به همراه مدل \lr{DIME+H} 
    ابزار تحلیل این کتاب خواهند بود.
    
    \item \textbf{ایران در تمام موج‌های دموکراتیزاسیون 
    حضور داشته} اما هر بار ناکام مانده. 
    فصل بعد به «چرایی» و «ویژگی‌های خاص ایران» 
    خواهد پرداخت.
\end{enumerate}

\vspace{6pt}
\begin{center}
    \textcolor{MainBlue}{
        \faArrowLeft\hspace{8pt}
        \textbf{فصل بعد: چرا ایران؟ 
        ویژگی‌ها، پیچیدگی‌ها و استثنائات}
        \hspace{8pt}\faArrowLeft
    }
\end{center}

\end{chaptersummary}

\chapterend
% ╔══════════════════════════════════════════════════════════════════╗
% ║  فصل ۲: چرا ایران؟                                             ║
% ║  ویژگی‌ها، پیچیدگی‌ها و استثنائات                               ║
% ╚══════════════════════════════════════════════════════════════════╝

% ---- صفحه‌ی آغازین فصل ----
\chapteropening{۲}
    {چرا ایران؟ ویژگی‌ها، پیچیدگی‌ها و استثنائات}
    {MainBlue}
    {ایران یک کشور نیست، یک تمدن است 
    که در قالب یک کشور گنجانده شده.}
    {ریچارد فرای، ایران‌شناس آمریکایی}

\chapter{چرا ایران؟ ویژگی‌ها، پیچیدگی‌ها و استثنائات}
\label{ch:why-iran}
\minitoc

% ---- خلاصه‌ی اجرایی فصل ----
\begin{executivesummary}
این فصل تبیین می‌کند که \emphblue{ایران یک مورد 
استاندارد گذار نیست}. ترکیب منحصربه‌فرد ویژگی‌های 
ژئوپلیتیکی (قدرت منطقه‌ای، بازیگر هسته‌ای)، 
جامعه‌شناختی (تنوع قومی، جوانی جمعیت، دیاسپورای 
عظیم)، و ساختار سیاسی (نظام دوگانه، سپاه پاسداران، 
روحانیت نهادینه) ایران را به یکی از پیچیده‌ترین 
موارد احتمالی گذار دموکراتیک در تاریخ معاصر 
تبدیل می‌کند. هرگونه الگوبرداری مکانیکی از 
نمونه‌های دیگر محکوم به شکست است.
\end{executivesummary}

% ============================================================
\section{ویژگی‌های ژئوپلیتیکی: ایران در تقاطع بحران‌ها}
\label{sec:geopolitics}
% ============================================================

ایران در یکی از حساس‌ترین نقاط ژئوپلیتیکی جهان 
قرار دارد. این ویژگی هر فرایند گذاری را تحت تأثیر 
قرار می‌دهد زیرا بازیگران منطقه‌ای و بین‌المللی 
منافع حیاتی در ایران دارند.

\subsection{قدرت منطقه‌ای و بازیگر هسته‌ای}

\begin{statsbox}
\begin{minipage}[t]{0.48\textwidth}
    \begin{center}
        {\statisticfont ۸۵}\\[2pt]
        {\small میلیون نفر جمعیت}\\[8pt]
        {\statisticfont ۱.۶}\\[2pt]
        {\small میلیون کیلومتر مربع}\\[8pt]
        {\statisticfont ۴}\\[2pt]
        {\small نیروی نظامی میلیونی 
        (فعال + ذخیره)}
    \end{center}
\end{minipage}%
\hfill
\begin{minipage}[t]{0.48\textwidth}
    \begin{center}
        {\statisticfont ۴}\\[2pt]
        {\small درصد ذخایر نفت جهان}\\[8pt]
        {\statisticfont ۱۷}\\[2pt]
        {\small درصد ذخایر گاز جهان}\\[8pt]
        {\statisticfont ۱۵}\\[2pt]
        {\small کشور همسایه (زمینی و دریایی)}
    \end{center}
\end{minipage}
\end{statsbox}

ایران صرفاً یک کشور اقتدارگرای کوچک مانند 
تونس یا حتی یک کشور متوسط مانند لهستان نیست. 
ایران یک \emphblue{قدرت منطقه‌ای} با ویژگی‌های 
زیر است:

\begin{itemize}[itemsep=4pt]
    \item \textbf{بازیگر هسته‌ای:} 
    برنامه‌ی هسته‌ای ایران — چه مسالمت‌آمیز و 
    چه نظامی — هر فرایند گذاری را به مسئله‌ای 
    بین‌المللی تبدیل می‌کند. آمریکا، اسرائیل و 
    اروپا نمی‌توانند نسبت به سرنوشت برنامه‌ی 
    هسته‌ای در دوره‌ی گذار بی‌تفاوت باشند.
    
    \item \textbf{شبکه‌ی نیابتی منطقه‌ای:} 
    حزب‌الله لبنان، حشد الشعبی عراق، حوثی‌های 
    یمن و گروه‌های فلسطینی — سقوط نظام ایران 
    تأثیر فوری بر تمام این بازیگران و در نتیجه 
    بر کل ژئوپلیتیک خاورمیانه خواهد داشت.
    
    \item \textbf{تنگه‌ی هرمز:} 
    حدود ۲۰ درصد نفت جهان از این تنگه عبور 
    می‌کند. بی‌ثباتی در ایران مستقیماً بر بازار 
    جهانی انرژی تأثیر می‌گذارد.
    
    \item \textbf{رقابت قدرت‌های بزرگ:} 
    آمریکا، چین و روسیه منافع متعارض در ایران 
    دارند. چین قرارداد ۲۵ ساله با ایران دارد؛ 
    روسیه ایران را شریک استراتژیک می‌داند؛ 
    آمریکا خواهان تغییر رفتار (یا تغییر رژیم) است.
\end{itemize}

\begin{warningbox}
\textbf{پیامد برای نظارت بین‌المللی:}
بر خلاف تیمور شرقی یا حتی آفریقای جنوبی، 
گذار در ایران یک \emphred{رویداد ژئوپلیتیکی 
جهانی} است. وتوی احتمالی چین و روسیه در 
شورای امنیت، تلاش بازیگران منطقه‌ای برای 
تأثیرگذاری بر نتیجه، و حساسیت بازار انرژی 
همگی فرایند نظارت را پیچیده می‌کنند. 
هر مدل نظارتی باید این واقعیت‌ها را از 
ابتدا در نظر بگیرد.
\end{warningbox}

\subsection{جغرافیای بحران: همسایگان ناآرام}

\begin{figure}[htbp]
    \centering
    \begin{tikzpicture}[
        country/.style={
            draw, rounded corners=2pt,
            minimum height=0.8cm,
            align=center, font=\tiny\bfseries
        },
        stable/.style={country, fill=GreenBG, draw=MainGreen},
        unstable/.style={country, fill=OrangeBG, draw=MainOrange},
        crisis/.style={country, fill=RedBG, draw=MainRed},
        rival/.style={country, fill=BlueBG, draw=MainBlue},
        iran/.style={
            draw=MainPurple, fill=PurpleBG,
            rounded corners=3pt,
            minimum height=1.5cm, minimum width=2.5cm,
            font=\small\bfseries
        }
    ]
    
    % ایران در مرکز
    \node[iran] (iran) at (0,0) {ایران};
    
    % همسایگان
    \node[crisis] (iraq) at (-3.5, 1) {عراق\\بی‌ثباتی};
    \node[crisis] (afghan) at (3.5, 1) {افغانستان\\بحران};
    \node[unstable] (pak) at (3.5, -1) {پاکستان\\شکننده};
    \node[rival] (turkey) at (-2, 2.5) {ترکیه\\رقیب};
    \node[rival] (saudi) at (-2, -2.5) {عربستان\\رقیب};
    \node[unstable] (azer) at (0, 2.5) {آذربایجان\\تنش};
    \node[stable] (turkmen) at (2, 2.5) {ترکمنستان\\اقتدارگرا};
    \node[stable] (arm) at (-1, 2.5) {ارمنستان\\شکننده};
    
    % دریایی
    \node[rival] (uae) at (-0.5, -2.5) 
        {امارات\\رقیب};
    \node[rival] (kuwait) at (-3, -1) 
        {کویت};
    
    % فلش‌ها
    \draw[MainRed, thick, ->] (iraq) -- (iran) 
        node[midway, above, font=\tiny] {شبه‌نظامیان};
    \draw[MainRed, thick, ->] (afghan) -- (iran) 
        node[midway, above, font=\tiny] {مهاجرت، مواد};
    \draw[MainOrange, thick, <->] (turkey) -- (iran) 
        node[midway, left, font=\tiny] {رقابت};
    \draw[MainOrange, thick, <->] (saudi) -- (iran) 
        node[midway, left, font=\tiny] {رقابت};
    
    \end{tikzpicture}
    \caption{جغرافیای ژئوپلیتیکی ایران: 
    همسایگان و تنش‌ها}
    \label{fig:geopolitical-map}
\end{figure}

\begin{lessonlearned}
\textbf{از تجربه‌ی عراق:}
پس از سقوط صدام (۲۰۰۳)، نبود امنیت مرزی 
منجر به سرازیر شدن جنگجویان خارجی، تأسیس 
\lr{ISIS} و سال‌ها جنگ داخلی شد. ایران با ۱۵ 
همسایه‌ی زمینی و دریایی — چندین مورد آن‌ها 
بی‌ثبات — با چالش مشابه اما در مقیاس بزرگ‌تر 
روبروست. \emphblue{امنیت مرزی باید اولویت 
نخست هر برنامه‌ی نظارتی باشد.}
\end{lessonlearned}

% ============================================================
\section{ویژگی‌های جامعه‌شناختی: سرمایه و چالش}
\label{sec:sociology}
% ============================================================

\subsection{تنوع قومی-زبانی و مذهبی}

ایران یکی از متنوع‌ترین کشورهای خاورمیانه است. 
این تنوع هم یک سرمایه‌ی ملی و هم یک چالش 
بالقوه برای دوره‌ی گذار است:

\begin{table}[htbp]
    \centering
    \caption{ترکیب قومی-زبانی و مذهبی ایران 
    (تخمینی)}
    \label{tab:ethnic-composition}
    \tablefontsize
    \begin{tabularx}{\textwidth}{
        L{2cm} C{1.5cm} X L{2.5cm}
    }
        \toprule
        \headerrow
        \textbf{گروه} & 
        \textbf{درصد تخمینی} & 
        \textbf{مناطق اصلی} & 
        \textbf{حساسیت سیاسی} \\
        \midrule
        
        فارس &
        ۵۰-۵۵\% &
        مرکز، شرق، جنوب &
        اکثریت — هویت مسلط \\
        \altrow
        
        ترک آذربایجانی &
        ۱۵-۲۰\% &
        شمال‌غرب (آذربایجان) &
        حساس — مسئله‌ی هویت \\
        
        کُرد &
        ۷-۱۰\% &
        غرب (کردستان، کرمانشاه، ایلام) &
        بسیار حساس — سابقه مسلح \\
        \altrow
        
        لُر و بختیاری &
        ۶-۸\% &
        غرب و جنوب‌غرب &
        متوسط \\
        
        عرب &
        ۲-۳\% &
        جنوب‌غرب (خوزستان) &
        حساس — مسئله‌ی منابع نفتی \\
        \altrow
        
        بلوچ &
        ۲-۳\% &
        جنوب‌شرق (سیستان) &
        بسیار حساس — محرومیت \\
        
        ترکمن &
        ۱-۲\% &
        شمال‌شرق &
        متوسط \\
        \altrow
        
        گیلک و مازندرانی &
        ۵-۷\% &
        شمال (سواحل خزر) &
        پایین \\
        
        سایر &
        ۳-۵\% &
        پراکنده &
        متفاوت \\
        
        \bottomrule
    \end{tabularx}
\end{table}

\begin{comparisonbox}{تنوع قومی ایران و نمونه‌های تاریخی}
\begin{itemize}[itemsep=3pt]
    \item \textbf{آفریقای جنوبی:} 
    تنوع نژادی شدید (سیاه، سفید، رنگین‌پوست، هندی) 
    اما مدل آشتی ملی مبتنی بر «ملت رنگین‌کمان» 
    موفق بود. \emphgreen{درس: هویت فراگیر ملی 
    کلید است.}
    
    \item \textbf{یوگسلاوی:} 
    تنوع قومی-مذهبی مشابه، اما فقدان هویت ملی 
    فراگیر منجر به تجزیه و جنگ داخلی شد. 
    \emphred{هشدار: اگر هویت ایرانی تضعیف شود.}
    
    \item \textbf{اسپانیا:} 
    تنوع منطقه‌ای (کاتالان، باسک، گالیسی) با مدل 
    خودمختاری مدیریت شد. 
    \emphgreen{درس: فدرالیسم/خودمختاری 
    می‌تواند تنوع را مدیریت کند.}
    
    \item \textbf{عراق:} 
    تنوع قومی-مذهبی (شیعه، سنی، کرد) با نظام 
    سهمیه‌ای مدیریت شد — نتیجه: فلج نهادی و 
    فرقه‌گرایی. 
    \emphred{هشدار: نظام سهمیه‌ای قومی-مذهبی 
    برای ایران مناسب نیست.}
\end{itemize}
\end{comparisonbox}

\subsection{جوانی جمعیت و شکاف نسلی}

\begin{statsbox}
\begin{center}
    {\statisticfont ۶۰\%}\\[2pt]
    {\small از جمعیت ایران زیر ۳۵ سال هستند}\\[8pt]
    {\statisticfont ۹۸\%}\\[2pt]
    {\small نرخ باسوادی (یکی از بالاترین 
    در خاورمیانه)}\\[8pt]
    {\statisticfont ۴.۵}\\[2pt]
    {\small میلیون دانشجو (بالاترین نسبت 
    به جمعیت در منطقه)}
\end{center}
\end{statsbox}

جوانی جمعیت ایران هم فرصت و هم چالش است:

\begin{table}[htbp]
    \centering
    \caption{جوانی جمعیت: فرصت‌ها و چالش‌ها 
    برای گذار}
    \label{tab:youth-opportunities}
    \begin{tabularx}{\textwidth}{
        C{0.5cm} X X
    }
        \toprule
        \headerrow
        & \textbf{\textcolor{MainGreen}{فرصت}} & 
        \textbf{\textcolor{MainRed}{چالش}} \\
        \midrule
        
        ۱ &
        \cellgreen{انرژی بالا برای تغییر و 
        مشارکت مدنی} &
        \cellred{بیکاری جوانان = بمب ساعتی 
        اجتماعی} \\
        \altrow
        
        ۲ &
        \cellgreen{آشنایی با فناوری و 
        شبکه‌های اجتماعی} &
        \cellred{آسیب‌پذیری در برابر 
        اطلاعات نادرست} \\
        
        ۳ &
        \cellgreen{تحصیلات بالا = ظرفیت 
        نهادسازی} &
        \cellred{انتظارات بالا = ناامیدی سریع 
        اگر تغییر کُند باشد} \\
        \altrow
        
        ۴ &
        \cellgreen{ارتباط با جهان از طریق 
        \lr{VPN} و دیاسپورا} &
        \cellred{شکاف نسلی عمیق = تعارض 
        بر سر ارزش‌ها} \\
        
        ۵ &
        \cellgreen{رهبری جنبش «زن، زندگی، 
        آزادی» توسط زنان جوان} &
        \cellred{خطر رادیکالیزه شدن بخشی 
        از جوانان} \\
        
        \bottomrule
    \end{tabularx}
\end{table}

\subsection{دیاسپورای ایرانی: پل یا شکاف؟}

\begin{statsbox}
\begin{center}
    {\statisticfont ۴-۵}\\[2pt]
    {\small میلیون ایرانی در خارج از کشور}\\[8pt]
    {\statisticfont ۱}\\[2pt]
    {\small میلیون نفر فقط در آمریکا 
    (بزرگ‌ترین جمعیت دیاسپورا)}
\end{center}
\end{statsbox}

دیاسپورای ایرانی یکی از تحصیل‌کرده‌ترین و 
موفق‌ترین جوامع مهاجر جهان است. اما رابطه‌ی 
دیاسپورا با داخل پیچیده است:

\begin{table}[htbp]
    \centering
    \caption{نقش دوگانه‌ی دیاسپورا در گذار}
    \label{tab:diaspora-dual}
    \tablefontsize
    \begin{tabularx}{\textwidth}{
        L{2.5cm} X X
    }
        \toprule
        \headerrow
        \textbf{بُعد} & 
        \textbf{\textcolor{MainGreen}{ظرفیت}} & 
        \textbf{\textcolor{MainRed}{ریسک}} \\
        \midrule
        
        مالی &
        منابع مالی عظیم برای حمایت از گذار &
        تلاش برای «خرید» نفوذ سیاسی \\
        \altrow
        
        فنی &
        تخصص در حوزه‌های حقوقی، فنی، 
        مدیریتی &
        ناآشنایی با واقعیت‌های روزمره‌ی 
        داخل ایران \\
        
        سیاسی &
        لابی بین‌المللی و جلب حمایت 
        دولت‌ها &
        تضادهای جناحی شدید بین گروه‌های 
        دیاسپورا \\
        \altrow
        
        فرهنگی &
        پل ارتباطی با جهان، 
        انتقال ارزش‌های دموکراتیک &
        فاصله‌ی فرهنگی فزاینده با 
        نسل جدید داخل \\
        
        انسانی &
        بازگشت نخبگان و متخصصان &
        تنش بازگشتی‌ها/مقیمان 
        (تجربه‌ی افغانستان) \\
        
        \bottomrule
    \end{tabularx}
\end{table}

\begin{lessonlearned}
\textbf{از تجربه‌ی افغانستان:}
پس از سقوط طالبان (۲۰۰۱)، بسیاری از 
افغان‌های دیاسپورا به مناصب کلیدی منصوب شدند. 
نتیجه: شکاف عمیق بین «غرب‌رفته‌ها» و مردم 
محلی، اتهام فساد و ناکارآمدی، و در نهایت 
فروپاشی ۲۰۲۱. \emphblue{مشارکت دیاسپورا 
باید ساختاریافته، شفاف و تابع اراده‌ی مردم 
داخل باشد — نه جایگزین آن.}
\end{lessonlearned}

% ============================================================
\section{ویژگی‌های نظام سیاسی: 
ساختار دوگانه‌ی قدرت}
\label{sec:political-structure}
% ============================================================

\subsection{معماری قدرت در جمهوری اسلامی}

نظام سیاسی ایران ساختاری منحصربه‌فرد دارد: 
ترکیبی از عناصر 
\termfn{جمهوری}{Republican} 
(انتخابات، مجلس، رئیس‌جمهور) و عناصر 
\termfn{تئوکراتیک-اقتدارگرا}{Theocratic-Authoritarian} 
(رهبر، شورای نگهبان، سپاه). این دوگانگی 
فرایند گذار را پیچیده می‌کند.

\begin{figure}[htbp]
    \centering
    \begin{tikzpicture}[
        inst/.style={
            draw, rounded corners=2pt,
            minimum height=1cm,
            align=center, font=\footnotesize
        },
        elected/.style={inst, fill=GreenBG, draw=MainGreen},
        appointed/.style={inst, fill=RedBG, draw=MainRed},
        military/.style={inst, fill=OrangeBG, draw=MainOrange},
        supreme/.style={
            inst, fill=DarkRed!80, text=white,
            font=\small\bfseries,
            minimum height=1.3cm, minimum width=3cm
        },
        arr/.style={-{Stealth[length=2mm]}, thick},
        appoints/.style={arr, MainRed, dashed},
        elects/.style={arr, MainGreen}
    ]
    
    % رهبر
    \node[supreme] (leader) at (0, 4.5) 
        {رهبر (ولی فقیه)};
    
    % نهادهای انتصابی
    \node[appointed, minimum width=2.5cm] (gc) at (-4, 2.5) 
        {شورای نگهبان};
    \node[appointed, minimum width=2.5cm] (jud) at (4, 2.5) 
        {قوه‌ی قضاییه};
    \node[appointed, minimum width=2.5cm] (exp) at (-4, 0.5) 
        {مجمع تشخیص};
    \node[military, minimum width=2.5cm] (irgc) at (4, 0.5) 
        {سپاه پاسداران};
    
    % نهادهای انتخابی
    \node[elected, minimum width=2.5cm] (pres) at (-1.5, -1.5) 
        {رئیس‌جمهور};
    \node[elected, minimum width=2.5cm] (parl) at (1.5, -1.5) 
        {مجلس شورا};
    
    % مردم
    \node[font=\small\bfseries, MainBlue] (people) at (0, -3.5) 
        {مردم (رأی‌دهندگان)};
    
    % فلش‌ها — انتصاب
    \draw[appoints] (leader) -- (gc) 
        node[midway, above, font=\tiny] {انتصاب};
    \draw[appoints] (leader) -- (jud)
        node[midway, above, font=\tiny] {انتصاب};
    \draw[appoints] (leader) -- (irgc)
        node[midway, right, font=\tiny] {فرماندهی};
    \draw[appoints] (leader) -- (exp);
    
    % فلش‌ها — انتخاب
    \draw[elects] (people) -- (pres)
        node[midway, left, font=\tiny] {انتخاب};
    \draw[elects] (people) -- (parl)
        node[midway, right, font=\tiny] {انتخاب};
    
    % فلش — فیلتر
    \draw[MainRed, ultra thick, ->] (gc) -- (pres)
        node[midway, left, font=\tiny\color{MainRed}] 
        {فیلتر نامزدها};
    \draw[MainRed, ultra thick, ->] (gc) -- (parl)
        node[midway, right, font=\tiny\color{MainRed}] 
        {فیلتر نامزدها};
    
    % کادر تقسیم‌بندی
    \draw[MainRed, thick, dashed] 
        (-6, 1.5) -- (6, 1.5);
    \node[font=\tiny\bfseries, MainRed] at (5.5, 1.8) 
        {اقتدارگرا};
    \node[font=\tiny\bfseries, MainGreen] at (5.5, 1.2) 
        {شبه‌دموکراتیک};
    
    \end{tikzpicture}
    \caption{ساختار دوگانه‌ی قدرت در 
    جمهوری اسلامی ایران}
    \label{fig:dual-power}
\end{figure}

\begin{keypoint}
\textbf{چرا این ساختار گذار را پیچیده می‌کند:}
\begin{enumerate}[itemsep=3pt, font=\small]
    \item هیچ «مرکز واحد قدرت» وجود ندارد 
    که با آن مذاکره شود.
    \item سپاه پاسداران هم بازیگر نظامی، هم 
    اقتصادی و هم سیاسی است.
    \item شورای نگهبان مکانیزم «وتوی ساختاری» 
    بر هرگونه تغییر درون‌نظامی است.
    \item نقش رهبر شخصی و غیرنهادی است — 
    بحران جانشینی می‌تواند خود محرک گذار باشد.
    \item بخش «انتخابی» نظام سوابق و 
    زیرساخت‌هایی دارد که می‌تواند در گذار 
    مفید باشد.
\end{enumerate}
\end{keypoint}

\subsection{سپاه پاسداران: دولت در دولت}

سپاه پاسداران انقلاب اسلامی (\lr{IRGC}) 
مهم‌ترین بازیگر در هر سناریوی گذار است. 
درک ابعاد قدرت سپاه برای طراحی هر مدل 
نظارتی ضروری است:

\begin{table}[htbp]
    \centering
    \caption{ابعاد قدرت سپاه پاسداران}
    \label{tab:irgc-dimensions}
    \begin{tabularx}{\textwidth}{
        L{2cm} X L{3cm}
    }
        \toprule
        \headerrow
        \textbf{بُعد} & 
        \textbf{توضیح} & 
        \textbf{مقایسه} \\
        \midrule
        
        نظامی &
        ۱۵۰,۰۰۰+ نیروی فعال، موشک‌های 
        بالستیک، نیروی دریایی، هوافضا، 
        نیروی قدس (عملیات خارجی) &
        بزرگ‌تر از ارتش 
        بسیاری کشورها \\
        \altrow
        
        اقتصادی &
        کنترل تخمینی ۲۰-۴۰\% اقتصاد 
        ایران از طریق بنیادها، شرکت‌ها و 
        قراردادهای دولتی &
        شبیه ارتش 
        مصر یا پاکستان \\
        
        اطلاعاتی &
        سازمان اطلاعات سپاه، 
        کنترل فضای سایبری، 
        بسیج مستضعفین (۵+ میلیون) &
        \lr{KGB} + 
        شبکه‌ی بسیج محلی \\
        \altrow
        
        سیاسی &
        حضور مستقیم و غیرمستقیم 
        سپاهیان در مجلس، دولت و 
        نهادهای محلی &
        حزب حاکم + ارتش 
        ترکیبی \\
        
        ایدئولوژیک &
        تعهد به ولایت فقیه و 
        صدور انقلاب &
        نیروی ایدئولوژیک 
        (نه حرفه‌ای صرف) \\
        
        \bottomrule
    \end{tabularx}
\end{table}

\begin{warningbox}
\textbf{مهم‌ترین درس تاریخی:}
\begin{itemize}[itemsep=3pt]
    \item \textbf{عراق (دی‌بعثی‌سازی ۲۰۰۳):} 
    انحلال کامل ارتش و حزب بعث → ۵۰۰,۰۰۰ 
    مرد مسلح بیکار و خشمگین → ظهور داعش. 
    \emphred{نابودی کامل سپاه فاجعه‌بار خواهد بود.}
    
    \item \textbf{مصر (۲۰۱۱-۲۰۱۳):} 
    ارتش نقش «محافظ گذار» ایفا کرد اما سپس 
    خودش قدرت را تصاحب کرد. 
    \emphred{اعتماد کورکورانه به نظامیان خطرناک است.}
    
    \item \textbf{اندونزی (۱۹۹۸):} 
    بازسازی تدریجی ارتش از نقش سیاسی به 
    نقش حرفه‌ای طی ۱۰-۱۵ سال. 
    \emphgreen{بهترین الگو برای مدیریت سپاه.}
\end{itemize}
\end{warningbox}

% ============================================================
\section{چرا مدل‌های موجود ناکافی‌اند: 
ماتریس استثنائات ایران}
\label{sec:iran-exceptions}
% ============================================================

با جمع‌بندی ویژگی‌های ایران، اکنون می‌توانیم 
نشان دهیم چرا هیچ مدل نظارتی موجودی 
بدون تطبیق اساسی برای ایران کار نمی‌کند:

\begin{table}[htbp]
    \centering
    \caption{ماتریس استثنائات ایران نسبت به 
    نمونه‌های تاریخی}
    \label{tab:iran-exceptions}
    \tablefontsize
    \begin{tabularx}{\textwidth}{
        L{3cm} C{1cm} X
    }
        \toprule
        \headerrow
        \textbf{ویژگی} & 
        \textbf{مشابه؟} & 
        \textbf{نزدیک‌ترین مشابه و تفاوت‌ها} \\
        \midrule
        
        جمعیت ۸۵ میلیون &
        \xmark &
        بزرگ‌تر از همه‌ی نمونه‌ها به جز 
        اندونزی (۲۷۰M) \\
        \altrow
        
        برنامه‌ی هسته‌ای &
        \xmark &
        هیچ نمونه‌ی گذاری با بازیگر 
        هسته‌ای وجود ندارد \\
        
        ارتش ایدئولوژیک + اقتصادی &
        \statuswarn &
        مصر (ارتش اقتصادی) + میانمار 
        (ارتش ایدئولوژیک) \\
        \altrow
        
        شبکه‌ی نیابتی منطقه‌ای &
        \xmark &
        هیچ نمونه‌ای ندارد \\
        
        تنوع قومی + وحدت ملی نسبی &
        \statuswarn &
        آفریقای جنوبی (تنوع + وحدت)، 
        عراق (تنوع - وحدت) \\
        \altrow
        
        دیاسپورای بزرگ و ثروتمند &
        \statuswarn &
        افغانستان (بزرگ اما فقیر)، 
        کوبا (ثروتمند اما کوچک‌تر) \\
        
        تمدن ۳۰۰۰ ساله &
        \statuswarn &
        مصر (تمدنی اما متفاوت) \\
        \altrow
        
        نظام تئوکراتیک نهادینه &
        \xmark &
        هیچ نمونه‌ی مشابهی وجود ندارد 
        (طالبان و داعش = غیرنهادی) \\
        
        تاریخ شکست‌خورده‌ی دموکراسی &
        \statuswarn &
        مشابه مصر و روسیه \\
        \altrow
        
        اهمیت انرژی جهانی &
        \statuswarn &
        عراق (نفت)، لیبی (نفت) \\
        
        \bottomrule
    \end{tabularx}
    
    {\footnotesize 
    \cmark = مشابه \hspace{1cm} 
    \statuswarn = تا حدی مشابه \hspace{1cm} 
    \xmark = بی‌سابقه}
\end{table}

\begin{recommendation}
\textbf{نتیجه‌گیری عملیاتی:}
ایران نیاز به یک \emphblue{مدل نظارتی 
طراحی‌شده‌ی سفارشی} دارد — نه کپی از 
هیچ نمونه‌ی موجود. این مدل باید:
\begin{enumerate}[itemsep=3pt]
    \item ابعاد ژئوپلیتیکی را در 
    طراحی خود لحاظ کند
    \item برای مدیریت سپاه 
    استراتژی مشخص داشته باشد
    \item مکانیزم‌های مدیریت تنوع 
    قومی را شامل شود
    \item نقش دیاسپورا را 
    ساختاریافته تعریف کند
    \item بُعد هسته‌ای را مدیریت کند
    \item بر فراگیری (زنان، جوانان، اقوام) 
    تأکید ویژه داشته باشد
\end{enumerate}
فصل ۳ چنین مدلی را ارائه خواهد کرد.
\end{recommendation}

\sectiondivider

% ============================================================
\section{جمع‌بندی فصل}
\label{sec:ch2-summary}
% ============================================================

\begin{chaptersummary}

\begin{chaptersummary}

\textbf{آنچه در این فصل آموختیم:}

\begin{enumerate}[
    label=\textcolor{DarkGray}{\bfseries\arabic*.},
    itemsep=4pt
]
    \item ایران یک \textbf{قدرت منطقه‌ای با 
    ویژگی‌های ژئوپلیتیکی منحصربه‌فرد} است: 
    بازیگر هسته‌ای، شبکه‌ی نیابتی منطقه‌ای، 
    کنترل‌کننده‌ی تنگه‌ی هرمز و میدان رقابت 
    قدرت‌های بزرگ.
    
    \item \textbf{تنوع قومی-زبانی} ایران هم سرمایه 
    و هم چالش است. مدیریت این تنوع در دوره‌ی 
    گذار حیاتی است — نه با حذف و نه با 
    سهمیه‌بندی قومی، بلکه با مدل‌های 
    خودمختاری و فدرالیسم مذاکره‌ای.
    
    \item \textbf{جوانی جمعیت و تحصیلات بالا} 
    سرمایه‌ی عظیمی برای گذار است، 
    اما بیکاری و انتظارات برآورده‌نشده 
    می‌تواند آن را به تهدید تبدیل کند.
    
    \item \textbf{دیاسپورای ایرانی} ظرفیت مالی، 
    فنی و دیپلماتیک دارد اما مشارکتش باید 
    ساختاریافته و تابع اراده‌ی داخل باشد.
    
    \item \textbf{ساختار دوگانه‌ی قدرت} 
    (جمهوری + تئوکراتیک) و نقش بی‌سابقه‌ی 
    \textbf{سپاه پاسداران} به‌عنوان بازیگر 
    نظامی-اقتصادی-سیاسی-ایدئولوژیک، 
    مهم‌ترین چالش‌های ساختاری گذار هستند.
    
    \item \textbf{هیچ مدل نظارتی موجودی} 
    بدون تطبیق اساسی برای ایران کار 
    نخواهد کرد. ایران به مدل سفارشی 
    نیاز دارد.
\end{enumerate}

\vspace{6pt}
\begin{center}
    \textcolor{MainGreen}{
        \faArrowLeft\hspace{8pt}
        \textbf{فصل بعد: رویکردها، ساختارها و 
        محدوده‌های نظارت — تحلیل مقایسه‌ای}
        \hspace{8pt}\faArrowLeft
    }
\end{center}

\end{chaptersummary}

\chapterend
% ╔══════════════════════════════════════════════════════════════════╗
% ║  فصل ۳: رویکردها، ساختارها و محدوده‌های نظارت                  ║
% ║  تحلیل مقایسه‌ای شش مدل + مدل ترکیبی پیشنهادی                 ║
% ║  ** قلب تحلیلی کتاب **                                         ║
% ╚══════════════════════════════════════════════════════════════════╝

% ---- صفحه‌ی آغازین فصل ----
\chapteropening{۳}
    {رویکردها، ساختارها و محدوده‌های نظارت}
    {MainGreen}
    {بدترین کار این است که مردم را وادار کنیم 
    چرخ را دوباره اختراع کنند، در حالی که تجربه‌ی 
    دیگران روی میز است. اما بدتر از آن، کپی 
    کورکورانه‌ی تجربه‌ی دیگران بدون درک 
    زمینه‌ی خودمان است.}
    {کوفی عنان، دبیرکل سابق سازمان ملل}

\chapter{رویکردها، ساختارها و محدوده‌های نظارت: 
تحلیل مقایسه‌ای}
\label{ch:approaches}
\minitoc

% ---- خلاصه‌ی اجرایی فصل ----
\begin{executivesummary}
این فصل \emphgreen{قلب تحلیلی کتاب} است. 
در اینجا شش مدل متمایز نظارت بین‌المللی بر 
گذار شناسایی، تشریح و مقایسه می‌شوند — 
از «نظارت انتخاباتی محدود» تا 
«مدیریت بین‌المللی مستقیم». هر مدل از نظر 
قوت‌ها، ضعف‌ها، نیروی انسانی، هزینه، 
تناسب با ایران و درس‌های تاریخی تحلیل 
می‌شود. سپس \emphgreen{مدل ششم (ترکیبی-تطبیقی)} 
به‌عنوان مدل پیشنهادی این کتاب معرفی می‌گردد: 
مدلی فازبندی‌شده که عناصر بهترین مدل‌ها را 
با ویژگی‌های خاص ایران ترکیب می‌کند.
\end{executivesummary}

% ============================================================
\section{طیف نظارت: از حداقل تا حداکثر}
\label{sec:monitoring-spectrum}
% ============================================================

پیش از بررسی تفصیلی هر مدل، مهم است که 
جایگاه آن‌ها را در یک طیف کلی ببینیم:

\begin{figure}[htbp]
    \centering
    \begin{tikzpicture}[
        scale=0.95,
        model/.style={
            draw, rounded corners=3pt,
            minimum height=1.8cm, minimum width=2.6cm,
            align=center, font=\footnotesize\bfseries,
            drop shadow={shadow xshift=0.5mm, 
            shadow yshift=-0.5mm, opacity=0.3}
        }
    ]
    
    % فلش طیف
    \draw[ultra thick, -{Stealth[length=4mm]}, 
        gray!60] (-1, -1.5) -- (15, -1.5);
    
    % برچسب‌های طیف
    \node[font=\small\bfseries, MainGreen] 
        at (0, -2.2) {حداقلی};
    \node[font=\small\bfseries, MainRed] 
        at (14, -2.2) {حداکثری};
    
    \node[font=\tiny, MediumGray] 
        at (0, -2.7) {تهاجم کم به حاکمیت};
    \node[font=\tiny, MediumGray] 
        at (14, -2.7) {تهاجم زیاد به حاکمیت};
    
    % مدل‌ها
    \node[model, fill=GreenBG, draw=MainGreen] 
        (m1) at (0.5, 0.5) {
        مدل ۱\\[2pt]
        \footnotesize نظارت\\انتخاباتی\\محدود
    };
    
    \node[model, fill=GreenBG, draw=MainGreen!70] 
        (m2) at (3.3, 0.5) {
        مدل ۲\\[2pt]
        \footnotesize نظارت\\مشورتی\\و فنی
    };
    
    \node[model, fill=BlueBG, draw=MainBlue] 
        (m3) at (6.1, 0.5) {
        مدل ۳\\[2pt]
        \footnotesize نظارت\\ساختاری\\و نهادی
    };
    
    \node[model, fill=OrangeBG, draw=MainOrange] 
        (m4) at (8.9, 0.5) {
        مدل ۴\\[2pt]
        \footnotesize نظارت\\اجرایی\\و تضمینی
    };
    
    \node[model, fill=RedBG, draw=MainRed] 
        (m5) at (11.7, 0.5) {
        مدل ۵\\[2pt]
        \footnotesize مدیریت\\بین‌المللی\\مستقیم
    };
    
    % مدل ۶ پیشنهادی
    \node[model, fill=PurpleBG, draw=MainPurple, 
        line width=1.5pt] 
        (m6) at (7.5, 3.2) {
        مدل ۶\\[2pt]
        \footnotesize ترکیبی\\تطبیقی\\(پیشنهادی)
    };
    
    % فلش‌های مدل ۶
    \draw[MainPurple, thick, dashed, ->] 
        (m6.south west) -- (m2.north east);
    \draw[MainPurple, thick, dashed, ->] 
        (m6.south) -- (m3.north);
    \draw[MainPurple, thick, dashed, ->] 
        (m6.south east) -- (m4.north west);
    
    % برچسب مدل ۶
    \node[font=\tiny\itshape, MainPurple, 
        text width=4cm, align=center] 
        at (7.5, 4.3) {
        ترکیب عناصر مدل‌های ۲، ۳ و ۴\\
        با فازبندی مرحله‌ای
    };
    
    \end{tikzpicture}
    \caption{طیف مدل‌های نظارت بین‌المللی 
    و جایگاه مدل ترکیبی پیشنهادی}
    \label{fig:monitoring-spectrum}
\end{figure}

\sectiondivider

% ============================================================
\section{مدل ۱: نظارت انتخاباتی محدود}
\label{sec:model1}
% ============================================================

\begin{definitionbox}{نظارت انتخاباتی محدود 
(\lr{Election-Only Monitoring})}
حضور ناظران بین‌المللی صرفاً در دوره‌ی 
انتخابات (معمولاً ۱ تا ۳ ماه) برای مشاهده‌ی 
فرایند رأی‌گیری، شمارش آرا و اعلام نتایج. 
بدون دخالت در طراحی نهادها، اصلاح قوانین 
یا نظارت بلندمدت.
\end{definitionbox}

\subsection{محدوده و مکانیزم‌ها}

\begin{itemize}[itemsep=4pt]
    \item \textbf{پیش از رأی‌گیری:} 
    بررسی فهرست رأی‌دهندگان، ثبت‌نام نامزدها، 
    دسترسی به رسانه، فضای مبارزاتی
    \item \textbf{روز رأی‌گیری:} 
    حضور در شعب، مشاهده‌ی فرایند، 
    گزارش تخلفات
    \item \textbf{پس از رأی‌گیری:} 
    نظارت بر شمارش، تأیید نتایج، 
    گزارش نهایی
\end{itemize}

\subsection{نهادهای مجری معمول}

\begin{table}[htbp]
    \centering
    \caption{نهادهای اصلی نظارت انتخاباتی 
    بین‌المللی}
    \label{tab:election-monitors}
    \tablefontsize
    \begin{tabularx}{\textwidth}{
        L{2.5cm} X C{1.5cm} C{2cm}
    }
        \toprule
        \headerrow
        \textbf{نهاد} & 
        \textbf{تخصص و سابقه} & 
        \textbf{تعداد مأموریت} &
        \textbf{اعتبار} \\
        \midrule
        
        \lr{OSCE/ODIHR} &
        معتبرترین نهاد نظارت انتخاباتی اروپا، 
        استانداردهای دقیق &
        ۴۰۰+ &
        \starrating{5} \\
        \altrow
        
        \lr{EU EOM} &
        مأموریت‌های نظارت انتخاباتی اتحادیه 
        اروپا در خارج از اروپا &
        ۱۵۰+ &
        \starrating{4} \\
        
        \lr{Carter Center} &
        نظارت انتخاباتی مستقل، 
        ۳۹ کشور &
        ۱۱۰+ &
        \starrating{5} \\
        \altrow
        
        \lr{Commonwealth} &
        نظارت در کشورهای مشترک‌المنافع &
        ۱۵۰+ &
        \starrating{3} \\
        
        \lr{African Union} &
        نظارت در آفریقا &
        ۱۰۰+ &
        \starrating{3} \\
        
        \bottomrule
    \end{tabularx}
\end{table}

\subsection{تحلیل قوت‌ها و ضعف‌ها}

\begin{table}[htbp]
    \centering
    \caption{تحلیل \lr{SWOT} مدل ۱: 
    نظارت انتخاباتی محدود}
    \label{tab:model1-swot}
    \begin{tabularx}{\textwidth}{
        C{0.5cm} X X
    }
        \toprule
        \headerrow
        & \textbf{مثبت} & \textbf{منفی} \\
        \midrule
        
        \rotatebox{90}{\footnotesize\bfseries درونی} &
        \cellgreen{
        \textbf{قوت‌ها:}\\
        • کم‌هزینه (\$۵-۲۰M)\\
        • کم‌تنش با حاکمیت\\
        • تجربه‌ی فراوان جهانی\\
        • استقرار سریع\\
        • استانداردهای شفاف
        } &
        \cellred{
        \textbf{ضعف‌ها:}\\
        • سطحی و ناکافی\\
        • فاقد نظارت ساختاری\\
        • قابل دور زدن\\
        • فقط روز انتخابات\\
        • بدون پی‌گیری
        } \\
        \altrow
        
        \rotatebox{90}{\footnotesize\bfseries بیرونی} &
        \cellgreen{
        \textbf{فرصت‌ها:}\\
        • نقطه‌ی شروع خوب\\
        • مشروعیت‌بخشی اولیه\\
        • پذیرش عمومی بالا
        } &
        \cellred{
        \textbf{تهدیدها:}\\
        • مشروعیت‌بخشی کاذب\\
        • توهم دموکراسی\\
        • مصادره توسط اقتدارگرایان
        } \\
        
        \bottomrule
    \end{tabularx}
\end{table}

\subsection{نیروی انسانی و هزینه}

\begin{table}[htbp]
    \centering
    \caption{نیازمندی‌های عملیاتی مدل ۱}
    \label{tab:model1-operations}
    \begin{tabularx}{\textwidth}{L{3.5cm} X}
        \toprule
        \headerrow
        \textbf{عنصر} & \textbf{تخمین} \\
        \midrule
        ناظران بلندمدت (\lr{LTO}) & 
        ۵۰-۱۰۰ نفر (۲ ماه قبل) \\
        \altrow
        ناظران کوتاه‌مدت (\lr{STO}) & 
        ۵۰۰-۲,۰۰۰ نفر (۱ هفته) \\
        تیم هسته‌ای (\lr{Core Team}) & 
        ۱۵-۳۰ نفر \\
        \altrow
        مدت استقرار & ۱-۳ ماه \\
        بودجه‌ی تخمینی & 
        \$۵-۲۰ میلیون \\
        \altrow
        نهادهای پشتیبان & 
        \lr{OSCE, EU, Carter Center} \\
        \bottomrule
    \end{tabularx}
\end{table}

\subsection{نمونه‌ی تاریخی و درس‌های آموخته}

\begin{casestudy}{انتخابات میانمار ۲۰۱۰}
در ۲۰۱۰ ارتش میانمار انتخاباتی برگزار کرد 
که ظاهراً «چندحزبی» بود. ناظران بین‌المللی 
محدودی حضور داشتند. حزب حامی ارتش با 
اکثریت قاطع «پیروز» شد. جامعه‌ی بین‌المللی 
نتوانست فرایند را رد کند زیرا نظارت کافی 
نداشت. نتیجه: مشروعیت‌بخشی به نظام نظامی 
برای ۵ سال دیگر.

\vspace{4pt}
\textbf{درس برای ایران:} 
\emphred{نظارت انتخاباتی بدون نظارت ساختاری، 
ابزار مشروعیت‌بخشی به اقتدارگرایی است.} 
اگر در ایران فقط به نظارت روز انتخابات 
بسنده شود — بدون نظارت بر فیلترینگ نامزدها، 
آزادی رسانه و دسترسی اپوزیسیون — نتیجه 
تکرار تجربه‌ی میانمار خواهد بود.
\end{casestudy}

\begin{keypoint}
\textbf{حکم نهایی درباره‌ی مدل ۱ برای ایران:}
\emphred{ناکافی و خطرناک} اگر به‌تنهایی 
به‌کار رود. می‌تواند \emphgreen{بخشی از 
یک مدل ترکیبی} باشد (فاز انتخاباتی)، 
اما هرگز نباید تمام برنامه‌ی نظارت باشد.
\end{keypoint}

\sectiondivider

% ============================================================
\section{مدل ۲: نظارت مشورتی و فنی}
\label{sec:model2}
% ============================================================

\begin{definitionbox}{نظارت مشورتی و فنی 
(\lr{Advisory \& Technical Monitoring})}
ارائه‌ی مشاوره‌ی تخصصی و کمک فنی 
به نهادهای دوره‌ی گذار در حوزه‌های مختلف 
(طراحی قانون انتخابات، ظرفیت‌سازی نهادی، 
آموزش قضات و...) بدون قدرت اجرایی یا 
حق وتو.
\end{definitionbox}

\subsection{محدوده و مکانیزم‌ها}

\begin{itemize}[itemsep=4pt]
    \item \textbf{مشاوره‌ی حقوقی:} 
    کمک به تدوین قانون اساسی، قانون انتخابات، 
    قانون احزاب و قانون رسانه
    \item \textbf{ظرفیت‌سازی نهادی:} 
    آموزش کارکنان کمیسیون انتخابات، قضات، 
    پلیس و کارمندان دولتی
    \item \textbf{انتقال دانش:} 
    به‌اشتراک‌گذاری تجربه‌ی کشورهای دیگر
    \item \textbf{ارزیابی و گزارش:} 
    تهیه‌ی گزارش‌های فنی درباره‌ی وضعیت 
    نهادها و پیشنهاد اصلاحات
\end{itemize}

\subsection{نهادهای مجری معمول}

\begin{table}[htbp]
    \centering
    \caption{نهادهای اصلی نظارت مشورتی}
    \label{tab:advisory-bodies}
    \tablefontsize
    \begin{tabularx}{\textwidth}{
        L{2.5cm} X L{3cm}
    }
        \toprule
        \headerrow
        \textbf{نهاد} & 
        \textbf{تخصص} & 
        \textbf{نمونه‌ی فعالیت} \\
        \midrule
        
        \lr{UNDP} &
        ظرفیت‌سازی حکمرانی، 
        توسعه‌ی نهادی، 
        مدیریت انتخابات &
        ۱۷۰ کشور \\
        \altrow
        
        \lr{International IDEA} &
        طراحی نظام انتخاباتی، 
        دموکراسی‌سنجی &
        تونس، مصر، نپال \\
        
        \lr{Venice Commission}\newline
        (شورای اروپا) &
        حقوق اساسی، 
        قانون اساسی تطبیقی &
        ۶۰+ کشور \\
        \altrow
        
        \lr{IFES} &
        زیرساخت انتخاباتی، 
        فناوری رأی‌گیری &
        ۱۴۵ کشور \\
        
        \lr{ICTJ} &
        عدالت انتقالی، 
        کمیسیون حقیقت &
        ۳۰+ کشور \\
        
        \bottomrule
    \end{tabularx}
\end{table}

\subsection{تحلیل قوت‌ها و ضعف‌ها}

\begin{table}[htbp]
    \centering
    \caption{تحلیل \lr{SWOT} مدل ۲: 
    نظارت مشورتی و فنی}
    \label{tab:model2-swot}
    \begin{tabularx}{\textwidth}{
        C{0.5cm} X X
    }
        \toprule
        \headerrow
        & \textbf{مثبت} & \textbf{منفی} \\
        \midrule
        
        \rotatebox{90}{\footnotesize\bfseries درونی} &
        \cellgreen{
        \textbf{قوت‌ها:}\\
        • ظرفیت‌سازی واقعی\\
        • انتقال دانش بین‌المللی\\
        • پذیرش بالای داخلی\\
        • حفظ مالکیت ملی\\
        • هزینه‌ی متوسط
        } &
        \cellred{
        \textbf{ضعف‌ها:}\\
        • فاقد قدرت اجرایی\\
        • وابسته به حسن نیت\\
        • توصیه‌ها قابل نادیده‌گیری\\
        • سرعت کم\\
        • بدون ضمانت اجرا
        } \\
        \altrow
        
        \rotatebox{90}{\footnotesize\bfseries بیرونی} &
        \cellgreen{
        \textbf{فرصت‌ها:}\\
        • ساخت ظرفیت بلندمدت\\
        • ایجاد شبکه‌ی متخصصان\\
        • مقبولیت بین‌المللی
        } &
        \cellred{
        \textbf{تهدیدها:}\\
        • نادیده‌گرفتن توسط بازیگران\\
        قدرتمند داخلی\\
        • سوءاستفاده‌ی نمایشی\\
        • فرسایش در طول زمان
        } \\
        
        \bottomrule
    \end{tabularx}
\end{table}

\subsection{نمونه‌ی تاریخی: تونس ۲۰۱۱-۲۰۱۴}

\begin{casestudy}{نظارت مشورتی در تونس}
پس از انقلاب یاسمین (۲۰۱۱)، تونس از 
مشاوره‌ی فنی گسترده‌ی نهادهای بین‌المللی 
بهره برد: \lr{UNDP} در طراحی سیستم انتخاباتی، 
\lr{Venice Commission} در تدوین قانون اساسی، 
\lr{International IDEA} در ظرفیت‌سازی احزاب. 
نتیجه: قانون اساسی ۲۰۱۴ یکی از مترقی‌ترین 
قوانین اساسی جهان عرب شد.

\vspace{4pt}
\textbf{اما...} بدون ضمانت اجرایی، رئیس‌جمهور 
قیس سعید در ۲۰۲۱ قانون اساسی را تعلیق کرد، 
مجلس را منحل نمود و عملاً به اقتدارگرایی 
بازگشت. ناظران مشورتی هیچ ابزاری برای 
جلوگیری نداشتند.

\vspace{4pt}
\textbf{درس برای ایران:}
\emphblue{مشاوره بدون ضمانت اجرا، در برابر 
بازیگران فرصت‌طلب آسیب‌پذیر است.} مدل 
مشورتی باید با مکانیزم‌های تضمینی ترکیب شود.
\end{casestudy}

\sectiondivider

% ============================================================
\section{مدل ۳: نظارت ساختاری و نهادی}
\label{sec:model3}
% ============================================================

\begin{definitionbox}{نظارت ساختاری و نهادی 
(\lr{Structural \& Institutional Oversight})}
نظارت بلندمدت و عمیق بر فرایند اصلاح 
ساختار نهادهای کلیدی کشور — از قانون اساسی 
و قوه‌ی قضاییه تا بخش امنیتی و نظام اقتصادی — 
با معیارهای مشخص (\lr{benchmarks}) 
و مکانیزم‌های مشوق/تنبیه.
\end{definitionbox}

\subsection{محدوده و مکانیزم‌ها}

\begin{itemize}[itemsep=4pt]
    \item \textbf{نظارت بر قانون اساسی:} 
    بررسی انطباق قانون اساسی جدید با 
    استانداردهای دموکراتیک بین‌المللی
    \item \textbf{اصلاح بخش امنیتی (\lr{SSR}):}
    نظارت بر بازسازی ارتش، پلیس و 
    نهادهای اطلاعاتی
    \item \textbf{اصلاح قضایی:} 
    استقلال قوه‌ی قضاییه، آموزش قضات، 
    بازنگری قوانین
    \item \textbf{معیارسنجی (\lr{Benchmarking}):}
    تعیین شاخص‌های مشخص و اندازه‌گیری 
    دوره‌ای پیشرفت
    \item \textbf{مشوق/تنبیه (\lr{Conditionality}):}
    پیوند پیشرفت در اصلاحات با مزایای 
    بین‌المللی (رفع تحریم، کمک مالی، عضویت)
\end{itemize}

\subsection{نمونه‌ی تاریخی: 
اروپای شرقی و مسیر الحاق به اتحادیه‌ی اروپا}

\begin{casestudy}{نظارت ساختاری اتحادیه‌ی اروپا 
بر اروپای شرقی (۱۹۹۰-۲۰۰۴)}
پس از فروپاشی شوروی، کشورهای اروپای شرقی 
(لهستان، چک، مجارستان، استونی و...) برای 
الحاق به اتحادیه‌ی اروپا باید «معیارهای 
کپنهاگ» را برآورده می‌کردند:

\begin{enumerate}[itemsep=2pt, font=\small]
    \item ثبات نهادهای تضمین‌کننده‌ی دموکراسی
    \item حاکمیت قانون
    \item حقوق بشر و حمایت از اقلیت‌ها
    \item اقتصاد بازار کارآمد
    \item ظرفیت پذیرش تعهدات عضویت
\end{enumerate}

\lr{EU} هر سال «گزارش پیشرفت» هر کشور را 
منتشر می‌کرد. پیشرفت → نزدیکی به عضویت؛ 
عقب‌گرد → تعویق. 
\emphgreen{این مشوق (عضویت در EU) 
قوی‌ترین اهرم نظارت ساختاری در تاریخ بود.}

\vspace{4pt}
\textbf{نتیجه:} ۱۰ کشور اروپای شرقی در 
۲۰۰۴ و ۲۰۰۷ به EU پیوستند. 
اکثر آن‌ها دموکراسی‌های پایدار شدند 
(البته مجارستان و لهستان اخیراً پس‌رفت داشته‌اند).

\vspace{4pt}
\textbf{درس برای ایران:}
\emphblue{مشوق خارجی (عضویت در EU) 
کلید موفقیت این مدل بود. ایران چنین مشوقی 
ندارد.} باید جایگزین‌هایی طراحی شود: 
رفع تحریم‌ها، بسته‌ی حمایت اقتصادی، 
و عضویت در سازمان‌های بین‌المللی.
\end{casestudy}

\subsection{تحلیل مقایسه‌ای: قوت‌ها، ضعف‌ها 
و نیازمندی‌ها}

\begin{table}[htbp]
    \centering
    \caption{مشخصات عملیاتی مدل ۳}
    \label{tab:model3-specs}
    \begin{tabularx}{\textwidth}{L{3.5cm} X}
        \toprule
        \headerrow
        \textbf{عنصر} & \textbf{تخمین/توضیح} \\
        \midrule
        نیروی انسانی & 
        ۵۰۰-۲,۰۰۰ کارشناس بخشی \\
        \altrow
        مدت زمان & ۳-۷ سال (حداقل) \\
        بودجه‌ی تخمینی & 
        \$۵۰۰M - \$۲B \\
        \altrow
        پیش‌نیاز اصلی & 
        اجماع داخلی + مشوق خارجی قوی \\
        قوت اصلی & 
        عمق بالا، اصلاح واقعی نهادها \\
        \altrow
        ضعف اصلی & 
        زمان‌بر، نیاز به مشوق 
        (ایران مشوق EU ندارد) \\
        تناسب با ایران & 
        \cellblue{\textbf{بالا — اما نیاز به 
        طراحی مشوق جایگزین}} \\
        \bottomrule
    \end{tabularx}
\end{table}

\sectiondivider

% ============================================================
\section{مدل ۴: نظارت اجرایی و تضمینی}
\label{sec:model4}
% ============================================================

\begin{definitionbox}{نظارت اجرایی و تضمینی 
(\lr{Executive \& Guarantee-Based Oversight})}
حضور بین‌المللی با قدرت اجرایی محدود 
در حوزه‌های خاص (مانند امنیت، مالیه یا 
انتخابات) و مکانیزم‌های تضمین‌کننده 
برای جلوگیری از بازگشت اقتدارگرایانه. 
شامل حق وتو یا تأیید بر تصمیمات 
کلیدی در دوره‌ی معین.
\end{definitionbox}

\subsection{محدوده و مکانیزم‌ها}

\begin{itemize}[itemsep=4pt]
    \item \textbf{قدرت اجرایی محدود:} 
    مثلاً تأیید فرمانده‌ی ارتش جدید، 
    نظارت بر بودجه‌ی دفاعی، 
    تأیید قضات ارشد
    
    \item \textbf{مکانیزم قفل‌کننده 
    (\lr{Locking Mechanism}):}
    ضمانت‌هایی که تغییر آن‌ها 
    نیاز به تأیید بین‌المللی دارد 
    (مثلاً تغییر قانون اساسی)
    
    \item \textbf{نیروهای بین‌المللی:} 
    حضور نظامی/انتظامی 
    بین‌المللی برای تضمین امنیت 
    (مشابه \lr{KFOR} در کوزوو)
    
    \item \textbf{نظارت مالی:}
    مدیریت مشترک درآمد نفت 
    یا صندوق امانی بین‌المللی
\end{itemize}

\subsection{نمونه‌های تاریخی}

\begin{table}[htbp]
    \centering
    \caption{نمونه‌های نظارت اجرایی در تاریخ}
    \label{tab:model4-examples}
    \tablefontsize
    \begin{tabularx}{\textwidth}{
        L{2cm} C{1.5cm} X C{1.5cm}
    }
        \toprule
        \headerrow
        \textbf{مورد} & 
        \textbf{نهاد} & 
        \textbf{محدوده‌ی اختیار} &
        \textbf{نتیجه} \\
        \midrule
        
        تیمور شرقی &
        \lr{UNTAET} &
        مدیریت کامل اجرایی → انتقال 
        تدریجی به دولت ملی &
        \cellgreen{نسبتاً موفق} \\
        \altrow
        
        کوزوو &
        \lr{UNMIK} &
        مدیریت اجرایی + \lr{KFOR} 
        (نیروی نظامی) &
        \cellorange{ناتمام} \\
        
        بوسنی &
        \lr{OHR} &
        نماینده‌ی عالی با قدرت 
        «بان» (حکم اجرایی) &
        \cellorange{وابستگی} \\
        \altrow
        
        کامبوج &
        \lr{UNTAC} &
        مدیریت مشترک + نظارت 
        انتخاباتی &
        \cellgreen{موفق اولیه} \\
        
        \bottomrule
    \end{tabularx}
\end{table}

\begin{lessonlearned}
\textbf{از تجربه‌ی تیمور شرقی (\lr{UNTAET}, 
۱۹۹۹-۲۰۰۲):}
\lr{UNTAET} موفق‌ترین نمونه‌ی نظارت اجرایی 
محسوب می‌شود. دلایل موفقیت:
\begin{enumerate}[itemsep=2pt, font=\small]
    \item جمعیت کم (۱ میلیون) → قابل مدیریت
    \item حمایت قاطع شورای امنیت
    \item همکاری مردم محلی
    \item \textbf{زمان‌بندی شفاف خروج}
\end{enumerate}

\textbf{اما:} تیمور ۱ میلیون نفر جمعیت داشت. 
ایران ۸۵ میلیون. مقیاس‌پذیری مدل \lr{UNTAET} 
برای ایران یک چالش بنیادین است. 
همچنین ایران — بر خلاف تیمور — 
یک قدرت منطقه‌ای با غرور ملی قوی است. 
\emphblue{نظارت اجرایی بدون زمان‌بندی 
خروج شفاف، به «اشغال» تبدیل می‌شود.}
\end{lessonlearned}

\begin{warningbox}
\textbf{هشدار درباره‌ی مدل ۴ برای ایران:}
هرگونه حضور «اجرایی» بین‌المللی در ایران 
با حساسیت شدید ملی‌گرایانه مواجه خواهد شد. 
ایرانیان حافظه‌ی تاریخی عمیقی از مداخله‌ی 
خارجی دارند (کودتای ۲۸ مرداد ۱۳۳۲، 
قرارداد ۱۹۱۹ وثوق‌الدوله). \emphred{هر 
عنصر اجرایی باید با دقت فوق‌العاده، شفافیت 
کامل و زمان‌بندی مشخص طراحی شود.}
\end{warningbox}

\sectiondivider

% ============================================================
\section{مدل ۵: مدیریت بین‌المللی مستقیم}
\label{sec:model5}
% ============================================================

\begin{definitionbox}{مدیریت بین‌المللی مستقیم 
(\lr{Full International Administration})}
کنترل کامل یا شبه‌کامل حکمرانی توسط 
نهاد بین‌المللی (معمولاً سازمان ملل یا 
ائتلاف بین‌المللی) برای مدت معین. 
تمام تصمیمات اجرایی، قضایی و تقنینی 
توسط مقام بین‌المللی اتخاذ می‌شود.
\end{definitionbox}

\begin{warningbox}
\textbf{موضع قاطع این کتاب:}
\emphred{مدل ۵ برای ایران نه ممکن است، 
نه مطلوب و نه مشروع.} بررسی این مدل 
صرفاً به‌منظور کامل بودن تحلیل و 
نشان دادن خطرات آن انجام می‌شود.
\end{warningbox}

\subsection{نمونه‌ی تاریخی: عراق (CPA, ۲۰۰۳-۲۰۰۴)}

\begin{casestudy}{مدیریت مستقیم آمریکا در عراق}
پس از سقوط صدام حسین (آوریل ۲۰۰۳)، 
\org{مرجع موقت ائتلاف}%
{Coalition Provisional Authority, CPA} 
به ریاست \person{پل برمر}{L. Paul Bremer} 
مدیریت کامل عراق را بر عهده گرفت.

\vspace{4pt}
\textbf{اقدامات فاجعه‌بار:}
\begin{enumerate}[itemsep=2pt, font=\small]
    \item \textbf{دی‌بعثی‌سازی:} اخراج 
    ۵۰۰,۰۰۰ نفر → بیکاری و خشم
    \item \textbf{انحلال ارتش:} ۳۵۰,۰۰۰ 
    سرباز مسلح بیکار → شورش مسلحانه
    \item \textbf{تصمیم‌گیری بدون مشورت:} 
    ناآشنایی با فرهنگ و جامعه‌ی عراق
    \item \textbf{فساد گسترده:} میلیاردها 
    دلار هدر رفت
\end{enumerate}

\vspace{4pt}
\textbf{نتیجه:} جنگ داخلی، ظهور داعش، 
بیش از ۵۰۰,۰۰۰ کشته، ۲+ تریلیون 
دلار هزینه، و عراقی که هنوز بی‌ثبات است.

\vspace{4pt}
\textbf{درس برای ایران:}
\emphred{مدیریت مستقیم بین‌المللی بر 
کشوری با ۸۵ میلیون جمعیت، تمدن ۳۰۰۰ 
ساله و هویت ملی قوی، فاجعه‌بار خواهد بود. 
این گزینه باید از ابتدا از روی میز 
برداشته شود.}
\end{casestudy}

\sectiondivider

% ============================================================
\section{مدل ۶ (پیشنهادی): مدل ترکیبی-تطبیقی ایران}
\label{sec:model6}
% ============================================================

\begin{keypoint}
\textbf{ایده‌ی محوری مدل ۶:}
هیچ مدل واحدی جواب نمی‌دهد. 
\emphpurple{مدل ترکیبی-تطبیقی} عناصر 
بهترین مدل‌ها (۲، ۳ و ۴) را ترکیب می‌کند 
و آن‌ها را در سه فاز زمانی سازمان‌دهی 
می‌کند — از نظارت سنگین‌تر در آغاز 
به نظارت سبک‌تر در انتها.
\end{keypoint}

\subsection{فازبندی مدل ترکیبی}

\begin{figure}[htbp]
    \centering
    \begin{tikzpicture}[
        phase/.style={
            draw, rounded corners=4pt,
            minimum height=3cm, minimum width=4.2cm,
            align=center, font=\small,
            drop shadow={opacity=0.2}
        }
    ]
    
    % فاز ۱
    \node[phase, fill=RedBG, draw=MainRed] 
        (p1) at (0, 0) {
        \textbf{\large فاز ۱}\\[4pt]
        \textbf{تثبیت}\\[2pt]
        ماه ۱ تا ۶\\[4pt]
        {\footnotesize 
        نظارت اجرایی (مدل ۴)\\
        بر امنیت + حقوق بشر\\
        + انتخابات اولیه}
    };
    
    % فاز ۲
    \node[phase, fill=OrangeBG, draw=MainOrange] 
        (p2) at (5.5, 0) {
        \textbf{\large فاز ۲}\\[4pt]
        \textbf{نهادسازی}\\[2pt]
        ماه ۶ تا ۲۴\\[4pt]
        {\footnotesize 
        نظارت ساختاری (مدل ۳)\\
        بر قانون اساسی + SSR\\
        + عدالت انتقالی}
    };
    
    % فاز ۳
    \node[phase, fill=GreenBG, draw=MainGreen] 
        (p3) at (11, 0) {
        \textbf{\large فاز ۳}\\[4pt]
        \textbf{تحکیم}\\[2pt]
        ماه ۲۴ تا ۶۰\\[4pt]
        {\footnotesize 
        نظارت مشورتی (مدل ۲)\\
        بر عملکرد نهادها\\
        + کاهش تدریجی حضور}
    };
    
    % فلش‌ها
    \draw[-{Stealth}, ultra thick, MainOrange] 
        (p1) -- (p2) 
        node[midway, above, font=\tiny\bfseries] 
        {ارزیابی};
    \draw[-{Stealth}, ultra thick, MainGreen] 
        (p2) -- (p3) 
        node[midway, above, font=\tiny\bfseries] 
        {ارزیابی};
    
    % شدت نظارت
    \draw[ultra thick, MainRed!70, -{Stealth}] 
        (0, -2.5) -- (11, -2.5)
        node[midway, below, font=\small\bfseries] 
        {کاهش تدریجی عمق نظارت};
    
    % مالکیت ملی
    \draw[ultra thick, MainGreen!70, -{Stealth}] 
        (0, -3.3) -- (11, -3.3)
        node[midway, below, font=\small\bfseries] 
        {افزایش تدریجی مالکیت ملی};
    
    \end{tikzpicture}
    \caption{فازبندی مدل ترکیبی-تطبیقی 
    پیشنهادی برای ایران}
    \label{fig:hybrid-model}
\end{figure}

\subsection{جزئیات هر فاز}

\subsubsection{فاز ۱: تثبیت (ماه ۱ تا ۶)}

\begin{table}[htbp]
    \centering
    \caption{جزئیات عملیاتی فاز ۱ (تثبیت)}
    \label{tab:phase1-details}
    \tablefontsize
    \begin{tabularx}{\textwidth}{
        L{3cm} X
    }
        \toprule
        \headerrow
        \textbf{عنصر} & \textbf{توضیح} \\
        \midrule
        
        هدف اصلی &
        جلوگیری از خلأ امنیتی، ثبت نقض 
        حقوق بشر، آماده‌سازی انتخابات اولیه \\
        \altrow
        
        مدل غالب &
        مدل ۴ (نظارت اجرایی) + عناصر مدل ۱ 
        (نظارت انتخاباتی) \\
        
        اختیارات اجرایی &
        تأیید فرماندهان امنیتی، نظارت بر بودجه‌ی 
        دفاعی، حفاظت از زیرساخت‌های حیاتی 
        (نفت، هسته‌ای) \\
        \altrow
        
        نیروی انسانی &
        ۳,۰۰۰-۵,۰۰۰ بین‌المللی + 
        ۱۰,۰۰۰+ ناظر ایرانی \\
        
        بودجه‌ی تخمینی &
        \$۵۰۰M - \$۱B \\
        \altrow
        
        اقدامات کلیدی &
        آزادی زندانیان سیاسی، بازگشایی 
        رسانه‌ها، اعلام تقویم انتخاباتی، 
        ایجاد کمیسیون انتخابات مستقل \\
        
        نقطه‌ی عطف خروج &
        برگزاری موفق انتخابات اولیه 
        (رفراندوم یا مجلس مؤسسان) \\
        
        \bottomrule
    \end{tabularx}
\end{table}

\subsubsection{فاز ۲: نهادسازی (ماه ۶ تا ۲۴)}

\begin{table}[htbp]
    \centering
    \caption{جزئیات عملیاتی فاز ۲ (نهادسازی)}
    \label{tab:phase2-details}
    \tablefontsize
    \begin{tabularx}{\textwidth}{
        L{3cm} X
    }
        \toprule
        \headerrow
        \textbf{عنصر} & \textbf{توضیح} \\
        \midrule
        
        هدف اصلی &
        تدوین قانون اساسی، اصلاح بخش امنیتی، 
        عدالت انتقالی، انتخابات پارلمانی \\
        \altrow
        
        مدل غالب &
        مدل ۳ (نظارت ساختاری) + عناصر مدل ۲ 
        (مشاوره‌ی فنی) \\
        
        اختیارات &
        مشاوره با قدرت نفوذ: توصیه‌های 
        \lr{Venice Commission} + معیارسنجی + 
        مشوق‌های مالی (رفع تحریم مرحله‌ای) \\
        \altrow
        
        نیروی انسانی &
        ۱,۵۰۰-۳,۰۰۰ کارشناس بخشی \\
        
        بودجه‌ی تخمینی &
        \$۸۰۰M - \$۱.۵B \\
        \altrow
        
        اقدامات کلیدی &
        رفراندوم قانون اساسی، تأسیس دادگاه 
        قانون اساسی، آغاز \lr{SSR}، 
        تأسیس کمیسیون حقیقت و آشتی \\
        
        نقطه‌ی عطف خروج &
        برگزاری انتخابات پارلمانی و 
        ریاست‌جمهوری آزاد \\
        
        \bottomrule
    \end{tabularx}
\end{table}

\subsubsection{فاز ۳: تحکیم (ماه ۲۴ تا ۶۰)}

\begin{table}[htbp]
    \centering
    \caption{جزئیات عملیاتی فاز ۳ (تحکیم)}
    \label{tab:phase3-details}
    \tablefontsize
    \begin{tabularx}{\textwidth}{
        L{3cm} X
    }
        \toprule
        \headerrow
        \textbf{عنصر} & \textbf{توضیح} \\
        \midrule
        
        هدف اصلی &
        نهادینه‌سازی دموکراسی، 
        انتقال کامل مسئولیت به نهادهای ملی، 
        کاهش تدریجی حضور بین‌المللی \\
        \altrow
        
        مدل غالب &
        مدل ۲ (نظارت مشورتی) + 
        ارزیابی‌های دوره‌ای مستقل \\
        
        اختیارات &
        مشاوره‌ی درخواست‌محور، ارزیابی 
        سالانه، گزارش‌دهی بین‌المللی \\
        \altrow
        
        نیروی انسانی &
        ۲۰۰-۵۰۰ مشاور بلندمدت \\
        
        بودجه‌ی تخمینی &
        \$۲۰۰-۵۰۰M (کل فاز) \\
        \altrow
        
        اقدامات کلیدی &
        دومین دوره‌ی انتخابات (اولین انتقال 
        مسالمت‌آمیز قدرت)، تکمیل \lr{SSR}، 
        جمع‌بندی عدالت انتقالی \\
        
        نقطه‌ی عطف خروج &
        اولین انتقال مسالمت‌آمیز قدرت + 
        شاخص‌های تحکیم محقق شده \\
        
        \bottomrule
    \end{tabularx}
\end{table}

\subsection{مکانیزم بازخورد و تنظیم مسیر}

\begin{recommendation}
\textbf{اصل انطباق‌پذیری:}
مدل ترکیبی نباید خشک و از‌پیش‌تعیین‌شده 
اجرا شود. در پایان هر فاز (و حتی درون هر فاز) 
باید ارزیابی مستقل انجام شود و بر اساس 
شرایط واقعی تنظیمات لازم اعمال گردد. 
سه سناریوی ممکن در هر نقطه‌ی ارزیابی:

\begin{enumerate}[itemsep=3pt]
    \item \textcolor{MainGreen}{\textbf{پیشرفت مطلوب}} 
    → انتقال به فاز بعد طبق برنامه
    \item \textcolor{MainOrange}{\textbf{پیشرفت جزئی}} 
    → تمدید فاز فعلی با اصلاحات
    \item \textcolor{MainRed}{\textbf{بحران یا بازگشت}} 
    → تقویت عناصر اجرایی و بازگشت 
    موقت به فاز قبل
\end{enumerate}
\end{recommendation}

\begin{figure}[htbp]
    \centering
    \begin{tikzpicture}[
        node distance=2cm,
        decision/.style={
            diamond, draw=MainPurple, fill=PurpleBG,
            text width=2.2cm, align=center, 
            inner sep=1pt, font=\tiny\bfseries,
            aspect=1.5
        },
        process/.style={
            rectangle, draw, rounded corners=3pt,
            minimum height=0.8cm, minimum width=2.5cm,
            align=center, font=\tiny\bfseries
        },
        good/.style={process, fill=GreenBG, draw=MainGreen},
        mid/.style={process, fill=OrangeBG, draw=MainOrange},
        bad/.style={process, fill=RedBG, draw=MainRed},
        arr/.style={-{Stealth[length=2mm]}, thick}
    ]
    
    % شروع
    \node[process, fill=BlueBG, draw=MainBlue] 
        (start) {پایان فاز N};
    
    % ارزیابی
    \node[decision, below=1.5cm of start] 
        (eval) {ارزیابی\\مستقل};
    
    % سه مسیر
    \node[good, below left=1.5cm and 2cm of eval] 
        (next) {انتقال به\\فاز N+1};
    \node[mid, below=1.5cm of eval] 
        (extend) {تمدید فاز N\\با اصلاحات};
    \node[bad, below right=1.5cm and 2cm of eval] 
        (back) {بازگشت به\\فاز N-1};
    
    % فلش‌ها
    \draw[arr, MainGreen] (eval) -- (next)
        node[midway, above left, font=\tiny] 
        {شاخص‌ها محقق};
    \draw[arr, MainOrange] (eval) -- (extend)
        node[midway, right, font=\tiny] 
        {پیشرفت جزئی};
    \draw[arr, MainRed] (eval) -- (back)
        node[midway, above right, font=\tiny] 
        {بحران/بازگشت};
    
    \draw[arr] (start) -- (eval);
    
    \end{tikzpicture}
    \caption{مکانیزم بازخورد و تنظیم مسیر 
    در مدل ترکیبی-تطبیقی}
    \label{fig:feedback-mechanism}
\end{figure}

\subsection{اصول بنیادین مدل ترکیبی}

مدل ششم بر هفت اصل بنیادین استوار است 
که باید در تمام فازها رعایت شوند:

\begin{table}[htbp]
    \centering
    \caption{هفت اصل بنیادین مدل ترکیبی-تطبیقی}
    \label{tab:model6-principles}
    \begin{tabularx}{\textwidth}{
        C{0.6cm} L{3cm} X
    }
        \toprule
        \headerrow
        \textbf{\#} & 
        \textbf{اصل} & 
        \textbf{معنای عملی} \\
        \midrule
        
        ۱ &
        \textbf{مالکیت ملی}
        \newline\lr{\tiny National Ownership} &
        ایرانیان تصمیم‌گیرنده‌ی نهایی هستند. 
        نقش بین‌المللی: همراهی و نظارت، 
        نه مدیریت \\
        \altrow
        
        ۲ &
        \textbf{فازبندی تدریجی}
        \newline\lr{\tiny Phased Approach} &
        شروع با نظارت عمیق‌تر، کاهش تدریجی 
        تا استقلال کامل \\
        
        ۳ &
        \textbf{انطباق‌پذیری}
        \newline\lr{\tiny Adaptability} &
        مدل نباید خشک باشد؛ بازنگری مستمر 
        بر اساس شرایط واقعی \\
        \altrow
        
        ۴ &
        \textbf{فراگیری}
        \newline\lr{\tiny Inclusivity} &
        همه‌ی گروه‌ها (زنان، اقوام، جوانان، 
        مذهبیون، سکولارها) باید نمایندگی شوند \\
        
        ۵ &
        \textbf{شفافیت}
        \newline\lr{\tiny Transparency} &
        تمام تصمیمات، بودجه‌ها و ارزیابی‌ها 
        باید عمومی باشند \\
        \altrow
        
        ۶ &
        \textbf{پاسخگویی دوسویه}
        \newline\lr{\tiny Mutual Accountability} &
        هم نهادهای ایرانی و هم نهادهای 
        بین‌المللی باید پاسخگو باشند \\
        
        ۷ &
        \textbf{خروج شفاف}
        \newline\lr{\tiny Clear Exit} &
        زمان‌بندی و شرایط خروج مأموریت 
        بین‌المللی از ابتدا تعریف شود \\
        
        \bottomrule
    \end{tabularx}
\end{table}

\sectiondivider

% ============================================================
\section{جدول مقایسه‌ای کلان شش مدل}
\label{sec:grand-comparison}
% ============================================================

جدول زیر خلاصه‌ی مقایسه‌ای شش مدل را 
ارائه می‌دهد. به دلیل گستردگی، این جدول 
در صفحه‌ی افقی چاپ شده است.

% ---- جدول بزرگ در صفحه‌ی افقی ----
\begin{landscape}
\begin{table}[htbp]
    \centering
    \caption{جدول مقایسه‌ای کلان شش مدل 
    نظارت بین‌المللی}
    \label{tab:grand-comparison}
    \bigtablefontsize
    \setlength{\tabcolsep}{4pt}
    \begin{tabularx}{\linewidth}{
        L{2.5cm}
        C{2cm} C{2cm} C{2cm} 
        C{2cm} C{2cm} C{2.5cm}
    }
        \toprule
        \headerrow
        \textbf{معیار} & 
        \rotsmall{\textbf{مدل ۱: انتخاباتی}} &
        \rotsmall{\textbf{مدل ۲: مشورتی}} &
        \rotsmall{\textbf{مدل ۳: ساختاری}} &
        \rotsmall{\textbf{مدل ۴: اجرایی}} &
        \rotsmall{\textbf{مدل ۵: مدیریت مستقیم}} &
        \rotsmall{\textbf{مدل ۶: ترکیبی (پیشنهادی)}} \\
        \midrule
        
        عمق نظارت & 
        \rating{1} & \rating{2} & \rating{3} & 
        \rating{4} & \rating{5} & \rating{4} \\
        \altrow
        
        پذیرش داخلی & 
        \rating{5} & \rating{4} & \rating{3} & 
        \rating{2} & \rating{1} & \rating{3} \\
        
        اثربخشی واقعی & 
        \rating{1} & \rating{2} & \rating{4} & 
        \rating{4} & \rating{3} & \rating{5} \\
        \altrow
        
        ریسک شکست & 
        \cellred{\rating{5}} & 
        \cellorange{\rating{4}} & 
        \cellgreen{\rating{2}} & 
        \cellorange{\rating{3}} & 
        \cellred{\rating{5}} & 
        \cellgreen{\rating{2}} \\
        
        هزینه & 
        \$۵-۲۰M & \$۵۰-۲۰۰M & 
        \$۵۰۰M-۲B & \$۱-۳B & 
        \$۳-۱۰B+ & \$۲.۵-۵B \\
        \altrow
        
        نیروی انسانی & 
        ۲۰۰-۲K & ۵۰-۱۵۰ & 
        ۵۰۰-۲K & ۲-۱۰K & 
        ۱۰-۵۰K & ۶-۱۲K \\
        
        مدت زمان & 
        ۱-۳ ماه & ۱-۳ سال & 
        ۳-۷ سال & ۲-۵ سال & 
        ۳-۱۰+ سال & ۵ سال (فازی) \\
        \altrow
        
        نمونه‌ی موفق & 
        غنا ۲۰۱۲ & تونس ۲۰۱۱ & 
        لهستان ۱۹۹۰ & تیمور ۱۹۹۹ & 
        --- & (پیشنهادی) \\
        
        نمونه‌ی شکست & 
        میانمار ۲۰۱۰ & تونس ۲۰۲۱ & 
        مجارستان ۲۰۱۰+ & کوزوو (وابستگی) & 
        عراق ۲۰۰۳ & --- \\
        \altrow
        
        \textbf{تناسب با ایران} & 
        \cellred{ناکافی} & 
        \cellorange{جزئی} & 
        \cellblue{بالا (بخشی)} & 
        \cellorange{پرریسک} & 
        \cellred{نامناسب} & 
        \cellcolor{PurpleBG}\textbf{بهینه} \\
        
        \bottomrule
    \end{tabularx}
    
    \vspace{6pt}
    {\footnotesize
    \rating{1} = خیلی کم \hspace{0.5cm}
    \rating{2} = کم \hspace{0.5cm}
    \rating{3} = متوسط \hspace{0.5cm}
    \rating{4} = بالا \hspace{0.5cm}
    \rating{5} = خیلی بالا
    }
\end{table}
\end{landscape}

\sectiondivider

% ============================================================
\section{نقش نظارت شهروندی و مکمل‌های 
غیررسمی}
\label{sec:citizen-monitoring}
% ============================================================

در کنار نظارت رسمی بین‌المللی، نظارت 
شهروندی و مکمل‌های غیررسمی نقش حیاتی 
دارند — به‌ویژه در کشوری مانند ایران که 
جامعه‌ی مدنی فعال و جمعیت جوان و 
آشنا به فناوری دارد.

\subsection{انواع نظارت شهروندی}

\begin{table}[htbp]
    \centering
    \caption{انواع نظارت شهروندی و 
    ابزارهای مرتبط}
    \label{tab:citizen-monitoring}
    \tablefontsize
    \begin{tabularx}{\textwidth}{
        L{2.5cm} X L{3cm}
    }
        \toprule
        \headerrow
        \textbf{نوع} & 
        \textbf{توضیح} & 
        \textbf{ابزار/نمونه} \\
        \midrule
        
        ناظران داخلی انتخابات &
        شهروندان آموزش‌دیده که در شعب 
        رأی‌گیری حضور دارند &
        \lr{ISFED} (گرجستان)، 
        \lr{NAMFREL} (فیلیپین) \\
        \altrow
        
        روزنامه‌نگاری شهروندی &
        مستندسازی رویدادها توسط 
        شهروندان عادی &
        تلگرام، توییتر، 
        \lr{YouTube} \\
        
        پلتفرم‌های گزارش‌دهی &
        سیستم‌های آنلاین گزارش 
        تخلفات و نقض حقوق &
        \lr{Ushahidi}، 
        \lr{Electionwatch} \\
        \altrow
        
        نظارت اجتماعی &
        پایش عملکرد نهادهای عمومی 
        توسط سازمان‌های مردم‌نهاد &
        \lr{Transparency Intl.}\\
        
        نظارت دیجیتال &
        تحلیل داده‌های باز، تصاویر 
        ماهواره‌ای، هوش مصنوعی &
        \lr{Bellingcat}، 
        \lr{Planet Labs} \\
        
        \bottomrule
    \end{tabularx}
\end{table}

\begin{operationalnote}
\textbf{برای ایران:}
با توجه به سابقه‌ی فعالیت مدنی ایرانیان 
(حتی تحت سرکوب) و مهارت فناورانه‌ی 
نسل جوان، نظارت شهروندی می‌تواند 
مکمل بسیار قوی‌ای برای نظارت رسمی 
بین‌المللی باشد. \emphgreen{باید از همین 
الان برنامه‌ی آموزش ناظران شهروندی 
ایرانی (داخل و دیاسپورا) آغاز شود.}
\end{operationalnote}

\sectiondivider

% ============================================================
\section{جمع‌بندی فصل}
\label{sec:ch3-summary}
% ============================================================

\begin{chaptersummary}

\textbf{آنچه در این فصل آموختیم:}

\begin{enumerate}[
    label=\textcolor{DarkGray}{\bfseries\arabic*.},
    itemsep=4pt
]
    \item \textbf{شش مدل متمایز} نظارت بین‌المللی 
    وجود دارد — از حداقلی (انتخاباتی) تا 
    حداکثری (مدیریت مستقیم).
    
    \item \textbf{مدل ۱ (انتخاباتی)} برای ایران 
    \emphred{ناکافی} است — نظارت بدون عمق 
    ساختاری، ابزار مشروعیت‌بخشی کاذب می‌شود.
    
    \item \textbf{مدل ۲ (مشورتی)} مفید اما 
    \emphred{ناکافی} است — بدون ضمانت اجرا، 
    تجربه‌ی تونس تکرار می‌شود.
    
    \item \textbf{مدل ۳ (ساختاری)} مناسب‌ترین 
    مدل منفرد است اما نیاز به \emphblue{مشوق 
    خارجی قوی} دارد که معادل عضویت EU 
    برای ایران طراحی شود.
    
    \item \textbf{مدل ۴ (اجرایی)} عناصر ضروری 
    دارد اما \emphred{حساسیت ملی‌گرایانه} و 
    \emphred{مقیاس‌ناپذیری} چالش‌های اصلی‌اند.
    
    \item \textbf{مدل ۵ (مدیریت مستقیم)} باید 
    \emphred{قاطعانه رد شود} — تجربه‌ی عراق 
    کافی است.
    
    \item \textbf{مدل ۶ (ترکیبی-تطبیقی)} — 
    پیشنهاد این کتاب — عناصر مدل‌های ۲، ۳ 
    و ۴ را در سه فاز زمانی ترکیب می‌کند: 
    تثبیت → نهادسازی → تحکیم.
    
    \item \textbf{هفت اصل بنیادین} مدل ترکیبی: 
    مالکیت ملی، فازبندی، انطباق‌پذیری، 
    فراگیری، شفافیت، پاسخگویی دوسویه 
    و خروج شفاف.
    
    \item \textbf{نظارت شهروندی} مکمل حیاتی 
    نظارت رسمی است و باید از الان 
    برنامه‌ریزی شود.
\end{enumerate}

\vspace{6pt}
\begin{center}
    \textcolor{MainOrange}{
        \faArrowLeft\hspace{8pt}
        \textbf{فصل بعد: سناریوهای گذار و 
        مدل‌های نظارتی متناظر}
        \hspace{8pt}\faArrowLeft
    }
\end{center}

\end{chaptersummary}

\chapterend
% ╔══════════════════════════════════════════════════════════════════╗
% ║  فصل ۴: سناریوهای گذار و مدل‌های نظارتی متناظر               ║
% ║  شش سناریو + ماتریس تطبیق + درخت تصمیم                       ║
% ╚══════════════════════════════════════════════════════════════════╝

% ---- صفحه‌ی آغازین فصل ----
\chapteropening{۴}
    {سناریوهای گذار و مدل‌های نظارتی متناظر}
    {MainOrange}
    {آینده قابل پیش‌بینی نیست، 
    اما می‌توان برای آن آماده شد. 
    بهترین استراتژی آن است که برای 
    چندین آینده‌ی ممکن هم‌زمان برنامه داشته باشیم.}
    {پیتر شوارتز، بنیان‌گذار سناریوپردازی مدرن}

\chapter{سناریوهای گذار و مدل‌های نظارتی متناظر}
\label{ch:scenarios}
\minitoc

% ---- خلاصه‌ی اجرایی ----
\begin{executivesummary}
هر سناریوی گذار، مدل نظارتی متفاوتی 
می‌طلبد. این فصل \emphorange{شش سناریوی 
محتمل} برای تغییر سیاسی در ایران را تحلیل 
می‌کند و برای هر یک مدل نظارتی مناسب، 
زمان واکنش لازم، ریسک‌های اصلی و 
نمونه‌های تاریخی مشابه را مشخص می‌سازد. 
هدف آن است که بازیگران داخلی و بین‌المللی 
\emphorange{برای هر آینده‌ی ممکن آماده باشند} 
و در لحظه‌ی بحران مجبور به تصمیم‌گیری 
ارتجالی نشوند. سناریوی مطلوب 
(گذار مذاکره‌ای) و سناریوی محتمل‌تر 
(فروپاشی یا انقلاب مردمی) تمایز داده شده 
و مدل ترکیبی-تطبیقی فصل ۳ برای هر 
سناریو تنظیم شده است.
\end{executivesummary}

% ============================================================
\section{چرا سناریوسازی؟ روش‌شناسی و 
محدودیت‌ها}
\label{sec:why-scenarios}
% ============================================================

\begin{definitionbox}{سناریوسازی (\lr{Scenario Planning})}
روشی ساختاریافته برای تصور چندین آینده‌ی 
ممکن و طراحی استراتژی‌های انطباقی برای 
هر یک. سناریوها پیش‌بینی نیستند — ابزار 
آمادگی هستند.
\end{definitionbox}

سناریوسازی برای موضوع این کتاب به سه 
دلیل ضروری است:

\begin{enumerate}[
    label=\textcolor{MainOrange}{\bfseries\arabic*.},
    itemsep=6pt
]
    \item \textbf{عدم قطعیت بالا:} 
    هیچ‌کس نمی‌داند تغییر در ایران دقیقاً 
    کِی و چگونه رخ خواهد داد. ممکن است 
    فردا باشد یا یک دهه‌ی دیگر.
    
    \item \textbf{تفاوت بنیادین مدل‌ها:} 
    همان‌طور که فصل ۳ نشان داد 
    (\seeChapter{ch:approaches})، 
    هر نوع گذار مدل نظارتی متفاوتی 
    می‌طلبد. آمادگی برای فقط یک سناریو 
    خطرناک است.
    
    \item \textbf{زمان واکنش محدود:} 
    در لحظه‌ی فروپاشی یا انقلاب، 
    فرصت طراحی از صفر وجود ندارد. 
    باید برنامه‌های آماده‌باش 
    (\lr{Contingency Plans}) 
    از پیش تدوین شده باشند.
\end{enumerate}

\begin{warningbox}
\textbf{محدودیت‌های سناریوسازی:}
\begin{itemize}[itemsep=2pt]
    \item سناریوها \emphred{پیش‌بینی نیستند} — 
    واقعیت احتمالاً ترکیبی از چند سناریو 
    خواهد بود
    \item ممکن است سناریویی رخ دهد که 
    اصلاً تصور نکرده‌ایم 
    (\lr{Black Swan})
    \item سناریوها ایستا نیستند — ممکن 
    است از یک سناریو به سناریوی دیگر 
    جابه‌جایی رخ دهد
    \item احتمالات ذکرشده تخمینی و ذهنی 
    هستند، نه محاسبات ریاضی
\end{itemize}
\end{warningbox}

\subsection{متغیرهای کلیدی سناریوسازی}

شش سناریوی این فصل بر مبنای تعامل 
چهار متغیر کلیدی ساخته شده‌اند:

\begin{figure}[htbp]
    \centering
    \begin{tikzpicture}[
        var/.style={
            draw=MainOrange, fill=OrangeBG,
            rounded corners=3pt,
            minimum height=1.3cm, minimum width=3.5cm,
            align=center, font=\small\bfseries
        },
        center/.style={
            draw=MainPurple, fill=PurpleBG,
            circle, minimum size=2cm,
            align=center, font=\footnotesize\bfseries
        },
        conn/.style={thick, MainOrange!60}
    ]
    
    \node[center] (c) {نوع\\گذار};
    
    \node[var] (v1) at (90:3.5cm) 
        {سرعت تغییر\\[2pt]
        \footnotesize تدریجی ← → ناگهانی};
    \node[var] (v2) at (0:4cm) 
        {نقش عامل خارجی\\[2pt]
        \footnotesize حداقلی ← → حداکثری};
    \node[var] (v3) at (270:3.5cm) 
        {سطح خشونت\\[2pt]
        \footnotesize مسالمت‌آمیز ← → خشن};
    \node[var] (v4) at (180:4cm) 
        {نقش نظام قدیم\\[2pt]
        \footnotesize حذف ← → مشارکت};
    
    \draw[conn] (c) -- (v1);
    \draw[conn] (c) -- (v2);
    \draw[conn] (c) -- (v3);
    \draw[conn] (c) -- (v4);
    
    \end{tikzpicture}
    \caption{چهار متغیر کلیدی تعیین‌کننده‌ی 
    نوع سناریوی گذار}
    \label{fig:scenario-variables}
\end{figure}

\sectiondivider

% ============================================================
\section{سناریوی A: فروپاشی ناگهانی نظام}
\label{sec:scenario-a}
% ============================================================

\begin{scenariobox}{فروپاشی ناگهانی 
(\lr{Sudden Collapse})}

\begin{tabularx}{\textwidth}{L{3cm} X}
    \textbf{احتمال:} & متوسط (۲۰-۳۰٪) \\
    \textbf{سرعت:} & بسیار بالا (روزها تا هفته‌ها) \\
    \textbf{خشونت:} & متوسط تا بالا \\
    \textbf{نقش نظام قدیم:} & حذف/فرار \\
    \textbf{عامل خارجی:} & واکنشی (نه آغازگر) \\
    \textbf{محرک احتمالی:} & بحران جانشینی رهبری، 
    شورش نظامی، فروپاشی اقتصادی ناگهانی، 
    اعتصاب عمومی فلج‌کننده \\
    \textbf{نمونه‌ی مشابه:} & فروپاشی شوروی ۱۹۹۱، 
    سقوط بن‌علی تونس ۲۰۱۱، 
    سقوط چائوشسکو رومانی ۱۹۸۹ \\
    \textbf{زمان واکنش نظارتی:} & ۴۸-۷۲ ساعت
\end{tabularx}

\end{scenariobox}

\subsection{ویژگی‌ها و پویایی}

در این سناریو، نظام بدون مذاکره و بدون 
برنامه‌ی انتقالی فرو می‌پاشد. ممکن است 
رهبر فوت کند و جانشینی بحرانی شود، 
یا بخشی از نیروهای مسلح از اطاعت سر 
باز زنند، یا شورشی شهری کنترل‌ناپذیر 
شود. ویژگی‌های اصلی:

\begin{itemize}[itemsep=4pt]
    \item \textbf{خلأ قدرت فوری:} 
    هیچ نهاد مشروعی برای مدیریت 
    انتقال وجود ندارد
    \item \textbf{ریسک بالای خشونت:} 
    نیروهای امنیتی ممکن است تجزیه 
    شوند — بخشی مقاومت و بخشی فرار
    \item \textbf{رقابت بر سر قدرت:} 
    گروه‌های مختلف (سپاه، اپوزیسیون، 
    گروه‌های قومی) هم‌زمان ادعای 
    قدرت می‌کنند
    \item \textbf{بحران انسانی:} 
    مهاجرت، کمبود مواد غذایی و دارو، 
    قطع خدمات
    \item \textbf{ریسک دخالت خارجی:} 
    همسایگان و قدرت‌های بزرگ وسوسه‌ی 
    مداخله پیدا می‌کنند
\end{itemize}

\subsection{مدل نظارتی متناسب}

\begin{table}[htbp]
    \centering
    \caption{مدل نظارتی برای سناریوی 
    فروپاشی ناگهانی}
    \label{tab:scenario-a-model}
    \tablefontsize
    \begin{tabularx}{\textwidth}{L{3cm} X}
        \toprule
        \headerrow
        \textbf{عنصر} & \textbf{توضیح} \\
        \midrule
        
        مدل غالب &
        مدل ۴ (اجرایی) تقویت‌شده + 
        عناصر امنیتی مدل ۵ در 
        هفته‌های نخست \\
        \altrow
        
        اولویت فوری &
        جلوگیری از خلأ امنیتی، 
        حفاظت از زیرساخت‌ها 
        (نفت، هسته‌ای، آب)، 
        مدیریت بحران انسانی \\
        
        نیروی انسانی فوری &
        تیم پیشرو (\lr{Advance Team}) 
        ۵۰-۱۰۰ نفر در ۴۸ ساعت، 
        سپس ۵,۰۰۰+ در ۲ هفته \\
        \altrow
        
        مکانیزم تصمیم‌گیری &
        جلسه‌ی اضطراری شورای امنیت 
        → قطعنامه‌ی فوری → 
        انتصاب \lr{SRSG} \\
        
        چالش اصلی &
        سرعت: آیا جامعه‌ی بین‌المللی 
        می‌تواند به اندازه‌ی کافی سریع 
        واکنش نشان دهد؟ \\
        \altrow
        
        پیش‌نیاز حیاتی &
        \emphred{برنامه‌ی آماده‌باش از الان 
        باید تدوین شده باشد} \\
        
        \bottomrule
    \end{tabularx}
\end{table}

\begin{lessonlearned}
\textbf{از تجربه‌ی رومانی (۱۹۸۹):}
سقوط چائوشسکو ظرف ۱۰ روز رخ داد. 
جامعه‌ی بین‌المللی هیچ برنامه‌ای نداشت. 
نتیجه: عناصر نظام قدیم (به رهبری ایلیسکو) 
قدرت را تصاحب کردند و «انقلاب دزدیده شد». 
دموکراسی واقعی سال‌ها به تأخیر افتاد.

\vspace{4pt}
\emphblue{درس: سرعت واکنش حیاتی است. 
بدون برنامه‌ی آماده‌باش از پیش، فروپاشی 
ناگهانی به مصادره‌ی گذار توسط فرصت‌طلبان 
منجر می‌شود.}
\end{lessonlearned}

\begin{keypoint}
\textbf{برنامه‌ی آماده‌باش (\lr{Contingency Plan}) 
باید شامل موارد زیر باشد:}
\begin{enumerate}[itemsep=2pt, font=\small]
    \item فهرست تماس اضطراری 
    (چه کسی با چه کسی تماس می‌گیرد)
    \item متن پیش‌نویس قطعنامه‌ی 
    شورای امنیت
    \item فهرست ناظران آماده‌باش 
    (\lr{Stand-by Roster})
    \item پروتکل امنیتی برای 
    حفاظت از تأسیسات هسته‌ای
    \item کانال‌های ارتباطی با 
    نیروهای داخلی قابل اعتماد
    \item برنامه‌ی رسانه‌ای بحران 
    (مقابله با اطلاعات نادرست)
\end{enumerate}
\end{keypoint}

\sectiondivider

% ============================================================
\section{سناریوی B: گذار مذاکره‌ای}
\label{sec:scenario-b}
% ============================================================

\begin{scenariobox}{گذار مذاکره‌ای 
(\lr{Negotiated Transition})}

\begin{tabularx}{\textwidth}{L{3cm} X}
    \textbf{احتمال:} & پایین تا متوسط (۱۰-۲۰٪) \\
    \textbf{سرعت:} & تدریجی (ماه‌ها تا سال‌ها) \\
    \textbf{خشونت:} & حداقلی \\
    \textbf{نقش نظام قدیم:} & مشارکت فعال \\
    \textbf{عامل خارجی:} & تسهیل‌گر/میانجی \\
    \textbf{محرک احتمالی:} & شکاف عمیق درون نظام 
    بین تندروها و نرم‌روها، فشار اقتصادی 
    غیرقابل تحمل، تهدید خارجی مشترک \\
    \textbf{نمونه‌ی مشابه:} & آفریقای جنوبی ۱۹۹۰-۱۹۹۴، 
    لهستان ۱۹۸۹ (میز گرد)، 
    اسپانیا ۱۹۷۵-۱۹۸۲ \\
    \textbf{زمان واکنش نظارتی:} & هفته‌ها تا ماه‌ها 
    (فرصت برنامه‌ریزی وجود دارد)
\end{tabularx}

\end{scenariobox}

\subsection{ویژگی‌ها و پویایی}

\begin{pullquote}
گذار مذاکره‌ای \emphorange{مطلوب‌ترین} سناریو 
است — کمترین هزینه‌ی انسانی، بالاترین شانس 
موفقیت بلندمدت، بیشترین فرصت برای طراحی 
نظارت. اما \emphorange{محتمل‌ترین} نیست.
\end{pullquote}

در این سناریو، بخشی از نخبگان نظام 
(\lr{softliners}) به این نتیجه می‌رسند 
که ادامه‌ی وضع موجود ناممکن یا 
بسیار پرهزینه است و حاضر می‌شوند 
با اپوزیسیون مذاکره کنند. 
شرایط لازم:

\begin{enumerate}[
    label=\textcolor{MainOrange}{\bfseries\alph*)},
    itemsep=4pt
]
    \item \textbf{شکاف درون نظام:} 
    تندروها (\lr{hardliners}) ضعیف 
    شده باشند یا درون خود تقسیم شوند — 
    مثلاً بحران جانشینی رهبری
    
    \item \textbf{اپوزیسیون سازمان‌یافته:} 
    طرف مقابل مذاکره باید وجود داشته 
    باشد — نه پراکنده و بی‌سازمان
    
    \item \textbf{اعتماد حداقلی:} 
    طرفین باید حداقلی از اعتماد 
    (یا تضمین خارجی جایگزین اعتماد) 
    داشته باشند
    
    \item \textbf{فشار کافی:} 
    فشار مردمی + اقتصادی + بین‌المللی 
    به حدی باشد که مذاکره را 
    جذاب‌تر از سرکوب کند
    
    \item \textbf{میانجی قابل اعتماد:} 
    شخص یا نهادی که هر دو طرف 
    بپذیرند (سازمان ملل؟ یک کشور 
    بی‌طرف؟ یک شخصیت مورد احترام؟)
\end{enumerate}

\subsection{مدل نظارتی متناسب}

\begin{table}[htbp]
    \centering
    \caption{مدل نظارتی برای سناریوی 
    گذار مذاکره‌ای}
    \label{tab:scenario-b-model}
    \tablefontsize
    \begin{tabularx}{\textwidth}{L{3cm} X}
        \toprule
        \headerrow
        \textbf{عنصر} & \textbf{توضیح} \\
        \midrule
        
        مدل غالب &
        مدل ۶ (ترکیبی-تطبیقی) استاندارد — 
        بهترین تناسب \\
        \altrow
        
        نقش بین‌المللی &
        تسهیل‌گری مذاکرات + تضمین 
        اجرای توافقات + نظارت ساختاری \\
        
        اولویت &
        طراحی قواعد بازی، تأمین امنیت 
        مذاکره‌کنندگان، نظارت بر 
        رعایت تعهدات طرفین \\
        \altrow
        
        زمان‌بندی &
        مرحله‌ی مذاکره: ۳-۱۲ ماه، 
        سپس ورود به فاز ۱ مدل ۶ \\
        
        نقش ویژه &
        \lr{SRSG} یا میانجی ویژه 
        به‌عنوان تسهیل‌گر مذاکرات \\
        \altrow
        
        مزیت اصلی &
        \cellgreen{فرصت برنامه‌ریزی: 
        می‌توان نظارت را از قبل 
        طراحی و مستقر کرد} \\
        
        \bottomrule
    \end{tabularx}
\end{table}

\begin{casestudy}{میز گرد لهستان (۱۹۸۹)}
در فوریه-آوریل ۱۹۸۹، دولت کمونیستی 
لهستان و اتحادیه‌ی همبستگی 
(\lr{Solidarity}) به رهبری 
\person{لخ والنسا}{Lech Walesa} 
بر سر میز مذاکره نشستند. نتیجه: 
انتخابات نیمه‌آزاد ژوئن ۱۹۸۹ 
که همبستگی ۹۹ از ۱۰۰ کرسی 
سنا را بُرد.

\vspace{4pt}
\textbf{عوامل موفقیت:}
\begin{enumerate}[itemsep=2pt, font=\small]
    \item فشار اقتصادی غیرقابل تحمل 
    بر رژیم
    \item حمایت پاپ ژان پل دوم 
    (مشروعیت مذهبی-ملی)
    \item اپوزیسیون سازمان‌یافته و 
    دارای رهبر شناخته‌شده
    \item تغییر موضع شوروی 
    (گورباچف دخالت نکرد)
    \item تضمین‌های امنیتی برای 
    نخبگان رژیم قدیم
\end{enumerate}

\vspace{4pt}
\textbf{درس برای ایران:}
\emphblue{گذار مذاکره‌ای در لهستان ممکن شد 
زیرا هم نظام ضعیف شده بود و هم اپوزیسیون 
قوی و متحد بود. در ایران، شرط دوم هنوز 
محقق نشده — اتحاد و سازمان‌دهی اپوزیسیون 
پیش‌شرط حیاتی این سناریو است.}
\end{casestudy}

\sectiondivider

% ============================================================
\section{سناریوی C: انقلاب مردمی}
\label{sec:scenario-c}
% ============================================================

\begin{scenariobox}{انقلاب مردمی 
(\lr{Popular Revolution})}

\begin{tabularx}{\textwidth}{L{3cm} X}
    \textbf{احتمال:} & متوسط (۲۰-۳۰٪) \\
    \textbf{سرعت:} & متوسط (هفته‌ها تا ماه‌ها) \\
    \textbf{خشونت:} & متوسط (بستگی به واکنش رژیم) \\
    \textbf{نقش نظام قدیم:} & سرنگونی/حذف \\
    \textbf{عامل خارجی:} & حمایتی (نه آغازگر) \\
    \textbf{محرک احتمالی:} & اعتراضات گسترده 
    (مشابه ۱۴۰۱ اما بزرگ‌تر و پایدارتر)، 
    شکاف در نیروهای امنیتی \\
    \textbf{نمونه‌ی مشابه:} & تونس ۲۰۱۱ (مثبت)، 
    مصر ۲۰۱۱ (ابتدا مثبت سپس شکست)، 
    لیبی ۲۰۱۱ (منفی) \\
    \textbf{زمان واکنش نظارتی:} & ۱-۴ هفته
\end{tabularx}

\end{scenariobox}

\subsection{ویژگی‌ها و پویایی}

این سناریو نزدیک‌ترین به تجربه‌ی اخیر 
ایران است — خیزش «زن، زندگی، آزادی» 
(۱۴۰۱) الگوی اولیه‌ی آن را نشان داد. 
تفاوت: این بار اعتراضات باید به حدی 
گسترده و پایدار باشند که نظام توان 
سرکوب نداشته باشد.

\begin{table}[htbp]
    \centering
    \caption{شرایط موفقیت و شکست 
    انقلاب مردمی}
    \label{tab:revolution-conditions}
    \begin{tabularx}{\textwidth}{
        C{0.5cm} X X
    }
        \toprule
        \headerrow
        & \textbf{\textcolor{MainGreen}{شرایط موفقیت}} & 
        \textbf{\textcolor{MainRed}{عوامل شکست}} \\
        \midrule
        
        ۱ &
        \cellgreen{مشارکت فراطبقاتی و 
        فراقومی — نه فقط طبقه متوسط} &
        \cellred{محدود ماندن به یک شهر 
        یا یک طبقه} \\
        \altrow
        
        ۲ &
        \cellgreen{حفظ خصلت مسالمت‌آمیز 
        (قدرت اخلاقی)} &
        \cellred{خشونت‌ورزی معترضان → 
        مشروعیت‌بخشی به سرکوب} \\
        
        ۳ &
        \cellgreen{شکاف در نیروهای امنیتی 
        — بخشی از سپاه/ارتش به مردم 
        بپیوندد} &
        \cellred{وحدت نیروهای سرکوبگر} \\
        \altrow
        
        ۴ &
        \cellgreen{رهبری شبکه‌ای 
        (غیرمتمرکز اما هماهنگ)} &
        \cellred{فقدان رهبری یا 
        رهبری تک‌صدایی} \\
        
        ۵ &
        \cellgreen{حمایت بین‌المللی سریع 
        و قاطع} &
        \cellred{سکوت یا تردید 
        جامعه بین‌المللی} \\
        \altrow
        
        ۶ &
        \cellgreen{اعتصاب عمومی 
        (فلج اقتصادی نظام)} &
        \cellred{ادامه‌ی چرخه‌ی اقتصادی 
        → تاب‌آوری نظام} \\
        
        \bottomrule
    \end{tabularx}
\end{table}

\subsection{مدل نظارتی متناسب}

در این سناریو، نظارت بین‌المللی باید 
در دو مرحله عمل کند:

\begin{enumerate}[
    label=\textcolor{MainOrange}{\bfseries 
    مرحله‌ی \arabic*:},
    itemsep=8pt
]
    \item \textbf{حین انقلاب (هفته‌ها):} 
    نظارت حقوق بشری فعال — مستندسازی 
    سرکوب، فشار دیپلماتیک بر رژیم، 
    حفاظت از شهروندان 
    (گزارشگر ویژه + \lr{OHCHR})
    
    \item \textbf{پس از سرنگونی:} 
    ورود فوری به فاز ۱ مدل ۶ 
    (مشابه سناریوی A اما با 
    مشروعیت مردمی بالاتر)
\end{enumerate}

\begin{lessonlearned}
\textbf{مقایسه‌ی تونس و مصر (۲۰۱۱):}

هر دو انقلاب مردمی بودند. اما نتایج 
بسیار متفاوت:

\begin{tabularx}{\textwidth}{L{3cm} X X}
    & \textbf{تونس} & \textbf{مصر} \\
    \midrule
    ارتش & بی‌طرف ماند & 
    «محافظ» شد سپس کودتا کرد \\
    اپوزیسیون & اجماع حداقلی & 
    تضاد شدید (اخوان vs سکولار) \\
    قانون اساسی & مشارکتی و مترقی & 
    یک‌جانبه و فرقه‌ای \\
    نظارت بین‌المللی & مشورتی فعال & 
    حداقلی و منفعل \\
    نتیجه & دموکراسی (تا ۲۰۲۱) & 
    کودتای ۲۰۱۳ \\
\end{tabularx}

\vspace{4pt}
\emphblue{درس: انقلاب مردمی فقط نیمی از 
کار است. نیمه‌ی دوم — نهادسازی — تعیین‌کننده 
است. و اینجاست که نظارت بین‌المللی 
تفاوت می‌سازد.}
\end{lessonlearned}

\sectiondivider

% ============================================================
\section{سناریوی D: تحول از درون نظام}
\label{sec:scenario-d}
% ============================================================

\begin{scenariobox}{تحول از درون نظام 
(\lr{Intra-Regime Reform})}

\begin{tabularx}{\textwidth}{L{3cm} X}
    \textbf{احتمال:} & پایین (۵-۱۰٪) \\
    \textbf{سرعت:} & بسیار تدریجی (سال‌ها) \\
    \textbf{خشونت:} & حداقلی \\
    \textbf{نقش نظام قدیم:} & هدایت‌کننده \\
    \textbf{عامل خارجی:} & فشار تدریجی \\
    \textbf{محرک احتمالی:} & به قدرت رسیدن 
    یک فرد اصلاح‌طلب واقعی پس از 
    مرگ رهبر، محاسبه‌ی عقلایی 
    بخشی از نظام \\
    \textbf{نمونه‌ی مشابه:} & میانمار ۲۰۱۰-۲۰۱۵ 
    (شکست‌خورده)، اتحاد شوروی 
    (گلاسنوست/پرسترویکا — ناخواسته 
    به فروپاشی انجامید) \\
    \textbf{زمان واکنش نظارتی:} & ماه‌ها تا سال‌ها
\end{tabularx}

\end{scenariobox}

\subsection{چرا این سناریو بعید است}

\begin{warningbox}
در ایران، بر خلاف میانمار یا شوروی، 
ساختار نظام به‌گونه‌ای طراحی شده که 
اصلاحات بنیادین را \emphred{ساختاراً مسدود} 
می‌کند:
\begin{itemize}[itemsep=2pt]
    \item شورای نگهبان وتوی مطلق 
    بر هرگونه تغییر قانونی دارد
    \item رهبر فراتر از قانون اساسی 
    قرار دارد
    \item سپاه منافع اقتصادی عظیمی 
    در حفظ وضع موجود دارد
    \item تجربه‌ی شکست اصلاحات 
    خاتمی (۱۳۷۶-۱۳۸۴) نشان داد 
    که «اصلاح از درون» در این 
    ساختار ناممکن است
\end{itemize}
\end{warningbox}

\subsection{اگر رخ دهد: مدل نظارتی}

\begin{table}[htbp]
    \centering
    \caption{مدل نظارتی برای سناریوی 
    تحول درونی}
    \label{tab:scenario-d-model}
    \begin{tabularx}{\textwidth}{L{3cm} X}
        \toprule
        \headerrow
        \textbf{عنصر} & \textbf{توضیح} \\
        \midrule
        مدل غالب & مدل ۲ (مشورتی) با 
        فشار برای حرکت به مدل ۳ \\
        \altrow
        نقش بین‌المللی & فشار دیپلماتیک + 
        مشاوره فنی + مشوق اقتصادی \\
        خطر اصلی & 
        \cellred{گذار نمایشی: اصلاحات 
        سطحی بدون تغییر واقعی قدرت} \\
        \altrow
        معیار سنجش & آیا اصلاحات واقعی‌اند؟ 
        آزادی زندانیان، رسانه آزاد، 
        انتخابات بدون فیلتر \\
        نمونه هشداردهنده & 
        \cellred{میانمار: ارتش «دموکراسی» 
        داد و ۵ سال بعد پس گرفت (کودتای ۲۰۲۱)} \\
        \bottomrule
    \end{tabularx}
\end{table}

\sectiondivider

% ============================================================
\section{سناریوی E: دخالت نظامی خارجی}
\label{sec:scenario-e}
% ============================================================

\begin{scenariobox}{دخالت نظامی خارجی 
(\lr{Foreign Military Intervention})}

\begin{tabularx}{\textwidth}{L{3cm} X}
    \textbf{احتمال:} & بسیار پایین (< ۵٪) \\
    \textbf{سرعت:} & متغیر \\
    \textbf{خشونت:} & بسیار بالا \\
    \textbf{نقش نظام قدیم:} & مقاومت مسلحانه \\
    \textbf{عامل خارجی:} & آغازگر و مجری \\
    \textbf{نمونه‌ی مشابه:} & عراق ۲۰۰۳، 
    افغانستان ۲۰۰۱، لیبی ۲۰۱۱ — 
    \emphred{همه شکست‌خورده} \\
    \textbf{مدل نظارتی:} & مدل ۵ اجباری
\end{tabularx}

\end{scenariobox}

\begin{warningbox}
\textbf{موضع قاطع این کتاب:}

\emphred{دخالت نظامی خارجی در ایران باید 
به هر قیمت اجتناب شود.} دلایل:

\begin{enumerate}[itemsep=3pt, font=\small]
    \item \textbf{ابعاد نظامی:} ایران ۱.۶ 
    میلیون کیلومتر مربع وسعت و ۸۵ 
    میلیون جمعیت دارد — مقایسه شود 
    با عراق (۲۵M) و افغانستان (۳۰M) 
    که هر دو شکست خوردند
    
    \item \textbf{مقاومت ملی:} حتی مخالفان 
    رژیم در برابر مداخله‌ی نظامی 
    خارجی مقاومت خواهند کرد — 
    ناسیونالیسم ایرانی قوی‌ترین 
    نیروی متحدکننده است
    
    \item \textbf{فاجعه‌ی انسانی:} جنگ در 
    ایران می‌تواند میلیون‌ها آواره و 
    صدها هزار کشته ایجاد کند
    
    \item \textbf{بی‌ثباتی جهانی:} 
    قیمت نفت چندبرابر، بحران 
    مهاجرت، گسترش جنگ به منطقه
    
    \item \textbf{تأسیسات هسته‌ای:} 
    خطر آلودگی رادیواکتیو
    
    \item \textbf{مشروعیت صفر:} 
    هر نظامی که پس از دخالت نظامی 
    ایجاد شود، مشروعیت مردمی 
    نخواهد داشت
\end{enumerate}
\end{warningbox}

\sectiondivider

% ============================================================
\section{سناریوی F: بحران ممتد و بی‌ثباتی مزمن}
\label{sec:scenario-f}
% ============================================================

\begin{scenariobox}{بحران ممتد 
(\lr{Prolonged Crisis / Chronic Instability})}

\begin{tabularx}{\textwidth}{L{3cm} X}
    \textbf{احتمال:} & متوسط تا بالا (۲۵-۳۵٪) \\
    \textbf{سرعت:} & بسیار کند / رکود \\
    \textbf{خشونت:} & پراکنده و مزمن \\
    \textbf{نقش نظام قدیم:} & تضعیف‌شده اما 
    حذف‌نشده \\
    \textbf{عامل خارجی:} & خسته و منفعل \\
    \textbf{محرک:} & نه نظام توان سرکوب 
    کامل دارد و نه مردم توان سرنگونی — 
    بن‌بست \\
    \textbf{نمونه‌ی مشابه:} & ونزوئلا (۲۰۱۹-حال)، 
    لبنان (۲۰۱۹-حال)، سودان (۲۰۱۹-۲۰۲۳) \\
    \textbf{زمان واکنش نظارتی:} & نامحدود (مزمن)
\end{tabularx}

\end{scenariobox}

\subsection{چرا این سناریو محتمل‌ترین است}

\begin{keypoint}
بسیاری از تحلیلگران معتقدند که 
\emphorange{محتمل‌ترین آینده‌ی کوتاه‌مدت 
ایران نه سقوط ناگهانی و نه اصلاحات، 
بلکه ادامه‌ی وضعیت بحرانی فعلی} است — 
اعتراضات دوره‌ای، سرکوب، فرسایش تدریجی 
نظام و جامعه، مهاجرت نخبگان و 
زوال اقتصادی.
\end{keypoint}

\begin{table}[htbp]
    \centering
    \caption{ویژگی‌ها و پیامدهای بحران ممتد}
    \label{tab:prolonged-crisis}
    \begin{tabularx}{\textwidth}{L{3cm} X}
        \toprule
        \headerrow
        \textbf{بُعد} & \textbf{توضیح} \\
        \midrule
        اقتصادی & فقر فزاینده، تورم مزمن، 
        فرار سرمایه، اقتصاد زیرزمینی \\
        \altrow
        اجتماعی & مهاجرت نخبگان (\lr{brain drain})، 
        افسردگی اجتماعی، افزایش اعتیاد 
        و جرم \\
        سیاسی & بن‌بست: نظام ناتوان از 
        اصلاح و مردم ناتوان از تغییر \\
        \altrow
        حقوق بشری & سرکوب مزمن، 
        اعدام‌های پراکنده، 
        زندانیان سیاسی \\
        منطقه‌ای & ادامه‌ی ماجراجویی 
        منطقه‌ای برای انحراف افکار عمومی \\
        \altrow
        بین‌المللی & خستگی (\lr{fatigue}) — 
        جهان ایران را فراموش می‌کند \\
        \bottomrule
    \end{tabularx}
\end{table}

\subsection{نقش نظارت بین‌المللی در بحران ممتد}

\begin{recommendation}
حتی اگر گذار فوری رخ ندهد، 
جامعه‌ی بین‌المللی وظایفی دارد:
\begin{enumerate}[itemsep=3pt]
    \item \textbf{نظارت حقوق بشری مستمر:} 
    گزارشگر ویژه + مستندسازی 
    (برای عدالت انتقالی آینده)
    \item \textbf{حمایت از جامعه مدنی:} 
    تأمین مالی رسانه‌های مستقل و 
    سازمان‌های حقوق بشری
    \item \textbf{آماده‌سازی:} 
    تدوین برنامه آماده‌باش، 
    آموزش ناظران ایرانی، 
    شبکه‌سازی
    \item \textbf{فشار دیپلماتیک:} 
    حفظ تحریم‌های هدفمند، 
    محاکمه بین‌المللی ناقضان 
    حقوق بشر
    \item \textbf{مقابله با خستگی:} 
    ایران را در دستور کار 
    بین‌المللی نگه داشتن
\end{enumerate}
\end{recommendation}

\sectiondivider

% ============================================================
\section{ماتریس جامع سناریو-مدل}
\label{sec:scenario-matrix}
% ============================================================

\begin{landscape}
\begin{table}[htbp]
    \centering
    \caption{ماتریس جامع سناریوها و 
    مشخصات نظارتی متناظر}
    \label{tab:scenario-matrix}
    \bigtablefontsize
    \setlength{\tabcolsep}{3pt}
    \begin{tabularx}{\linewidth}{
        L{2.2cm}
        C{2.2cm} C{2.2cm} C{2.2cm} 
        C{2.2cm} C{2.2cm} C{2.2cm}
    }
        \toprule
        \headerrow
        \textbf{معیار} & 
        \rotsmall{\textbf{A: فروپاشی}} &
        \rotsmall{\textbf{B: مذاکره}} &
        \rotsmall{\textbf{C: انقلاب}} &
        \rotsmall{\textbf{D: اصلاح درونی}} &
        \rotsmall{\textbf{E: مداخله خارجی}} &
        \rotsmall{\textbf{F: بحران ممتد}} \\
        \midrule
        
        احتمال & ۲۰-۳۰٪ & ۱۰-۲۰٪ & 
        ۲۰-۳۰٪ & ۵-۱۰٪ & <۵٪ & ۲۵-۳۵٪ \\
        \altrow
        
        مطلوبیت & \cellorange{متوسط} & 
        \cellgreen{بالا} & \cellorange{متوسط} & 
        \cellorange{متوسط} & \cellred{خیلی کم} & 
        \cellred{کم} \\
        
        فوریت نظارت & 
        \rating{5} & \rating{3} & \rating{4} & 
        \rating{2} & \rating{5} & \rating{3} \\
        \altrow
        
        پیچیدگی & 
        \rating{5} & \rating{3} & \rating{4} & 
        \rating{2} & \rating{5} & \rating{4} \\
        
        شانس موفقیت & 
        \rating{2} & \rating{4} & \rating{3} & 
        \rating{2} & \rating{1} & \rating{2} \\
        \altrow
        
        مدل غالب & ۴+۵ → ۶ & ۶ استاندارد & 
        ۴ → ۶ & ۲ → ۳ & ۵ (اجباری) & ۲+HR \\
        
        زمان واکنش & ۴۸ ساعت & هفته‌ها & 
        ۱-۴ هفته & ماه‌ها & فوری & مزمن \\
        \altrow
        
        هزینه نظارت & \$\$\$\$ & \$\$\$ & 
        \$\$\$\$ & \$\$ & \$\$\$\$\$ & \$\$ \\
        
        نیروی انسانی & ۵-۱۰K & ۶-۱۲K & 
        ۵-۱۰K & ۱۰۰-۵۰۰ & ۱۰-۵۰K & ۵۰-۲۰۰ \\
        \altrow
        
        خطر اصلی & سوریه‌ای شدن & 
        مصادره & رادیکالیزم & 
        نمایشی بودن & فاجعه & فرسایش \\
        
        پیش‌نیاز & آماده‌باش & اپوزیسیون متحد & 
        شبکه مدنی & شکاف درون نظام & 
        — & صبر استراتژیک \\
        
        \bottomrule
    \end{tabularx}
\end{table}
\end{landscape}

\sectiondivider

% ============================================================
\section{درخت تصمیم: انتخاب مدل نظارتی 
بر اساس سناریو}
\label{sec:decision-tree}
% ============================================================

\begin{figure}[htbp]
    \centering
    \begin{tikzpicture}[
        decision/.style={
            diamond, draw=MainOrange, fill=OrangeBG,
            text width=2.4cm, align=center,
            inner sep=2pt, font=\tiny\bfseries,
            aspect=1.8
        },
        outcome/.style={
            rectangle, draw, rounded corners=3pt,
            text width=2.8cm, align=center,
            font=\tiny\bfseries, minimum height=0.9cm
        },
        good/.style={outcome, fill=GreenBG, draw=MainGreen},
        mid/.style={outcome, fill=OrangeBG, draw=MainOrange},
        bad/.style={outcome, fill=RedBG, draw=MainRed},
        arr/.style={-{Stealth[length=2mm]}, thick}
    ]
    
    % ریشه
    \node[decision] (root) at (0,0) 
        {نوع تغییر\\چیست؟};
    
    % سطح ۱
    \node[decision] (sudden) at (-5,-3) 
        {خلأ امنیتی\\وجود دارد؟};
    \node[decision] (negot) at (0,-3) 
        {سپاه در\\مذاکره شرکت\\می‌کند؟};
    \node[decision] (reform) at (5,-3) 
        {اصلاحات\\واقعی است؟};
    
    % سطح ۲
    \node[bad] (m45) at (-7,-6) 
        {مدل ۴+۵\\تثبیت فوری};
    \node[mid] (m4) at (-3,-6) 
        {مدل ۶\\فاز ۱ تقویت‌شده};
    \node[good] (m6std) at (-0.5,-6) 
        {مدل ۶\\استاندارد};
    \node[mid] (m6plus) at (2.5,-6) 
        {مدل ۶\\فاز ۱ تقویت‌شده};
    \node[good] (m23) at (4,-6) 
        {مدل ۲→۳\\تدریجی};
    \node[bad] (m2hr) at (7,-6) 
        {مدل ۲+HR\\فشار};
    
    % فلش‌های سطح ۰→۱
    \draw[arr] (root) -- (sudden) 
        node[midway, above left, font=\tiny] 
        {ناگهانی/انقلابی};
    \draw[arr] (root) -- (negot)
        node[midway, right, font=\tiny] 
        {مذاکره‌ای};
    \draw[arr] (root) -- (reform)
        node[midway, above right, font=\tiny] 
        {تدریجی/درونی};
    
    % فلش‌های سطح ۱→۲
    \draw[arr, MainRed] (sudden) -- (m45)
        node[midway, left, font=\tiny] {بله};
    \draw[arr, MainGreen] (sudden) -- (m4)
        node[midway, right, font=\tiny] {خیر};
    \draw[arr, MainGreen] (negot) -- (m6std)
        node[midway, left, font=\tiny] {بله};
    \draw[arr, MainOrange] (negot) -- (m6plus)
        node[midway, right, font=\tiny] {خیر};
    \draw[arr, MainGreen] (reform) -- (m23)
        node[midway, left, font=\tiny] {بله};
    \draw[arr, MainRed] (reform) -- (m2hr)
        node[midway, right, font=\tiny] {خیر};
    
    \end{tikzpicture}
    \caption{درخت تصمیم ساده‌شده: 
    انتخاب مدل نظارتی بر اساس 
    سناریوی گذار}
    \label{fig:decision-tree}
\end{figure}

\begin{operationalnote}
این درخت تصمیم ابزاری \emphgreen{ساده‌شده} 
برای تصمیم‌گیری سریع در لحظه‌ی بحران است. 
در واقعیت، تصمیم‌گیری نیاز به تحلیل 
عمیق‌تر و مشورت گسترده دارد. اما داشتن 
چنین ابزاری از تصمیم‌گیری ارتجالی 
بهتر است.
\end{operationalnote}

\sectiondivider

% ============================================================
\section{جمع‌بندی فصل}
\label{sec:ch4-summary}
% ============================================================

\begin{chaptersummary}

\textbf{آنچه در این فصل آموختیم:}

\begin{enumerate}[
    label=\textcolor{DarkGray}{\bfseries\arabic*.},
    itemsep=4pt
]
    \item \textbf{شش سناریوی محتمل} 
    برای گذار شناسایی شد: فروپاشی 
    ناگهانی، مذاکره، انقلاب مردمی، 
    اصلاح درونی، مداخله خارجی 
    و بحران ممتد.
    
    \item \textbf{مطلوب‌ترین سناریو} 
    گذار مذاکره‌ای است (کمترین هزینه 
    انسانی) اما \textbf{محتمل‌ترین} 
    ادامه‌ی بحران ممتد یا فروپاشی/ 
    انقلاب است.
    
    \item \textbf{هر سناریو مدل نظارتی 
    متفاوتی} می‌طلبد — از نظارت 
    اجرایی فوری (فروپاشی) تا 
    نظارت مشورتی صبورانه 
    (بحران ممتد).
    
    \item \textbf{مداخله‌ی نظامی خارجی} 
    باید قاطعانه رد شود — تمام 
    نمونه‌های تاریخی شکست‌خورده‌اند.
    
    \item \textbf{برنامه‌ی آماده‌باش} 
    حیاتی‌ترین توصیه‌ی این فصل 
    است: منتظر نمانید — از الان 
    برای هر سناریو آماده شوید.
    
    \item \textbf{واقعیت احتمالاً ترکیبی} 
    از چند سناریو خواهد بود — 
    مدل ترکیبی-تطبیقی (فصل ۳) 
    دقیقاً برای همین انعطاف‌پذیری 
    طراحی شده است.
    
    \item \textbf{درخت تصمیم} ابزاری 
    ساده اما مفید برای واکنش سریع 
    در لحظه‌ی بحران ارائه شد — 
    باید از پیش تمرین و بازبینی شود.
    
    \item \textbf{حتی در سناریوی بحران ممتد} 
    (محتمل‌ترین)، جامعه‌ی بین‌المللی 
    وظایف فعالی دارد: نظارت حقوق بشری، 
    حمایت از جامعه‌ی مدنی، آماده‌سازی 
    و مقابله با خستگی بین‌المللی.
\end{enumerate}

\vspace{6pt}
\begin{center}
    \textcolor{MainOrange}{
        \faArrowLeft\hspace{8pt}
        \textbf{فصل بعد: نهادها، بازیگران، 
        سازمان‌ها و نقش هر یک}
        \hspace{8pt}\faArrowLeft
    }
\end{center}

\end{chaptersummary}

\chapterend
% ╔══════════════════════════════════════════════════════════════════╗
% ║  فصل ۵: نهادها، بازیگران، سازمان‌ها و نقش هر یک              ║
% ║  نقشه‌ی جامع بازیگران و تحلیل نقش‌ها                          ║
% ╚══════════════════════════════════════════════════════════════════╝

% ---- صفحه‌ی آغازین فصل ----
\chapteropening{۵}
    {نهادها، بازیگران، سازمان‌ها و نقش هر یک}
    {MainOrange}
    {در سیاست بین‌الملل، هیچ‌کس بی‌طرف نیست. 
    هنر آن است که منافع متعارض بازیگران را 
    به‌گونه‌ای مدیریت کنیم که نتیجه به نفع 
    مردم ایران باشد.}
    {داگ هامرشولد، دبیرکل سابق سازمان ملل}

\chapter{نهادها، بازیگران، سازمان‌ها و نقش هر یک}
\label{ch:actors}
\minitoc

% ---- خلاصه‌ی اجرایی ----
\begin{executivesummary}
گذار دموکراتیک ایران در خلأ رخ نمی‌دهد. 
ده‌ها نهاد بین‌المللی، دولت، سازمان غیردولتی، 
رسانه و بازیگر داخلی هر یک نقش، منافع 
و ظرفیت‌های متفاوتی دارند. این فصل 
\emphorange{نقشه‌ی جامع بازیگران} را ترسیم 
و نقش، اهمیت، محدودیت‌ها و نوع تعامل مطلوب 
با هر یک را تحلیل می‌کند. هدف آن است که 
هم بازیگران ایرانی بدانند از چه کسی 
چه انتظاری داشته باشند، و هم بازیگران 
بین‌المللی بدانند نقش‌شان چیست و 
مرز آن کجاست.
\end{executivesummary}

% ============================================================
\section{نقشه‌ی کلان بازیگران}
\label{sec:actors-map}
% ============================================================

بازیگران مرتبط با فرایند نظارت بر گذار 
ایران را در نُه دسته سازمان‌دهی می‌کنیم:

\begin{figure}[htbp]
    \centering
    \begin{tikzpicture}[
        ring/.style={
            draw=#1, fill=#1!8,
            rounded corners=3pt,
            minimum height=1cm, minimum width=3cm,
            align=center, font=\footnotesize\bfseries
        },
        center/.style={
            draw=MainPurple, fill=PurpleBG,
            circle, minimum size=2.2cm,
            align=center, font=\small\bfseries
        },
        conn/.style={thick, #1!50}
    ]
    
    % مرکز
    \node[center] (c) at (0,0) {گذار\\ایران};
    
    % ۹ دسته
    \node[ring=MainBlue] (un) at (90:4cm) 
        {۱. سازمان ملل};
    \node[ring=MainBlue] (reg) at (50:4cm) 
        {۲. سازمان‌های منطقه‌ای};
    \node[ring=MainOrange] (gov) at (10:4cm) 
        {۳. دولت‌های کلیدی};
    \node[ring=MainGreen] (ngo) at (330:4cm) 
        {۴. \lr{NGO}های بین‌المللی};
    \node[ring=MainRed] (media) at (290:4cm) 
        {۵. رسانه‌ها};
    \node[ring=DarkYellow] (ifi) at (250:4cm) 
        {۶. نهادهای مالی};
    \node[ring=MainPurple] (iran) at (210:4cm) 
        {۷. بازیگران ایرانی};
    \node[ring=DarkGray] (legal) at (170:4cm) 
        {۸. نهادهای حقوقی};
    \node[ring=MediumGray] (acad) at (130:4cm) 
        {۹. آکادمیا و اتاق فکر};
    
    % اتصالات
    \foreach \n/\col in {
        un/MainBlue, reg/MainBlue, gov/MainOrange,
        ngo/MainGreen, media/MainRed, ifi/DarkYellow,
        iran/MainPurple, legal/DarkGray, acad/MediumGray
    } {
        \draw[conn=\col] (c) -- (\n);
    }
    
    \end{tikzpicture}
    \caption{نقشه‌ی کلان نُه دسته‌ی بازیگران 
    مرتبط با نظارت بر گذار ایران}
    \label{fig:actors-map}
\end{figure}

\sectiondivider

% ============================================================
\section{دسته‌ی ۱: سازمان ملل متحد}
\label{sec:actors-un}
% ============================================================

سازمان ملل \emphblue{محوری‌ترین نهاد} 
در هر فرایند نظارت بین‌المللی است. 
اما سازمان ملل یک نهاد واحد نیست — 
مجموعه‌ای پیچیده از ارگان‌ها با 
اختیارات و فرهنگ‌های سازمانی متفاوت.

\begin{table}[htbp]
    \centering
    \caption{نهادهای کلیدی سازمان ملل 
    و نقش هر یک در گذار ایران}
    \label{tab:un-bodies}
    \tablefontsize
    \begin{tabularx}{\textwidth}{
        L{2.2cm} X C{1.8cm} L{2.5cm}
    }
        \toprule
        \headerrow
        \textbf{نهاد} & 
        \textbf{نقش احتمالی} & 
        \textbf{اهمیت} &
        \textbf{چالش اصلی} \\
        \midrule
        
        شورای امنیت &
        صدور قطعنامه، تأسیس مأموریت 
        نظارتی، تعیین مأموریت \lr{SRSG} &
        \cellred{\textbf{حیاتی}} &
        وتوی روسیه و چین \\
        \altrow
        
        مجمع عمومی &
        مشروعیت‌بخشی سیاسی، 
        تصویب قطعنامه‌ی غیرالزامی &
        مهم &
        فاقد قدرت اجرایی \\
        
        دبیرکل / \lr{SRSG} &
        میانجی‌گری، هماهنگی کل فرایند، 
        چهره‌ی عمومی مأموریت &
        \cellred{\textbf{حیاتی}} &
        انتخاب شخص مناسب \\
        \altrow
        
        \lr{UNDP} &
        ظرفیت‌سازی نهادی، طراحی 
        سیستم انتخاباتی، حکمرانی محلی &
        بالا &
        بروکراسی کند \\
        
        \lr{OHCHR} &
        نظارت بر حقوق بشر، مستندسازی 
        نقض‌ها، گزارش‌دهی مستقل &
        بالا &
        فشار سیاسی دولت‌ها \\
        \altrow
        
        \lr{DPPA} &
        تحلیل سیاسی، پیشگیری از خشونت، 
        هشدار زودهنگام &
        بالا &
        محدودیت نیرو \\
        
        \lr{UN Women} &
        تضمین حقوق زنان در قانون اساسی 
        و نهادهای جدید &
        مهم &
        مقاومت فرهنگی \\
        \altrow
        
        \lr{UNHCR} &
        مدیریت بازگشت پناهندگان و 
        آوارگان داخلی &
        مهم (فاز ۲-۳) &
        مقیاس بزرگ \\
        
        \lr{UNICEF} &
        حفاظت از کودکان در دوره‌ی 
        بی‌ثباتی &
        مهم &
        دسترسی \\
        \altrow
        
        \lr{IAEA} &
        نظارت بر تأسیسات هسته‌ای 
        در دوره‌ی گذار &
        \cellred{\textbf{حیاتی}} &
        حساسیت امنیتی \\
        
        \bottomrule
    \end{tabularx}
\end{table}

\subsection{چالش وتو در شورای امنیت}

\begin{warningbox}
\textbf{بزرگ‌ترین مانع نهادی:}
روسیه و چین هر دو روابط استراتژیک 
با جمهوری اسلامی دارند و احتمالاً 
هرگونه قطعنامه‌ی شورای امنیت علیه 
ایران یا برای تأسیس مأموریت نظارتی 
را وتو خواهند کرد.

\vspace{4pt}
\textbf{راه‌حل‌های جایگزین:}
\begin{enumerate}[itemsep=2pt, font=\small]
    \item \textbf{قطعنامه‌ی مجمع عمومی:} 
    غیرالزامی اما مشروعیت‌بخش 
    (\lr{Uniting for Peace})
    \item \textbf{دعوت دولت موقت ایران:} 
    اگر دولت جدید خودش دعوت کند، 
    نیازی به قطعنامه نیست
    \item \textbf{ائتلاف اختیاری:} 
    \lr{Coalition of the Willing} 
    مشابه گروه تماس بالکان
    \item \textbf{مکانیزم EU:} 
    اتحادیه اروپا مستقلاً 
    مأموریت نظارتی اعزام کند
    \item \textbf{فشار بر روسیه و چین:} 
    بسته‌ی تشویقی (حفظ قراردادهای 
    اقتصادی در ایران جدید)
\end{enumerate}
\end{warningbox}

\subsection{انتخاب نماینده‌ی ویژه‌ی دبیرکل}

\begin{keypoint}
انتخاب \lr{SRSG} مناسب یکی از 
تعیین‌کننده‌ترین تصمیمات فرایند است. 
پروفایل ایده‌آل:

\begin{table}[H]
    \tablefontsize
    \begin{tabularx}{\textwidth}{L{3cm} X}
        \toprule
        \headerrow
        \textbf{ویژگی} & \textbf{توضیح} \\
        \midrule
        ملیت & 
        ترجیحاً نه آمریکایی، نه اروپایی 
        غربی، نه روسی/چینی — 
        شخصیتی از جنوب جهانی 
        (آمریکای لاتین، آفریقا، 
        آسیای جنوب‌شرقی) \\
        \altrow
        تجربه & 
        سابقه‌ی میانجی‌گری، مدیریت 
        بحران، رهبری مأموریت UN \\
        زبان & 
        آشنایی با فارسی مزیت بزرگ 
        (اما الزامی نیست) \\
        \altrow
        شخصیت & 
        صبور، شنونده، قاطع اما 
        انعطاف‌پذیر \\
        شناخت منطقه & 
        درک ژئوپلیتیک خاورمیانه \\
        \altrow
        جنسیت & 
        انتخاب یک زن بسیار نمادین 
        خواهد بود (جنبش زن، زندگی، آزادی) \\
        \bottomrule
    \end{tabularx}
\end{table}
\end{keypoint}

\begin{lessonlearned}
\textbf{از تجربه‌ی \person{سرجیو ویئرا دملو}
{Sergio Vieira de Mello} در تیمور شرقی:}
دملو (برزیلی) به‌عنوان \lr{SRSG} در تیمور 
شرقی (۱۹۹۹-۲۰۰۲) یکی از موفق‌ترین 
مدیریت‌های گذار را رهبری کرد. 
رمز موفقیتش: \emphblue{شنیدن صدای مردم 
محلی + قاطعیت در تصمیمات امنیتی + 
شفافیت + برنامه‌ی مشخص خروج.} 
متأسفانه او در حمله به دفتر UN 
در بغداد (۲۰۰۳) کشته شد — 
یادآوری تلخ ریسک‌هایی که 
ناظران بین‌المللی با آن مواجه‌اند.
\end{lessonlearned}

\sectiondivider

% ============================================================
\section{دسته‌ی ۲: سازمان‌های منطقه‌ای}
\label{sec:actors-regional}
% ============================================================

\begin{table}[htbp]
    \centering
    \caption{سازمان‌های منطقه‌ای و 
    نقش احتمالی در گذار ایران}
    \label{tab:regional-orgs}
    \tablefontsize
    \begin{tabularx}{\textwidth}{
        L{2cm} X C{1.5cm} L{2.5cm}
    }
        \toprule
        \headerrow
        \textbf{سازمان} & 
        \textbf{نقش احتمالی} & 
        \textbf{اعتبار} &
        \textbf{محدودیت} \\
        \midrule
        
        اتحادیه اروپا &
        نظارت انتخاباتی (\lr{EU EOM})، 
        حمایت مالی، مشاوره حقوقی، 
        رفع تحریم‌ها مشروط &
        \starrating{4} &
        کندی تصمیم‌گیری، سیاست 
        داخلی اعضا \\
        \altrow
        
        شورای اروپا &
        \lr{Venice Commission} 
        برای قانون اساسی، 
        استانداردهای حقوق بشر &
        \starrating{5} &
        ایران عضو نیست 
        (اما مشاوره ممکن) \\
        
        \lr{OSCE} &
        نظارت انتخاباتی 
        (\lr{ODIHR})، 
        رسانه آزاد &
        \starrating{5} &
        ایران عضو نیست \\
        \altrow
        
        اتحادیه عرب &
        مشروعیت منطقه‌ای، 
        کاهش تنش‌های 
        ایران-عرب &
        \starrating{2} &
        ضعف ساختاری، 
        تعارض منافع \\
        
        \lr{OIC} &
        مشروعیت اسلامی، 
        پل فرهنگی &
        \starrating{2} &
        سیاسی‌زدگی، 
        ناکارآمدی \\
        \altrow
        
        \lr{SCO} &
        مدیریت نقش 
        روسیه و چین &
        \starrating{1} &
        ایران تازه عضو شده، 
        سازمان اقتدارگرامحور \\
        
        \bottomrule
    \end{tabularx}
\end{table}

\begin{recommendation}
\textbf{استراتژی پیشنهادی برای سازمان‌های منطقه‌ای:}
\begin{itemize}[itemsep=3pt]
    \item \textbf{EU:} شریک اصلی — 
    از ظرفیت‌های \lr{EU EOM}، 
    \lr{Venice Commission} و 
    ابزارهای مالی بهره ببرید
    \item \textbf{OSCE:} حتی بدون 
    عضویت ایران، می‌تواند 
    استانداردها و ناظران 
    ارائه دهد
    \item \textbf{اتحادیه عرب و OIC:} 
    نقش نمادین — برای کاهش 
    روایت «غرب علیه ایران»
    \item \textbf{SCO:} مدیریت کنید 
    نه نادیده بگیرید — کانال 
    ارتباط با روسیه و چین
\end{itemize}
\end{recommendation}

\sectiondivider

% ============================================================
\section{دسته‌ی ۳: دولت‌های کلیدی}
\label{sec:actors-states}
% ============================================================

دولت‌ها مهم‌ترین بازیگران واقعی 
(نه رسمی) هستند. هر دولت منافع 
خاص خود را دارد و نقشش باید 
با درک این منافع مدیریت شود.

\begin{landscape}
\begin{table}[htbp]
    \centering
    \caption{دولت‌های کلیدی: نقش، منافع، 
    ریسک و اهرم}
    \label{tab:key-states}
    \bigtablefontsize
    \setlength{\tabcolsep}{3pt}
    \begin{tabularx}{\linewidth}{
        L{1.8cm} X X L{2cm} L{2cm}
    }
        \toprule
        \headerrow
        \textbf{دولت} & 
        \textbf{منافع اصلی} & 
        \textbf{نقش احتمالی} &
        \textbf{ریسک/محدودیت} &
        \textbf{اهرم ایران} \\
        \midrule
        
        آمریکا &
        هسته‌ای، امنیت اسرائیل، 
        نفوذ منطقه‌ای، حقوق بشر &
        بزرگ‌ترین اهرم فشار و 
        حمایت مالی، رفع تحریم &
        بی‌اعتمادی تاریخی 
        ایرانیان، ابزاری شدن 
        دموکراسی &
        تحریم‌ها، 
        دارایی‌های بلوکه \\
        \altrow
        
        آلمان/فرانسه &
        ثبات منطقه، مهاجرت، 
        تجارت، هسته‌ای &
        میانجی قابل اعتمادتر، 
        حمایت مالی-فنی EU &
        کندی تصمیم‌گیری، 
        وابستگی انرژی &
        دیپلماسی، 
        بازار \\
        
        بریتانیا &
        نفوذ پسااستعماری، 
        حقوق بین‌الملل &
        تجربه حقوقی، 
        BBC فارسی &
        سابقه‌ی استعماری 
        در ذهنیت ایرانیان &
        حقوق بین‌الملل، 
        رسانه \\
        \altrow
        
        ژاپن/کره‌جنوبی &
        ثبات انرژی، 
        تجارت &
        کمک مالی بی‌طرف، 
        تجربه بازسازی &
        نفوذ سیاسی محدود &
        کم‌ریسک‌ترین 
        کمک‌کننده \\
        
        ترکیه &
        مرز مشترک، 
        رقابت منطقه‌ای، 
        کُردها &
        همسایگی، فهم فرهنگی نسبی، 
        مدیریت مرز &
        تعارض منافع، 
        اردوغان &
        مرز، تجارت، 
        کُردها \\
        \altrow
        
        روسیه &
        حفظ نفوذ، فروش سلاح، 
        ضد آمریکایی‌گری &
        \cellred{اسپویلر بالقوه} — 
        وتو در شورای امنیت &
        مانع اصلی در UN &
        باید مدیریت شود \\
        
        چین &
        نفت، قرارداد ۲۵ ساله، 
        جاده ابریشم &
        \cellred{اسپویلر بالقوه} — 
        وتو + نفوذ اقتصادی &
        منافع اقتصادی قوی‌تر 
        از ایدئولوژی &
        تضمین ادامه 
        قراردادها \\
        \altrow
        
        هند &
        بندر چابهار، انرژی، 
        توازن منطقه‌ای &
        بازیگر متوازن، 
        میانجی احتمالی &
        محافظه‌کاری ذاتی &
        چابهار، 
        روابط تاریخی \\
        
        عربستان/امارات &
        پایان تهدید ایران، 
        ثبات خلیج فارس &
        مالی و منطقه‌ای، 
        مشروعیت عربی &
        \cellred{تلاش برای 
        نفوذ بیش از حد} &
        اقتصادی، 
        منطقه‌ای \\
        \altrow
        
        اسرائیل &
        هسته‌ای، امنیت، 
        پایان تهدید &
        اطلاعاتی، فنی 
        (هسته‌ای) &
        \cellred{حضور علنی 
        سمّی برای مشروعیت} &
        پشت‌پرده، 
        نه علنی \\
        
        \bottomrule
    \end{tabularx}
\end{table}
\end{landscape}

\begin{keypoint}
\textbf{اصل طلایی مدیریت دولت‌ها:}
\begin{itemize}[itemsep=3pt]
    \item \emphpurple{هیچ دولتی 
    بی‌طرف نیست} — همه منافع دارند
    \item \emphpurple{اسپویلرها را 
    مدیریت کنید نه نادیده بگیرید} — 
    روسیه و چین را با بسته‌ی 
    تشویقی همراه کنید
    \item \emphpurple{تنوع شرکا} — 
    وابستگی به یک دولت خطرناک است
    \item \emphpurple{حضور علنی 
    برخی بازیگران سمّی است} — 
    نقش اسرائیل باید پشت‌پرده باشد
    \item \emphpurple{ایرانیان حساس‌اند} — 
    هرگونه نقش آمریکایی باید با 
    ظرافت مدیریت شود
\end{itemize}
\end{keypoint}

\sectiondivider

% ============================================================
\section{دسته‌ی ۴: سازمان‌های غیردولتی بین‌المللی}
\label{sec:actors-ngos}
% ============================================================

\lr{NGO}ها نقش‌های تخصصی و اغلب 
غیرقابل جایگزین در فرایند نظارت دارند. 
اما برخی از آن‌ها در ایران 
حساسیت‌برانگیز هستند.

\begin{table}[htbp]
    \centering
    \caption{\lr{NGO}های بین‌المللی کلیدی 
    و نقش هر یک}
    \label{tab:ngos}
    \tablefontsize
    \begin{tabularx}{\textwidth}{
        L{2cm} X C{1.5cm} C{1.5cm}
    }
        \toprule
        \headerrow
        \textbf{سازمان} & 
        \textbf{تخصص و نقش} & 
        \textbf{اعتبار} &
        \textbf{حساسیت ایران} \\
        \midrule
        
        \lr{Carter Center} &
        نظارت انتخاباتی مستقل و 
        معتبر — ۳۹ کشور &
        \starrating{5} &
        \cellgreen{کم} \\
        \altrow
        
        \lr{ICG} &
        تحلیل بحران، هشدار 
        زودهنگام، توصیه سیاستی &
        \starrating{4} &
        \cellgreen{کم} \\
        
        \lr{HRW} &
        مستندسازی نقض حقوق بشر، 
        فشار عمومی &
        \starrating{4} &
        \cellorange{متوسط} \\
        \altrow
        
        \lr{Amnesty Intl.} &
        مشابه \lr{HRW} — 
        تمرکز بر زندانیان &
        \starrating{4} &
        \cellorange{متوسط} \\
        
        \lr{Transparency Intl.} &
        ضد فساد — نظارت بر 
        شفافیت دوره گذار &
        \starrating{4} &
        \cellgreen{کم} \\
        \altrow
        
        \lr{Intl. IDEA} &
        طراحی نظام انتخاباتی، 
        دموکراسی‌سنجی &
        \starrating{4} &
        \cellgreen{کم} \\
        
        \lr{IFES} &
        زیرساخت فنی انتخابات، 
        ثبت رأی‌دهندگان &
        \starrating{4} &
        \cellgreen{کم} \\
        \altrow
        
        \lr{ICTJ} &
        عدالت انتقالی، طراحی 
        کمیسیون حقیقت &
        \starrating{5} &
        \cellgreen{کم} \\
        
        \lr{NED} &
        حمایت مالی از نهادهای 
        دموکراتیک &
        \starrating{3} &
        \cellred{بالا — تأمین 
        مالی دولت آمریکا} \\
        \altrow
        
        \lr{OSF (Soros)} &
        جامعه باز، حمایت مالی، 
        شبکه‌سازی &
        \starrating{3} &
        \cellred{بالا — 
        تئوری توطئه} \\
        
        \lr{NDI / IRI} &
        آموزش احزاب و 
        نهادهای سیاسی &
        \starrating{3} &
        \cellred{بالا — وابسته 
        به احزاب آمریکایی} \\
        
        \bottomrule
    \end{tabularx}
\end{table}

\begin{warningbox}
\textbf{مدیریت حساسیت \lr{NGO}ها:}
سازمان‌هایی مانند \lr{NED}، \lr{OSF} و 
\lr{NDI/IRI} در ایران (و حتی در 
بخش‌هایی از اپوزیسیون) 
\emphred{حساسیت‌برانگیز} هستند — 
به دلیل ارتباط با دولت آمریکا یا 
روایت‌های توطئه‌ای.

\textbf{راه‌حل:}
\begin{itemize}[itemsep=2pt]
    \item نقش عملیاتی آن‌ها 
    \textbf{پشت‌صحنه} باشد
    \item تأمین مالی از طریق 
    \textbf{صندوق‌های چندجانبه} 
    (نه مستقیم) کانالیزه شود
    \item از \textbf{نهادهای کم‌حساسیت‌تر} 
    (مانند \lr{Carter Center}، \lr{ICTJ}، 
    \lr{IDEA}) به‌عنوان چهره‌ی 
    عمومی استفاده شود
\end{itemize}
\end{warningbox}

\sectiondivider

% ============================================================
\section{دسته‌ی ۵: رسانه‌ها}
\label{sec:actors-media}
% ============================================================

رسانه‌ها در دوره‌ی گذار سه نقش حیاتی دارند: 
\textbf{اطلاع‌رسانی}، \textbf{نظارت شهروندی} 
و \textbf{شکل‌دهی روایت عمومی}.

\begin{table}[htbp]
    \centering
    \caption{رسانه‌های کلیدی و نقش‌شان 
    در گذار}
    \label{tab:media-actors}
    \tablefontsize
    \begin{tabularx}{\textwidth}{
        L{2.2cm} X L{3cm}
    }
        \toprule
        \headerrow
        \textbf{رسانه} & 
        \textbf{نقش و نفوذ} & 
        \textbf{ملاحظات} \\
        \midrule
        
        \lr{BBC} فارسی &
        بالاترین نفوذ در میان 
        رسانه‌های بین‌المللی فارسی‌زبان، 
        اعتماد نسبی مخاطبان &
        باید استقلال تحریریه حفظ شود — 
        ابزار دولت بریتانیا نباشد \\
        \altrow
        
        صدای آمریکا / فردا &
        پوشش خبری + تحلیل، 
        دسترسی گسترده &
        سابقه‌ی دولتی — 
        اعتماد کمتر از \lr{BBC} \\
        
        \lr{DW} فارسی &
        اطلاع‌رسانی متوازن، 
        اعتماد نسبی &
        منابع محدودتر \\
        \altrow
        
        ایران‌اینترنشنال &
        پربیننده‌ترین رسانه 
        فارسی‌زبان ماهواره‌ای &
        تأمین مالی عربستان → 
        چالش استقلال \\
        
        رسانه‌های مستقل 
        ایرانی آنلاین &
        \lr{IranWire}، \lr{Iranians.com} 
        و ده‌ها رسانه‌ی کوچک &
        پراکنده اما اصیل — 
        نیاز به حمایت مالی \\
        \altrow
        
        شبکه‌های اجتماعی &
        تلگرام، اینستاگرام، 
        توییتر/\lr{X} — 
        ابزار بسیج و نظارت &
        اطلاعات نادرست، 
        نیاز به سواد رسانه‌ای \\
        
        \bottomrule
    \end{tabularx}
\end{table}

\begin{operationalnote}
\textbf{اقدام فوری پیشنهادی:}
تأسیس یک \emphgreen{«هاب اطلاعاتی گذار»} 
(\lr{Transition Information Hub}) — 
پلتفرم آنلاین چندزبانه (فارسی + کردی + 
ترکی + عربی + بلوچی + انگلیسی) 
برای انتشار اطلاعات موثق درباره‌ی 
فرایند گذار، مقابله با اطلاعات نادرست 
و ارائه‌ی آموزش مدنی. مدیریت مشترک 
توسط نهاد بین‌المللی + رسانه‌های 
مستقل ایرانی.
\end{operationalnote}

\sectiondivider

% ============================================================
\section{دسته‌ی ۶: نهادهای مالی بین‌المللی}
\label{sec:actors-ifi}
% ============================================================

\begin{table}[htbp]
    \centering
    \caption{نهادهای مالی بین‌المللی و 
    نقش‌شان}
    \label{tab:ifis}
    \tablefontsize
    \begin{tabularx}{\textwidth}{
        L{2cm} X L{3cm}
    }
        \toprule
        \headerrow
        \textbf{نهاد} & 
        \textbf{نقش} & 
        \textbf{زمان ورود} \\
        \midrule
        
        \lr{IMF} &
        تثبیت اقتصاد کلان، مشاوره 
        ارزی و مالی، وام اضطراری &
        فاز ۱ — فوری \\
        \altrow
        
        \lr{World Bank} &
        بازسازی زیرساخت، 
        توسعه نهادی، 
        کاهش فقر &
        فاز ۲ — میان‌مدت \\
        
        \lr{EBRD} &
        حمایت از بخش خصوصی، 
        اصلاحات اقتصادی &
        فاز ۲-۳ \\
        \altrow
        
        \lr{ADB} &
        زیرساخت منطقه‌ای، 
        انرژی، حمل‌ونقل &
        فاز ۲-۳ \\
        
        \lr{AIIB} &
        زیرساخت — 
        کانال ارتباط با چین &
        فاز ۲-۳ \\
        
        \bottomrule
    \end{tabularx}
\end{table}

\sectiondivider

% ============================================================
\section{دسته‌ی ۷: بازیگران ایرانی}
\label{sec:actors-iranian}
% ============================================================

\begin{keypoint}
مهم‌ترین دسته‌ی بازیگران، خود ایرانیان 
هستند. اصل «مالکیت ملی» یعنی ایرانیان 
باید در مرکز فرایند باشند — 
نه حاشیه‌ی آن.
\end{keypoint}

\begin{table}[htbp]
    \centering
    \caption{بازیگران ایرانی کلیدی}
    \label{tab:iranian-actors}
    \tablefontsize
    \begin{tabularx}{\textwidth}{
        L{2.5cm} X L{2.5cm}
    }
        \toprule
        \headerrow
        \textbf{بازیگر} & 
        \textbf{نقش در گذار} & 
        \textbf{چالش اصلی} \\
        \midrule
        
        جامعه مدنی داخل &
        پایه‌ی اصلی نظارت شهروندی، 
        مشارکت در نهادهای گذار &
        سرکوب‌شدگی، 
        ضعف سازمانی \\
        \altrow
        
        جنبش زنان &
        پیشران اصلی تغییر 
        (زن، زندگی، آزادی)، 
        تضمین حقوق در قانون اساسی &
        مقاومت سنتی، 
        نیاز به نهادسازی \\
        
        جنبش‌های قومی &
        نمایندگی تنوع، مذاکره 
        خودمختاری/فدرالیسم &
        ریسک تجزیه‌طلبی، 
        مسلح بودن برخی \\
        \altrow
        
        جنبش کارگری &
        نمایندگی طبقاتی، 
        اعتصاب به‌عنوان اهرم &
        سرکوب‌شدگی شدید \\
        
        اپوزیسیون سازمان‌یافته &
        طرف مذاکره، طراحی 
        نهادهای جدید &
        پراکندگی شدید، 
        تضادهای جناحی \\
        \altrow
        
        دیاسپورا &
        منابع مالی-فنی-دیپلماتیک، 
        لابی بین‌المللی &
        فاصله از داخل، 
        تضاد نسلی \\
        
        روحانیت مستقل &
        مشروعیت‌زدایی از ولایت فقیه، 
        پل به جامعه سنتی &
        ضعف نهادی، 
        تعارض درونی \\
        \altrow
        
        نخبگان فکری &
        روایت‌سازی، طراحی 
        گفتمان دموکراتیک &
        تبعید/زندان/سکوت \\
        
        \bottomrule
    \end{tabularx}
\end{table}

\begin{casestudy}{نقش جامعه مدنی تونسی 
در گذار}
چهار سازمان مدنی تونسی در ۲۰۱۳ 
به‌عنوان «چهارتای گفت‌وگوی ملی» 
(\lr{Tunisian National Dialogue Quartet}) 
میانجی‌گری بحران سیاسی را بر عهده گرفتند 
و مانع از تکرار سناریوی مصر شدند. 
در ۲۰۱۵ جایزه نوبل صلح دریافت کردند.

\vspace{4pt}
\emphblue{درس: جامعه مدنی قوی می‌تواند 
در لحظات بحرانی نقش نجات‌بخش ایفا کند. 
تقویت جامعه مدنی ایران — حتی در شرایط 
سرکوب — سرمایه‌گذاری حیاتی برای 
آینده است.}
\end{casestudy}

\sectiondivider

% ============================================================
\section{نقشه‌ی تعامل: چه کسی با چه کسی 
چگونه کار می‌کند}
\label{sec:interaction-map}
% ============================================================

\begin{table}[htbp]
    \centering
    \caption{نوع تعامل مطلوب با هر 
    دسته‌ی بازیگران}
    \label{tab:interaction-types}
    \tablefontsize
    \begin{tabularx}{\textwidth}{
        L{2.5cm} L{2.5cm} X
    }
        \toprule
        \headerrow
        \textbf{دسته} & 
        \textbf{نوع تعامل} & 
        \textbf{ابزار اصلی} \\
        \midrule
        
        نهادهای UN &
        رسمی-دیپلماتیک &
        قطعنامه، نشست، 
        تفاهم‌نامه فنی \\
        \altrow
        
        دولت‌های کلیدی &
        دیپلماسی دو/چندجانبه &
        مذاکره، بسته تشویقی، 
        فشار/اهرم \\
        
        \lr{NGO}ها &
        قرارداد پروژه‌ای &
        \lr{MOU}، تأمین مالی 
        از صندوق مشترک \\
        \altrow
        
        رسانه‌ها &
        شفافیت + دسترسی &
        نشست خبری، اسناد عمومی، 
        هاب اطلاعاتی \\
        
        نهادهای مالی &
        برنامه‌ی مشترک &
        وام مشروط، 
        کمک فنی \\
        \altrow
        
        بازیگران ایرانی &
        مشارکت ساختاریافته &
        شورای مشورتی ملی، 
        انتخابات، رفراندوم \\
        
        آکادمیا &
        سفارش تحقیق + مشاوره &
        گرنت، کنفرانس، 
        بررسی همتا \\
        
        \bottomrule
    \end{tabularx}
\end{table}

\sectiondivider

% ============================================================
\section{جمع‌بندی فصل}
\label{sec:ch5-summary}
% ============================================================

\begin{chaptersummary}

\textbf{آنچه در این فصل آموختیم:}

\begin{enumerate}[
    label=\textcolor{DarkGray}{\bfseries\arabic*.},
    itemsep=4pt
]
    \item \textbf{نُه دسته‌ی بازیگر} شناسایی و 
    تحلیل شد: از سازمان ملل تا جامعه مدنی 
    ایرانی.
    
    \item \textbf{سازمان ملل محوری‌ترین نهاد} 
    است اما چالش وتوی روسیه و چین باید 
    با راه‌حل‌های جایگزین مدیریت شود.
    
    \item \textbf{انتخاب \lr{SRSG} مناسب} 
    یکی از تعیین‌کننده‌ترین تصمیمات است — 
    ترجیحاً شخصیتی از جنوب جهانی.
    
    \item \textbf{هیچ دولتی بی‌طرف نیست} — 
    مدیریت منافع متعارض دولت‌ها کلید 
    موفقیت است. اسپویلرها (روسیه، چین) 
    باید مدیریت شوند نه نادیده گرفته.
    
    \item \textbf{برخی \lr{NGO}ها حساسیت‌برانگیز}اند — 
    از نهادهای کم‌حساسیت‌تر به‌عنوان 
    چهره‌ی عمومی استفاده شود.
    
    \item \textbf{رسانه‌ها اکسیژن دموکراسی}اند — 
    تأسیس هاب اطلاعاتی گذار توصیه شد.
    
    \item \textbf{بازیگران ایرانی در مرکز فرایند} 
    باید باشند — نه حاشیه. جامعه مدنی، 
    جنبش زنان، اقوام و دیاسپورا 
    همه نقش‌های حیاتی دارند.
    
    \item \textbf{تنوع شرکا} — وابستگی به 
    یک دولت یا نهاد واحد خطرناک است.
\end{enumerate}

\vspace{6pt}
\begin{center}
    \textcolor{MainGreen}{
        \faArrowLeft\hspace{8pt}
        \textbf{فصل بعد: تضمین‌های موفقیت 
        و پیش‌شرط‌های ساختاری}
        \hspace{8pt}\faArrowLeft
    }
\end{center}

\end{chaptersummary}

\chapterend
% ╔══════════════════════════════════════════════════════════════════╗
% ║  فصل ۶: تضمین‌های موفقیت و پیش‌شرط‌های ساختاری               ║
% ║  شش حوزه‌ی تضمین + ماتریس پیش‌شرط‌ها                         ║
% ╚══════════════════════════════════════════════════════════════════╝

% ---- صفحه‌ی آغازین فصل ----
\chapteropening{۶}
    {تضمین‌های موفقیت و پیش‌شرط‌های ساختاری}
    {MainGreen}
    {دموکراسی بدون پیش‌شرط ممکن نیست. 
    اما پیش‌شرط‌ها بهانه‌ای برای تأخیر نیستند — 
    ابزاری برای آمادگی هستند.}
    {آمارتیا سن، اقتصاددان و فیلسوف هندی، برنده نوبل}

\chapter{تضمین‌های موفقیت و پیش‌شرط‌های ساختاری}
\label{ch:guarantees}
\minitoc

% ---- خلاصه‌ی اجرایی ----
\begin{executivesummary}
نظارت بین‌المللی در خلأ عمل نمی‌کند. 
موفقیت آن مستلزم فراهم بودن مجموعه‌ای 
از \emphgreen{پیش‌شرط‌ها و تضمین‌ها} در 
شش حوزه است: سیاسی، امنیتی، حقوقی، 
اقتصادی، اجتماعی و نهادی. این فصل 
هر حوزه را تشریح می‌کند، پیش‌شرط‌ها 
را اولویت‌بندی می‌کند (حیاتی / مهم / مطلوب) 
و مسئول تأمین هر یک را مشخص می‌سازد. 
هدف آن است که بازیگران بدانند 
\emphgreen{قبل از آغاز، حین اجرا و 
برای تداوم} نظارت چه چیزهایی باید 
فراهم باشد.
\end{executivesummary}

% ============================================================
\section{چارچوب شش‌گانه‌ی تضمین‌ها}
\label{sec:guarantee-framework}
% ============================================================

\begin{figure}[htbp]
    \centering
    \begin{tikzpicture}[
        pillar/.style={
            draw=#1, fill=#1!10,
            rounded corners=3pt,
            minimum height=3cm, minimum width=2.2cm,
            align=center, font=\footnotesize\bfseries
        },
        base/.style={
            draw=DarkGray, fill=VeryLightGray,
            minimum height=0.8cm, minimum width=15cm,
            align=center, font=\small\bfseries
        },
        roof/.style={
            draw=MainPurple, fill=PurpleBG,
            minimum height=1cm, minimum width=15cm,
            align=center, font=\normalsize\bfseries
        }
    ]
    
    % پایه
    \node[base] (foundation) at (0, 0) 
        {مالکیت ملی + اراده‌ی مردم ایران};
    
    % ستون‌ها
    \node[pillar=MainBlue] (pol) at (-6.2, 2.5) 
        {سیاسی\\[4pt]\faUsers\\[4pt]
        \tiny اجماع\\حمایت\\قواعد بازی};
    
    \node[pillar=MainRed] (sec) at (-3.7, 2.5) 
        {امنیتی\\[4pt]\faShieldAlt\\[4pt]
        \tiny سپاه\\DDR\\مرزها};
    
    \node[pillar=MainOrange] (leg) at (-1.2, 2.5) 
        {حقوقی\\[4pt]\faGavel\\[4pt]
        \tiny SOMA\\قانون\\معاهدات};
    
    \node[pillar=MainGreen] (eco) at (1.3, 2.5) 
        {اقتصادی\\[4pt]\faChartLine\\[4pt]
        \tiny تحریم\\مارشال\\صندوق};
    
    \node[pillar=DarkYellow] (soc) at (3.8, 2.5) 
        {اجتماعی\\[4pt]\faHeart\\[4pt]
        \tiny زنان\\اقوام\\آشتی};
    
    \node[pillar=MainPurple] (inst) at (6.3, 2.5) 
        {نهادی\\[4pt]\faUniversity\\[4pt]
        \tiny انتخابات\\قضا\\رسانه};
    
    % سقف
    \node[roof] (success) at (0, 4.8) 
        {\faCheckDouble\hspace{8pt}
        گذار دموکراتیک موفق و پایدار
        \hspace{8pt}\faCheckDouble};
    
    \end{tikzpicture}
    \caption{شش ستون تضمین موفقیت 
    گذار دموکراتیک}
    \label{fig:six-pillars}
\end{figure}

\sectiondivider

% ============================================================
\section{تضمین‌های سیاسی}
\label{sec:political-guarantees}
% ============================================================

\subsection{اجماع حداقلی نیروهای سیاسی}

\begin{definitionbox}{اجماع حداقلی 
(\lr{Minimum Consensus})}
توافق طرف‌های اصلی (نه لزوماً همه) 
بر «قواعد بازی» — یعنی فرایند گذار، 
نقش نظارت بین‌المللی و اصول 
غیرقابل مذاکره (مانند حقوق بشر و 
دموکراسی) — حتی اگر بر محتوای 
سیاسی (نوع نظام آینده) 
اختلاف داشته باشند.
\end{definitionbox}

\begin{table}[htbp]
    \centering
    \caption{عناصر اجماع حداقلی سیاسی}
    \label{tab:political-consensus}
    \begin{tabularx}{\textwidth}{
        C{0.6cm} X L{3cm}
    }
        \toprule
        \headerrow
        \textbf{\#} & 
        \textbf{عنصر اجماع} & 
        \textbf{نمونه‌ی تاریخی} \\
        \midrule
        
        ۱ &
        پذیرش اصل انتخابات آزاد 
        به‌عنوان تنها مکانیزم 
        مشروع تعیین قدرت &
        لهستان: میز گرد ۱۹۸۹ \\
        \altrow
        
        ۲ &
        توافق بر فرایند تدوین 
        قانون اساسی جدید 
        (چه کسی، چگونه، چه وقت) &
        آفریقای جنوبی: \lr{CODESA} \\
        
        ۳ &
        تعهد به عدم خشونت و 
        حل اختلاف از طریق 
        مکانیزم‌های مدنی &
        اسپانیا: پیمان مونکلوا \\
        \altrow
        
        ۴ &
        پذیرش نظارت بین‌المللی 
        به‌عنوان تسهیل‌کننده 
        (نه مداخله‌گر) &
        تیمور شرقی \\
        
        ۵ &
        توافق بر خطوط قرمز مشترک: 
        عدم انتقام‌جویی کور، 
        عدم تجزیه‌طلبی خشونت‌آمیز &
        آفریقای جنوبی \\
        \altrow
        
        ۶ &
        قبول حق مشارکت همه‌ی 
        گروه‌ها (حتی اعضای 
        سابق نظام قدیم که 
        مرتکب جنایت نشده‌اند) &
        اسپانیا: عفو مشروط \\
        
        \bottomrule
    \end{tabularx}
\end{table}

\begin{warningbox}
\textbf{هشدار: اجماع = اتفاق‌نظر کامل نیست}

اجماع حداقلی به معنای توافق بر 
\emphred{قواعد بازی} است، نه بر 
\emphred{نتیجه‌ی بازی}. طرف‌ها 
می‌توانند (و باید) بر سر نوع نظام، 
سیاست‌های اقتصادی و ارزش‌های 
فرهنگی رقابت کنند — به شرط آنکه 
بر قواعد رقابت دموکراتیک توافق 
داشته باشند.

\vspace{4pt}
\textbf{بزرگ‌ترین چالش ایران:} 
اپوزیسیون ایران هنوز حتی بر 
این قواعد حداقلی اجماع ندارد. 
بخشی خواهان جمهوری‌اند، بخشی 
مشروطه، بخشی فدرالیسم و بخشی 
مخالف هر سه. \emphred{بدون حل 
این معضل، هیچ مدل نظارتی 
کار نخواهد کرد.}
\end{warningbox}

\subsection{حمایت بین‌المللی}

\begin{table}[htbp]
    \centering
    \caption{سطوح حمایت بین‌المللی مورد نیاز}
    \label{tab:international-support}
    \begin{tabularx}{\textwidth}{
        L{2.5cm} X C{2cm}
    }
        \toprule
        \headerrow
        \textbf{سطح} & 
        \textbf{محتوا} & 
        \textbf{اولویت} \\
        \midrule
        
        حداقلی &
        عدم مخالفت فعال — 
        روسیه و چین مانع نشوند 
        (حداقل رأی ممتنع در 
        شورای امنیت) &
        \cellred{\textbf{حیاتی}} \\
        \altrow
        
        مطلوب &
        حمایت فعال اکثریت 
        اعضای شورای امنیت + 
        مجمع عمومی &
        مهم \\
        
        ایده‌آل &
        اجماع جهانی + بسته‌ی 
        جامع حمایتی (مالی، 
        فنی، دیپلماتیک) &
        مطلوب \\
        
        \bottomrule
    \end{tabularx}
\end{table}

\sectiondivider

% ============================================================
\section{تضمین‌های امنیتی}
\label{sec:security-guarantees}
% ============================================================

\begin{keypoint}
تضمین‌های امنیتی \emphgreen{مهم‌ترین و 
حساس‌ترین} حوزه هستند. بدون امنیت، 
هیچ فرایند سیاسی ممکن نیست. و مهم‌ترین 
متغیر امنیتی ایران: \emphgreen{سپاه پاسداران}.
\end{keypoint}

\subsection{مدیریت سپاه پاسداران: 
سه رویکرد ممکن}

\begin{table}[htbp]
    \centering
    \caption{سه رویکرد مدیریت سپاه: 
    مقایسه‌ی مزایا و ریسک‌ها}
    \label{tab:irgc-approaches}
    \tablefontsize
    \begin{tabularx}{\textwidth}{
        L{2cm} X X L{2cm}
    }
        \toprule
        \headerrow
        \textbf{رویکرد} & 
        \textbf{مزایا} & 
        \textbf{ریسک‌ها} &
        \textbf{نمونه} \\
        \midrule
        
        \textbf{انحلال کامل}
        \newline\lr{\tiny Full Dissolution} &
        \cellgreen{حذف تهدید، 
        نمادین قوی، پاسخ به 
        خواست مردم} &
        \cellred{۵۰۰K+ نیروی مسلح 
        بیکار و خشمگین، خلأ 
        امنیتی، شورش مسلحانه} &
        عراق ۲۰۰۳ \newline \emphred{فاجعه} \\
        \altrow
        
        \textbf{حفظ با اصلاح}
        \newline\lr{\tiny Reform \& Retain} &
        \cellgreen{ثبات، حفظ ظرفیت 
        دفاعی، کاهش مقاومت} &
        \cellred{خطر کودتا (مصر)، 
        حفظ نفوذ اقتصادی-سیاسی، 
        عدم اعتماد مردم} &
        مصر ۲۰۱۱ \newline \emphred{کودتا} \\
        
        \textbf{بازسازی تدریجی}
        \newline\lr{\tiny Gradual Restructuring} &
        \cellgreen{تعادل بین امنیت و 
        اصلاح، ادغام تدریجی 
        در ارتش حرفه‌ای، 
        جداسازی اقتصاد} &
        \cellorange{زمان‌بر (۱۰-۱۵ سال)، 
        نیاز به نظارت مستمر، 
        مقاومت فرماندهان} &
        اندونزی ۱۹۹۸ \newline \emphgreen{نسبتاً موفق} \\
        
        \bottomrule
    \end{tabularx}
\end{table}

\begin{recommendation}
\textbf{رویکرد پیشنهادی: بازسازی تدریجی 
(مدل اندونزی)}

\begin{enumerate}[itemsep=4pt]
    \item \textbf{فاز ۱ (ماه ۱-۶):} 
    برکناری فرماندهان ارشد دخیل 
    در جنایات + تعلیق فعالیت‌های 
    سیاسی و اقتصادی سپاه + 
    نظارت بین‌المللی بر تأسیسات 
    حساس
    
    \item \textbf{فاز ۲ (ماه ۶-۲۴):} 
    جداسازی بازوی اقتصادی سپاه 
    (واگذاری شرکت‌ها) + ادغام 
    نیروهای پایین‌رتبه در ارتش 
    ملی + آغاز فرایند حسابرسی 
    (\lr{vetting})
    
    \item \textbf{فاز ۳ (سال ۲-۵):} 
    ادغام کامل در ساختار دفاعی 
    حرفه‌ای + حذف سازمان 
    اطلاعات مستقل + نظارت 
    غیرنظامی بر ارتش
    
    \item \textbf{فاز ۴ (سال ۵-۱۵):} 
    حرفه‌ای‌سازی کامل ارتش + 
    آموزش نسل جدید نظامیان 
    در ارزش‌های دموکراتیک
\end{enumerate}
\end{recommendation}

\begin{lessonlearned}
\textbf{از تجربه‌ی اندونزی (اصلاح \lr{TNI}، ۱۹۹۸-۲۰۱۴):}

پس از سقوط سوهارتو (۱۹۹۸)، ارتش 
اندونزی (\lr{TNI}) — مشابه سپاه ایران — 
هم نظامی، هم اقتصادی و هم سیاسی بود. 
اصلاحات:
\begin{enumerate}[itemsep=2pt, font=\small]
    \item حذف نمایندگی نظامی از مجلس (فوری)
    \item جداسازی پلیس از ارتش (سال اول)
    \item واگذاری کسب‌وکارهای نظامی (تدریجی — ۱۰ سال)
    \item حذف ساختار سرزمینی نظامی (تدریجی)
    \item آموزش حقوق بشر به نظامیان (مستمر)
\end{enumerate}

\textbf{نتیجه:} اندونزی امروز بزرگ‌ترین 
دموکراسی مسلمان جهان است و ارتش آن 
تحت نظارت غیرنظامی قرار دارد.

\vspace{4pt}
\emphblue{درس: بازسازی تدریجی ممکن 
است اما نیاز به صبر، منابع و 
نظارت مستمر بین‌المللی دارد.}
\end{lessonlearned}

\subsection{امنیت مرزی و منطقه‌ای}

\begin{table}[htbp]
    \centering
    \caption{اولویت‌های امنیت مرزی 
    در دوره‌ی گذار}
    \label{tab:border-security}
    \begin{tabularx}{\textwidth}{
        L{2.5cm} X C{1.5cm}
    }
        \toprule
        \headerrow
        \textbf{مرز/منطقه} & 
        \textbf{تهدید اصلی} & 
        \textbf{اولویت} \\
        \midrule
        
        غرب (عراق) &
        نفوذ شبه‌نظامیان حشد الشعبی، 
        قاچاق سلاح &
        \cellred{\textbf{بسیار بالا}} \\
        \altrow
        
        شرق (افغانستان) &
        نفوذ طالبان/داعش، 
        مواد مخدر، مهاجرت &
        \cellred{\textbf{بسیار بالا}} \\
        
        جنوب‌شرق (پاکستان) &
        گروه‌های مسلح بلوچ، 
        قاچاق &
        \cellorange{بالا} \\
        \altrow
        
        شمال‌غرب (ترکیه) &
        مسئله‌ی کُرد، 
        تنش مرزی &
        \cellorange{بالا} \\
        
        شمال (آذربایجان) &
        تنش‌های قومی، 
        نفوذ پان‌ترکیسم &
        \cellorange{متوسط} \\
        \altrow
        
        جنوب (خلیج فارس) &
        تنگه‌ی هرمز، 
        جزایر مورد اختلاف &
        \cellred{\textbf{بسیار بالا}} \\
        
        تأسیسات هسته‌ای &
        سرقت مواد، خرابکاری، 
        دسترسی غیرمجاز &
        \cellred{\textbf{حیاتی}} \\
        
        \bottomrule
    \end{tabularx}
\end{table}

\begin{warningbox}
\textbf{هشدار ویژه: تأسیسات هسته‌ای}

ایران حداقل ۱۰ تأسیسات هسته‌ای شناخته‌شده 
دارد (نطنز، فردو، اصفهان، اراک، بوشهر...). 
در هرگونه سناریوی فروپاشی یا بی‌ثباتی:
\begin{itemize}[itemsep=2pt]
    \item \emphred{حفاظت از مواد شکافت‌پذیر} 
    اولویت مطلق است
    \item \lr{IAEA} باید \emphred{فوراً} 
    دسترسی کامل داشته باشد
    \item نیروی حفاظتی مشترک 
    (بین‌المللی + ایرانی) تشکیل شود
    \item هرگونه تلاش برای انتقال 
    مواد هسته‌ای باید جرم‌انگاری شود
\end{itemize}
\end{warningbox}

\sectiondivider

% ============================================================
\section{تضمین‌های حقوقی}
\label{sec:legal-guarantees}
% ============================================================

\begin{table}[htbp]
    \centering
    \caption{تضمین‌های حقوقی مورد نیاز}
    \label{tab:legal-guarantees}
    \tablefontsize
    \begin{tabularx}{\textwidth}{
        L{3cm} X C{1.5cm} C{1.5cm}
    }
        \toprule
        \headerrow
        \textbf{تضمین حقوقی} & 
        \textbf{محتوا} & 
        \textbf{اولویت} &
        \textbf{زمان} \\
        \midrule
        
        \bilingual{توافق‌نامه وضعیت مأموریت}{SOMA} &
        وضعیت حقوقی، مصونیت‌ها و 
        امتیازات ناظران بین‌المللی &
        \cellred{حیاتی} &
        هفته‌ی ۱ \\
        \altrow
        
        چارچوب حقوقی موقت &
        قانون اساسی موقت یا اعلامیه‌ی 
        اصول دوره‌ی گذار &
        \cellred{حیاتی} &
        ماه ۱ \\
        
        قانون انتخابات موقت &
        قواعد ثبت‌نام، نامزدی، رأی‌گیری 
        و شمارش برای انتخابات اولیه &
        \cellred{حیاتی} &
        ماه ۲-۳ \\
        \altrow
        
        قانون احزاب موقت &
        حق تأسیس و فعالیت احزاب، 
        شفافیت مالی &
        مهم &
        ماه ۳-۶ \\
        
        قانون رسانه‌ی موقت &
        آزادی مطبوعات و رسانه، 
        تنظیم‌گری مستقل &
        مهم &
        ماه ۱-۳ \\
        \altrow
        
        مکانیزم عدالت انتقالی &
        قانون تأسیس کمیسیون حقیقت، 
        حدود عفو، حقوق قربانیان &
        مهم &
        ماه ۶-۱۲ \\
        
        الحاق به معاهدات &
        \lr{ICCPR, ICESCR, CAT, CEDAW, 
        CRC, Rome Statute} &
        مطلوب &
        سال ۱-۲ \\
        
        \bottomrule
    \end{tabularx}
\end{table}

\begin{casestudy}{قانون اساسی موقت 
آفریقای جنوبی (۱۹۹۳)}
آفریقای جنوبی قبل از انتخابات ۱۹۹۴ 
یک «قانون اساسی موقت» تدوین کرد 
که ۳۴ اصل غیرقابل مذاکره 
(\lr{Constitutional Principles}) 
تعیین می‌کرد. هر قانون اساسی 
نهایی باید با این اصول سازگار 
می‌بود. دادگاه قانون اساسی 
(\lr{Constitutional Court}) 
ناظر رعایت این اصول بود.

\vspace{4pt}
\emphblue{درس برای ایران: تدوین 
«اعلامیه‌ی اصول گذار» قبل از 
قانون اساسی نهایی — شامل اصول 
غیرقابل مذاکره مانند حقوق بشر، 
تفکیک دین و دولت، برابری جنسیتی 
و حقوق اقلیت‌ها.}
\end{casestudy}

\sectiondivider

% ============================================================
\section{تضمین‌های اقتصادی}
\label{sec:economic-guarantees}
% ============================================================

\begin{keypoint}
\emphgreen{اقتصاد ناموفق = دموکراسی ناموفق.}
اگر مردم در ماه‌های نخست گذار شاهد 
بهبود ملموس اقتصادی نباشند، سرخوردگی 
سیاسی و نوستالژی نظام قدیم رشد می‌کند. 
تجربه‌ی روسیه دهه‌ی ۱۹۹۰ هشدار روشنی است.
\end{keypoint}

\begin{table}[htbp]
    \centering
    \caption{بسته‌ی تضمین‌های اقتصادی}
    \label{tab:economic-guarantees}
    \tablefontsize
    \begin{tabularx}{\textwidth}{
        L{2.5cm} X C{1.5cm} L{2.5cm}
    }
        \toprule
        \headerrow
        \textbf{تضمین} & 
        \textbf{محتوا} & 
        \textbf{اولویت} &
        \textbf{مسئول} \\
        \midrule
        
        رفع تحریم‌ها &
        مرحله‌ای و مشروط به 
        پیشرفت گذار — نه 
        یکباره و نه معطل &
        \cellred{حیاتی} &
        آمریکا + EU \\
        \altrow
        
        بسته‌ی حمایت فوری &
        کمک غذایی و دارویی، 
        تثبیت نرخ ارز، 
        وام اضطراری \lr{IMF} &
        \cellred{حیاتی} &
        \lr{IMF} + دولت‌ها \\
        
        آزادسازی دارایی‌ها &
        دسترسی به دارایی‌های 
        بلوکه‌شده‌ی ایران 
        (تخمین: \$۱۰۰-۱۵۰B) &
        \cellred{حیاتی} &
        آمریکا + EU \\
        \altrow
        
        صندوق امانی نفت &
        مدیریت شفاف درآمد نفت 
        در دوره‌ی گذار — 
        جلوگیری از غارت &
        مهم &
        \lr{UN} + دولت موقت \\
        
        برنامه ضد فساد &
        ردیابی و بازپس‌گیری 
        دارایی‌های غارت‌شده 
        توسط سران نظام قدیم &
        مهم &
        \lr{TI} + نهاد ملی \\
        \altrow
        
        بازسازی زیرساخت &
        آب، برق، حمل‌ونقل، 
        ارتباطات &
        مهم (فاز ۲) &
        \lr{World Bank} + بخش خصوصی \\
        
        حمایت از اشتغال &
        برنامه‌ی اضطراری 
        اشتغال‌زایی — 
        به‌ویژه جوانان &
        مهم &
        \lr{ILO} + دولت موقت \\
        
        \bottomrule
    \end{tabularx}
\end{table}

\begin{lessonlearned}
\textbf{از تجربه‌ی روسیه (شوک‌درمانی دهه‌ی ۱۹۹۰):}

روسیه پس از فروپاشی شوروی 
«شوک‌درمانی» اقتصادی را پیش گرفت: 
آزادسازی ناگهانی قیمت‌ها + خصوصی‌سازی 
سریع. نتیجه:
\begin{itemize}[itemsep=2pt, font=\small]
    \item تورم ۲,۵۰۰٪ در ۱۹۹۲
    \item ظهور الیگارشی (دزدیدن دارایی‌های ملی)
    \item فقر ۴۰٪ جمعیت
    \item سرخوردگی از دموکراسی → ظهور پوتین
\end{itemize}

\emphblue{درس: اصلاحات اقتصادی باید 
تدریجی، عادلانه و با شبکه‌ی 
حمایت اجتماعی باشد. 
«شوک‌درمانی» برای ایران فاجعه‌بار 
خواهد بود.}
\end{lessonlearned}

\sectiondivider

% ============================================================
\section{تضمین‌های اجتماعی-فرهنگی}
\label{sec:social-guarantees}
% ============================================================

\subsection{مشارکت زنان: نه امتیاز بلکه حق}

\begin{statsbox}
\begin{center}
    {\statisticfont حداقل ۳۰\%}\\[4pt]
    {\small مشارکت زنان در تمام نهادهای 
    دوره‌ی گذار — از مجلس مؤسسان 
    تا کمیسیون انتخابات}\\[10pt]
    {\small هدف بلندمدت:}\\[2pt]
    {\statisticfont ۵۰\%}\\[4pt]
    {\small برابری کامل در نمایندگی}
\end{center}
\end{statsbox}

\begin{table}[htbp]
    \centering
    \caption{مکانیزم‌های تضمین مشارکت زنان}
    \label{tab:women-participation}
    \begin{tabularx}{\textwidth}{
        L{3cm} X L{2.5cm}
    }
        \toprule
        \headerrow
        \textbf{مکانیزم} & 
        \textbf{توضیح} & 
        \textbf{نمونه‌ی موفق} \\
        \midrule
        
        سهمیه‌ی قانونی &
        حداقل ۳۰٪ کرسی‌ها 
        برای زنان در مجلس مؤسسان 
        و نهادهای گذار &
        رواندا (۶۱٪!), تونس \\
        \altrow
        
        \lr{UN SCR 1325} &
        اجرای قطعنامه‌ی زنان، صلح 
        و امنیت — مشارکت زنان 
        در مذاکرات صلح/گذار &
        کلمبیا \\
        
        فهرست زیپ &
        فهرست‌های انتخاباتی 
        متناوب زن-مرد &
        بولیوی, اکوادور \\
        \altrow
        
        ممیزی جنسیتی &
        ارزیابی تأثیر جنسیتی 
        هر قانون و سیاست &
        سوئد, کانادا \\
        
        \bottomrule
    \end{tabularx}
\end{table}

\subsection{حقوق اقوام و مدل‌های 
خودمختاری}

\begin{table}[htbp]
    \centering
    \caption{مدل‌های مدیریت تنوع قومی}
    \label{tab:ethnic-models}
    \tablefontsize
    \begin{tabularx}{\textwidth}{
        L{2.5cm} X X C{1.5cm}
    }
        \toprule
        \headerrow
        \textbf{مدل} & 
        \textbf{مزایا} & 
        \textbf{ریسک‌ها} &
        \textbf{نمونه} \\
        \midrule
        
        دولت متمرکز با حقوق فرهنگی &
        \cellgreen{وحدت ملی, سادگی} &
        \cellred{سرکوب تنوع, نارضایتی} &
        فرانسه \\
        \altrow
        
        خودمختاری فرهنگی &
        \cellgreen{حفظ هویت, انعطاف} &
        \cellorange{ناکافی برای اقوام 
        بزرگ با تمرکز جغرافیایی} &
        استونی \\
        
        فدرالیسم جغرافیایی &
        \cellgreen{خودگردانی, نمایندگی} &
        \cellorange{خطر تجزیه, رقابت 
        منابع} &
        آلمان, هند \\
        \altrow
        
        فدرالیسم قومی &
        \cellgreen{نمایندگی مستقیم اقوام} &
        \cellred{قومیت‌سازی سیاست, 
        تجزیه (اتیوپی)} &
        اتیوپی (مشکل‌دار) \\
        
        ترکیبی (پیشنهادی) &
        \cellgreen{خودمختاری استانی + 
        حقوق فرهنگی + تضمین 
        اقلیت‌ها در مرکز} &
        \cellorange{پیچیدگی طراحی} &
        اسپانیا \\
        
        \bottomrule
    \end{tabularx}
\end{table}

\begin{recommendation}
\textbf{مدل پیشنهادی برای ایران: 
ترکیبی-اسپانیایی}

\begin{enumerate}[itemsep=3pt]
    \item \textbf{تقسیمات استانی} 
    (نه قومی) با اختیارات گسترده‌ی 
    خودگردانی (آموزش, فرهنگ, 
    زبان محلی, عمران)
    \item \textbf{زبان‌های رسمی منطقه‌ای} 
    در کنار فارسی به‌عنوان زبان مشترک
    \item \textbf{تضمین نمایندگی} 
    اقوام در نهادهای مرکزی 
    (مجلس دوم/سنا)
    \item \textbf{خطوط قرمز:} تمامیت 
    ارضی غیرقابل مذاکره، 
    خودمختاری ≠ استقلال
\end{enumerate}
\end{recommendation}

\subsection{عدالت انتقالی و آشتی ملی}

\begin{table}[htbp]
    \centering
    \caption{ابزارهای عدالت انتقالی 
    و اولویت‌بندی}
    \label{tab:transitional-justice}
    \begin{tabularx}{\textwidth}{
        L{2.5cm} X C{1.5cm} C{1.5cm}
    }
        \toprule
        \headerrow
        \textbf{ابزار} & 
        \textbf{توضیح} & 
        \textbf{اولویت} &
        \textbf{زمان} \\
        \midrule
        
        کمیسیون حقیقت &
        ثبت شهادات قربانیان, 
        مستندسازی جامع نقض‌ها &
        \cellred{حیاتی} &
        ماه ۶-۱۲ \\
        \altrow
        
        محاکمه‌ی مسئولان &
        محاکمه‌ی عاملان جنایات 
        بزرگ (اعدام‌ها, شکنجه, 
        کشتار ۶۷) &
        \cellred{حیاتی} &
        سال ۱-۳ \\
        
        غرامت به قربانیان &
        جبران مادی و معنوی 
        برای قربانیان و 
        خانواده‌هایشان &
        مهم &
        سال ۱-۵ \\
        \altrow
        
        حسابرسی نهادی &
        بررسی پیشینه‌ی کارکنان 
        دولتی (\lr{vetting}) &
        مهم &
        ماه ۶-۲۴ \\
        
        بزرگداشت و یادبود &
        ساختن حافظه‌ی جمعی, 
        موزه‌ها, روز ملی &
        مطلوب &
        سال ۲-۱۰ \\
        \altrow
        
        آشتی ملی &
        گفت‌وگوی ملی, 
        عذرخواهی رسمی, 
        نمادهای مشترک &
        مهم &
        مستمر \\
        
        \bottomrule
    \end{tabularx}
\end{table}

\begin{warningbox}
\textbf{تعادل حیاتی: عدالت vs ثبات}

بزرگ‌ترین معضل عدالت انتقالی تعادل بین 
خواست عدالت و نیاز به ثبات است:
\begin{itemize}[itemsep=2pt]
    \item \emphred{عدالت بیش از حد → 
    بی‌ثباتی} (عراق: دی‌بعثی‌سازی → 
    داعش)
    \item \emphred{فراموشی → 
    نارضایتی مزمن} (اسپانیا: 
    «پیمان فراموشی» نیم‌قرن بعد 
    هنوز محل بحث)
    \item \emphgreen{تعادل = 
    محاکمه‌ی مسئولان اصلی + 
    آشتی برای بقیه} 
    (آفریقای جنوبی)
\end{itemize}
\end{warningbox}

\sectiondivider

% ============================================================
\section{تضمین‌های نهادی}
\label{sec:institutional-guarantees}
% ============================================================

\begin{table}[htbp]
    \centering
    \caption{نهادهای کلیدی دوره‌ی گذار}
    \label{tab:transitional-institutions}
    \tablefontsize
    \begin{tabularx}{\textwidth}{
        L{3cm} X C{1.5cm}
    }
        \toprule
        \headerrow
        \textbf{نهاد} & 
        \textbf{وظیفه و مشخصات} & 
        \textbf{زمان تأسیس} \\
        \midrule
        
        کمیسیون مستقل انتخابات &
        مدیریت تمام انتخابات و رفراندوم‌ها, 
        استقلال کامل از دولت, 
        ترکیب مشورتی بین‌المللی &
        ماه ۱-۳ \\
        \altrow
        
        دادگاه قانون اساسی &
        نظارت بر انطباق قوانین 
        با اصول دموکراتیک, 
        حکمیت اختلافات &
        ماه ۶-۱۲ \\
        
        نهاد تنظیم‌گر رسانه &
        تضمین آزادی و مسئولیت 
        رسانه, صدور مجوز, 
        استقلال از دولت &
        ماه ۳-۶ \\
        \altrow
        
        کمیسیون ملی حقوق بشر &
        نظارت بر رعایت حقوق بشر, 
        دریافت شکایات, 
        گزارش‌دهی عمومی &
        ماه ۳-۶ \\
        
        نهاد ضد فساد &
        پیشگیری و مبارزه با فساد 
        در دوره‌ی گذار &
        ماه ۳-۶ \\
        \altrow
        
        شورای مشورتی ملی &
        نمایندگی همه‌ی گروه‌ها 
        تا تشکیل مجلس مؤسسان &
        هفته‌ی ۱-۲ \\
        
        \bottomrule
    \end{tabularx}
\end{table}

\sectiondivider

% ============================================================
\section{ماتریس جامع تضمین‌ها}
\label{sec:guarantee-matrix}
% ============================================================

\begin{landscape}
\begin{table}[htbp]
    \centering
    \caption{ماتریس جامع تضمین‌ها 
    و اولویت‌بندی}
    \label{tab:guarantee-matrix}
    \bigtablefontsize
    \setlength{\tabcolsep}{3pt}
    \begin{tabularx}{\linewidth}{
        L{2cm} X X X L{2.5cm}
    }
        \toprule
        \headerrow
        \textbf{حوزه} & 
        \textbf{حیاتی (بدون آن شروع نکنید)} & 
        \textbf{مهم (لازم برای موفقیت)} & 
        \textbf{مطلوب (تقویت‌کننده)} &
        \textbf{مسئول اصلی} \\
        \midrule
        
        سیاسی &
        اجماع حداقلی بر قواعد بازی &
        حمایت فعال بین‌المللی, 
        مکانیزم حل اختلاف &
        مشارکت دیاسپورا &
        نیروهای سیاسی ایرانی \\
        \altrow
        
        امنیتی &
        کنترل سپاه, حفاظت 
        از تأسیسات هسته‌ای &
        امنیت مرزی, \lr{DDR}, 
        مدیریت نیروهای نیابتی &
        اصلاح کامل بخش 
        امنیتی &
        شورای امنیت + 
        بازیگران داخلی \\
        
        حقوقی &
        \lr{SOMA}, چارچوب 
        حقوقی موقت &
        قانون انتخابات/احزاب/رسانه, 
        مکانیزم عدالت انتقالی &
        الحاق به معاهدات 
        بین‌المللی &
        دولت موقت + UN \\
        \altrow
        
        اقتصادی &
        رفع تحریم‌ها, 
        بسته حمایت فوری &
        صندوق امانی نفت, 
        ضد فساد &
        بازسازی زیرساخت, 
        جذب سرمایه &
        قدرت‌های بزرگ + 
        \lr{IFIs} \\
        
        اجتماعی &
        مشارکت زنان (۳۰٪+), 
        نمایندگی اقوام &
        گفت‌وگوی ملی, 
        عدالت انتقالی &
        آشتی کامل, 
        سواد دموکراتیک &
        جامعه مدنی + UN \\
        \altrow
        
        نهادی &
        کمیسیون انتخابات, 
        شورای مشورتی &
        استقلال قضا, 
        آزادی رسانه &
        کمیسیون حقوق بشر, 
        نهاد ضد فساد &
        دولت موقت + 
        مشاوران بین‌المللی \\
        
        \bottomrule
    \end{tabularx}
\end{table}
\end{landscape}

\sectiondivider

% ============================================================
\section{جمع‌بندی فصل}
\label{sec:ch6-summary}
% ============================================================

\begin{chaptersummary}

\textbf{آنچه در این فصل آموختیم:}

\begin{enumerate}[
    label=\textcolor{DarkGray}{\bfseries\arabic*.},
    itemsep=4pt
]
    \item \textbf{شش حوزه‌ی تضمین} 
    شناسایی شد: سیاسی, امنیتی, 
    حقوقی, اقتصادی, اجتماعی و نهادی.
    
    \item \textbf{اجماع حداقلی سیاسی} 
    اولین و مهم‌ترین پیش‌شرط است — 
    و بزرگ‌ترین ضعف فعلی اپوزیسیون ایران.
    
    \item \textbf{مدیریت سپاه} با مدل 
    «بازسازی تدریجی» (الگوی اندونزی) 
    پیشنهاد شد — نه انحلال کامل 
    (خطای عراق) و نه حفظ بدون اصلاح 
    (خطای مصر).
    
    \item \textbf{تأسیسات هسته‌ای} 
    اولویت مطلق امنیتی هستند و 
    \lr{IAEA} باید فوراً دسترسی 
    کامل داشته باشد.
    
    \item \textbf{رفع تحریم‌ها و 
    بسته‌ی حمایت اقتصادی} شرط لازم 
    موفقیت است — «شوک‌درمانی» روسی 
    تکرار نشود.
    
    \item \textbf{حداقل ۳۰٪ مشارکت زنان} 
    در همه‌ی نهادهای گذار باید 
    تضمین شود.
    
    \item \textbf{مدل ترکیبی مدیریت 
    تنوع قومی} (الگوی اسپانیایی) 
    پیشنهاد شد: خودمختاری استانی + 
    حقوق فرهنگی + تضمین نمایندگی.
    
    \item \textbf{عدالت انتقالی} باید 
    آشتی‌محور باشد: محاکمه‌ی مسئولان 
    اصلی + آشتی برای بقیه.
    
    \item \textbf{نهادهای کلیدی} باید 
    در هفته‌ها و ماه‌های نخست 
    تأسیس شوند — از شورای مشورتی 
    تا کمیسیون انتخابات.
\end{enumerate}

\vspace{6pt}
\begin{center}
    \textcolor{MainRed}{
        \faArrowLeft\hspace{8pt}
        \textbf{فصل بعد: آسیب‌شناسی، 
        ریسک‌ها و چالش‌های پیش‌رو}
        \hspace{8pt}\faArrowLeft
    }
\end{center}

\end{chaptersummary}

\chapterend
% ═══════════════════════════════════════════════════════════════════════════════
% فصل ۷: آسیب‌شناسی، ریسک‌ها و چالش‌های پیش‌رو
% فایل: chapters/ch07-risks.tex
% رنگ فصل: قرمز (MainRed)
% ═══════════════════════════════════════════════════════════════════════════════

\chapteropening{۷}{آسیب‌شناسی، ریسک‌ها و چالش‌های پیش‌رو}{MainRed}{%
هیچ طرحی برای نبرد از اولین برخورد با دشمن جان سالم به در نمی‌برد؛ اما طرح نداشتن به معنای شکست حتمی است.%
}{هلموت فون مولتکه}

\chapter{آسیب‌شناسی، ریسک‌ها و چالش‌های پیش‌رو}
\label{ch:risks}

\minitoc

% ─────────────────────────────────────────────────────────────────────────────
% خلاصه اجرایی
% ─────────────────────────────────────────────────────────────────────────────

\begin{executivesummary}
نظارت بین‌المللی بر گذار دموکراتیک، علی‌رغم ضرورت و فواید بالقوه، با ریسک‌ها و آسیب‌های جدی مواجه است که عدم شناسایی و مدیریت آن‌ها می‌تواند کل فرایند را به شکست بکشاند. این فصل به تحلیل انتقادی شش دسته ریسک می‌پردازد: \emph{آسیب‌شناسی مفهومی} (ادراک نئواستعماری، خستگی بین‌المللی)، \emph{ریسک‌های امنیتی} (بازگشت اقتدارگرایی، تجزیه، خشونت)، \emph{ریسک‌های سیاسی} (مصادره گذار، پوپولیسم)، \emph{ریسک‌های اقتصادی} (فروپاشی، الیگارشی)، \emph{ریسک‌های اجتماعی} (انتقام‌جویی، تضاد دیاسپورا-داخل)، و \emph{ریسک‌های خود نظارت} (ناکارآمدی، فساد، سوگیری). برای هر ریسک، نمونه‌های تاریخی، احتمال وقوع در ایران، و راهکارهای پیشگیری ارائه می‌شود. نقشه حرارتی ریسک و ماتریس پاسخ در پایان فصل، ابزار تصمیم‌گیری عملیاتی را فراهم می‌آورد.
\end{executivesummary}

\section{درآمد: چرا آسیب‌شناسی ضروری است؟}
\label{sec:risks-intro}

تاریخ گذارهای دموکراتیک مملو از شکست‌ها، عقب‌گردها و فاجعه‌های انسانی است. از هر سه تلاش گذار، تقریباً یکی به دموکراسی تحکیم‌یافته می‌رسد، یکی به اقتدارگرایی بازمی‌گردد، و یکی در منطقه خاکستری بین این دو معلق می‌ماند.\footnote{\متن‌لاتین{Diamond, Larry. "Facing Up to the Democratic Recession." \emph{Journal of Democracy} 26, no. 1 (2015): 141-155.}} نظارت بین‌المللی که قرار است این ریسک‌ها را کاهش دهد، خود می‌تواند منشأ ریسک‌های جدید باشد.

\begin{keypoint}
شناسایی پیشینی ریسک‌ها و طراحی مکانیزم‌های پاسخ، تفاوت میان گذار موفق و فاجعه است. تجربه عراق، لیبی، و میانمار نشان می‌دهد که حتی نیت خوب و منابع فراوان، بدون مدیریت ریسک، به شکست می‌انجامد.
\end{keypoint}

این فصل با رویکردی انتقادی و واقع‌بینانه، نه برای منصرف کردن از نظارت بین‌المللی، بلکه برای \emph{طراحی هوشمندتر} آن نگاشته شده است. هر ریسکی که شناسایی و برنامه‌ریزی شود، ریسکی است که می‌توان مدیریتش کرد.

\sectiondivider

% ═══════════════════════════════════════════════════════════════════════════════
\section{آسیب‌شناسی مفهومی نظارت بین‌المللی}
\label{sec:conceptual-pathology}
% ═══════════════════════════════════════════════════════════════════════════════

پیش از بررسی ریسک‌های عینی، باید به نقدهای بنیادین نظارت بین‌المللی پرداخت که در ادبیات آکادمیک و گفتمان‌های ضداستعماری مطرح‌اند.

\subsection{نقد نئواستعماری و «مهندسی دموکراسی»}
\label{subsec:neocolonial-critique}

منتقدان پسااستعماری معتقدند نظارت بین‌المللی اغلب پوششی برای تحمیل مدل‌های غربی حکمرانی است.\footnote{\متن‌لاتین{Paris, Roland. "International Peacebuilding and the 'Mission Civilisatrice'." \emph{Review of International Studies} 28, no. 4 (2002): 637-656.}} مفاهیمی چون \bilingual{ظرفیت‌سازی}{Capacity Building} یا \bilingual{حکمرانی خوب}{Good Governance} می‌توانند ابزار سلطه فرهنگی تعبیر شوند.

\begin{warningbox}
در جامعه ایرانی با حافظه تاریخی مداخلات خارجی (کودتای ۱۳۳۲، قرارداد ۱۹۱۹، حمایت از صدام)، حساسیت به هرگونه «قیمومیت» بین‌المللی بسیار بالاست. حتی کمک‌های صادقانه می‌توانند با مقاومت روبرو شوند.
\end{warningbox}

\textbf{راهکارهای پیشگیری:}
\begin{itemize}[nosep]
    \item تأکید مکرر بر \emph{مالکیت ملی} (ایرانیان تصمیم‌گیرنده، بین‌المللی‌ها تسهیل‌گر)
    \item اجتناب از زبان «نجات‌بخشانه» در ارتباطات عمومی
    \item حضور پررنگ چهره‌های بین‌المللی از جنوب جهانی (نه فقط غرب)
    \item شفافیت کامل درباره منافع کشورهای حامی
    \item مکانیزم شکایت و اعتراض برای شهروندان ایرانی
\end{itemize}

\subsection{خستگی بین‌المللی و کاهش تعهد}
\label{subsec:international-fatigue}

\bilingual{خستگی بین‌المللی}{International Fatigue} پدیده‌ای مستند است که طی آن توجه و منابع جامعه جهانی پس از ماه‌های اولیه کاهش می‌یابد.\footnote{\متن‌لاتین{Autesserre, Séverine. \emph{Peaceland: Conflict Resolution and the Everyday Politics of International Intervention}. Cambridge University Press, 2014.}}

\begin{casestudy}{افغانستان: از «ملت‌سازی» تا فراموشی}
تعهد اولیه جامعه بین‌المللی به افغانستان پس از ۲۰۰۱ با شعار «دیگر هرگز تنها نخواهید بود» آغاز شد. اما تدریجاً توجه به عراق، سپس سوریه، و سپس اوکراین معطوف شد. بودجه کمک‌ها از ۱۵.۷ میلیارد دلار در ۲۰۱۱ به ۴.۲ میلیارد در ۲۰۲۰ کاهش یافت. نتیجه: سقوط ۲۰۲۱ و بازگشت طالبان.
\end{casestudy}

\textbf{چرخه توجه بین‌المللی:}

\begin{figure}[htbp]
\centering
\begin{tikzpicture}[
    node distance=2.5cm,
    every node/.style={font=\small},
    phase/.style={rectangle, rounded corners, draw=MainRed, fill=LightRed, minimum width=2.8cm, minimum height=1cm, align=center},
    arrow/.style={-{Stealth[length=3mm]}, thick, MainRed}
]
    \node[phase] (crisis) {بحران اولیه\\توجه حداکثری};
    \node[phase, left=of crisis] (commit) {تعهد منابع\\کنفرانس کمک‌ها};
    \node[phase, below=of commit] (routine) {روتین شدن\\کاهش پوشش رسانه‌ای};
    \node[phase, below=of crisis] (fatigue) {خستگی\\کاهش بودجه};
    \node[phase, below right=1.5cm and -1cm of fatigue] (exit) {استراتژی خروج\\«پیروزی» اعلامی};
    
    \draw[arrow] (crisis) -- (commit);
    \draw[arrow] (commit) -- (routine);
    \draw[arrow] (routine) -- (fatigue);
    \draw[arrow] (fatigue) -- (exit);
    \draw[arrow, dashed, DarkGray] (exit) to[bend right=40] node[right, font=\footnotesize] {بحران جدید} (crisis);
\end{tikzpicture}
\caption{چرخه توجه بین‌المللی و خطر خستگی}
\label{fig:attention-cycle}
\end{figure}

\textbf{راهکارهای پیشگیری:}
\begin{itemize}[nosep]
    \item تعهدات چندساله الزام‌آور (نه سالانه)
    \item پیوند به منافع ملی کشورهای حامی (امنیت انرژی، مهاجرت، تروریسم)
    \item تنوع‌بخشی به منابع مالی (نه وابستگی به یک حامی)
    \item ایجاد ذی‌نفعان داخلی در کشورهای حامی (شرکت‌ها، دیاسپورا)
\end{itemize}

\subsection{رقابت ناظران و تضاد منافع}
\label{subsec:competing-monitors}

هنگامی که چندین نهاد بین‌المللی همزمان وارد می‌شوند، رقابت بر سر حوزه نفوذ، منابع، و اعتبار می‌تواند به ناهماهنگی و حتی تضاد بینجامد.

\begin{lessonlearned}{بوسنی: سه پا در یک کفش}
در بوسنی پس از جنگ، \lr{OHR}، \lr{OSCE}، \lr{EU}، \lr{NATO}، و \lr{UNDP} هر یک مأموریت‌های همپوشان داشتند. تعارض میان \lr{OHR} و \lr{EU} بر سر اولویت‌های اصلاحات، فرایند الحاق به اتحادیه اروپا را سال‌ها به تأخیر انداخت.
\end{lessonlearned}

\subsection{مشروعیت‌بخشی کاذب}
\label{subsec:false-legitimization}

نظارت بین‌المللی می‌تواند ناخواسته به فرایندهای غیردموکراتیک مشروعیت ببخشد.

\begin{casestudy}{میانمار ۲۰۱۰: انتخابات تأییدشده، دموکراسی کاذب}
انتخابات ۲۰۱۰ میانمار با حضور برخی ناظران بین‌المللی برگزار شد. اعلام «پیشرفت» توسط برخی نهادها، فشار بین‌المللی برای اصلاحات واقعی را کاهش داد. نتیجه: دموکراسی ناقصی که در ۲۰۲۱ با کودتا فروپاشید.
\end{casestudy}

\textbf{راهکارهای پیشگیری:}
\begin{itemize}[nosep]
    \item معیارهای صریح و علنی برای «موفقیت» از ابتدا
    \item گزارش‌دهی صادقانه حتی اگر منفی باشد
    \item اجتناب از فشار سیاسی برای اعلام «پیروزی» زودهنگام
    \item مکانیزم نظارت بر ناظران (ارزیابی مستقل)
\end{itemize}

% ═══════════════════════════════════════════════════════════════════════════════
\section{ریسک‌های امنیتی}
\label{sec:security-risks}
% ═══════════════════════════════════════════════════════════════════════════════

ریسک‌های امنیتی حیاتی‌ترین تهدیدات برای گذار موفق‌اند و می‌توانند تمام دستاوردها را یک‌شبه نابود کنند.

\subsection{بازگشت اقتدارگرایی}
\label{subsec:authoritarian-reversal}

\bilingual{بازگشت اقتدارگرایی}{Authoritarian Reversal} یا \bilingual{موج معکوس}{Reverse Wave} پدیده‌ای رایج است که طی آن نیروهای قدیم یا جدید، دستاوردهای دموکراتیک را به عقب می‌رانند.

\begin{warningbox}
مصر ۲۰۱۳ نمونه کلاسیک است: انتخابات آزاد ← رئیس‌جمهور منتخب ← کودتای نظامی ← سرکوب شدیدتر از قبل. این سناریو در ایران با حضور سپاه پاسداران، محتمل‌ترین و خطرناک‌ترین ریسک است.
\end{warningbox}

\begin{table}[htbp]
\centering
\caption{نمونه‌های بازگشت اقتدارگرایی پس از گذار}
\label{tab:reversal-examples}
\begin{tabularx}{\textwidth}{>{\raggedleft\arraybackslash}p{2cm} 
                             >{\raggedleft\arraybackslash}p{2cm}
                             >{\raggedleft\arraybackslash}X
                             >{\raggedleft\arraybackslash}X}
\toprule
\headerrow کشور & سال بازگشت & مکانیزم & پیامد \\
\midrule
مصر & ۲۰۱۳ & کودتای نظامی با حمایت مردمی & دیکتاتوری نظامی \\
\altrow تایلند & ۲۰۱۴ & کودتای نظامی & حکومت نظامی ممتد \\
ترکیه & ۲۰۱۶+ & تمرکز قدرت پس از کودتای نافرجام & اقتدارگرایی رقابتی \\
\altrow میانمار & ۲۰۲۱ & کودتای نظامی & جنگ داخلی \\
ونزوئلا & ۲۰۰۰+ & تضعیف تدریجی نهادها & اقتدارگرایی پوپولیستی \\
\bottomrule
\end{tabularx}
\end{table}

\textbf{عوامل افزایش‌دهنده ریسک در ایران:}
\begin{enumerate}[nosep]
    \item قدرت اقتصادی-نظامی سپاه پاسداران
    \item شبکه اطلاعاتی گسترده
    \item امکان بسیج بخشی از جامعه با روایت «هرج‌ومرج»
    \item حمایت احتمالی بازیگران خارجی (روسیه، چین)
    \item ضعف احتمالی نهادهای دموکراتیک نوپا
\end{enumerate}

\textbf{راهکارهای پیشگیری:}
\begin{enumerate}[nosep]
    \item اصلاح ساختاری نیروهای مسلح در فاز اول (\seeChapter{ch:guarantees})
    \item ممنوعیت قانون اساسی از دخالت نظامیان در سیاست
    \item نظارت بین‌المللی بر بخش امنیتی (نه فقط انتخابات)
    \item تقویت سریع نهادهای مدنی به‌عنوان موازنه
    \item ضمانت‌های بین‌المللی علیه کودتا (تهدید به تحریم فوری)
\end{enumerate}

\subsection{تجزیه و جنگ داخلی}
\label{subsec:fragmentation}

\begin{keypoint}
ایران با تنوع قومی-زبانی (فارس ۶۱٪، آذری ۱۶٪، کرد ۱۰٪، لر ۶٪، عرب ۲٪، بلوچ ۲٪، ترکمن ۲٪، و دیگران)، در صورت مدیریت نادرست گذار، مستعد بحران‌های قومی است.
\end{keypoint}

\begin{casestudy}{یوگسلاوی: از فدراسیون تا جنگ}
یوگسلاوی پس از مرگ تیتو و فروپاشی کمونیسم، نتوانست گذار مسالمت‌آمیز داشته باشد. رقابت رهبران قومی، ضعف نهادهای مشترک، و دخالت خارجی ناهماهنگ به جنگ‌های خونین ۱۹۹۱-۲۰۰۱ با بیش از ۱۴۰,۰۰۰ کشته و ۴ میلیون آواره انجامید.
\end{casestudy}

\textbf{تفاوت‌های ایران با یوگسلاوی:}
\begin{itemize}[nosep]
    \item[\cmark] هویت ملی ایرانی قوی‌تر از هویت یوگسلاوی
    \item[\cmark] سابقه طولانی همزیستی (بر خلاف ترکیب مصنوعی یوگسلاوی)
    \item[\cmark] نبود مرزهای داخلی فدرالی که تجزیه را تسهیل کند
    \item[\xmark] سابقه سرکوب اقوام در چهار دهه اخیر
    \item[\xmark] مرزهای طولانی با کشورهای بی‌ثبات
    \item[\xmark] امکان حمایت خارجی از جنبش‌های تجزیه‌طلب
\end{itemize}

\textbf{راهکارهای پیشگیری:}
\begin{enumerate}[nosep]
    \item تعهد صریح همه بازیگران به تمامیت ارضی
    \item طراحی ساختار نامتمرکز اما نه فدرال
    \item تضمین حقوق فرهنگی-زبانی اقوام
    \item مشارکت نمایندگان اقوام در دولت انتقالی
    \item نظارت بین‌المللی بر مناطق حساس مرزی
\end{enumerate}

\subsection{خشونت فرقه‌ای و انتقام‌جویی}
\label{subsec:sectarian-violence}

\begin{lessonlearned}{عراق: حمام خون فرقه‌ای}
سیاست «بعث‌زدایی» افراطی پس از ۲۰۰۳، همراه با انحلال ارتش، میلیون‌ها سنّی را به حاشیه راند. نتیجه: جنگ داخلی ۲۰۰۶-۲۰۰۸ با دهها هزار کشته، و سپس ظهور داعش. نظارت بین‌المللی نه تنها جلوی این فاجعه را نگرفت، بلکه آمریکا به‌عنوان «ناظر» خود عامل اصلی آن بود.
\end{lessonlearned}

\textbf{عوامل ریسک خشونت در ایران:}
\begin{itemize}[nosep]
    \item انباشت کینه‌های ۴۵ ساله
    \item گستردگی نیروهای امنیتی و بسیجی در محلات
    \item سلاح‌های پراکنده در صورت فروپاشی نظم
    \item امکان تحریک خارجی (داعش، القاعده)
\end{itemize}

\subsection{تروریسم و ناامنی}
\label{subsec:terrorism}

گذار می‌تواند خلأ امنیتی ایجاد کند که گروه‌های تروریستی از آن بهره‌برداری کنند.

\begin{table}[htbp]
\centering
\caption{تهدیدات تروریستی بالقوه در فرایند گذار ایران}
\label{tab:terror-threats}
\begin{tabularx}{\textwidth}{>{\raggedleft\arraybackslash}p{3cm}
                             >{\raggedleft\arraybackslash}p{2cm}
                             >{\raggedleft\arraybackslash}X
                             >{\raggedleft\arraybackslash}X}
\toprule
\headerrow تهدید & احتمال & مکانیزم & پیامد بالقوه \\
\midrule
داعش/القاعده & متوسط & نفوذ از مرزهای شرقی و غربی & حملات در مناطق سنّی‌نشین \\
\altrow گروه‌های شیعه افراطی & متوسط-بالا & مقاومت علیه «تغییر ضداسلامی» & ترور شخصیت‌های اصلاح‌طلب \\
جریان‌های قومی مسلح & متوسط & بهره‌برداری از خلأ امنیتی & ناامنی در مناطق مرزی \\
\altrow عناصر سپاه زیرزمینی & بالا & ضدانقلاب مسلح & خرابکاری و ترور \\
\bottomrule
\end{tabularx}
\end{table}

\sectiondivider

% ═══════════════════════════════════════════════════════════════════════════════
\section{ریسک‌های سیاسی}
\label{sec:political-risks}
% ═══════════════════════════════════════════════════════════════════════════════

\subsection{مصادره گذار توسط یک جناح}
\label{subsec:capture}

\bilingual{مصادره گذار}{Transition Capture} زمانی رخ می‌دهد که یک گروه سیاسی، به نام «انقلاب» یا «دموکراسی»، قدرت را قبضه می‌کند.

\begin{warningbox}
در تاریخ ایران، انقلاب ۱۳۵۷ نمونه کلاسیک مصادره است: جنبش گسترده با خواسته‌های متنوع، توسط یک جریان خاص مصادره شد. خطر تکرار این الگو جدی است.
\end{warningbox}

\textbf{سناریوهای مصادره:}
\begin{enumerate}[nosep]
    \item \textbf{مصادره نظامی}: ارتش یا سپاه «انتظامی» به نام «ثبات» قدرت را می‌گیرد
    \item \textbf{مصادره ایدئولوژیک}: یک جریان سیاسی (چپ یا راست افراطی) رقبا را حذف می‌کند
    \item \textbf{مصادره قومی}: یک قوم غالب، دیگران را به حاشیه می‌راند
    \item \textbf{مصادره الیگارشیک}: نخبگان اقتصادی، دموکراسی صوری ایجاد می‌کنند
\end{enumerate}

\textbf{راهکارهای پیشگیری:}
\begin{itemize}[nosep]
    \item ائتلاف گذار با نمایندگان واقعی همه جریان‌ها
    \item قانون انتخابات تناسبی (نه اکثریتی)
    \item تضمین‌های حقوق اقلیت در قانون اساسی
    \item نظارت بین‌المللی بر فراگیری فرایند (نه فقط انتخابات)
\end{itemize}

\subsection{بن‌بست سیاسی و فلج نهادی}
\label{subsec:deadlock}

\begin{casestudy}{بلژیک و عراق: دولت‌های بدون دولت}
بلژیک در ۲۰۱۰-۲۰۱۱ برای ۵۴۱ روز بدون دولت بود. عراق پس از انتخابات ۲۰۱۰ نُه ماه طول کشید تا دولت تشکیل شود. در ایران با چنددستگی اپوزیسیون، این سناریو محتمل است و می‌تواند مردم را از دموکراسی سرخورده کند.
\end{casestudy}

\subsection{پوپولیسم و عوام‌فریبی}
\label{subsec:populism}

گذارهای دموکراتیک زمین حاصلخیزی برای پوپولیست‌هاست که با وعده‌های ساده‌انگارانه، رأی‌دهندگان سرخورده را جذب می‌کنند.

\begin{lessonlearned}{ونزوئلا: از دموکراسی تا چاوزیسم}
ونزوئلا در دهه ۱۹۹۰ یکی از باثبات‌ترین دموکراسی‌های آمریکای لاتین بود. فساد نخبگان و نابرابری، زمینه را برای ظهور چاوز فراهم کرد که با وعده «انقلاب» بر سر کار آمد و تدریجاً نهادهای دموکراتیک را تخریب کرد. ناظران بین‌المللی تا دیر متوجه روند نشدند.
\end{lessonlearned}

\sectiondivider

% ═══════════════════════════════════════════════════════════════════════════════
\section{ریسک‌های اقتصادی}
\label{sec:economic-risks}
% ═══════════════════════════════════════════════════════════════════════════════

\subsection{فروپاشی اقتصادی}
\label{subsec:economic-collapse}

گذار سیاسی اغلب با بحران اقتصادی همراه است: کاهش سرمایه‌گذاری، فرار سرمایه، تورم، و بیکاری.

\begin{keypoint}
اقتصاد ایران پیش از گذار در وضعیت بحرانی است: تورم ۴۰-۵۰٪، بیکاری رسمی ۱۰٪ (واقعی ۲۵-۳۰٪)، ارزش ریال در ۱۰ سال ۹۰٪ کاهش، ذخایر ارزی محدود. گذار می‌تواند این بحران را تشدید یا تخفیف دهد.
\end{keypoint}

\textbf{سناریوی بدبینانه:}
\begin{itemize}[nosep]
    \item فرار سرمایه گسترده در هفته‌های اول
    \item توقف صادرات نفت به دلیل بی‌ثباتی
    \item هجوم به بانک‌ها و فروپاشی سیستم مالی
    \item کاهش ۳۰-۵۰٪ تولید ناخالص داخلی (مشابه عراق پس از ۲۰۰۳)
\end{itemize}

\textbf{سناریوی خوش‌بینانه:}
\begin{itemize}[nosep]
    \item رفع سریع تحریم‌ها و آزادسازی دارایی‌های بلوکه‌شده (۱۰۰+ میلیارد دلار)
    \item بسته کمک بین‌المللی فوری (مشابه طرح مارشال)
    \item بازگشت اعتماد و سرمایه‌گذاری
    \item رشد ۵-۱۰٪ سالانه در سال‌های اول
\end{itemize}

\subsection{غارت دارایی‌ها و الیگارشی}
\label{subsec:looting-oligarchy}

\begin{casestudy}{روسیه دهه ۱۹۹۰: از کمونیسم تا الیگارشی}
خصوصی‌سازی سریع در روسیه، با نظارت ناکافی بین‌المللی، به انتقال دارایی‌های عمومی به گروهی کوچک از «الیگارش‌ها» انجامید. ثروت ۷ نفر از ثروتمندترین روس‌ها برابر نیمی از کل جمعیت شد. نارضایتی عمومی، زمینه‌ساز ظهور پوتین بود.
\end{casestudy}

\textbf{ریسک ویژه ایران:}
\begin{itemize}[nosep]
    \item امپراتوری اقتصادی سپاه (۲۰-۴۰٪ اقتصاد)
    \item بنیادها و نهادهای شبه‌دولتی
    \item شبکه‌های رانت دهه‌ها قدیمی
    \item فقدان شفافیت مالکیت واقعی
\end{itemize}

\begin{recommendation}
قبل از هرگونه خصوصی‌سازی، باید:
\begin{enumerate}[nosep]
    \item حسابرسی کامل از دارایی‌های دولتی و شبه‌دولتی
    \item شناسایی مالکیت واقعی شرکت‌ها (نه صوری)
    \item ایجاد صندوق امانی بین‌المللی برای دارایی‌های استراتژیک
    \item قوانین ضدانحصار و ضدفساد قبل از خصوصی‌سازی
\end{enumerate}
\end{recommendation}

\sectiondivider

% ═══════════════════════════════════════════════════════════════════════════════
\section{ریسک‌های اجتماعی}
\label{sec:social-risks}
% ═══════════════════════════════════════════════════════════════════════════════

\subsection{انتقام‌جویی و عدالت انتقامی}
\label{subsec:revenge}

تمایل طبیعی قربانیان سرکوب به انتقام می‌تواند چرخه خشونت ایجاد کند.

\begin{warningbox}
تجربه عراق نشان داد که «بعث‌زدایی» افراطی می‌تواند میلیون‌ها نفر را به دشمن گذار تبدیل کند. در ایران با میلیون‌ها عضو و وابسته به نهادهای حکومتی (سپاه، بسیج، نهادهای مذهبی)، سیاست انتقامی فاجعه‌بار خواهد بود.
\end{warningbox}

\textbf{تعادل دشوار:}
\begin{itemize}[nosep]
    \item عدالت برای قربانیان ضروری است
    \item انتقام جمعی چرخه خشونت ایجاد می‌کند
    \item بخشش بدون پاسخگویی، بی‌عدالتی است
    \item عفو عمومی، مصونیت برای جنایتکاران است
\end{itemize}

\textbf{راهکار پیشنهادی:}
\begin{enumerate}[nosep]
    \item تمرکز بر رهبران و آمران (نه مجریان رده‌پایین)
    \item کمیسیون حقیقت با فرصت اعتراف و کاهش مجازات
    \item جبران خسارت مادی و نمادین برای قربانیان
    \item منع از مناصب حساس (نه مجازات کیفری) برای همکاران رده‌میانی
\end{enumerate}

\subsection{تضاد دیاسپورا و داخل}
\label{subsec:diaspora-conflict}

\begin{keypoint}
دیاسپورای ایرانی (۴-۵ میلیون نفر) دارای سرمایه، تخصص، و شبکه‌های بین‌المللی است، اما ممکن است با واقعیت‌های داخل فاصله داشته باشد. تنش میان «آن‌ها که ماندند» و «آن‌ها که رفتند» می‌تواند به شکاف سیاسی بدل شود.
\end{keypoint}

\begin{table}[htbp]
\centering
\caption{منابع تنش دیاسپورا-داخل}
\label{tab:diaspora-tension}
\begin{tabularx}{\textwidth}{>{\raggedleft\arraybackslash}p{3.5cm}
                             >{\raggedleft\arraybackslash}X
                             >{\raggedleft\arraybackslash}X}
\toprule
\headerrow موضوع & دیدگاه دیاسپورا (احتمالی) & دیدگاه داخل (احتمالی) \\
\midrule
سرعت تغییرات & تغییرات رادیکال و سریع & محتاط‌تر، ترس از بی‌ثباتی \\
\altrow عدالت انتقالی & مجازات شدید عاملان & مصالحه و آرامش \\
رابطه با غرب & استقبال از حمایت غربی & نگرانی از وابستگی \\
\altrow اولویت‌های اقتصادی & آزادسازی سریع، جذب سرمایه & حمایت از صنایع داخلی \\
هویت ملی & تأکید بر هویت پیشااسلامی & تنوع بیشتر در تعریف هویت \\
\bottomrule
\end{tabularx}
\end{table}

\textbf{راهکارهای پیشگیری:}
\begin{itemize}[nosep]
    \item سهمیه معقول (نه غالب) برای دیاسپورا در نهادهای انتقالی
    \item مشوق بازگشت تدریجی، نه هجوم
    \item گفتگوی ساختاریافته دیاسپورا-داخل از همین حالا
    \item نظارت بر جریان سرمایه دیاسپورا برای جلوگیری از «استعمار اقتصادی»
\end{itemize}

\subsection{بحران هویت و خلأ ارزشی}
\label{subsec:identity-crisis}

فروپاشی ایدئولوژی حاکم می‌تواند به خلأ ارزشی بینجامد.

\begin{casestudy}{روسیه پس از شوروی: از کمونیسم تا نیهیلیسم}
فروپاشی شوروی نه تنها یک نظام سیاسی، بلکه یک جهان‌بینی را فرو ریخت. نرخ خودکشی، اعتیاد، و جرم در دهه ۱۹۹۰ به‌شدت افزایش یافت. امید به زندگی مردان از ۶۴ به ۵۷ سال کاهش یافت. این خلأ، زمینه‌ساز بازگشت به ناسیونالیسم اقتدارگرای پوتین شد.
\end{casestudy}

\sectiondivider

% ═══════════════════════════════════════════════════════════════════════════════
\section{ریسک‌های ناشی از خود نظارت بین‌المللی}
\label{sec:monitoring-risks}
% ═══════════════════════════════════════════════════════════════════════════════

نظارت بین‌المللی نه تنها ممکن است ریسک‌ها را کاهش ندهد، بلکه می‌تواند خود منشأ ریسک‌های جدید باشد.

\subsection{ناکافی بودن و نمادین شدن}
\label{subsec:insufficient}

\begin{table}[htbp]
\centering
\caption{طیف نظارت: از ناکافی تا بیش‌ازحد}
\label{tab:monitoring-spectrum}
\begin{tabularx}{\textwidth}{>{\raggedleft\arraybackslash}p{2.5cm}
                             >{\raggedleft\arraybackslash}X
                             >{\raggedleft\arraybackslash}X}
\toprule
\headerrow سطح & مشکل & نمونه \\
\midrule
ناکافی & مشروعیت‌بخشی به فرایند معیوب & میانمار ۲۰۱۰ \\
\altrow متوسط ضعیف & عدم توانایی جلوگیری از نقض & افغانستان ۲۰۱۴ \\
متوسط مناسب & تأثیر مثبت با محدودیت & تونس ۲۰۱۱-۲۰۱۴ \\
\altrow قوی & تأثیر قابل‌توجه & تیمور شرقی ۱۹۹۹ \\
بیش‌ازحد & تضعیف مالکیت ملی & عراق ۲۰۰۳ \\
\bottomrule
\end{tabularx}
\end{table}

\subsection{ناآشنایی فرهنگی و زبانی}
\label{subsec:cultural-ignorance}

\begin{warningbox}
بسیاری از ناظران بین‌المللی با زبان، فرهنگ، و پیچیدگی‌های ایران آشنا نیستند. این می‌تواند به:
\begin{itemize}[nosep]
    \item سوءتفاهم‌های ارتباطی
    \item تفسیر نادرست رویدادها
    \item نادیده گرفتن نشانه‌های هشدار
    \item توصیه‌های نامناسب
\end{itemize}
بینجامد.
\end{warningbox}

\textbf{راهکارهای پیشگیری:}
\begin{enumerate}[nosep]
    \item آموزش اجباری زبان و فرهنگ برای همه ناظران
    \item تیم‌های مختلط ایرانی-بین‌المللی
    \item مشاوران فرهنگی در همه سطوح
    \item فروتنی نهادی: اعتراف به محدودیت‌های دانش
\end{enumerate}

\subsection{فساد و سوءاستفاده}
\label{subsec:corruption}

\begin{casestudy}{بوسنی و کوزوو: فساد صلح‌بانان}
گزارش‌های متعدد از فساد، قاچاق، و حتی بهره‌برداری جنسی توسط کارکنان بین‌المللی در بوسنی و کوزوو منتشر شده است. در یک مورد، کارکنان \lr{DynCorp} (پیمانکار آمریکایی) در قاچاق زنان برای فحشا دست داشتند.
\end{casestudy}

\textbf{راهکارهای پیشگیری:}
\begin{itemize}[nosep]
    \item غربالگری دقیق کارکنان بین‌المللی
    \item کدهای رفتاری الزام‌آور با ضمانت اجرا
    \item مکانیزم شکایت امن برای شهروندان محلی
    \item نظارت بر ناظران توسط نهاد مستقل
    \item عدم مصونیت برای جرایم سنگین
\end{itemize}

\subsection{سوگیری و جانبداری}
\label{subsec:bias}

نهادهای بین‌المللی ممکن است به دلایل ژئوپلیتیکی، مالی، یا ایدئولوژیک، به نفع یک جناح داخلی سوگیری داشته باشند.

\textbf{منابع سوگیری:}
\begin{enumerate}[nosep]
    \item فشار کشورهای تأمین‌کننده بودجه
    \item ارتباطات پیشین با برخی گروه‌های اپوزیسیون
    \item ترجیحات ایدئولوژیک کارکنان
    \item تأثیر لابی‌های دیاسپورا
\end{enumerate}

\begin{recommendation}
برای کاهش سوگیری:
\begin{enumerate}[nosep]
    \item تنوع در منابع مالی (هیچ کشوری بیش از ۲۵٪)
    \item تنوع در ترکیب کارکنان (ملیتی، جنسیتی، تخصصی)
    \item ارزیابی مستقل دوره‌ای
    \item مکانیزم اعتراض برای طرف‌های معترض
\end{enumerate}
\end{recommendation}

\sectiondivider

% ═══════════════════════════════════════════════════════════════════════════════
\section{نقشه حرارتی ریسک}
\label{sec:risk-heatmap}
% ═══════════════════════════════════════════════════════════════════════════════

نمودار زیر ماتریس ریسک را بر اساس دو محور \emph{احتمال وقوع} و \emph{شدت پیامد} نشان می‌دهد:

\begin{figure}[htbp]
\centering
\begin{tikzpicture}[
    font=\footnotesize,
    cell/.style={minimum width=2.5cm, minimum height=1.2cm, align=center},
]
    % Grid
    \draw[step=2.5cm, DarkGray, thin] (0,0) grid (10,6);
    
    % Axis labels
    \node[rotate=90, anchor=center] at (-0.8,3) {\textbf{شدت پیامد}};
    \node[anchor=center] at (5,-0.5) {\textbf{احتمال وقوع}};
    
    % Y-axis labels
    \node[anchor=east] at (0,0.6) {پایین};
    \node[anchor=east] at (0,1.8) {};
    \node[anchor=east] at (0,3) {متوسط};
    \node[anchor=east] at (0,4.2) {};
    \node[anchor=east] at (0,5.4) {بالا};
    
    % X-axis labels  
    \node[anchor=north] at (1.25,0) {پایین};
    \node[anchor=north] at (3.75,0) {متوسط};
    \node[anchor=north] at (6.25,0) {متوسط-بالا};
    \node[anchor=north] at (8.75,0) {بالا};
    
    % Color zones (background)
    \fill[green!20] (0,0) rectangle (2.5,2);
    \fill[yellow!30] (2.5,0) rectangle (5,2);
    \fill[yellow!30] (0,2) rectangle (2.5,4);
    \fill[orange!30] (5,0) rectangle (7.5,2);
    \fill[orange!30] (2.5,2) rectangle (5,4);
    \fill[orange!30] (0,4) rectangle (2.5,6);
    \fill[red!30] (7.5,0) rectangle (10,2);
    \fill[red!30] (5,2) rectangle (7.5,4);
    \fill[red!30] (2.5,4) rectangle (5,6);
    \fill[red!40] (7.5,2) rectangle (10,4);
    \fill[red!40] (5,4) rectangle (7.5,6);
    \fill[red!50] (7.5,4) rectangle (10,6);
    
    % Risk items
    \node[cell, fill=white, draw=MainRed, thick, rounded corners] at (8.75,5.4) {\textbf{بازگشت اقتدار}\\(سپاه)};
    \node[cell, fill=white, draw=MainRed, thick, rounded corners] at (6.25,5.4) {\textbf{فروپاشی}\\اقتصادی};
    \node[cell, fill=white, draw=MainOrange, thick, rounded corners] at (6.25,3) {\textbf{خشونت}\\فرقه‌ای};
    \node[cell, fill=white, draw=MainOrange, thick, rounded corners] at (3.75,5.4) {\textbf{تجزیه}\\ارضی};
    \node[cell, fill=white, draw=MainOrange, thick, rounded corners] at (8.75,3) {\textbf{مصادره}\\گذار};
    \node[cell, fill=white, draw=MainYellow, thick, rounded corners] at (6.25,1.2) {\textbf{خستگی}\\بین‌المللی};
    \node[cell, fill=white, draw=MainYellow, thick, rounded corners] at (3.75,3) {\textbf{پوپولیسم}};
    \node[cell, fill=white, draw=MainGreen, thick, rounded corners] at (1.25,3) {\textbf{مداخله}\\نظامی};
    
    % Legend
    \node[anchor=west] at (11,5) {\colorbox{red!50}{\ \ } بحرانی};
    \node[anchor=west] at (11,4) {\colorbox{red!30}{\ \ } بالا};
    \node[anchor=west] at (11,3) {\colorbox{orange!30}{\ \ } متوسط};
    \node[anchor=west] at (11,2) {\colorbox{yellow!30}{\ \ } پایین-متوسط};
    \node[anchor=west] at (11,1) {\colorbox{green!20}{\ \ } پایین};
    
\end{tikzpicture}
\caption{نقشه حرارتی ریسک‌های گذار ایران}
\label{fig:risk-heatmap}
\end{figure}

\sectiondivider

% ═══════════════════════════════════════════════════════════════════════════════
\section{ماتریس پاسخ به ریسک}
\label{sec:risk-response-matrix}
% ═══════════════════════════════════════════════════════════════════════════════

جدول زیر برای هر ریسک اصلی، راهکارهای پیشگیری، کشف زودهنگام، و پاسخ را خلاصه می‌کند:

\begin{landscape}
\begin{table}[htbp]
\centering
\bigtablefontsize
\caption{ماتریس جامع پاسخ به ریسک}
\label{tab:risk-response-matrix}
\begin{tabularx}{\linewidth}{>{\raggedleft\arraybackslash}p{2cm}
                             >{\raggedleft\arraybackslash}p{1.5cm}
                             >{\raggedleft\arraybackslash}p{1.5cm}
                             >{\raggedleft\arraybackslash}X
                             >{\raggedleft\arraybackslash}X
                             >{\raggedleft\arraybackslash}X
                             >{\raggedleft\arraybackslash}p{2cm}}
\toprule
\headerrow ریسک & احتمال & شدت & پیشگیری & کشف زودهنگام & پاسخ & مسئول اصلی \\
\midrule
بازگشت اقتدارگرایی & \riskhigh & \riskhigh & اصلاح سپاه، ممنوعیت قانونی، نظارت امنیتی & پایش جابجایی نظامی، اطلاعات انسانی & فشار فوری بین‌المللی، تحریم، انزوا & شورای امنیت \\
\altrow تجزیه ارضی & \riskmedium & \riskhigh & فدرالیسم نامتقارن، حقوق اقوام، مشارکت & پایش تنش‌های قومی، رسانه‌های محلی & میانجیگری فوری، مذاکره، حفظ صلح & \lr{SRSG} \\
خشونت فرقه‌ای & \riskmedium & \riskhigh & عدالت انتقالی متوازن، گفتگوی ملی & گزارش خشونت، نشانه‌های تحریک & نیروی حفظ صلح، میانجیگری، عدالت & \lr{OHCHR} \\
\altrow فروپاشی اقتصادی & \riskmedium & \riskhigh & رفع تحریم سریع، بسته کمک، صندوق امانی & شاخص‌های اقتصادی، نرخ ارز & تزریق نقدینگی، کمک غذایی فوری & \lr{IMF/WB} \\
مصادره گذار & \riskhigh & \riskmedium & ائتلاف فراگیر، قانون تناسبی، نظارت بر فراگیری & پایش تمرکز قدرت، شکایات احزاب & فشار دیپلماتیک، مشروط‌سازی کمک & \lr{EU/UN} \\
\altrow پوپولیسم & \riskmedium & \riskmedium & آموزش شهروندی، رسانه مستقل، نهادهای قوی & نظرسنجی، تحلیل گفتمان & افشاگری، حمایت از رسانه، آموزش & جامعه مدنی \\
خستگی بین‌المللی & \riskmedium & \riskmedium & تعهدات چندساله، پیوند به منافع ملی & کاهش بودجه، کاهش پوشش رسانه‌ای & کمپین یادآوری، لابی دیاسپورا & دولت انتقالی \\
\altrow الیگارشی & \riskmedium & \riskmedium & شفافیت مالکیت، ضدانحصار، خصوصی‌سازی تدریجی & تمرکز ثروت، معاملات مشکوک & تحقیق، مصادره، اصلاح قوانین & \lr{TI/WB} \\
ناآشنایی فرهنگی & \riskhigh & \risklow & آموزش اجباری، تیم مختلط، مشاوران محلی & شکایات، سوءتفاهم‌ها & بازآموزی، جایگزینی & \lr{SRSG} \\
\altrow فساد ناظران & \risklow & \riskmedium & غربالگری، کد رفتاری، نظارت بر ناظر & گزارش‌های افشاگر، شکایات & اخراج، پیگرد، جبران & بازرسی \lr{UN} \\
\bottomrule
\end{tabularx}
\end{table}
\end{landscape}

\sectiondivider

% ═══════════════════════════════════════════════════════════════════════════════
\section{نظام هشدار زودهنگام}
\label{sec:early-warning}
% ═══════════════════════════════════════════════════════════════════════════════

برای مدیریت مؤثر ریسک، نظام \bilingual{هشدار زودهنگام}{Early Warning System} ضروری است. این نظام باید شاخص‌های پیشرو را پایش کند و قبل از بحران، هشدار دهد.

\subsection{شاخص‌های پیشرو برای هر دسته ریسک}
\label{subsec:leading-indicators}

\begin{table}[htbp]
\centering
\caption{شاخص‌های هشدار زودهنگام}
\label{tab:early-warning-indicators}
\begin{tabularx}{\textwidth}{>{\raggedleft\arraybackslash}p{2.5cm}
                             >{\raggedleft\arraybackslash}X
                             >{\raggedleft\arraybackslash}p{2.5cm}
                             >{\raggedleft\arraybackslash}p{2cm}}
\toprule
\headerrow دسته ریسک & شاخص‌های کلیدی & منبع داده & فرکانس پایش \\
\midrule
امنیتی & جابجایی نیروها، خرید سلاح، بیانیه‌های تهدیدآمیز، ترور & اطلاعاتی، رسانه، \lr{OSINT} & روزانه \\
\altrow سیاسی & نظرسنجی اعتماد، شکایات انتخاباتی، خشونت سیاسی & نظرسنجی، گزارش احزاب & هفتگی \\
اقتصادی & نرخ ارز، تورم، بیکاری، صف نان/بنزین & بانک مرکزی، میدانی & روزانه \\
\altrow اجتماعی & تنش‌های قومی، خشونت محلی، مهاجرت داخلی & گزارش‌های محلی، \lr{UNHCR} & هفتگی \\
نظارتی & شکایات از ناظران، تأخیر در تصمیم‌گیری & سامانه شکایت & ماهانه \\
\bottomrule
\end{tabularx}
\end{table}

\subsection{ساختار نظام هشدار}
\label{subsec:warning-structure}

\begin{figure}[htbp]
\centering
\begin{tikzpicture}[
    node distance=1.5cm,
    every node/.style={font=\small},
    source/.style={rectangle, rounded corners, draw=MainBlue, fill=LightBlue, minimum width=2cm, minimum height=0.8cm, align=center},
    process/.style={rectangle, rounded corners, draw=MainOrange, fill=LightOrange, minimum width=2.5cm, minimum height=0.8cm, align=center},
    output/.style={rectangle, rounded corners, draw=MainRed, fill=LightRed, minimum width=2cm, minimum height=0.8cm, align=center},
    arrow/.style={-{Stealth[length=2.5mm]}, thick}
]
    % Sources
    \node[source] (s1) {منابع اطلاعاتی};
    \node[source, below=0.5cm of s1] (s2) {گزارش‌های میدانی};
    \node[source, below=0.5cm of s2] (s3) {رسانه‌ها و \lr{OSINT}};
    \node[source, below=0.5cm of s3] (s4) {نظرسنجی‌ها};
    \node[source, below=0.5cm of s4] (s5) {شکایات شهروندی};
    
    % Processing
    \node[process, right=2cm of s3] (collect) {جمع‌آوری\\و تأیید};
    \node[process, right=1.5cm of collect] (analyze) {تحلیل\\و امتیازدهی};
    \node[process, right=1.5cm of analyze] (decide) {تصمیم‌گیری\\و اولویت‌بندی};
    
    % Outputs
    \node[output, right=2cm of decide] (green) {\textcolor{MainGreen}{\faCheckCircle} عادی};
    \node[output, above=0.3cm of green] (yellow) {\textcolor{MainYellow}{\faExclamationCircle} هشدار};
    \node[output, above=0.3cm of yellow] (red) {\textcolor{MainRed}{\faExclamationTriangle} بحران};
    
    % Arrows
    \draw[arrow] (s1.east) -- ++(0.5,0) |- (collect.west);
    \draw[arrow] (s2.east) -- ++(0.5,0) |- (collect.west);
    \draw[arrow] (s3.east) -- (collect.west);
    \draw[arrow] (s4.east) -- ++(0.5,0) |- (collect.west);
    \draw[arrow] (s5.east) -- ++(0.5,0) |- (collect.west);
    
    \draw[arrow] (collect) -- (analyze);
    \draw[arrow] (analyze) -- (decide);
    
    \draw[arrow] (decide.east) -- ++(0.5,0) |- (green.west);
    \draw[arrow] (decide.east) -- ++(0.5,0) |- (yellow.west);
    \draw[arrow] (decide.east) -- ++(0.5,0) |- (red.west);
    
\end{tikzpicture}
\caption{ساختار نظام هشدار زودهنگام}
\label{fig:early-warning-structure}
\end{figure}

\begin{keypoint}
نظام هشدار زودهنگام باید:
\begin{itemize}[nosep]
    \item \textbf{چندمنبعی} باشد (اتکا به یک منبع خطرناک است)
    \item \textbf{محلی‌محور} باشد (تهران‌محوری کافی نیست)
    \item \textbf{سریع} باشد (تأخیر = فاجعه)
    \item \textbf{مستقل} باشد (تحت فشار سیاسی نباشد)
    \item \textbf{عملیاتی} باشد (هشدار بدون پاسخ بی‌فایده است)
\end{itemize}
\end{keypoint}

\sectiondivider

% ═══════════════════════════════════════════════════════════════════════════════
\section{سطوح پاسخ به بحران}
\label{sec:crisis-response-levels}
% ═══════════════════════════════════════════════════════════════════════════════

برای هر سطح هشدار، پروتکل پاسخ از پیش تعریف‌شده باید وجود داشته باشد:

\begin{table}[htbp]
\centering
\caption{پروتکل پاسخ به سطوح مختلف بحران}
\label{tab:crisis-protocol}
\begin{tabularx}{\textwidth}{>{\raggedleft\arraybackslash}p{2cm}
                             >{\raggedleft\arraybackslash}p{2cm}
                             >{\raggedleft\arraybackslash}X
                             >{\raggedleft\arraybackslash}X
                             >{\raggedleft\arraybackslash}p{2.5cm}}
\toprule
\headerrow سطح & رنگ & شرایط فعال‌سازی & اقدامات & تصمیم‌گیر \\
\midrule
۱ & \cellgreen{سبز} & شاخص‌ها در محدوده عادی & پایش روتین، گزارش ماهانه & تیم پایش \\
\altrow ۲ & \cellorange{زرد} & یک شاخص در آستانه هشدار & افزایش فرکانس پایش، تحلیل علت & رئیس بخش \\
۳ & \cellorange{نارنجی} & چند شاخص در آستانه یا یک شاخص بحرانی & فعال‌سازی تیم بحران، گزارش فوری & معاون \lr{SRSG} \\
\altrow ۴ & \cellred{قرمز} & بحران فعال یا قریب‌الوقوع & جلسه اضطراری، اقدام میدانی، اطلاع به \lr{SC} & \lr{SRSG} \\
۵ & \cellred{قرمز تیره} & فاجعه گسترده & درخواست مداخله، تخلیه، بازنگری کل مأموریت & دبیرکل/\lr{SC} \\
\bottomrule
\end{tabularx}
\end{table}

\begin{warningbox}
مهم‌ترین درس از شکست‌های گذشته (رواندا ۱۹۹۴، سربرنیتسا ۱۹۹۵): هشدار بدون پاسخ بی‌فایده است. در هر دو مورد، هشدارهای کافی وجود داشت اما اراده سیاسی برای اقدام نبود. ساختار نظارت باید شامل تعهد قبلی به اقدام در صورت هشدار باشد.
\end{warningbox}

\sectiondivider

% ═══════════════════════════════════════════════════════════════════════════════
\section{راهکارهای کلان کاهش ریسک}
\label{sec:macro-risk-mitigation}
% ═══════════════════════════════════════════════════════════════════════════════

فراتر از پاسخ به ریسک‌های خاص، راهکارهای کلانی وجود دارد که مجموعه ریسک‌ها را کاهش می‌دهد:

\subsection{تقویت تاب‌آوری سیستمی}
\label{subsec:systemic-resilience}

\bilingual{تاب‌آوری}{Resilience} توانایی سیستم برای جذب شوک و بازیابی است. گذار تاب‌آور گذاری است که یک بحران آن را به عقب نمی‌راند.

\textbf{عناصر تاب‌آوری:}
\begin{enumerate}[nosep]
    \item \textbf{تنوع}: عدم اتکا به یک نهاد، یک رهبر، یک منبع مالی
    \item \textbf{افزونگی}: پشتیبان برای هر عنصر حیاتی
    \item \textbf{مدولاریت}: توانایی عملکرد بخش‌ها به‌صورت مستقل در بحران
    \item \textbf{انطباق‌پذیری}: ظرفیت تغییر سریع استراتژی
    \item \textbf{یادگیری}: مکانیزم درس‌آموزی از اشتباهات
\end{enumerate}

\subsection{ایجاد ذی‌نفعان متعدد در موفقیت}
\label{subsec:multiple-stakeholders}

هرچه افراد و گروه‌های بیشتری منفعت در موفقیت گذار داشته باشند، شکست دادن آن سخت‌تر است.

\begin{recommendation}
استراتژی «ذی‌نفع‌سازی گسترده»:
\begin{itemize}[nosep]
    \item توزیع فرصت‌های شغلی در مناطق مختلف
    \item سهام‌دار کردن کارگران در صنایع خصوصی‌شده
    \item مشارکت محلی در پروژه‌های بازسازی
    \item پیوند منافع اقتصادی طبقه متوسط به ثبات دموکراتیک
    \item ایجاد انگیزه برای بازگشت دیاسپورا
\end{itemize}
\end{recommendation}

\subsection{حفظ انعطاف استراتژیک}
\label{subsec:strategic-flexibility}

\begin{lessonlearned}{تونس: انعطاف نجات‌بخش}
وقتی پیش‌نویس اول قانون اساسی تونس با اعتراض روبرو شد، به‌جای اصرار، فرایند بازنگری شد. این انعطاف که برخی آن را «ضعف» می‌خواندند، در نهایت به اجماع گسترده‌تر انجامید. در مقابل، اصرار مرسی در مصر بر قانون اساسی اسلام‌گرایانه، زمینه‌ساز کودتا شد.
\end{lessonlearned}

\textbf{اصول انعطاف استراتژیک:}
\begin{itemize}[nosep]
    \item اهداف غیرقابل مذاکره (دموکراسی، حقوق بشر) ثابت
    \item روش‌ها و زمان‌بندی قابل تعدیل
    \item بازنگری دوره‌ای استراتژی (هر ۶ ماه)
    \item پذیرش اشتباه و اصلاح مسیر
\end{itemize}

\sectiondivider

% ═══════════════════════════════════════════════════════════════════════════════
\section{تحلیل سناریوی بدترین حالت}
\label{sec:worst-case-scenario}
% ═══════════════════════════════════════════════════════════════════════════════

برنامه‌ریزی برای بدترین حالت، نه نشانه بدبینی، بلکه الزام حرفه‌ای است.

\subsection{سناریوی فاجعه: «سوریه‌ای شدن»}
\label{subsec:syria-scenario}

\begin{warningbox}
\textbf{سناریوی فاجعه:} گذار با فروپاشی ناگهانی آغاز می‌شود ← خلأ امنیتی ← سپاه در برخی مناطق مقاومت می‌کند ← گروه‌های قومی مسلح می‌شوند ← مداخله کشورهای منطقه (عربستان، ترکیه، اسرائیل) ← جنگ داخلی تمام‌عیار ← میلیون‌ها آواره ← ظهور گروه‌های تروریستی ← فروپاشی کامل دولت ← بحران بشردوستانه عظیم.

\textbf{احتمال:} پایین (۵-۱۰٪) اما نه صفر
\textbf{پیامد:} فاجعه‌بار برای ایران، منطقه و جهان
\end{warningbox}

\textbf{عوامل جلوگیری:}
\begin{enumerate}[nosep]
    \item اجماع بین‌المللی قوی قبل از گذار
    \item آمادگی قبلی نیروهای جایگزین امنیتی
    \item توافق با بخشی از سپاه (شکستن یکپارچگی)
    \item مداخله سریع بشردوستانه و امنیتی
    \item جلوگیری از تسلیح گروه‌های غیردولتی
\end{enumerate}

\subsection{طرح اضطراری}
\label{subsec:contingency-plan}

\begin{table}[htbp]
\centering
\caption{اقدامات اضطراری در سناریوی فاجعه}
\label{tab:contingency-actions}
\begin{tabularx}{\textwidth}{>{\raggedleft\arraybackslash}p{2.5cm}
                             >{\raggedleft\arraybackslash}X
                             >{\raggedleft\arraybackslash}p{3cm}
                             >{\raggedleft\arraybackslash}p{2.5cm}}
\toprule
\headerrow بازه زمانی & اقدام & مسئول & پیش‌نیاز \\
\midrule
ساعات اول & تماس با همه طرف‌ها برای آتش‌بس & دبیرکل \lr{UN} & کانال ارتباطی از قبل \\
\altrow ۲۴-۷۲ ساعت & نشست اضطراری شورای امنیت & اعضای \lr{P5} & پیش‌نویس قطعنامه آماده \\
هفته اول & استقرار ناظران غیرمسلح در مناطق امن & \lr{DPPA} & تیم آماده‌باش \\
\altrow هفته ۱-۴ & کریدور بشردوستانه، کمک غذایی & \lr{OCHA}, \lr{WFP} & پیش‌موقعیت‌یابی کمک‌ها \\
ماه ۱-۳ & استقرار نیروی حفظ صلح (در صورت تصویب) & \lr{DPKO} & آمادگی کشورهای مشارکت‌کننده \\
\bottomrule
\end{tabularx}
\end{table}

\sectiondivider

% ═══════════════════════════════════════════════════════════════════════════════
% جمع‌بندی فصل
% ═══════════════════════════════════════════════════════════════════════════════

\begin{chaptersummary}
یافته‌های کلیدی این فصل:

\begin{enumerate}
    \item \textbf{ریسک‌ها اجتناب‌ناپذیرند}: هر گذاری با ریسک همراه است؛ هدف مدیریت است نه حذف.
    
    \item \textbf{بازگشت اقتدارگرایی جدی‌ترین تهدید است}: با حضور سپاه پاسداران، این ریسک در ایران بالاتر از میانگین است و نیاز به توجه ویژه دارد.
    
    \item \textbf{ریسک تجزیه قابل مدیریت است}: برخلاف نگرانی‌های رایج، ایران یوگسلاوی نیست و با سیاست‌های درست، می‌توان وحدت ملی را حفظ کرد.
    
    \item \textbf{اقتصاد می‌تواند ناجی یا قاتل گذار باشد}: رفع سریع تحریم‌ها و بسته کمک بین‌المللی حیاتی است.
    
    \item \textbf{نظارت بین‌المللی خود ریسک‌هایی دارد}: از ناکافی بودن تا فساد و سوگیری که باید با طراحی هوشمند کنترل شوند.
    
    \item \textbf{نظام هشدار زودهنگام ضروری است}: با شاخص‌های تعریف‌شده، منابع متنوع، و پروتکل پاسخ.
    
    \item \textbf{انعطاف استراتژیک کلید بقاست}: اصرار بر طرح‌های از پیش تعیین‌شده می‌تواند فاجعه‌بار باشد.
    
    \item \textbf{طرح اضطراری باید از قبل آماده باشد}: «امیدوار به بهترین، آماده برای بدترین».
\end{enumerate}

\vspace{0.5cm}
\textit{در فصل بعد (\ref{ch:requirements})، نیازمندی‌های انسانی، نهادی، فنی و حقوقی برای اجرای نظارت بین‌المللی مؤثر بررسی خواهد شد.}
\end{chaptersummary}

\chapterend
% ═══════════════════════════════════════════════════════════════════════════════
% فصل ۸: نیازمندی‌ها — انسانی، نهادی، فنی، حقوقی
% فایل: chapters/ch08-requirements.tex
% رنگ فصل: زرد (MainYellow)
% ═══════════════════════════════════════════════════════════════════════════════

\chapteropening{۸}{نیازمندی‌ها: انسانی، نهادی، فنی، حقوقی}{MainYellow}{%
بدون ابزار مناسب، حتی بهترین معمار هم نمی‌تواند خانه‌ای بسازد. نظارت مؤثر نیازمند منابع مؤثر است.%
}{ضرب‌المثل}

\chapter{نیازمندی‌ها: انسانی، نهادی، فنی، حقوقی}
\label{ch:requirements}

\minitoc

% ─────────────────────────────────────────────────────────────────────────────
% خلاصه اجرایی
% ─────────────────────────────────────────────────────────────────────────────

\begin{executivesummary}
اجرای مؤثر نظارت بین‌المللی بر گذار ایران نیازمند چهار دسته منابع است: \emph{نیروی انسانی} (۶,۰۰۰-۱۲,۰۰۰ بین‌المللی و ۲۰,۰۰۰-۵۰,۰۰۰ ایرانی در اوج استقرار)، \emph{ساختارهای نهادی} (از دفتر نماینده ویژه تا کمیسیون‌های تخصصی)، \emph{زیرساخت‌های فنی} (ارتباطات، سامانه‌های اطلاعاتی، امنیت سایبری)، و \emph{چارچوب حقوقی} (قطعنامه‌ها، توافق‌نامه‌ها، قوانین موقت). این فصل برای هر دسته، جزئیات، استانداردها، و برآورد کمّی ارائه می‌دهد. تأکید اصلی بر تعادل میان ظرفیت بین‌المللی و مالکیت ملی، و انتقال تدریجی مسئولیت به نهادهای ایرانی است.
\end{executivesummary}

\section{درآمد: از طراحی تا اجرا}
\label{sec:req-intro}

فصول پیشین چارچوب مفهومی، سناریوها، بازیگران، تضمین‌ها و ریسک‌ها را بررسی کردند. اکنون به پرسش عملیاتی می‌رسیم: \emph{برای اجرای این طرح، دقیقاً چه چیزهایی لازم است؟}

\begin{keypoint}
تفاوت میان طرح‌های موفق و ناموفق، اغلب نه در طراحی بلکه در اجراست. طراحی بدون منابع کافی، آرزوپردازی است. منابع بدون طراحی، هدررفت. این فصل پل میان این دو را می‌سازد.
\end{keypoint}

نیازمندی‌ها در چهار حوزه دسته‌بندی می‌شوند:

\begin{enumerate}[nosep]
    \item \textbf{انسانی}: چه کسانی، چند نفر، با چه تخصص‌هایی
    \item \textbf{نهادی}: چه ساختارها و سازمان‌هایی باید ایجاد شوند
    \item \textbf{فنی}: چه زیرساخت‌ها و ابزارهایی لازم است
    \item \textbf{حقوقی}: چه چارچوب قانونی و توافق‌نامه‌هایی ضروری است
\end{enumerate}

\sectiondivider

% ═══════════════════════════════════════════════════════════════════════════════
\section{نیازمندی‌های انسانی}
\label{sec:human-requirements}
% ═══════════════════════════════════════════════════════════════════════════════

\subsection{برآورد کلی نیروی انسانی}
\label{subsec:hr-overview}

بر اساس تجربه مأموریت‌های مشابه و با در نظر گرفتن ویژگی‌های ایران (جمعیت ۸۵ میلیون، مساحت ۱.۶ میلیون کیلومتر مربع، ۳۱ استان)، برآورد نیروی انسانی به شرح زیر است:

\begin{table}[htbp]
\centering
\caption{برآورد کلی نیروی انسانی بر حسب فاز}
\label{tab:hr-overview}
\begin{tabularx}{\textwidth}{>{\raggedleft\arraybackslash}p{3cm}
                             >{\centering\arraybackslash}p{2.5cm}
                             >{\centering\arraybackslash}p{2.5cm}
                             >{\centering\arraybackslash}p{2.5cm}
                             >{\centering\arraybackslash}X}
\toprule
\headerrow دسته & فاز ۱ (۱-۶ ماه) & فاز ۲ (۶-۲۴ ماه) & فاز ۳ (۲۴-۶۰ ماه) & توضیح \\
\midrule
کارکنان بین‌المللی & ۳,۰۰۰-۵,۰۰۰ & ۶,۰۰۰-۱۲,۰۰۰ & ۱,۵۰۰-۳,۰۰۰ & اوج در فاز ۲ (انتخابات) \\
\altrow کارکنان ایرانی (مستقیم) & ۵,۰۰۰-۱۰,۰۰۰ & ۲۰,۰۰۰-۵۰,۰۰۰ & ۱۰,۰۰۰-۲۰,۰۰۰ & ناظران، مترجمان، پشتیبان \\
ناظران انتخاباتی موقت & --- & ۵۰,۰۰۰-۱۰۰,۰۰۰ & --- & فقط روز انتخابات \\
\altrow مشاوران کوتاه‌مدت & ۵۰۰-۱,۰۰۰ & ۱,۰۰۰-۲,۰۰۰ & ۵۰۰-۱,۰۰۰ & مأموریت‌های خاص \\
\midrule
\textbf{مجموع (بدون موقت)} & \textbf{۸,۵۰۰-۱۶,۰۰۰} & \textbf{۲۷,۰۰۰-۶۴,۰۰۰} & \textbf{۱۲,۰۰۰-۲۴,۰۰۰} & \\
\bottomrule
\end{tabularx}
\end{table}

\begin{lessonlearned}{تیمور شرقی: نسبت جمعیت به ناظر}
در \lr{UNTAET} تیمور شرقی (جمعیت ۸۰۰,۰۰۰)، حدود ۱۱,۰۰۰ کارمند بین‌المللی و محلی حضور داشتند — نسبت ۱ به ۷۳. برای ایران با همین نسبت، بیش از ۱ میلیون نفر لازم بود که نه ممکن است و نه مطلوب. نسبت واقع‌بینانه برای ایران: ۱ به ۱,۵۰۰-۳,۰۰۰ (با تأکید بر ظرفیت‌سازی محلی).
\end{lessonlearned}

\subsection{تفکیک تخصصی نیروی انسانی بین‌المللی}
\label{subsec:hr-specialization}

\begin{table}[htbp]
\centering
\caption{تفکیک تخصصی کارکنان بین‌المللی (فاز ۲ — اوج)}
\label{tab:hr-specialization}
\begin{tabularx}{\textwidth}{>{\raggedleft\arraybackslash}p{0.5cm}
                             >{\raggedleft\arraybackslash}p{3.5cm}
                             >{\centering\arraybackslash}p{2cm}
                             >{\centering\arraybackslash}p{1.5cm}
                             >{\raggedleft\arraybackslash}X}
\toprule
\headerrow \# & حوزه تخصصی & تعداد & درصد & وظایف اصلی \\
\midrule
۱ & نظارت انتخاباتی & ۲,۵۰۰-۵,۰۰۰ & ۴۰٪ & نظارت بر ثبت‌نام، رأی‌گیری، شمارش \\
\altrow ۲ & حقوق بشر & ۸۰۰-۱,۵۰۰ & ۱۲٪ & پایش، مستندسازی، گزارش‌دهی \\
۳ & اصلاح بخش امنیتی & ۵۰۰-۱,۰۰۰ & ۸٪ & مشاوره به ارتش/پلیس، نظارت بر \lr{DDR} \\
\altrow ۴ & حاکمیت قانون/قضایی & ۴۰۰-۸۰۰ & ۶٪ & اصلاح قضایی، عدالت انتقالی \\
۵ & اقتصادی/مالی & ۳۰۰-۶۰۰ & ۵٪ & نظارت مالی، ضدفساد، بانکداری \\
\altrow ۶ & رسانه/ارتباطات & ۲۰۰-۴۰۰ & ۳٪ & نظارت رسانه، آموزش روزنامه‌نگاران \\
۷ & جامعه مدنی/جنسیت & ۲۰۰-۴۰۰ & ۳٪ & توانمندسازی \lr{NGO}ها، زنان، جوانان \\
\altrow ۸ & اداری/مالی/لجستیک & ۸۰۰-۱,۵۰۰ & ۱۲٪ & پشتیبانی عملیاتی \\
۹ & امنیت/حفاظت & ۴۰۰-۸۰۰ & ۶٪ & امنیت کارکنان و تأسیسات \\
\altrow ۱۰ & ارتباطات/\lr{IT} & ۲۰۰-۴۰۰ & ۳٪ & زیرساخت فنی، امنیت سایبری \\
۱۱ & هماهنگی/سیاست‌گذاری & ۲۰۰-۴۰۰ & ۳٪ & دفتر \lr{SRSG}، تحلیل، برنامه‌ریزی \\
\midrule
& \textbf{مجموع} & \textbf{۶,۵۰۰-۱۲,۸۰۰} & \textbf{۱۰۰٪} & \\
\bottomrule
\end{tabularx}
\end{table}

\subsection{معیارهای استخدام کارکنان بین‌المللی}
\label{subsec:hr-criteria}

\begin{recommendation}
معیارهای کلیدی برای انتخاب کارکنان بین‌المللی:
\begin{enumerate}[nosep]
    \item \textbf{تجربه مرتبط}: حداقل ۵ سال در حوزه تخصصی، ترجیحاً در کشورهای مسلمان یا خاورمیانه
    \item \textbf{مهارت زبانی}: فارسی (ترجیحی)، انگلیسی (الزامی)، عربی یا ترکی (مفید)
    \item \textbf{حساسیت فرهنگی}: آموزش اجباری پیش از استقرار + ارزیابی
    \item \textbf{سلامت و آمادگی}: توانایی کار در شرایط سخت
    \item \textbf{تعهد اخلاقی}: پیشینه پاک، امضای کد رفتاری
    \item \textbf{تنوع}: حداقل ۴۰٪ زن، حداکثر ۱۵٪ از هر کشور واحد
\end{enumerate}
\end{recommendation}

\subsection{نیروی انسانی ایرانی}
\label{subsec:iranian-staff}

نیروی انسانی ایرانی ستون فقرات عملیات است و بدون آن، هیچ نظارتی امکان‌پذیر نیست.

\begin{table}[htbp]
\centering
\caption{دسته‌بندی نیروی انسانی ایرانی}
\label{tab:iranian-staff}
\begin{tabularx}{\textwidth}{>{\raggedleft\arraybackslash}p{3cm}
                             >{\centering\arraybackslash}p{2.5cm}
                             >{\raggedleft\arraybackslash}X
                             >{\raggedleft\arraybackslash}p{3cm}}
\toprule
\headerrow دسته & تعداد (فاز ۲) & وظایف & منبع جذب \\
\midrule
مترجمان & ۲,۰۰۰-۴,۰۰۰ & ترجمه همزمان و کتبی & دانشگاه‌ها، دیاسپورا \\
\altrow ناظران محلی (دائم) & ۵,۰۰۰-۱۰,۰۰۰ & پایش روزانه، گزارش‌دهی & جامعه مدنی، دانشجویان \\
ناظران انتخابات (موقت) & ۵۰,۰۰۰-۱۰۰,۰۰۰ & نظارت روز رأی‌گیری & داوطلب عمومی \\
\altrow کارشناسان فنی & ۱,۰۰۰-۲,۰۰۰ & حقوقی، مالی، \lr{IT} & متخصصان داخل و دیاسپورا \\
پشتیبانی اداری & ۳,۰۰۰-۶,۰۰۰ & دفتری، لجستیک، رانندگی & بازار کار محلی \\
\altrow امنیت محلی & ۲,۰۰۰-۴,۰۰۰ & حفاظت تأسیسات & نیروهای امنیتی اصلاح‌شده \\
رابطان اجتماعی & ۱,۰۰۰-۲,۰۰۰ & ارتباط با جوامع محلی & فعالان مدنی، معتمدان \\
\midrule
\textbf{مجموع} & \textbf{۶۴,۰۰۰-۱۲۸,۰۰۰} & & \\
\bottomrule
\end{tabularx}
\end{table}

\begin{warningbox}
\textbf{چالش غربالگری}: چگونه می‌توان اطمینان یافت که کارکنان ایرانی وابسته به نظام قبلی نیستند؟ غربالگری افراطی (مثل بعث‌زدایی عراق) فاجعه‌بار است، اما بدون غربالگری، نفوذ محتمل است.

\textbf{راه‌حل پیشنهادی}: 
\begin{itemize}[nosep]
    \item غربالگری فقط برای پست‌های حساس
    \item تمرکز بر رفتار آینده نه سوابق صرف
    \item دوره آزمایشی با نظارت
    \item مکانیزم گزارش‌دهی و اخراج سریع
\end{itemize}
\end{warningbox}

\subsection{نقش دیاسپورا در تأمین نیروی انسانی}
\label{subsec:diaspora-hr}

دیاسپورای ایرانی (۴-۵ میلیون نفر) منبع ارزشمندی از نیروی متخصص دوزبانه است.

\begin{table}[htbp]
\centering
\caption{ظرفیت‌ها و محدودیت‌های نیروی دیاسپورا}
\label{tab:diaspora-hr}
\begin{tabularx}{\textwidth}{>{\centering\arraybackslash}p{0.8cm}
                             >{\raggedleft\arraybackslash}X
                             >{\raggedleft\arraybackslash}X}
\toprule
\headerrow & مزایا & محدودیت‌ها/ریسک‌ها \\
\midrule
\cmark & تسلط به فارسی و زبان‌های بین‌المللی & فاصله از واقعیت روزمره داخل \\
\altrow \cmark & آشنایی با فرهنگ ایرانی و بین‌المللی & ممکن است وابستگی جناحی داشته باشند \\
\cmark & تحصیلات و تجربه در نهادهای معتبر & انتظارات حقوقی بالاتر \\
\altrow \cmark & انگیزه بالا برای خدمت & ممکن است تنش با نیروی داخلی ایجاد کنند \\
\cmark & شبکه‌های بین‌المللی & تعهد بلندمدت به ماندن نامطمئن \\
\bottomrule
\end{tabularx}
\end{table}

\begin{recommendation}
سیاست پیشنهادی برای دیاسپورا:
\begin{itemize}[nosep]
    \item سقف ۲۰-۳۰٪ از کارکنان ایرانی از دیاسپورا
    \item ترجیح برای پست‌های فنی و مشاوره‌ای (نه اجرایی ارشد)
    \item تعهد حداقل ۲ ساله برای پست‌های کلیدی
    \item مکانیزم انتقال دانش به نیروی داخلی
\end{itemize}
\end{recommendation}

\sectiondivider

% ═══════════════════════════════════════════════════════════════════════════════
\section{نیازمندی‌های نهادی}
\label{sec:institutional-requirements}
% ═══════════════════════════════════════════════════════════════════════════════

ساختارهای نهادی چارچوبی فراهم می‌آورند که در آن افراد می‌توانند مؤثر عمل کنند.

\subsection{ساختار کلان نهادی}
\label{subsec:institutional-structure}

\begin{figure}[htbp]
\centering
\begin{tikzpicture}[
    node distance=1.2cm,
    every node/.style={font=\small, align=center},
    top/.style={rectangle, rounded corners, draw=MainPurple, fill=LightPurple, minimum width=4cm, minimum height=1cm},
    mid/.style={rectangle, rounded corners, draw=MainBlue, fill=LightBlue, minimum width=3.5cm, minimum height=0.9cm},
    low/.style={rectangle, rounded corners, draw=MainGreen, fill=LightGreen, minimum width=3cm, minimum height=0.8cm},
    support/.style={rectangle, rounded corners, draw=MainOrange, fill=LightOrange, minimum width=2.5cm, minimum height=0.8cm},
    arrow/.style={-{Stealth[length=2.5mm]}, thick}
]
    % Top level
    \node[top] (sc) {شورای امنیت \lr{UN}};
    \node[top, right=2cm of sc] (sg) {دبیرکل \lr{UN}};
    
    % SRSG level
    \node[mid, below=1.5cm of sg] (srsg) {\lr{SRSG}\\نماینده ویژه دبیرکل};
    
    % Deputy level
    \node[low, below left=1.5cm and 2cm of srsg] (d1) {معاون سیاسی};
    \node[low, below=1.5cm of srsg] (d2) {معاون عملیاتی};
    \node[low, below right=1.5cm and 2cm of srsg] (d3) {معاون حقوق بشر};
    
    % Pillars
    \node[support, below=1.2cm of d1] (p1) {انتخابات};
    \node[support, left=0.3cm of p1] (p2) {امنیت};
    \node[support, below=1.2cm of d2] (p3) {اقتصاد};
    \node[support, right=0.3cm of p3] (p4) {قضایی};
    \node[support, below=1.2cm of d3] (p5) {رسانه};
    \node[support, right=0.3cm of p5] (p6) {مدنی};
    
    % Arrows
    \draw[arrow] (sc) -- (sg);
    \draw[arrow] (sg) -- (srsg);
    \draw[arrow] (srsg) -- (d1);
    \draw[arrow] (srsg) -- (d2);
    \draw[arrow] (srsg) -- (d3);
    \draw[arrow] (d1) -- (p1);
    \draw[arrow] (d1) -- (p2);
    \draw[arrow] (d2) -- (p3);
    \draw[arrow] (d2) -- (p4);
    \draw[arrow] (d3) -- (p5);
    \draw[arrow] (d3) -- (p6);
    
    % Side boxes
    \node[mid, right=3cm of srsg] (contact) {گروه تماس\\بین‌المللی};
    \node[mid, below=0.5cm of contact] (consult) {شورای مشورتی\\ایرانی};
    
    \draw[arrow, dashed] (srsg) -- (contact);
    \draw[arrow, dashed] (srsg) -- (consult);
    
\end{tikzpicture}
\caption{ساختار کلان نهادی نظارت بین‌المللی}
\label{fig:institutional-structure}
\end{figure}

\subsection{نهادهای اصلی مورد نیاز}
\label{subsec:key-institutions}

\subsubsection{دفتر نماینده ویژه دبیرکل (\lr{SRSG})}

\begin{table}[htbp]
\centering
\caption{مشخصات دفتر \lr{SRSG}}
\label{tab:srsg-office}
\begin{tabularx}{\textwidth}{>{\raggedleft\arraybackslash}p{3.5cm}
                             >{\raggedleft\arraybackslash}X}
\toprule
\headerrow ویژگی & شرح \\
\midrule
عنوان رسمی & \lr{United Nations Mission for Oversight of Iran's Transition (UNMOIT)} \\
\altrow رئیس & نماینده ویژه دبیرکل (\lr{SRSG}) در سطح معاون دبیرکل \\
معاونان & ۳-۵ معاون (سیاسی، عملیاتی، حقوق بشر، هماهنگی، اداری) \\
\altrow کارکنان دفتر مرکزی & ۳۰۰-۵۰۰ نفر \\
دفاتر منطقه‌ای & ۳۱ دفتر استانی + ۵-۷ دفتر منطقه‌ای \\
\altrow محل استقرار مرکزی & تهران (ترجیحاً در ساختمان‌های موجود، نه پایگاه ایزوله) \\
مدت مأموریت اولیه & ۲ سال با امکان تمدید \\
\altrow گزارش‌دهی & دبیرکل → شورای امنیت (ماهانه) \\
\bottomrule
\end{tabularx}
\end{table}

\begin{casestudy}{انتخاب \lr{SRSG}: درس‌های گذشته}
انتخاب \lr{SRSG} تأثیر عمیقی بر موفقیت مأموریت دارد:
\begin{itemize}[nosep]
    \item \textbf{موفق}: \person{سرجیو ویرا دملو}{Sergio Vieira de Mello} در تیمور شرقی — دیپلمات باتجربه، کاریزماتیک، محترم
    \item \textbf{ناموفق}: \person{پل برمر}{Paul Bremer} در عراق — بدون تجربه منطقه، تصمیمات یکجانبه
\end{itemize}
\textbf{معیارهای پیشنهادی}: تجربه خاورمیانه/جهان اسلام، زبان (ترجیحاً فارسی یا عربی)، اعتبار در میان همه طرف‌ها، تعهد به مالکیت ملی، سابقه موفق در مأموریت‌های مشابه.
\end{casestudy}

\subsubsection{کمیسیون مستقل انتخابات}

\begin{table}[htbp]
\centering
\caption{ساختار کمیسیون انتخابات}
\label{tab:election-commission}
\begin{tabularx}{\textwidth}{>{\raggedleft\arraybackslash}p{3cm}
                             >{\raggedleft\arraybackslash}X}
\toprule
\headerrow عنصر & شرح \\
\midrule
ترکیب & ۷-۱۱ عضو: اکثریت ایرانی + ۲-۳ بین‌المللی (بدون حق رأی) \\
\altrow انتخاب اعضا & پیشنهاد شورای مشورتی، تأیید مجلس موقت/انتقالی \\
استقلال & بودجه مستقل، مصونیت از فشار سیاسی \\
\altrow وظایف & ثبت رأی‌دهندگان، تأیید نامزدها، اجرای انتخابات، رسیدگی به شکایات \\
کارکنان & ۵,۰۰۰-۱۰,۰۰۰ دائم + ۱۰۰,۰۰۰+ موقت (روز انتخابات) \\
\altrow نظارت بین‌المللی & مشاوران \lr{UNDP}/\lr{IFES} در همه سطوح \\
\bottomrule
\end{tabularx}
\end{table}

\subsubsection{کمیسیون حقیقت‌یابی و آشتی}

\begin{keypoint}
کمیسیون حقیقت نه جایگزین دادگاه است و نه مانع آن. وظیفه‌اش کشف حقیقت، شنیدن صدای قربانیان، و ایجاد روایت مشترک ملی است. تجربه آفریقای جنوبی نشان داد که عفو مشروط (در ازای اعتراف کامل) می‌تواند به آشتی کمک کند.
\end{keypoint}

\begin{table}[htbp]
\centering
\caption{ساختار کمیسیون حقیقت‌یابی}
\label{tab:truth-commission}
\begin{tabularx}{\textwidth}{>{\raggedleft\arraybackslash}p{3cm}
                             >{\raggedleft\arraybackslash}X}
\toprule
\headerrow عنصر & شرح \\
\midrule
ترکیب & ۱۵-۲۱ عضو (کاملاً ایرانی) + هیئت مشاوران بین‌المللی \\
\altrow معیار انتخاب & اعتبار اخلاقی، استقلال از همه جناح‌ها، تخصص (حقوقدان، روانشناس، مورخ، فعال حقوق بشر) \\
محدوده زمانی & بررسی نقض حقوق بشر از ۱۳۵۷ تا پایان رژیم \\
\altrow مدت فعالیت & ۳-۵ سال \\
اختیارات & احضار شهود، دسترسی به اسناد، توصیه عفو مشروط \\
\altrow خروجی & گزارش نهایی عمومی + توصیه‌ها برای جبران و اصلاحات \\
\bottomrule
\end{tabularx}
\end{table}

\subsubsection{دادگاه ویژه برای جنایات سنگین}

\begin{warningbox}
دادگاه‌های ویژه می‌توانند به عدالت یا انتقام بینجامند. تفاوت در طراحی است:
\begin{itemize}[nosep]
    \item \textbf{موفق}: دادگاه‌های رواندا و سیرالئون — استانداردهای بین‌المللی، وکیل مدافع، شفافیت
    \item \textbf{ناموفق}: دادگاه‌های انقلابی ایران ۱۳۵۷ — بدون وکیل، محاکمات چنددقیقه‌ای، اعدام فوری
\end{itemize}
\end{warningbox}

\begin{table}[htbp]
\centering
\caption{ساختار دادگاه ویژه}
\label{tab:special-tribunal}
\begin{tabularx}{\textwidth}{>{\raggedleft\arraybackslash}p{3cm}
                             >{\raggedleft\arraybackslash}X}
\toprule
\headerrow عنصر & شرح \\
\midrule
صلاحیت & جنایات علیه بشریت، شکنجه سیستماتیک، کشتار جمعی (۱۳۶۷ و...) \\
\altrow ترکیب قضات & ترکیبی: ۳ ایرانی + ۲ بین‌المللی در هر شعبه \\
دادستان & ایرانی با معاون بین‌المللی \\
\altrow استانداردها & مطابق اساسنامه رم (حقوق متهم، وکیل، استیناف) \\
تعداد پرونده & تمرکز بر ۱۰۰-۵۰۰ مسئول اصلی (نه هزاران نفر) \\
\altrow مجازات‌ها & حبس (بدون اعدام — مطابق استانداردهای بین‌المللی) \\
\bottomrule
\end{tabularx}
\end{table}

\subsubsection{صندوق امانی بین‌المللی}

\begin{table}[htbp]
\centering
\caption{ساختار صندوق امانی}
\label{tab:trust-fund}
\begin{tabularx}{\textwidth}{>{\raggedleft\arraybackslash}p{3cm}
                             >{\raggedleft\arraybackslash}X}
\toprule
\headerrow عنصر & شرح \\
\midrule
هدف & مدیریت شفاف کمک‌های بین‌المللی و دارایی‌های آزادشده ایران \\
\altrow مدیریت & هیئت امنای مشترک (ایرانی-بین‌المللی) \\
حسابرس & شرکت حسابرسی بین‌المللی مستقل \\
\altrow شفافیت & انتشار ماهانه گزارش مالی عمومی \\
حوزه‌های هزینه & بازسازی، انتخابات، عدالت انتقالی، خدمات عمومی \\
\altrow کنترل & هیچ برداشتی بدون تأیید مشترک ایرانی-بین‌المللی \\
\bottomrule
\end{tabularx}
\end{table}

\subsubsection{سایر نهادهای ضروری}

\begin{itemize}[nosep]
    \item \textbf{شورای مشورتی ایرانی}: نمایندگان همه طیف‌ها، مشورت با \lr{SRSG}
    \item \textbf{کمیسیون رسانه}: تنظیم‌گری، صدور مجوز، نظارت بر تعادل
    \item \textbf{کمیسیون ضدفساد}: پیشگیری، تحقیق، آموزش
    \item \textbf{نهاد حقوق اقوام}: تضمین حقوق زبانی-فرهنگی
    \item \textbf{دفتر جبران خسارت قربانیان}: مالی و نمادین
\end{itemize}

\sectiondivider

% ═══════════════════════════════════════════════════════════════════════════════
\section{نیازمندی‌های فنی}
\label{sec:technical-requirements}
% ═══════════════════════════════════════════════════════════════════════════════

\subsection{زیرساخت ارتباطات}
\label{subsec:communications}

\begin{table}[htbp]
\centering
\caption{نیازمندی‌های ارتباطی}
\label{tab:comms-requirements}
\begin{tabularx}{\textwidth}{>{\raggedleft\arraybackslash}p{3cm}
                             >{\raggedleft\arraybackslash}X
                             >{\raggedleft\arraybackslash}p{3cm}}
\toprule
\headerrow حوزه & نیاز & اولویت \\
\midrule
اینترنت & رفع فیلترینگ، پهنای باند کافی & فوری (روز ۱) \\
\altrow تلفن همراه & پوشش سراسری، امنیت مکالمات & فوری \\
شبکه داخلی \lr{UN} & \lr{VPN} امن، ویدئوکنفرانس & هفته اول \\
\altrow رادیو & ارتباط اضطراری، مناطق دوردست & ماه اول \\
ماهواره & پشتیبان اینترنت، مناطق بدون پوشش & ماه اول \\
\bottomrule
\end{tabularx}
\end{table}

\begin{lessonlearned}{افغانستان: اهمیت ارتباطات}
در انتخابات ۲۰۰۴ افغانستان، قطع ارتباط با صدها ایستگاه رأی‌گیری در مناطق دوردست، امکان تقلب را فراهم کرد و اعتماد به نتایج را تضعیف کرد. سیستم ارتباط چندلایه (اینترنت + تلفن + رادیو + ماهواره) ضروری است.
\end{lessonlearned}

\subsection{سامانه‌های اطلاعاتی}
\label{subsec:information-systems}

\begin{table}[htbp]
\centering
\caption{سامانه‌های اطلاعاتی مورد نیاز}
\label{tab:info-systems}
\begin{tabularx}{\textwidth}{>{\raggedleft\arraybackslash}p{3.5cm}
                             >{\raggedleft\arraybackslash}X
                             >{\raggedleft\arraybackslash}p{2.5cm}}
\toprule
\headerrow سامانه & کارکرد & فاز استقرار \\
\midrule
ثبت رأی‌دهندگان & پایگاه داده ۶۰+ میلیون واجد شرایط & فاز ۱ \\
\altrow مدیریت انتخابات & برنامه‌ریزی، توزیع مواد، گزارش‌دهی & فاز ۱-۲ \\
مستندسازی حقوق بشر & ثبت شهادت‌ها، شواهد، پرونده‌ها & فاز ۱ \\
\altrow هشدار زودهنگام & پایش شاخص‌ها، تحلیل ریسک & فاز ۱ \\
مدیریت مالی & حسابداری، شفافیت، گزارش‌دهی & فاز ۱ \\
\altrow مدیریت منابع انسانی & استخدام، آموزش، ارزیابی & فاز ۱ \\
گزارش‌دهی عمومی & اطلاع‌رسانی به مردم و رسانه‌ها & فاز ۱ \\
\bottomrule
\end{tabularx}
\end{table}

\subsection{امنیت سایبری}
\label{subsec:cybersecurity}

\begin{warningbox}
ایران دارای ظرفیت سایبری تهاجمی قابل‌توجه است (گروه‌هایی مثل \lr{APT33}, \lr{APT34}). در صورت گذار، عناصر مقاوم ممکن است از حملات سایبری برای اختلال استفاده کنند:
\begin{itemize}[nosep]
    \item حمله به سامانه انتخابات
    \item نشت اطلاعات محرمانه شهود
    \item اختلال در ارتباطات
    \item انتشار اطلاعات جعلی
\end{itemize}
\end{warningbox}

\textbf{راهکارهای امنیت سایبری:}
\begin{enumerate}[nosep]
    \item سامانه‌های حیاتی آفلاین یا ایزوله
    \item رمزنگاری سرتاسری همه ارتباطات
    \item پشتیبان‌گیری چندلایه
    \item تیم واکنش سریع سایبری (\lr{CERT})
    \item همکاری با شرکت‌های امنیتی معتبر
    \item آموزش امنیت سایبری به همه کارکنان
\end{enumerate}

\subsection{فناوری انتخابات}
\label{subsec:election-tech}

\begin{keypoint}
انتخاب میان رأی‌گیری کاغذی و الکترونیکی تصمیمی استراتژیک است. در شرایط گذار با اعتماد پایین، \textbf{رأی کاغذی با شمارش علنی} معمولاً امن‌تر و قابل‌اعتمادتر است.
\end{keypoint}

\begin{table}[htbp]
\centering
\caption{مقایسه گزینه‌های فناوری انتخابات}
\label{tab:election-tech}
\begin{tabularx}{\textwidth}{>{\raggedleft\arraybackslash}p{2.5cm}
                             >{\raggedleft\arraybackslash}X
                             >{\raggedleft\arraybackslash}X
                             >{\centering\arraybackslash}p{2cm}}
\toprule
\headerrow گزینه & مزایا & معایب & توصیه \\
\midrule
کاملاً کاغذی & شفافیت، سادگی، قابل شمارش مجدد & کُند، خطای انسانی & \cmark توصیه \\
\altrow الکترونیک + کاغذ & سرعت، کاهش خطا، پشتیبان کاغذی & هزینه، پیچیدگی & قابل بررسی \\
کاملاً الکترونیک & سرعت بالا & ریسک هک، عدم قابلیت بازشماری & \xmark رد \\
\altrow رأی اینترنتی & دسترسی آسان & ریسک امنیتی بالا & \xmark رد \\
\bottomrule
\end{tabularx}
\end{table}

\sectiondivider

% ═══════════════════════════════════════════════════════════════════════════════
\section{نیازمندی‌های حقوقی}
\label{sec:legal-requirements}
% ═══════════════════════════════════════════════════════════════════════════════

\subsection{چارچوب بین‌المللی}
\label{subsec:international-legal}

\subsubsection{قطعنامه شورای امنیت}

\begin{table}[htbp]
\centering
\caption{عناصر کلیدی قطعنامه پیشنهادی شورای امنیت}
\label{tab:unsc-resolution}
\begin{tabularx}{\textwidth}{>{\raggedleft\arraybackslash}p{3cm}
                             >{\raggedleft\arraybackslash}X}
\toprule
\headerrow عنصر & محتوا \\
\midrule
فصل & ترجیحاً فصل ۶ (توافقی)، در صورت نیاز فصل ۷ (الزامی) \\
\altrow مأموریت & ایجاد \lr{UNMOIT}، تعیین اختیارات، مدت، گزارش‌دهی \\
اصول & تأکید بر مالکیت ملی، تمامیت ارضی، عدم مداخله \\
\altrow تحریم‌ها & تعلیق یا لغو تحریم‌های قبلی مشروط به پیشرفت \\
تعهدات کشورها & همکاری، عدم مداخله یکجانبه، کمک مالی \\
\altrow مکانیزم بازنگری & بررسی هر ۶ ماه، امکان تمدید یا تعدیل \\
\bottomrule
\end{tabularx}
\end{table}

\subsubsection{توافق‌نامه وضعیت مأموریت (\lr{SOMA})}

\bilingual{توافق‌نامه وضعیت مأموریت}{Status of Mission Agreement — SOMA} میان \lr{UN} و دولت انتقالی ایران، چارچوب حقوقی حضور بین‌المللی را تعیین می‌کند.

\begin{table}[htbp]
\centering
\caption{عناصر کلیدی \lr{SOMA}}
\label{tab:soma}
\begin{tabularx}{\textwidth}{>{\raggedleft\arraybackslash}p{3.5cm}
                             >{\raggedleft\arraybackslash}X}
\toprule
\headerrow موضوع & محتوا \\
\midrule
مصونیت & مصونیت کارکردی (نه مطلق) برای کارکنان، لغو برای جرایم سنگین \\
\altrow تردد & آزادی حرکت در سراسر کشور بدون مجوز قبلی \\
ارتباطات & آزادی ارتباطات رمزنگاری‌شده \\
\altrow مالیات و گمرک & معافیت برای تجهیزات و واردات مأموریت \\
تأسیسات & دسترسی به ساختمان‌ها، امنیت تأسیسات \\
\altrow حل اختلاف & کمیته مشترک + داوری بین‌المللی \\
\bottomrule
\end{tabularx}
\end{table}

\subsection{چارچوب داخلی موقت}
\label{subsec:domestic-legal}

در دوره گذار، قبل از تصویب قانون اساسی جدید، قوانین موقت ضروری‌اند:

\begin{table}[htbp]
\centering
\caption{قوانین موقت ضروری}
\label{tab:interim-laws}
\begin{tabularx}{\textwidth}{>{\raggedleft\arraybackslash}p{3.5cm}
                             >{\raggedleft\arraybackslash}X
                             >{\raggedleft\arraybackslash}p{2.5cm}}
\toprule
\headerrow قانون & محتوای کلیدی & مرجع تصویب \\
\midrule
قانون انتخابات موقت & نظام انتخاباتی، حق رأی، ثبت‌نام، شکایات & شورای انتقالی \\
\altrow قانون احزاب موقت & آزادی تشکیل، ثبت، تأمین مالی & شورای انتقالی \\
قانون رسانه موقت & آزادی مطبوعات، صدور مجوز، ضدانحصار & شورای انتقالی \\
\altrow قانون تجمعات & آزادی تجمع، محدودیت‌های امنیتی معقول & شورای انتقالی \\
منشور حقوق بنیادین & حقوق اساسی غیرقابل تعلیق & اعلامیه \lr{SRSG} \\
\altrow قانون عدالت انتقالی & کمیسیون حقیقت، دادگاه ویژه، جبران & شورای انتقالی \\
\bottomrule
\end{tabularx}
\end{table}

\begin{casestudy}{عراق: خلأ قانونی}
یکی از اشتباهات بزرگ \lr{CPA} در عراق، انحلال قوانین موجود بدون جایگزین بود. این خلأ به هرج‌ومرج، غارت، و تضعیف حاکمیت قانون انجامید. در ایران باید از «قانون‌زدایی» افراطی اجتناب شود. قوانین موجود که با حقوق بشر سازگارند، موقتاً حفظ شوند.
\end{casestudy}

\subsection{الحاق به معاهدات بین‌المللی}
\label{subsec:treaties}

\begin{recommendation}
الحاق فوری (ماه‌های اول):
\begin{itemize}[nosep]
    \item پروتکل اختیاری میثاق حقوق مدنی-سیاسی
    \item کنوانسیون علیه شکنجه (پروتکل اختیاری)
    \item اساسنامه رم دادگاه کیفری بین‌المللی
    \item کنوانسیون رفع تبعیض علیه زنان (\lr{CEDAW}) بدون شرط
\end{itemize}

الحاق میان‌مدت (سال اول):
\begin{itemize}[nosep]
    \item کنوانسیون ضد فساد سازمان ملل
    \item کنوانسیون‌های \lr{ILO} (کار اجباری، آزادی انجمن)
    \item کنوانسیون حقوق کودک (پروتکل‌های اختیاری)
\end{itemize}
\end{recommendation}

\sectiondivider

% ═══════════════════════════════════════════════════════════════════════════════
\section{ماتریس یکپارچه نیازمندی‌ها}
\label{sec:requirements-matrix}
% ═══════════════════════════════════════════════════════════════════════════════

جدول زیر نیازمندی‌های کلیدی را با فاز، اولویت، و مسئول خلاصه می‌کند:

\begin{landscape}
\begin{table}[htbp]
\centering
\bigtablefontsize
\caption{ماتریس یکپارچه نیازمندی‌های کلیدی}
\label{tab:requirements-matrix}
\begin{tabularx}{\linewidth}{>{\raggedleft\arraybackslash}p{1cm}
                             >{\raggedleft\arraybackslash}p{3cm}
                             >{\centering\arraybackslash}p{1.5cm}
                             >{\centering\arraybackslash}p{1.5cm}
                             >{\raggedleft\arraybackslash}X
                             >{\raggedleft\arraybackslash}p{2.5cm}
                             >{\raggedleft\arraybackslash}p{2.5cm}}
\toprule
\headerrow \# & نیازمندی & حوزه & فاز & شرح مختصر & مسئول اصلی & پیش‌نیاز \\
\midrule
۱ & \lr{SRSG} و دفتر مرکزی & نهادی & ۱ & انتخاب و استقرار نماینده ویژه & دبیرکل \lr{UN} & قطعنامه \lr{SC} \\
\altrow ۲ & قطعنامه شورای امنیت & حقوقی & ۰ & چارچوب حقوقی مأموریت & \lr{P5} & اجماع \\
۳ & \lr{SOMA} & حقوقی & ۱ & توافق با دولت انتقالی & \lr{UN} + دولت & دولت انتقالی \\
\altrow ۴ & کمیسیون انتخابات & نهادی & ۱ & تأسیس و تجهیز & دولت + \lr{UNDP} & قانون موقت \\
۵ & سامانه ثبت رأی‌دهندگان & فنی & ۱ & پایگاه داده ۶۰M+ & کمیسیون + \lr{IFES} & زیرساخت \lr{IT} \\
\altrow ۶ & ۶-۱۲K کارمند بین‌المللی & انسانی & ۱-۲ & استخدام و استقرار & \lr{UN HR} & بودجه \\
۷ & ۲۰-۵۰K کارمند ایرانی & انسانی & ۱-۲ & استخدام و آموزش & \lr{UNMOIT} + محلی & غربالگری \\
\altrow ۸ & کمیسیون حقیقت & نهادی & ۲ & تأسیس و شروع کار & مجلس موقت & قانون عدالت انتقالی \\
۹ & دادگاه ویژه & نهادی & ۲ & تأسیس و شروع محاکمات & دولت + \lr{UN} & تحقیقات اولیه \\
\altrow ۱۰ & صندوق امانی & نهادی & ۱ & مدیریت کمک‌ها و دارایی‌ها & \lr{UN} + دولت & توافق حامیان \\
۱۱ & زیرساخت ارتباطات & فنی & ۱ & اینترنت، تلفن، رادیو & وزارت \lr{ICT} + \lr{UN} & رفع فیلترینگ \\
\altrow ۱۲ & امنیت سایبری & فنی & ۱ & تیم و پروتکل‌ها & \lr{UNMOIT IT} & متخصصان \\
۱۳ & قوانین موقت & حقوقی & ۱ & انتخابات، احزاب، رسانه & شورای انتقالی & مشروعیت \\
\altrow ۱۴ & ۳۱ دفتر استانی & نهادی & ۱-۲ & حضور سراسری & \lr{UNMOIT} & کارمند + تأسیسات \\
\bottomrule
\end{tabularx}
\end{table}
\end{landscape}

\sectiondivider

% ═══════════════════════════════════════════════════════════════════════════════
% جمع‌بندی فصل
% ═══════════════════════════════════════════════════════════════════════════════

\begin{chaptersummary}
یافته‌های کلیدی این فصل:

\begin{enumerate}
    \item \textbf{نیروی انسانی عظیم اما تدریجی}: اوج نیاز در فاز ۲ (۲۷,۰۰۰-۶۴,۰۰۰ نفر) است، اما استقرار باید تدریجی و با آموزش کافی باشد. اتکای بیش‌ازحد به نیروی بین‌المللی نه ممکن است و نه مطلوب.
    
    \item \textbf{ظرفیت ایرانی ستون فقرات است}: ۸۰-۹۰٪ نیروی انسانی باید ایرانی باشد. دیاسپورا منبع ارزشمند اما باید با دقت مدیریت شود (سقف ۲۰-۳۰٪).
    
    \item \textbf{هشت نهاد کلیدی باید ایجاد شوند}: دفتر \lr{SRSG}، کمیسیون انتخابات، کمیسیون حقیقت، دادگاه ویژه، صندوق امانی، کمیسیون رسانه، کمیسیون ضدفساد، و نهاد حقوق اقوام.
    
    \item \textbf{زیرساخت فنی حیاتی‌تر از آنچه به نظر می‌رسد}: رفع فیلترینگ اینترنت در روز اول، سامانه ثبت رأی‌دهندگان، و امنیت سایبری از اولویت‌های بالا هستند.
    
    \item \textbf{انتخاب فناوری انتخابات}: رأی کاغذی با شمارش علنی در مرحله اول، امن‌ترین و قابل‌اعتمادترین گزینه است.
    
    \item \textbf{چارچوب حقوقی سه‌لایه}: قطعنامه شورای امنیت (بین‌المللی) + \lr{SOMA} (دوجانبه) + قوانین موقت (داخلی) باید همزمان آماده شوند.
    
    \item \textbf{از خلأ قانونی اجتناب شود}: قوانین موجود که با حقوق بشر سازگارند، تا تصویب قانون جدید حفظ شوند (درس عراق).
    
    \item \textbf{الحاق به معاهدات بین‌المللی}: اساسنامه رم، \lr{CEDAW}، پروتکل شکنجه و سایر معاهدات حقوق بشری باید در اولویت باشند.
\end{enumerate}

\vspace{0.5cm}
\textit{در فصل بعد (\ref{ch:timeline})، زمان‌بندی تفصیلی، ساختار تیم‌ها و مکانیزم هماهنگی بررسی خواهد شد.}
\end{chaptersummary}

\chapterend
% ═══════════════════════════════════════════════════════════════════════════════
% فصل ۹: زمان‌بندی، تیم‌سازی و ساختارسازی
% فایل: chapters/ch09-timeline.tex
% رنگ فصل: زرد (MainYellow)
% ═══════════════════════════════════════════════════════════════════════════════

\chapteropening{۹}{زمان‌بندی، تیم‌سازی و ساختارسازی}{MainYellow}{%
برنامه‌ریزی بدون اجرا، رؤیاپردازی است. اجرا بدون برنامه‌ریزی، کابوس.%
}{پیتر دراکر}

\chapter{زمان‌بندی، تیم‌سازی و ساختارسازی}
\label{ch:timeline}

\minitoc

% ─────────────────────────────────────────────────────────────────────────────
% خلاصه اجرایی
% ─────────────────────────────────────────────────────────────────────────────

\begin{executivesummary}
نظارت بین‌المللی بر گذار ایران یک عملیات چندفازی است که از مرحله پیش‌گذار (۶-۱۲ ماه قبل از سقوط رژیم) آغاز و تا ۱۰ سال پس از آن ادامه می‌یابد. این فصل زمان‌بندی تفصیلی پنج فاز، ساختار تیم‌ها در هر فاز، مکانیزم هماهنگی، و ابزارهای مدیریت پروژه را ارائه می‌دهد. تأکید اصلی بر \emph{آمادگی پیش‌گذار} (که اغلب نادیده گرفته می‌شود)، \emph{سرعت عمل در ۷۲ ساعت اول}، و \emph{انتقال تدریجی مسئولیت} به نهادهای ایرانی است. ماتریس \lr{RACI} و نمودار گانت عملیاتی، ابزارهای اجرایی این فصل‌اند.
\end{executivesummary}

\section{درآمد: زمان به‌عنوان متغیر حیاتی}
\label{sec:timeline-intro}

در فرایند گذار دموکراتیک، زمان‌بندی صرفاً یک ابزار مدیریتی نیست؛ \emph{متغیر استراتژیک} است. اقدام زودهنگام یا دیرهنگام، هر دو می‌توانند فاجعه‌بار باشند.

\begin{keypoint}
\textbf{پارادوکس زمان در گذار}: 
\begin{itemize}[nosep]
    \item خیلی سریع = بی‌ثباتی، اشتباهات جبران‌ناپذیر
    \item خیلی آهسته = از دست رفتن فرصت، خستگی مردم، بازگشت اقتدارگرایی
    \item نقطه بهینه = «به‌اندازه کافی سریع برای حفظ momentum، به‌اندازه کافی محتاط برای جلوگیری از خطا»
\end{itemize}
\end{keypoint}

\begin{casestudy}{مصر در مقابل تونس: اهمیت زمان‌بندی}
مصر پس از سقوط مبارک، به فشار خیابان و نظامیان، انتخابات سریع برگزار کرد (کمتر از ۱ سال) — قبل از آنکه نهادها و احزاب آماده شوند. نتیجه: پیروزی اخوان‌المسلمین، واکنش نظامیان، کودتا. تونس ۲.۵ سال صبر کرد، قانون اساسی توافقی نوشت، و سپس انتخابات برگزار کرد. نتیجه: گذار نسبتاً موفق (هرچند شکننده).
\end{casestudy}

\sectiondivider

% ═══════════════════════════════════════════════════════════════════════════════
\section{نمای کلی فازبندی}
\label{sec:phasing-overview}
% ═══════════════════════════════════════════════════════════════════════════════

\subsection{پنج فاز عملیات}
\label{subsec:five-phases}

\begin{figure}[htbp]
\centering
\begin{tikzpicture}[
    node distance=0.3cm,
    every node/.style={font=\small},
    phase/.style={rectangle, rounded corners=5pt, minimum width=3cm, minimum height=1.8cm, align=center, text=white, font=\small\bfseries},
    arrow/.style={-{Stealth[length=3mm]}, very thick, DarkGray},
    label/.style={font=\footnotesize, align=center, text=DarkGray}
]
    % Phases
    \node[phase, fill=MediumGray] (p0) {فاز ۰\\پیش‌گذار\\{\footnotesize ۶-۱۲ ماه قبل}};
    \node[phase, fill=MainRed, right=0.5cm of p0] (p1) {فاز ۱\\تثبیت\\{\footnotesize ماه ۱-۶}};
    \node[phase, fill=MainOrange, right=0.5cm of p1] (p2) {فاز ۲\\نهادسازی\\{\footnotesize ماه ۶-۲۴}};
    \node[phase, fill=MainGreen, right=0.5cm of p2] (p3) {فاز ۳\\تحکیم\\{\footnotesize ماه ۲۴-۶۰}};
    \node[phase, fill=MainBlue, right=0.5cm of p3] (p4) {فاز ۴\\پی‌گیری\\{\footnotesize ماه ۶۰-۱۲۰}};
    
    % Arrows
    \draw[arrow] (p0) -- (p1);
    \draw[arrow] (p1) -- (p2);
    \draw[arrow] (p2) -- (p3);
    \draw[arrow] (p3) -- (p4);
    
    % Labels below
    \node[label, below=0.4cm of p0] {آمادگی\\برنامه‌ریزی};
    \node[label, below=0.4cm of p1] {امنیت\\نظم اولیه};
    \node[label, below=0.4cm of p2] {قانون اساسی\\انتخابات};
    \node[label, below=0.4cm of p3] {دموکراسی\\نهادها};
    \node[label, below=0.4cm of p4] {نظارت\\مشاوره};
    
    % Personnel indicator
    \node[label, above=0.4cm of p0] {۵۰-۲۰۰};
    \node[label, above=0.4cm of p1] {۳-۵K};
    \node[label, above=0.4cm of p2] {۶-۱۲K};
    \node[label, above=0.4cm of p3] {۱.۵-۳K};
    \node[label, above=0.4cm of p4] {۲۰۰-۵۰۰};
    \node[font=\footnotesize, anchor=south] at ($(p2)+(0,1.5)$) {\textbf{کارکنان بین‌المللی}};
    
\end{tikzpicture}
\caption{نمای کلی پنج فاز عملیات نظارت بین‌المللی}
\label{fig:five-phases}
\end{figure}

\begin{table}[htbp]
\centering
\caption{خلاصه ویژگی‌های هر فاز}
\label{tab:phase-summary}
\begin{tabularx}{\textwidth}{>{\raggedleft\arraybackslash}p{1.5cm}
                             >{\raggedleft\arraybackslash}p{2.5cm}
                             >{\raggedleft\arraybackslash}X
                             >{\raggedleft\arraybackslash}p{2cm}
                             >{\raggedleft\arraybackslash}p{2.5cm}}
\toprule
\headerrow فاز & بازه زمانی & هدف اصلی & مدل غالب & شاخص پایان \\
\midrule
۰ & ۶-۱۲ ماه قبل & آمادگی، برنامه‌ریزی، شبکه‌سازی & --- & شروع گذار \\
\altrow ۱ & ماه ۱-۶ & تثبیت، امنیت، نظم اولیه & مدل ۴ & امنیت نسبی \\
۲ & ماه ۶-۲۴ & قانون اساسی، انتخابات، نهادها & مدل ۳+۴ & انتخابات آزاد \\
\altrow ۳ & ماه ۲۴-۶۰ & تحکیم دموکراسی، انتقال کامل & مدل ۲+۳ & دولت منتخب پایدار \\
۴ & ماه ۶۰-۱۲۰ & پی‌گیری، مشاوره، ارزیابی & مدل ۲ & خروج کامل \\
\bottomrule
\end{tabularx}
\end{table}

\sectiondivider

% ═══════════════════════════════════════════════════════════════════════════════
\section{فاز ۰: پیش‌گذار (۶-۱۲ ماه قبل)}
\label{sec:phase-zero}
% ═══════════════════════════════════════════════════════════════════════════════

این فاز، مهم‌ترین و اغلب نادیده‌گرفته‌ترین مرحله است. آمادگی قبل از بحران، تفاوت میان موفقیت و فاجعه است.

\begin{warningbox}
در اکثر گذارهای تاریخی، جامعه بین‌المللی \emph{غافلگیر} شده است. فروپاشی شوروی (۱۹۹۱)، سقوط بن‌علی (۲۰۱۱)، سقوط بشار اسد — هیچ‌کدام با آمادگی کافی بین‌المللی همراه نبود. این سند دقیقاً برای پر کردن این خلأ نوشته شده است.
\end{warningbox}

\subsection{اقدامات فاز ۰}
\label{subsec:phase0-actions}

\begin{table}[htbp]
\centering
\caption{اقدامات تفصیلی فاز ۰ (پیش‌گذار)}
\label{tab:phase0-actions}
\begin{tabularx}{\textwidth}{>{\raggedleft\arraybackslash}p{0.5cm}
                             >{\raggedleft\arraybackslash}p{3cm}
                             >{\raggedleft\arraybackslash}X
                             >{\raggedleft\arraybackslash}p{2.5cm}
                             >{\raggedleft\arraybackslash}p{2cm}}
\toprule
\headerrow \# & اقدام & شرح & مسئول & زمان \\
\midrule
۱ & تشکیل گروه برنامه‌ریزی & تیم ۵۰-۱۰۰ نفره در نیویورک/ژنو & \lr{DPPA/UN} & ماه ۱-۳ \\
\altrow ۲ & تحلیل سناریو & بروزرسانی سناریوهای فصل ۴ بر اساس اطلاعات جدید & گروه برنامه‌ریزی & ماه ۱-۶ \\
۳ & شناسایی \lr{SRSG} بالقوه & فهرست ۵-۱۰ نامزد، مذاکرات محرمانه & دبیرکل & ماه ۳-۶ \\
\altrow ۴ & پیش‌نویس قطعنامه & نسخه‌های مختلف برای سناریوهای مختلف & \lr{P3+} & ماه ۳-۹ \\
۵ & شبکه‌سازی با اپوزیسیون & ارتباط غیررسمی با همه گروه‌ها & \lr{DPPA} + دولت‌ها & مستمر \\
\altrow ۶ & نقشه‌برداری ظرفیت & شناسایی ایرانیان متخصص (داخل و دیاسپورا) & \lr{UNDP} & ماه ۱-۶ \\
۷ & پیش‌موقعیت‌یابی & ذخیره تجهیزات در کشورهای همسایه & \lr{DPKO/DOS} & ماه ۶-۱۲ \\
\altrow ۸ & آموزش پیشینی & آماده‌سازی ناظران از فهرست آماده‌باش & \lr{UN/EU} & ماه ۶-۱۲ \\
۹ & طرح ارتباطات عمومی & پیام‌ها، مخاطبان، رسانه‌ها & تیم ارتباطات & ماه ۹-۱۲ \\
\altrow ۱۰ & طرح بودجه اولیه & برآورد فاز ۱ و ۲، شناسایی حامیان & \lr{OCHA/WB} & ماه ۶-۱۲ \\
\bottomrule
\end{tabularx}
\end{table}

\begin{lessonlearned}{تیمور شرقی: ارزش آمادگی}
در تیمور شرقی، سازمان ملل ماه‌ها قبل از رفراندوم ۱۹۹۹ برنامه‌ریزی کرده بود. وقتی خشونت پس از رفراندوم آغاز شد، \lr{INTERFET} (نیروی چندملیتی به رهبری استرالیا) ظرف ۱۰ روز مستقر شد. بدون این آمادگی، فاجعه بسیار بزرگ‌تر بود.
\end{lessonlearned}

\sectiondivider

% ═══════════════════════════════════════════════════════════════════════════════
\section{فاز ۱: تثبیت (ماه ۱-۶)}
\label{sec:phase-one}
% ═══════════════════════════════════════════════════════════════════════════════

\subsection{۷۲ ساعت اول: لحظه سرنوشت‌ساز}
\label{subsec:first-72}

\begin{warningbox}
۷۲ ساعت اول پس از سقوط رژیم، پنجره فرصتی است که آینده گذار را رقم می‌زند. تصمیمات این ساعات، بازگشت‌ناپذیرند.
\end{warningbox}

\begin{table}[htbp]
\centering
\caption{اقدامات حیاتی ۷۲ ساعت اول}
\label{tab:first-72}
\begin{tabularx}{\textwidth}{>{\raggedleft\arraybackslash}p{1.5cm}
                             >{\raggedleft\arraybackslash}p{3cm}
                             >{\raggedleft\arraybackslash}X
                             >{\raggedleft\arraybackslash}p{2.5cm}}
\toprule
\headerrow ساعت & اقدام & شرح & مسئول \\
\midrule
۰-۶ & ارزیابی وضعیت & تماس با همه طرف‌ها، ارزیابی امنیتی & \lr{DPPA} \\
\altrow ۶-۱۲ & بیانیه دبیرکل & حمایت از گذار مسالمت‌آمیز، دعوت به خویشتنداری & دبیرکل \\
۱۲-۲۴ & نشست شورای امنیت & بحث اولیه، بیانیه ریاست & شورای امنیت \\
\altrow ۲۴-۴۸ & اعزام تیم ارزیابی & تیم ۲۰-۳۰ نفره به تهران & \lr{DPPA/OCHA} \\
۴۸-۷۲ & تماس با دولت موقت & شناسایی طرف‌های مذاکره، ارزیابی نیازها & تیم ارزیابی \\
\altrow ۷۲+ & پیش‌نویس قطعنامه & ارائه متن به شورای امنیت & \lr{P3+} \\
\bottomrule
\end{tabularx}
\end{table}

\subsection{اقدامات ماه اول}
\label{subsec:month-one}

\begin{enumerate}[nosep]
    \item \textbf{انتخاب و اعزام \lr{SRSG}} (از فهرست آماده فاز ۰)
    \item \textbf{تصویب قطعنامه شورای امنیت} (ایجاد \lr{UNMOIT})
    \item \textbf{مذاکره \lr{SOMA}} با دولت انتقالی/موقت
    \item \textbf{استقرار تیم پیشتاز} (۵۰۰-۱,۰۰۰ نفر)
    \item \textbf{ایجاد دفتر مرکزی} در تهران
    \item \textbf{شروع پایش حقوق بشر} (فوری)
    \item \textbf{برقراری ارتباط با نیروهای امنیتی} (ارتش، پلیس)
    \item \textbf{ارزیابی وضعیت اقتصادی} و نیازهای فوری
\end{enumerate}

\subsection{اقدامات ماه ۲-۶}
\label{subsec:months-2-6}

\begin{table}[htbp]
\centering
\caption{اقدامات کلیدی فاز ۱ (ماه ۲-۶)}
\label{tab:phase1-actions}
\begin{tabularx}{\textwidth}{>{\raggedleft\arraybackslash}p{2.5cm}
                             >{\raggedleft\arraybackslash}X
                             >{\raggedleft\arraybackslash}p{2.5cm}
                             >{\raggedleft\arraybackslash}p{2cm}}
\toprule
\headerrow حوزه & اقدام‌های کلیدی & مسئول & زمان \\
\midrule
امنیت & نظارت بر نیروهای مسلح، شروع \lr{DDR}، کنترل مرزها & معاون امنیتی & ماه ۱-۶ \\
\altrow سیاسی & تشکیل شورای مشورتی، گفتگوی ملی، زمان‌بندی انتخابات & معاون سیاسی & ماه ۲-۶ \\
حقوقی & تصویب قوانین موقت، منشور حقوق بنیادین & \lr{SRSG}+شورا & ماه ۲-۴ \\
\altrow حقوق بشر & مستندسازی نقض‌ها، حفظ اسناد، محافظت از شهود & \lr{OHCHR} & ماه ۱-۶ \\
اقتصاد & رفع تحریم‌ها، تثبیت ارز، کمک بشردوستانه & \lr{IMF/WB} & ماه ۱-۶ \\
\altrow نهادسازی & ایجاد کمیسیون انتخابات، شروع ثبت رأی‌دهندگان & \lr{UNDP/IFES} & ماه ۳-۶ \\
رسانه & رفع فیلترینگ، صدور مجوز رسانه، نظارت بر تعادل & کمیسیون رسانه & ماه ۱-۶ \\
\altrow استقرار & ایجاد ۳۱ دفتر استانی، استخدام ۳-۵K بین‌المللی & \lr{UNMOIT} اداری & ماه ۲-۶ \\
\bottomrule
\end{tabularx}
\end{table}

\sectiondivider

% ═══════════════════════════════════════════════════════════════════════════════
\section{فاز ۲: نهادسازی (ماه ۶-۲۴)}
\label{sec:phase-two}
% ═══════════════════════════════════════════════════════════════════════════════

فاز ۲ قلب عملیات است. در این فاز، نهادهای دموکراتیک ساخته می‌شوند و اولین انتخابات آزاد برگزار می‌شود.

\subsection{قانون اساسی}
\label{subsec:constitution}

\begin{table}[htbp]
\centering
\caption{فرایند تدوین قانون اساسی}
\label{tab:constitution-process}
\begin{tabularx}{\textwidth}{>{\raggedleft\arraybackslash}p{2.5cm}
                             >{\raggedleft\arraybackslash}X
                             >{\raggedleft\arraybackslash}p{2.5cm}
                             >{\raggedleft\arraybackslash}p{2cm}}
\toprule
\headerrow مرحله & شرح & مشارکت‌کنندگان & بازه زمانی \\
\midrule
مشاوره عمومی & جمع‌آوری نظرات مردم (آنلاین و حضوری) & مردم، \lr{NGO}ها & ماه ۶-۱۰ \\
\altrow انتخاب مجلس مؤسسان & انتخابات برای ۲۰۰-۳۰۰ نماینده & کمیسیون انتخابات & ماه ۱۰-۱۲ \\
تدوین پیش‌نویس & نگارش با کمک مشاوران بین‌المللی & مجلس مؤسسان & ماه ۱۲-۱۸ \\
\altrow بازخورد عمومی & انتشار پیش‌نویس، جمع‌آوری نظرات & مردم، احزاب & ماه ۱۸-۲۰ \\
بازنگری و اصلاح & اصلاح بر اساس بازخورد & مجلس مؤسسان & ماه ۲۰-۲۲ \\
\altrow رفراندوم & تصویب عمومی قانون اساسی & مردم & ماه ۲۲-۲۴ \\
\bottomrule
\end{tabularx}
\end{table}

\begin{keypoint}
\textbf{دو رویکرد به قانون اساسی:}
\begin{itemize}[nosep]
    \item \textbf{اول انتخابات، بعد قانون اساسی} (مدل عراق): سریع‌تر اما ریسک مصادره
    \item \textbf{اول قانون اساسی، بعد انتخابات} (مدل تونس): کندتر اما پایدارتر
\end{itemize}
\textbf{توصیه}: مدل تونس، با تعدیل — یعنی مجلس مؤسسان منتخب ابتدا قانون اساسی بنویسد، سپس انتخابات پارلمانی و ریاست‌جمهوری برگزار شود.
\end{keypoint}

\subsection{انتخابات}
\label{subsec:elections}

\begin{table}[htbp]
\centering
\caption{تقویم انتخاباتی پیشنهادی}
\label{tab:election-calendar}
\begin{tabularx}{\textwidth}{>{\raggedleft\arraybackslash}p{3cm}
                             >{\centering\arraybackslash}p{2.5cm}
                             >{\raggedleft\arraybackslash}X
                             >{\raggedleft\arraybackslash}p{2.5cm}}
\toprule
\headerrow انتخابات & ماه (تقریبی) & هدف & نظام انتخاباتی \\
\midrule
مجلس مؤسسان & ماه ۱۰-۱۲ & تدوین قانون اساسی & تناسبی \\
\altrow رفراندوم قانون اساسی & ماه ۲۲-۲۴ & تصویب مردمی & اکثریت ساده \\
پارلمان & ماه ۲۶-۳۰ & قوه مقننه & تناسبی مختلط \\
\altrow ریاست‌جمهوری & ماه ۲۸-۳۲ & قوه مجریه & دومرحله‌ای \\
شوراهای محلی & ماه ۳۰-۳۶ & حکمرانی محلی & تناسبی \\
\bottomrule
\end{tabularx}
\end{table}

\begin{lessonlearned}{آفریقای جنوبی: صبر استراتژیک}
آفریقای جنوبی ۴ سال (۱۹۹۰-۱۹۹۴) صبر کرد تا اولین انتخابات دموکراتیک برگزار شود. این زمان صرف مذاکره، ساخت اعتماد، و آماده‌سازی نهادها شد. اولین انتخابات ۱۹۹۴ یکی از موفق‌ترین انتخابات گذار در تاریخ بود: مشارکت ۸۶٪، پذیرش نتایج توسط همه طرف‌ها.
\end{lessonlearned}

\sectiondivider

% ═══════════════════════════════════════════════════════════════════════════════
\section{فاز ۳: تحکیم (ماه ۲۴-۶۰)}
\label{sec:phase-three}
% ═══════════════════════════════════════════════════════════════════════════════

پس از انتخابات و تشکیل دولت منتخب، فاز تحکیم آغاز می‌شود. نظارت بین‌المللی تدریجاً کاهش و مالکیت ملی افزایش می‌یابد.

\subsection{اقدامات کلیدی فاز ۳}
\label{subsec:phase3-actions}

\begin{table}[htbp]
\centering
\caption{اقدامات فاز ۳ (تحکیم)}
\label{tab:phase3-actions}
\begin{tabularx}{\textwidth}{>{\raggedleft\arraybackslash}p{3cm}
                             >{\raggedleft\arraybackslash}X
                             >{\raggedleft\arraybackslash}p{2.5cm}}
\toprule
\headerrow حوزه & اقدامات & تغییر نسبت به فاز ۲ \\
\midrule
نظارت انتخاباتی & نظارت بر انتخابات دوم، آموزش ناظران داخلی & کاهش ۵۰٪ \\
\altrow حقوق بشر & ادامه پایش، کمیسیون حقیقت فعال & ثابت \\
اصلاح امنیتی & ادامه \lr{DDR}، آموزش پلیس، نظارت بر ارتش & کاهش ۳۰٪ \\
\altrow قضایی & دادگاه ویژه فعال، اصلاح قوه قضاییه & ثابت \\
اقتصادی & خصوصی‌سازی نظارت‌شده، جذب سرمایه‌گذاری & مشاوره‌ای \\
\altrow ظرفیت‌سازی & آموزش کارکنان دولت، انتقال مهارت & افزایش \\
رسانه & حمایت از رسانه مستقل، آموزش روزنامه‌نگاران & کاهش ۴۰٪ \\
\altrow کاهش حضور & بسته شدن دفاتر استانی، کاهش کارکنان بین‌المللی & ۶-۱۲K → ۱.۵-۳K \\
\bottomrule
\end{tabularx}
\end{table}

\subsection{معیارهای تحکیم}
\label{subsec:consolidation-criteria}

\begin{recommendation}
دموکراسی ایران «تحکیم‌یافته» تلقی می‌شود اگر:
\begin{enumerate}[nosep]
    \item حداقل ۲ انتخابات آزاد و عادلانه برگزار شده باشد
    \item حداقل ۱ انتقال مسالمت‌آمیز قدرت رخ داده باشد
    \item نظامیان مداخله‌ای در سیاست نداشته باشند
    \item آزادی مطبوعات در سطح «نسبتاً آزاد» باشد
    \item استقلال قوه قضاییه تضمین شده باشد
    \item نرخ اعتماد عمومی به دموکراسی بالای ۵۰٪ باشد
    \item حقوق اقلیت‌ها رعایت شود
\end{enumerate}
\end{recommendation}

\sectiondivider

% ═══════════════════════════════════════════════════════════════════════════════
\section{فاز ۴: پی‌گیری و خروج (ماه ۶۰-۱۲۰)}
\label{sec:phase-four}
% ═══════════════════════════════════════════════════════════════════════════════

\subsection{استراتژی خروج}
\label{subsec:exit-strategy}

\begin{keypoint}
\textbf{سه سناریوی خروج:}
\begin{enumerate}[nosep]
    \item \textbf{خروج موفق}: معیارهای تحکیم محقق شده → خروج تدریجی با ۱-۲ سال هم‌پوشانی
    \item \textbf{خروج تعدیل‌شده}: برخی معیارها محقق → کاهش اما نه خروج کامل → تمدید ۲-۳ ساله
    \item \textbf{بازنگری اساسی}: عقب‌گرد جدی → بازنگری کل استراتژی → ممکن است نیاز به افزایش
\end{enumerate}
\end{keypoint}

\begin{table}[htbp]
\centering
\caption{مراحل خروج تدریجی}
\label{tab:exit-steps}
\begin{tabularx}{\textwidth}{>{\raggedleft\arraybackslash}p{2.5cm}
                             >{\raggedleft\arraybackslash}X
                             >{\raggedleft\arraybackslash}p{2cm}
                             >{\raggedleft\arraybackslash}p{2.5cm}}
\toprule
\headerrow مرحله & اقدام & ماه & باقیمانده \\
\midrule
تبدیل مأموریت & تغییر \lr{UNMOIT} به دفتر سیاسی \lr{UN} & ۶۰ & ۵۰۰ نفر \\
\altrow بسته شدن دفاتر & ادغام ۳۱ استان در ۵-۷ منطقه & ۶۰-۷۲ & ۳۰۰ نفر \\
کاهش تخصصی & پایان مأموریت انتخاباتی و امنیتی & ۷۲-۸۴ & ۲۰۰ نفر \\
\altrow مشاوره محض & فقط مشاوران ارشد و ارزیابان & ۸۴-۱۰۸ & ۵۰-۱۰۰ نفر \\
خروج نهایی & بسته شدن مأموریت، گزارش نهایی & ۱۰۸-۱۲۰ & ۰ \\
\bottomrule
\end{tabularx}
\end{table}

\begin{warningbox}
خروج زودهنگام به‌اندازه ماندن بیش‌ازحد خطرناک است. خروج آمریکا از عراق (۲۰۱۱) و افغانستان (۲۰۲۱) نشان داد که خروج بدون تحکیم واقعی، می‌تواند همه دستاوردها را نابود کند. خروج باید بر اساس شاخص‌ها باشد، نه تقویم سیاسی.
\end{warningbox}

\sectiondivider

% ═══════════════════════════════════════════════════════════════════════════════
\section{ساختار هماهنگی}
\label{sec:coordination-structure}
% ═══════════════════════════════════════════════════════════════════════════════

\subsection{سلسله‌مراتب هماهنگی}
\label{subsec:coordination-hierarchy}

\begin{figure}[htbp]
\centering
\begin{tikzpicture}[
    node distance=1cm,
    every node/.style={font=\small, align=center},
    global/.style={rectangle, rounded corners, draw=MainPurple, fill=LightPurple, minimum width=4.5cm, minimum height=0.9cm},
    strategic/.style={rectangle, rounded corners, draw=MainBlue, fill=LightBlue, minimum width=4cm, minimum height=0.9cm},
    operational/.style={rectangle, rounded corners, draw=MainGreen, fill=LightGreen, minimum width=3.5cm, minimum height=0.9cm},
    tactical/.style={rectangle, rounded corners, draw=MainOrange, fill=LightOrange, minimum width=3cm, minimum height=0.9cm},
    arrow/.style={-{Stealth[length=2.5mm]}, thick}
]
    % Global level
    \node[global] (sc) {شورای امنیت \lr{UN}};
    \node[global, right=2.5cm of sc] (cg) {گروه تماس بین‌المللی\\{\footnotesize \lr{P5} + \lr{EU} + ژاپن + ترکیه + ...}};
    
    % Strategic level
    \node[strategic, below=1.2cm of sc] (srsg) {\lr{SRSG}\\ستاد مرکزی تهران};
    \node[strategic, right=2.5cm of srsg] (advisory) {شورای مشورتی ایرانی\\{\footnotesize ۲۰-۳۰ نماینده}};
    
    % Operational level
    \node[operational, below left=1.2cm and 1.5cm of srsg] (pol) {بخش سیاسی};
    \node[operational, below=1.2cm of srsg] (ops) {بخش عملیاتی};
    \node[operational, below right=1.2cm and 1.5cm of srsg] (hr) {بخش حقوق بشر};
    
    % Tactical level
    \node[tactical, below=1.2cm of pol] (prov1) {دفاتر استانی\\منطقه ۱-۳};
    \node[tactical, below=1.2cm of ops] (prov2) {دفاتر استانی\\منطقه ۴-۵};
    \node[tactical, below=1.2cm of hr] (prov3) {دفاتر استانی\\منطقه ۶-۷};
    
    % Arrows
    \draw[arrow] (sc) -- (srsg);
    \draw[arrow, dashed] (cg) -- (srsg);
    \draw[arrow, dashed] (advisory) -- (srsg);
    \draw[arrow] (srsg) -- (pol);
    \draw[arrow] (srsg) -- (ops);
    \draw[arrow] (srsg) -- (hr);
    \draw[arrow] (pol) -- (prov1);
    \draw[arrow] (ops) -- (prov2);
    \draw[arrow] (hr) -- (prov3);
    
    % Labels
    \node[font=\footnotesize\bfseries, MainPurple, anchor=east] at (-3.5,0) {سطح جهانی};
    \node[font=\footnotesize\bfseries, MainBlue, anchor=east] at (-3.5,-2.2) {سطح استراتژیک};
    \node[font=\footnotesize\bfseries, MainGreen, anchor=east] at (-3.5,-4.4) {سطح عملیاتی};
    \node[font=\footnotesize\bfseries, MainOrange, anchor=east] at (-3.5,-6.6) {سطح تاکتیکی};
    
\end{tikzpicture}
\caption{سلسله‌مراتب هماهنگی نظارت بین‌المللی}
\label{fig:coordination-hierarchy}
\end{figure}

\subsection{گروه تماس بین‌المللی}
\label{subsec:contact-group}

\begin{table}[htbp]
\centering
\caption{ترکیب و وظایف گروه تماس بین‌المللی}
\label{tab:contact-group}
\begin{tabularx}{\textwidth}{>{\raggedleft\arraybackslash}p{3cm}
                             >{\raggedleft\arraybackslash}X}
\toprule
\headerrow ویژگی & شرح \\
\midrule
اعضا & \lr{P5} + آلمان + ژاپن + ترکیه + عربستان + هند + \lr{EU} + اتحادیه آفریقا \\
\altrow سطح نمایندگی & وزیر خارجه یا معاون \\
دبیرخانه & دفتر \lr{SRSG} \\
\altrow فرکانس نشست & ماهانه (عادی)، فوری (بحران) \\
وظایف & هماهنگی سیاسی، تضمین منابع، جلوگیری از مداخله یکجانبه \\
\altrow تصمیم‌گیری & اجماعی (نه رأی‌گیری) \\
\bottomrule
\end{tabularx}
\end{table}

\subsection{شورای مشورتی ایرانی}
\label{subsec:advisory-council}

\begin{table}[htbp]
\centering
\caption{ترکیب شورای مشورتی ایرانی}
\label{tab:advisory-council}
\begin{tabularx}{\textwidth}{>{\raggedleft\arraybackslash}p{3.5cm}
                             >{\centering\arraybackslash}p{2cm}
                             >{\raggedleft\arraybackslash}X}
\toprule
\headerrow دسته & تعداد & معیار انتخاب \\
\midrule
احزاب/جنبش‌ها سیاسی & ۶-۸ & نمایندگان همه طیف‌ها \\
\altrow جامعه مدنی & ۴-۶ & \lr{NGO}های معتبر، حقوق بشری \\
زنان & ۳-۴ & فعالان حقوق زنان \\
\altrow اقوام & ۴-۶ & نمایندگان اقوام اصلی \\
اقلیت‌های مذهبی & ۲-۳ & سنّی، بهایی، مسیحی، یهودی، زرتشتی \\
\altrow جوانان & ۲-۳ & زیر ۳۵ سال \\
دیاسپورا & ۲-۳ & نمایندگان جوامع مهاجر \\
\altrow حقوقدانان & ۲-۳ & متخصصان حقوق اساسی \\
\midrule
\textbf{مجموع} & \textbf{۲۵-۳۶} & \\
\bottomrule
\end{tabularx}
\end{table}

\begin{casestudy}{عراق: شورای حکومتی ناکارآمد}
شورای حکومتی عراق (۲۰۰۳-۲۰۰۴) که توسط \lr{CPA} منصوب شد، فاقد مشروعیت مردمی بود و ترکیب آن بر اساس سهمیه‌بندی قومی-مذهبی بود. این الگو به «محاصصه» (تقسیم طایفه‌ای قدرت) دامن زد و تا امروز عراق از آن رنج می‌برد. شورای مشورتی ایران باید \emph{مشاوره‌ای} باشد، نه \emph{حکومتی}، و ترکیب آن بر اساس تخصص و اعتبار باشد نه صرفاً نمایندگی قومی.
\end{casestudy}

\sectiondivider

% ═══════════════════════════════════════════════════════════════════════════════
\section{تیم‌سازی و ساختار منابع انسانی}
\label{sec:team-building}
% ═══════════════════════════════════════════════════════════════════════════════

\subsection{ساختار رهبری}
\label{subsec:leadership-structure}

\begin{table}[htbp]
\centering
\caption{ساختار رهبری \lr{UNMOIT}}
\label{tab:leadership}
\begin{tabularx}{\textwidth}{>{\raggedleft\arraybackslash}p{3.5cm}
                             >{\raggedleft\arraybackslash}p{2.5cm}
                             >{\raggedleft\arraybackslash}X}
\toprule
\headerrow مقام & سطح \lr{UN} & مسئولیت اصلی \\
\midrule
\lr{SRSG} & \lr{USG} & رهبری کل مأموریت \\
\altrow معاون سیاسی & \lr{ASG} & مذاکرات، احزاب، گفتگوی ملی \\
معاون عملیاتی & \lr{ASG} & لجستیک، امور اداری، مالی \\
\altrow معاون حقوق بشر & \lr{D-2} & پایش، مستندسازی، عدالت انتقالی \\
معاون امنیتی & \lr{D-2} & اصلاح بخش امنیتی، \lr{DDR} \\
\altrow مشاور ارشد انتخابات & \lr{D-2} & کمیسیون انتخابات، فنی \\
سخنگو & \lr{D-1} & ارتباطات عمومی، رسانه \\
\bottomrule
\end{tabularx}
\end{table}

\subsection{دفاتر استانی}
\label{subsec:provincial-offices}

\begin{table}[htbp]
\centering
\caption{ساختار دفاتر استانی}
\label{tab:provincial-offices}
\begin{tabularx}{\textwidth}{>{\raggedleft\arraybackslash}p{3cm}
                             >{\raggedleft\arraybackslash}X}
\toprule
\headerrow عنصر & شرح \\
\midrule
تعداد & ۳۱ دفتر (یکی در هر استان) + ۵-۷ دفتر منطقه‌ای \\
\altrow رئیس & کارمند بین‌المللی \lr{P-5/D-1} با معاون ایرانی \\
کارکنان (فاز ۲) & ۵۰-۲۰۰ نفر بسته به اندازه استان \\
\altrow وظایف & پایش محلی، ارتباط با جوامع، گزارش‌دهی، نظارت انتخاباتی \\
اولویت استقرار & ابتدا استان‌های حساس (مرزی، قومی، پرجمعیت) \\
\bottomrule
\end{tabularx}
\end{table}

\begin{recommendation}
\textbf{اولویت‌بندی استقرار دفاتر استانی:}
\begin{enumerate}[nosep]
    \item \textbf{فوری (هفته ۱-۴)}: تهران، اصفهان، مشهد (خراسان رضوی)، تبریز (آذربایجان شرقی)، اهواز (خوزستان)
    \item \textbf{اولویت بالا (ماه ۱-۲)}: کردستان، سیستان‌وبلوچستان، کرمانشاه، فارس (شیراز)، هرمزگان
    \item \textbf{اولویت متوسط (ماه ۲-۴)}: سایر استان‌های مرزی و پرجمعیت
    \item \textbf{تکمیلی (ماه ۴-۶)}: استان‌های کوچک‌تر مرکزی
\end{enumerate}
\end{recommendation}

\sectiondivider

% ═══════════════════════════════════════════════════════════════════════════════
\section{ماتریس \lr{RACI}}
\label{sec:raci-matrix}
% ═══════════════════════════════════════════════════════════════════════════════

\bilingual{ماتریس مسئولیت}{RACI Matrix} ابزاری برای تعیین نقش هر بازیگر در هر فعالیت است:
\begin{itemize}[nosep]
    \item \textbf{\lr{R} = مسئول اجرا} (\lr{Responsible})
    \item \textbf{\lr{A} = پاسخگو} (\lr{Accountable})
    \item \textbf{\lr{C} = مشاور} (\lr{Consulted})
    \item \textbf{\lr{I} = اطلاع‌یافته} (\lr{Informed})
\end{itemize}

\begin{landscape}
\begin{table}[htbp]
\centering
\bigtablefontsize
\caption{ماتریس \lr{RACI} فعالیت‌های کلیدی}
\label{tab:raci-matrix}
\begin{tabularx}{\linewidth}{>{\raggedleft\arraybackslash}p{3cm}
                             >{\centering\arraybackslash}p{1.2cm}
                             >{\centering\arraybackslash}p{1.5cm}
                             >{\centering\arraybackslash}p{1.5cm}
                             >{\centering\arraybackslash}p{1.5cm}
                             >{\centering\arraybackslash}p{1.2cm}
                             >{\centering\arraybackslash}p{1.2cm}
                             >{\centering\arraybackslash}p{1.2cm}
                             >{\centering\arraybackslash}p{1.5cm}
                             >{\centering\arraybackslash}p{1.5cm}}
\toprule
\headerrow فعالیت & \rot{\lr{SC}} & \rot{\lr{SRSG}} & \rot{دولت انتقالی} & \rot{شورای مشورتی} & \rot{کمیسیون انتخابات} & \rot{\lr{OHCHR}} & \rot{\lr{IMF/WB}} & \rot{جامعه مدنی} & \rot{گروه تماس} \\
\midrule
قطعنامه & \lr{A,R} & \lr{C} & \lr{C} & \lr{I} & \lr{I} & \lr{C} & \lr{I} & \lr{I} & \lr{C} \\
\altrow انتخاب \lr{SRSG} & \lr{A} & --- & \lr{C} & \lr{I} & \lr{I} & \lr{I} & \lr{I} & \lr{I} & \lr{C} \\
امنیت عمومی & \lr{I} & \lr{A} & \lr{R} & \lr{C} & \lr{I} & \lr{C} & \lr{I} & \lr{I} & \lr{I} \\
\altrow اصلاح سپاه & \lr{A} & \lr{R} & \lr{R} & \lr{C} & \lr{I} & \lr{C} & \lr{I} & \lr{I} & \lr{C} \\
برگزاری انتخابات & \lr{I} & \lr{A} & \lr{C} & \lr{C} & \lr{R} & \lr{I} & \lr{I} & \lr{C} & \lr{I} \\
\altrow قانون اساسی & \lr{I} & \lr{C} & \lr{A,R} & \lr{C} & \lr{I} & \lr{C} & \lr{I} & \lr{C} & \lr{I} \\
حقوق بشر & \lr{I} & \lr{A} & \lr{C} & \lr{C} & \lr{I} & \lr{R} & \lr{I} & \lr{C} & \lr{I} \\
\altrow اقتصاد & \lr{I} & \lr{C} & \lr{A,R} & \lr{C} & \lr{I} & \lr{I} & \lr{R} & \lr{I} & \lr{C} \\
رفع تحریم & \lr{A,R} & \lr{C} & \lr{C} & \lr{I} & \lr{I} & \lr{I} & \lr{C} & \lr{I} & \lr{R} \\
\altrow عدالت انتقالی & \lr{I} & \lr{C} & \lr{A,R} & \lr{C} & \lr{I} & \lr{R} & \lr{I} & \lr{C} & \lr{I} \\
\bottomrule
\end{tabularx}
\end{table}
\end{landscape}

\sectiondivider

% ═══════════════════════════════════════════════════════════════════════════════
\section{نمودار گانت عملیاتی}
\label{sec:gantt-chart}
% ═══════════════════════════════════════════════════════════════════════════════

\begin{figure}[htbp]
\centering
\begin{ganttchart}[
    x unit=0.4cm,
    y unit chart=0.6cm,
    y unit title=0.8cm,
    title height=1,
    bar height=0.5,
    group peaks height=0.3,
    title label font=\footnotesize\bfseries,
    bar label font=\footnotesize,
    group label font=\footnotesize\bfseries,
    milestone label font=\footnotesize,
    bar/.append style={fill=MainYellow!60},
    group/.append style={fill=MainBlue!60},
    milestone/.append style={fill=MainRed},
    vgrid={*{11}{draw=LightGray, thin}},
    hgrid={draw=VeryLightGray},
    today=0,
    today rule/.style={draw=MainRed, ultra thick, dashed},
    today label=گذار
]{-6}{54}
    % Titles
    \gantttitlelist{-6,...,0,...,54}{1} \\
    
    % Phase 0
    \ganttgroup{فاز ۰: پیش‌گذار}{-6}{0} \\
    \ganttbar{تشکیل تیم برنامه‌ریزی}{-6}{-3} \\
    \ganttbar{شناسایی \lr{SRSG}}{-6}{-1} \\
    \ganttbar{پیش‌نویس قطعنامه}{-4}{0} \\
    \ganttbar{پیش‌موقعیت‌یابی}{-3}{0} \\
    
    % Phase 1
    \ganttgroup{فاز ۱: تثبیت}{1}{6} \\
    \ganttbar{استقرار اولیه}{1}{3} \\
    \ganttbar{امنیت و نظم}{1}{6} \\
    \ganttbar{قوانین موقت}{2}{5} \\
    \ganttmilestone{تصویب قطعنامه}{1} \\
    
    % Phase 2
    \ganttgroup{فاز ۲: نهادسازی}{7}{24} \\
    \ganttbar{مشاوره عمومی قانون اساسی}{7}{12} \\
    \ganttbar{مجلس مؤسسان}{10}{12} \\
    \ganttbar{تدوین قانون اساسی}{12}{20} \\
    \ganttbar{رفراندوم}{22}{24} \\
    \ganttbar{انتخابات پارلمان}{26}{30} \\
    \ganttmilestone{رفراندوم قانون اساسی}{24} \\
    
    % Phase 3
    \ganttgroup{فاز ۳: تحکیم}{25}{48} \\
    \ganttbar{تشکیل دولت منتخب}{30}{33} \\
    \ganttbar{انتقال تدریجی مسئولیت}{30}{48} \\
    \ganttbar{انتخابات دوم}{42}{48} \\
    \ganttmilestone{انتقال قدرت}{33} \\
    
    % Phase 4
    \ganttgroup{فاز ۴: خروج}{49}{54} \\
    \ganttbar{کاهش حضور}{49}{54} \\
    \ganttmilestone{خروج کامل}{54}
    
\end{ganttchart}
\caption{نمودار گانت کلان عملیات نظارت (ماه -۶ تا ۵۴)}
\label{fig:gantt-chart}
\end{figure}

\begin{lessonlearned}{بوسنی: طولانی‌ترین مأموریت}
مأموریت بین‌المللی بوسنی (\lr{OHR}) از ۱۹۹۵ آغاز شد و تا ۲۰۲۴ (۲۹ سال!) ادامه داشت. دلیل: شروع بدون معیارهای خروج مشخص + ساختار سیاسی ناکارآمد دیتون + بن‌بست قومی. درس ایران: \textbf{معیارهای خروج باید از روز اول تعریف شوند} و هر فاز بر اساس شاخص‌های عینی (نه تقویم سیاسی) پایان یابد.
\end{lessonlearned}

\sectiondivider

% ═══════════════════════════════════════════════════════════════════════════════
\section{مکانیزم بازخورد و اصلاح مسیر}
\label{sec:feedback-mechanism}
% ═══════════════════════════════════════════════════════════════════════════════

هیچ طرحی بدون تغییر اجرا نمی‌شود. مکانیزم بازخورد تضمین می‌کند که اشتباهات شناسایی و اصلاح شوند.

\subsection{سطوح ارزیابی}
\label{subsec:evaluation-levels}

\begin{table}[htbp]
\centering
\caption{نظام ارزیابی پنج‌سطحی}
\label{tab:evaluation-levels}
\begin{tabularx}{\textwidth}{>{\raggedleft\arraybackslash}p{0.5cm}
                             >{\raggedleft\arraybackslash}p{2.5cm}
                             >{\centering\arraybackslash}p{2cm}
                             >{\raggedleft\arraybackslash}X
                             >{\raggedleft\arraybackslash}p{2.5cm}}
\toprule
\headerrow \# & نوع ارزیابی & فرکانس & شرح & مسئول \\
\midrule
۱ & پایش روزانه & روزانه & شاخص‌های هشدار زودهنگام (\seeChapter{ch:risks}) & تیم پایش \\
\altrow ۲ & گزارش ماهانه & ماهانه & گزارش \lr{SRSG} به دبیرکل و شورای امنیت & دفتر \lr{SRSG} \\
۳ & بازنگری فصلی & هر ۳ ماه & بررسی پیشرفت شاخص‌ها، تعدیل برنامه & تیم برنامه‌ریزی \\
\altrow ۴ & ارزیابی مستقل & هر ۶ ماه & تیم خارجی مستقل از \lr{UNMOIT} & \lr{OIOS/UN} \\
۵ & بازنگری استراتژیک & سالانه & بازنگری کل استراتژی، تصمیم ادامه/تعدیل/خروج & شورای امنیت \\
\bottomrule
\end{tabularx}
\end{table}

\subsection{نمودار چرخه بازخورد}
\label{subsec:feedback-cycle}

\begin{figure}[htbp]
\centering
\begin{tikzpicture}[
    node distance=2cm,
    every node/.style={font=\small, align=center},
    step/.style={circle, draw=MainYellow, fill=LightYellow, minimum size=2.2cm, thick},
    decision/.style={diamond, draw=MainRed, fill=LightRed, minimum width=2.5cm, minimum height=2cm, thick, aspect=1.5},
    arrow/.style={-{Stealth[length=3mm]}, thick, MainYellow!80!black}
]
    % Cycle nodes
    \node[step] (plan) {برنامه‌ریزی\\{\footnotesize \lr{Plan}}};
    \node[step, right=2.5cm of plan] (exec) {اجرا\\{\footnotesize \lr{Do}}};
    \node[step, below=2.5cm of exec] (check) {ارزیابی\\{\footnotesize \lr{Check}}};
    \node[decision, below=2.5cm of plan] (decide) {تصمیم\\{\footnotesize \lr{Act}}};
    
    % Arrows
    \draw[arrow] (plan) -- node[above, font=\footnotesize] {استقرار} (exec);
    \draw[arrow] (exec) -- node[right, font=\footnotesize] {داده‌ها} (check);
    \draw[arrow] (check) -- node[below, font=\footnotesize] {تحلیل} (decide);
    
    % Decision outcomes
    \draw[arrow, MainGreen] (decide) -- node[left, font=\footnotesize, MainGreen] {ادامه/بهبود} (plan);
    \draw[arrow, MainOrange, dashed] (decide) to[bend left=50] node[right=0.3cm, font=\footnotesize, MainOrange] {تعدیل جدی} (plan);
    \draw[arrow, MainRed, dotted] (decide) -- ++(0,-1.5) node[below, font=\footnotesize, MainRed] {بازنگری اساسی};
    
    % Center label
    \node[font=\bfseries] at ($(plan)!0.5!(check)$) {\lr{PDCA}};
    
\end{tikzpicture}
\caption{چرخه بازخورد مداوم (مدل \lr{PDCA})}
\label{fig:feedback-cycle}
\end{figure}

\begin{keypoint}
\textbf{سه نتیجه ممکن هر ارزیابی:}
\begin{enumerate}[nosep]
    \item \textbf{\textcolor{MainGreen}{سبز — ادامه}}: شاخص‌ها مثبت، برنامه طبق زمان‌بندی → ادامه با بهبودهای جزئی
    \item \textbf{\textcolor{MainOrange}{نارنجی — تعدیل}}: برخی شاخص‌ها منفی، تأخیر → تعدیل برنامه، تمدید فاز، تغییر اولویت
    \item \textbf{\textcolor{MainRed}{قرمز — بازنگری}}: عقب‌گرد جدی، بحران → بازنگری کل استراتژی، ممکن است نیاز به قطعنامه جدید
\end{enumerate}
\end{keypoint}

\subsection{شاخص‌های کلیدی عملکرد (\lr{KPI})}
\label{subsec:kpis}

\begin{table}[htbp]
\centering
\caption{شاخص‌های کلیدی عملکرد در هر فاز}
\label{tab:kpis}
\begin{tabularx}{\textwidth}{>{\raggedleft\arraybackslash}p{3cm}
                             >{\raggedleft\arraybackslash}X
                             >{\raggedleft\arraybackslash}p{2.5cm}
                             >{\raggedleft\arraybackslash}p{2cm}}
\toprule
\headerrow شاخص & تعریف عملیاتی & هدف & فاز مرتبط \\
\midrule
امنیت عمومی & تعداد حوادث امنیتی در ماه & کاهش ۸۰٪ & فاز ۱ \\
\altrow آزادی رسانه & امتیاز \lr{RSF} آزادی مطبوعات & بالای ۵۰ & فاز ۱-۲ \\
ثبت رأی‌دهندگان & درصد واجدین ثبت‌نام‌شده & بالای ۸۰٪ & فاز ۲ \\
\altrow مشارکت انتخاباتی & درصد رأی‌دهندگان & بالای ۶۰٪ & فاز ۲ \\
اعتماد عمومی & نظرسنجی اعتماد به نهادها & بالای ۵۰٪ & فاز ۲-۳ \\
\altrow حقوق بشر & تعداد موارد نقض مستند & کاهش مستمر & همه فازها \\
برابری جنسیتی & درصد زنان در نهادها & بالای ۳۰٪ & فاز ۲-۳ \\
\altrow فساد & امتیاز \lr{TI CPI} & بالای ۴۰ & فاز ۲-۳ \\
رشد اقتصادی & نرخ رشد \lr{GDP} & مثبت & فاز ۱-۳ \\
\altrow انتقال مسئولیت & درصد فعالیت‌ها با مدیریت ایرانی & بالای ۹۰٪ & فاز ۳ \\
\bottomrule
\end{tabularx}
\end{table}

\sectiondivider

% ═══════════════════════════════════════════════════════════════════════════════
\section{مدیریت انتقال}
\label{sec:transition-management}
% ═══════════════════════════════════════════════════════════════════════════════

\subsection{اصل بنیادین: از حضور به مشارکت به مالکیت}
\label{subsec:presence-to-ownership}

\begin{figure}[htbp]
\centering
\begin{tikzpicture}[
    font=\small
]
    % Axes
    \draw[-{Stealth}, thick] (0,0) -- (12,0) node[below] {زمان (ماه)};
    \draw[-{Stealth}, thick] (0,0) -- (0,6) node[above, rotate=90, anchor=south] {سهم مسئولیت (\%)};
    
    % Grid
    \foreach \y in {1,...,5} {
        \draw[VeryLightGray] (0,\y) -- (11.5,\y);
    }
    
    % Y labels
    \node[anchor=east, font=\footnotesize] at (0,1) {۲۰};
    \node[anchor=east, font=\footnotesize] at (0,2) {۴۰};
    \node[anchor=east, font=\footnotesize] at (0,3) {۶۰};
    \node[anchor=east, font=\footnotesize] at (0,4) {۸۰};
    \node[anchor=east, font=\footnotesize] at (0,5) {۱۰۰};
    
    % X labels
    \node[anchor=north, font=\footnotesize] at (1,0) {۶};
    \node[anchor=north, font=\footnotesize] at (3,0) {۱۲};
    \node[anchor=north, font=\footnotesize] at (5,0) {۲۴};
    \node[anchor=north, font=\footnotesize] at (7,0) {۳۶};
    \node[anchor=north, font=\footnotesize] at (9,0) {۴۸};
    \node[anchor=north, font=\footnotesize] at (11,0) {۶۰};
    
    % International line (decreasing)
    \draw[MainBlue, very thick] plot[smooth] coordinates {(0,4.5) (1,4) (3,3.5) (5,2.5) (7,1.5) (9,1) (11,0.3)};
    \node[MainBlue, anchor=west, font=\footnotesize] at (11.2,0.3) {بین‌المللی};
    
    % Iranian line (increasing)
    \draw[MainGreen, very thick] plot[smooth] coordinates {(0,0.5) (1,1) (3,1.5) (5,2.5) (7,3.5) (9,4) (11,4.7)};
    \node[MainGreen, anchor=west, font=\footnotesize] at (11.2,4.7) {ایرانی};
    
    % Crossover point
    \fill[MainRed] (5,2.5) circle (3pt);
    \node[MainRed, anchor=south, font=\footnotesize\bfseries] at (5,2.7) {نقطه تقاطع (ماه ۲۴)};
    
    % Phase labels
    \draw[dashed, DarkGray] (1,0) -- (1,5.5);
    \draw[dashed, DarkGray] (5,0) -- (5,5.5);
    \draw[dashed, DarkGray] (9,0) -- (9,5.5);
    
    \node[font=\footnotesize\bfseries, DarkGray] at (0.5,5.8) {فاز ۱};
    \node[font=\footnotesize\bfseries, DarkGray] at (3,5.8) {فاز ۲};
    \node[font=\footnotesize\bfseries, DarkGray] at (7,5.8) {فاز ۳};
    \node[font=\footnotesize\bfseries, DarkGray] at (10,5.8) {فاز ۴};
    
\end{tikzpicture}
\caption{منحنی انتقال مسئولیت از بین‌المللی به ایرانی}
\label{fig:transition-curve}
\end{figure}

\begin{warningbox}
\textbf{خطر «وابستگی نهادی»}: اگر نهادهای ایرانی به حضور بین‌المللی عادت کنند و ظرفیت مستقل نسازند، خروج ناظران به خلأ می‌انجامد. تجربه افغانستان نشان داد که ۲۰ سال حضور بدون انتقال واقعی مسئولیت، یک‌شبه فرو می‌ریزد. \textbf{انتقال باید از روز اول آغاز شود}، نه فقط در فاز خروج.
\end{warningbox}

\subsection{ابزارهای انتقال مسئولیت}
\label{subsec:transfer-tools}

\begin{table}[htbp]
\centering
\caption{ابزارهای انتقال مسئولیت}
\label{tab:transfer-tools}
\begin{tabularx}{\textwidth}{>{\raggedleft\arraybackslash}p{3cm}
                             >{\raggedleft\arraybackslash}X
                             >{\raggedleft\arraybackslash}p{2cm}
                             >{\raggedleft\arraybackslash}p{2.5cm}}
\toprule
\headerrow ابزار & شرح & فاز شروع & شاخص موفقیت \\
\midrule
آموزش حین خدمت & هر بین‌المللی یک همتای ایرانی آموزش می‌دهد & فاز ۱ & ۹۰٪ پست‌ها ایرانی \\
\altrow مدیریت سایه‌ای & ایرانی ابتدا ناظر، سپس مدیر مشترک، سپس مستقل & فاز ۱ & عملکرد مستقل \\
بورسیه و اعزام & اعزام ایرانیان به مأموریت‌های \lr{UN} در کشورهای دیگر & فاز ۲ & ۵۰۰+ بورسیه \\
\altrow مستندسازی دانش & ثبت رویه‌ها، درس‌آموخته‌ها، راهنماها به فارسی & فاز ۱ & کتابخانه کامل \\
آزمون آمادگی & ارزیابی دوره‌ای توانایی نهاد ایرانی برای مدیریت مستقل & فاز ۲ & قبولی ۸۰٪+ \\
\altrow خروج آزمایشی & ناظران ۱ ماه غایب، عملکرد ارزیابی می‌شود & فاز ۳ & بدون افت عملکرد \\
\bottomrule
\end{tabularx}
\end{table}

\begin{recommendation}
\textbf{قاعده ۳۰-۵۰-۸۰}:
\begin{itemize}[nosep]
    \item \textbf{فاز ۱}: حداقل ۳۰٪ مدیریت توسط ایرانیان
    \item \textbf{فاز ۲}: حداقل ۵۰٪ مدیریت توسط ایرانیان (نقطه تقاطع)
    \item \textbf{فاز ۳}: حداقل ۸۰٪ مدیریت توسط ایرانیان
    \item \textbf{فاز ۴}: ۱۰۰٪ مدیریت ایرانی (بین‌المللی‌ها فقط مشاور)
\end{itemize}
\end{recommendation}

\sectiondivider

% ═══════════════════════════════════════════════════════════════════════════════
\section{مدیریت ارتباطات و اطلاع‌رسانی}
\label{sec:communications-management}
% ═══════════════════════════════════════════════════════════════════════════════

\subsection{استراتژی ارتباطات}
\label{subsec:comms-strategy}

\begin{table}[htbp]
\centering
\caption{مخاطبان و پیام‌های کلیدی}
\label{tab:comms-audiences}
\begin{tabularx}{\textwidth}{>{\raggedleft\arraybackslash}p{2.5cm}
                             >{\raggedleft\arraybackslash}X
                             >{\raggedleft\arraybackslash}p{3cm}}
\toprule
\headerrow مخاطب & پیام کلیدی & کانال اصلی \\
\midrule
مردم ایران & «این گذار متعلق به شماست؛ ما کمک می‌کنیم» & تلویزیون، رادیو، شبکه‌های اجتماعی \\
\altrow نخبگان سیاسی & «فراگیری و مشارکت، نه حذف» & جلسات مستقیم، بیانیه‌ها \\
نیروهای امنیتی & «جایی برای شما در آینده هست — اگر به قانون احترام بگذارید» & کانال‌های خاص، میانجیان \\
\altrow جامعه بین‌المللی & «سرمایه‌گذاری در ثبات ایران، سرمایه‌گذاری در امنیت جهانی» & رسانه‌های بین‌المللی، کنفرانس‌ها \\
دیاسپورا & «تخصص شما ارزشمند است — با مسئولیت مشارکت کنید» & رسانه‌های فارسی‌زبان خارج \\
\altrow مخالفان گذار & «فرایند عادلانه است؛ صدای شما هم شنیده می‌شود» & گفتگوی مستقیم، رسانه \\
\bottomrule
\end{tabularx}
\end{table}

\begin{casestudy}{تونس: قدرت شفافیت}
یکی از عوامل موفقیت نسبی گذار تونس، شفافیت فرایند قانون‌نویسی بود. جلسات مجلس مؤسسان زنده پخش می‌شد. مردم پیش‌نویس قانون اساسی را آنلاین می‌خواندند و نظر می‌دادند. این شفافیت اعتماد عمومی را بالا برد و مقاومت اسلام‌گرایان افراطی را خنثی کرد.
\end{casestudy}

\subsection{مقابله با اطلاعات نادرست}
\label{subsec:counter-disinfo}

\begin{warningbox}
در عصر شبکه‌های اجتماعی، \bilingual{اطلاعات نادرست}{Disinformation} بزرگ‌ترین تهدید اطلاعاتی است. عناصر رژیم قبلی، قدرت‌های خارجی (روسیه، چین)، و گروه‌های افراطی همگی انگیزه و ظرفیت انتشار اطلاعات نادرست دارند:
\begin{itemize}[nosep]
    \item «گذار توطئه خارجی است»
    \item «ناظران جاسوس‌اند»
    \item «انتخابات تقلبی بود»
    \item «فلان قوم می‌خواهد جدا شود»
\end{itemize}
\end{warningbox}

\textbf{راهکارهای مقابله:}
\begin{enumerate}[nosep]
    \item تیم واکنش سریع رسانه‌ای (پاسخ در کمتر از ۱ ساعت)
    \item همکاری با پلتفرم‌های اجتماعی برای حذف محتوای مخرب
    \item شبکه «بررسی واقعیت» (\lr{Fact-Checking}) فارسی‌زبان
    \item آموزش سواد رسانه‌ای عمومی
    \item شفافیت حداکثری خود مأموریت (بهترین پادزهر شایعه)
\end{enumerate}

\sectiondivider

% ═══════════════════════════════════════════════════════════════════════════════
\section{خلاصه زمانی یکپارچه}
\label{sec:integrated-timeline}
% ═══════════════════════════════════════════════════════════════════════════════

\begin{table}[htbp]
\centering
\caption{خلاصه نقاط عطف کلیدی}
\label{tab:milestones-summary}
\begin{tabularx}{\textwidth}{>{\centering\arraybackslash}p{2cm}
                             >{\raggedleft\arraybackslash}p{4cm}
                             >{\raggedleft\arraybackslash}X}
\toprule
\headerrow ماه & نقطه عطف & شاخص تحقق \\
\midrule
ماه ۰ & \emphred{سقوط رژیم / آغاز گذار} & رویداد سیاسی \\
\altrow ساعت ۷۲ & اعزام تیم ارزیابی & تیم در تهران \\
هفته ۲ & تصویب قطعنامه شورای امنیت & متن مصوب \\
\altrow ماه ۱ & استقرار \lr{SRSG} و تیم پیشتاز & دفتر فعال \\
ماه ۳ & ۳۱ دفتر استانی فعال & پوشش سراسری \\
\altrow ماه ۶ & پایان فاز تثبیت & امنیت نسبی، قوانین موقت \\
ماه ۱۲ & انتخابات مجلس مؤسسان & مجلس تشکیل‌شده \\
\altrow ماه ۲۴ & رفراندوم قانون اساسی & تصویب مردمی \\
ماه ۳۰ & انتخابات پارلمان & پارلمان منتخب \\
\altrow ماه ۳۳ & انتقال قدرت به دولت منتخب & مراسم تحلیف \\
ماه ۴۸ & انتخابات دوم (محلی/پارلمانی) & انتقال مسالمت‌آمیز \\
\altrow ماه ۶۰ & تبدیل مأموریت به دفتر سیاسی & کاهش ۸۰٪ حضور \\
ماه ۱۲۰ & \emphgreen{خروج کامل} & پایان مأموریت \\
\bottomrule
\end{tabularx}
\end{table}

\sectiondivider

% ═══════════════════════════════════════════════════════════════════════════════
% جمع‌بندی فصل
% ═══════════════════════════════════════════════════════════════════════════════

\begin{chaptersummary}
یافته‌های کلیدی این فصل:

\begin{enumerate}
    \item \textbf{فاز ۰ (پیش‌گذار) مهم‌ترین فاز است}: آمادگی قبل از بحران تفاوت میان موفقیت و فاجعه است. تشکیل تیم برنامه‌ریزی، شناسایی \lr{SRSG}، و پیش‌موقعیت‌یابی باید \emph{الان} آغاز شود.
    
    \item \textbf{۷۲ ساعت اول سرنوشت‌ساز است}: اعزام تیم ارزیابی، بیانیه دبیرکل، و نشست شورای امنیت باید در سه روز اول اتفاق بیفتد.
    
    \item \textbf{مدل «اول قانون اساسی، بعد انتخابات»}: تجربه تونس نشان داد که صبر برای تدوین قانون اساسی توافقی، از انتخابات عجولانه (مدل مصر/عراق) بهتر است.
    
    \item \textbf{خروج بر اساس شاخص، نه تقویم}: معیارهای خروج باید از روز اول تعریف و پایش شوند. خروج زودهنگام به‌اندازه ماندن بیش‌ازحد خطرناک است.
    
    \item \textbf{قاعده ۳۰-۵۰-۸۰}: سهم مدیریت ایرانی باید از ۳۰٪ در فاز ۱ به ۱۰۰٪ در فاز ۴ برسد. انتقال از روز اول آغاز می‌شود.
    
    \item \textbf{ماتریس \lr{RACI} ابهام نقش‌ها را رفع می‌کند}: تعیین دقیق مسئول اجرا، پاسخگو، مشاور، و اطلاع‌یافته برای هر فعالیت.
    
    \item \textbf{مکانیزم بازخورد پنج‌سطحی}: از پایش روزانه تا بازنگری استراتژیک سالانه، با شاخص‌های عینی.
    
    \item \textbf{ارتباطات استراتژیک حیاتی است}: پیام‌های متناسب با هر مخاطب، شفافیت حداکثری، و مقابله فعال با اطلاعات نادرست.
\end{enumerate}

\vspace{0.5cm}
\textit{در فصل بعد (\ref{ch:budget})، بودجه‌بندی تفصیلی، منابع مالی، و مکانیزم شفافیت مالی بررسی خواهد شد.}
\end{chaptersummary}

\chapterend
% ═══════════════════════════════════════════════════════════════════════════════
% فصل ۱۰: بودجه‌بندی و تأمین مالی
% فایل: chapters/ch10-budget.tex
% رنگ فصل: زرد (MainYellow)
% ═══════════════════════════════════════════════════════════════════════════════

\chapteropening{۱۰}{بودجه‌بندی و تأمین مالی}{MainYellow}{%
اگر فکر می‌کنید هزینه صلح گران است، هزینه جنگ را حساب کنید.%
}{کوفی عنان}

\chapter{بودجه‌بندی و تأمین مالی}
\label{ch:budget}

\minitoc

% ─────────────────────────────────────────────────────────────────────────────
% خلاصه اجرایی
% ─────────────────────────────────────────────────────────────────────────────

\begin{executivesummary}
برآورد کل بودجه نظارت بین‌المللی بر گذار ایران در بازه ۱۰ ساله، \textbf{۲.۵ تا ۵ میلیارد دلار} است — رقمی قابل‌توجه اما در مقایسه با هزینه‌های مداخله نظامی (عراق: ۶۰+ میلیارد دلار فقط برای بازسازی) یا هزینه عدم اقدام (بحران پناهندگی، تروریسم، بی‌ثباتی منطقه‌ای) بسیار معقول. این فصل بودجه را به تفکیک فاز، بخش، و نوع هزینه ارائه می‌دهد. منابع مالی شامل مشارکت سازمان ملل (۳۰٪)، اتحادیه اروپا (۲۰٪)، آمریکا (۱۵٪)، ژاپن/کره (۱۰٪)، کشورهای خلیج فارس (۱۰٪)، دارایی‌های آزادشده ایران (۱۰٪)، و بخش خصوصی (۵٪) است. شفافیت مالی و حسابرسی مستقل، ستون‌های اعتمادسازی در این حوزه‌اند.
\end{executivesummary}

\section{درآمد: اقتصاد صلح‌سازی}
\label{sec:budget-intro}

بودجه‌بندی نظارت بین‌المللی صرفاً یک تمرین حسابداری نیست؛ بیان \emph{اولویت‌های استراتژیک} در قالب ارقام است. تخصیص منابع مالی نشان می‌دهد که جامعه بین‌المللی واقعاً چه چیزی را مهم می‌داند.

\begin{keypoint}
\textbf{اقتصاد پیشگیری}: هر ۱ دلار سرمایه‌گذاری در پیشگیری از بحران، ۱۶ دلار در هزینه‌های واکنش صرفه‌جویی می‌کند.\footnote{\lr{United Nations/World Bank. \emph{Pathways for Peace: Inclusive Approaches to Preventing Violent Conflict}. 2018.}} نظارت مؤثر بر گذار ایران، سرمایه‌گذاری در پیشگیری از بحرانی است که می‌تواند ده‌ها میلیارد دلار و صدها هزار جان انسانی هزینه داشته باشد.
\end{keypoint}

\begin{table}[htbp]
\centering
\caption{مقایسه هزینه‌های مداخله بین‌المللی در نمونه‌های تاریخی}
\label{tab:cost-comparison}
\begin{tabularx}{\textwidth}{>{\raggedleft\arraybackslash}p{3cm}
                             >{\centering\arraybackslash}p{2cm}
                             >{\centering\arraybackslash}p{2cm}
                             >{\centering\arraybackslash}p{2cm}
                             >{\raggedleft\arraybackslash}X}
\toprule
\headerrow کشور/مأموریت & جمعیت (M) & مدت (سال) & هزینه کل (\$B) & هزینه سرانه \\
\midrule
عراق (بازسازی) & ۳۸ & ۸ & ۶۰+ & ۱,۵۸۰ \\
\altrow افغانستان (کل) & ۳۸ & ۲۰ & ۱۴۵+ & ۳,۸۰۰ \\
تیمور شرقی (\lr{UNTAET}) & ۰.۸ & ۳ & ۳ & ۳,۷۵۰ \\
\altrow کوزوو (\lr{UNMIK}) & ۱.۸ & ۲۰+ & ۵+ & ۲,۸۰۰ \\
بوسنی (\lr{OHR}+\lr{SFOR}) & ۳.۵ & ۲۵+ & ۱۵+ & ۴,۳۰۰ \\
\altrow کامبوج (\lr{UNTAC}) & ۱۰ & ۲ & ۱.۶ & ۱۶۰ \\
\midrule
\textbf{ایران (پیشنهادی)} & \textbf{۸۵} & \textbf{۱۰} & \textbf{۲.۵-۵} & \textbf{۳۰-۶۰} \\
\bottomrule
\end{tabularx}
\end{table}

\begin{lessonlearned}{عراق: هزینه عدم برنامه‌ریزی}
آمریکا بیش از ۶۰ میلیارد دلار صرف «بازسازی» عراق کرد، اما حسابرسی \lr{SIGIR} نشان داد که حداقل ۸ میلیارد دلار به هدر رفته یا سرقت شده است. دلیل: عدم برنامه‌ریزی پیشینی، نبود مکانیزم نظارت مالی، و عجله سیاسی. هزینه سرانه بازسازی عراق (۱,۵۸۰ دلار) ۲۵-۵۰ برابر بیشتر از برآورد ایران (۳۰-۶۰ دلار) است — زیرا مدل ایران مبتنی بر نظارت است نه مداخله.
\end{lessonlearned}

\sectiondivider

% ═══════════════════════════════════════════════════════════════════════════════
\section{برآورد بودجه به تفکیک فاز}
\label{sec:budget-by-phase}
% ═══════════════════════════════════════════════════════════════════════════════

\subsection{بودجه کلی فازها}
\label{subsec:phase-budgets}

\begin{table}[htbp]
\centering
\caption{برآورد بودجه به تفکیک فاز (میلیون دلار)}
\label{tab:budget-by-phase}
\begin{tabularx}{\textwidth}{>{\raggedleft\arraybackslash}p{3cm}
                             >{\centering\arraybackslash}p{2cm}
                             >{\centering\arraybackslash}p{2cm}
                             >{\centering\arraybackslash}p{2cm}
                             >{\raggedleft\arraybackslash}X}
\toprule
\headerrow فاز & مدت & حداقل (\$M) & حداکثر (\$M) & توضیح \\
\midrule
فاز ۰ (پیش‌گذار) & ۶-۱۲ ماه & ۵۰ & ۱۰۰ & برنامه‌ریزی، آموزش، پیش‌موقعیت \\
\altrow فاز ۱ (تثبیت) & ۶ ماه & ۴۰۰ & ۸۰۰ & استقرار سریع، امنیت، نظم اولیه \\
فاز ۲ (نهادسازی) & ۱۸ ماه & ۸۰۰ & ۱,۵۰۰ & انتخابات، قانون اساسی، نهادها \\
\altrow فاز ۳ (تحکیم) & ۳۶ ماه & ۶۰۰ & ۱,۲۰۰ & تحکیم، انتقال، انتخابات دوم \\
فاز ۴ (پی‌گیری) & ۶۰ ماه & ۱۵۰ & ۴۰۰ & مشاوره، ارزیابی، خروج \\
\midrule
\textbf{مجموع} & \textbf{۱۰ سال} & \textbf{۲,۰۰۰} & \textbf{۴,۰۰۰} & \\
\altrow \textbf{+ ذخیره احتیاطی ۲۵٪} & & \textbf{۵۰۰} & \textbf{۱,۰۰۰} & پوشش ریسک و تغییرات \\
\midrule
\textbf{مجموع نهایی} & & \textbf{۲,۵۰۰} & \textbf{۵,۰۰۰} & \\
\bottomrule
\end{tabularx}
\end{table}

\begin{casestudy}{ذخیره احتیاطی: درس افغانستان}
بودجه اولیه \lr{ISAF} در افغانستان بارها بازنگری شد — هر بار افزایش. عوامل غیرقابل‌پیش‌بینی (شورش طالبان، فساد، بلایای طبیعی) بودجه را ۲-۳ برابر تخمین اولیه بالا بردند. ذخیره احتیاطی ۲۵٪ ضروری و حداقلی است.
\end{casestudy}

\subsection{نمودار توزیع زمانی بودجه}
\label{subsec:budget-timeline-chart}

\begin{figure}[htbp]
\centering
\begin{tikzpicture}
\begin{axis}[
    ybar,
    width=14cm,
    height=7cm,
    bar width=1.2cm,
    ylabel={میلیون دلار},
    xlabel={سال},
    symbolic x coords={سال ۰, سال ۱, سال ۲, سال ۳, سال ۴, سال ۵, سال ۶-۷, سال ۸-۱۰},
    xtick=data,
    x tick label style={font=\footnotesize},
    y tick label style={font=\footnotesize},
    ymin=0,
    ymax=1100,
    nodes near coords,
    every node near coord/.append style={font=\tiny},
    legend style={at={(0.5,-0.25)}, anchor=north, legend columns=2, font=\footnotesize},
    axis lines=left,
    enlarge x limits=0.1,
]
    \addplot[fill=MainRed!60] coordinates {
        (سال ۰, 75) (سال ۱, 800) (سال ۲, 600) (سال ۳, 400) 
        (سال ۴, 300) (سال ۵, 250) (سال ۶-۷, 150) (سال ۸-۱۰, 100)
    };
    \addplot[fill=MainYellow!60] coordinates {
        (سال ۰, 0) (سال ۱, 200) (سال ۲, 300) (سال ۳, 200) 
        (سال ۴, 100) (سال ۵, 50) (سال ۶-۷, 25) (سال ۸-۱۰, 0)
    };
    \legend{هزینه‌های عملیاتی, هزینه‌های سرمایه‌ای}
\end{axis}
\end{tikzpicture}
\caption{توزیع زمانی بودجه (سناریوی میانه)}
\label{fig:budget-timeline}
\end{figure}

\sectiondivider

% ═══════════════════════════════════════════════════════════════════════════════
\section{برآورد بودجه به تفکیک بخش}
\label{sec:budget-by-sector}
% ═══════════════════════════════════════════════════════════════════════════════

\begin{landscape}
\begin{table}[htbp]
\centering
\bigtablefontsize
\caption{برآورد بودجه به تفکیک بخش و فاز (میلیون دلار — سناریوی میانه)}
\label{tab:budget-by-sector}
\begin{tabularx}{\linewidth}{>{\raggedleft\arraybackslash}p{3cm}
                             >{\centering\arraybackslash}p{1.5cm}
                             >{\centering\arraybackslash}p{1.5cm}
                             >{\centering\arraybackslash}p{1.5cm}
                             >{\centering\arraybackslash}p{1.5cm}
                             >{\centering\arraybackslash}p{1.5cm}
                             >{\centering\arraybackslash}p{1.5cm}
                             >{\centering\arraybackslash}p{1.2cm}
                             >{\raggedleft\arraybackslash}X}
\toprule
\headerrow بخش & فاز ۰ & فاز ۱ & فاز ۲ & فاز ۳ & فاز ۴ & مجموع & \% & توضیح \\
\midrule
نیروی انسانی (حقوق و مزایا) & ۲۰ & ۲۵۰ & ۵۰۰ & ۳۵۰ & ۱۰۰ & ۱,۲۲۰ & ۳۶٪ & بزرگ‌ترین قلم \\
\altrow نظارت انتخاباتی & ۵ & ۳۰ & ۲۵۰ & ۱۰۰ & ۲۰ & ۴۰۵ & ۱۲٪ & شامل ثبت‌نام و رأی‌گیری \\
حقوق بشر و عدالت انتقالی & ۵ & ۴۰ & ۸۰ & ۸۰ & ۳۰ & ۲۳۵ & ۷٪ & کمیسیون حقیقت + دادگاه \\
\altrow اصلاح بخش امنیتی & ۰ & ۸۰ & ۱۲۰ & ۸۰ & ۲۰ & ۳۰۰ & ۹٪ & \lr{DDR}، آموزش، نظارت \\
لجستیک و زیرساخت & ۱۰ & ۱۰۰ & ۸۰ & ۵۰ & ۲۰ & ۲۶۰ & ۸٪ & ساختمان، حمل‌ونقل، تجهیزات \\
\altrow فناوری اطلاعات & ۱۰ & ۵۰ & ۶۰ & ۳۰ & ۱۰ & ۱۶۰ & ۵٪ & سامانه‌ها، ارتباطات، سایبری \\
آموزش و ظرفیت‌سازی & ۱۰ & ۲۰ & ۵۰ & ۶۰ & ۳۰ & ۱۷۰ & ۵٪ & ایرانی و بین‌المللی \\
\altrow رسانه و ارتباطات & ۵ & ۲۰ & ۴۰ & ۳۰ & ۱۰ & ۱۰۵ & ۳٪ & ارتباطات عمومی، ضد اطلاعات نادرست \\
امنیت کارکنان & ۵ & ۳۰ & ۵۰ & ۳۰ & ۱۰ & ۱۲۵ & ۴٪ & حفاظت، بیمه، تخلیه \\
\altrow هماهنگی و مدیریت & ۵ & ۳۰ & ۴۰ & ۳۰ & ۱۰ & ۱۱۵ & ۳٪ & دفتر \lr{SRSG}، ارزیابی \\
ذخیره احتیاطی & --- & --- & --- & --- & --- & ۶۵۵ & ۸٪ (مبنا) & \\
\midrule
\textbf{مجموع} & \textbf{۷۵} & \textbf{۶۵۰} & \textbf{۱,۲۷۰} & \textbf{۸۴۰} & \textbf{۲۶۰} & \textbf{۳,۷۵۰} & \textbf{۱۰۰٪} & سناریوی میانه \\
\bottomrule
\end{tabularx}
\end{table}
\end{landscape}

\subsection{تحلیل اقلام بودجه‌ای}
\label{subsec:budget-analysis}

\subsubsection{نیروی انسانی (۳۶٪ — بزرگ‌ترین قلم)}

\begin{table}[htbp]
\centering
\caption{تفکیل هزینه‌های نیروی انسانی}
\label{tab:hr-costs}
\begin{tabularx}{\textwidth}{>{\raggedleft\arraybackslash}p{3.5cm}
                             >{\centering\arraybackslash}p{2cm}
                             >{\centering\arraybackslash}p{2cm}
                             >{\raggedleft\arraybackslash}X}
\toprule
\headerrow دسته & تعداد (اوج) & هزینه سالانه (\$K/نفر) & مجموع ۱۰ ساله (\$M) \\
\midrule
کارکنان ارشد بین‌المللی (\lr{D/P5}) & ۲۰۰ & ۲۰۰-۳۰۰ & ۱۵۰ \\
\altrow کارکنان حرفه‌ای بین‌المللی (\lr{P2-P4}) & ۲,۰۰۰ & ۱۲۰-۲۰۰ & ۴۰۰ \\
کارکنان عمومی بین‌المللی (\lr{GS}) & ۱,۵۰۰ & ۶۰-۱۰۰ & ۱۵۰ \\
\altrow مشاوران کوتاه‌مدت & ۱,۰۰۰ & ۱۵۰-۲۵۰ & ۱۲۰ \\
کارکنان ایرانی (دائم) & ۲۰,۰۰۰ & ۱۵-۳۰ & ۳۰۰ \\
\altrow ناظران موقت انتخابات & ۱۰۰,۰۰۰ & ۰.۵-۱ (روزانه) & ۱۰۰ \\
\midrule
\textbf{مجموع} & & & \textbf{۱,۲۲۰} \\
\bottomrule
\end{tabularx}
\end{table}

\begin{warningbox}
تفاوت حقوق کارکنان بین‌المللی و ایرانی (۵-۱۰ برابر) می‌تواند منبع نارضایتی باشد. راهکارها:
\begin{itemize}[nosep]
    \item شفافیت در توضیح ساختار حقوقی \lr{UN}
    \item فوق‌العاده سختی کار و ریسک برای کارکنان ایرانی
    \item مسیر ارتقا برای کارکنان ایرانی به پست‌های بین‌المللی
    \item هزینه زندگی محلی در تعیین حقوق ایرانیان لحاظ شود
\end{itemize}
\end{warningbox}

\subsubsection{نظارت انتخاباتی (۱۲٪)}

\begin{table}[htbp]
\centering
\caption{تفکیک هزینه‌های انتخاباتی}
\label{tab:election-costs}
\begin{tabularx}{\textwidth}{>{\raggedleft\arraybackslash}p{3.5cm}
                             >{\centering\arraybackslash}p{2.5cm}
                             >{\raggedleft\arraybackslash}X}
\toprule
\headerrow قلم & هزینه (\$M) & توضیح \\
\midrule
ثبت‌نام رأی‌دهندگان & ۸۰ & سامانه دیجیتال + محلی، ۶۰M+ نفر \\
\altrow تجهیزات رأی‌گیری & ۶۰ & صندوق، برگه، جوهر، حمل‌ونقل \\
آموزش کارکنان انتخاباتی & ۴۰ & ۱۰۰,۰۰۰+ نفر \\
\altrow اطلاع‌رسانی عمومی & ۳۰ & رسانه، آموزش رأی‌دهندگان \\
نظارت بین‌المللی & ۸۰ & ناظران \lr{OSCE/EU/Carter} \\
\altrow لجستیک انتخاباتی & ۵۰ & حمل‌ونقل مواد به ۶۵,۰۰۰+ شعبه \\
فناوری شمارش و گزارش & ۳۰ & انتقال نتایج، پایگاه داده \\
\altrow رسیدگی به شکایات & ۱۵ & هیئت‌های حل اختلاف \\
ناظران داخلی & ۲۰ & آموزش و تجهیز \\
\midrule
\textbf{مجموع} & \textbf{۴۰۵} & برای ۳-۴ انتخابات \\
\bottomrule
\end{tabularx}
\end{table}

\begin{keypoint}
هزینه سرانه انتخابات در ایران حدود ۵-۸ دلار به ازای هر رأی‌دهنده برآورد می‌شود. مقایسه: عراق ۲۰۰۵ حدود ۱۲ دلار، افغانستان ۲۰۰۴ حدود ۲۰ دلار، تیمور شرقی ۱۹۹۹ حدود ۳۵ دلار. هزینه پایین‌تر ایران به دلیل زیرساخت‌های بهتر و باسوادی بالاتر است.
\end{keypoint}

\sectiondivider

% ═══════════════════════════════════════════════════════════════════════════════
\section{منابع مالی}
\label{sec:funding-sources}
% ═══════════════════════════════════════════════════════════════════════════════

\subsection{ترکیب پیشنهادی منابع}
\label{subsec:funding-mix}

\begin{figure}[htbp]
\centering
\begin{tikzpicture}
\begin{axis}[
    width=10cm,
    height=10cm,
    xbar,
    bar width=0.6cm,
    xlabel={درصد از بودجه کل},
    symbolic y coords={
        بخش خصوصی,
        دارایی‌های ایران,
        کشورهای خلیج فارس,
        ژاپن و کره,
        آمریکا,
        اتحادیه اروپا,
        سازمان ملل
    },
    ytick=data,
    y tick label style={font=\small},
    xmin=0,
    xmax=35,
    nodes near coords,
    every node near coord/.append style={font=\small},
    axis lines=left,
    enlarge y limits=0.15,
]
    \addplot[fill=MainBlue!60] coordinates {
        (30, سازمان ملل)
        (20, اتحادیه اروپا)
        (15, آمریکا)
        (10, ژاپن و کره)
        (10, کشورهای خلیج فارس)
        (10, دارایی‌های ایران)
        (5, بخش خصوصی)
    };
\end{axis}
\end{tikzpicture}
\caption{ترکیب پیشنهادی منابع مالی}
\label{fig:funding-sources}
\end{figure}

\subsection{تحلیل هر منبع}
\label{subsec:source-analysis}

\begin{table}[htbp]
\centering
\caption{تحلیل تفصیلی منابع مالی}
\label{tab:funding-analysis}
\begin{tabularx}{\textwidth}{>{\raggedleft\arraybackslash}p{2.5cm}
                             >{\centering\arraybackslash}p{1.5cm}
                             >{\centering\arraybackslash}p{1.5cm}
                             >{\raggedleft\arraybackslash}X
                             >{\raggedleft\arraybackslash}p{2.5cm}}
\toprule
\headerrow منبع & سهم (\%) & مبلغ (\$M) & توجیه & ریسک اصلی \\
\midrule
بودجه ارزیابی‌شده \lr{UN} & ۳۰ & ۷۵۰-۱,۵۰۰ & تعهد جمعی، مشروعیت & تأخیر تصویب \\
\altrow اتحادیه اروپا & ۲۰ & ۵۰۰-۱,۰۰۰ & بزرگ‌ترین کمک‌دهنده صلح & بروکراسی \\
آمریکا (دوجانبه) & ۱۵ & ۳۷۵-۷۵۰ & منفعت استراتژیک & تغییر دولت \\
\altrow ژاپن + کره جنوبی & ۱۰ & ۲۵۰-۵۰۰ & سابقه مشارکت + منافع انرژی & محدودیت قانون اساسی \\
کشورهای خلیج فارس & ۱۰ & ۲۵۰-۵۰۰ & ثبات منطقه‌ای & شرط‌گذاری سیاسی \\
\altrow دارایی‌های آزادشده ایران & ۱۰ & ۲۵۰-۵۰۰ & دارایی‌های بلوکه‌شده ۱۰۰B+ & اختلاف حقوقی \\
بخش خصوصی + بنیادها & ۵ & ۱۲۵-۲۵۰ & \lr{OSF}, \lr{Gates}, شرکت‌ها & پایداری نامطمئن \\
\bottomrule
\end{tabularx}
\end{table}

\begin{warningbox}
\textbf{اصل تنوع مالی}: هیچ منبع واحدی نباید بیش از ۳۰٪ بودجه را تأمین کند. وابستگی مالی = وابستگی سیاسی. در عراق، تأمین ۹۰٪+ بودجه توسط آمریکا، مأموریت را عملاً آمریکایی کرد و مشروعیت آن را تضعیف نمود.
\end{warningbox}

\subsection{دارایی‌های آزادشده ایران}
\label{subsec:iranian-assets}

\begin{table}[htbp]
\centering
\caption{برآورد دارایی‌های بلوکه‌شده ایران}
\label{tab:frozen-assets}
\begin{tabularx}{\textwidth}{>{\raggedleft\arraybackslash}p{3.5cm}
                             >{\centering\arraybackslash}p{2.5cm}
                             >{\raggedleft\arraybackslash}X}
\toprule
\headerrow نوع دارایی & برآورد (\$B) & توضیح \\
\midrule
ذخایر ارزی بلوکه‌شده & ۵۰-۸۰ & در بانک‌های خارجی \\
\altrow دارایی‌های بانک مرکزی & ۲۰-۴۰ & طلا و ارز خارجی \\
درآمد نفتی معوقه & ۲۰-۳۰ & فروش‌های بلوکه‌شده \\
\altrow دارایی‌های سپاه/بنیادها خارجی & ۵-۱۵ & شرکت‌ها و حساب‌های خارجی \\
\midrule
\textbf{مجموع} & \textbf{۹۵-۱۶۵} & \\
\bottomrule
\end{tabularx}
\end{table}

\begin{recommendation}
اختصاص ۱-۳٪ از دارایی‌های آزادشده (۱-۵ میلیارد دلار) به صندوق امانی گذار:
\begin{itemize}[nosep]
    \item مشروعیت بالا: «پول ایرانی‌ها برای ایرانی‌ها»
    \item کاهش وابستگی به حامیان خارجی
    \item مدیریت مشترک (ایرانی-بین‌المللی) صندوق امانی
    \item شفافیت کامل در مصرف
\end{itemize}
\end{recommendation}

\sectiondivider

% ═══════════════════════════════════════════════════════════════════════════════
\section{مکانیزم‌های مالی}
\label{sec:financial-mechanisms}
% ═══════════════════════════════════════════════════════════════════════════════

\subsection{صندوق امانی چندجانبه}
\label{subsec:multi-donor-trust}

\begin{figure}[htbp]
\centering
\begin{tikzpicture}[
    node distance=1.2cm,
    every node/.style={font=\small, align=center},
    source/.style={rectangle, rounded corners, draw=MainBlue, fill=LightBlue, minimum width=2cm, minimum height=0.7cm},
    fund/.style={rectangle, rounded corners, draw=MainYellow, fill=LightYellow, minimum width=4cm, minimum height=1.2cm, thick},
    dest/.style={rectangle, rounded corners, draw=MainGreen, fill=LightGreen, minimum width=2.5cm, minimum height=0.7cm},
    audit/.style={rectangle, rounded corners, draw=MainRed, fill=LightRed, minimum width=3cm, minimum height=0.7cm},
    arrow/.style={-{Stealth[length=2.5mm]}, thick}
]
    % Sources
    \node[source] (s1) {\lr{UN}};
    \node[source, below=0.4cm of s1] (s2) {\lr{EU}};
    \node[source, below=0.4cm of s2] (s3) {آمریکا};
    \node[source, below=0.4cm of s3] (s4) {ژاپن/کره};
    \node[source, below=0.4cm of s4] (s5) {خلیج فارس};
    \node[source, below=0.4cm of s5] (s6) {دارایی‌های ایران};
    \node[source, below=0.4cm of s6] (s7) {بخش خصوصی};
    
    % Trust Fund
    \node[fund, right=3cm of s4] (fund) {\textbf{صندوق امانی}\\{\footnotesize هیئت امنا: ایرانی + بین‌المللی}};
    
    % Destinations
    \node[dest, right=3cm of fund] (d1) {انتخابات};
    \node[dest, above=0.3cm of d1] (d2) {حقوق بشر};
    \node[dest, above=0.3cm of d2] (d3) {امنیت};
    \node[dest, below=0.3cm of d1] (d4) {قضایی};
    \node[dest, below=0.3cm of d4] (d5) {ظرفیت‌سازی};
    
    % Audit
    \node[audit, below=1.5cm of fund] (audit) {حسابرسی مستقل\\{\footnotesize + گزارش عمومی}};
    
    % Arrows - sources to fund
    \foreach \s in {s1,s2,s3,s4,s5,s6,s7} {
        \draw[arrow, MainBlue] (\s.east) -- ++(0.5,0) |- (fund.west);
    }
    
    % Arrows - fund to destinations
    \foreach \d in {d1,d2,d3,d4,d5} {
        \draw[arrow, MainGreen] (fund.east) -- ++(0.5,0) |- (\d.west);
    }
    
    % Audit arrow
    \draw[arrow, MainRed, dashed] (fund) -- (audit);
    \draw[arrow, MainRed, dashed] (audit.east) -- ++(2,0) node[right, font=\footnotesize] {گزارش عمومی};
    
\end{tikzpicture}
\caption{نمودار جریان مالی صندوق امانی}
\label{fig:trust-fund-flow}
\end{figure}

\subsection{مکانیزم‌های تخصیص بودجه}
\label{subsec:allocation-mechanisms}

\begin{table}[htbp]
\centering
\caption{سه مکانیزم تخصیص بودجه}
\label{tab:allocation-mechanisms}
\begin{tabularx}{\textwidth}{>{\raggedleft\arraybackslash}p{3cm}
                             >{\raggedleft\arraybackslash}X
                             >{\raggedleft\arraybackslash}X
                             >{\centering\arraybackslash}p{2cm}}
\toprule
\headerrow مکانیزم & مزایا & معایب & سهم پیشنهادی \\
\midrule
بودجه ارزیابی‌شده \lr{UN} & پایدار، الزام‌آور & کُند، بوروکراتیک & ۳۰-۴۰٪ \\
\altrow صندوق امانی چندجانبه & انعطاف‌پذیر، هماهنگ & وابسته به حسن‌نیت حامیان & ۴۰-۵۰٪ \\
کمک‌های دوجانبه & سریع، انعطاف‌پذیر & خطر سوگیری، ناهماهنگی & ۱۰-۲۰٪ \\
\bottomrule
\end{tabularx}
\end{table}

\begin{lessonlearned}{تیمور شرقی: موفقیت صندوق امانی}
در تیمور شرقی، صندوق امانی مدیریت‌شده توسط بانک جهانی (\lr{TFET}) یکی از مؤثرترین مکانیزم‌ها بود: ۱۸۰ میلیون دلار از ۱۱ کشور جمع‌آوری شد، با حسابرسی مستقل و گزارش‌دهی شفاف. نکته کلیدی: مالکیت تیموری‌ها بر اولویت‌گذاری هزینه‌ها.
\end{lessonlearned}

\sectiondivider

% ═══════════════════════════════════════════════════════════════════════════════
\section{شفافیت مالی و ضدفساد}
\label{sec:financial-transparency}
% ═══════════════════════════════════════════════════════════════════════════════

\subsection{اصول شفافیت مالی}
\label{subsec:transparency-principles}

\begin{keypoint}
\textbf{شش اصل شفافیت مالی:}
\begin{enumerate}[nosep]
    \item \textbf{انتشار عمومی}: بودجه، هزینه‌ها، و قراردادها به‌صورت آنلاین منتشر شوند
    \item \textbf{حسابرسی مستقل}: حداقل سالانه توسط شرکت بین‌المللی معتبر
    \item \textbf{دسترسی رسانه}: روزنامه‌نگاران حق پرسش و دسترسی به اسناد مالی را دارند
    \item \textbf{مکانیزم شکایت}: کانال امن برای گزارش فساد توسط کارکنان و شهروندان
    \item \textbf{استاندارد \lr{IATI}}: گزارش‌دهی مطابق ابتکار شفافیت کمک بین‌المللی
    \item \textbf{پاسخگویی دوسویه}: هم به حامیان مالی و هم به مردم ایران
\end{enumerate}
\end{keypoint}

\subsection{ساختار نظارت مالی}
\label{subsec:financial-oversight}

\begin{table}[htbp]
\centering
\caption{لایه‌های نظارت مالی}
\label{tab:financial-oversight}
\begin{tabularx}{\textwidth}{>{\raggedleft\arraybackslash}p{0.5cm}
                             >{\raggedleft\arraybackslash}p{3cm}
                             >{\raggedleft\arraybackslash}X
                             >{\raggedleft\arraybackslash}p{2.5cm}}
\toprule
\headerrow \# & لایه & وظیفه & فرکانس \\
\midrule
۱ & حسابرسی داخلی \lr{UNMOIT} & بررسی روزانه تراکنش‌ها & مستمر \\
\altrow ۲ & بازرسی داخلی \lr{UN} (\lr{OIOS}) & حسابرسی مستقل از مدیریت مأموریت & شش‌ماهه \\
۳ & حسابرسی خارجی & شرکت بین‌المللی (مثل \lr{Deloitte/PwC}) & سالانه \\
\altrow ۴ & هیئت امنای صندوق & نظارت بر تخصیص و مصرف & ماهانه \\
۵ & کمیته بودجه شورای امنیت & بررسی بودجه ارزیابی‌شده & سالانه \\
\altrow ۶ & نظارت عمومی (رسانه/مدنی) & افشاگری، پرسش، پیگیری & مستمر \\
\bottomrule
\end{tabularx}
\end{table}

\begin{warningbox}
\textbf{فساد در مأموریت‌های \lr{UN} واقعی است}:
\begin{itemize}[nosep]
    \item رسوایی «نفت در برابر غذا» عراق: ۱.۸ میلیارد دلار سوءاستفاده
    \item گزارش‌های فساد در \lr{UNMIK} کوزوو و \lr{MONUC} کنگو
    \item حیف‌ومیل در پروژه‌های بازسازی افغانستان
\end{itemize}
اعتماد مردم ایران — که سال‌ها شاهد فساد نهادهای حکومتی بوده‌اند — بسیار شکننده است. هر مورد فساد، حتی کوچک، می‌تواند کل مأموریت را بی‌اعتبار کند.
\end{warningbox}

\begin{recommendation}
\textbf{سیاست تحمل‌صفر با فساد:}
\begin{enumerate}[nosep]
    \item اخراج فوری کارکنان متخلف
    \item ارجاع پرونده‌ها به مراجع قضایی (بین‌المللی و ملی)
    \item جبران کامل خسارت از محل حقوق/ضمانت متخلف
    \item انتشار عمومی نتایج تحقیقات (بدون سانسور)
    \item مکانیزم حفاظت از افشاگران (\lr{Whistleblower Protection})
\end{enumerate}
\end{recommendation}

\sectiondivider

% ═══════════════════════════════════════════════════════════════════════════════
\section{تحلیل هزینه-فایده}
\label{sec:cost-benefit}
% ═══════════════════════════════════════════════════════════════════════════════

\subsection{هزینه عدم اقدام}
\label{subsec:cost-of-inaction}

\begin{table}[htbp]
\centering
\caption{برآورد هزینه عدم اقدام (سناریوی فروپاشی بدون نظارت)}
\label{tab:cost-of-inaction}
\begin{tabularx}{\textwidth}{>{\raggedleft\arraybackslash}p{3.5cm}
                             >{\centering\arraybackslash}p{2.5cm}
                             >{\raggedleft\arraybackslash}X}
\toprule
\headerrow حوزه & هزینه برآوردی (\$B) & توضیح \\
\midrule
بحران پناهندگی & ۲۰-۵۰ & ۵-۱۰M آواره، هزینه برای کشورهای همسایه و اروپا \\
\altrow بی‌ثباتی نفتی & ۵۰-۲۰۰ & افزایش ۳۰-۵۰٪ قیمت نفت جهانی \\
تروریسم & ۵-۲۰ & گسترش گروه‌های افراطی، عملیات ضدتروریسم \\
\altrow بحران هسته‌ای & غیرقابل‌محاسبه & ریسک اشاعه یا استفاده \\
از دست رفتن بازار & ۱۰-۳۰ & بازار ۸۵M نفری غیرقابل‌دسترس \\
\altrow هزینه بشردوستانه & ۵-۱۵ & کمک غذایی، پزشکی، سرپناه \\
\midrule
\textbf{مجموع} & \textbf{۹۰-۳۱۵} & بسیار بیشتر از بودجه نظارت \\
\bottomrule
\end{tabularx}
\end{table}

\begin{keypoint}
\textbf{نسبت هزینه-فایده}: بودجه ۲.۵-۵ میلیارد دلاری نظارت، در مقابل هزینه ۹۰-۳۱۵ میلیارد دلاری عدم اقدام. حتی اگر نظارت فقط ۵٪ احتمال فاجعه را کاهش دهد، بازگشت سرمایه ۱:۱-۱:۳ خواهد بود. در واقعیت، نظارت مؤثر می‌تواند احتمال موفقیت را ۳۰-۵۰٪ افزایش دهد.
\end{keypoint}

\subsection{فواید اقتصادی گذار موفق}
\label{subsec:benefits-of-success}

\begin{table}[htbp]
\centering
\caption{فواید اقتصادی گذار موفق ایران}
\label{tab:benefits}
\begin{tabularx}{\textwidth}{>{\raggedleft\arraybackslash}p{3.5cm}
                             >{\centering\arraybackslash}p{2.5cm}
                             >{\raggedleft\arraybackslash}X}
\toprule
\headerrow حوزه & ارزش برآوردی (\$B/سال) & توضیح \\
\midrule
رفع تحریم‌ها و صادرات نفت & ۶۰-۱۰۰ & بازگشت به ظرفیت ۴-۵ M بشکه/روز \\
\altrow جذب سرمایه‌گذاری خارجی & ۱۰-۳۰ & بازار بکر ۸۵M نفری \\
گردشگری & ۵-۱۵ & میراث فرهنگی عظیم \\
\altrow بازگشت سرمایه فراری & ۲۰-۵۰ & سرمایه ایرانیان خارج \\
رشد اقتصادی & ۵-۱۰٪/سال & اثر آزادسازی اقتصادی \\
\bottomrule
\end{tabularx}
\end{table}

\sectiondivider

% ═══════════════════════════════════════════════════════════════════════════════
\section{کنفرانس کمک‌دهندگان}
\label{sec:donors-conference}
% ═══════════════════════════════════════════════════════════════════════════════

\begin{casestudy}{کنفرانس‌های موفق: بن (افغانستان) و پاریس (لبنان)}
کنفرانس بن ۲۰۰۱ (افغانستان) و کنفرانس‌های پاریس (لبنان، \lr{CEDRE 2018}) نشان دادند که کنفرانس کمک‌دهندگان می‌تواند میلیاردها دلار تعهد جذب کند — \emph{به شرط آنکه}:
\begin{itemize}[nosep]
    \item طرح روشن و قابل‌اعتمادی وجود داشته باشد
    \item نمایندگان قانونی کشور میزبان حضور داشته باشند
    \item مکانیزم شفافیت و پاسخگویی تعریف شده باشد
    \item تعهدات مالی به برنامه‌های مشخص متصل باشند
\end{itemize}
\end{casestudy}

\begin{table}[htbp]
\centering
\caption{طرح پیشنهادی کنفرانس کمک‌دهندگان ایران}
\label{tab:donors-conference}
\begin{tabularx}{\textwidth}{>{\raggedleft\arraybackslash}p{3cm}
                             >{\raggedleft\arraybackslash}X}
\toprule
\headerrow عنصر & شرح \\
\midrule
زمان & ماه ۲-۳ پس از گذار \\
\altrow مکان & ژنو یا وین (بی‌طرف) \\
میزبان & دبیرکل \lr{UN} + دولت انتقالی ایران \\
\altrow مدعوین & ۵۰+ کشور، نهادهای مالی بین‌المللی، بخش خصوصی \\
هدف مالی & ۳-۵ میلیارد دلار تعهد اولیه \\
\altrow ساختار & ۱) جلسه عمومی + ۲) میزهای تخصصی + ۳) تعهدات \\
پیگیری & کنفرانس سالانه بازنگری \\
\bottomrule
\end{tabularx}
\end{table}

\begin{recommendation}
برای موفقیت کنفرانس:
\begin{enumerate}[nosep]
    \item ارائه طرح جامع با ارقام مشخص (همین سند می‌تواند مبنا باشد)
    \item حضور دولت انتقالی ایران با مشروعیت
    \item تعهد شفاف به حسابرسی و گزارش‌دهی
    \item پیوند تعهدات مالی به شاخص‌های پیشرفت
    \item مکانیزم پیگیری تعهدات (بسیاری از تعهدات کنفرانس‌ها محقق نمی‌شوند)
\end{enumerate}
\end{recommendation}

\sectiondivider

% ═══════════════════════════════════════════════════════════════════════════════
% جمع‌بندی فصل
% ═══════════════════════════════════════════════════════════════════════════════

\begin{chaptersummary}
یافته‌های کلیدی این فصل:

\begin{enumerate}
    \item \textbf{بودجه ۲.۵-۵ میلیارد دلار معقول و توجیه‌پذیر است}: در مقایسه با عراق (۶۰B+)، افغانستان (۱۴۵B+)، و هزینه عدم اقدام (۹۰-۳۱۵B)، این رقم سرمایه‌گذاری عاقلانه‌ای است.

    \item \textbf{هزینه سرانه پایین}: ۳۰-۶۰ دلار به ازای هر ایرانی در ۱۰ سال — کمتر از قیمت یک وعده غذا در رستوران نیویورک.

    \item \textbf{نیروی انسانی بزرگ‌ترین قلم بودجه (۳۶٪)}: کنترل هزینه‌های حقوقی و تعادل بین کارکنان بین‌المللی و ایرانی حیاتی است.

    \item \textbf{تنوع منابع مالی ضروری}: حداکثر ۳۰٪ از هر منبع واحد. ترکیب \lr{UN} + \lr{EU} + دوجانبه + دارایی‌های ایران + خصوصی.

    \item \textbf{صندوق امانی چندجانبه مکانیزم مرکزی}: با مدیریت مشترک ایرانی-بین‌المللی و حسابرسی مستقل.

    \item \textbf{شفافیت مالی شرط اعتماد}: شش اصل شفافیت، شش لایه نظارت، و سیاست تحمل‌صفر با فساد.

    \item \textbf{هزینه عدم اقدام بسیار بیشتر است}: بحران پناهندگی، بی‌ثباتی نفتی، تروریسم — مجموعاً ۹۰-۳۱۵ میلیارد دلار.

    \item \textbf{کنفرانس کمک‌دهندگان در ماه ۲-۳}: با طرح روشن، حضور دولت انتقالی، و مکانیزم پیگیری تعهدات.
\end{enumerate}

\vspace{0.5cm}
\textit{در فصل بعد (\ref{ch:roadmap})، نقشه راه اجرایی و توصیه‌های نهایی برای همه ذی‌نفعان ارائه خواهد شد.}
\end{chaptersummary}

\chapterend
% ═══════════════════════════════════════════════════════════════════════════════
% فصل ۱۱: نقشه راه اجرایی و توصیه‌های نهایی
% فایل: chapters/ch11-roadmap.tex
% رنگ فصل: بنفش (MainPurple)
% ═══════════════════════════════════════════════════════════════════════════════

\chapteropening{۱۱}{نقشه راه اجرایی و توصیه‌های نهایی}{MainPurple}{%
بهترین زمان برای کاشتن درخت بیست سال پیش بود. دومین بهترین زمان، همین الان است.%
}{ضرب‌المثل چینی}

\chapter{نقشه راه اجرایی و توصیه‌های نهایی}
\label{ch:roadmap}

\minitoc

% ─────────────────────────────────────────────────────────────────────────────
% خلاصه اجرایی
% ─────────────────────────────────────────────────────────────────────────────

\begin{executivesummary}
این فصل، عصاره عملیاتی تمام یافته‌های کتاب است. ده توصیه کلیدی به تفکیک مخاطب، شاخص‌های کمّی سنجش موفقیت، مکانیزم پایش پنج‌سطحی، استراتژی خروج تفصیلی، و جدول اقدامات فوری/میان‌مدت/بلندمدت ارائه می‌شود. هدف: هر خواننده — از رهبر اپوزیسیون تا دیپلمات سازمان ملل تا فعال مدنی ایرانی — پس از خواندن این فصل بداند \emph{دقیقاً چه باید بکند}. نقشه راه بصری در پایان فصل، تمام عناصر را در یک نمای واحد ترکیب می‌کند.
\end{executivesummary}

\section{درآمد: از تحلیل به اقدام}
\label{sec:roadmap-intro}

در فصول پیشین، مبانی نظری (فصل \ref{ch:theoretical})، ویژگی‌های ایران (فصل \ref{ch:why-iran})، مدل‌های نظارت (فصل \ref{ch:approaches})، سناریوها (فصل \ref{ch:scenarios})، بازیگران (فصل \ref{ch:actors})، تضمین‌ها (فصل \ref{ch:guarantees})، ریسک‌ها (فصل \ref{ch:risks})، نیازمندی‌ها (فصل \ref{ch:requirements})، زمان‌بندی (فصل \ref{ch:timeline})، و بودجه (فصل \ref{ch:budget}) بررسی شدند. اکنون همه این عناصر را در یک نقشه راه عملیاتی یکپارچه ترکیب می‌کنیم.

\begin{keypoint}
این فصل سه پرسش را پاسخ می‌دهد:
\begin{enumerate}[nosep]
    \item \textbf{چه باید کرد؟} — ده توصیه کلیدی
    \item \textbf{چگونه می‌فهمیم موفق شده‌ایم؟} — شاخص‌های سنجش
    \item \textbf{اگر مسیر منحرف شد چه؟} — مکانیزم اصلاح و خروج
\end{enumerate}
\end{keypoint}

\sectiondivider

% ═══════════════════════════════════════════════════════════════════════════════
\section{ده توصیه کلیدی}
\label{sec:ten-recommendations}
% ═══════════════════════════════════════════════════════════════════════════════

\subsection{توصیه ۱: مالکیت ملی — خط قرمز غیرقابل‌مذاکره}
\label{subsec:rec-national-ownership}

\begin{recommendation}
\textbf{مخاطب اصلی: همه بازیگران}

ایرانیان باید مالک فرایند گذار باشند. نظارت بین‌المللی ابزار تسهیل است، نه مدیریت. در هر سطح تصمیم‌گیری، ایرانیان باید حرف آخر را بزنند.

\textbf{اقدامات مشخص:}
\begin{itemize}[nosep]
    \item رئیس دولت انتقالی و اکثریت شورای مشورتی ایرانی باشند
    \item هر نهاد بین‌المللی معاون ایرانی داشته باشد
    \item تصمیمات استراتژیک (قانون اساسی، انتخابات) صرفاً توسط ایرانیان گرفته شود
    \item قاعده ۳۰-۵۰-۸۰ انتقال مسئولیت رعایت شود (\seeChapter{ch:timeline})
\end{itemize}
\end{recommendation}

\subsection{توصیه ۲: آماده‌باش — همین الان شروع کنید}
\label{subsec:rec-preparedness}

\begin{recommendation}
\textbf{مخاطب اصلی: سازمان ملل، اتحادیه اروپا، اپوزیسیون}

فاز ۰ (پیش‌گذار) باید \emph{الان} آغاز شود — نه وقتی رژیم سقوط کرده. تجربه تاریخی نشان می‌دهد غافلگیری = فاجعه.

\textbf{اقدامات مشخص:}
\begin{itemize}[nosep]
    \item \lr{DPPA} تیم برنامه‌ریزی ایران تشکیل دهد
    \item فهرست ۵-۱۰ نامزد \lr{SRSG} آماده شود
    \item پیش‌نویس قطعنامه شورای امنیت نوشته شود
    \item شبکه ناظران آماده‌باش فعال شود
    \item اپوزیسیون درباره ساختار دولت انتقالی توافق کند
\end{itemize}
\end{recommendation}

\subsection{توصیه ۳: مدل ترکیبی-تطبیقی — نه حداقل، نه حداکثر}
\label{subsec:rec-hybrid-model}

\begin{recommendation}
\textbf{مخاطب اصلی: شورای امنیت، SRSG}

مدل ۶ (ترکیبی-تطبیقی) ارائه‌شده در فصل \ref{ch:approaches} بهترین گزینه برای ایران است: ترکیب عناصر مدل‌های ۲ (مشورتی)، ۳ (ساختاری)، و ۴ (تضمینی) در سه فاز.

\textbf{اصول کلیدی:}
\begin{itemize}[nosep]
    \item فاز ۱: مدل ۴ غالب (نظارت تضمینی برای تثبیت)
    \item فاز ۲: مدل ۳ غالب (نظارت ساختاری برای نهادسازی)
    \item فاز ۳: مدل ۲ غالب (مشاوره برای تحکیم)
    \item مدل ۵ (مدیریت مستقیم) قاطعانه رد شود
    \item انطباق‌پذیری: مدل باید با تحولات میدانی تعدیل شود
\end{itemize}
\end{recommendation}

\subsection{توصیه ۴: فراگیری — هیچ‌کس نباید حذف شود}
\label{subsec:rec-inclusivity}

\begin{recommendation}
\textbf{مخاطب اصلی: دولت انتقالی، اپوزیسیون}

فراگیری شرط بقای گذار است. مصادره توسط یک جناح، قومیت، جنسیت، یا طبقه = شکست حتمی.

\textbf{معیارهای کمّی:}
\begin{itemize}[nosep]
    \item حداقل ۳۰٪ زنان در همه نهادها
    \item نمایندگی همه اقوام اصلی (آذری، کرد، بلوچ، عرب، ترکمن، لر) در شورا
    \item نمایندگی اقلیت‌های مذهبی (سنّی، بهایی، مسیحی، یهودی، زرتشتی)
    \item سهمیه جوانان (زیر ۳۵): حداقل ۲۰٪
    \item دیاسپورا: حداکثر ۲۰-۳۰٪ (نه بیشتر)
\end{itemize}
\end{recommendation}

\subsection{توصیه ۵: عدالت آشتی‌محور — نه انتقام، نه فراموشی}
\label{subsec:rec-justice}

\begin{recommendation}
\textbf{مخاطب اصلی: دولت انتقالی، جامعه مدنی}

\textbf{فرمول پیشنهادی:}
\begin{itemize}[nosep]
    \item \textbf{رهبران و آمران}: محاکمه در دادگاه ویژه (۱۰۰-۵۰۰ نفر)
    \item \textbf{مجریان رده‌میانی}: کمیسیون حقیقت + عفو مشروط در ازای اعتراف
    \item \textbf{اعضای عادی}: بازگشت به زندگی عادی بدون تعقیب
    \item \textbf{قربانیان}: جبران مادی + نمادین + حق شنیده شدن
    \item \textbf{ممنوعیت}: اعدام، شکنجه، مجازات جمعی
\end{itemize}
\end{recommendation}

\begin{lessonlearned}{آفریقای جنوبی در مقابل عراق}
آفریقای جنوبی با مدل آشتی (کمیسیون حقیقت + عفو مشروط) به ثبات رسید. عراق با مدل انتقامی (بعث‌زدایی افراطی) به جنگ داخلی کشیده شد. ایران باید مدل آفریقای جنوبی را الگو قرار دهد — نه عراق.
\end{lessonlearned}

\subsection{توصیه ۶: مدیریت سپاه — بزرگ‌ترین چالش}
\label{subsec:rec-irgc}

\begin{recommendation}
\textbf{مخاطب اصلی: دولت انتقالی، شورای امنیت}

سپاه پاسداران بزرگ‌ترین بازیگر مخرب بالقوه و بزرگ‌ترین چالش گذار است. استراتژی باید ترکیبی از فشار و انگیزه باشد:

\textbf{فشار:}
\begin{itemize}[nosep]
    \item ممنوعیت قانونی از فعالیت سیاسی و اقتصادی
    \item تهدید قاطع بین‌المللی به تحریم در صورت کودتا
    \item نظارت بین‌المللی بر جابجایی نیروها و سلاح
\end{itemize}

\textbf{انگیزه:}
\begin{itemize}[nosep]
    \item برنامه \lr{DDR} با بسته مالی مناسب برای افراد رده‌پایین
    \item ادغام نیروهای فنی در ارتش حرفه‌ای جدید
    \item عدم تعقیب اعضای عادی (فقط آمران جنایات)
    \item تبدیل شرکت‌های سپاه به شرکت‌های سهامی عام (نه مصادره)
\end{itemize}
\end{recommendation}

\subsection{توصیه ۷: اقتصاد — نان مقدم بر رأی}
\label{subsec:rec-economy}

\begin{recommendation}
\textbf{مخاطب اصلی: IMF، بانک جهانی، دولت انتقالی}

بدون ثبات اقتصادی، دموکراسی بقا نخواهد داشت. مردمی که نان ندارند، به صندوق رأی اعتماد نمی‌کنند.

\textbf{اقدامات فوری (ماه ۱-۶):}
\begin{itemize}[nosep]
    \item رفع تحریم‌های بین‌المللی (فوری و بدون شرط)
    \item آزادسازی دارایی‌های بلوکه‌شده
    \item بسته کمک بشردوستانه ($۲-۵B)
    \item تثبیت نرخ ارز با حمایت \lr{IMF}
\end{itemize}

\textbf{اقدامات میان‌مدت (ماه ۶-۲۴):}
\begin{itemize}[nosep]
    \item برنامه اصلاحات اقتصادی با مشاوره \lr{IMF/WB}
    \item جذب سرمایه‌گذاری خارجی
    \item شبکه ایمنی اجتماعی برای اقشار آسیب‌پذیر
\end{itemize}
\end{recommendation}

\subsection{توصیه ۸: رسانه آزاد — اکسیژن دموکراسی}
\label{subsec:rec-media}

\begin{recommendation}
\textbf{مخاطب اصلی: دولت انتقالی، رسانه‌ها}

\textbf{اقدامات:}
\begin{itemize}[nosep]
    \item رفع فیلترینگ اینترنت در روز اول
    \item آزادی فوری روزنامه‌نگاران زندانی
    \item صدور مجوز رسانه بدون سانسور
    \item کمیسیون مستقل تنظیم رسانه (نه دولتی)
    \item حمایت مالی از رسانه‌های مستقل (بدون کنترل محتوا)
    \item آموزش سواد رسانه‌ای عمومی
    \item مقابله فعال با اطلاعات نادرست (\seeChapter{ch:timeline})
\end{itemize}
\end{recommendation}

\subsection{توصیه ۹: هسته‌ای — حل مسئله، نه تشدید بحران}
\label{subsec:rec-nuclear}

\begin{recommendation}
\textbf{مخاطب اصلی: P5+1، آژانس بین‌المللی انرژی اتمی}

\textbf{اصول:}
\begin{itemize}[nosep]
    \item مسئله هسته‌ای نباید گروگان گذار شود
    \item توافق جامع جدید (جایگزین \lr{JCPOA}) در فاز ۲
    \item تضمین حق غنی‌سازی صلح‌آمیز
    \item نظارت تقویت‌شده آژانس (پروتکل الحاقی + دسترسی گسترده)
    \item رفع کامل تحریم‌های هسته‌ای در ازای شفافیت کامل
\end{itemize}
\end{recommendation}

\subsection{توصیه ۱۰: خروج شفاف — از ابتدا بدانید کی تمام می‌شود}
\label{subsec:rec-exit}

\begin{recommendation}
\textbf{مخاطب اصلی: شورای امنیت، SRSG}

\textbf{اصول خروج:}
\begin{itemize}[nosep]
    \item معیارهای خروج از روز اول اعلام شوند (شاخص‌محور نه زمان‌محور)
    \item خروج تدریجی (۵ مرحله مطابق فصل \ref{ch:timeline})
    \item انتقال واقعی مسئولیت (نه فقط کاهش حضور)
    \item ارزیابی مستقل قبل از هر مرحله خروج
    \item حفظ رابطه پس از خروج (دفتر سیاسی \lr{UN})
\end{itemize}
\end{recommendation}

\sectiondivider

% ═══════════════════════════════════════════════════════════════════════════════
\section{توصیه‌ها به تفکیک مخاطب}
\label{sec:recommendations-by-audience}
% ═══════════════════════════════════════════════════════════════════════════════

\begin{landscape}
\begin{table}[htbp]
\centering
\bigtablefontsize
\caption{ماتریس توصیه‌ها به تفکیک مخاطب و بازه زمانی}
\label{tab:recommendations-matrix}
\begin{tabularx}{\linewidth}{>{\raggedleft\arraybackslash}p{2.5cm}
                             >{\raggedleft\arraybackslash}X
                             >{\raggedleft\arraybackslash}X
                             >{\raggedleft\arraybackslash}X}
\toprule
\headerrow مخاطب & فوری (الان — ماه ۶) & میان‌مدت (ماه ۶-۲۴) & بلندمدت (ماه ۲۴-۶۰) \\
\midrule
اپوزیسیون ایرانی & توافق بر ساختار دولت انتقالی، ائتلاف فراگیر، منشور دموکراتیک & مشارکت در مجلس مؤسسان، پذیرش نتایج انتخابات & حزب‌سازی، نهادسازی، پذیرش چرخش قدرت \\
\altrow جامعه مدنی ایران & مستندسازی حقوق بشر، آموزش ناظران محلی & نظارت شهروندی، مشارکت در قانون‌نویسی & نهاد دیده‌بان، آموزش دموکراتیک \\
سازمان ملل & تشکیل تیم برنامه‌ریزی، شناسایی \lr{SRSG} & ایجاد \lr{UNMOIT}، استقرار ناظران & کاهش تدریجی، انتقال مسئولیت \\
\altrow اتحادیه اروپا & پیش‌نویس بسته حمایتی، آموزش ناظران & نظارت انتخاباتی، مشروط‌سازی کمک & مذاکرات تجاری، پیوند نهادی \\
آمریکا & رفع تحریم‌ها، آزادسازی دارایی‌ها & حمایت مالی و فنی، مذاکره هسته‌ای & عادی‌سازی روابط، سرمایه‌گذاری \\
\altrow کشورهای منطقه & عدم مداخله، حمایت از ثبات & مشارکت در گروه تماس، کمک مالی & همکاری امنیتی و اقتصادی منطقه‌ای \\
دیاسپورا & بسیج تخصص و سرمایه، پرهیز از جناح‌گرایی & مشارکت فنی، بازگشت تدریجی & ادغام، انتقال دانش \\
\altrow رسانه‌ها & پوشش مسئولانه، ضد اطلاعات نادرست & آموزش روزنامه‌نگاری حرفه‌ای & رسانه مستقل پایدار \\
\bottomrule
\end{tabularx}
\end{table}
\end{landscape}

\sectiondivider

% ═══════════════════════════════════════════════════════════════════════════════
\section{شاخص‌های سنجش موفقیت}
\label{sec:success-indicators}
% ═══════════════════════════════════════════════════════════════════════════════

\begin{table}[htbp]
\centering
\caption{نُه شاخص کمّی سنجش موفقیت گذار}
\label{tab:success-indicators}
\begin{tabularx}{\textwidth}{>{\centering\arraybackslash}p{0.5cm}
                             >{\raggedleft\arraybackslash}p{3cm}
                             >{\raggedleft\arraybackslash}X
                             >{\centering\arraybackslash}p{2cm}
                             >{\centering\arraybackslash}p{2cm}}
\toprule
\headerrow \# & شاخص & تعریف عملیاتی & هدف سال ۳ & هدف سال ۵ \\
\midrule
۱ & آزادی سیاسی & امتیاز \lr{Freedom House} (۱-۷) & ۳.۰ & ۲.۵ \\
\altrow ۲ & آزادی مطبوعات & رتبه \lr{RSF} (از ۱۸۰) & زیر ۱۰۰ & زیر ۷۰ \\
۳ & مشارکت انتخاباتی & درصد رأی‌دهندگان & ۶۰٪ & ۶۵٪ \\
\altrow ۴ & شاخص فساد & امتیاز \lr{TI CPI} (از ۱۰۰) & ۳۵ & ۴۵ \\
۵ & حقوق بشر & تعداد موارد نقض مستند/سال & کاهش ۷۰٪ & کاهش ۹۰٪ \\
\altrow ۶ & برابری جنسیتی & درصد زنان در پارلمان & ۲۵٪ & ۳۰٪ \\
۷ & رشد اقتصادی & نرخ رشد \lr{GDP} سالانه & ۳٪ & ۵٪ \\
\altrow ۸ & اعتماد عمومی & نظرسنجی اعتماد به نهادها & ۴۵٪ & ۵۵٪ \\
۹ & امنیت & تعداد حوادث امنیتی/ماه & کاهش ۸۰٪ & کاهش ۹۵٪ \\
\bottomrule
\end{tabularx}
\end{table}

\begin{warningbox}
\textbf{هشدار درباره شاخص‌ها}: شاخص‌های کمّی ابزار مفید اما ناقصی هستند. رتبه‌بندی‌های بین‌المللی ممکن است تعصب داشته باشند. شاخص‌ها باید با تحلیل کیفی (مصاحبه، مشاهده، تحلیل روایت) تکمیل شوند. «آنچه اندازه‌گیری می‌شود مدیریت می‌شود» — اما «آنچه فقط اندازه‌گیری می‌شود تحریف می‌شود».
\end{warningbox}

\begin{casestudy}{گرجستان: موفقیت قابل‌اندازه‌گیری}
گرجستان پس از انقلاب رز (۲۰۰۳) یکی از موفق‌ترین گذارهای پساشوروی بود. شاخص‌های کمّی:
\begin{itemize}[nosep]
    \item فساد (\lr{TI CPI}): از ۱.۸ (۲۰۰۳) به ۵.۲ (۲۰۱۲) — بهبود ۱۹۰٪
    \item آزادی اقتصادی: از رتبه ۱۰۱ به ۲۱ جهانی
    \item سهولت کسب‌وکار: از رتبه ۱۱۲ به ۸ جهانی
\end{itemize}
البته گرجستان در آزادی سیاسی عقب‌گرد داشت — نشان‌دهنده اهمیت نگاه جامع به همه شاخص‌ها.
\end{casestudy}

\sectiondivider

% ═══════════════════════════════════════════════════════════════════════════════
\section{مکانیزم پایش و اصلاح}
\label{sec:monitoring-mechanism}
% ═══════════════════════════════════════════════════════════════════════════════

\subsection{پنج سطح پایش}
\label{subsec:five-level-monitoring}

\begin{figure}[htbp]
\centering
\begin{tikzpicture}[
    every node/.style={font=\small, align=center},
    level/.style={rectangle, rounded corners, minimum width=10cm, minimum height=1cm, thick},
    arrow/.style={-{Stealth[length=2.5mm]}, thick, DarkGray}
]
    % Levels (bottom to top)
    \node[level, draw=MainGreen, fill=LightGreen] (l1) at (0,0) {\textbf{سطح ۱: پایش روزانه} — شاخص‌های هشدار زودهنگام | تیم پایش};
    \node[level, draw=MainYellow, fill=LightYellow] (l2) at (0,1.5) {\textbf{سطح ۲: گزارش ماهانه} — گزارش \lr{SRSG} به دبیرکل و شورای امنیت};
    \node[level, draw=MainOrange, fill=LightOrange] (l3) at (0,3) {\textbf{سطح ۳: بازنگری فصلی} — ارزیابی شاخص‌ها، تعدیل برنامه | تیم برنامه‌ریزی};
    \node[level, draw=MainRed, fill=LightRed] (l4) at (0,4.5) {\textbf{سطح ۴: ارزیابی مستقل} — تیم خارجی هر ۶ ماه | \lr{OIOS}};
    \node[level, draw=MainPurple, fill=LightPurple] (l5) at (0,6) {\textbf{سطح ۵: بازنگری استراتژیک} — شورای امنیت، سالانه | ادامه/تعدیل/خروج};
    
    % Arrows
    \draw[arrow] (l1) -- (l2);
    \draw[arrow] (l2) -- (l3);
    \draw[arrow] (l3) -- (l4);
    \draw[arrow] (l4) -- (l5);
    
    % Side label
    \node[font=\footnotesize, rotate=90, anchor=south] at (-6,3) {\textbf{افزایش سطح اختیار تصمیم‌گیری} →};
    
\end{tikzpicture}
\caption{پنج سطح پایش و اصلاح مسیر}
\label{fig:five-monitoring-levels}
\end{figure}

\subsection{پروتکل تصمیم‌گیری}
\label{subsec:decision-protocol}

\begin{table}[htbp]
\centering
\caption{پروتکل تصمیم‌گیری بر اساس نتایج ارزیابی}
\label{tab:decision-protocol}
\begin{tabularx}{\textwidth}{>{\centering\arraybackslash}p{2cm}
                             >{\raggedleft\arraybackslash}p{3cm}
                             >{\raggedleft\arraybackslash}X
                             >{\raggedleft\arraybackslash}p{2.5cm}}
\toprule
\headerrow وضعیت & شرایط & اقدام & تصمیم‌گیر \\
\midrule
\cellgreen{سبز} & ۷+ شاخص از ۹ در مسیر هدف & ادامه طبق برنامه، بهبود جزئی & \lr{SRSG} \\
\altrow \cellorange{زرد} & ۴-۶ شاخص در مسیر، بقیه تأخیر & تعدیل برنامه، تمرکز بر حوزه‌های ضعیف & \lr{SRSG} + دبیرکل \\
\cellorange{نارنجی} & ۲-۳ شاخص در مسیر، عقب‌گرد جزئی & بازنگری جدی، ممکن است تمدید فاز & شورای امنیت \\
\altrow \cellred{قرمز} & ۱ یا کمتر در مسیر، عقب‌گرد جدی & بازنگری اساسی استراتژی & شورای امنیت \\
\cellred{مشکی} & بحران حاد (کودتا، جنگ) & فعال‌سازی طرح اضطراری (\seeChapter{ch:risks}) & دبیرکل + \lr{SC} \\
\bottomrule
\end{tabularx}
\end{table}

\begin{lessonlearned}{موزامبیک: اصلاح مسیر موفق}
در موزامبیک، وقتی فرایند \lr{DDR} کندتر از برنامه پیش رفت، سازمان ملل به‌جای اصرار بر زمان‌بندی اولیه، فاز را ۶ ماه تمدید کرد. این انعطاف باعث شد فرایند با کیفیت بهتر تکمیل شود. \textbf{درس}: اصلاح مسیر نشانه ضعف نیست، نشانه هوشمندی است.
\end{lessonlearned}

\sectiondivider

% ═══════════════════════════════════════════════════════════════════════════════
\section{استراتژی خروج تفصیلی}
\label{sec:detailed-exit}
% ═══════════════════════════════════════════════════════════════════════════════

\subsection{شرایط خروج}
\label{subsec:exit-conditions}

\begin{table}[htbp]
\centering
\caption{معیارهای خروج به تفکیک مرحله}
\label{tab:exit-criteria}
\begin{tabularx}{\textwidth}{>{\raggedleft\arraybackslash}p{3cm}
                             >{\raggedleft\arraybackslash}X
                             >{\raggedleft\arraybackslash}p{3cm}}
\toprule
\headerrow مرحله خروج & شرط لازم & شاخص تأیید \\
\midrule
تبدیل به دفتر سیاسی (ماه ۶۰) & ۲ انتخابات آزاد + ۱ انتقال مسالمت‌آمیز قدرت & تأیید ناظران بین‌المللی \\
\altrow کاهش ۸۰٪ (ماه ۷۲) & استقلال قوه قضاییه + رسانه نسبتاً آزاد & \lr{Freedom House} بالای ۳.۰ \\
کاهش ۹۵٪ (ماه ۹۶) & نهادهای ملی عملکرد مستقل & ارزیابی مستقل مثبت \\
\altrow خروج کامل (ماه ۱۲۰) & همه ۹ شاخص در سطح هدف ۵ ساله & گزارش نهایی \\
\bottomrule
\end{tabularx}
\end{table}

\subsection{چرخه تصمیم خروج}
\label{subsec:exit-decision-cycle}

\begin{figure}[htbp]
\centering
\begin{tikzpicture}[
    node distance=2cm,
    every node/.style={font=\small, align=center},
    step/.style={rectangle, rounded corners, draw=MainPurple, fill=LightPurple, minimum width=3cm, minimum height=1cm, thick},
    decision/.style={diamond, draw=MainRed, fill=LightRed, minimum width=2cm, minimum height=1.5cm, thick, aspect=2},
    outcome/.style={rectangle, rounded corners, minimum width=2.5cm, minimum height=0.8cm, thick},
    arrow/.style={-{Stealth[length=2.5mm]}, thick}
]
    \node[step] (assess) {ارزیابی\\شاخص‌ها};
    \node[decision, right=2cm of assess] (decide) {آیا معیارها\\محقق شده؟};
    \node[outcome, above right=1cm and 2cm of decide, draw=MainGreen, fill=LightGreen] (yes) {خروج مرحله‌ای\\(گام بعدی)};
    \node[outcome, right=2.5cm of decide, draw=MainOrange, fill=LightOrange] (partial) {تمدید ۶-۱۲ ماه\\تعدیل برنامه};
    \node[outcome, below right=1cm and 2cm of decide, draw=MainRed, fill=LightRed] (no) {بازنگری اساسی\\ممکن: افزایش حضور};
    
    \draw[arrow] (assess) -- (decide);
    \draw[arrow, MainGreen] (decide) -- node[above, font=\footnotesize] {بله} (yes);
    \draw[arrow, MainOrange] (decide) -- node[above, font=\footnotesize] {تا حدی} (partial);
    \draw[arrow, MainRed] (decide) -- node[below, font=\footnotesize] {خیر} (no);
    
    % Feedback
    \draw[arrow, dashed, DarkGray] (partial.south) -- ++(0,-1) -| (assess.south);
    \draw[arrow, dashed, DarkGray] (no.west) -- ++(-1,0) |- (assess.south east);
    
\end{tikzpicture}
\caption{چرخه تصمیم‌گیری خروج}
\label{fig:exit-decision-cycle}
\end{figure}

\begin{warningbox}
\textbf{دام «خروج سیاسی»}: فشار داخلی در کشورهای حامی (آمریکا، اروپا) ممکن است خروج زودهنگام را دیکته کند — مثل افغانستان ۲۰۲۱. خروج باید بر اساس شاخص‌های میدانی باشد، نه انتخابات کشورهای حامی. مکانیزم دفاع: تعهدات حقوقی چندساله + نقش \lr{SC} (نه یک کشور) در تصمیم خروج.
\end{warningbox}

\sectiondivider

% ═══════════════════════════════════════════════════════════════════════════════
\section{جدول اقدامات به تفکیک بازه زمانی}
\label{sec:action-table}
% ═══════════════════════════════════════════════════════════════════════════════

\subsection{اقدامات فوری (الان تا ماه ۶)}
\label{subsec:immediate-actions}

\begin{table}[htbp]
\centering
\caption{اقدامات فوری — اولویت حیاتی}
\label{tab:immediate-actions}
\begin{tabularx}{\textwidth}{>{\centering\arraybackslash}p{0.5cm}
                             >{\raggedleft\arraybackslash}p{3.5cm}
                             >{\raggedleft\arraybackslash}X
                             >{\raggedleft\arraybackslash}p{2.5cm}}
\toprule
\headerrow \# & اقدام & شرح & مسئول \\
\midrule
۱ & تشکیل تیم آماده‌باش \lr{UN} & ۵۰-۱۰۰ نفر، برنامه‌ریزی سناریوها & \lr{DPPA} \\
\altrow ۲ & ائتلاف اپوزیسیون & منشور دموکراتیک، ساختار دولت انتقالی & اپوزیسیون \\
۳ & پیش‌نویس قطعنامه & نسخه‌های مختلف برای سناریوهای مختلف & \lr{P3+} \\
\altrow ۴ & شبکه ناظران & فهرست آماده‌باش ۵,۰۰۰+ ناظر & \lr{EU/OSCE/Carter} \\
۵ & آموزش مدنی & آموزش ناظران داخلی، مستندسازی & جامعه مدنی \\
\altrow ۶ & نقشه ظرفیت & شناسایی ایرانیان متخصص (داخل+دیاسپورا) & \lr{UNDP} \\
۷ & پیش‌موقعیت‌یابی & ذخیره تجهیزات در کشورهای همسایه & \lr{DPKO} \\
\altrow ۸ & گفتگوی ملی & آغاز گفتگوی غیررسمی بین همه گروه‌ها & میانجیان بین‌المللی \\
\bottomrule
\end{tabularx}
\end{table}

\subsection{اقدامات میان‌مدت (ماه ۶-۲۴)}
\label{subsec:medium-term-actions}

\begin{table}[htbp]
\centering
\caption{اقدامات میان‌مدت — ساختارسازی}
\label{tab:medium-actions}
\begin{tabularx}{\textwidth}{>{\centering\arraybackslash}p{0.5cm}
                             >{\raggedleft\arraybackslash}p{3.5cm}
                             >{\raggedleft\arraybackslash}X
                             >{\raggedleft\arraybackslash}p{2.5cm}}
\toprule
\headerrow \# & اقدام & شرح & مسئول \\
\midrule
۱ & قانون اساسی & مشاوره → مجلس مؤسسان → پیش‌نویس → رفراندوم & مجلس مؤسسان \\
\altrow ۲ & انتخابات & مجلس مؤسسان + رفراندوم + پارلمان & کمیسیون انتخابات \\
۳ & اصلاح سپاه & \lr{DDR}، ادغام، نظارت & معاون امنیتی \\
\altrow ۴ & کمیسیون حقیقت & تأسیس، شروع جلسات استماع & مجلس موقت \\
۵ & اصلاحات اقتصادی & رفع تحریم، صندوق امانی، جذب سرمایه & \lr{IMF/WB} \\
\altrow ۶ & نهادسازی & قوه قضاییه مستقل، کمیسیون رسانه، ضدفساد & دولت انتقالی \\
\bottomrule
\end{tabularx}
\end{table}

\subsection{اقدامات بلندمدت (ماه ۲۴-۶۰)}
\label{subsec:long-term-actions}

\begin{table}[htbp]
\centering
\caption{اقدامات بلندمدت — تحکیم}
\label{tab:long-term-actions}
\begin{tabularx}{\textwidth}{>{\centering\arraybackslash}p{0.5cm}
                             >{\raggedleft\arraybackslash}p{3.5cm}
                             >{\raggedleft\arraybackslash}X
                             >{\raggedleft\arraybackslash}p{2.5cm}}
\toprule
\headerrow \# & اقدام & شرح & مسئول \\
\midrule
۱ & تحکیم دموکراسی & انتخابات دوم، انتقال مسالمت‌آمیز قدرت & دولت منتخب \\
\altrow ۲ & انتقال مسئولیت & ۸۰→۱۰۰٪ مدیریت ایرانی & \lr{SRSG} \\
۳ & عدالت انتقالی & تکمیل کار کمیسیون، دادگاه ویژه & نهادهای قضایی \\
\altrow ۴ & توسعه اقتصادی & خصوصی‌سازی نظارت‌شده، رشد پایدار & دولت منتخب \\
۵ & الحاق به نهادهای بین‌المللی & \lr{WTO}، شورای حقوق بشر، و... & وزارت خارجه \\
\altrow ۶ & آماده‌سازی خروج & ارزیابی، تبدیل مأموریت، خروج تدریجی & \lr{SC/SRSG} \\
\bottomrule
\end{tabularx}
\end{table}

\sectiondivider

% ═══════════════════════════════════════════════════════════════════════════════
\section{نقشه راه بصری یکپارچه}
\label{sec:visual-roadmap}
% ═══════════════════════════════════════════════════════════════════════════════

\begin{figure}[htbp]
\centering
\begin{tikzpicture}[
    font=\footnotesize,
    milestone/.style={circle, draw=#1, fill=#1!20, minimum size=0.6cm, thick},
    phase/.style={rectangle, rounded corners, draw=#1, fill=#1!10, minimum width=2cm, minimum height=0.5cm},
]
    % Timeline
    \draw[very thick, DarkGray, -{Stealth}] (0,0) -- (16,0);
    
    % Phase backgrounds
    \fill[MediumGray!15] (0,-2) rectangle (2,2);
    \fill[MainRed!10] (2,-2) rectangle (5,2);
    \fill[MainOrange!10] (5,-2) rectangle (9,2);
    \fill[MainGreen!10] (9,-2) rectangle (13,2);
    \fill[MainBlue!10] (13,-2) rectangle (16,2);
    
    % Phase labels
    \node[font=\footnotesize\bfseries, MediumGray] at (1,2.3) {فاز ۰};
    \node[font=\footnotesize\bfseries, MainRed] at (3.5,2.3) {فاز ۱};
    \node[font=\footnotesize\bfseries, MainOrange] at (7,2.3) {فاز ۲};
    \node[font=\footnotesize\bfseries, MainGreen] at (11,2.3) {فاز ۳};
    \node[font=\footnotesize\bfseries, MainBlue] at (14.5,2.3) {فاز ۴};
    
    % Time markers
    \foreach \x/\t in {0/الان, 2/ماه ۰, 3.5/ماه ۶, 5/ماه ۱۲, 7/ماه ۲۴, 9/ماه ۳۶, 11/ماه ۴۸, 13/ماه ۶۰, 16/ماه ۱۲۰} {
        \draw[DarkGray] (\x,0.1) -- (\x,-0.1);
        \node[anchor=north, font=\tiny] at (\x,-0.2) {\t};
    }
    
    % Milestones (above line)
    \node[milestone=MainRed] (m1) at (2,0.8) {};
    \node[anchor=south, font=\tiny, text width=1.5cm] at (m1.north) {تیم ارزیابی\\قطعنامه};
    
    \node[milestone=MainRed] (m2) at (3.5,0.8) {};
    \node[anchor=south, font=\tiny, text width=1.5cm] at (m2.north) {UNMOIT\\فعال};
    
    \node[milestone=MainOrange] (m3) at (5,0.8) {};
    \node[anchor=south, font=\tiny, text width=1.5cm] at (m3.north) {مجلس\\مؤسسان};
    
    \node[milestone=MainOrange] (m4) at (7,0.8) {};
    \node[anchor=south, font=\tiny, text width=1.5cm] at (m4.north) {رفراندوم\\قانون اساسی};
    
    \node[milestone=MainGreen] (m5) at (9,0.8) {};
    \node[anchor=south, font=\tiny, text width=1.5cm] at (m5.north) {انتقال\\قدرت};
    
    \node[milestone=MainGreen] (m6) at (11,0.8) {};
    \node[anchor=south, font=\tiny, text width=1.5cm] at (m6.north) {انتخابات\\دوم};
    
    \node[milestone=MainBlue] (m7) at (13,0.8) {};
    \node[anchor=south, font=\tiny, text width=1.5cm] at (m7.north) {تبدیل\\مأموریت};
    
    \node[milestone=MainPurple] (m8) at (16,0.8) {};
    \node[anchor=south, font=\tiny, text width=1.5cm] at (m8.north) {خروج\\کامل};
    
    % Activities (below line)
    \node[anchor=north, font=\tiny, text width=1.8cm] at (1,-0.5) {آماده‌باش\\برنامه‌ریزی};
    \node[anchor=north, font=\tiny, text width=1.8cm] at (3,-0.5) {تثبیت\\امنیت};
    \node[anchor=north, font=\tiny, text width=1.8cm] at (6,-0.5) {نهادسازی\\قانون‌نویسی};
    \node[anchor=north, font=\tiny, text width=1.8cm] at (8,-0.5) {انتخابات\\پارلمان};
    \node[anchor=north, font=\tiny, text width=1.8cm] at (10,-0.5) {تحکیم\\انتقال};
    \node[anchor=north, font=\tiny, text width=1.8cm] at (14,-0.5) {مشاوره\\ارزیابی};
    
    % Personnel curve
    \draw[MainPurple, very thick, dashed] plot[smooth] coordinates {
        (0,0) (1,0.2) (2,0.5) (3,1) (3.5,1.2) (5,1.5) (6,1.7) (7,1.8)
        (8,1.5) (9,1.2) (10,1) (11,0.7) (13,0.3) (16,0.05)
    };
    \node[MainPurple, font=\tiny, anchor=west] at (7.5,1.9) {حضور بین‌المللی};
    
\end{tikzpicture}
\caption{نقشه راه بصری یکپارچه — از آماده‌باش تا خروج}
\label{fig:visual-roadmap}
\end{figure}

\sectiondivider

% ═══════════════════════════════════════════════════════════════════════════════
\section{پیام نهایی به هر مخاطب}
\label{sec:final-message}
% ═══════════════════════════════════════════════════════════════════════════════

\begin{keypoint}
\textbf{به اپوزیسیون ایرانی}: وحدت یابید. اختلافات امروز، ضعف فردا خواهد بود. بر اصول توافق کنید، جزئیات را به دموکراسی بسپارید.

\textbf{به جامعه مدنی ایران}: شما ستون فقرات گذارید. آماده شوید. مستند کنید. شبکه بسازید. مهارت بیاموزید.

\textbf{به سازمان ملل}: غافلگیر نشوید. تیم بسازید. برنامه بریزید. \lr{SRSG} را شناسایی کنید. الان.

\textbf{به اتحادیه اروپا}: بزرگ‌ترین حامی صلح جهان باشید — نه فقط ناظر.

\textbf{به آمریکا}: تحریم‌ها را ابزار فشار بدانید نه هدف. آماده رفع سریع باشید.

\textbf{به کشورهای منطقه}: ثبات ایران به نفع شماست. مداخله نکنید. حمایت کنید.

\textbf{به دیاسپورا}: تخصص‌تان ارزشمند است. با فروتنی بیایید. گوش دهید قبل از حرف زدن.

\textbf{به نیروهای سپاه}: جایی برای شما در آینده هست — اگر به مردم شلیک نکنید.
\end{keypoint}

\sectiondivider

% ═══════════════════════════════════════════════════════════════════════════════
% جمع‌بندی فصل
% ═══════════════════════════════════════════════════════════════════════════════

\begin{chaptersummary}
یافته‌های کلیدی این فصل:

\begin{enumerate}
    \item \textbf{ده توصیه کلیدی}: مالکیت ملی، آماده‌باش، مدل ترکیبی، فراگیری، عدالت آشتی‌محور، مدیریت سپاه، اقتصاد، رسانه آزاد، حل مسئله هسته‌ای، خروج شفاف.
    
    \item \textbf{نُه شاخص کمّی}: آزادی سیاسی، آزادی مطبوعات، مشارکت انتخاباتی، فساد، حقوق بشر، برابری جنسیتی، رشد اقتصادی، اعتماد عمومی، امنیت — هر کدام با هدف عددی برای سال ۳ و ۵.
    
    \item \textbf{پایش پنج‌سطحی}: از روزانه تا سالانه، با پروتکل تصمیم‌گیری سبز/زرد/نارنجی/قرمز/مشکی.
    
    \item \textbf{خروج شاخص‌محور}: معیارهای مشخص برای هر مرحله خروج، نه تقویم سیاسی.
    
    \item \textbf{اقدامات به سه بازه}: فوری (الان-۶ ماه)، میان‌مدت (۶-۲۴)، بلندمدت (۲۴-۶۰).
    
    \item \textbf{توصیه‌ها به تفکیک مخاطب}: هر بازیگر — از اپوزیسیون تا سازمان ملل — می‌داند دقیقاً چه باید بکند.
    
    \item \textbf{زمان اقدام الان است}: بهترین زمان برای آمادگی، قبل از بحران است.
\end{enumerate}

\vspace{0.5cm}
\textit{در فصل پایانی (\ref{ch:conclusion})، جمع‌بندی کل کتاب و کلام آخر نویسنده ارائه خواهد شد.}
\end{chaptersummary}

\chapterend
% ═══════════════════════════════════════════════════════════════════════════════
% فصل ۱۲: جمع‌بندی و کلام آخر
% فایل: chapters/ch12-conclusion.tex
% رنگ فصل: بنفش (MainPurple)
% ═══════════════════════════════════════════════════════════════════════════════

\chapteropening{۱۲}{جمع‌بندی و کلام آخر}{MainPurple}{%
تاریخ جهان چیزی نیست جز پیشرفت آگاهی از آزادی.%
}{گئورگ ویلهلم فریدریش هگل}

\chapter{جمع‌بندی و کلام آخر}
\label{ch:conclusion}

\minitoc

% ─────────────────────────────────────────────────────────────────────────────
% خلاصه اجرایی
% ─────────────────────────────────────────────────────────────────────────────

\begin{executivesummary}
این فصل پایانی، عصاره یافته‌های ۱۱ فصل پیشین را در قالبی فشرده مرور می‌کند، پنج یافته کلان را برجسته می‌سازد، هزینه عدم اقدام را یادآور می‌شود، و با دعوتی صمیمانه به اقدام و نقل‌قولی الهام‌بخش به پایان می‌رسد. پیام محوری ساده است: \emph{گذار دموکراتیک ایران ممکن است، مشروط به آمادگی، هوشمندی، و اراده جمعی}.
\end{executivesummary}

% ═══════════════════════════════════════════════════════════════════════════════
\section{مرور فشرده یافته‌ها}
\label{sec:summary-review}
% ═══════════════════════════════════════════════════════════════════════════════

\begin{table}[htbp]
\centering
\caption{خلاصه یافته‌های هر فصل در یک نگاه}
\label{tab:chapter-findings}
\begin{tabularx}{\textwidth}{>{\centering\arraybackslash}p{0.8cm}
                             >{\raggedleft\arraybackslash}p{3cm}
                             >{\raggedleft\arraybackslash}X}
\toprule
\headerrow فصل & عنوان & یافته محوری \\
\midrule
۱ & مبانی نظری & گذار فرایندی سه‌مرحله‌ای (آزادسازی→گذار→تحکیم) است؛ نظارت بین‌المللی طیفی از حداقلی تا حداکثری دارد؛ حاکمیت دیگر مطلق نیست. \\
\altrow ۲ & چرا ایران؟ & ایران با ۱۰ ویژگی منحصربه‌فرد (هسته‌ای، سپاه، تنوع قومی، دوگانگی قدرت) پیچیده‌ترین آزمون گذار معاصر خواهد بود. \\
۳ & رویکردها و ساختارها & از ۶ مدل نظارت، مدل ترکیبی-تطبیقی (مدل ۶) بهترین گزینه است: ترکیب سه‌فازی مدل‌های ۲، ۳، و ۴ با ۷ اصل بنیادین. \\
\altrow ۴ & سناریوها & از ۶ سناریوی گذار، مذاکره‌ای (B) مطلوب‌ترین و بحران ممتد (F) محتمل‌ترین است. برای هر سناریو، مدل نظارتی متفاوتی لازم است. \\
۵ & نهادها و بازیگران & ۹ دسته بازیگر باید هماهنگ عمل کنند؛ هیچ نهاد واحدی به‌تنهایی کافی نیست. ایرانیان (داخل و دیاسپورا) عامل اصلی‌اند. \\
\altrow ۶ & تضمین‌ها & ۶ حوزه تضمین (سیاسی، امنیتی، حقوقی، اقتصادی، اجتماعی، نهادی) باید همزمان فعال شوند. \\
۷ & ریسک‌ها & بازگشت اقتدارگرایی (سپاه) جدی‌ترین تهدید است. ۶ دسته ریسک شناسایی شد با ماتریس پاسخ و نظام هشدار زودهنگام. \\
\altrow ۸ & نیازمندی‌ها & ۶-۱۲K بین‌المللی + ۲۰-۵۰K ایرانی، ۸ نهاد کلیدی، زیرساخت فنی، و چارچوب حقوقی سه‌لایه ضروری است. \\
۹ & زمان‌بندی & ۵ فاز از پیش‌گذار تا خروج (۱۰ سال)، با تأکید بر آمادگی قبلی و ۷۲ ساعت سرنوشت‌ساز. قاعده ۳۰-۵۰-۸۰ انتقال مسئولیت. \\
\altrow ۱۰ & بودجه & \$۲.۵-۵B در ۱۰ سال — معقول در مقایسه با هزینه عدم اقدام (\$۹۰-۳۱۵B). تنوع منابع و شفافیت حیاتی. \\
۱۱ & نقشه راه & ۱۰ توصیه کلیدی، ۹ شاخص کمّی، پایش ۵‌سطحی، خروج شاخص‌محور. «بهترین زمان اقدام، الان.» \\
\bottomrule
\end{tabularx}
\end{table}

\sectiondivider

% ═══════════════════════════════════════════════════════════════════════════════
\section{پنج یافته کلان}
\label{sec:five-findings}
% ═══════════════════════════════════════════════════════════════════════════════

\subsection{یافته ۱: ایران منحصربه‌فرد اما نه استثنا}
\label{subsec:finding-unique}

ایران با ترکیبی بی‌سابقه از ویژگی‌ها (تئوکراسی + جمهوری، سپاه اقتصادی-نظامی، برنامه هسته‌ای، تنوع قومی، دیاسپورای بزرگ، جمعیت جوان باسواد) پیچیده‌ترین آزمون گذار دموکراتیک معاصر خواهد بود. اما «پیچیده» به معنای «ناممکن» نیست. هر کشوری که گذار موفق داشته، در زمان خود «استثنا» به نظر می‌رسید:

\begin{itemize}[nosep]
    \item آفریقای جنوبی: آپارتاید ۵۰ ساله + سلاح هسته‌ای + تنوع نژادی
    \item لهستان: ۴۵ سال کمونیسم + ارتش سرخ در مرز
    \item اسپانیا: ۳۶ سال دیکتاتوری فرانکو + ارتش قدرتمند
    \item تونس: اولین دموکراسی عربی؟ «غیرممکن» بود — تا شد.
\end{itemize}

\subsection{یافته ۲: مدل ترکیبی-تطبیقی بهترین گزینه است}
\label{subsec:finding-hybrid}

\begin{keypoint}
نه نظارت حداقلی (مدل ۱-۲) کافی است و نه مدیریت حداکثری (مدل ۵) مطلوب. مدل ۶ (ترکیبی-تطبیقی) با سه فاز و هفت اصل، تعادل بهینه میان حمایت بین‌المللی و مالکیت ملی را فراهم می‌کند. این مدل، نوآوری اصلی این کتاب است.
\end{keypoint}

\subsection{یافته ۳: مالکیت ملی غیرقابل‌مذاکره است}
\label{subsec:finding-ownership}

هیچ مدل نظارتی — هرقدر هم هوشمندانه طراحی شده باشد — بدون مالکیت و رهبری ایرانیان موفق نخواهد بود. تجربه عراق (مدیریت خارجی) و افغانستان (وابستگی نهادی) نشان داد که دموکراسی وارداتی پایدار نیست. ایرانیان باید مالک فرایند باشند؛ جامعه بین‌المللی، تسهیل‌گر.

\subsection{یافته ۴: آمادگی مساوی موفقیت}
\label{subsec:finding-preparedness}

\begin{warningbox}
تقریباً همه شکست‌های گذار یک ویژگی مشترک دارند: \textbf{غافلگیری}. فروپاشی شوروی (۱۹۹۱)، بهار عربی (۲۰۱۱)، سقوط کابل (۲۰۲۱) — در هیچ‌کدام جامعه بین‌المللی آماده نبود. و تقریباً همه موفقیت‌ها یک ویژگی مشترک دارند: \textbf{آمادگی قبلی}. آفریقای جنوبی (سال‌ها مذاکره)، لهستان (میزگرد)، تیمور شرقی (برنامه‌ریزی ماه‌ها قبل).

\textbf{این کتاب همین آمادگی را فراهم می‌کند.}
\end{warningbox}

\subsection{یافته ۵: هزینه عدم اقدام بسیار بیشتر از هزینه اقدام}
\label{subsec:finding-cost}

\begin{table}[htbp]
\centering
\caption{مقایسه نهایی: هزینه اقدام در مقابل عدم اقدام}
\label{tab:action-vs-inaction}
\begin{tabularx}{\textwidth}{>{\raggedleft\arraybackslash}p{4cm}
                             >{\centering\arraybackslash}p{3cm}
                             >{\centering\arraybackslash}p{3cm}}
\toprule
\headerrow & اقدام (نظارت مؤثر) & عدم اقدام (فروپاشی بدون نظارت) \\
\midrule
هزینه مالی & \$۲.۵-۵B & \$۹۰-۳۱۵B \\
\altrow هزینه انسانی & حداقل & هزاران تا صدها هزار کشته \\
آوارگان & حداقل & ۵-۱۰ میلیون \\
\altrow ریسک هسته‌ای & مدیریت‌شده & غیرقابل‌پیش‌بینی \\
ثبات منطقه‌ای & تقویت & فروپاشی \\
\altrow بازار جهانی انرژی & تثبیت & بحران \\
\bottomrule
\end{tabularx}
\end{table}

\sectiondivider

% ═══════════════════════════════════════════════════════════════════════════════
\section{دعوت به اقدام}
\label{sec:call-to-action}
% ═══════════════════════════════════════════════════════════════════════════════

این کتاب با یک پرسش آغاز شد: \emph{آیا می‌توان گذار دموکراتیک ایران را به‌گونه‌ای مدیریت کرد که به فاجعه‌ای مشابه عراق، لیبی، یا سوریه ختم نشود؟}

پاسخ ما: \textbf{بله — مشروط به آمادگی، هوشمندی، فراگیری، و اراده جمعی.}

ابزارها موجودند. تجربه تاریخی آموزنده است. مدل‌ها طراحی شده‌اند. نیازمندی‌ها شناسایی شده‌اند. ریسک‌ها تحلیل شده‌اند. بودجه برآورد شده. نقشه راه ترسیم شده.

آنچه باقی می‌ماند، \emph{اراده} است.

\begin{lessonlearned}{نلسون ماندلا: اراده تغییر}
ماندلا ۲۷ سال در زندان بود. وقتی آزاد شد، می‌توانست انتقام بگیرد — ارتش و پلیس سفیدپوست را منحل کند، مصادره اموال کند، محاکمه‌های نمایشی برگزار کند. به‌جای آن، دست آشتی دراز کرد. پیراهن تیم راگبی آفریقای جنوبی (نماد آپارتاید) را پوشید. گفت: «شجاعت یعنی نبود ترس نیست؛ یعنی غلبه بر ترس.» و کشورش را از لبه پرتگاه نجات داد.
\end{lessonlearned}

ایران نیز لبه پرتگاه ایستاده است. مسیر جنگ داخلی، تجزیه، یا اقتدارگرایی جدید، مسیری ساده و شناخته‌شده است — نیازی به برنامه‌ریزی ندارد. مسیر دموکراسی، مسیر دشوار اما ممکن است.

این کتاب نقشه آن مسیر دشوار است.

\begin{recommendation}
\textbf{فراخوان اقدام فوری:}
\begin{enumerate}[nosep]
    \item این سند را با سیاست‌گذاران، فعالان، و نهادهای بین‌المللی به اشتراک بگذارید
    \item درباره محتوای آن بحث، نقد، و بازنگری کنید
    \item اقدامات فوری فصل ۱۱ را آغاز کنید — \emph{الان}
    \item شبکه‌سازی برای آمادگی پیش‌گذار را شروع کنید
    \item «بهترین زمان کاشتن درخت بیست سال پیش بود. دومین بهترین زمان، همین الان است.»
\end{enumerate}
\end{recommendation}

\sectiondivider

% ═══════════════════════════════════════════════════════════════════════════════
\section{کلام آخر}
\label{sec:final-word}
% ═══════════════════════════════════════════════════════════════════════════════

نگارش این کتاب با یک باور عمیق آغاز شد: مردم ایران شایسته آزادی، عدالت، و حکومتی هستند که پاسخگوی آنان باشد. این باور نه آرمان‌گرایانه است و نه ساده‌انگارانه. ریسک‌ها واقعی‌اند. چالش‌ها عظیم‌اند. شکست ممکن است. اما تسلیم شدن به وضع موجود — که خود بزرگ‌ترین ریسک است — پذیرفتنی نیست.

تاریخ نشان داده که دموکراسی‌ها در شرایطی ساخته شده‌اند که بسیاری آن‌ها را «ناممکن» می‌خواندند. آلمان از خاکستر جنگ جهانی دوم. ژاپن از ویرانه هیروشیما. لهستان از زیر سایه ارتش سرخ. آفریقای جنوبی از جهنم آپارتاید.

ایران نیز خواهد توانست — اگر آماده باشیم.

\vspace{1cm}

\begin{center}
\begin{tikzpicture}
    \node[text width=10cm, align=center, font=\large\itshape, text=MainPurple] {
        «تنها چیزی که برای پیروزی شر لازم است،\\[0.3cm]
        این است که انسان‌های خوب هیچ نکنند.»\\[0.5cm]
    };
    \node[font=\normalsize, text=DarkGray] at (0,-1.5) {— ادموند بِرک};
\end{tikzpicture}
\end{center}

\vspace{0.5cm}

\begin{flushright}
\textbf{مهدی سالم}\\
تابستان ۲۰۲۵
\end{flushright}

\chapterend

% ---- پیوست‌ها ----
\appendix
%══════════════════════════════════════════════════════════════
% پیوست الف: مقایسه جامع ۹ نمونه تاریخی گذار
% فایل: appendices/app-a-comparison.tex
% رنگ: خاکستری/چندرنگ (پیوست)
% حجم هدف: ۲۰-۲۵ صفحه
%══════════════════════════════════════════════════════════════

\chapter{مقایسه جامع نُه نمونهٔ تاریخی گذار دموکراتیک}
\label{app:comparison}

%-----------------------------------------------------------
\begin{executivesummary}
این پیوست، جامع‌ترین بخش مقایسه‌ای کتاب حاضر است و نُه نمونهٔ تاریخی گذار دموکراتیک را در قالب جداول افقی گسترده بررسی می‌کند. کشورهای مورد مطالعه عبارت‌اند از: \textbf{اسپانیا} (۱۹۷۵-۱۹۸۲)، \textbf{لهستان} (۱۹۸۹-۱۹۹۱)، \textbf{شیلی} (۱۹۸۸-۱۹۹۰)، \textbf{آفریقای جنوبی} (۱۹۹۰-۱۹۹۴)، \textbf{اندونزی} (۱۹۹۸-۲۰۰۴)، \textbf{تیمور شرقی} (۱۹۹۹-۲۰۰۲)، \textbf{عراق} (۲۰۰۳-۲۰۱۰)، \textbf{تونس} (۲۰۱۱-۲۰۱۴) و \textbf{میانمار} (۲۰۱۰-۲۰۲۱). این نمونه‌ها در ۱۰ بُعد تحلیلی و بیش از ۶۰ شاخص مقایسه شده‌اند تا \textbf{درس‌آموخته‌های قابل‌انتقال} به سناریوی گذار ایران استخراج شود.
\end{executivesummary}

%-----------------------------------------------------------
\section{مقدمه و روش‌شناسی مقایسه}
\label{app:a:intro}

\subsection{معیارهای انتخاب نمونه‌ها}

انتخاب نُه نمونهٔ تاریخی بر اساس پنج معیار صورت گرفته است:

\begin{enumerate}[nosep]
\item \textbf{تنوع جغرافیایی:} اروپا (اسپانیا، لهستان)، آمریکای لاتین (شیلی)، آفریقا (آفریقای جنوبی)، آسیای جنوب‌شرقی (اندونزی، تیمور شرقی، میانمار)، خاورمیانه/شمال آفریقا (عراق، تونس)
\item \textbf{تنوع نوع گذار:} مذاکره‌ای (اسپانیا، لهستان، آفریقای جنوبی)، انقلابی/مردمی (تونس، اندونزی)، مداخلهٔ خارجی (عراق، تیمور شرقی)، ترکیبی (شیلی)، ناتمام (میانمار)
\item \textbf{تنوع نتیجه:} موفق (اسپانیا، شیلی)، نسبتاً موفق (لهستان، آفریقای جنوبی، تیمور شرقی)، ناموفق (عراق)، بازگشتی (میانمار، تونس پس از ۲۰۲۱)
\item \textbf{تنوع زمانی:} موج سوم (اسپانیا ۱۹۷۵)، موج چهارم (لهستان ۱۹۸۹)، پسا-یازده‌سپتامبر (عراق ۲۰۰۳)، بهار عربی (تونس ۲۰۱۱)
\item \textbf{مشابهت تحلیلی با ایران:} در حداقل سه بُعد مرتبط با پروندهٔ ایران
\end{enumerate}

\subsection{ابعاد ده‌گانهٔ مقایسه}

\begin{keypoint}
هر نمونه در \textbf{ده بُعد} تحلیل شده است: ۱)مشخصات عمومی، ۲)رژیم پیشین، ۳)مسیر و محرک گذار، ۴)بازیگران اصلی، ۵)نقش بین‌المللی، ۶)نظارت و مدیریت گذار، ۷)عدالت انتقالی، ۸)اصلاحات بخش امنیتی، ۹)نتایج و شاخص‌ها، ۱۰)درس‌آموخته‌ها برای ایران.
\end{keypoint}

\subsection{نظام نمادها و امتیازدهی}

در جداول این پیوست از نظام نمادی زیر استفاده شده است:

\begin{center}
\begin{tabularx}{0.85\textwidth}{>{\centering\arraybackslash}p{2.5cm} >{\raggedleft\arraybackslash}X}
\toprule
\headerrow \textbf{نماد} & \textbf{معنا} \\
\midrule
\cmark & وجود / اجرا / موفقیت \\
\altrow \xmark & عدم وجود / شکست \\
$\sim$ & نسبی / ناقص \\
\altrow \statusok & وضعیت مطلوب \\
\statuswarn & وضعیت هشداری \\
\altrow \statusbad & وضعیت بحرانی \\
\rating{5} & امتیاز ۵ از ۵ (بسیار بالا) \\
\altrow \rating{1} & امتیاز ۱ از ۵ (بسیار پایین) \\
\bottomrule
\end{tabularx}
\end{center}

\sectiondivider

%═══════════════════════════════════════════════════════════
\section{جدول اول: مشخصات عمومی و زمینه‌ای}
\label{app:a:table1}
%═══════════════════════════════════════════════════════════

\begin{landscape}
\pagestyle{empty}
\bigtablefontsize

\begin{longtable}{
  >{\raggedleft\arraybackslash}p{2.1cm}|
  >{\raggedleft\arraybackslash}p{1.7cm}
  >{\raggedleft\arraybackslash}p{1.7cm}
  >{\raggedleft\arraybackslash}p{1.7cm}
  >{\raggedleft\arraybackslash}p{1.7cm}
  >{\raggedleft\arraybackslash}p{1.7cm}
  >{\raggedleft\arraybackslash}p{1.7cm}
  >{\raggedleft\arraybackslash}p{1.7cm}
  >{\raggedleft\arraybackslash}p{1.7cm}
  >{\raggedleft\arraybackslash}p{1.7cm}
}
\caption{مشخصات عمومی نُه نمونهٔ تاریخی گذار}
\label{tab:app-a-general} \\

\toprule
\headerrow
\rot{\textbf{شاخص}} &
\rot{\textbf{اسپانیا}} &
\rot{\textbf{لهستان}} &
\rot{\textbf{شیلی}} &
\rot{\textbf{آفریقای جنوبی}} &
\rot{\textbf{اندونزی}} &
\rot{\textbf{تیمور شرقی}} &
\rot{\textbf{عراق}} &
\rot{\textbf{تونس}} &
\rot{\textbf{میانمار}} \\
\midrule
\endfirsthead

\multicolumn{10}{c}{\small\textit{ادامهٔ جدول \ref{tab:app-a-general}: مشخصات عمومی}} \\
\toprule
\headerrow
\rot{\textbf{شاخص}} &
\rot{\textbf{اسپانیا}} &
\rot{\textbf{لهستان}} &
\rot{\textbf{شیلی}} &
\rot{\textbf{آفریقای جنوبی}} &
\rot{\textbf{اندونزی}} &
\rot{\textbf{تیمور شرقی}} &
\rot{\textbf{عراق}} &
\rot{\textbf{تونس}} &
\rot{\textbf{میانمار}} \\
\midrule
\endhead

\midrule
\multicolumn{10}{l}{\small\textit{ادامه در صفحهٔ بعد...}} \\
\endfoot

\bottomrule
\endlastfoot

%--- ردیف‌ها ---
\textbf{منطقه} &
اروپای جنوبی &
اروپای شرقی &
آمریکای لاتین &
آفریقای جنوبی &
آسیای جنوب‌شرقی &
آسیای جنوب‌شرقی &
خاورمیانه &
شمال آفریقا &
آسیای جنوب‌شرقی \\

\altrow
\textbf{جمعیت (زمان گذار)} &
۳۷ میلیون &
۳۸ میلیون &
۱۳ میلیون &
۴۰ میلیون &
۲۱۰ میلیون &
۰.۸ میلیون &
۲۶ میلیون &
۱۱ میلیون &
۵۲ میلیون \\

\textbf{مساحت (\lr{km²})} &
۵۰۵,۰۰۰ &
۳۱۲,۰۰۰ &
۷۵۶,۰۰۰ &
۱,۲۲۰,۰۰۰ &
۱,۹۰۵,۰۰۰ &
۱۵,۰۰۰ &
۴۳۸,۰۰۰ &
۱۶۳,۰۰۰ &
۶۷۶,۰۰۰ \\

\altrow
\textbf{تنوع قومی-مذهبی} &
متوسط (کاتالان، باسک) &
پایین (۹۷٪ لهستانی) &
پایین &
بسیار بالا (۱۱ زبان رسمی) &
بسیار بالا (۳۰۰+ قوم) &
متوسط &
بالا (عرب/کرد/ ترکمن، شیعه/سنی) &
پایین &
بسیار بالا (۱۳۵ قوم) \\

\textbf{منابع طبیعی کلیدی} &
محدود &
زغال‌سنگ &
مس &
طلا، الماس &
نفت، گاز &
نفت &
نفت (ذخایر ۵) &
فسفات &
یشم، گاز \\

\altrow
\textbf{\lr{GDP per capita} (زمان گذار)} &
$\sim$\$۵,۰۰۰ &
$\sim$\$۱,۷۰۰ &
$\sim$\$۲,۵۰۰ &
$\sim$\$۳,۵۰۰ &
$\sim$\$۱,۰۰۰ &
$\sim$\$۴۰۰ &
$\sim$\$۹۰۰ &
$\sim$\$۴,۲۰۰ &
$\sim$\$۱,۲۰۰ \\

\textbf{سال آغاز گذار} &
۱۹۷۵ &
۱۹۸۹ &
۱۹۸۸ &
۱۹۹۰ &
۱۹۹۸ &
۱۹۹۹ &
۲۰۰۳ &
۲۰۱۱ &
۲۰۱۰ \\

\altrow
\textbf{مدت فاز انتقالی} &
$\sim$۷ سال &
$\sim$۲ سال &
$\sim$۲ سال &
$\sim$۴ سال &
$\sim$۶ سال &
$\sim$۳ سال &
$\sim$۷ سال+ &
$\sim$۳ سال &
$\sim$۵ سال (سپس بازگشت) \\

\textbf{سال اولین انتخابات آزاد} &
۱۹۷۷ &
۱۹۸۹ (نسبی) &
۱۹۸۹ &
۱۹۹۴ &
۱۹۹۹ &
۲۰۰۱ &
۲۰۰۵ &
۲۰۱۱ (مجلس مؤسسان) &
۲۰۱۵ \\

\altrow
\textbf{قانون اساسی جدید} &
۱۹۷۸ &
اصلاحیهٔ ۱۹۹۷ &
اصلاحیهٔ ۱۹۸۹ &
۱۹۹۶ &
اصلاحیه‌های متعدد &
۲۰۰۲ &
۲۰۰۵ &
۲۰۱۴ &
۲۰۰۸ (نظامی) \\

\textbf{\termfn{شاخص دموکراسی فعلی}{V-Dem ۲۰۲۳}} &
\cellgreen{بالا (۰.۸۱)} &
\cellorange{متوسط (۰.۵۵)} &
\cellgreen{بالا (۰.۷۹)} &
\cellgreen{بالا (۰.۷۲)} &
\cellorange{متوسط (۰.۵۲)} &
\cellorange{متوسط (۰.۵۸)} &
\cellred{پایین (۰.۲۵)} &
\cellred{پایین (۰.۱۸)} &
\cellred{بسیار پایین (۰.۰۸)} \\

\altrow
\textbf{ارزیابی کلی نتیجه} &
\statusok موفق &
\statusok موفق (با افت اخیر) &
\statusok موفق &
\statusok نسبتاً موفق &
\statuswarn نسبتاً موفق &
\statuswarn نسبتاً موفق &
\statusbad ناموفق &
\statusbad بازگشت (۲۰۲۱) &
\statusbad بازگشت (۲۰۲۱) \\

\textbf{مقایسه‌پذیری با ایران} &
\rating{4} &
\rating{3} &
\rating{4} &
\rating{5} &
\rating{4} &
\rating{3} &
\rating{3} &
\rating{4} &
\rating{3} \\

\end{longtable}
\end{landscape}

\sectiondivider

%═══════════════════════════════════════════════════════════
\section{جدول دوم: ویژگی‌های رژیم پیشین}
\label{app:a:table2}
%═══════════════════════════════════════════════════════════

\begin{landscape}
\pagestyle{empty}
\bigtablefontsize

\begin{longtable}{
  >{\raggedleft\arraybackslash}p{2.1cm}|
  >{\raggedleft\arraybackslash}p{1.7cm}
  >{\raggedleft\arraybackslash}p{1.7cm}
  >{\raggedleft\arraybackslash}p{1.7cm}
  >{\raggedleft\arraybackslash}p{1.7cm}
  >{\raggedleft\arraybackslash}p{1.7cm}
  >{\raggedleft\arraybackslash}p{1.7cm}
  >{\raggedleft\arraybackslash}p{1.7cm}
  >{\raggedleft\arraybackslash}p{1.7cm}
  >{\raggedleft\arraybackslash}p{1.7cm}
}
\caption{ویژگی‌های رژیم پیشین در نُه نمونه}
\label{tab:app-a-regime} \\

\toprule
\headerrow
\rot{\textbf{شاخص}} &
\rot{\textbf{اسپانیا}} &
\rot{\textbf{لهستان}} &
\rot{\textbf{شیلی}} &
\rot{\textbf{آفریقای جنوبی}} &
\rot{\textbf{اندونزی}} &
\rot{\textbf{تیمور شرقی}} &
\rot{\textbf{عراق}} &
\rot{\textbf{تونس}} &
\rot{\textbf{میانمار}} \\
\midrule
\endfirsthead

\multicolumn{10}{c}{\small\textit{ادامهٔ جدول \ref{tab:app-a-regime}: رژیم پیشین}} \\
\toprule
\headerrow
\rot{\textbf{شاخص}} &
\rot{\textbf{اسپانیا}} &
\rot{\textbf{لهستان}} &
\rot{\textbf{شیلی}} &
\rot{\textbf{آفریقای جنوبی}} &
\rot{\textbf{اندونزی}} &
\rot{\textbf{تیمور شرقی}} &
\rot{\textbf{عراق}} &
\rot{\textbf{تونس}} &
\rot{\textbf{میانمار}} \\
\midrule
\endhead

\bottomrule
\endlastfoot

\textbf{نوع رژیم} &
فاشیست/ اقتدارگرای شخصی &
کمونیست تک‌حزبی &
دیکتاتوری نظامی &
آپارتاید نژادی &
اقتدارگرای شخصی-نظامی &
اشغال خارجی (اندونزی) &
اقتدارگرای تمامیت‌خواه &
اقتدارگرای شخصی &
جونتای نظامی \\

\altrow
\textbf{طول عمر رژیم} &
۳۶ سال (۱۹۳۹-۱۹۷۵) &
۴۵ سال (۱۹۴۴-۱۹۸۹) &
۱۷ سال (۱۹۷۳-۱۹۹۰) &
۴۶ سال (۱۹۴۸-۱۹۹۴) &
۳۲ سال (۱۹۶۶-۱۹۹۸) &
۲۴ سال (۱۹۷۵-۱۹۹۹) &
۳۵ سال (۱۹۶۸-۲۰۰۳) &
۲۳ سال (۱۹۸۷-۲۰۱۱) &
۴۹ سال (۱۹۶۲-۲۰۱۱) \\

\textbf{ایدئولوژی مسلط} &
ناسیونال-کاتولیک &
مارکسیسم-لنینیسم &
ضدکمونیسم/ نئولیبرال &
سفیدبرتری نژادی &
پنچاسیلا/ ضدکمونیسم &
-- (اشغال) &
بعثیسم/ ناسیونالیسم عربی &
سکولاریسم اقتدارگرا &
ناسیونالیسم بامار \\

\altrow
\textbf{نقش ارتش/نیروهای مسلح} &
\riskhigh بالا (ارتش حامی فرانکو) &
\riskmedium متوسط &
\riskhigh بسیار بالا (پینوشه) &
\riskhigh بالا (\lr{SADF}) &
\riskhigh بسیار بالا (\lr{TNI}) &
\riskhigh بسیار بالا &
\riskhigh بسیار بالا (حرس جمهوری) &
\riskmedium متوسط &
\riskhigh بسیار بالا (تاتمادو) \\

\textbf{نقش نهاد مذهبی} &
کلیسای کاتولیک (حامی سپس منتقد) &
کلیسای کاتولیک (حامی جنبش) &
کلیسا (حامی حقوق‌بشر) &
کلیساها (متنوع) &
سازمان‌های اسلامی (متنوع) &
کلیسای کاتولیک (حامی) &
مرجعیت شیعه (دوگانه) &
محدود &
سنگها (بودایی، ملی‌گرا) \\

\altrow
\textbf{سطح سرکوب} &
بالا (اعدام، شکنجه) &
متوسط (حکومت نظامی ۱۹۸۱) &
بسیار بالا (۳,۰۰۰+ ناپدیدشده) &
بسیار بالا (شارپویل، سوتو) &
بالا (تیانمن اندونزیایی) &
بسیار بالا (نسل‌کشی) &
بسیار بالا (انفال، حلبچه) &
بالا (ترور) &
بسیار بالا \\

\textbf{وضعیت جامعهٔ مدنی} &
محدود اما فعال (اواخر) &
قوی (همبستگی) &
بازسازی‌شده (اواخر) &
قوی (\lr{UDF/ANC}) &
نسبتاً فعال &
ضعیف &
سرکوب‌شده &
نسبتاً فعال &
ضعیف \\

\altrow
\textbf{وجود اپوزیسیون سازمان‌یافته} &
\cmark (احزاب مخفی) &
\cmark (همبستگی ۱۰M) &
\cmark (ائتلاف «نه») &
\cmark (\lr{ANC}) &
$\sim$ (پراکنده) &
\cmark (\lr{FRETILIN}) &
$\sim$ (تبعیدی) &
$\sim$ (ضعیف) &
\cmark (\lr{NLD}) \\

\textbf{وضعیت اقتصاد (آستانهٔ گذار)} &
رشد (معجزهٔ اسپانیا) &
بحران شدید &
رشد نسبی &
رکود + تحریم &
بحران مالی آسیا &
فروپاشی &
تحریم + جنگ &
رشد نسبی اما نابرابر &
رکود + تحریم \\

\altrow
\textbf{مشابهت با ج.ا.ایران} &
\rating{3} (شخصی، ارتش، ایدئولوژیک) &
\rating{2} (تک‌حزبی، بحران اقتصادی) &
\rating{4} (نظامی-اقتصادی، رفراندوم) &
\rating{4} (ایدئولوژیک، تحریم، سرکوب) &
\rating{4} (سپاه، تنوع، نفت) &
\rating{2} (اشغال، کوچک) &
\rating{3} (شیعه، نفت، سپاه) &
\rating{3} (خاورمیانه، جوان) &
\rating{3} (نظامی-اقتصادی) \\

\end{longtable}
\end{landscape}

\begin{lessonlearned}
از میان نُه نمونه، \textbf{آفریقای جنوبی} و \textbf{اندونزی} بیشترین مشابهت ساختاری با ایران دارند: هر دو دارای نیروهای امنیتی قدرتمند با منافع اقتصادی، جمعیت بزرگ، تنوع قومی-مذهبی، و ایدئولوژی رسمی سرکوبگر بودند. تفاوت کلیدی ایران: \textbf{ترکیب ایدئولوژی مذهبی + قدرت نظامی-اقتصادی سپاه + برنامه هسته‌ای} این پرونده را منحصربه‌فرد می‌سازد.
\end{lessonlearned}

\sectiondivider

%═══════════════════════════════════════════════════════════
\section{جدول سوم: مسیر و محرک‌های گذار}
\label{app:a:table3}
%═══════════════════════════════════════════════════════════

\begin{landscape}
\pagestyle{empty}
\bigtablefontsize

\begin{longtable}{
  >{\raggedleft\arraybackslash}p{2.1cm}|
  >{\raggedleft\arraybackslash}p{1.7cm}
  >{\raggedleft\arraybackslash}p{1.7cm}
  >{\raggedleft\arraybackslash}p{1.7cm}
  >{\raggedleft\arraybackslash}p{1.7cm}
  >{\raggedleft\arraybackslash}p{1.7cm}
  >{\raggedleft\arraybackslash}p{1.7cm}
  >{\raggedleft\arraybackslash}p{1.7cm}
  >{\raggedleft\arraybackslash}p{1.7cm}
  >{\raggedleft\arraybackslash}p{1.7cm}
}
\caption{مسیر و محرک‌های گذار در نُه نمونه}
\label{tab:app-a-drivers} \\

\toprule
\headerrow
\rot{\textbf{شاخص}} &
\rot{\textbf{اسپانیا}} &
\rot{\textbf{لهستان}} &
\rot{\textbf{شیلی}} &
\rot{\textbf{آفریقای جنوبی}} &
\rot{\textbf{اندونزی}} &
\rot{\textbf{تیمور شرقی}} &
\rot{\textbf{عراق}} &
\rot{\textbf{تونس}} &
\rot{\textbf{میانمار}} \\
\midrule
\endfirsthead

\multicolumn{10}{c}{\small\textit{ادامهٔ جدول \ref{tab:app-a-drivers}: مسیر و محرک‌های گذار}} \\
\toprule
\headerrow
\rot{\textbf{شاخص}} &
\rot{\textbf{اسپانیا}} &
\rot{\textbf{لهستان}} &
\rot{\textbf{شیلی}} &
\rot{\textbf{آفریقای جنوبی}} &
\rot{\textbf{اندونزی}} &
\rot{\textbf{تیمور شرقی}} &
\rot{\textbf{عراق}} &
\rot{\textbf{تونس}} &
\rot{\textbf{میانمار}} \\
\midrule
\endhead

\bottomrule
\endlastfoot

\textbf{نوع گذار (تیپولوژی)} &
\termfn{مذاکره‌ای از بالا}{Pacted} &
\termfn{میزگرد}{Round Table} &
\termfn{ترکیبی}{Hybrid} (رفراندوم) &
\termfn{مذاکره‌ای}{Negotiated} &
\termfn{فروپاشی}{Collapse} + مذاکره &
\termfn{مداخله + رفراندوم}{Intervention} &
\termfn{مداخله نظامی}{Military Intervention} &
\termfn{انقلاب مردمی}{Popular Revolution} &
\termfn{تحول از بالا}{Top-down} \\

\altrow
\textbf{محرک اصلی} &
مرگ فرانکو &
بحران اقتصادی + همبستگی &
فشار مدنی + رفراندوم &
فشار بین‌المللی + بحران &
بحران مالی آسیا &
رفراندوم + مداخله &
حملهٔ نظامی آمریکا &
خودسوزی بوعزیزی &
تصمیم ژنرال‌ها \\

\textbf{نقش مرگ/حذف رهبر} &
\cmark (مرگ طبیعی) &
\xmark &
\xmark (پینوشه ماند) &
\xmark (دکلرک مذاکره کرد) &
\cmark (سقوط سوهارتو) &
-- &
\cmark (سقوط صدام) &
\cmark (فرار بن‌علی) &
\xmark (انتقال تدریجی) \\

\altrow
\textbf{نقش شکاف نخبگان} &
\cmark (اصلاح‌طلبان سوآرز) &
\cmark (یاروزلسکی) &
$\sim$ &
\cmark (دکلرک) &
\cmark (شکاف ارتش) &
$\sim$ &
\xmark &
$\sim$ &
\cmark \\

\textbf{نقش اعتراضات مردمی} &
$\sim$ &
\cmark (قوی) &
\cmark (قوی) &
\cmark (بسیار قوی) &
\cmark (بسیار قوی) &
\cmark &
\xmark &
\cmark (بسیار قوی) &
$\sim$ \\

\altrow
\textbf{نقش فشار خارجی} &
\cmark (\lr{EC}) &
\cmark (واتیکان + آمریکا) &
\cmark (آمریکا) &
\cmark (تحریم‌ها) &
\cmark (\lr{IMF}) &
\cmark (سازمان ملل) &
\cmark (تعیین‌کننده) &
$\sim$ &
$\sim$ \\

\textbf{خشونت فاز گذار} &
\risklow محدود (کودتای ۲۳-F) &
\risklow بسیار محدود &
\risklow محدود &
\riskmedium متوسط (IFP/ANC) &
\riskmedium متوسط (شورش‌ها) &
\riskhigh شدید (میلیشیا) &
\riskhigh بسیار شدید (جنگ داخلی) &
\risklow محدود &
\risklow محدود (در فاز اول) \\

\altrow
\textbf{توافق‌نامهٔ سیاسی رسمی} &
\cmark \lr{Moncloa Pacts} (۱۹۷۷) &
\cmark توافق میزگرد (۱۹۸۹) &
$\sim$ رفراندوم ۱۹۸۸ &
\cmark \lr{CODESA/} قانون اساسی موقت &
$\sim$ توافق ضمنی &
\cmark توافق نیویورک ۱۹۹۹ &
\xmark (اشغال نظامی) &
\cmark گفت‌وگوی ملی &
$\sim$ قانون اساسی ۲۰۰۸ \\

\textbf{درس برای ایران} &
اهمیت اصلاح‌طلب درون نظام &
قدرت جنبش اجتماعی سازمان‌یافته &
ابزار رفراندوم &
مذاکره با فشار &
ارتش منشعب‌شدنی &
نیاز به حضور بین‌المللی &
مداخله نظامی = فاجعه &
سرعت بالا + خطر بازگشت &
گشایش کنترل‌شده ≠ دموکراسی \\

\end{longtable}
\end{landscape}

\begin{warningbox}
از نُه نمونه، تنها \textbf{عراق} با \textbf{مداخلهٔ نظامی خارجی} وارد گذار شد و تنها نمونه‌ای است که به \textbf{جنگ داخلی} منجر شد. هزینهٔ انسانی: بیش از ۲۰۰,۰۰۰ کشته. هزینهٔ مالی: بیش از ۲ تریلیون دلار. این نمونه \textbf{قوی‌ترین شاهد ضد مداخلهٔ نظامی} در پروندهٔ ایران است (\seeChapter{ch:scenarios}).
\end{warningbox}

\sectiondivider

%═══════════════════════════════════════════════════════════
\section{جدول چهارم: بازیگران اصلی داخلی}
\label{app:a:table4}
%═══════════════════════════════════════════════════════════

\begin{landscape}
\pagestyle{empty}
\bigtablefontsize

\begin{longtable}{
  >{\raggedleft\arraybackslash}p{2.1cm}|
  >{\raggedleft\arraybackslash}p{1.7cm}
  >{\raggedleft\arraybackslash}p{1.7cm}
  >{\raggedleft\arraybackslash}p{1.7cm}
  >{\raggedleft\arraybackslash}p{1.7cm}
  >{\raggedleft\arraybackslash}p{1.7cm}
  >{\raggedleft\arraybackslash}p{1.7cm}
  >{\raggedleft\arraybackslash}p{1.7cm}
  >{\raggedleft\arraybackslash}p{1.7cm}
  >{\raggedleft\arraybackslash}p{1.7cm}
}
\caption{بازیگران اصلی داخلی گذار}
\label{tab:app-a-actors} \\

\toprule
\headerrow
\rot{\textbf{شاخص}} &
\rot{\textbf{اسپانیا}} &
\rot{\textbf{لهستان}} &
\rot{\textbf{شیلی}} &
\rot{\textbf{آفریقای جنوبی}} &
\rot{\textbf{اندونزی}} &
\rot{\textbf{تیمور شرقی}} &
\rot{\textbf{عراق}} &
\rot{\textbf{تونس}} &
\rot{\textbf{میانمار}} \\
\midrule
\endfirsthead

\multicolumn{10}{c}{\small\textit{ادامهٔ جدول \ref{tab:app-a-actors}: بازیگران داخلی}} \\
\toprule
\headerrow
\rot{\textbf{شاخص}} &
\rot{\textbf{اسپانیا}} &
\rot{\textbf{لهستان}} &
\rot{\textbf{شیلی}} &
\rot{\textbf{آفریقای جنوبی}} &
\rot{\textbf{اندونزی}} &
\rot{\textbf{تیمور شرقی}} &
\rot{\textbf{عراق}} &
\rot{\textbf{تونس}} &
\rot{\textbf{میانمار}} \\
\midrule
\endhead

\bottomrule
\endlastfoot

\textbf{رهبر کلیدی گذار} &
\person{خوان‌کارلوس + سوآرز}{Juan Carlos \& Suárez} &
\person{والسا + مازوویتسکی}{Wałęsa} &
\person{آیلوین}{Aylwin} &
\person{ماندلا + دکلرک}{Mandela \& de Klerk} &
\person{حبیبی + واحد}{Habibie \& Wahid} &
\person{گوسمائو}{Gusmão} &
\person{برمر (خارجی)}{Bremer} &
\person{السبسی + غنوشی}{Essebsi \& Ghannouchi} &
\person{تئین‌سئین}{Thein Sein} \\

\altrow
\textbf{نقش رهبر پیشین رژیم} &
فرانکو مُرد &
یاروزلسکی تسلیم شد &
پینوشه ماند (سناتور) &
دکلرک شریک شد &
سوهارتو استعفا داد &
-- (اشغالگر رفت) &
صدام اعدام شد &
بن‌علی فرار کرد &
تان‌شوی بازنشسته شد \\

\textbf{حزب/جنبش اصلی اپوزیسیون} &
\lr{PSOE + PCE} &
\lr{Solidarność} &
\lr{Concertación} &
\lr{ANC + COSATU + SACP} &
\lr{PDI-P} + اسلامی‌ها &
\lr{FRETILIN + CNRT} &
احزاب شیعه/کرد &
\lr{Ennahda + Nidaa} &
\lr{NLD} \\

\altrow
\textbf{نقش جنبش کارگری} &
\cmark (\lr{CCOO}) &
\cmark (تعیین‌کننده) &
\cmark (\lr{CUT}) &
\cmark (\lr{COSATU}) &
$\sim$ &
\xmark &
\xmark &
\cmark (\lr{UGTT}) &
\xmark \\

\textbf{نقش زنان} &
$\sim$ (محدود) &
$\sim$ (محدود) &
$\sim$ (محدود) &
\cmark (منشور حقوق) &
$\sim$ &
$\sim$ &
\xmark &
\cmark (مادهٔ ۴۶ قانون اساسی) &
\cmark (\lr{NLD}: سوچی) \\

\altrow
\textbf{نقش دیاسپورا} &
$\sim$ &
\cmark (لابی در غرب) &
\cmark (تبعیدیان) &
\xmark &
$\sim$ &
\cmark (لابی استقلال) &
\cmark (شورای حاکمیتی) &
$\sim$ &
\cmark \\

\textbf{مقایسه با اپوزیسیون ایران} &
ایران: احزاب ضعیف‌تر &
ایران: فاقد تشکل ۱۰M &
ایران: رفراندوم ممکن &
ایران: دیاسپورای قوی‌تر &
ایران: سپاه = \lr{TNI}+ &
ایران: بزرگ‌تر &
ایران: اپوزیسیون پراکنده‌تر &
ایران: جنبش زنان قوی &
ایران: فاقد سوچی \\

\end{longtable}
\end{landscape}

\sectiondivider

%═══════════════════════════════════════════════════════════
\section{جدول پنجم: نقش بین‌المللی و نظارت}
\label{app:a:table5}
%═══════════════════════════════════════════════════════════

\begin{landscape}
\pagestyle{empty}
\bigtablefontsize

\begin{longtable}{
  >{\raggedleft\arraybackslash}p{2.1cm}|
  >{\raggedleft\arraybackslash}p{1.7cm}
  >{\raggedleft\arraybackslash}p{1.7cm}
  >{\raggedleft\arraybackslash}p{1.7cm}
  >{\raggedleft\arraybackslash}p{1.7cm}
  >{\raggedleft\arraybackslash}p{1.7cm}
  >{\raggedleft\arraybackslash}p{1.7cm}
  >{\raggedleft\arraybackslash}p{1.7cm}
  >{\raggedleft\arraybackslash}p{1.7cm}
  >{\raggedleft\arraybackslash}p{1.7cm}
}
\caption{نقش بین‌المللی و ساختار نظارت}
\label{tab:app-a-intl} \\

\toprule
\headerrow
\rot{\textbf{شاخص}} &
\rot{\textbf{اسپانیا}} &
\rot{\textbf{لهستان}} &
\rot{\textbf{شیلی}} &
\rot{\textbf{آفریقای جنوبی}} &
\rot{\textbf{اندونزی}} &
\rot{\textbf{تیمور شرقی}} &
\rot{\textbf{عراق}} &
\rot{\textbf{تونس}} &
\rot{\textbf{میانمار}} \\
\midrule
\endfirsthead

\multicolumn{10}{c}{\small\textit{ادامهٔ جدول \ref{tab:app-a-intl}: نقش بین‌المللی}} \\
\toprule
\headerrow
\rot{\textbf{شاخص}} &
\rot{\textbf{اسپانیا}} &
\rot{\textbf{لهستان}} &
\rot{\textbf{شیلی}} &
\rot{\textbf{آفریقای جنوبی}} &
\rot{\textbf{اندونزی}} &
\rot{\textbf{تیمور شرقی}} &
\rot{\textbf{عراق}} &
\rot{\textbf{تونس}} &
\rot{\textbf{میانمار}} \\
\midrule
\endhead

\bottomrule
\endlastfoot

\textbf{مدل نظارت (از فصل ۳)} &
مدل ۱ (انتخاباتی محدود) &
مدل ۲ (مشورتی) &
مدل ۲ (مشورتی) &
مدل ۳ (ساختاری) &
مدل ۲ (مشورتی) &
مدل ۵ (مدیریت مستقیم) &
مدل ۵ (مدیریت مستقیم) &
مدل ۱-۲ (انتخاباتی + مشورتی) &
مدل ۱ (انتخاباتی محدود) \\

\altrow
\textbf{قطعنامهٔ شورای امنیت} &
\xmark &
\xmark &
\xmark &
\xmark &
\xmark &
\cmark (Res.\ 1272) &
\cmark (Res.\ 1483+) &
\xmark &
\xmark \\

\textbf{مأموریت سازمان ملل} &
\xmark &
\xmark &
\xmark &
ناظران (UNOMSA) &
\xmark &
\cmark (UNTAET/UNMISET) &
\cmark (UNAMI) &
\xmark &
$\sim$ (نمایندهٔ ویژه) \\

\altrow
\textbf{نیروی حافظ صلح} &
\xmark &
\xmark &
\xmark &
\xmark &
\xmark (INTERFET استرالیا) &
\cmark (INTERFET→PKF) &
\cmark (MNF-I) &
\xmark &
\xmark \\

\textbf{ناظران انتخاباتی بین‌المللی} &
$\sim$ (محدود) &
\cmark &
\cmark (قوی) &
\cmark (۲,۱۲۰ ناظر) &
\cmark (کارتر) &
\cmark (سازمان ملل) &
\cmark (محدود) &
\cmark (\lr{EU + Carter}) &
\cmark \\

\altrow
\textbf{نقش سازمان‌های منطقه‌ای} &
\cmark (\lr{EC}→عضویت) &
\cmark (\lr{CSCE/NATO}) &
\cmark (\lr{OAS}) &
\cmark (\lr{OAU/SADC}) &
\cmark (\lr{ASEAN} محدود) &
$\sim$ &
$\sim$ (اتحادیهٔ عرب ضعیف) &
$\sim$ (اتحادیهٔ عرب) &
\cmark (\lr{ASEAN}) \\

\textbf{نقش \lr{NGO}های بین‌المللی} &
$\sim$ &
\cmark (\lr{NED, Soros}) &
\cmark (\lr{HRW, AI}) &
\cmark (بسیار فعال) &
\cmark &
\cmark &
\cmark &
\cmark (بسیار فعال) &
\cmark \\

\altrow
\textbf{تحریم‌ها (قبل از گذار)} &
\xmark &
\xmark &
$\sim$ (محدود) &
\cmark (جامع و مؤثر) &
\xmark &
-- &
\cmark (جامع) &
\xmark &
\cmark (هدفمند) \\

\textbf{کمک‌های بین‌المللی پس از گذار} &
\cmark (صندوق‌های \lr{EC}) &
\cmark (\lr{PHARE, IMF}) &
$\sim$ &
\cmark (\lr{EU + US}) &
\cmark (\lr{IMF} مشروط) &
\cmark ($\sim$\$۵B) &
\cmark ($\sim$\$۶۰B+) &
\cmark ($\sim$\$۱B) &
\cmark (محدود) \\

\altrow
\textbf{مدل پیشنهادی برای ایران (ارجاع)} &
عنصر: مشوق عضویت &
عنصر: حمایت جامعهٔ مدنی &
عنصر: رفراندوم &
مدل ۶ اصلی &
عنصر: \lr{DDR} سپاه &
عنصر: مدیریت مرحله‌ای &
\textbf{ضد الگو} &
عنصر: گفت‌وگوی ملی &
درس: خطر بازگشت \\

\end{longtable}
\end{landscape}

\begin{recommendation}
\textbf{مدل ۶ (ترکیبی-تطبیقی) پیشنهادی برای ایران} بر مبنای مقایسهٔ این جدول طراحی شده: ناظران انتخاباتی از شیلی و آفریقای جنوبی، مشاوره فنی از لهستان، نظارت ساختاری از آفریقای جنوبی، تضمین‌های اجرایی از تیمور شرقی — اما بدون مدیریت مستقیم (ضد الگوی عراق). مالکیت ملی ایرانی: اصل غیرقابل‌مذاکره (\seeChapter{ch:approaches}).
\end{recommendation}

\sectiondivider

%═══════════════════════════════════════════════════════════
\section{جدول ششم: عدالت انتقالی}
\label{app:a:table6}
%═══════════════════════════════════════════════════════════

\begin{landscape}
\pagestyle{empty}
\bigtablefontsize

\begin{longtable}{
  >{\raggedleft\arraybackslash}p{2.1cm}|
  >{\raggedleft\arraybackslash}p{1.7cm}
  >{\raggedleft\arraybackslash}p{1.7cm}
  >{\raggedleft\arraybackslash}p{1.7cm}
  >{\raggedleft\arraybackslash}p{1.7cm}
  >{\raggedleft\arraybackslash}p{1.7cm}
  >{\raggedleft\arraybackslash}p{1.7cm}
  >{\raggedleft\arraybackslash}p{1.7cm}
  >{\raggedleft\arraybackslash}p{1.7cm}
  >{\raggedleft\arraybackslash}p{1.7cm}
}
\caption{مقایسهٔ سازوکارهای عدالت انتقالی}
\label{tab:app-a-tj} \\

\toprule
\headerrow
\rot{\textbf{شاخص}} &
\rot{\textbf{اسپانیا}} &
\rot{\textbf{لهستان}} &
\rot{\textbf{شیلی}} &
\rot{\textbf{آفریقای جنوبی}} &
\rot{\textbf{اندونزی}} &
\rot{\textbf{تیمور شرقی}} &
\rot{\textbf{عراق}} &
\rot{\textbf{تونس}} &
\rot{\textbf{میانمار}} \\
\midrule
\endfirsthead

\multicolumn{10}{c}{\small\textit{ادامهٔ جدول \ref{tab:app-a-tj}: عدالت انتقالی}} \\
\toprule
\headerrow
\rot{\textbf{شاخص}} &
\rot{\textbf{اسپانیا}} &
\rot{\textbf{لهستان}} &
\rot{\textbf{شیلی}} &
\rot{\textbf{آفریقای جنوبی}} &
\rot{\textbf{اندونزی}} &
\rot{\textbf{تیمور شرقی}} &
\rot{\textbf{عراق}} &
%══════════════════════════════════════════════════════════════
% تکمیل جدول ششم + جداول ۷-۱۰ + نمودار حبابی
% این بلوک بین endhead جدول ششم و نمودار حبابی قرار می‌گیرد
%══════════════════════════════════════════════════════════════

% ── ادامهٔ endhead جدول ششم ──
\rot{\textbf{تونس}} &
\rot{\textbf{میانمار}} \\
\midrule
\endhead

\bottomrule
\endlastfoot

%--- ردیف‌های جدول ششم ---

\textbf{کمیسیون حقیقت} &
\xmark (عفو ۱۹۷۷) &
$\sim$ (\lr{IPN} بایگانی) &
\cmark (کمیسیون رِتیگ ۱۹۹۱ + والِش ۲۰۰۴) &
\cmark (\lr{TRC} ۱۹۹۵-۲۰۰۲، ریاست توتو) &
\xmark (لایحه رد شد) &
\cmark (\lr{CAVR} ۲۰۰۲-۲۰۰۵) &
\xmark &
\cmark (\lr{IVD} ۲۰۱۴-۲۰۱۹) &
\xmark \\

\altrow
\textbf{دادگاه ویژه / محاکمات} &
\xmark &
$\sim$ (محدود، اواخر) &
\cmark (پینوشه ۱۹۹۸ بازداشت لندن) &
$\sim$ (محدود، بیشتر عفو) &
\xmark &
\cmark (دادگاه ویژه ترکیبی \lr{UN}) &
\cmark (دادگاه صدام ۲۰۰۶) &
$\sim$ (محاکمهٔ بن‌علی غیابی) &
\xmark \\

\textbf{مکانیزم عفو} &
\cmark (عفو عمومی بلانکت ۱۹۷۷) &
\xmark &
$\sim$ (عفو نظامی ۱۹۷۸ — بعداً لغو) &
\cmark (عفو مشروط فردی در \lr{TRC}) &
\xmark (بدون مکانیزم) &
$\sim$ &
\xmark &
$\sim$ &
\xmark \\

\altrow
\textbf{جبران خسارت قربانیان} &
\cmark (قانون ۲۰۰۷ — ۳۰ سال بعد!) &
$\sim$ (بسیار محدود) &
\cmark (غرامت مالی + نمادین) &
\cmark (برنامهٔ جبران، ناقص) &
$\sim$ (بسیار محدود) &
\cmark (برنامهٔ جبران ملی) &
\xmark &
\cmark (صندوق جبران) &
\xmark \\

\textbf{لوستراسیون / پاک‌سازی} &
\xmark &
\cmark (قانون ۱۹۹۷ — محدود و جنجالی) &
$\sim$ (ممنوعیت مناصب محدود) &
$\sim$ (عمدتاً نه) &
\xmark &
$\sim$ &
\cmark (بعث‌زدایی افراطی — فاجعه) &
$\sim$ (\lr{IVD} توصیه کرد، اجرا نشد) &
\xmark \\

\altrow
\textbf{تعداد تقریبی قربانیان رسیدگی‌شده} &
--- &
$\sim$۲۰,۰۰۰ پرونده &
$\sim$۳,۱۰۰ ناپدید + ۳۰,۰۰۰+ شکنجه &
۲۱,۰۰۰ شهادت، ۷,۱۱۲ درخواست عفو &
$\sim$۱۰۰ پرونده &
$\sim$۸,۰۰۰ شهادت &
$\sim$۳۰,۰۰۰ بعثی محکوم/اخراج &
$\sim$۶۲,۰۰۰ پرونده &
--- \\

\textbf{ارزیابی کلی عدالت انتقالی} &
\cellred{ناکافی: عفو بدون حقیقت} &
\cellorange{متوسط: تأخیر ۸ سال} &
\cellgreen{مؤثر: ترکیب کمیسیون + دادگاه} &
\cellgreen{الگویی: \lr{TRC} مدل جهانی شد} &
\cellred{ناکافی: بدون مکانیزم} &
\cellorange{نسبتاً مؤثر} &
\cellred{فاجعه: بعث‌زدایی = جنگ داخلی} &
\cellorange{مؤثر اما ناتمام} &
\cellred{صفر} \\

\altrow
\textbf{درس برای ایران} &
عفو بدون حقیقت نارضایتی می‌آورد &
تأخیر پذیرفتنی اما نه بیش از ۳ سال &
ترکیب حقیقت + محاکمه بهترین &
مدل \lr{TRC} با تعدیل ایرانی &
بدون عدالت ≠ ثبات &
مدل ترکیبی قابل‌استفاده &
اجتثاث افراطی = خطر مرگبار &
شروع سریع + صبر برای تکمیل &
بدون عدالت = بازگشت \\

\end{longtable}
\end{landscape}

\begin{casestudy}{مقایسهٔ سه مدل عدالت انتقالی}
\textbf{مدل اسپانیایی (فراموشی):} عفو عمومی ۱۹۷۷ باعث «پیمان فراموشی» شد. ثبات کوتاه‌مدت فراهم کرد اما زخم‌های بلندمدت ماند. ۴۰ سال بعد (۲۰۰۷) قانون حافظهٔ تاریخی تصویب شد — خیلی دیر. \textbf{مدل عراقی (انتقام):} بعث‌زدایی بدون تمایز، ۴۰۰,۰۰۰ نفر را اخراج و میلیون‌ها نفر را به‌حاشیه راند. نتیجه: داعش. \textbf{مدل آفریقای جنوبی (آشتی):} عفو فردی مشروط به اعتراف علنی و کامل. حقیقت آشکار شد بدون جنگ داخلی. \textbf{توصیه برای ایران:} مدل سوم با تقویت بُعد محاکمهٔ آمران جنایات (\seeChapter{ch:requirements}).
\end{casestudy}

\sectiondivider

%═══════════════════════════════════════════════════════════
\section{جدول هفتم: اصلاحات بخش امنیتی}
\label{app:a:table7}
%═══════════════════════════════════════════════════════════

\begin{landscape}
\pagestyle{empty}
\bigtablefontsize

\begin{longtable}{
  >{\raggedleft\arraybackslash}p{2.1cm}|
  >{\raggedleft\arraybackslash}p{1.7cm}
  >{\raggedleft\arraybackslash}p{1.7cm}
  >{\raggedleft\arraybackslash}p{1.7cm}
  >{\raggedleft\arraybackslash}p{1.7cm}
  >{\raggedleft\arraybackslash}p{1.7cm}
  >{\raggedleft\arraybackslash}p{1.7cm}
  >{\raggedleft\arraybackslash}p{1.7cm}
  >{\raggedleft\arraybackslash}p{1.7cm}
  >{\raggedleft\arraybackslash}p{1.7cm}
}
\caption{مقایسهٔ اصلاحات بخش امنیتی (\lr{SSR})}
\label{tab:app-a-ssr} \\

\toprule
\headerrow
\rot{\textbf{شاخص}} &
\rot{\textbf{اسپانیا}} &
\rot{\textbf{لهستان}} &
\rot{\textbf{شیلی}} &
\rot{\textbf{آفریقای جنوبی}} &
\rot{\textbf{اندونزی}} &
\rot{\textbf{تیمور شرقی}} &
\rot{\textbf{عراق}} &
\rot{\textbf{تونس}} &
\rot{\textbf{میانمار}} \\
\midrule
\endfirsthead

\multicolumn{10}{c}{\small\textit{ادامهٔ جدول \ref{tab:app-a-ssr}: اصلاحات بخش امنیتی}} \\
\toprule
\headerrow
\rot{\textbf{شاخص}} &
\rot{\textbf{اسپانیا}} &
\rot{\textbf{لهستان}} &
\rot{\textbf{شیلی}} &
\rot{\textbf{آفریقای جنوبی}} &
\rot{\textbf{اندونزی}} &
\rot{\textbf{تیمور شرقی}} &
\rot{\textbf{عراق}} &
\rot{\textbf{تونس}} &
\rot{\textbf{میانمار}} \\
\midrule
\endhead

\bottomrule
\endlastfoot

\textbf{اندازهٔ نیروهای مسلح (زمان گذار)} &
$\sim$۳۰۰K &
$\sim$۴۰۰K &
$\sim$۱۰۰K &
$\sim$۹۰K (\lr{SADF}) + ۲۰K+ (\lr{MK}) &
$\sim$۴۰۰K (\lr{TNI}) &
-- (اشغالگر) &
$\sim$۴۰۰K + ۵۰K حرس &
$\sim$۵۰K &
$\sim$۴۰۰K (تاتمادو) \\

\altrow
\textbf{سهم اقتصادی ارتش} &
\riskmedium متوسط &
\risklow پایین &
\riskmedium متوسط &
\riskmedium متوسط &
\riskhigh بسیار بالا (۲۰-۳۰٪) &
-- &
\riskhigh بالا &
\risklow پایین &
\riskhigh بسیار بالا (۴۰٪+) \\

\textbf{استراتژی اصلاح} &
حفظ + تبعیت تدریجی &
ادغام در \lr{NATO} &
حفظ خودمختاری (تدریجاً کاهش) &
ادغام \lr{SADF+MK+APLA} در \lr{SANDF} &
تفکیک تدریجی نقش نظامی/اقتصادی &
ایجاد نیروی جدید (\lr{F-FDTL}) &
\cellred{انحلال کامل ارتش} &
بی‌طرف‌سازی + اصلاح تدریجی &
\cellred{هیچ اصلاحی (فریب)} \\

\altrow
\textbf{برنامهٔ \lr{DDR}} &
\xmark (نیازی نبود) &
\xmark &
\xmark &
\cmark (ادغام چندنیرویی) &
$\sim$ (محدود) &
\cmark (\lr{FALINTIL} → \lr{F-FDTL}) &
\cellred{\xmark انحلال بدون \lr{DDR}} &
$\sim$ &
\xmark \\

\textbf{نظارت مدنی بر ارتش} &
\cmark (تدریجی — کودتای نافرجام ۲۳-F ۱۹۸۱) &
\cmark (وزیر دفاع غیرنظامی) &
$\sim$ (پینوشه فرمانده ماند تا ۱۹۹۸) &
\cmark (وزیر دفاع غیرنظامی) &
$\sim$ (هنوز نفوذ) &
\cmark (از ابتدا مدنی) &
\cellred{\xmark} &
\cmark &
\cellred{\xmark (ارتش ۲۵٪ پارلمان)} \\

\altrow
\textbf{خلع‌سلاح هسته‌ای} &
-- &
-- &
-- &
\cmark (۶ کلاهک تحویل داوطلبانه ۱۹۹۳) &
-- &
-- &
\cmark (WMD وجود نداشت) &
-- &
-- \\

\textbf{اصلاح پلیس} &
\cmark (حل \lr{Guardia Civil} → پلیس ملی) &
\cmark &
$\sim$ &
\cmark (ادغام \lr{SAP+KZP}→\lr{SAPS}) &
$\sim$ (محدود) &
\cmark (ایجاد \lr{PNTL}) &
$\sim$ &
$\sim$ &
\xmark \\

\altrow
\textbf{نتیجهٔ \lr{SSR}} &
\cellgreen{موفق — ارتش حرفه‌ای غیرسیاسی} &
\cellgreen{موفق — عضو \lr{NATO}} &
\cellorange{ناقص — ارتش تا ۱۹۹۸ خودمختار} &
\cellgreen{موفق — \lr{SANDF} چندنژادی} &
\cellorange{ناقص — نفوذ ادامه دارد} &
\cellorange{شکننده — بحران ۲۰۰۶} &
\cellred{فاجعه — منشأ شورش و داعش} &
\cellorange{نسبی — بدون اصلاح ساختاری} &
\cellred{فاجعه — کودتای ۲۰۲۱} \\

\textbf{درس برای سپاه ایران} &
صبر + تبعیت تدریجی &
مشوق خارجی (معادل \lr{NATO}) &
پذیرش موقت خودمختاری &
ادغام بهترین مدل &
تفکیک نقش اقتصادی حیاتی &
ایجاد نهاد جدید اگر لازم &
\textbf{هرگز انحلال!} &
بی‌طرف‌سازی ممکن اگر ارتش حرفه‌ای &
گشایش کنترل‌شده ≠ اصلاح \\

\end{longtable}
\end{landscape}

\begin{warningbox}
\textbf{درس حیاتی برای سپاه پاسداران:} سپاه با $\sim$۱۹۰,۰۰۰ نفر نیروی نظامی، ۲۰-۴۰٪ اقتصاد، شبکهٔ اطلاعاتی گسترده، و ایدئولوژی تثبیت‌شده، ترکیبی از ارتش اندونزی (\lr{TNI})، حرس جمهوری عراق، و تاتمادوی میانمار است — و به‌مراتب پیچیده‌تر. تنها مدل موفقِ مقایسه‌ای: \textbf{ادغام تدریجی آفریقای جنوبی} + \textbf{تفکیک اقتصادی اندونزی} + \textbf{مشوق بین‌المللی لهستان}. انحلال (مدل عراق) \emphred{به هیچ وجه} نباید تکرار شود (\seeChapter{ch:guarantees}).
\end{warningbox}

\sectiondivider

%═══════════════════════════════════════════════════════════
\section{جدول هشتم: نتایج اقتصادی و اجتماعی گذار}
\label{app:a:table8}
%═══════════════════════════════════════════════════════════

\begin{landscape}
\pagestyle{empty}
\bigtablefontsize

\begin{longtable}{
  >{\raggedleft\arraybackslash}p{2.1cm}|
  >{\raggedleft\arraybackslash}p{1.7cm}
  >{\raggedleft\arraybackslash}p{1.7cm}
  >{\raggedleft\arraybackslash}p{1.7cm}
  >{\raggedleft\arraybackslash}p{1.7cm}
  >{\raggedleft\arraybackslash}p{1.7cm}
  >{\raggedleft\arraybackslash}p{1.7cm}
  >{\raggedleft\arraybackslash}p{1.7cm}
  >{\raggedleft\arraybackslash}p{1.7cm}
  >{\raggedleft\arraybackslash}p{1.7cm}
}
\caption{نتایج اقتصادی و اجتماعی گذار}
\label{tab:app-a-econ} \\

\toprule
\headerrow
\rot{\textbf{شاخص}} &
\rot{\textbf{اسپانیا}} &
\rot{\textbf{لهستان}} &
\rot{\textbf{شیلی}} &
\rot{\textbf{آفریقای جنوبی}} &
\rot{\textbf{اندونزی}} &
\rot{\textbf{تیمور شرقی}} &
\rot{\textbf{عراق}} &
\rot{\textbf{تونس}} &
\rot{\textbf{میانمار}} \\
\midrule
\endfirsthead

\multicolumn{10}{c}{\small\textit{ادامهٔ جدول \ref{tab:app-a-econ}: نتایج اقتصادی}} \\
\toprule
\headerrow
\rot{\textbf{شاخص}} &
\rot{\textbf{اسپانیا}} &
\rot{\textbf{لهستان}} &
\rot{\textbf{شیلی}} &
\rot{\textbf{آفریقای جنوبی}} &
\rot{\textbf{اندونزی}} &
\rot{\textbf{تیمور شرقی}} &
\rot{\textbf{عراق}} &
\rot{\textbf{تونس}} &
\rot{\textbf{میانمار}} \\
\midrule
\endhead

\bottomrule
\endlastfoot

\textbf{رشد \lr{GDP} ۵ سال اول} &
\cellgreen{+۳.۵٪ میانگین} &
\cellred{-۷٪ (۱۹۹۰) سپس +۵٪} &
\cellgreen{+۷٪ میانگین} &
\cellorange{+۲.۵٪ میانگین} &
\cellred{-۱۳٪ (۱۹۹۸) سپس +۴٪} &
\cellorange{+۲٪ (وابسته)} &
\cellred{-۳۰٪ (۲۰۰۳) سپس نوسانی} &
\cellorange{+۱.۵٪ ضعیف} &
\cellorange{+۶٪ (آمارهای مشکوک)} \\

\altrow
\textbf{تورم فاز انتقالی} &
بالا (۲۰٪) — تدریجاً کاهش &
بسیار بالا (۵۸۶٪ ۱۹۹۰) — شوک‌درمانی &
پایین (کنترل‌شده) &
متوسط (۹٪) &
بالا (۷۸٪ ۱۹۹۸) &
بالا &
بسیار بالا &
متوسط (۵٪) &
بالا \\

\textbf{بیکاری} &
بالا (۱۵-۲۰٪) &
بالا (۱۵٪ — شوک) &
متوسط (۸٪) &
بسیار بالا (۲۵-۳۰٪ — نماند) &
بالا (۱۵٪+) &
بسیار بالا &
بسیار بالا (۵۰٪+) &
بالا (۱۵-۱۸٪ جوانان ۳۵٪) &
بالا \\

\altrow
\textbf{نابرابری (جینی)} &
۰.۳۴ → ۰.۳۲ (بهبود) &
۰.۲۷ → ۰.۳۴ (بدتر) &
۰.۵۶ → ۰.۴۷ (بهبود کند) &
۰.۵۹ → ۰.۶۳ (بدتر!) &
۰.۳۶ → ۰.۳۹ (بدتر) &
--- &
--- (بدتر) &
۰.۳۶ → ۰.۴۰ (بدتر) &
--- \\

\textbf{فقر} &
کاهش تدریجی &
افزایش موقت سپس کاهش شدید &
کاهش قابل‌توجه (۴۵٪→۱۲٪) &
کاهش (۵۰٪→۲۵٪) اما ناکافی &
کاهش تدریجی &
بسیار بالا (ماند) &
افزایش شدید &
بدون تغییر محسوس &
بسیار بالا (ماند) \\

\altrow
\textbf{نقش تحریم/رفع آن} &
-- (تحریم نبود) &
-- &
$\sim$ (تحریم محدود — رفع) &
\cmark (رفع تحریم = محرک رشد) &
\cmark (\lr{IMF} مشروط) &
-- &
\cmark (برداشته شد — اثر محدود) &
-- &
\cmark (تحریم‌ها باقی ماند) \\

\textbf{کمک بین‌المللی مالی} &
\cmark (صندوق‌های \lr{EC}) &
\cmark (\lr{PHARE}: \$۱.۵B+) &
$\sim$ (محدود) &
\cmark (\lr{EU + US}: \$۱B+) &
\cmark (\lr{IMF}: \$۴۳B) &
\cmark ($\sim$\$۵B) &
\cmark ($\sim$\$۶۰B+) &
\cmark ($\sim$\$۱B) &
$\sim$ (محدود) \\

\altrow
\textbf{ارزیابی کلی اقتصادی} &
\cellgreen{موفق — عضو \lr{EU}} &
\cellgreen{موفق — معجزهٔ لهستان} &
\cellgreen{بسیار موفق — ببر آمریکای لاتین} &
\cellorange{مختلط — رشد اما نابرابری} &
\cellorange{بازیابی تدریجی} &
\cellorange{وابسته به کمک} &
\cellred{فاجعه} &
\cellred{ضعیف — عامل عقب‌گرد} &
\cellorange{رشد صوری بدون توسعه} \\

\textbf{درس اقتصادی برای ایران} &
مشوق عضویت = تسریع‌کننده &
شوک‌درمانی هزینهٔ اجتماعی دارد &
ثبات ماکرو + رشد = ضامن دموکراسی &
رشد بدون عدالت = ناپایداری &
بستهٔ \lr{IMF} با شرط = دوسویه &
وابستگی بلندمدت خطرناک &
بازسازی بدون برنامه = هدررفت &
اقتصاد ضعیف = سرخوردگی = بازگشت &
رشد بدون آزادی ≠ توسعه \\

\end{longtable}
\end{landscape}

\begin{lessonlearned}
\textbf{فرمول اقتصادی گذار موفق (از مقایسهٔ ۹ نمونه):}
\begin{enumerate}[nosep]
    \item \textbf{تثبیت فوری} (ماه ۱-۶): جلوگیری از فروپاشی ارزی و بانکی (درس اندونزی ۱۹۹۸)
    \item \textbf{رفع تحریم سریع} (ماه ۱-۳): آفریقای جنوبی نشان داد رفع تحریم یکی از مؤثرترین محرک‌هاست
    \item \textbf{بستهٔ کمک هدفمند} (ماه ۳-۱۲): نه مثل عراق (پول‌پاشی بدون نظارت) بلکه مثل لهستان (\lr{PHARE} مشروط)
    \item \textbf{اصلاحات ساختاری تدریجی} (سال ۱-۵): مدل شیلی (ثبات + آزادسازی تدریجی) نه شوک لهستانی
    \item \textbf{شبکهٔ ایمنی اجتماعی}: در همهٔ نمونه‌هایی که اقشار آسیب‌پذیر فراموش شدند، مردم از دموکراسی سرخورده شدند
\end{enumerate}
\end{lessonlearned}

\sectiondivider

%═══════════════════════════════════════════════════════════
\section{جدول نهم: نتایج سیاسی و شاخص‌های فعلی}
\label{app:a:table9}
%═══════════════════════════════════════════════════════════

\begin{landscape}
\pagestyle{empty}
\bigtablefontsize

\begin{longtable}{
  >{\raggedleft\arraybackslash}p{2.1cm}|
  >{\raggedleft\arraybackslash}p{1.7cm}
  >{\raggedleft\arraybackslash}p{1.7cm}
  >{\raggedleft\arraybackslash}p{1.7cm}
  >{\raggedleft\arraybackslash}p{1.7cm}
  >{\raggedleft\arraybackslash}p{1.7cm}
  >{\raggedleft\arraybackslash}p{1.7cm}
  >{\raggedleft\arraybackslash}p{1.7cm}
  >{\raggedleft\arraybackslash}p{1.7cm}
  >{\raggedleft\arraybackslash}p{1.7cm}
}
\caption{نتایج سیاسی و شاخص‌های فعلی (۲۰۲۳-۲۰۲۴)}
\label{tab:app-a-results} \\

\toprule
\headerrow
\rot{\textbf{شاخص}} &
\rot{\textbf{اسپانیا}} &
\rot{\textbf{لهستان}} &
\rot{\textbf{شیلی}} &
\rot{\textbf{آفریقای جنوبی}} &
\rot{\textbf{اندونزی}} &
\rot{\textbf{تیمور شرقی}} &
\rot{\textbf{عراق}} &
\rot{\textbf{تونس}} &
\rot{\textbf{میانمار}} \\
\midrule
\endfirsthead

\multicolumn{10}{c}{\small\textit{ادامهٔ جدول \ref{tab:app-a-results}: نتایج سیاسی}} \\
\toprule
\headerrow
\rot{\textbf{شاخص}} &
\rot{\textbf{اسپانیا}} &
\rot{\textbf{لهستان}} &
\rot{\textbf{شیلی}} &
\rot{\textbf{آفریقای جنوبی}} &
\rot{\textbf{اندونزی}} &
\rot{\textbf{تیمور شرقی}} &
\rot{\textbf{عراق}} &
\rot{\textbf{تونس}} &
\rot{\textbf{میانمار}} \\
\midrule
\endhead

\bottomrule
\endlastfoot

\textbf{تعداد انتقال مسالمت‌آمیز قدرت} &
\cellgreen{۸+} &
\cellgreen{۶+} &
\cellgreen{۶+} &
\cellgreen{۴} &
\cellgreen{۴} &
\cellorange{۳} &
\cellorange{۲ (با بحران)} &
\cellred{۱ (سپس کودتا)} &
\cellred{۱ (سپس کودتا)} \\

\altrow
\textbf{وضعیت \lr{Freedom House} (۲۰۲۴)} &
\cellgreen{آزاد (۱.۰)} &
\cellgreen{آزاد (۲.۰)} &
\cellgreen{آزاد (۱.۰)} &
\cellgreen{آزاد (۲.۰)} &
\cellorange{نیمه‌آزاد (۳.۰)} &
\cellorange{نیمه‌آزاد (۳.۰)} &
\cellred{غیرآزاد (۵.۵)} &
\cellred{غیرآزاد (۳.۵)} &
\cellred{غیرآزاد (۶.۵)} \\

\textbf{وضعیت آزادی مطبوعات (\lr{RSF} ۲۰۲۴)} &
\cellgreen{رتبهٔ ۳۶} &
\cellorange{رتبهٔ ۴۷} &
\cellgreen{رتبهٔ ۵۲} &
\cellorange{رتبهٔ ۵۵} &
\cellorange{رتبهٔ ۱۱۱} &
\cellorange{رتبهٔ ۱۰۵} &
\cellred{رتبهٔ ۱۶۹} &
\cellred{رتبهٔ ۱۱۸} &
\cellred{رتبهٔ ۱۷۱} \\

\altrow
\textbf{وضعیت \lr{TI CPI} (فساد) ۲۰۲۳} &
\cellgreen{۶۰} &
\cellorange{۵۴} &
\cellgreen{۶۷} &
\cellorange{۴۱} &
\cellorange{۳۴} &
\cellorange{۴۲} &
\cellred{۲۳} &
\cellorange{۴۰} &
\cellred{۲۰} \\

\textbf{درصد زنان در پارلمان} &
\cellgreen{۴۴٪} &
\cellorange{۲۹٪} &
\cellgreen{۳۵٪} &
\cellorange{۲۸٪} &
\cellorange{۲۲٪} &
\cellgreen{۳۸٪} &
\cellorange{۲۶٪} &
\cellorange{۲۶٪} &
--- \\

\altrow
\textbf{کودتا یا بازگشت اقتدارگرایی} &
\cmark نافرجام (۲۳-F ۱۹۸۱) &
$\sim$ (افت دموکراتیک \lr{PiS} ۲۰۱۵-۲۰۲۳) &
\xmark &
\xmark (اما فساد \lr{ANC}) &
\xmark (اما افت اخیر) &
$\sim$ (بحران ۲۰۰۶) &
$\sim$ (بی‌ثباتی مزمن) &
\cmark (سعید ۲۰۲۱) &
\cmark (کودتای ۲۰۲۱) \\

\textbf{وضعیت فعلی نظام سیاسی} &
\cellgreen{دموکراسی تحکیم‌یافتهٔ پارلمانی} &
\cellgreen{دموکراسی بازیابی‌شده (پس از ۲۰۲۳)} &
\cellgreen{دموکراسی تحکیم‌یافته} &
\cellgreen{دموکراسی با چالش} &
\cellorange{دموکراسی انتخاباتی ناقص} &
\cellorange{دموکراسی شکننده} &
\cellred{اقتدارگرایی رقابتی} &
\cellred{اقتدارگرایی نوین} &
\cellred{خونتای نظامی + جنگ داخلی} \\

\altrow
\textbf{آیا «تحکیم» محقق شد؟} &
\cmark (قطعی) &
\cmark (با نوسان) &
\cmark (قطعی) &
\cmark (با چالش) &
$\sim$ (ناقص) &
$\sim$ (شکننده) &
\xmark &
\xmark (بازگشت) &
\xmark (بازگشت) \\

\end{longtable}
\end{landscape}

\sectiondivider

%═══════════════════════════════════════════════════════════
\section{جدول دهم: نگاشت سناریوها به نمونه‌ها}
\label{app:a:table10}
%═══════════════════════════════════════════════════════════

\begin{table}[htbp]
\centering
\caption{نگاشت سناریوهای گذار ایران (فصل ۴) به نمونه‌های تاریخی}
\label{tab:app-a-mapping}
\begin{tabularx}{\textwidth}{>{\raggedleft\arraybackslash}p{3cm}
                             >{\raggedleft\arraybackslash}p{3cm}
                             >{\raggedleft\arraybackslash}X
                             >{\centering\arraybackslash}p{2cm}}
\toprule
\headerrow سناریوی ایران & نمونهٔ نزدیک‌ترین & درس اصلی & نتیجهٔ نمونه \\
\midrule
A: فروپاشی ناگهانی & اندونزی ۱۹۹۸ + لیبی ۲۰۱۱ & سرعت + خلأ = خطر شدید؛ نیاز به آمادگی قبلی & \statuswarn مختلط \\
\altrow B: مذاکره‌ای (مطلوب) & آفریقای جنوبی ۱۹۹۰ + اسپانیا ۱۹۷۵ & صبر + فراگیری + آشتی = بهترین نتیجه & \statusok موفق \\
C: انقلاب مردمی & تونس ۲۰۱۱ + اندونزی ۱۹۹۸ & سرعت ← نهادسازی سریع ضروری؛ خطر بازگشت & \statuswarn مختلط \\
\altrow D: تحول از درون & اسپانیا ۱۹۷۵ + میانمار ۲۰۱۰ & اصلاح‌طلب واقعی لازم (خوان‌کارلوس ≠ تاتمادو) & \statuswarn خطرناک \\
E: مداخلهٔ نظامی (رد) & عراق ۲۰۰۳ & \emphred{به هیچ وجه تکرار نشود} & \statusbad فاجعه \\
\altrow F: بحران ممتد & ونزوئلا + لبنان & خستگی = فرسایش ← فرصت از دست رفته & \statusbad بن‌بست \\
\bottomrule
\end{tabularx}
\end{table}

\begin{keypoint}
\textbf{مهم‌ترین یافتهٔ نگاشت:} سناریوی B (مذاکره‌ای) تنها سناریویی است که نمونه‌های تاریخی آن (آفریقای جنوبی، اسپانیا، لهستان) همگی به دموکراسی تحکیم‌یافته رسیده‌اند. سایر سناریوها نتایج مختلط یا منفی دارند. این تأیید مجدد اولویت مدل مذاکره‌ای است — هرچند آمادگی برای سناریوهای دیگر ضروری است (\seeChapter{ch:scenarios}).
\end{keypoint}

\sectiondivider

%═══════════════════════════════════════════════════════════
\section{نمودار حبابی: هزینه در مقابل نتیجه}
\label{app:a:bubble}
%═══════════════════════════════════════════════════════════

\begin{figure}[htbp]
\centering
\begin{tikzpicture}
\begin{axis}[
    width=14cm,
    height=10cm,
    xlabel={هزینهٔ بین‌المللی (میلیارد دلار)},
    ylabel={شاخص دموکراسی \lr{V-Dem} (۲۰۲۳)},
    xmin=-2, xmax=70,
    ymin=0, ymax=1,
    xtick={0,10,20,30,40,50,60},
    ytick={0,0.2,0.4,0.6,0.8,1.0},
    x tick label style={font=\footnotesize},
    y tick label style={font=\footnotesize},
    xlabel style={font=\small},
    ylabel style={font=\small},
    legend pos=outer north east,
    legend style={font=\tiny, draw=none, fill=none},
    grid=major,
    grid style={gray!20},
    clip=false,
]

% حباب‌ها: (هزینه بین‌المللی میلیارد$, V-Dem Score, اندازه=جمعیت/۱۰M)
% اسپانیا
\addplot[
  only marks, mark=*, mark size=5pt,
  color=MainBlue, fill=MainBlue!40
] coordinates {(2, 0.81)};
\addlegendentry{اسپانیا}

% لهستان
\addplot[
  only marks, mark=*, mark size=5pt,
  color=MainBlue!70, fill=MainBlue!25
] coordinates {(5, 0.55)};
\addlegendentry{لهستان}

% شیلی
\addplot[
  only marks, mark=*, mark size=3pt,
  color=MainGreen, fill=MainGreen!40
] coordinates {(1, 0.79)};
\addlegendentry{شیلی}

% آفریقای جنوبی
\addplot[
  only marks, mark=*, mark size=5.5pt,
  color=MainGreen!70, fill=MainGreen!25
] coordinates {(3, 0.72)};
\addlegendentry{آفریقای جنوبی}

% اندونزی
\addplot[
  only marks, mark=*, mark size=8pt,
  color=MainOrange, fill=MainOrange!30
] coordinates {(8, 0.52)};
\addlegendentry{اندونزی}

% تیمور شرقی
\addplot[
  only marks, mark=*, mark size=2pt,
  color=MainPurple, fill=MainPurple!30
] coordinates {(5, 0.58)};
\addlegendentry{تیمور شرقی}

% عراق
\addplot[
  only marks, mark=*, mark size=5pt,
  color=MainRed, fill=MainRed!30
] coordinates {(60, 0.25)};
\addlegendentry{عراق}

% تونس
\addplot[
  only marks, mark=*, mark size=3pt,
  color=MainYellow!80!black, fill=MainYellow!30
] coordinates {(1.5, 0.18)};
\addlegendentry{تونس}

% میانمار
\addplot[
  only marks, mark=*, mark size=6pt,
  color=MainRed!70, fill=MainRed!15
] coordinates {(2, 0.08)};
\addlegendentry{میانمار}

% ایران (پیش‌بینی — ستاره)
\addplot[
  only marks, mark=star, mark size=7pt,
  color=MainPurple, fill=MainPurple!50,
  line width=1.5pt
] coordinates {(4, 0.65)};
\addlegendentry{ایران (هدف مدل ۶)}

% خط هدف
\draw[dashed, MainGreen!60, thick] (axis cs:0,0.6) -- (axis cs:70,0.6);
\node[font=\tiny, MainGreen!80!black, anchor=west] at (axis cs:45,0.62) {آستانهٔ دموکراسی تحکیم‌یافته};

% ناحیهٔ «بازده بالا»
\fill[MainGreen!8] (axis cs:0,0.6) rectangle (axis cs:10,1);
\node[font=\tiny, MainGreen!60!black, rotate=90] at (axis cs:0.8,0.8) {بازده بالا / هزینهٔ پایین};

% ناحیهٔ «فاجعه»
\fill[MainRed!8] (axis cs:40,0) rectangle (axis cs:70,0.4);
\node[font=\tiny, MainRed!60!black] at (axis cs:55,0.15) {هزینهٔ بالا / شکست};

\end{axis}
\end{tikzpicture}
\caption{نمودار حبابی: هزینهٔ بین‌المللی در برابر نتیجهٔ دموکراتیک (اندازهٔ حباب $\propto$ جمعیت)}
\label{fig:app-a-bubble}
\end{figure}

\begin{keypoint}
\textbf{یافتهٔ کلیدی نمودار حبابی:} هیچ رابطهٔ مستقیمی بین \textbf{حجم هزینهٔ بین‌المللی} و \textbf{نتیجهٔ دموکراتیک} وجود ندارد. عراق با بیش از ۶۰ میلیارد دلار هزینهٔ بین‌المللی، بدترین نتیجه را داشت؛ شیلی با کمتر از ۱ میلیارد، بهترین. آنچه تعیین‌کننده است: \textbf{مالکیت ملی}، \textbf{مسیر مذاکره‌ای}، و \textbf{طراحی هوشمند نظارت}. بودجهٔ $۲.۵$-$۵$ میلیارد دلاری پیشنهادی برای ایران (\seeChapter{ch:budget}) در «ناحیهٔ بازده بالا» قرار می‌گیرد.
\end{keypoint}

\sectiondivider

%═══════════════════════════════════════════════════════════
\section{خلاصهٔ مقایسه‌ای: کارت امتیاز نُه نمونه}
\label{app:a:scorecard}
%═══════════════════════════════════════════════════════════

\begin{table}[htbp]
\centering
\caption{کارت امتیاز مقایسه‌ای: ارزیابی کلی نُه نمونه در ۸ بُعد}
\label{tab:app-a-scorecard}
\bigtablefontsize
\begin{tabularx}{\textwidth}{
  >{\raggedleft\arraybackslash}p{2.2cm}
  >{\centering\arraybackslash}p{1.1cm}
  >{\centering\arraybackslash}p{1.1cm}
  >{\centering\arraybackslash}p{1.1cm}
  >{\centering\arraybackslash}p{1.1cm}
  >{\centering\arraybackslash}p{1.1cm}
  >{\centering\arraybackslash}p{1.1cm}
  >{\centering\arraybackslash}p{1.1cm}
  >{\centering\arraybackslash}p{1.1cm}
  >{\centering\arraybackslash}p{1.3cm}
}
\toprule
\headerrow
\textbf{بُعد} &
\rot{\textbf{اسپانیا}} &
\rot{\textbf{لهستان}} &
\rot{\textbf{شیلی}} &
\rot{\textbf{آفریقا}} &
\rot{\textbf{اندونزی}} &
\rot{\textbf{تیمور}} &
\rot{\textbf{عراق}} &
\rot{\textbf{تونس}} &
\rot{\textbf{میانمار}} \\
\midrule

مسالمت‌آمیز بودن &
\starrating{4} &
\starrating{5} &
\starrating{4} &
\starrating{3} &
\starrating{3} &
\starrating{2} &
\starrating{1} &
\starrating{4} &
\starrating{3} \\

\altrow
توافق سیاسی &
\starrating{5} &
\starrating{5} &
\starrating{4} &
\starrating{5} &
\starrating{3} &
\starrating{3} &
\starrating{1} &
\starrating{4} &
\starrating{2} \\

عدالت انتقالی &
\starrating{1} &
\starrating{2} &
\starrating{4} &
\starrating{5} &
\starrating{1} &
\starrating{3} &
\starrating{1} &
\starrating{4} &
\starrating{0} \\

\altrow
اصلاح امنیتی &
\starrating{3} &
\starrating{4} &
\starrating{3} &
\starrating{4} &
\starrating{3} &
\starrating{3} &
\starrating{1} &
\starrating{3} &
\starrating{0} \\

قانون اساسی &
\starrating{5} &
\starrating{3} &
\starrating{3} &
\starrating{5} &
\starrating{3} &
\starrating{4} &
\starrating{2} &
\starrating{5} &
\starrating{1} \\

\altrow
فراگیری &
\starrating{3} &
\starrating{2} &
\starrating{2} &
\starrating{5} &
\starrating{2} &
\starrating{3} &
\starrating{1} &
\starrating{4} &
\starrating{2} \\

نتیجهٔ بلندمدت &
\starrating{5} &
\starrating{4} &
\starrating{5} &
\starrating{4} &
\starrating{3} &
\starrating{3} &
\starrating{1} &
\starrating{1} &
\starrating{1} \\

\altrow
انتقال به ایران &
\starrating{3} &
\starrating{3} &
\starrating{4} &
\starrating{5} &
\starrating{4} &
\starrating{3} &
\starrating{2} &
\starrating{3} &
\starrating{2} \\

\midrule
\headerrow
\textbf{مجموع (/۴۰)} &
\textbf{۲۹} &
\textbf{۲۸} &
\textbf{۲۹} &
\textbf{۳۶} &
\textbf{۲۲} &
\textbf{۲۴} &
\textbf{۱۰} &
\textbf{۲۸} &
\textbf{۱۱} \\

\bottomrule
\end{tabularx}
\end{table}

\begin{lessonlearned}
\textbf{آفریقای جنوبی با امتیاز ۳۶ از ۴۰}، جامع‌ترین الگوی قابل‌انتقال به ایران است. اسپانیا و شیلی (هر دو ۲۹) در بُعد نتیجهٔ بلندمدت برتر هستند اما در عدالت انتقالی (اسپانیا) یا فراگیری (شیلی) ضعیف‌تر. عراق (۱۰) و میانمار (۱۱) \textbf{ضد الگوهای اصلی} هستند. نکتهٔ حیاتی: حتی آفریقای جنوبی هم \textbf{مدل کامل نیست} — نابرابری اقتصادی و خشونت جنسیتی همچنان چالش‌های جدی هستند.
\end{lessonlearned}

\sectiondivider

%═══════════════════════════════════════════════════════════
\section{ده یافتهٔ کلان مقایسه‌ای}
\label{app:a:findings}
%═══════════════════════════════════════════════════════════

بر اساس تحلیل جامع نُه نمونه در ده بُعد، ده یافتهٔ کلان استخراج شده است:

\begin{enumerate}
\item \textbf{مسیر گذار تعیین‌کنندهٔ نتیجه است:} گذارهای \textbf{مذاکره‌ای} (اسپانیا، لهستان، آفریقای جنوبی) نتایج پایدارتری از گذارهای \textbf{یک‌جانبه} (مداخله، فروپاشی) داشته‌اند. احتمال موفقیت گذار مذاکره‌ای: $\sim$۷۵٪ در مقابل $\sim$۲۵٪ برای مداخله.

\item \textbf{اپوزیسیون سازمان‌یافته شرط لازم است:} در هر پنج نمونهٔ موفق، اپوزیسیون قبل از گذار \textbf{حداقل یک سازمان فراگیر} داشت (\lr{Solidarność}، \lr{ANC}، \lr{Concertación}). \emphorange{هشدار ایرانی:} اپوزیسیون فعلی ایران فاقد این ویژگی است.

\item \textbf{انحلال نیروهای امنیتی = فاجعه:} عراق این را اثبات کرد. مدل درست: \textbf{ادغام + تفکیک اقتصادی + نظارت مدنی تدریجی}. زمان مورد نیاز: ۵ تا ۱۵ سال.

\item \textbf{عدالت انتقالی بدون حقیقت، عدالت نیست:} مدل اسپانیایی (عفو بدون حقیقت) و مدل عراقی (انتقام بدون عدالت) هر دو شکست خوردند. مدل \lr{TRC} بهینه‌ترین است.

\item \textbf{قانون اساسی فراگیر سنگ بنای تحکیم است:} بهترین نتایج در کشورهایی حاصل شد که قانون اساسی از طریق \textbf{مجلس مؤسسان منتخب و فراگیر} نوشته شد (آفریقای جنوبی، تونس).

\item \textbf{مشوق اقتصادی بین‌المللی تسریع‌کننده است:} عضویت در \lr{EC/EU} (اسپانیا، لهستان) قوی‌ترین مشوق بود. \emphorange{چالش ایرانی:} معادل \lr{EU} برای ایران وجود ندارد؛ باید بسته‌ای ترکیبی (لغو تحریم + سرمایه‌گذاری + عضویت \lr{WTO}) طراحی شود.

\item \textbf{زنان و اقلیت‌ها: سهمیهٔ قانونی لازم است:} آفریقای جنوبی و تونس نشان دادند که بدون \textbf{سهمیهٔ قانونی}، مشارکت زنان و اقلیت‌ها در ساختارهای جدید محدود می‌ماند.

\item \textbf{نقش بین‌المللی مؤثر ≠ مدیریت مستقیم:} بهترین نتایج زمانی حاصل شد که نقش بین‌المللی \textbf{حمایتی-نظارتی} بود (مدل ۳-۴ از فصل ۳)، نه مدیریت مستقیم (مدل ۵). مدل ۶ پیشنهادی این درس را رعایت می‌کند.

\item \textbf{بُعد هسته‌ای ایران بی‌سابقه است:} تنها آفریقای جنوبی تجربهٔ خلع‌سلاح هسته‌ای داوطلبانه را داشت (۶ کلاهک، ۱۹۹۳). ایران با برنامه‌ای بسیار پیچیده‌تر، نیازمند \textbf{مکانیزم ویژه} است (\seeChapter{ch:guarantees}).

\item \textbf{هزینهٔ پیشگیری بسیار کمتر از هزینهٔ شکست است:} مقایسهٔ شیلی ($<$\$۱B) با عراق ($>$\$۶۰B بین‌المللی + \$۲T آمریکا) نشان می‌دهد که \textbf{سرمایه‌گذاری در مدل درست}، هزارها برابر ارزان‌تر از \textbf{مدیریت شکست} است.
\end{enumerate}

\sectiondivider

%═══════════════════════════════════════════════════════════
\section{نمودار تکمیلی: عوامل موفقیت و شکست}
\label{app:a:success-factors}
%═══════════════════════════════════════════════════════════

\begin{figure}[htbp]
\centering
\begin{tikzpicture}[
  node distance=0.6cm,
  factor/.style={
    draw, rounded corners=3pt, minimum width=6cm,
    minimum height=0.7cm, font=\small, align=center
  },
  success/.style={factor, fill=MainGreen!15, draw=MainGreen!60, text=DarkGray},
  failure/.style={factor, fill=MainRed!15, draw=MainRed!60, text=DarkGray},
  title/.style={font=\bfseries\small, anchor=south}
]

% عنوان‌ها
\node[title, MainGreen] at (-4, 6.5) {عوامل موفقیت (الگوها)};
\node[title, MainRed] at (4, 6.5) {عوامل شکست (ضد الگوها)};

% خط جداکننده
\draw[gray!40, thick, dashed] (0,-0.5) -- (0,6.5);

% عوامل موفقیت (چپ)
\node[success] at (-4, 5.8) {۱. گذار مذاکره‌ای (اسپانیا، آفریقای جنوبی)};
\node[success] at (-4, 5.0) {۲. اپوزیسیون سازمان‌یافته (لهستان، شیلی)};
\node[success] at (-4, 4.2) {۳. ادغام تدریجی نیروها (آفریقای جنوبی)};
\node[success] at (-4, 3.4) {۴. کمیسیون حقیقت (\lr{TRC})};
\node[success] at (-4, 2.6) {۵. قانون اساسی فراگیر (آفریقای جنوبی، تونس)};
\node[success] at (-4, 1.8) {۶. مشوق اقتصادی خارجی (اسپانیا/\lr{EC})};
\node[success] at (-4, 1.0) {۷. سهمیهٔ زنان/اقلیت‌ها (آفریقای جنوبی)};
\node[success] at (-4, 0.2) {۸. نظارت بین‌المللی حمایتی (مدل ۳-۴)};

% عوامل شکست (راست)
\node[failure] at (4, 5.8) {۱. مداخلهٔ نظامی خارجی (عراق)};
\node[failure] at (4, 5.0) {۲. انحلال کامل ارتش (عراق)};
\node[failure] at (4, 4.2) {۳. اجتثاث بدون عدالت (عراق)};
\node[failure] at (4, 3.4) {۴. عفو بدون حقیقت (اسپانیا)};
\node[failure] at (4, 2.6) {۵. گشایش صوری (میانمار)};
\node[failure] at (4, 1.8) {۶. بدون \lr{SSR} واقعی (میانمار)};
\node[failure] at (4, 1.0) {۷. فقدان اپوزیسیون منسجم (عراق)};
\node[failure] at (4, 0.2) {۸. مدیریت مستقیم خارجی (عراق، تیمور)};

\end{tikzpicture}
\caption{مقایسهٔ بصری ۸ عامل موفقیت و ۸ عامل شکست از نُه نمونهٔ تاریخی}
\label{fig:app-a-factors}
\end{figure}

\sectiondivider

%═══════════════════════════════════════════════════════════
\section{جمع‌بندی پیوست}
\label{app:a:conclusion}
%═══════════════════════════════════════════════════════════

\begin{chaptersummary}
جمع‌بندی پیوست الف — مقایسهٔ جامع نُه نمونه:

\begin{enumerate}[nosep]
\item نُه نمونه در ده بُعد و بیش از ۶۰ شاخص مقایسه شدند.
\item \textbf{آفریقای جنوبی} (امتیاز ۳۶/۴۰) جامع‌ترین الگوی قابل‌انتقال به ایران است.
\item \textbf{عراق} و \textbf{میانمار} مهم‌ترین ضد الگوها هستند.
\item مسیر مذاکره‌ای، اپوزیسیون سازمان‌یافته، و اصلاح تدریجی امنیتی سه عامل کلیدی موفقیت‌اند.
\item مداخلهٔ نظامی، انحلال ارتش، و اجتثاث افراطی سه عامل کلیدی شکست‌اند.
\item مدل ۶ (ترکیبی-تطبیقی) از بهترین عناصر هر نمونه طراحی شده است.
\item بُعد هسته‌ای و ساختار سپاه، ایران را منحصربه‌فرد اما نه استثنا می‌سازد.
\item رابطهٔ معناداری بین حجم هزینهٔ بین‌المللی و نتیجهٔ دموکراتیک وجود ندارد؛ \textbf{طراحی هوشمند} تعیین‌کننده است.
\item هر سناریوی گذار ایران (فصل ۴) نمونهٔ تاریخی مرتبط خود را دارد (جدول ۱۰).
\item جزئیات هر نمونه در پیوست‌های ب تا ح آمده است.
\end{enumerate}

\vspace{0.3cm}
\textit{برای مطالعهٔ تفصیلی هر نمونه:}
\begin{itemize}[nosep]
\item آفریقای جنوبی: \seeChapter{app:south-africa}
\item شیلی: \seeChapter{app:chile}
\item تونس: \seeChapter{app:tunisia}
\item لهستان و اروپای شرقی: \seeChapter{app:poland}
\item عراق (ضد الگو): \seeChapter{app:iraq}
\item میانمار (گذار ناتمام): \seeChapter{app:myanmar}
\item تیمور شرقی: \seeChapter{app:timor}
\end{itemize}
\end{chaptersummary}

\chapterend

%══════════════════════════════════════════════════════════════
% پایان پیوست الف
%══════════════════════════════════════════════════════════════
%══════════════════════════════════════════════════════════════
% پیوست ب: مطالعه موردی آفریقای جنوبی
% فایل: appendices/app-b-south-africa.tex
% حجم هدف: ۸-۱۰ صفحه
%══════════════════════════════════════════════════════════════

\chapter{مطالعهٔ موردی: آفریقای جنوبی (۱۹۹۰-۱۹۹۹)}
\label{app:south-africa}

\begin{executivesummary}
آفریقای جنوبی یکی از موفق‌ترین نمونه‌های گذار دموکراتیک در نیمهٔ دوم قرن بیستم و \textbf{جامع‌ترین الگوی قابل‌انتقال} به پروندهٔ ایران است. گذار از نظام \termfn{آپارتاید}{Apartheid} (۱۹۴۸-۱۹۹۴) به دموکراسی چندنژادی، از طریق \textbf{مذاکرهٔ چندجانبه} (\lr{CODESA})، \textbf{قانون اساسی فراگیر}، \textbf{کمیسیون حقیقت و آشتی} (\lr{TRC})، و \textbf{ادغام نیروهای مسلح} صورت گرفت. این پیوست هفت بُعد کلیدی این تجربه را تحلیل و درس‌آموخته‌های آن را برای ایران استخراج می‌کند. وجوه مشابهت: ایدئولوژی رسمی سرکوبگر، تحریم‌های بین‌المللی، نیروهای امنیتی قدرتمند، تنوع قومی-زبانی، و دیاسپورای فعال.
\end{executivesummary}

%═══════════════════════════════════════════════════════════
\section{زمینه و بافت تاریخی}
\label{app:sa:context}
%═══════════════════════════════════════════════════════════

\subsection{نظام آپارتاید: ساختار و ویژگی‌ها}

نظام \termfn{آپارتاید}{Apartheid} (به معنای «جدایی» در زبان آفریکانس) از ۱۹۴۸ تا ۱۹۹۴ حاکم بود و جامعهٔ آفریقای جنوبی را بر اساس نژاد به چهار دستهٔ سلسله‌مراتبی تقسیم کرده بود:

\begin{itemize}[nosep]
\item \textbf{سفیدپوستان} (۱۳٪ جمعیت): انحصار کامل قدرت سیاسی و اقتصادی
\item \textbf{رنگین‌پوستان (\lr{Coloured})} (۹٪): حقوق محدود
\item \textbf{آسیایی‌تبارها} (۳٪): حقوق محدود
\item \textbf{سیاه‌پوستان} (۷۵٪): محرومیت کامل از حقوق شهروندی
\end{itemize}

\begin{table}[htbp]
\centering
\caption{مشخصات آفریقای جنوبی در آستانهٔ گذار (۱۹۹۰)}
\label{tab:app-sa-profile}
\begin{tabularx}{\textwidth}{>{\raggedleft\arraybackslash}p{4.5cm} >{\raggedleft\arraybackslash}X}
\toprule
\headerrow \textbf{شاخص} & \textbf{مقدار} \\
\midrule
جمعیت & ۴۰ میلیون نفر \\
\altrow مساحت & ۱,۲۲۰,۰۰۰ \lr{km²} \\
تنوع زبانی & ۱۱ زبان رسمی \\
\altrow \lr{GDP per capita} & $\sim$\$۳,۵۰۰ \\
ضریب جینی & ۰.۶۳ (یکی از بالاترین جهان) \\
\altrow نرخ بیکاری & $\sim$۳۰٪ (سیاه‌پوستان: $\sim$۴۵٪) \\
طول عمر رژیم آپارتاید & ۴۶ سال (۱۹۴۸-۱۹۹۴) \\
\altrow اندازهٔ نیروهای مسلح (\lr{SADF}) & $\sim$۱۰۰,۰۰۰ \\
برنامهٔ هسته‌ای & ۶ کلاهک (خلع‌سلاح ۱۹۸۹-۱۹۹۳) \\
\altrow تحریم‌های بین‌المللی & جامع (تسلیحاتی ۱۹۷۷ + اقتصادی ۱۹۸۶) \\
\bottomrule
\end{tabularx}
\end{table}

\subsection{ستون‌های قدرت نظام آپارتاید}

نظام آپارتاید بر پنج ستون استوار بود — مشابهت‌های هر ستون با جمهوری اسلامی در ادامه تحلیل خواهد شد:

\begin{enumerate}[nosep]
\item \textbf{ایدئولوژی نژادی-مذهبی:} کالوینیسم آفریکانر + نظریهٔ «توسعهٔ جداگانه» (\lr{Separate Development}) — مشابه ایدئولوژی ولایت فقیه
\item \textbf{نیروهای امنیتی:} \org{نیروی دفاعی آفریقای جنوبی}{\lr{SADF}} + پلیس امنیتی (\lr{SAP}) + نیروهای ویژه — مشابه سپاه + بسیج + اطلاعات
\item \textbf{نظام حقوقی تبعیض‌آمیز:} بیش از ۳۰۰ قانون نژادپرستانه — مشابه قوانین تبعیض جنسیتی/مذهبی
\item \textbf{کنترل اقتصادی:} شرکت‌های دولتی-نظامی (\lr{ARMSCOR}) + معادن — مشابه خاتم‌الانبیاء + بنیادها
\item \textbf{سرکوب جامعهٔ مدنی:} ممنوعیت \lr{ANC/PAC} + سانسور + حکومت نظامی (\lr{State of Emergency} ۱۹۸۵-۱۹۹۰) — مشابه سرکوب جنبش‌های اعتراضی
\end{enumerate}

\begin{casestudy}
\textbf{مقایسهٔ ساختاری آپارتاید و جمهوری اسلامی:}
هر دو نظام بر \textbf{ایدئولوژی رسمی تبعیض‌آمیز} بنا شده‌اند (نژادی در آفریقا، مذهبی-جنسیتی در ایران). هر دو \textbf{نیروهای امنیتی موازی} با منافع اقتصادی دارند. هر دو تحت \textbf{تحریم‌های بین‌المللی} قرار گرفتند. تفاوت کلیدی: آپارتاید اقلیتِ حاکم بر اکثریت بود (۱۳٪ بر ۸۷٪)؛ در ایران نخبگان حاکم لزوماً اقلیت قومی/نژادی نیستند. تفاوت دوم: آفریقای جنوبی فاقد مسئلهٔ برنامهٔ هسته‌ای فعال در \textbf{زمان} گذار بود (خلع‌سلاح \textbf{قبل} از مذاکرات).
\end{casestudy}

\sectiondivider

%═══════════════════════════════════════════════════════════
\section{مسیر گذار: از بحران تا مذاکره}
\label{app:sa:transition}
%═══════════════════════════════════════════════════════════

\subsection{محرک‌های گذار (۱۹۸۵-۱۹۹۰)}

ترکیبی از فشارهای داخلی و خارجی رژیم آپارتاید را به مذاکره واداشت:

\begin{table}[htbp]
\centering
\caption{محرک‌های چندگانهٔ گذار آفریقای جنوبی}
\label{tab:app-sa-drivers}
\begin{tabularx}{\textwidth}{>{\centering\arraybackslash}p{2.5cm} >{\raggedleft\arraybackslash}X >{\centering\arraybackslash}p{2cm}}
\toprule
\headerrow \textbf{نوع فشار} & \textbf{توضیح} & \textbf{شدت} \\
\midrule
اقتصادی-بین‌المللی & تحریم‌های جامع (اقتصادی ۱۹۸۶ + تسلیحاتی ۱۹۷۷) + خروج سرمایه + کاهش ارزش رَند & \riskhigh \\
\altrow مدنی-داخلی & اعتصابات عمومی + شورش‌های تاونشیپ + کمپین نافرمانی + \lr{UDF} & \riskhigh \\
مسلحانه & عملیات‌های \lr{MK} (بازوی نظامی \lr{ANC}) + تشدید ناامنی & \riskmedium \\
\altrow ژئوپلیتیکی & فروپاشی شوروی = حذف بهانهٔ ضدکمونیستی + استقلال نامیبیا & \riskhigh \\
جمعیت‌شناختی & رشد سریع جمعیت سیاه‌پوست + فشار بر منابع & \riskmedium \\
\altrow شکاف نخبگان & اصلاح‌طلبان درون حزب ملی (\person{دکلرک}{\lr{F.W. de Klerk}}) & \riskhigh \\
\bottomrule
\end{tabularx}
\end{table}

\subsection{لحظهٔ تاریخی: ۲ فوریه ۱۹۹۰}

\person{اف.دبلیو. دکلرک}{\lr{F.W. de Klerk}}، رئیس‌جمهور جدید حزب ملی، در ۲ فوریه ۱۹۹۰ سخنرانی تاریخی خود در پارلمان ایراد کرد و اعلام نمود:

\begin{itemize}[nosep]
\item آزادی \person{نلسون ماندلا}{\lr{Nelson Mandela}} و سایر زندانیان سیاسی
\item رفع ممنوعیت \lr{ANC}، \lr{PAC}، حزب کمونیست و ۳۰+ سازمان
\item لغو حکومت نظامی
\item آغاز مذاکرات برای قانون اساسی جدید
\end{itemize}

\begin{keypoint}
\textbf{درس کلیدی:} دکلرک از موضع \textbf{قدرت نسبی} (نه فروپاشی) مذاکره را آغاز کرد. ارتش هنوز قوی بود، اما او فهمید که \textbf{هزینهٔ ادامهٔ وضع موجود} بیش از هزینهٔ مذاکره است. این مشابه سناریوی \lr{D} (تحول از درون) یا \lr{B} (مذاکره‌ای) برای ایران است (\seeChapter{ch:scenarios}).
\end{keypoint}

\subsection{گاه‌شمار گذار}

\begin{table}[htbp]
\centering
\caption{گاه‌شمار کلیدی گذار آفریقای جنوبی (۱۹۹۰-۱۹۹۹)}
\label{tab:app-sa-timeline}
\begin{tabularx}{\textwidth}{>{\centering\arraybackslash}p{2.5cm} >{\raggedleft\arraybackslash}X >{\centering\arraybackslash}p{2.2cm}}
\toprule
\headerrow \textbf{تاریخ} & \textbf{رویداد} & \textbf{فاز معادل ایران} \\
\midrule
فوریه ۱۹۹۰ & سخنرانی دکلرک + آزادی ماندلا (۱۱ فوریه) & فاز ۱ (تثبیت) \\
\altrow دسامبر ۱۹۹۱ & آغاز \lr{CODESA I} (۱۹ حزب + سازمان) & فاز ۱-۲ \\
مه ۱۹۹۲ & \lr{CODESA II}: بن‌بست بر سر وتو & فاز ۲ (بحران) \\
\altrow ژوئن ۱۹۹۲ & کشتار بویپاتونگ + \lr{ANC} مذاکرات را ترک کرد & بحران \\
سپتامبر ۱۹۹۲ & توافق ریکورد (\lr{Record of Understanding}) بین دکلرک و ماندلا & بازگشت به مذاکره \\
\altrow آوریل ۱۹۹۳ & ترور \person{کریس هانی}{\lr{Chris Hani}} — ماندلا آرامش خواست & آزمون بحران \\
نوامبر ۱۹۹۳ & تصویب قانون اساسی موقت (فصل ۱۵: عفو) & فاز ۲ \\
\altrow آوریل ۱۹۹۴ & \textbf{اولین انتخابات آزاد} (۲۷ آوریل): مشارکت ۸۶٪ & فاز ۲ (نقطهٔ عطف) \\
مه ۱۹۹۴ & ماندلا رئیس‌جمهور + دولت وحدت ملی & فاز ۲-۳ \\
\altrow دسامبر ۱۹۹۵ & تشکیل \lr{TRC} به ریاست اسقف \person{دزموند توتو}{\lr{Desmond Tutu}} & فاز ۳ \\
دسامبر ۱۹۹۶ & تصویب قانون اساسی دائمی (منشور حقوق) & فاز ۳ \\
\altrow اکتبر ۱۹۹۸ & ارائهٔ گزارش نهایی \lr{TRC} (۵ جلد) & فاز ۳ \\
ژوئن ۱۹۹۹ & \textbf{دومین انتخابات آزاد}: انتقال از ماندلا به مبکی & فاز ۴ (تحکیم) \\
\bottomrule
\end{tabularx}
\end{table}

\sectiondivider

%═══════════════════════════════════════════════════════════
\section{مذاکرات \lr{CODESA}: الگوی گفت‌وگوی ملی}
\label{app:sa:codesa}
%═══════════════════════════════════════════════════════════

\subsection{ساختار و شرکت‌کنندگان}

\org{مجمع یک آفریقای جنوبی دموکراتیک}{\lr{Convention for a Democratic South Africa (CODESA)}} در دو مرحله (۱۹۹۱ و ۱۹۹۲) برگزار شد و \textbf{۱۹ حزب و سازمان} در آن شرکت کردند:

\begin{table}[htbp]
\centering
\caption{ترکیب شرکت‌کنندگان \lr{CODESA} و معادل ایرانی}
\label{tab:app-sa-codesa}
\begin{tabularx}{\textwidth}{>{\raggedleft\arraybackslash}p{3.5cm} >{\raggedleft\arraybackslash}X >{\raggedleft\arraybackslash}p{3.5cm}}
\toprule
\headerrow \textbf{جریان آفریقایی} & \textbf{نقش} & \textbf{معادل ایرانی احتمالی} \\
\midrule
حزب ملی (\lr{NP}: دکلرک) & رژیم حاکم (اصلاح‌طلب) & بخش اصلاح‌طلب نظام \\
\altrow \lr{ANC} (ماندلا) & اپوزیسیون اصلی & ائتلاف فراگیر اپوزیسیون \\
حزب آزادی اینکاتا (\lr{IFP}: بوتلزی) & ملی‌گرای زولو & جریان‌های قومی/فدرالیست \\
\altrow حزب کمونیست (\lr{SACP}) & چپ & جریان چپ/سوسیالیست \\
\lr{PAC} & ملی‌گرای رادیکال & جریان‌های رادیکال \\
\altrow \lr{COSATU} & اتحادیهٔ کارگری & تشکل‌های کارگری مستقل \\
نمایندگان بانتوستان‌ها & خودمختاری‌های قومی & نمایندگان اقوام \\
\altrow احزاب لیبرال سفیدپوست (\lr{DP}) & میانه‌رو & اصلاح‌طلبان مستقل \\
نمایندگان مذهبی & کلیساها & نمایندگان ادیان و مذاهب \\
\altrow ناظران بین‌المللی & \lr{UN + OAU + EC} & \lr{UN + EU + AU} \\
\bottomrule
\end{tabularx}
\end{table}

\subsection{اصول کلیدی \lr{CODESA}}

پنج اصل بنیادین \lr{CODESA} که برای ایران الگو هستند:

\begin{enumerate}[nosep]
\item \textbf{اصل «فراگیری کافی» (\lr{Sufficient Inclusivity}):} هیچ جریان مهمی نباید از میز مذاکره حذف شود — حتی جریان‌های ناخوشایند
\item \textbf{اصل «اجماع کافی» (\lr{Sufficient Consensus}):} نه اتفاق آرا (غیرممکن)، بلکه توافق اکثریت قریب به اتفاق
\item \textbf{اصل «هم‌زمانی مذاکره و فشار» (\lr{Rolling Mass Action}):} \lr{ANC} هم‌زمان با مذاکره، اعتراضات خیابانی را ادامه داد
\item \textbf{اصل «غروب» (\lr{Sunset Clauses}):} تضمین‌های موقت برای کاهش ترس حاکمان (مثلاً: ۵ سال دولت وحدت ملی، امنیت شغلی کارمندان)
\item \textbf{اصل «دو مرحله‌ای بودن» (\lr{Two-Phase Constitution}):} ابتدا قانون اساسی موقت (۱۹۹۳)، سپس دائمی (۱۹۹۶)
\end{enumerate}

\begin{lessonlearned}
\textbf{اصل «بندهای غروب» (\lr{Sunset Clauses}) مهم‌ترین نوآوری آفریقای جنوبی} بود و توسط \person{جو اسلوو}{\lr{Joe Slovo}} (رهبر حزب کمونیست!) پیشنهاد شد. این بندها به سفیدپوستان \textbf{تضمین موقت} دادند: ۱) ادامهٔ خدمت کارمندان دولت به مدت ۵ سال، ۲) دولت وحدت ملی (دکلرک معاون ماندلا شد)، ۳) عدم مصادرهٔ اموال. بدون این تضمین‌ها، سفیدپوستان مذاکره نمی‌کردند. \emphorange{کاربرد ایرانی:} تضمین‌های مشابه برای بخش‌هایی از نظام فعلی که حاضر به همکاری شوند (\seeChapter{ch:guarantees}).
\end{lessonlearned}

\sectiondivider

%═══════════════════════════════════════════════════════════
\section{قانون اساسی: مدل مجلس مؤسسان فراگیر}
\label{app:sa:constitution}
%═══════════════════════════════════════════════════════════

\subsection{فرآیند دو مرحله‌ای}

\begin{enumerate}[nosep]
\item \textbf{قانون اساسی موقت (۱۹۹۳):} محصول مذاکرات \lr{CODESA} — ۳۴ اصل قانون اساسی (\lr{Constitutional Principles}) تعیین شد که قانون اساسی دائمی نمی‌توانست نقض کند
\item \textbf{قانون اساسی دائمی (۱۹۹۶):} توسط \textbf{مجلس مؤسسان} (= مجلس ملی + سنا) نوشته شد با مشارکت عمومی گسترده
\end{enumerate}

\subsection{مشارکت عمومی بی‌سابقه}

\begin{table}[htbp]
\centering
\caption{آمار مشارکت عمومی در تدوین قانون اساسی ۱۹۹۶}
\label{tab:app-sa-constitution-participation}
\begin{tabularx}{\textwidth}{>{\raggedleft\arraybackslash}p{5cm} >{\centering\arraybackslash}X}
\toprule
\headerrow \textbf{شاخص مشارکت} & \textbf{آمار} \\
\midrule
تعداد پیشنهادات مردمی دریافتی & بیش از ۲ میلیون \\
\altrow جلسات عمومی در سراسر کشور & ۲۶ کارگاه منطقه‌ای \\
مدت تدوین & ۲ سال (۱۹۹۴-۱۹۹۶) \\
\altrow تأیید دادگاه قانون اساسی & \cmark (بررسی انطباق با ۳۴ اصل) \\
تعداد زبان‌های رسمی در قانون اساسی & ۱۱ زبان \\
\altrow حقوق زنان & مادهٔ ۹: برابری کامل + ممنوعیت تبعیض جنسیتی \\
حقوق اقلیت‌ها & فصل ۲: منشور حقوق جامع (\lr{Bill of Rights}) \\
\bottomrule
\end{tabularx}
\end{table}

\subsection{ویژگی‌های منشور حقوق (\lr{Bill of Rights})}

\textbf{فصل دوم قانون اساسی ۱۹۹۶} یکی از پیشرفته‌ترین منشورهای حقوقی جهان است:

\begin{itemize}[nosep]
\item \textbf{حق حیات} (ممنوعیت اعدام — رأی دادگاه قانون اساسی ۱۹۹۵)
\item \textbf{برابری و ممنوعیت تبعیض} بر اساس نژاد، جنسیت، گرایش جنسی، مذهب، قومیت
\item \textbf{حقوق اجتماعی-اقتصادی:} حق مسکن، آب، غذا، تأمین اجتماعی، بهداشت
\item \textbf{حقوق زبانی:} ۱۱ زبان رسمی + هیئت زبان‌های آفریقای جنوبی (\lr{PanSALB})
\item \textbf{حقوق فرهنگی و مذهبی:} آزادی مذهب + حق آموزش به زبان مادری
\item \textbf{حقوق محیط‌زیستی:} نسل سوم حقوق بشر
\end{itemize}

\begin{recommendation}
\textbf{مدل قانون اساسی آفریقای جنوبی} بهترین الگو برای ایران در بُعد «فراگیری» است. قانون اساسی آیندهٔ ایران باید حداقل شامل: ۱) منشور حقوق جامع (شامل حقوق زنان، اقوام، اقلیت‌های مذهبی)، ۲) ۳+ زبان رسمی یا «زبان‌های ملی»، ۳) دادگاه قانون اساسی مستقل، ۴) حقوق اجتماعی-اقتصادی، ۵) فرآیند تدوین مشارکتی با دریافت پیشنهادات مردمی باشد (\seeChapter{ch:guarantees}).
\end{recommendation}

\sectiondivider

%═══════════════════════════════════════════════════════════
\section{کمیسیون حقیقت و آشتی (\lr{TRC})}
\label{app:sa:trc}
%═══════════════════════════════════════════════════════════

\subsection{ساختار و مأموریت}

\org{کمیسیون حقیقت و آشتی}{\lr{Truth and Reconciliation Commission (TRC)}} در دسامبر ۱۹۹۵ تشکیل شد و تا ۱۹۹۸ (گزارش نهایی ۲۰۰۳) فعالیت کرد:

\begin{table}[htbp]
\centering
\caption{ساختار و آمار \lr{TRC} آفریقای جنوبی}
\label{tab:app-sa-trc}
\begin{tabularx}{\textwidth}{>{\raggedleft\arraybackslash}p{5cm} >{\raggedleft\arraybackslash}X}
\toprule
\headerrow \textbf{شاخص} & \textbf{جزئیات} \\
\midrule
رئیس & اسقف اعظم \person{دزموند توتو}{\lr{Desmond Tutu}} \\
\altrow تعداد اعضا & ۱۷ کمیسیونر \\
سه کمیتهٔ فرعی & ۱) نقض حقوق بشر، ۲) عفو، ۳) غرامت و بازتوانی \\
\altrow تعداد شهادت‌های دریافتی & ۲۱,۰۰۰+ \\
تعداد جلسات علنی & ۸۰+ جلسه در سراسر کشور \\
\altrow درخواست‌های عفو & ۷,۱۱۲ درخواست \\
عفو اعطاشده & ۱,۵۲۷ (۲۱.۵٪) \\
\altrow عفو رد شده & ۵,۳۹۲ (۷۵.۸٪) \\
پخش زندهٔ تلویزیونی & \cmark (بی‌سابقه) \\
\altrow حجم گزارش نهایی & ۵ جلد + ۲ جلد تکمیلی (۲۰۰۳) \\
بودجه & $\sim$\$۱۸ میلیون دلار \\
\bottomrule
\end{tabularx}
\end{table}

\subsection{مکانیزم «عفو مشروط»}

نوآوری محوری \lr{TRC} مکانیزم \textbf{عفو مشروط} (\lr{Conditional Amnesty}) بود:

\begin{enumerate}[nosep]
\item عامل خشونت باید \textbf{شخصاً} درخواست عفو می‌کرد (نه عفو عمومی)
\item باید \textbf{اعتراف کامل و علنی} می‌کرد (نه پشت درهای بسته)
\item عمل باید \textbf{با انگیزهٔ سیاسی} بوده باشد (نه جنایت شخصی)
\item باید \textbf{تناسب} بین عمل و هدف سیاسی وجود می‌داشت
\item \textbf{قربانیان حق بودند} که حضور یابند و سؤال بپرسند
\item تصمیم توسط \textbf{هیئت قضایی مستقل} گرفته می‌شد
\end{enumerate}

\begin{keypoint}
\textbf{فلسفهٔ \lr{TRC}} بر مفهوم آفریقایی \termfn{اوبونتو}{\lr{Ubuntu}} — «من هستم چون ما هستیم» — استوار بود. این فلسفه نه \textbf{عفو بدون حقیقت} (مدل اسپانیا) و نه \textbf{انتقام بدون آشتی} (مدل نورنبرگ) را می‌پذیرفت، بلکه راه سومی ارائه کرد: \textbf{حقیقت در ازای عفو، آشتی در ازای اعتراف}.
\end{keypoint}

\subsection{نقد و محدودیت‌های \lr{TRC}}

با وجود موفقیت‌ها، \lr{TRC} با انتقادات جدی مواجه شد:

\begin{warningbox}
\textbf{محدودیت‌های \lr{TRC} که ایران باید بداند:}
\begin{enumerate}[nosep]
\item \textbf{غرامت ناکافی:} توصیه‌های کمیتهٔ غرامت عمدتاً اجرا نشد — متوسط پرداخت: فقط $\sim$\$۳,۹۰۰ به هر قربانی
\item \textbf{عدم تعقیب مجرمان عفو‌نشده:} دولت فاقد ارادهٔ سیاسی برای پیگیری بود
\item \textbf{تمرکز بر خشونت فیزیکی:} خشونت ساختاری آپارتاید (فقر، بیکاری، محرومیت) کمتر بررسی شد
\item \textbf{نابرابری اقتصادی پایدار:} \lr{TRC} نتوانست عدالت اقتصادی ایجاد کند — ضریب جینی هنوز ۰.۶۳ است
\item \textbf{«آشتی» سطحی:} بسیاری از قربانیان احساس عدالت نکردند
\end{enumerate}
\end{warningbox}

\sectiondivider

%═══════════════════════════════════════════════════════════
\section{اصلاح بخش امنیتی: مدل ادغام}
\label{app:sa:ssr}
%═══════════════════════════════════════════════════════════

\subsection{چالش: ادغام ۷ نیروی مسلح}

یکی از پیچیده‌ترین ابعاد گذار آفریقای جنوبی، ادغام نیروهای مسلح رقیب در یک ارتش واحد بود:

\begin{table}[htbp]
\centering
\caption{ادغام نیروهای مسلح در \lr{SANDF}}
\label{tab:app-sa-sandf}
\begin{tabularx}{\textwidth}{>{\raggedleft\arraybackslash}p{4cm} >{\centering\arraybackslash}p{2cm} >{\raggedleft\arraybackslash}X}
\toprule
\headerrow \textbf{نیرو} & \textbf{تعداد تقریبی} & \textbf{ماهیت} \\
\midrule
\lr{SADF} (ارتش رژیم) & ۹۰,۰۰۰ & حرفه‌ای، سفیدپوست‌محور \\
\altrow \lr{MK} (بازوی نظامی \lr{ANC}) & ۲۸,۰۰۰ & چریکی، تبعیدی \\
\lr{APLA} (بازوی نظامی \lr{PAC}) & ۶,۰۰۰ & چریکی \\
\altrow نیروهای بانتوستان‌ها (۴ ارتش) & ۱۱,۰۰۰ & نیمه‌حرفه‌ای \\
\lr{KZP} (پلیس کوازولو) & ۸,۰۰۰ & وابسته به \lr{IFP} \\
\midrule
\headerrow \textbf{\lr{SANDF} (نیروی جدید)} & \textbf{$\sim$۱۰۰,۰۰۰} & \textbf{ملی-فراگیر} \\
\bottomrule
\end{tabularx}
\end{table}

\subsection{استراتژی ادغام}

\begin{enumerate}[nosep]
\item \textbf{ادغام، نه انحلال:} برخلاف عراق، هیچ نیرویی منحل نشد
\item \textbf{نظارت مدنی:} وزیر دفاع غیرنظامی + نظارت پارلمانی
\item \textbf{کاهش تدریجی:} از ۱۴۳,۰۰۰ (مجموع) به $\sim$۱۰۰,۰۰۰
\item \textbf{آموزش مشترک:} نیروهای \lr{MK} آموزش حرفه‌ای دیدند
\item \textbf{ترکیب رهبری:} فرماندهی ترکیبی (ژنرال \person{گئورگ مرینگ}{\lr{Georg Meiring}} از \lr{SADF} + ژنرال \person{سیفیوه نیاندا}{\lr{Siphiwe Nyanda}} از \lr{MK})
\item \textbf{بازنشستگی افتخاری:} بستهٔ مالی برای نظامیانی که نمی‌خواستند بمانند
\end{enumerate}

\begin{lessonlearned}
\textbf{مدل ادغام آفریقای جنوبی} برای سپاه پاسداران ایران با تعدیل‌هایی قابل‌استفاده است: ۱) سپاه منحل نمی‌شود بلکه در ارتش ملی واحد ادغام می‌شود؛ ۲) فرماندهی ترکیبی (از هر دو طرف)؛ ۳) بُعد اقتصادی سپاه باید \textbf{جداگانه} مدیریت شود (مدل اندونزی، نه آفریقای جنوبی)؛ ۴) بستهٔ مالی بازنشستگی افتخاری برای کاهش مقاومت؛ ۵) زمان‌بندی: ۵-۱۰ سال (\seeChapter{ch:guarantees}).
\end{lessonlearned}

\subsection{خلع‌سلاح هسته‌ای: تنها الگوی موجود}

آفریقای جنوبی تنها کشور جهان است که \textbf{داوطلبانه} برنامهٔ سلاح هسته‌ای خود را کنار گذاشت:

\begin{table}[htbp]
\centering
\caption{خلع‌سلاح هسته‌ای آفریقای جنوبی و مقایسه با ایران}
\label{tab:app-sa-nuclear}
\begin{tabularx}{\textwidth}{>{\raggedleft\arraybackslash}p{4cm} >{\raggedleft\arraybackslash}X >{\raggedleft\arraybackslash}X}
\toprule
\headerrow \textbf{بُعد} & \textbf{آفریقای جنوبی} & \textbf{ایران (وضعیت فعلی)} \\
\midrule
تعداد کلاهک‌ها & ۶ عدد & -- (غنی‌سازی بالای ۶۰٪) \\
\altrow زمان خلع‌سلاح & ۱۹۸۹-۱۹۹۳ (قبل از گذار) & باید در فاز ۱-۲ تعیین شود \\
انگیزهٔ خلع‌سلاح & ترس از دست‌یابی \lr{ANC} + فشار بین‌المللی & پیش‌شرط لغو تحریم؟ \\
\altrow مکانیزم تأیید & بازرسی \lr{IAEA} (پس از الحاق به \lr{NPT} ۱۹۹۱) & پروتکل الحاقی + بازرسی‌های ویژه \\
نقش در مذاکرات & ابزار اعتمادسازی & شرط لازم برای حمایت بین‌المللی \\
\altrow درس کلیدی & خلع‌سلاح \textbf{قبل} از انتقال قدرت & ترجیحاً در فاز ۰-۱ \\
\bottomrule
\end{tabularx}
\end{table}

\begin{warningbox}
تفاوت حیاتی: دکلرک برنامهٔ هسته‌ای را \textbf{قبل از} مذاکرات خلع کرد تا سلاح به دست \lr{ANC} نیفتد. در ایران، جمهوری اسلامی ممکن است برعکس عمل کند و برنامهٔ هسته‌ای را \textbf{ابزار چانه‌زنی} قرار دهد. مدل ایرانی باید خلع هسته‌ای را به \textbf{بسته‌ای از مشوق‌ها} (لغو تحریم + سرمایه‌گذاری + تضمین امنیتی) گره بزند (\seeChapter{ch:guarantees}).
\end{warningbox}

\sectiondivider

%═══════════════════════════════════════════════════════════
\section{نقش بین‌المللی: نظارت حمایتی}
\label{app:sa:international}
%═══════════════════════════════════════════════════════════

\subsection{سطوح مداخلهٔ بین‌المللی}

نقش بین‌المللی در آفریقای جنوبی \textbf{حمایتی-نظارتی} بود (مدل ۳ از فصل ۳)، نه مدیریت مستقیم:

\begin{table}[htbp]
\centering
\caption{نقش بازیگران بین‌المللی در گذار آفریقای جنوبی}
\label{tab:app-sa-intl}
\begin{tabularx}{\textwidth}{>{\raggedleft\arraybackslash}p{3.5cm} >{\raggedleft\arraybackslash}X >{\centering\arraybackslash}p{2cm}}
\toprule
\headerrow \textbf{بازیگر} & \textbf{نقش} & \textbf{اثرگذاری} \\
\midrule
سازمان ملل (\lr{UNOMSA}) & ناظران انتخاباتی (۲,۱۲۰ نفر) + ناظران حقوق بشر & \rating{4} \\
\altrow \lr{OAU/AU} & حمایت دیپلماتیک + مشروعیت‌بخشی & \rating{3} \\
اتحادیهٔ اروپا & تحریم (قبل) + کمک مالی (بعد) + ناظران & \rating{4} \\
\altrow ایالات متحده & تحریم‌ها (\lr{CAAA} ۱۹۸۶) + فشار دیپلماتیک & \rating{4} \\
جنبش ضد آپارتاید جهانی & فشار افکار عمومی + تحریم فرهنگی-ورزشی & \rating{5} \\
\altrow دولت‌های جبهه‌ای (\lr{Frontline States}) & پناهندگی + پایگاه \lr{ANC} & \rating{3} \\
کمنولث & تعلیق عضویت + فشار & \rating{3} \\
\altrow \lr{IAEA} & تأیید خلع‌سلاح هسته‌ای & \rating{4} \\
\lr{NGO}ها (\lr{HRW, AI, IDASA}) & مستندسازی + آموزش مدنی & \rating{4} \\
\bottomrule
\end{tabularx}
\end{table}

\begin{keypoint}
\textbf{کلید موفقیت:} نقش بین‌المللی در آفریقای جنوبی \textbf{مکمل} بود نه \textbf{جایگزین} مالکیت ملی. \lr{UNOMSA} فقط ۲,۱۲۰ ناظر فرستاد (مقایسه کنید با ۸,۰۰۰ نفر تیمور شرقی یا ۱۵۰,۰۰۰ نفر عراق). مذاکرات \lr{CODESA} توسط \textbf{خود آفریقایی‌ها} هدایت شد. این دقیقاً مدلی است که برای ایران توصیه می‌شود: نظارت بین‌المللی حمایتی (مدل ۳-۴ → مدل ۶) با مالکیت ملی ایرانی (\seeChapter{ch:approaches}).
\end{keypoint}

\sectiondivider

%═══════════════════════════════════════════════════════════
\section{نتایج و ارزیابی بلندمدت}
\label{app:sa:outcomes}
%═══════════════════════════════════════════════════════════

\subsection{دستاوردها}

\begin{table}[htbp]
\centering
\caption{دستاوردها و چالش‌های پایدار آفریقای جنوبی}
\label{tab:app-sa-outcomes}
\begin{tabularx}{\textwidth}{>{\centering\arraybackslash}p{1cm} >{\raggedleft\arraybackslash}X >{\centering\arraybackslash}p{2cm}}
\toprule
\headerrow & \textbf{دستاوردها} & \textbf{امتیاز} \\
\midrule
\cmark & گذار مسالمت‌آمیز (بدون جنگ داخلی) & \starrating{5} \\
\altrow \cmark & قانون اساسی فراگیر و مترقی & \starrating{5} \\
\cmark & سه انتقال مسالمت‌آمیز قدرت (۱۹۹۴، ۱۹۹۹، ۲۰۰۹+) & \starrating{5} \\
\altrow \cmark & آزادی مطبوعات و جامعهٔ مدنی فعال & \starrating{4} \\
\cmark & ادغام موفق نیروهای مسلح & \starrating{4} \\
\altrow \cmark & خلع‌سلاح هسته‌ای داوطلبانه & \starrating{5} \\
\cmark & لغو مجازات اعدام & \starrating{5} \\
\midrule
\headerrow & \textbf{چالش‌های پایدار} & \textbf{شدت} \\
\midrule
\xmark & نابرابری اقتصادی شدید (جینی ۰.۶۳) & \riskhigh \\
\altrow \xmark & بیکاری بالا ($\sim$۳۳٪، جوانان: $\sim$۶۰٪) & \riskhigh \\
\xmark & فساد سیستماتیک (دورهٔ زوما: \lr{State Capture}) & \riskhigh \\
\altrow \xmark & خشونت جنسیتی بالا & \riskhigh \\
\xmark & بحران خدمات عمومی (آب، برق) & \riskmedium \\
\altrow \xmark & تسلط تک‌حزبی \lr{ANC} (تا ۲۰۲۴) & \riskmedium \\
\bottomrule
\end{tabularx}
\end{table}

\subsection{شاخص‌های کمّی (۲۰۲۳)}

\begin{center}
\begin{tabularx}{0.85\textwidth}{>{\raggedleft\arraybackslash}X >{\centering\arraybackslash}p{3cm}}
\toprule
\headerrow \textbf{شاخص} & \textbf{مقدار} \\
\midrule
\lr{V-Dem Liberal Democracy Index} & ۰.۷۲ \\
\altrow \lr{Freedom House} & آزاد (۷۹/۱۰۰) \\
\lr{Transparency International CPI} & ۴۳/۱۰۰ (رتبهٔ ۸۳) \\
\altrow \lr{RSF Press Freedom} & رتبهٔ ۳۸ \\
\lr{GDP per capita (PPP)} & $\sim$\$۱۵,۰۰۰ \\
\altrow \lr{HDI} & ۰.۷۱۳ (متوسط-بالا) \\
\bottomrule
\end{tabularx}
\end{center}

\sectiondivider

%═══════════════════════════════════════════════════════════
\section{ماتریس درس‌آموخته‌ها برای ایران}
\label{app:sa:lessons}
%═══════════════════════════════════════════════════════════

\begin{table}[htbp]
\centering
\caption{ماتریس انتقال درس‌آموخته‌های آفریقای جنوبی به ایران}
\label{tab:app-sa-lessons}
\begin{tabularx}{\textwidth}{
  >{\raggedleft\arraybackslash}p{2.5cm}
  >{\raggedleft\arraybackslash}p{3.5cm}
  >{\raggedleft\arraybackslash}X
  >{\centering\arraybackslash}p{1.5cm}
}
\toprule
\headerrow \textbf{بُعد} & \textbf{درس آفریقای جنوبی} & \textbf{کاربرد ایرانی} & \textbf{انتقال‌پذیری} \\
\midrule
مذاکره & \lr{CODESA}: ۱۹ طرف، فراگیر & «کنفرانس ملی ایران» با حضور همهٔ جریان‌ها & \rating{5} \\
\altrow
بندهای غروب & تضمین‌های موقت برای رژیم پیشین & تضمین ۵ ساله برای همکاری‌کنندگان نظام & \rating{5} \\
قانون اساسی & مجلس مؤسسان + مشارکت ۲M + ۱۱ زبان & مجلس مؤسسان + سهمیهٔ اقوام/زنان + ۳+ زبان & \rating{5} \\
\altrow
\lr{TRC} & حقیقت + عفو مشروط + پخش زنده & «کمیسیون حقیقت ایران» + رسانهٔ ملی & \rating{4} \\
ادغام نظامی & ۷ نیرو → \lr{SANDF} & سپاه + ارتش → ارتش ملی واحد & \rating{4} \\
\altrow
هسته‌ای & خلع‌سلاح قبل از گذار & الحاق به پروتکل الحاقی + بازرسی \lr{IAEA} & \rating{3} \\
نظارت بین‌المللی & \lr{UNOMSA} (حمایتی) & مدل ۶: حمایتی-نظارتی & \rating{5} \\
\altrow
عدالت اقتصادی & ضعیف (نابرابری پایدار) & ایران باید از این ضعف بیاموزد & \rating{4} \\
فراگیری & ۱۱ زبان + منشور حقوق & زبان‌های ملی + منشور حقوق اقلیت‌ها & \rating{5} \\
\altrow
ضد فساد & ضعیف (\lr{State Capture}) & نهاد ضد فساد مستقل از روز اول & \rating{4} \\
\midrule
\headerrow \multicolumn{3}{l}{\textbf{میانگین انتقال‌پذیری}} & \textbf{\rating{4}} \\
\bottomrule
\end{tabularx}
\end{table}

\sectiondivider

%═══════════════════════════════════════════════════════════
\section{نمودار: مسیر گذار آفریقای جنوبی و نقاط انتقال به ایران}
\label{app:sa:diagram}
%═══════════════════════════════════════════════════════════

\begin{figure}[htbp]
\centering
\begin{tikzpicture}[
  node distance=1.2cm and 2.5cm,
  phase/.style={
    draw, rounded corners=5pt, minimum width=3.2cm,
    minimum height=1.5cm, font=\small\bfseries, align=center,
    text=white
  },
  arrow/.style={->, thick, >=stealth},
  label/.style={font=\tiny, align=center, text width=3cm},
  iranlabel/.style={font=\tiny\itshape, align=center, text width=3cm, MainPurple}
]

% فازها
\node[phase, fill=MainRed!80] (crisis) {بحران و فشار\\(۱۹۸۵-۱۹۸۹)};
\node[phase, fill=MainOrange!80, right=of crisis] (opening) {گشایش\\(فوریه ۱۹۹۰)};
\node[phase, fill=MainYellow!90!black, right=of opening] (negotiation) {مذاکره\\(\lr{CODESA}\\۱۹۹۱-۱۹۹۳)};
\node[phase, fill=MainGreen!80, below=2cm of crisis] (election) {انتخابات\\(آوریل ۱۹۹۴)};
\node[phase, fill=MainBlue!80, right=of election] (consolidation) {تحکیم\\(۱۹۹۴-۱۹۹۹)};
\node[phase, fill=MainPurple!80, right=of consolidation] (democracy) {دموکراسی\\(۱۹۹۹+)};

% فلش‌ها
\draw[arrow] (crisis) -- (opening);
\draw[arrow] (opening) -- (negotiation);
\draw[arrow] (negotiation) -- (election);
\draw[arrow] (election) -- (consolidation);
\draw[arrow] (consolidation) -- (democracy);

% بحران‌ها (نقاط خطر)
\node[label, MainRed] at ($(opening)!0.5!(negotiation)+(0,1.2)$) {بحران: \lr{CODESA II}\\بن‌بست + بویپاتونگ};
\node[label, MainRed] at ($(negotiation)!0.5!(election)+(0,-0.8)$) {بحران: ترور هانی\\(آوریل ۱۹۹۳)};

% معادل ایرانی
\node[iranlabel] at ($(crisis)+(0,-1.2)$) {ایران: فاز ۰\\(پیش‌گذار)};
\node[iranlabel] at ($(opening)+(0,-1.2)$) {ایران: فاز ۱\\(تثبیت)};
\node[iranlabel] at ($(negotiation)+(0,-1.2)$) {ایران: فاز ۲\\(نهادسازی)};
\node[iranlabel] at ($(election)+(0,1.2)$) {ایران: فاز ۲\\(انتخابات)};
\node[iranlabel] at ($(consolidation)+(0,1.2)$) {ایران: فاز ۳\\(تحکیم)};
\node[iranlabel] at ($(democracy)+(0,1.2)$) {ایران: فاز ۴\\(خروج نظارت)};
\node[label, DarkGray] at ($(crisis)+(-1.5,-2.5)$) {تحریم‌ها\\اعتراضات\\فروپاشی شوروی};
\node[label, DarkGray] at ($(opening)+(0,-2.5)$) {آزادی ماندلا\\رفع ممنوعیت \lr{ANC}\\لغو قوانین آپارتاید};
\node[label, DarkGray] at ($(negotiation)+(1.5,-2.5)$) {۱۹ حزب\\بندهای غروب\\ق.ا.\ موقت};

\end{tikzpicture}
\caption{مسیر گذار آفریقای جنوبی و فازهای معادل ایرانی}
\label{fig:app-sa-path}
\end{figure}

\sectiondivider

%═══════════════════════════════════════════════════════════
\section{تحلیل مقایسه‌ای تفصیلی: آفریقای جنوبی و ایران}
\label{app:sa:comparison}
%═══════════════════════════════════════════════════════════

\subsection{ماتریس مشابهت‌ها و تفاوت‌ها}

\begin{table}[htbp]
\centering
\caption{ماتریس تفصیلی مشابهت‌ها و تفاوت‌های آفریقای جنوبی و ایران}
\label{tab:app-sa-iran-comparison}
\begin{tabularx}{\textwidth}{
  >{\raggedleft\arraybackslash}p{2.5cm}
  >{\raggedleft\arraybackslash}X
  >{\raggedleft\arraybackslash}X
  >{\centering\arraybackslash}p{1.5cm}
}
\toprule
\headerrow \textbf{بُعد} & \textbf{آفریقای جنوبی} & \textbf{ایران} & \textbf{مشابهت} \\
\midrule
ایدئولوژی رسمی & نژادپرستی (\lr{Apartheid}) & ولایت فقیه + تبعیض مذهبی-جنسیتی & \rating{4} \\
\altrow
جمعیت & ۴۰ میلیون & ۸۵+ میلیون & \rating{2} \\
تنوع قومی-زبانی & بسیار بالا (۱۱ زبان) & بالا (۸+ قوم اصلی) & \rating{4} \\
\altrow
تحریم‌های بین‌المللی & جامع و مؤثر & جامع (هسته‌ای + حقوق بشر) & \rating{5} \\
نیروهای امنیتی & \lr{SADF} + پلیس + نیروهای ویژه & سپاه + ارتش + بسیج + اطلاعات & \rating{4} \\
\altrow
منافع اقتصادی نظامیان & \lr{ARMSCOR} + صنایع دفاعی & خاتم‌الانبیاء + بنیادها + قاچاق & \rating{4} \\
برنامهٔ هسته‌ای & ۶ کلاهک (خلع ۱۹۹۳) & غنی‌سازی ۶۰٪+ (فعال) & \rating{3} \\
\altrow
اپوزیسیون سازمان‌یافته & \lr{ANC}: ۱۰۰+ سال، ۳M عضو & پراکنده، فاقد تشکل فراگیر & \rating{1} \\
دیاسپورا & محدود & بسیار بزرگ (۴-۵M) & \rating{2} \\
\altrow
جامعهٔ مدنی & قوی (\lr{UDF, COSATU}) & قوی اما سرکوب‌شده & \rating{3} \\
رهبر کاریزماتیک & ماندلا (نماد جهانی) & فاقد (چالش اصلی) & \rating{1} \\
\altrow
موقعیت ژئوپلیتیکی & منطقه‌ای مهم اما نه حساس & بسیار حساس (هرمز، هسته‌ای، همسایگان) & \rating{2} \\
نقش مذهب در سیاست & محدود (کلیسا حامی آشتی) & مرکزی (دین = قدرت) & \rating{2} \\
\altrow
سابقهٔ دموکراتیک & محدود (سفیدپوستان) & تجربهٔ مشروطه + مصدق (محدود) & \rating{3} \\
\midrule
\headerrow \multicolumn{3}{l}{\textbf{میانگین مشابهت کلی}} & \textbf{\rating{3}} \\
\bottomrule
\end{tabularx}
\end{table}

\subsection{پنج تفاوت حیاتی که تعدیل مدل را ضروری می‌سازد}

\begin{warningbox}
\textbf{پنج تفاوت حیاتی ایران با آفریقای جنوبی} که مانع کپی‌برداری مستقیم می‌شود:

\begin{enumerate}[nosep]
\item \textbf{فقدان ماندلا:} آفریقای جنوبی رهبری کاریزماتیک با مشروعیت جهانی و ۲۷ سال زندان داشت. ایران فاقد چنین شخصیتی است. \emphred{راه‌حل:} جایگزینی رهبر فردی با \textbf{ائتلاف فراگیر} و \textbf{نهادسازی} (رهبری جمعی).

\item \textbf{فقدان \lr{ANC}:} \lr{ANC} تشکیلاتی ۱۰۰+ ساله با میلیون‌ها عضو بود. اپوزیسیون ایرانی پراکنده و متفرق است. \emphred{راه‌حل:} تشکیل \textbf{پلتفرم هماهنگی اپوزیسیون} با ساختار فدرالی (نه تک‌حزبی) قبل از گذار.

\item \textbf{بُعد هسته‌ای فعال:} آفریقای جنوبی هسته‌ای را \textbf{قبل} از مذاکره خلع کرد. ایران برنامهٔ هسته‌ای فعال دارد که ابزار چانه‌زنی است. \emphred{راه‌حل:} گره زدن خلع هسته‌ای به \textbf{بستهٔ جامع مشوق‌ها} (تحریم + سرمایه‌گذاری + امنیت).

\item \textbf{ژئوپلیتیک حساس‌تر:} آفریقای جنوبی در منطقه‌ای نسبتاً باثبات بود. ایران ۱۵ همسایه دارد، تنگهٔ هرمز را کنترل می‌کند، و در ۴ جنگ نیابتی درگیر است. \emphred{راه‌حل:} \textbf{گروه تماس بین‌المللی} گسترده‌تر و حضور دیپلماتیک فعال‌تر.

\item \textbf{دین = قدرت:} در آفریقای جنوبی کلیسا نقش \textbf{میانجی‌گر} داشت. در ایران دین \textbf{ابزار قدرت} است و جدایی دین از دولت چالش اصلی خواهد بود. \emphred{راه‌حل:} تأکید بر \textbf{آزادی مذهب} (نه ضدیت با دین) و جلب حمایت روحانیون مستقل.
\end{enumerate}
\end{warningbox}

\subsection{پنج عنصر مستقیماً قابل‌انتقال}

\begin{recommendation}
\textbf{پنج عنصر از مدل آفریقای جنوبی} که مستقیماً به ایران قابل‌انتقال است:

\begin{enumerate}[nosep]
\item \textbf{بندهای غروب (\lr{Sunset Clauses}):} تضمین‌های موقت ۵ ساله برای کاهش ترس بخش‌های همکاری‌کنندهٔ نظام فعلی — امنیت شغلی کارمندان + عدم مصادره + مشارکت در دولت انتقالی. بدون این بندها، مذاکره آغاز نخواهد شد.

\item \textbf{مدل \lr{CODESA} (کنفرانس ملی فراگیر):} تشکیل «کنفرانس ملی ایران» با حضور همهٔ جریان‌ها — ملی-مذهبی، جمهوری‌خواه، فدرالیست، زنان، اقوام، جوانان، و حتی بخش‌هایی از نظام فعلی. قاعدهٔ «اجماع کافی» (نه اتفاق آرا).

\item \textbf{قانون اساسی دو مرحله‌ای:} ابتدا منشور موقت (اصول بنیادین غیرقابل‌تغییر) در فاز ۱-۲، سپس قانون اساسی دائمی توسط مجلس مؤسسان منتخب در فاز ۲-۳. دادگاه قانون اساسی بررسی انطباق کند.

\item \textbf{مدل \lr{TRC} با تعدیل:} «کمیسیون حقیقت و کرامت ایران» با مکانیزم عفو مشروط + پخش زنده + حقوق قربانیان + \textbf{افزودن بُعد اقتصادی} (برخلاف آفریقای جنوبی که این بُعد را نادیده گرفت).

\item \textbf{ادغام نیروهای مسلح (نه انحلال):} ادغام سپاه و ارتش در نیروی دفاعی ملی واحد + نظارت مدنی + بازنشستگی افتخاری + \textbf{تفکیک اقتصادی} (ترکیب مدل آفریقای جنوبی و اندونزی).
\end{enumerate}
\end{recommendation}

\sectiondivider

%═══════════════════════════════════════════════════════════
\section{نمودار: شبکهٔ انتقال درس‌آموخته‌ها}
\label{app:sa:network}
%═══════════════════════════════════════════════════════════

\begin{figure}[htbp]
\centering
\begin{tikzpicture}[
  node distance=2cm,
  sanode/.style={
    draw=MainGreen, fill=MainGreen!10, rounded corners=5pt,
    minimum width=2.8cm, minimum height=1cm, font=\small,
    align=center, thick
  },
  irannode/.style={
    draw=MainPurple, fill=MainPurple!10, rounded corners=5pt,
    minimum width=2.8cm, minimum height=1cm, font=\small,
    align=center, thick
  },
  transferarrow/.style={->, thick, >=stealth, MainOrange, dashed},
  directarrow/.style={->, very thick, >=stealth, MainGreen}
]

% گره‌های آفریقای جنوبی (بالا)
\node[sanode] (codesa) at (0,4) {\lr{CODESA}\\مذاکرهٔ فراگیر};
\node[sanode] (sunset) at (4,4) {بندهای غروب\\تضمین موقت};
\node[sanode] (trc) at (8,4) {\lr{TRC}\\حقیقت+عفو};
\node[sanode] (sandf) at (0,2) {\lr{SANDF}\\ادغام نیروها};
\node[sanode] (const) at (4,2) {قانون اساسی\\فراگیر};
\node[sanode] (nuclear) at (8,2) {خلع هسته‌ای\\داوطلبانه};

% عنوان آفریقای جنوبی
\node[font=\bfseries\small, MainGreen] at (4,5.3) {درس‌آموخته‌های آفریقای جنوبی};

% گره‌های ایران (پایین)
\node[irannode] (iran-conf) at (0,-1) {کنفرانس ملی\\ایران};
\node[irannode] (iran-sunset) at (4,-1) {تضمین‌ها برای\\همکاری‌کنندگان};
\node[irannode] (iran-trc) at (8,-1) {کمیسیون حقیقت\\و کرامت ایران};
\node[irannode] (iran-army) at (0,-3) {ارتش ملی\\واحد};
\node[irannode] (iran-const) at (4,-3) {مجلس مؤسسان\\+ منشور حقوق};
\node[irannode] (iran-nuke) at (8,-3) {توافق هسته‌ای\\جامع};

% عنوان ایران
\node[font=\bfseries\small, MainPurple] at (4,-4.3) {کاربردهای ایرانی (مدل ۶)};

% فلش‌های انتقال مستقیم
\draw[directarrow] (codesa) -- (iran-conf);
\draw[directarrow] (sunset) -- (iran-sunset);
\draw[directarrow] (trc) -- (iran-trc);
\draw[directarrow] (sandf) -- (iran-army);
\draw[directarrow] (const) -- (iran-const);

% فلش انتقال با تعدیل (هسته‌ای)
\draw[transferarrow] (nuclear) -- node[right, font=\tiny, MainOrange] {نیاز به تعدیل} (iran-nuke);

% خط جداکننده
\draw[gray!30, thick] (-2,0.5) -- (10,0.5);
\node[font=\tiny, gray] at (10.8,0.5) {خط انتقال};

\end{tikzpicture}
\caption{شبکهٔ انتقال درس‌آموخته‌ها: از آفریقای جنوبی به مدل ایران}
\label{fig:app-sa-transfer}
\end{figure}

\sectiondivider

%═══════════════════════════════════════════════════════════
\section{مصاحبه‌ها و منابع اصلی}
\label{app:sa:sources}
%═══════════════════════════════════════════════════════════

\subsection{منابع کلیدی دربارهٔ گذار آفریقای جنوبی}

\begin{enumerate}[nosep]
\item \textbf{ماندلا، نلسون} (۱۹۹۴). \textit{\lr{Long Walk to Freedom}}. لیتل براون. — خودزندگی‌نامهٔ ماندلا با جزئیات مذاکرات.

\item \textbf{اسپارکز، آلیستر} (۱۹۹۵). \textit{\lr{Tomorrow is Another Country: The Inside Story of South Africa's Negotiated Revolution}}. مندارین. — بهترین روایت ژورنالیستی از مذاکرات.

\item \textbf{توتو، دزموند} (۱۹۹۹). \textit{\lr{No Future Without Forgiveness}}. دابلدی. — فلسفهٔ \lr{TRC} از زبان رئیس آن.

\item \textbf{گزارش نهایی \lr{TRC}} (۱۹۹۸-۲۰۰۳). ۷ جلد. قابل‌دسترسی: \lr{justice.gov.za/trc}

\item \textbf{اسلوو، جو} (۱۹۹۲). «\lr{Negotiations: What Room for Compromise?}» — مقالهٔ تاریخی دربارهٔ بندهای غروب.

\item \textbf{واله، فردریک ون زیل} (۲۰۱۲). \textit{\lr{The Last Trek — A New Beginning}}. مکمیلان. — دیدگاه دکلرک.

\item \textbf{ساسکسمن، مارک} (۲۰۱۶). \textit{\lr{The State Capture Report}}. — تحلیل فساد دورهٔ زوما.

\item \textbf{لاند، کریس} (۲۰۰۶). «\lr{War and Peace in the Democratic Republic of Congo and South Africa}». \textit{\lr{Strategic Review for Southern Africa}}. — مقایسهٔ مدل‌های \lr{SSR}.
\end{enumerate}

\subsection{شخصیت‌های کلیدی برای مشاوره}

\begin{casestudy}
\textbf{کارشناسان آفریقای جنوبی که تیم آماده‌سازی ایران باید مشورت کند:}

\begin{itemize}[nosep]
\item \person{سیریل رامافوزا}{\lr{Cyril Ramaphosa}}: رئیس تیم مذاکره‌کنندهٔ \lr{ANC} در \lr{CODESA} (اکنون رئیس‌جمهور) — تجربهٔ مذاکره
\item \person{آلبی ساکس}{\lr{Albie Sachs}}: قاضی دادگاه قانون اساسی — تجربهٔ تدوین منشور حقوق
\item \person{الکس بورین}{\lr{Alex Boraine}}: معاون رئیس \lr{TRC} — تجربهٔ عدالت انتقالی (مؤسس \lr{ICTJ})
\item \person{ویلی استورهوف}{\lr{Willy Esterhuyse}}: واسطهٔ مذاکرات محرمانه — تجربهٔ کانال‌های پنهان
\item \person{هنری باردن}{\lr{Henry Memory Barton}}: مدیر ادغام \lr{SANDF} — تجربهٔ \lr{SSR}
\item \person{یاسمین سوکا}{\lr{Yasmin Sooka}}: کمیسیونر \lr{TRC} — تجربهٔ حقوق زنان در عدالت انتقالی
\end{itemize}

\textbf{توصیه:} در فاز ۰ (پیش‌گذار)، تشکیل \textbf{گروه مشاورهٔ آفریقای جنوبی} (\lr{South Africa Advisory Group}) متشکل از ۵-۱۰ کارشناس فوق برای انتقال تجربه به تیم ایرانی (\seeChapter{ch:timeline}).
\end{casestudy}

\sectiondivider

%═══════════════════════════════════════════════════════════
\section{جمع‌بندی پیوست}
\label{app:sa:conclusion}
%═══════════════════════════════════════════════════════════

\begin{chaptersummary}
جمع‌بندی پیوست ب — مطالعهٔ موردی آفریقای جنوبی:

\begin{enumerate}[nosep]
\item آفریقای جنوبی با امتیاز \textbf{۳۶ از ۴۰} جامع‌ترین الگوی قابل‌انتقال به ایران است.
\item پنج عنصر مستقیماً قابل‌انتقال: \lr{CODESA}، بندهای غروب، قانون اساسی فراگیر، \lr{TRC}، و ادغام نیروهای مسلح.
\item پنج تفاوت حیاتی (فقدان ماندلا، فقدان \lr{ANC}، هسته‌ای فعال، ژئوپلیتیک حساس‌تر، دین=قدرت) کپی‌برداری مستقیم را ناممکن و \textbf{تعدیل هوشمند} را ضروری می‌سازد.
\item مهم‌ترین نوآوری آفریقای جنوبی: \textbf{بندهای غروب} (تضمین موقت برای رژیم پیشین) که مذاکره را ممکن ساخت.
\item ضعف اصلی آفریقای جنوبی: \textbf{عدالت اقتصادی ناکافی} — ایران باید از این شکست بیاموزد و بُعد اقتصادی را از روز اول در مدل عدالت انتقالی بگنجاند.
\item \lr{TRC} الگوی اصلی عدالت انتقالی برای ایران است، اما باید با \textbf{بُعد اقتصادی} (غرامت واقعی) و \textbf{محاکمهٔ تکمیلی} (برای موارد عفو‌نشده) ارتقا یابد.
\item نقش بین‌المللی در آفریقای جنوبی \textbf{حمایتی} بود (مدل ۳) نه مدیریت مستقیم — این دقیقاً مبنای مدل ۶ پیشنهادی است.
\item تشکیل \textbf{گروه مشاورهٔ آفریقای جنوبی} در فاز ۰ پیش‌گذار توصیه می‌شود.
\end{enumerate}

\vspace{0.3cm}
\textit{مطالعهٔ تکمیلی:}
\begin{itemize}[nosep]
\item مقایسهٔ جامع ۹ نمونه: \seeChapter{app:comparison}
\item تضمین‌های موفقیت: \seeChapter{ch:guarantees}
\item عدالت انتقالی و \lr{SSR}: \seeChapter{ch:requirements}
\item سناریوی مذاکره‌ای (\lr{B}): \seeChapter{ch:scenarios}
\item شیلی (الگوی مکمل): \seeChapter{app:chile}
\end{itemize}
\end{chaptersummary}

\chapterend

%══════════════════════════════════════════════════════════════
% پایان پیوست ب
%══════════════════════════════════════════════════════════════
%══════════════════════════════════════════════════════════════
% پیوست پ: مطالعه موردی شیلی
% فایل: appendices/app-c-chile.tex
% حجم هدف: ۸-۱۰ صفحه
%══════════════════════════════════════════════════════════════

\chapter{مطالعهٔ موردی: شیلی (۱۹۸۸-۱۹۹۸)}
\label{app:chile}

\begin{executivesummary}
شیلی یکی از موفق‌ترین نمونه‌های گذار دموکراتیک در جهان است که از \textbf{دیکتاتوری نظامی} ژنرال \person{آگوستو پینوشه}{\lr{Augusto Pinochet}} (۱۹۷۳-۱۹۹۰) به دموکراسی باثبات انتقال یافت. ویژگی‌های منحصربه‌فرد این گذار عبارت‌اند از: ۱) استفاده از \textbf{رفراندوم} به‌عنوان ابزار گذار (پلبیسیت ۱۹۸۸)، ۲) \textbf{ائتلاف فراگیر اپوزیسیون} (\lr{Concertación})، ۳) \textbf{عدالت انتقالی تدریجی} (از کمیسیون رتیگ تا بازداشت لندن)، ۴) \textbf{مدل اقتصادی موفق} (رشد پایدار با عدالت اجتماعی)، و ۵) \textbf{مدیریت ریسک نظامی} (پینوشه ماند اما قدرتش کاهش یافت). این تجربه برای ایران در ابعاد رفراندوم، ائتلاف‌سازی، و عدالت تدریجی بسیار آموزنده است.
\end{executivesummary}

%═══════════════════════════════════════════════════════════
\section{زمینه و بافت تاریخی}
\label{app:chile:context}
%═══════════════════════════════════════════════════════════

\subsection{کودتای ۱۹۷۳ و دیکتاتوری پینوشه}

در ۱۱ سپتامبر ۱۹۷۳، ارتش شیلی به فرماندهی ژنرال \person{آگوستو پینوشه}{\lr{Augusto Pinochet}} با حمایت \lr{CIA} علیه دولت منتخب سوسیالیست \person{سالوادور آلنده}{\lr{Salvador Allende}} کودتا کرد. آلنده در کاخ ریاست‌جمهوری (\lr{La Moneda}) جان باخت و ۱۷ سال دیکتاتوری نظامی آغاز شد.

\begin{table}[htbp]
\centering
\caption{مشخصات شیلی در آستانهٔ گذار (۱۹۸۸)}
\label{tab:app-chile-profile}
\begin{tabularx}{\textwidth}{>{\raggedleft\arraybackslash}p{4.5cm} >{\raggedleft\arraybackslash}X}
\toprule
\headerrow \textbf{شاخص} & \textbf{مقدار} \\
\midrule
جمعیت & ۱۳ میلیون نفر \\
\altrow مساحت & ۷۵۶,۰۰۰ \lr{km²} \\
تنوع قومی & پایین (۹۵٪ مستیزو + اسپانیایی‌تبار) \\
\altrow \lr{GDP per capita} & $\sim$\$۲,۵۰۰ \\
طول عمر دیکتاتوری & ۱۷ سال (۱۹۷۳-۱۹۹۰) \\
\altrow اندازهٔ نیروهای مسلح & $\sim$۱۰۰,۰۰۰ \\
تعداد کشته‌شدگان/ناپدیدشدگان & ۳,۲۰۰+ (رتیگ) — تا ۴۰,۰۰۰+ (والش) \\
\altrow تعداد شکنجه‌شدگان & ۲۸,۰۰۰+ (والش) \\
مدل اقتصادی & نئولیبرال (\lr{Chicago Boys}) \\
\altrow تحریم‌های بین‌المللی & محدود (آمریکا حامی بود) \\
\bottomrule
\end{tabularx}
\end{table}

\subsection{ویژگی‌های دیکتاتوری پینوشه}

\begin{enumerate}[nosep]
\item \textbf{شخصی‌سازی قدرت:} پینوشه هم رئیس‌جمهور بود، هم فرمانده کل ارتش — تمرکز قدرت مشابه ساختار ولایت فقیه
\item \textbf{سرکوب سیستماتیک:} \org{پلیس مخفی}{\lr{DINA}} (بعداً \lr{CNI}) — عملیات کندور (\lr{Operation Condor}) در سطح منطقه‌ای
\item \textbf{اصلاحات اقتصادی نئولیبرال:} «پسران شیکاگو» (\lr{Chicago Boys}) اقتصاد را آزادسازی کردند — رشد اقتصادی با نابرابری
\item \textbf{قانون اساسی ۱۹۸۰:} پینوشه قانون اساسی جدیدی تصویب کرد که خودش را برای ۸ سال دیگر تثبیت می‌کرد و سپس رفراندوم (\lr{Plebiscite}) تعیین‌کننده بود
\item \textbf{نقش آمریکا:} دولت ریگان در ابتدا حامی پینوشه بود اما از ۱۹۸۶ فشار برای دموکراتیزه‌شدن آغاز شد
\end{enumerate}

\begin{casestudy}
\textbf{مقایسهٔ ساختاری پینوشه و جمهوری اسلامی:} هر دو نظام ترکیبی از \textbf{شخصی‌سازی قدرت + نیروهای امنیتی قدرتمند + ایدئولوژی (ضدکمونیسم/اسلام سیاسی) + سرکوب سازمان‌یافته} هستند. تفاوت‌ها: ۱) پینوشه فاقد مشروعیت مذهبی بود؛ ۲) اقتصاد شیلی رو به رشد بود (نه بحران)؛ ۳) جمعیت شیلی بسیار کمتر (۱۳M vs ۸۵M)؛ ۴) تنوع قومی پایین بود. نکتهٔ مهم: پینوشه \textbf{خودش} رفراندوم را در قانون اساسی‌اش گنجانده بود — اشتباه راهبردی که اپوزیسیون از آن بهره برد.
\end{casestudy}

\sectiondivider

%═══════════════════════════════════════════════════════════
\section{رفراندوم ۱۹۸۸: نقطهٔ عطف تاریخی}
\label{app:chile:plebiscite}
%═══════════════════════════════════════════════════════════

\subsection{زمینه‌سازی: از تفرقه تا ائتلاف}

مهم‌ترین درس شیلی برای اپوزیسیون ایران: چگونه جریان‌های متفرق و متخاصم توانستند در یک \textbf{ائتلاف فراگیر} متحد شوند.

\begin{table}[htbp]
\centering
\caption{ترکیب ائتلاف «نه» (\lr{Concertación}) و معادل ایرانی}
\label{tab:app-chile-coalition}
\begin{tabularx}{\textwidth}{>{\raggedleft\arraybackslash}p{3.2cm} >{\raggedleft\arraybackslash}X >{\raggedleft\arraybackslash}p{3.5cm}}
\toprule
\headerrow \textbf{جریان شیلیایی} & \textbf{ویژگی} & \textbf{معادل ایرانی احتمالی} \\
\midrule
\lr{PDC} (دموکرات‌مسیحی) & میانه‌رو، بزرگ‌ترین حزب & ملی-مذهبیون \\
\altrow \lr{PS} (سوسیالیست) & چپ میانه (حزب آلنده) & چپ دموکراتیک \\
\lr{PPD} (دموکراسی برای مردم) & لیبرال-چپ & جمهوری‌خواهان \\
\altrow \lr{PR} (رادیکال) & لائیک، لیبرال & سکولارها \\
کمونیست‌ها & چپ رادیکال (خارج ائتلاف رسمی) & چپ رادیکال (حاشیه‌ای) \\
\altrow جنبش زنان & حقوق زنان + مادران ناپدیدشدگان & جنبش زن-زندگی-آزادی \\
اتحادیهٔ کارگری (\lr{CUT}) & کارگران & تشکل‌های کارگری \\
\altrow روشنفکران و هنرمندان & فرهنگ‌سازی + کمپین خلاقانه & روشنفکران و هنرمندان ایرانی \\
\bottomrule
\end{tabularx}
\end{table}

\subsection{کمپین «نه» (\lr{NO}): نبوغ خلاقانه}

کمپین رفراندوم «نه» به پینوشه یکی از \textbf{خلاقانه‌ترین کمپین‌های سیاسی تاریخ} بود:

\begin{itemize}[nosep]
\item \textbf{۱۵ دقیقه تلویزیون شبانه:} طبق قانون، هر طرف ۱۵ دقیقه وقت تبلیغاتی تلویزیونی داشت. تیم «نه» به‌جای نشان دادن سرکوب و خشونت، \textbf{شادی و امید} را برگزید
\item \textbf{شعار «شادی در راه است» (\lr{La Alegría Ya Viene}):} پیام مثبت + رنگین‌کمان به‌جای پیام خشم و انتقام
\item \textbf{ثبت‌نام رأی‌دهندگان:} کمپین گستردهٔ ثبت‌نام ۷.۵ میلیون نفر
\item \textbf{ناظران داخلی:} سازمان \lr{Participa} شبکهٔ ۳۰,۰۰۰ ناظر داوطلب سازماندهی کرد
\item \textbf{شمارش موازی:} اپوزیسیون سیستم شمارش موازی (\lr{Parallel Vote Tabulation}) راه‌اندازی کرد تا تقلب غیرممکن شود
\end{itemize}

\begin{keypoint}
\textbf{نتیجهٔ رفراندوم ۵ اکتبر ۱۹۸۸:} «نه» ۵۵.۹۹٪ — «بله» ۴۴.۰۱٪. مشارکت: ۹۷٪ ثبت‌نام‌شدگان. پینوشه ابتدا نتیجه را نپذیرفت و خواست حکومت نظامی اعلام کند، اما \textbf{فرماندهان ارتش} (به‌ویژه ژنرال \person{ماتیاس}{Matthei}) از اجرای دستور امتناع کردند. این نشان داد که حتی در درون ارتش، \textbf{شکاف نخبگان} وجود داشت.
\end{keypoint}

\subsection{آمار و تحلیل رفراندوم}

\begin{table}[htbp]
\centering
\caption{آمار رفراندوم شیلی ۱۹۸۸ و الگوی ایرانی}
\label{tab:app-chile-referendum}
\begin{tabularx}{\textwidth}{>{\raggedleft\arraybackslash}p{4.5cm} >{\centering\arraybackslash}p{3cm} >{\raggedleft\arraybackslash}X}
\toprule
\headerrow \textbf{شاخص} & \textbf{شیلی ۱۹۸۸} & \textbf{کاربرد ایرانی} \\
\midrule
واجدین شرایط رأی & ۷,۴۳۵,۹۱۳ & $\sim$۶۰ میلیون \\
\altrow مشارکت & ۹۷.۵٪ & هدف: ۷۰٪+ \\
آرای «نه» & ۵۵.۹۹٪ (۳,۹۶۷,۵۷۹) & -- \\
\altrow آرای «بله» & ۴۴.۰۱٪ (۳,۱۱۹,۱۱۰) & -- \\
ناظران بین‌المللی & $\sim$۱,۰۰۰ & نیاز: ۱۰,۰۰۰+ \\
\altrow ناظران داخلی & $\sim$۳۰,۰۰۰ & نیاز: ۲۰۰,۰۰۰+ \\
شمارش موازی & \cmark (حیاتی) & \cmark (ضروری) \\
\altrow خشونت انتخاباتی & \risklow حداقل & باید تضمین شود \\
\bottomrule
\end{tabularx}
\end{table}

\begin{lessonlearned}
\textbf{ابزار رفراندوم برای ایران:} شیلی نشان داد که رفراندوم می‌تواند ابزار قدرتمند گذار باشد، \textbf{اما تنها در شرایطی:} ۱) اپوزیسیون متحد باشد؛ ۲) ناظران داخلی و بین‌المللی کافی وجود داشته باشد؛ ۳) شمارش موازی امکان‌پذیر باشد؛ ۴) رسانهٔ آزاد (حتی محدود) وجود داشته باشد. در سناریوی \lr{D} ایران (تحول از درون)، رفراندوم بر سر قانون اساسی جدید ممکن‌ترین ابزار است (\seeChapter{ch:scenarios}).
\end{lessonlearned}

\sectiondivider

%═══════════════════════════════════════════════════════════
\section{دولت آیلوین و انتقال مسالمت‌آمیز قدرت}
\label{app:chile:aylwin}
%═══════════════════════════════════════════════════════════

\subsection{انتخابات ۱۹۸۹ و محدودیت‌ها}

پس از شکست در رفراندوم، پینوشه انتخابات ریاست‌جمهوری و پارلمانی برگزار کرد (دسامبر ۱۹۸۹). \person{پاتریسیو آیلوین}{\lr{Patricio Aylwin}} از ائتلاف \lr{Concertación} با ۵۵٪ آرا پیروز شد.

اما پینوشه \textbf{قبل از ترک قدرت}، مجموعه‌ای از «قفل‌های نهادی» (\lr{Institutional Locks}) ایجاد کرد:

\begin{table}[htbp]
\centering
\caption{«قفل‌های نهادی» پینوشه و معادل ایرانی}
\label{tab:app-chile-locks}
\begin{tabularx}{\textwidth}{>{\raggedleft\arraybackslash}p{3.5cm} >{\raggedleft\arraybackslash}X >{\raggedleft\arraybackslash}p{3.5cm}}
\toprule
\headerrow \textbf{قفل پینوشه} & \textbf{توضیح} & \textbf{معادل ایرانی احتمالی} \\
\midrule
سناتورهای منصوب & ۹ سناتور منصوب (از جمله خود پینوشه) + مادام‌العمر & شورای نگهبان / مجلس خبرگان \\
\altrow فرمانده‌کل ارتش & پینوشه ۸ سال دیگر فرمانده ماند (تا ۱۹۹۸) & رهبر نظامی-امنیتی باقیمانده \\
عدم قابلیت عزل فرماندهان & رئیس‌جمهور حق عزل فرماندهان نظامی را نداشت & ساختار نظامی مستقل \\
\altrow قانون عفو ۱۹۷۸ & عفو عمومی برای جرایم ۱۹۷۳-۱۹۷۸ & مصونیت‌های قانونی \\
شورای امنیت ملی & ارتش اکثریت داشت (۴ از ۸ عضو) & شورای عالی امنیت ملی \\
\altrow نظام انتخاباتی دوعضوی & (\lr{Binomial}) که اقلیت (راست) را بیش‌نمایندگی می‌کرد & مهندسی انتخابات \\
\bottomrule
\end{tabularx}
\end{table}

\begin{warningbox}
\textbf{درس قفل‌های نهادی:} شیلی نشان داد که حتی پس از پیروزی دموکراتیک، رژیم پیشین می‌تواند از طریق \textbf{قفل‌های نهادی} قدرت خود را حفظ کند. شیلی ۱۵ سال طول کشید تا اکثر این قفل‌ها را بردارد (اصلاحیهٔ قانون اساسی ۲۰۰۵). \emphred{هشدار ایرانی:} در هر سناریوی مذاکره‌ای، باید مراقب ایجاد «قفل‌های نهادی» توسط جمهوری اسلامی بود. مدل ۶ پیشنهادی باید مکانیزم بازبینی و حذف تدریجی این قفل‌ها را پیش‌بینی کند (\seeChapter{ch:risks}).
\end{warningbox}

\subsection{استراتژی آیلوین: «عدالت تا حد ممکن»}

\person{آیلوین}{\lr{Aylwin}} عبارت معروف خود را به‌کار برد: «\textit{\lr{Justicia en la medida de lo posible}}» (عدالت تا حد ممکن). این به معنای \textbf{واقع‌بینی} در مواجهه با محدودیت‌ها بود:

\begin{itemize}[nosep]
\item پینوشه هنوز فرماندهٔ ارتش بود و تهدید کرده بود
\item قانون عفو ۱۹۷۸ مانع محاکمه بود
\item سناتورهای منصوب مانع اصلاح قانون اساسی بودند
\item اما آیلوین \textbf{کمیسیون حقیقت} تشکیل داد — «اگر نمی‌توانیم محاکمه کنیم، حداقل حقیقت را روشن می‌کنیم»
\end{itemize}

\sectiondivider

%═══════════════════════════════════════════════════════════
\section{عدالت انتقالی: مدل تدریجی شیلی}
\label{app:chile:tj}
%═══════════════════════════════════════════════════════════

شیلی مدل منحصربه‌فردی از عدالت انتقالی ارائه کرد: نه عفو کامل (اسپانیا) و نه محاکمهٔ فوری (نورنبرگ)، بلکه \textbf{پیشروی تدریجی} در طول ۳۰ سال.

\subsection{گاه‌شمار عدالت تدریجی}

\begin{table}[htbp]
\centering
\caption{گاه‌شمار عدالت انتقالی تدریجی شیلی (۱۹۹۰-۲۰۲۳)}
\label{tab:app-chile-tj-timeline}
\begin{tabularx}{\textwidth}{>{\centering\arraybackslash}p{2.2cm} >{\raggedleft\arraybackslash}X >{\centering\arraybackslash}p{2cm}}
\toprule
\headerrow \textbf{سال} & \textbf{اقدام} & \textbf{سطح عدالت} \\
\midrule
۱۹۹۱ & \textbf{کمیسیون رتیگ}: شناسایی ۳,۱۹۷ قربانی (کشته/ناپدید) & \statuswarn حقیقت \\
\altrow ۱۹۹۲ & \textbf{برنامهٔ غرامت:} مستمری ماهانه به خانواده‌ها ($\sim$\$۵۰۰/ماه) & \statuswarn غرامت \\
۱۹۹۶ & تفسیر جدید دادگاه عالی: ناپدیدشدن = «جرم مستمر» (مشمول عفو نمی‌شود) & \statuswarn حقوقی \\
\altrow ۱۹۹۸ & \textbf{بازداشت پینوشه در لندن} (حکم قاضی گارسون اسپانیا) & \statusok نقطهٔ عطف \\
۲۰۰۰ & بازگشت پینوشه + لغو مصونیت توسط دادگاه عالی شیلی & \statusok محاکمه \\
\altrow ۲۰۰۴ & \textbf{کمیسیون والش}: شناسایی ۲۸,۰۰۰+ قربانی شکنجه + غرامت & \statusok تعمیق \\
۲۰۰۵ & \textbf{اصلاح قانون اساسی}: حذف سناتورهای منصوب + کنترل مدنی ارتش & \statusok نهادی \\
\altrow ۲۰۰۶ & مرگ پینوشه (بدون محکومیت نهایی) & $\sim$ \\
۲۰۱۰-۲۰۲۳ & ادامهٔ محاکمات: ۵۰۰+ نظامی محکوم، ۱۲۰+ زندانی & \statusok عدالت \\
\altrow ۲۰۲۳ & پنجاهمین سالگرد کودتا + اعلام «برنامهٔ ملی جستجوی ناپدیدشدگان» & \statusok حافظه \\
\bottomrule
\end{tabularx}
\end{table}

\subsection{کمیسیون‌های حقیقت: رتیگ و والش}

\begin{table}[htbp]
\centering
\caption{مقایسهٔ دو کمیسیون حقیقت شیلی}
\label{tab:app-chile-commissions}
\begin{tabularx}{\textwidth}{>{\raggedleft\arraybackslash}p{3.5cm} >{\raggedleft\arraybackslash}X >{\raggedleft\arraybackslash}X}
\toprule
\headerrow \textbf{شاخص} & \textbf{کمیسیون رتیگ (۱۹۹۱)} & \textbf{کمیسیون والش (۲۰۰۳-۲۰۰۴)} \\
\midrule
نام رسمی & \lr{Comisión Nacional de Verdad y Reconciliación} & \lr{Comisión Nacional sobre Prisión Política y Tortura} \\
\altrow رئیس & \person{رائول رتیگ}{\lr{Raúl Rettig}} & \person{سرخیو والش}{\lr{Sergio Valech}} \\
تمرکز & کشته‌شدگان و ناپدیدشدگان & شکنجه‌شدگان و زندانیان سیاسی \\
\altrow تعداد قربانیان شناسایی‌شده & ۳,۱۹۷ (بعداً ۳,۲۰۰+) & ۲۸,۴۵۹ (بعداً ۴۰,۰۰۰+) \\
اختیار عفو & \xmark (فقط حقیقت‌یابی) & \xmark \\
\altrow غرامت & \cmark (مستمری ماهانه) & \cmark (مبلغ یکباره + مستمری) \\
تأثیر & گشایش فضای حقیقت & تعمیق + شناسایی ابعاد جدید \\
\bottomrule
\end{tabularx}
\end{table}

\begin{keypoint}
\textbf{نوآوری شیلی: عدالت تدریجی (\lr{Incremental Justice}).} برخلاف مدل آفریقای جنوبی (عفو مشروط یکباره) یا مدل عراقی (محاکمهٔ فوری)، شیلی مسیر \textbf{سی‌ساله‌ای} را طی کرد: ابتدا حقیقت (رتیگ ۱۹۹۱) ← سپس غرامت (۱۹۹۲) ← سپس تفسیر حقوقی جدید (۱۹۹۶) ← سپس بازداشت بین‌المللی (۱۹۹۸) ← سپس محاکمه‌های داخلی (۲۰۰۰+). این مدل نشان می‌دهد که عدالت \textbf{لزوماً فوری نیست}، اما باید \textbf{مسیر مشخص و رو به جلو} داشته باشد.
\end{keypoint}

\subsection{بازداشت لندن: نقطهٔ عطف جهانی}

در ۱۶ اکتبر ۱۹۹۸، پینوشه در لندن به‌دستور قاضی \person{بالتاسار گارسون}{\lr{Baltasar Garzón}} اسپانیایی بازداشت شد. این اولین بار بود که یک دیکتاتور سابق بر اساس اصل \termfn{صلاحیت جهانی}{Universal Jurisdiction} بازداشت می‌شد.

\begin{lessonlearned}
\textbf{درس بازداشت لندن برای ایران:} ۱) هیچ مصونیتی دائمی نیست — حتی بندهای غروب و عفوهای داخلی مانع صلاحیت جهانی نمی‌شوند؛ ۲) \textbf{زمان به نفع عدالت} است — اگر امروز محاکمه ممکن نیست، فردا ممکن می‌شود؛ ۳) دادگاه‌های بین‌المللی و خارجی \textbf{مکمل} عدالت داخلی هستند؛ ۴) تهدید صلاحیت جهانی \textbf{انگیزهٔ مذاکره} ایجاد می‌کند (مقامات جمهوری اسلامی بهتر است در عدالت انتقالی داخلی مشارکت کنند تا با دادگاه‌های بین‌المللی مواجه شوند).
\end{lessonlearned}

\sectiondivider

%═══════════════════════════════════════════════════════════
\section{اصلاح بخش امنیتی: مدل «حفظ با نظارت تدریجی»}
\label{app:chile:ssr}
%═══════════════════════════════════════════════════════════

\subsection{چالش: ارتشی که نرفت}

برخلاف بسیاری از نمونه‌ها، در شیلی \textbf{ارتش سقوط نکرد} — پینوشه فرماندهٔ ارتش ماند و ساختار نظامی دست‌نخورده باقی ماند. این ریسک بزرگی بود اما دولت آیلوین آن را مدیریت کرد:

\begin{table}[htbp]
\centering
\caption{استراتژی تدریجی نظارت مدنی بر ارتش شیلی}
\label{tab:app-chile-ssr}
\begin{tabularx}{\textwidth}{>{\centering\arraybackslash}p{2cm} >{\raggedleft\arraybackslash}X >{\centering\arraybackslash}p{2cm}}
\toprule
\headerrow \textbf{دوره} & \textbf{اقدام} & \textbf{ریسک} \\
\midrule
۱۹۹۰-۱۹۹۳ & احتیاط: پینوشه هنوز فرمانده بود. دو بار تهدید نظامی (\lr{Ejercicio de Enlace} ۱۹۹۰ + \lr{Boinazo} ۱۹۹۳) & \riskhigh \\
\altrow ۱۹۹۴-۱۹۹۸ & تثبیت: دولت فری (\lr{Frei}) ادامه‌داد. بازداشت پینوشه در لندن (۱۹۹۸) قدرت نمادین ارتش را شکست & \riskmedium \\
۱۹۹۸-۲۰۰۵ & اصلاح: حذف مصونیت پینوشه + محاکمات نظامیان + کاهش بودجهٔ نظامی & \riskmedium \\
\altrow ۲۰۰۵-۲۰۱۰ & نهادسازی: اصلاح قانون اساسی (۲۰۰۵): رئیس‌جمهور حق عزل فرماندهان + حذف شورای امنیت نظامی‌محور & \risklow \\
۲۰۱۰+ & تحکیم: ارتش کاملاً تحت نظارت مدنی. نظامیان در زندان. فرهنگ دموکراتیک نهادینه & \risklow \\
\bottomrule
\end{tabularx}
\end{table}

\begin{warningbox}
\textbf{ریسک مدل شیلی:} «حفظ ارتش با نظارت تدریجی» ذاتاً \textbf{خطرناک} است. پینوشه دو بار نیروهای مسلح را به حالت آماده‌باش درآورد (۱۹۹۰ و ۱۹۹۳) تا مانع تعقیب قضایی شود. فقط به‌دلیل \textbf{عزم سیاسی دولت + فشار بین‌المللی + شکاف درون ارتش}، کودتا رخ نداد. \emphred{هشدار ایرانی:} اگر سپاه بعد از گذار در ساختار باقی بماند، خطر «پینوشه ایرانی» (فرمانده‌ای که از درون تهدید کند) واقعی است. مدل ایران باید ترکیبی از شیلی (نظارت تدریجی) و آفریقای جنوبی (ادغام) باشد (\seeChapter{ch:guarantees}).
\end{warningbox}

\sectiondivider

%═══════════════════════════════════════════════════════════
\section{مدل اقتصادی: رشد با عدالت}
\label{app:chile:economy}
%═══════════════════════════════════════════════════════════

\subsection{«رشد با برابری» (\lr{Growth with Equity})}

یکی از مهم‌ترین درس‌های شیلی، مدیریت اقتصادی دورهٔ گذار بود:

\begin{table}[htbp]
\centering
\caption{شاخص‌های اقتصادی شیلی قبل و بعد از گذار}
\label{tab:app-chile-economy}
\begin{tabularx}{\textwidth}{>{\raggedleft\arraybackslash}p{4.5cm} >{\centering\arraybackslash}p{3cm} >{\centering\arraybackslash}p{3cm}}
\toprule
\headerrow \textbf{شاخص} & \textbf{۱۹۹۰ (گذار)} & \textbf{۲۰۰۰ (یک دهه بعد)} \\
\midrule
\lr{GDP per capita} & \$۲,۵۰۰ & \$۵,۰۰۰ \\
\altrow نرخ رشد \lr{GDP} & ۳.۷٪ & ۵.۴٪ (میانگین دهه) \\
نرخ فقر & ۴۰٪ & ۱۴٪ \\
\altrow نرخ بیکاری & ۱۰٪ & ۹.۲٪ \\
ضریب جینی & ۰.۵۵ & ۰.۵۲ \\
\altrow تورم & ۲۷٪ & ۴.۵٪ \\
سرمایه‌گذاری خارجی & \$۱.۵B & \$۴.۹B \\
\altrow عضویت بین‌المللی & -- & \lr{WTO} (۱۹۹۵), \lr{APEC} \\
\bottomrule
\end{tabularx}
\end{table}

\begin{enumerate}[nosep]
\item \textbf{حفظ ثبات اقتصادی:} دولت آیلوین ساختار اقتصادی نئولیبرال را \textbf{ناگهان} تغییر نداد — اصلاح تدریجی
\item \textbf{افزایش مالیات:} اصلاح مالیاتی (۱۹۹۰) برای تأمین هزینهٔ برنامه‌های اجتماعی
\item \textbf{سرمایه‌گذاری اجتماعی:} افزایش بودجهٔ بهداشت، آموزش، مسکن
\item \textbf{کاهش فقر:} از ۴۰٪ به ۱۴٪ در یک دهه (موفق‌ترین کاهش فقر در آمریکای لاتین)
\item \textbf{ادغام در اقتصاد جهانی:} عضویت \lr{WTO}، توافقات تجارت آزاد
\end{enumerate}

\begin{recommendation}
\textbf{مدل اقتصادی شیلی} برای ایران آموزنده است: ۱) در دورهٔ گذار، ثبات اقتصادی را فدای شعارهای انقلابی نکنید؛ ۲) اصلاح تدریجی اقتصاد (نه شوک‌تراپی لهستان و نه حفظ وضع موجود)؛ ۳) کاهش فقر = مشروعیت دموکراسی نوپا؛ ۴) ادغام در اقتصاد جهانی مهم‌ترین مشوق بلندمدت است. البته ضعف شیلی (نابرابری پایدار: جینی ۰.۵۲) درس منفی هم دارد: \textbf{رشد بدون بازتوزیع} کافی نیست (\seeChapter{ch:guarantees}).
\end{recommendation}

\sectiondivider

%═══════════════════════════════════════════════════════════
\section{ماتریس درس‌آموخته‌ها برای ایران}
\label{app:chile:lessons}
%═══════════════════════════════════════════════════════════

\begin{table}[htbp]
\centering
\caption{ماتریس انتقال درس‌آموخته‌های شیلی به ایران}
\label{tab:app-chile-lessons}
\begin{tabularx}{\textwidth}{
  >{\raggedleft\arraybackslash}p{2.2cm}
  >{\raggedleft\arraybackslash}p{3.5cm}
  >{\raggedleft\arraybackslash}X
  >{\centering\arraybackslash}p{1.5cm}
}
\toprule
\headerrow \textbf{بُعد} & \textbf{درس شیلی} & \textbf{کاربرد ایرانی} & \textbf{انتقال‌پذیری} \\
\midrule
رفراندوم & پلبیسیت ۱۹۸۸: ابزار تغییر & رفراندوم قانون اساسی جدید & \rating{5} \\
\altrow
ائتلاف‌سازی & \lr{Concertación}: ۱۶ حزب متحد شدند & ائتلاف فراگیر اپوزیسیون ایران & \rating{5} \\
کمپین خلاقانه & «شادی در راه است» + امید (نه خشم) & رسانهٔ امیدبخش + فضای مجازی & \rating{4} \\
\altrow
عدالت تدریجی & حقیقت ← غرامت ← محاکمه (۳۰ سال) & مسیر تدریجی اما رو به جلو & \rating{5} \\
قفل‌های نهادی & پینوشه قفل‌ها گذاشت → ۱۵ سال رفع & پیش‌بینی و مقابله با قفل‌های ج.ا. & \rating{5} \\
\altrow
مدیریت ارتش & حفظ + نظارت تدریجی (ریسکی) & ترکیب با مدل ادغام آفریقای جنوبی & \rating{3} \\
مدل اقتصادی & ثبات + اصلاح تدریجی + کاهش فقر & ثبات اقتصادی اولویت + اصلاح تدریجی & \rating{4} \\
\altrow
شمارش موازی & \lr{PVT} + ۳۰,۰۰۰ ناظر داوطلب & شبکهٔ ناظران مدنی + فناوری & \rating{5} \\
صلاحیت جهانی & بازداشت لندن: مصونیت دائمی نیست & هشدار به مقامات: مذاکره کنید & \rating{4} \\
\altrow
اصلاح قانون اساسی & اصلاحیهٔ ۲۰۰۵: ۱۵ سال طول کشید & مجلس مؤسسان از ابتدا (نه اصلاحیه) & \rating{3} \\
\midrule
\headerrow \multicolumn{3}{l}{\textbf{میانگین انتقال‌پذیری}} & \textbf{\rating{4}} \\
\bottomrule
\end{tabularx}
\end{table}

\sectiondivider

%═══════════════════════════════════════════════════════════
\section{نمودار: مقایسهٔ مسیر شیلی و مسیر پیشنهادی ایران}
\label{app:chile:diagram}
%═══════════════════════════════════════════════════════════

\begin{figure}[htbp]
\centering
\begin{tikzpicture}[
  node distance=0.5cm and 0.3cm,
  timeline/.style={very thick, ->, >=stealth},
  event/.style={
    draw, rounded corners=3pt, minimum width=2.2cm,
    minimum height=0.8cm, font=\tiny, align=center
  },
  chile/.style={event, fill=MainGreen!15, draw=MainGreen!60},
  iran/.style={event, fill=MainPurple!15, draw=MainPurple!60},
  danger/.style={event, fill=MainRed!15, draw=MainRed!60},
  success/.style={event, fill=MainGreen!25, draw=MainGreen!80}
]

% عنوان
\node[font=\small\bfseries, MainGreen] at (0,5.5) {مسیر شیلی (۱۹۸۸-۲۰۰۵)};
\node[font=\small\bfseries, MainPurple] at (0,0) {مسیر پیشنهادی ایران};

% خط زمانی شیلی
\draw[timeline, MainGreen!60] (-6,4.5) -- (6,4.5);

\node[chile] at (-5,4.5) {رفراندوم\\۱۹۸۸};
\node[chile] at (-3,4.5) {انتخابات\\۱۹۸۹};
\node[chile] at (-1,4.5) {رتیگ\\۱۹۹۱};
\node[danger] at (0.5,3.3) {تهدید نظامی\\بویناسو ۱۹۹۳};
\node[chile] at (1.5,4.5) {بازداشت\\لندن ۱۹۹۸};
\node[chile] at (3.5,4.5) {والش\\۲۰۰۴};
\node[success] at (5.5,4.5) {اصلاح ق.ا.\\۲۰۰۵};

% خط زمانی ایران
\draw[timeline, MainPurple!60] (-6,-1) -- (6,-1);

\node[iran] at (-5,-1) {رفراندوم\\ق.ا.\ جدید};
\node[iran] at (-3,-1) {انتخابات\\آزاد};
\node[iran] at (-1,-1) {کمیسیون\\حقیقت};
\node[danger] at (0.5,-2.2) {خطر: سپاه\\تهدید می‌کند؟};
\node[iran] at (1.5,-1) {محاکمات\\تدریجی};
\node[iran] at (3.5,-1) {تحکیم\\نهادی};
\node[success] at (5.5,-1) {دموکراسی\\پایدار};

% فلش‌های ارتباطی
\draw[dashed, MainOrange, ->] (-5,3.8) -- (-5,-0.3);
\draw[dashed, MainOrange, ->] (-3,3.8) -- (-3,-0.3);
\draw[dashed, MainOrange, ->] (-1,3.8) -- (-1,-0.3);
\draw[dashed, MainRed, ->] (0.5,3.0) -- (0.5,-1.7);
\draw[dashed, MainOrange, ->] (1.5,3.8) -- (1.5,-0.3);

% راهنما
\node[font=\tiny, MainOrange] at (7, 2) {انتقال درس};
\node[font=\tiny, MainRed] at (7, 1.5) {انتقال هشدار};

\end{tikzpicture}
\caption{مقایسهٔ مسیر شیلی و مسیر پیشنهادی ایران}
\label{fig:app-chile-comparison}
\end{figure}

\sectiondivider

%═══════════════════════════════════════════════════════════
\section{جمع‌بندی پیوست}
\label{app:chile:conclusion}
%═══════════════════════════════════════════════════════════

\begin{chaptersummary}
جمع‌بندی پیوست پ — مطالعهٔ موردی شیلی:

\begin{enumerate}[nosep]
\item شیلی با امتیاز \textbf{۲۹ از ۴۰} دومین الگوی مهم (پس از آفریقای جنوبی) برای ایران است، به‌ویژه در ابعاد \textbf{رفراندوم}، \textbf{ائتلاف‌سازی}، و \textbf{عدالت تدریجی}.
\item \textbf{رفراندوم ۱۹۸۸} نشان داد که ابزار مسالمت‌آمیز می‌تواند دیکتاتوری نظامی را سرنگون کند — بدون خشونت.
\item \textbf{ائتلاف \lr{Concertación}} الگوی ائتلاف‌سازی اپوزیسیون متفرق است: ۱۶ حزب از دموکرات‌مسیحی تا سوسیالیست متحد شدند.
\item \textbf{کمپین «نه»} نشان داد که پیام \textbf{امید و شادی} مؤثرتر از خشم و انتقام است — درس مستقیم برای رسانه‌های اپوزیسیون ایران.
\item \textbf{عدالت تدریجی} مدل منحصربه‌فردی است: حقیقت امروز ← غرامت فردا ← محاکمه پس‌فردا. مسیر ۳۰ ساله اما رو به جلو.
\item \textbf{قفل‌های نهادی} پینوشه هشدار جدی است: رژیم پیشین می‌تواند حتی پس از شکست، قدرت نهادی حفظ کند. مدل ۶ باید مکانیزم رفع تدریجی قفل‌ها داشته باشد.
\item \textbf{مدل اقتصادی} شیلی (ثبات + اصلاح تدریجی + کاهش فقر) الگوی مدیریت اقتصاد دورهٔ گذار است.
\item ضعف شیلی: \textbf{پینوشه ماند} و ۱۵ سال تهدید کرد + \textbf{نابرابری پایدار}. ایران باید مدل شیلی را با ادغام آفریقای جنوبی ترکیب کند.
\end{enumerate}

\vspace{0.3cm}
\textit{مطالعهٔ تکمیلی:}
\begin{itemize}[nosep]
\item مقایسهٔ جامع ۹ نمونه: \seeChapter{app:comparison}
\item آفریقای جنوبی (الگوی مکمل): \seeChapter{app:south-africa}
\item سناریوی رفراندوم: \seeChapter{ch:scenarios}
\item عدالت انتقالی: \seeChapter{ch:guarantees}
\item تونس (الگوی خاورمیانه‌ای): \seeChapter{app:tunisia}
\end{itemize}
\end{chaptersummary}

\chapterend

%══════════════════════════════════════════════════════════════
% پایان پیوست پ
%══════════════════════════════════════════════════════════════
%══════════════════════════════════════════════════════════════
% پیوست ت: مطالعه موردی تونس
% فایل: appendices/app-d-tunisia.tex
% حجم هدف: ۸-۱۰ صفحه
%══════════════════════════════════════════════════════════════

\chapter{مطالعهٔ موردی: تونس (۲۰۱۱-۲۰۲۱)}
\label{app:tunisia}

\begin{executivesummary}
تونس تنها نمونهٔ «موفقِ اولیه» در \termfn{بهار عربی}{\lr{Arab Spring}} بود که از دیکتاتوری \person{زین‌العابدین بن‌علی}{\lr{Zine El Abidine Ben Ali}} (۱۹۸۷-۲۰۱۱) به دموکراسی نوپا گذار کرد. ویژگی‌های منحصربه‌فرد: ۱) \textbf{انقلاب مردمی غیرمنتظره} (دسامبر ۲۰۱۰)، ۲) \textbf{گفت‌وگوی ملی} میان اسلام‌گرایان و سکولارها (چهارگانهٔ نوبل ۲۰۱۵)، ۳) \textbf{قانون اساسی مترقی ۲۰۱۴} (شامل مادهٔ ۴۶: برابری جنسیتی)، ۴) \textbf{نهاد حقیقت و کرامت} (\lr{IVD})، و ۵) متأسفانه \textbf{بازگشت اقتدارگرایی} پس از کودتای قیس سعید (۲۰۲۱). تونس هم \textbf{الگوی امیدبخش} (۲۰۱۱-۲۰۱۹) و هم \textbf{هشدار جدی} (۲۰۲۱+) برای ایران است. ارتباط ویژه با ایران: نزدیکی فرهنگی-منطقه‌ای (خاورمیانه/شمال آفریقا)، رابطهٔ دین و سیاست، جوانی جمعیت، و چالش اقتصادی پس از گذار.
\end{executivesummary}

%═══════════════════════════════════════════════════════════
\section{زمینه و بافت تاریخی}
\label{app:tunisia:context}
%═══════════════════════════════════════════════════════════

\subsection{رژیم بن‌علی: اقتدارگرایی مدرن}

\person{زین‌العابدین بن‌علی}{\lr{Ben Ali}} در کودتای «پزشکی» ۱۹۸۷ به قدرت رسید و تا ۲۰۱۱ حکومت کرد. رژیم او ترکیبی بود از:

\begin{enumerate}[nosep]
\item \textbf{سکولاریسم اقتدارگرا:} سرکوب همزمان اسلام‌گرایان (\lr{Ennahda}) و چپ‌ها
\item \textbf{لیبرالیسم اقتصادی ظاهری:} رشد اقتصادی اما فساد خانوادگی (خاندان طرابلسی)
\item \textbf{دولت پلیسی:} ۱۸۰,۰۰۰ نیروی امنیتی برای ۱۱ میلیون جمعیت (نسبت بسیار بالا)
\item \textbf{کنترل رسانه و اینترنت:} سانسور گسترده (لقب «عمّار ۴۰۴»)
\item \textbf{نمای دموکراتیک:} انتخابات فرمایشی با ۹۰٪+ رأی
\end{enumerate}

\begin{table}[htbp]
\centering
\caption{مشخصات تونس در آستانهٔ انقلاب (۲۰۱۰)}
\label{tab:app-tunisia-profile}
\begin{tabularx}{\textwidth}{>{\raggedleft\arraybackslash}p{4.5cm} >{\raggedleft\arraybackslash}X}
\toprule
\headerrow \textbf{شاخص} & \textbf{مقدار} \\
\midrule
جمعیت & ۱۰.۶ میلیون نفر \\
\altrow مساحت & ۱۶۳,۰۰۰ \lr{km²} \\
تنوع قومی & پایین (عرب-بربر) \\
\altrow \lr{GDP per capita} & $\sim$\$۴,۲۰۰ \\
طول عمر رژیم بن‌علی & ۲۳ سال (۱۹۸۷-۲۰۱۱) \\
\altrow نرخ بیکاری جوانان & $\sim$۳۰٪ \\
نیروهای امنیتی & $\sim$۱۸۰,۰۰۰ (پلیس + حرس ملی) \\
\altrow ارتش & $\sim$۵۰,۰۰۰ (حرفه‌ای، غیرسیاسی) \\
سطح سواد & ۸۲٪ \\
\altrow جامعهٔ مدنی & نسبتاً فعال (اتحادیهٔ کارگری \lr{UGTT} قوی) \\
\bottomrule
\end{tabularx}
\end{table}

\begin{casestudy}
\textbf{مقایسهٔ بن‌علی و جمهوری اسلامی:} مشابهت‌ها: دولت پلیسی + سانسور + فساد خانوادگی + جوانی جمعیت + بیکاری بالا + انتخابات فرمایشی. تفاوت‌ها: ۱) بن‌علی ایدئولوژی مذهبی نداشت (سکولار بود)؛ ۲) ارتش تونس \textbf{غیرسیاسی و کوچک} بود (برخلاف سپاه)؛ ۳) تونس فاقد منابع نفتی بود (وابستگی به گردشگری)؛ ۴) جمعیت بسیار کمتر (۱۱M vs ۸۵M)؛ ۵) تونس فاقد بُعد هسته‌ای بود. مهم‌ترین مشابهت: \textbf{رابطهٔ اسلام‌گرایان و سکولارها} — چالشی که تونس (موقتاً) حل کرد اما ایران حل نکرده است.
\end{casestudy}

\sectiondivider

%═══════════════════════════════════════════════════════════
\section{انقلاب یاسمن: از خودسوزی تا سقوط}
\label{app:tunisia:revolution}
%═══════════════════════════════════════════════════════════

\subsection{گاه‌شمار انقلاب (دسامبر ۲۰۱۰ — ژانویه ۲۰۱۱)}

\begin{table}[htbp]
\centering
\caption{گاه‌شمار انقلاب تونس (۲۸ روز تا سقوط)}
\label{tab:app-tunisia-revolution-timeline}
\begin{tabularx}{\textwidth}{>{\centering\arraybackslash}p{3cm} >{\raggedleft\arraybackslash}X >{\centering\arraybackslash}p{2cm}}
\toprule
\headerrow \textbf{تاریخ} & \textbf{رویداد} & \textbf{شدت} \\
\midrule
۱۷ دسامبر ۲۰۱۰ & خودسوزی \person{محمد بوعزیزی}{\lr{Mohamed Bouazizi}} در سیدی‌بوزید & نقطهٔ آغاز \\
\altrow ۱۸-۲۴ دسامبر & اعتراضات در شهرهای مرکزی (سیدی‌بوزید، قصرین، تالا) & \riskmedium \\
۲۵ دسامبر-۳ ژانویه & گسترش به شهرهای بزرگ‌تر + سرکوب خشن (۲۰+ کشته) & \riskhigh \\
\altrow ۴ ژانویه ۲۰۱۱ & مرگ بوعزیزی → تشدید اعتراضات & نقطهٔ عطف \\
۸-۱۰ ژانویه & اعتصاب عمومی \lr{UGTT} + اعتراضات سراسری & \riskhigh \\
\altrow ۱۱-۱۲ ژانویه & ارتش از شلیک به مردم \textbf{امتناع} کرد (ژنرال \person{رشید عمّار}{\lr{Rachid Ammar}}) & تعیین‌کننده \\
۱۳ ژانویه & بن‌علی سخنرانی آخر: «فهمیدم» (\lr{Fhemtkom}) & ناکافی \\
\altrow \textbf{۱۴ ژانویه ۲۰۱۱} & \textbf{فرار بن‌علی به عربستان} — سقوط رژیم ۲۸ روزه & \cmark \\
\bottomrule
\end{tabularx}
\end{table}

\subsection{نقش تعیین‌کنندهٔ ارتش}

\begin{keypoint}
\textbf{مهم‌ترین درس تونس:} ارتش \textbf{از شلیک به مردم امتناع کرد}. ژنرال \person{رشید عمّار}{\lr{Rachid Ammar}} فرمان بن‌علی را اجرا نکرد و بدین ترتیب سرنوشت انقلاب را تعیین کرد. دلایل: ۱) ارتش تونس \textbf{حرفه‌ای و غیرسیاسی} بود (برخلاف مصر و لیبی)؛ ۲) ارتش فاقد منافع اقتصادی بود؛ ۳) وفاداری به «ملت» نه «رژیم». \emphorange{سؤال کلیدی ایرانی:} آیا بخشی از سپاه حاضر است از شلیک امتناع کند؟ بدون «لحظهٔ عمّار»، سناریوی \lr{C} (انقلاب مردمی) ایران بسیار خشونت‌بارتر خواهد بود (\seeChapter{ch:scenarios}).
\end{keypoint}

\subsection{نقش فضای مجازی و رسانه}

تونس اولین «انقلاب فیسبوکی» نام گرفت:

\begin{itemize}[nosep]
\item \textbf{فیسبوک:} ۲ میلیون کاربر (از ۱۱M جمعیت) — انتشار ویدیوهای سرکوب
\item \textbf{الجزیره:} پوشش زنده اعتراضات — شکستن انحصار خبری رژیم
\item \textbf{بلاگرها:} شبکهٔ وبلاگ‌نویسان (\person{لینا بن مهنی}{\lr{Lina Ben Mhenni}}) مستندسازی کردند
\item \textbf{ویکی‌لیکس:} کابل‌های سفارت آمریکا دربارهٔ فساد خاندان طرابلسی فاش شد
\end{itemize}

\begin{lessonlearned}
\textbf{فضای مجازی سلاح دولبه است:} تونس نشان داد که شبکه‌های اجتماعی می‌توانند \textbf{بسیج سریع} و \textbf{شکستن سانسور} ایجاد کنند. اما همان ابزار بعداً برای \textbf{اطلاعات نادرست} و \textbf{قطبی‌سازی} نیز استفاده شد. \emphorange{درس ایرانی:} ۱) استارلینک و VPN برای دور زدن فیلترینگ حیاتی‌اند؛ ۲) اما باید مکانیزم \textbf{راستی‌آزمایی} و \textbf{ضد اطلاعات نادرست} از ابتدا طراحی شود (\seeChapter{ch:timeline}).
\end{lessonlearned}

\sectiondivider

%═══════════════════════════════════════════════════════════
\section{گفت‌وگوی ملی: چهارگانهٔ نوبل}
\label{app:tunisia:dialogue}
%═══════════════════════════════════════════════════════════

\subsection{بحران ۲۰۱۳: آستانهٔ شکست}

در ۲۰۱۳، تونس در آستانهٔ سقوط به مدل مصری (کودتای نظامی) یا لیبیایی (جنگ داخلی) بود:

\begin{itemize}[nosep]
\item \textbf{ترور دو رهبر سکولار:} \person{شکری بلعید}{\lr{Chokri Belaïd}} (فوریه) و \person{محمد البراهمی}{\lr{Mohamed Brahmi}} (ژوئیه)
\item \textbf{قطبی‌شدن شدید:} اسلام‌گرایان (\lr{Ennahda}) در مقابل سکولارها
\item \textbf{بحران اقتصادی:} بیکاری + تورم + کاهش گردشگری
\item \textbf{مدل مصر:} کودتای سیسی (ژوئیه ۲۰۱۳) الگوی خطرناکی ایجاد کرد
\item \textbf{اعتراضات خیابانی:} مردم خواهان انحلال مجلس مؤسسان شدند
\end{itemize}

\subsection{چهارگانهٔ گفت‌وگوی ملی (\lr{Quartet})}

در بحرانی‌ترین لحظه، چهار سازمان جامعهٔ مدنی تونس نقش \textbf{میانجی} ایفا کردند و «گفت‌وگوی ملی» (\lr{Dialogue National}) را سازماندهی کردند:

\begin{table}[htbp]
\centering
\caption{چهارگانهٔ گفت‌وگوی ملی تونس (برندهٔ نوبل صلح ۲۰۱۵)}
\label{tab:app-tunisia-quartet}
\begin{tabularx}{\textwidth}{>{\raggedleft\arraybackslash}p{3.5cm} >{\raggedleft\arraybackslash}X >{\raggedleft\arraybackslash}p{3.5cm}}
\toprule
\headerrow \textbf{سازمان} & \textbf{نقش} & \textbf{معادل ایرانی احتمالی} \\
\midrule
\org{اتحادیهٔ عمومی کارگران تونس}{\lr{UGTT}} & بزرگ‌ترین تشکل مدنی (۷۰۰K عضو) — میانجی اصلی & کانون‌های صنفی + تشکل‌های کارگری \\
\altrow \org{اتحادیهٔ صنعت و تجارت}{\lr{UTICA}} & نمایندهٔ بخش خصوصی — اعتمادسازی اقتصادی & اتاق بازرگانی + بخش خصوصی \\
\org{انجمن حقوق بشر تونس}{\lr{LTDH}} & نظارت حقوق بشری — مشروعیت‌بخشی & کانون وکلا + فعالان حقوق بشر \\
\altrow \org{کانون وکلای تونس}{\lr{ONAT}} & تضمین حقوقی — چارچوب قانونی & کانون وکلای دادگستری \\
\bottomrule
\end{tabularx}
\end{table}

\subsection{دستاوردهای گفت‌وگوی ملی}

\begin{enumerate}[nosep]
\item \textbf{استعفای دولت \lr{Ennahda}:} حزب اسلام‌گرای حاکم داوطلبانه قدرت را واگذار کرد — بی‌سابقه در جهان عرب
\item \textbf{تشکیل دولت تکنوکرات:} دولت بی‌طرف \person{مهدی جمعه}{\lr{Mehdi Jomaa}} برای مدیریت انتقال
\item \textbf{تکمیل قانون اساسی:} مجلس مؤسسان قانون اساسی ۲۰۱۴ را تصویب کرد
\item \textbf{جلوگیری از مدل مصری:} تونس کودتا را تجربه نکرد
\item \textbf{جایزهٔ نوبل صلح ۲۰۱۵:} به چهارگانه اعطا شد
\end{enumerate}

\begin{keypoint}
\textbf{نوآوری تونس: نقش جامعهٔ مدنی به‌عنوان میانجی.} نه دولت، نه ارتش، نه بازیگران خارجی — بلکه \textbf{جامعهٔ مدنی سازمان‌یافته} بحران را حل کرد. \lr{UGTT} با ۷۰۰,۰۰۰ عضو، اعتبار کافی برای نشاندن دو طرف پشت میز مذاکره داشت. \emphgreen{کاربرد ایرانی:} آیا تشکل‌های مدنی ایران (کانون وکلا، انجمن‌های صنفی، شبکه‌های زنان) می‌توانند نقش مشابهی ایفا کنند؟ چالش: سطح سازمان‌یافتگی جامعهٔ مدنی ایران پایین‌تر از تونس است (\seeChapter{ch:actors}).
\end{keypoint}

\begin{recommendation}
\textbf{مدل «چهارگانهٔ ایرانی» (پیشنهاد):} بر اساس الگوی تونس، تشکیل ائتلاف میانجی‌گر از: ۱) \textbf{تشکل‌های کارگری و صنفی} (معلمان، کارگران، پرستاران)؛ ۲) \textbf{اتاق بازرگانی و بخش خصوصی}؛ ۳) \textbf{کانون وکلای دادگستری}؛ ۴) \textbf{شبکهٔ زنان ایران} (جنبش «زن، زندگی، آزادی»). این چهارگانه می‌تواند در فاز ۱-۲ نقش میانجی بین جریان‌های سیاسی ایفا کند.
\end{recommendation}

\sectiondivider

%═══════════════════════════════════════════════════════════
\section{قانون اساسی ۲۰۱۴: پیشرفته‌ترین در جهان عرب}
\label{app:tunisia:constitution}
%═══════════════════════════════════════════════════════════

\subsection{فرآیند تدوین}

\begin{table}[htbp]
\centering
\caption{فرآیند تدوین قانون اساسی تونس (۲۰۱۱-۲۰۱۴)}
\label{tab:app-tunisia-constitution}
\begin{tabularx}{\textwidth}{>{\centering\arraybackslash}p{2.5cm} >{\raggedleft\arraybackslash}X}
\toprule
\headerrow \textbf{مرحله} & \textbf{جزئیات} \\
\midrule
اکتبر ۲۰۱۱ & انتخابات مجلس مؤسسان (\lr{ANC}): ۲۱۷ نماینده، مشارکت ۵۲٪ \\
\altrow ۲۰۱۲-۲۰۱۳ & تدوین پیش‌نویس‌های متعدد + مشاوره عمومی + بحران ۲۰۱۳ \\
ژانویه ۲۰۱۴ & تصویب نهایی: ۲۰۰ رأی موافق از ۲۱۶ (۹۳٪) \\
\altrow ویژگی مهم & اجماع بین اسلام‌گرایان و سکولارها (نه رأی اکثریت ساده) \\
\bottomrule
\end{tabularx}
\end{table}

\subsection{نوآوری‌های قانون اساسی ۲۰۱۴}

\begin{enumerate}[nosep]
\item \textbf{مادهٔ ۱:} «تونس دولتی آزاد، مستقل و با حاکمیت است؛ اسلام دین آن، عربی زبان آن و جمهوری نظام آن است» — ترکیب هویت اسلامی با جمهوریت (بدون شریعت)
\item \textbf{مادهٔ ۲:} «تونس دولتی مدنی (\lr{État civil}) است، مبتنی بر شهروندی، ارادهٔ مردم و برتری قانون» — مهم‌ترین ماده: \textbf{«دولت مدنی»} نه دینی و نه نظامی
\item \textbf{مادهٔ ۶:} آزادی وجدان و اعتقاد + حمایت از مقدسات + ممنوعیت تکفیر — \textbf{سازش خلاقانه} بین اسلام‌گرایان و سکولارها
\item \textbf{مادهٔ ۴۶:} «دولت حقوق زنان و برابری فرصت‌ها را تضمین می‌کند... و تلاش می‌کند نمایندگی برابر (\lr{parité}) در مجالس منتخب تحقق یابد» — پیشرفته‌ترین مادهٔ جنسیتی در جهان عرب
\item \textbf{مادهٔ ۴۹:} محدودیت حقوق فقط به‌موجب قانون و با رعایت تناسب — اصل تناسب
\item \textbf{نظام نیمه‌ریاستی:} تقسیم قدرت بین رئیس‌جمهور و نخست‌وزیر
\item \textbf{تمرکززدایی:} فصل هفتم: حکومت‌های محلی منتخب
\end{enumerate}

\begin{lessonlearned}
\textbf{سازش \lr{Ennahda}:} \person{راشد غنوشی}{\lr{Rached Ghannouchi}}، رهبر \lr{Ennahda}، تصمیم تاریخی گرفت: ۱) «شریعت» را به‌عنوان منبع قانون‌گذاری مطالبه نکرد؛ ۲) «دولت مدنی» را پذیرفت؛ ۳) قدرت را داوطلبانه واگذار کرد. این بی‌سابقه در تاریخ اسلام‌گرایی بود. \emphgreen{درس ایرانی:} آیا بخشی از روحانیت ایران حاضر است «مدل غنوشی» را بپذیرد — یعنی دین بدون ادعای حکمرانی؟ این یکی از کلیدی‌ترین سؤالات فاز پیش‌گذار است.
\end{lessonlearned}

\sectiondivider

%═══════════════════════════════════════════════════════════
\section{نهاد حقیقت و کرامت (\lr{IVD})}
\label{app:tunisia:ivd}
%═══════════════════════════════════════════════════════════

\subsection{ساختار و مأموریت}

\org{نهاد حقیقت و کرامت}{\lr{Instance Vérité et Dignité (IVD)}} در ژوئن ۲۰۱۴ تشکیل شد و تا دسامبر ۲۰۱۸ فعالیت کرد:

\begin{table}[htbp]
\centering
\caption{ساختار و آمار \lr{IVD} تونس}
\label{tab:app-tunisia-ivd}
\begin{tabularx}{\textwidth}{>{\raggedleft\arraybackslash}p{5cm} >{\raggedleft\arraybackslash}X}
\toprule
\headerrow \textbf{شاخص} & \textbf{جزئیات} \\
\midrule
رئیس & \person{سهام بن‌سدرین}{\lr{Sihem Bensedrine}} \\
\altrow تعداد اعضای هیئت‌رئیسه & ۱۵ کمیسیونر \\
بازهٔ زمانی پوشش & ۱ ژوئیه ۱۹۵۵ تا ۳۱ دسامبر ۲۰۱۳ (۵۸ سال!) \\
\altrow تعداد پرونده‌های دریافتی & ۶۲,۷۲۰ پرونده \\
جلسات علنی & ۱۴ جلسه (برخی پخش زندهٔ تلویزیونی) \\
\altrow دوایر تخصصی & شکنجه، ناپدیدشدگی، خشونت جنسی، فساد مالی، تبعیض منطقه‌ای \\
نوآوری ویژه & \textbf{بُعد اقتصادی:} فساد مالی و تبعیض اقتصادی منطقه‌ای بررسی شد \\
\altrow بودجه & $\sim$\$۱۵ میلیون \\
گزارش نهایی & ۲۰۱۹ (۵+ جلد، ۸۰,۰۰۰+ صفحه) \\
\bottomrule
\end{tabularx}
\end{table}

\subsection{نوآوری‌های \lr{IVD} نسبت به \lr{TRC}}

\begin{enumerate}[nosep]
\item \textbf{بُعد اقتصادی:} برخلاف \lr{TRC} آفریقای جنوبی که فقط خشونت فیزیکی را بررسی کرد، \lr{IVD} \textbf{فساد مالی و تبعیض اقتصادی} را نیز پوشش داد
\item \textbf{بازهٔ زمانی بلند:} ۵۸ سال (از استقلال تا ۲۰۱۳) — نه فقط دورهٔ بن‌علی
\item \textbf{حقوق زنان:} کمیسیون ویژهٔ خشونت جنسیتی
\item \textbf{تبعیض منطقه‌ای:} توجه به محرومیت مناطق مرکزی و جنوبی (ریشهٔ انقلاب)
\item \textbf{«آشتی اقتصادی»:} مکانیزم بازپرداخت مالی توسط فاسدان (بدون محاکمه)
\end{enumerate}

\begin{keypoint}
\textbf{ارزش نوآوری اقتصادی \lr{IVD} برای ایران:} در ایران، فساد اقتصادی (سپاه، بنیادها، خودی‌ها) \textbf{ابعاد بسیار بزرگ‌تری} از خشونت فیزیکی دارد. کمیسیون حقیقت ایران \textbf{باید} بُعد اقتصادی داشته باشد: مستندسازی فساد سیستماتیک + تبعیض منطقه‌ای (بلوچستان، کردستان، خوزستان) + مصادرهٔ اموال + قاچاق سازمان‌یافته. مدل \lr{IVD} در این بُعد از \lr{TRC} بهتر است.
\end{keypoint}

\sectiondivider

%═══════════════════════════════════════════════════════════
\section{بازگشت اقتدارگرایی: کودتای قیس سعید (۲۰۲۱)}
\label{app:tunisia:backsliding}
%═══════════════════════════════════════════════════════════

\subsection{چگونه دموکراسی شکست خورد}

در ۲۵ ژوئیه ۲۰۲۱، \person{قیس سعید}{\lr{Kais Saied}} رئیس‌جمهور تونس، پارلمان را تعلیق و قدرت را قبضه کرد. مراحل بازگشت:

\begin{table}[htbp]
\centering
\caption{مراحل بازگشت اقتدارگرایی در تونس (۲۰۱۹-۲۰۲۳)}
\label{tab:app-tunisia-backsliding}
\begin{tabularx}{\textwidth}{>{\centering\arraybackslash}p{2.5cm} >{\raggedleft\arraybackslash}X >{\centering\arraybackslash}p{2cm}}
\toprule
\headerrow \textbf{تاریخ} & \textbf{رویداد} & \textbf{وخامت} \\
\midrule
اکتبر ۲۰۱۹ & انتخاب قیس سعید (پوپولیست ضدنظام) با ۷۳٪ & \statuswarn \\
\altrow ۲۰۲۰ & بحران کووید + بحران اقتصادی + بن‌بست سیاسی & \statuswarn \\
ژوئیه ۲۰۲۱ & \textbf{تعلیق پارلمان + اخراج نخست‌وزیر} (مادهٔ ۸۰) & \statusbad \\
\altrow سپتامبر ۲۰۲۱ & تعلیق قانون اساسی ۲۰۱۴ + حکومت فردی & \statusbad \\
مارس ۲۰۲۲ & انحلال شورای عالی قضایی & \statusbad \\
\altrow ژوئیه ۲۰۲۲ & رفراندوم قانون اساسی جدید (مشارکت ۳۰٪: مشروعیت پایین) & \statusbad \\
فوریه ۲۰۲۳ & بازداشت رهبران اپوزیسیون + غنوشی & \statusbad \\
\altrow ۲۰۲۳-۲۰۲۴ & \lr{Freedom House}: «غیرآزاد» — بازگشت کامل & \statusbad \\
\bottomrule
\end{tabularx}
\end{table}

\subsection{علل شکست دموکراسی تونس}

\begin{warningbox}
\textbf{هفت علت شکست دموکراسی تونس — هشدار حیاتی برای ایران:}

\begin{enumerate}[nosep]
\item \textbf{شکست اقتصادی:} دموکراسی نتوانست وضع معیشتی مردم را بهبود دهد. بیکاری جوانان همچنان ۳۵٪+. انقلاب برای «نان و کرامت» بود اما فقط «کرامت سیاسی» آمد.

\item \textbf{بن‌بست سیاسی:} نظام نیمه‌ریاستی ناکارآمد بود. رئیس‌جمهور و نخست‌وزیر دائماً در تضاد بودند.

\item \textbf{فساد پابرجا:} نخبگان رژیم پیشین در اقتصاد باقی ماندند. اصلاحات ضدفساد ناکافی بود.

\item \textbf{خستگی دموکراتیک (\lr{Democratic Fatigue}):} مردم از بی‌ثباتی، چندپارچگی حزبی و جدال‌های سیاسی خسته شدند.

\item \textbf{پوپولیسم:} قیس سعید با شعار «ضد نظام» و «پاکدستی» رأی آورد اما استبداد آورد.

\item \textbf{ضعف جامعهٔ مدنی:} همان \lr{UGTT} که در ۲۰۱۳ ناجی بود، در ۲۰۲۱ ضعیف و منفعل شد.

\item \textbf{فقدان حمایت بین‌المللی مؤثر:} اتحادیهٔ اروپا و آمریکا واکنش ضعیفی نشان دادند.
\end{enumerate}
\end{warningbox}

\sectiondivider

%═══════════════════════════════════════════════════════════
\section{ماتریس درس‌آموخته‌ها برای ایران: الگو + هشدار}
\label{app:tunisia:lessons}
%═══════════════════════════════════════════════════════════

\begin{table}[htbp]
\centering
\caption{ماتریس انتقال درس‌آموخته‌های تونس به ایران (الگو + هشدار)}
\label{tab:app-tunisia-lessons}
\begin{tabularx}{\textwidth}{
  >{\raggedleft\arraybackslash}p{2.2cm}
  >{\raggedleft\arraybackslash}p{3.5cm}
  >{\raggedleft\arraybackslash}X
  >{\centering\arraybackslash}p{1.3cm}
  >{\centering\arraybackslash}p{1cm}
}
\toprule
\headerrow \textbf{بُعد} & \textbf{درس تونس} & \textbf{کاربرد ایرانی} & \textbf{نوع} & \textbf{اهمیت} \\
\midrule
انقلاب مردمی & سرعت (۲۸ روز) + غیرمنتظره & آمادگی برای سناریوی \lr{A/C} & الگو & \rating{4} \\
\altrow
نقش ارتش & امتناع از شلیک = سقوط & آیا بخشی از سپاه جدا می‌شود؟ & الگو & \rating{5} \\
گفت‌وگوی ملی & چهارگانه + اجماع & «چهارگانهٔ ایرانی» & الگو & \rating{5} \\
\altrow
سازش اسلام‌گرایان & غنوشی قدرت واگذار کرد & مدل «دین بدون حکمرانی» & الگو & \rating{5} \\
قانون اساسی & مجلس مؤسسان + اجماع ۹۳٪ & مجلس مؤسسان فراگیر & الگو & \rating{5} \\
\altrow
برابری جنسیتی & مادهٔ ۴۶: \lr{parité} & سهمیهٔ ۳۰٪+ زنان & الگو & \rating{5} \\
\lr{IVD} & بُعد اقتصادی عدالت انتقالی & کمیسیون حقیقت با بُعد مالی & الگو & \rating{5} \\
\altrow
فضای مجازی & فیسبوک = بسیج سریع & استارلینک + ضد اطلاعات نادرست & الگو/هشدار & \rating{4} \\
\midrule
شکست اقتصادی & بیکاری → خستگی → بازگشت & اقتصاد = مشروعیت: اولویت اول & \cellred{هشدار} & \rating{5} \\
\altrow
پوپولیسم & قیس سعید = «ناجی» → مستبد & ضد پوپولیسم: نهادسازی & \cellred{هشدار} & \rating{5} \\
بن‌بست نهادی & نظام نیمه‌ریاستی ناکارآمد & طراحی دقیق نظام سیاسی & \cellred{هشدار} & \rating{4} \\
\altrow
ضعف جامعهٔ مدنی & \lr{UGTT} ۲۰۱۳ vs ۲۰۲۱ & تقویت مستمر جامعهٔ مدنی & \cellred{هشدار} & \rating{4} \\
فقدان حمایت بین‌المللی & واکنش ضعیف \lr{EU/US} & تضمین حمایت بلندمدت & \cellred{هشدار} & \rating{4} \\
\midrule
\headerrow \multicolumn{3}{l}{\textbf{میانگین اهمیت}} & & \textbf{\rating{5}} \\
\bottomrule
\end{tabularx}
\end{table}

\sectiondivider

%═══════════════════════════════════════════════════════════
\section{نمودار: دو مسیر تونس — الگو و هشدار}
\label{app:tunisia:diagram}
%═══════════════════════════════════════════════════════════

\begin{figure}[htbp]
\centering
\begin{tikzpicture}[
  node distance=0.8cm,
  phase/.style={
    draw, rounded corners=5pt, minimum width=2.8cm,
    minimum height=1.2cm, font=\small, align=center,
    thick
  },
  good/.style={phase, fill=MainGreen!15, draw=MainGreen!60},
  bad/.style={phase, fill=MainRed!15, draw=MainRed!60},
  neutral/.style={phase, fill=MainYellow!15, draw=MainYellow!80!black},
  arrow/.style={->, thick, >=stealth},
  fork/.style={->, very thick, >=stealth, dashed}
]

% خط بالا: مسیر موفق (۲۰۱۱-۲۰۱۴)
\node[good] (rev) at (0,3) {انقلاب\\ژانویه ۲۰۱۱};
\node[good] (election) at (3.5,3) {انتخابات\\مؤسسان ۲۰۱۱};
\node[neutral] (crisis) at (7,3) {بحران\\۲۰۱۳};
\node[good] (dialogue) at (10.5,3) {گفت‌وگوی\\ملی ۲۰۱۳};
\node[good] (const) at (14,3) {قانون اساسی\\۲۰۱۴ \cmark};

\draw[arrow, MainGreen] (rev) -- (election);
\draw[arrow, MainYellow!80!black] (election) -- (crisis);
\draw[arrow, MainGreen] (crisis) -- (dialogue);
\draw[arrow, MainGreen] (dialogue) -- (const);

% عنوان
\node[font=\small\bfseries, MainGreen] at (7,4.5) {مسیر موفق: الگو برای ایران};

% خط پایین: مسیر شکست (۲۰۱۹-۲۰۲۳)
\node[neutral] (elect19) at (0,-1) {انتخابات\\۲۰۱۹};
\node[neutral] (saied) at (3.5,-1) {قیس سعید\\«ناجی»};
\node[bad] (econ) at (7,-1) {بحران\\اقتصادی};
\node[bad] (coup) at (10.5,-1) {کودتا\\ژوئیه ۲۰۲۱};
\node[bad] (auth) at (14,-1) {بازگشت\\اقتدارگرایی \xmark};

\draw[arrow, MainYellow!80!black] (elect19) -- (saied);
\draw[arrow, MainRed] (saied) -- (econ);
\draw[arrow, MainRed] (econ) -- (coup);
\draw[arrow, MainRed] (coup) -- (auth);

% عنوان
\node[font=\small\bfseries, MainRed] at (7,-2.5) {مسیر شکست: هشدار برای ایران};

% نقطهٔ انشعاب
\draw[fork, MainPurple] (const.south) -- ++(0,-0.8) -- (elect19.north);
\node[font=\tiny, MainPurple, anchor=east] at (1.5,1) {۵ سال بعد...};

% ایران
\node[draw=MainPurple, fill=MainPurple!10, rounded corners=3pt,
  minimum width=3.5cm, minimum height=1cm, font=\small\bfseries,
  align=center, thick] (iran) at (7,0.7) {ایران باید از هر دو بیاموزد};

\draw[fork, MainGreen] (iran.north) -- ++(0,0.5);
\draw[fork, MainRed] (iran.south) -- ++(0,-0.5);

\end{tikzpicture}
\caption{دو مسیر تونس: الگوی موفقیت (۲۰۱۱-۲۰۱۴) و هشدار شکست (۲۰۱۹-۲۰۲۳)}
\label{fig:app-tunisia-paths}
\end{figure}

\sectiondivider

%═══════════════════════════════════════════════════════════
\section{جمع‌بندی پیوست}
\label{app:tunisia:conclusion}
%═══════════════════════════════════════════════════════════

\begin{chaptersummary}
جمع‌بندی پیوست ت — مطالعهٔ موردی تونس:

\begin{enumerate}[nosep]
\item تونس \textbf{هم الگو و هم هشدار} برای ایران است: الگوی موفقیت (۲۰۱۱-۲۰۱۴) + هشدار شکست (۲۰۲۱+).
\item \textbf{گفت‌وگوی ملی تونس} (چهارگانهٔ نوبل) مهم‌ترین الگوی قابل‌انتقال: جامعهٔ مدنی به‌عنوان میانجی.
\item \textbf{سازش \lr{Ennahda}} بی‌سابقه بود: حزب اسلام‌گرا داوطلبانه قدرت را واگذار کرد و «دولت مدنی» را پذیرفت — مدل «دین بدون حکمرانی».
\item \textbf{قانون اساسی ۲۰۱۴} پیشرفته‌ترین در جهان عرب بود: مادهٔ ۲ (دولت مدنی) + مادهٔ ۴۶ (برابری جنسیتی) + مادهٔ ۶ (آزادی وجدان).
\item \textbf{\lr{IVD}} نوآوری ارزشمندی داشت: بُعد اقتصادی عدالت انتقالی — مستقیماً به ایران قابل‌انتقال.
\item \textbf{هفت علت شکست} تونس مهم‌ترین هشدارهای ایران هستند: شکست اقتصادی، پوپولیسم، بن‌بست نهادی، خستگی دموکراتیک، ضعف جامعهٔ مدنی، فساد پابرجا، و فقدان حمایت بین‌المللی.
\item \textbf{مهم‌ترین درس:} دموکراسی بدون \textbf{بهبود اقتصادی} پایدار نیست. اگر مردم ایران در ۵ سال اول پس از گذار بهبود معیشتی نبینند، خطر «قیس سعید ایرانی» واقعی است.
\item مدل ایران باید از مسیر اول تونس الهام بگیرد (گفت‌وگو + قانون اساسی فراگیر) و از مسیر دوم (شکست اقتصادی + پوپولیسم) اجتناب کند.
\end{enumerate}

\vspace{0.3cm}
\textit{مطالعهٔ تکمیلی:}
\begin{itemize}[nosep]
\item مقایسهٔ جامع ۹ نمونه: \seeChapter{app:comparison}
\item آفریقای جنوبی (\lr{TRC}): \seeChapter{app:south-africa}
\item شیلی (عدالت تدریجی): \seeChapter{app:chile}
\item ریسک بازگشت اقتدارگرایی: \seeChapter{ch:risks}
\item بودجه‌بندی و تأمین اقتصادی: \seeChapter{ch:budget}
\item لهستان و اروپای شرقی: \seeChapter{app:poland}
\end{itemize}
\end{chaptersummary}

\chapterend

%══════════════════════════════════════════════════════════════
% پایان پیوست ت
%══════════════════════════════════════════════════════════════
%══════════════════════════════════════════════════════════════
% پیوست ث: مطالعه موردی لهستان و اروپای شرقی
% فایل: appendices/app-e-poland.tex
% حجم هدف: ۸-۱۰ صفحه
%══════════════════════════════════════════════════════════════

\chapter{مطالعهٔ موردی: لهستان و اروپای شرقی (۱۹۸۹-۲۰۰۴)}
\label{app:poland}

\begin{executivesummary}
لهستان پیشگام گذار دموکراتیک در اروپای شرقی بود و سقوط کمونیسم در این منطقه را رقم زد. ویژگی‌های منحصربه‌فرد: ۱) \textbf{جنبش همبستگی} (\lr{Solidarność}) به‌عنوان بزرگ‌ترین جنبش اجتماعی غیرخشونت‌آمیز قرن بیستم (۱۰ میلیون عضو)، ۲) \textbf{مذاکرات میزگرد} (فوریه-آوریل ۱۹۸۹) که مدل «گذار مذاکره‌ای» در شرایط بحران اقتصادی را ارائه کرد، ۳) \textbf{نقش کلیسای کاتولیک} و شخصیت \person{پاپ ژان‌پل دوم}{\lr{Pope John Paul II}} در مشروعیت‌زدایی از رژیم، ۴) \textbf{مشوق عضویت در \lr{EU} و \lr{NATO}} به‌عنوان قوی‌ترین ابزار تحکیم دموکراسی، و ۵) هشدار اخیر \textbf{بازگشت اقتدارگرایی} (دورهٔ حزب \lr{PiS} ۲۰۱۵-۲۰۲۳). این پیوست همچنین به تجربه‌های مشابه در \textbf{جمهوری چک}، \textbf{مجارستان}، و \textbf{آلمان شرقی} اشاره می‌کند.
\end{executivesummary}

%═══════════════════════════════════════════════════════════
\section{زمینه و بافت تاریخی}
\label{app:poland:context}
%═══════════════════════════════════════════════════════════

\subsection{لهستان کمونیستی: ویژگی‌ها و تفاوت‌ها}

لهستان در بلوک شرقی وضعیت ویژه‌ای داشت — نه تمامیت‌خواه‌ترین (مانند رومانی) و نه آزادترین (مانند مجارستان):

\begin{table}[htbp]
\centering
\caption{مشخصات لهستان در آستانهٔ گذار (۱۹۸۹)}
\label{tab:app-poland-profile}
\begin{tabularx}{\textwidth}{>{\raggedleft\arraybackslash}p{4.5cm} >{\raggedleft\arraybackslash}X}
\toprule
\headerrow \textbf{شاخص} & \textbf{مقدار} \\
\midrule
جمعیت & ۳۸ میلیون نفر \\
\altrow مساحت & ۳۱۲,۰۰۰ \lr{km²} \\
تنوع قومی & بسیار پایین (۹۷٪ لهستانی) \\
\altrow \lr{GDP per capita} & $\sim$\$۱,۷۰۰ \\
طول عمر رژیم کمونیستی & ۴۵ سال (۱۹۴۴-۱۹۸۹) \\
\altrow اندازهٔ ارتش & $\sim$۴۰۰,۰۰۰ \\
حزب حاکم & \lr{PZPR} (حزب متحد کارگران لهستان) \\
\altrow وضعیت اقتصاد & بحران شدید (تورم ۶۰۰٪، صف‌های طولانی) \\
جامعهٔ مدنی & قوی‌ترین در بلوک شرقی (همبستگی) \\
\altrow نقش کلیسا & بسیار بالا (۹۵٪ کاتولیک) \\
حکومت نظامی & ۱۹۸۱-۱۹۸۳ (ژنرال \lr{Jaruzelski}) \\
\altrow سلاح‌های هسته‌ای شوروی & مستقر در خاک لهستان \\
\bottomrule
\end{tabularx}
\end{table}

\subsection{ستون‌های رژیم و مقایسه با ایران}

\begin{table}[htbp]
\centering
\caption{مقایسهٔ ساختاری رژیم کمونیستی لهستان و جمهوری اسلامی}
\label{tab:app-poland-iran-structure}
\begin{tabularx}{\textwidth}{>{\raggedleft\arraybackslash}p{3cm} >{\raggedleft\arraybackslash}X >{\raggedleft\arraybackslash}X}
\toprule
\headerrow \textbf{بُعد} & \textbf{لهستان کمونیستی} & \textbf{جمهوری اسلامی} \\
\midrule
ایدئولوژی & مارکسیسم-لنینیسم (تحمیلی) & اسلام سیاسی / ولایت فقیه \\
\altrow رهبری & حزب واحد (\lr{PZPR}) + شوروی & ولی‌فقیه + سپاه + روحانیت \\
نیروی خارجی حامی & شوروی (تعیین‌کننده) & خودبسنده (فاقد حامی مستقیم) \\
\altrow ارتش & تحت فرمان حزب + پیمان ورشو & سپاه مستقل + ارتش تحت‌فرمان \\
اقتصاد & دولتی-برنامه‌ای (بحرانی) & نیمه‌دولتی (نفت + سپاه + بنیادها) \\
\altrow مذهب & مخالف (اما ناتوان از حذف کلیسا) & پایهٔ مشروعیت (دین = قدرت) \\
جامعهٔ مدنی & قوی (همبستگی ۱۰M) & فعال اما سرکوب‌شده \\
\altrow دیاسپورا & محدود (مهاجرت بعد از گذار) & بسیار بزرگ (۴-۵M) \\
\bottomrule
\end{tabularx}
\end{table}

\begin{casestudy}
\textbf{تفاوت ساختاری کلیدی:} رژیم کمونیستی لهستان \textbf{وابسته به شوروی} بود — با فروپاشی شوروی، پایهٔ خارجی رژیم از بین رفت. جمهوری اسلامی فاقد «شوروی» است و \textbf{خودبسنده‌تر} عمل می‌کند. در عوض، تفاوت دوم مهم‌تر است: لهستان \textbf{مشوق عضویت \lr{EU} و \lr{NATO}} داشت — قوی‌ترین ابزار تحکیم. ایران فاقد معادل این مشوق است (\seeChapter{ch:guarantees}).
\end{casestudy}

\sectiondivider

%═══════════════════════════════════════════════════════════
\section{جنبش همبستگی: الگوی جنبش اجتماعی}
\label{app:poland:solidarity}
%═══════════════════════════════════════════════════════════

\subsection{تاریخچه و ویژگی‌ها}

\org{اتحادیهٔ کارگری مستقل و خودگردان همبستگی}{\lr{Niezależny Samorządny Związek Zawodowy "Solidarność"}} در اوت ۱۹۸۰ در کارخانهٔ کشتی‌سازی گدانسک تأسیس شد:

\begin{table}[htbp]
\centering
\caption{ویژگی‌های جنبش همبستگی و مقایسه با اپوزیسیون ایران}
\label{tab:app-poland-solidarity}
\begin{tabularx}{\textwidth}{>{\raggedleft\arraybackslash}p{3.5cm} >{\raggedleft\arraybackslash}X >{\raggedleft\arraybackslash}p{3.5cm}}
\toprule
\headerrow \textbf{ویژگی} & \textbf{همبستگی} & \textbf{اپوزیسیون ایران} \\
\midrule
سال تأسیس & ۱۹۸۰ & فاقد تشکیلات واحد \\
\altrow اعضا (اوج) & ۱۰ میلیون (از ۳۸M) = ۲۶٪ & فاقد عضویت رسمی \\
رهبر & \person{لخ والسا}{\lr{Lech Wałęsa}} (نوبل ۱۹۸۳) & فاقد رهبر واحد \\
\altrow ماهیت & اتحادیهٔ کارگری + جنبش اجتماعی & شبکه‌های پراکنده \\
ایدئولوژی & فراگیر (کارگر + روشنفکر + مذهبی) & متنوع (گاه متضاد) \\
\altrow پشتیبان نهادی & کلیسای کاتولیک & فاقد پشتیبان نهادی \\
ابزار & اعتصاب + نافرمانی مدنی + نشریه & اعتراض خیابانی + فضای مجازی \\
\altrow ممنوعیت & ممنوع ۱۹۸۱-۱۹۸۹ (اما فعال مخفیانه) & سرکوب مستمر \\
بازگشت & ۱۹۸۹: قانونی شد و انتخابات برد & -- \\
\bottomrule
\end{tabularx}
\end{table}

\subsection{رمز موفقیت همبستگی}

\begin{enumerate}[nosep]
\item \textbf{پایگاه طبقاتی:} ریشه در \textbf{طبقهٔ کارگر} — ایدئولوژی رژیم نمی‌توانست کارگران را «دشمن طبقاتی» بنامد
\item \textbf{فراگیری:} روشنفکران (\lr{KOR}) + کارگران + کلیسا + دانشجویان = ائتلاف فراطبقاتی
\item \textbf{حمایت نهادی کلیسا:} کلیسای کاتولیک فضا، پول، و مشروعیت فراهم کرد
\item \textbf{رهبری کاریزماتیک:} والسا نماد وحدت بود (با وجود ضعف‌هایش)
\item \textbf{خشونت‌پرهیزی:} حتی در حکومت نظامی (۱۹۸۱)، جنبش به خشونت روی نیاورد
\item \textbf{ساختار سازمانی:} حتی در دوران ممنوعیت، شبکهٔ زیرزمینی (\lr{podziemie}) فعال بود: ۵۰۰+ نشریهٔ مخفی
\end{enumerate}

\begin{keypoint}
\textbf{فاصلهٔ اپوزیسیون ایران با مدل همبستگی:} اپوزیسیون ایران فاقد سه عنصر کلیدی همبستگی است: ۱) \textbf{تشکیلات گسترده} (۱۰M عضو)؛ ۲) \textbf{رهبر واحد} (والسا)؛ ۳) \textbf{پشتیبان نهادی} (کلیسا). جنبش «زن، زندگی، آزادی» پتانسیل فراگیری دارد اما هنوز \textbf{ساختار سازمانی} ندارد. \emphred{توصیه:} تشکیل شبکه‌های صنفی-حرفه‌ای (معلمان، پرستاران، کارگران، وکلا) به‌عنوان «همبستگی‌های کوچک» که در لحظهٔ مناسب به «همبستگی بزرگ» تبدیل شوند (\seeChapter{ch:actors}).
\end{keypoint}

\sectiondivider

%═══════════════════════════════════════════════════════════
\section{مذاکرات میزگرد: مدل گذار مذاکره‌ای}
\label{app:poland:roundtable}
%═══════════════════════════════════════════════════════════

\subsection{زمینه‌سازی: چرا رژیم مذاکره کرد}

\begin{enumerate}[nosep]
\item \textbf{بحران اقتصادی شدید:} تورم ۶۰۰٪ + صف‌های طولانی + کمبود کالا
\item \textbf{موج اعتصابات ۱۹۸۸:} کارخانه‌ها، معادن، بنادر
\item \textbf{سیاست گورباچف:} \lr{Perestroika} + \lr{Glasnost} = شوروی دیگر مداخله نمی‌کند
\item \textbf{شکاف درون حزب:} جناح اصلاح‌طلب (ژنرال \person{یاروزلسکی}{\lr{Jaruzelski}} + وزیر کشور \person{کیشچاک}{\lr{Kiszczak}})
\item \textbf{واقع‌بینی:} رژیم فهمید بدون مشروعیت مردمی نمی‌تواند بحران اقتصادی را حل کند
\end{enumerate}

\subsection{ساختار میزگرد}

\begin{table}[htbp]
\centering
\caption{مذاکرات میزگرد لهستان (۶ فوریه — ۵ آوریل ۱۹۸۹)}
\label{tab:app-poland-roundtable}
\begin{tabularx}{\textwidth}{>{\raggedleft\arraybackslash}p{3cm} >{\raggedleft\arraybackslash}X}
\toprule
\headerrow \textbf{بُعد} & \textbf{جزئیات} \\
\midrule
مدت & ۵۹ روز (۶ فوریه — ۵ آوریل ۱۹۸۹) \\
\altrow مکان & کاخ نامیستنیکوفسکی، ورشو \\
شرکت‌کنندگان & ۵۷ نفر (۲۹ حکومتی + ۲۶ اپوزیسیون + ۲ ناظر کلیسا) \\
\altrow سه کارگروه اصلی & ۱) اصلاحات سیاسی، ۲) اقتصاد و سیاست اجتماعی، ۳) تکثرگرایی اتحادیه‌ای \\
تعداد کارگروه‌های فرعی & ۱۱ (آموزش، بهداشت، محیط‌زیست، قضا...) \\
\altrow مشارکت‌کنندگان کل & بیش از ۴۵۰ نفر (در همهٔ سطوح) \\
\bottomrule
\end{tabularx}
\end{table}

\subsection{توافقات میزگرد}

\begin{enumerate}[nosep]
\item \textbf{قانونی‌سازی همبستگی:} اتحادیه دوباره قانونی شد
\item \textbf{انتخابات نیمه‌آزاد:} سنا: ۱۰۰٪ آزاد | مجلس (سِیم): ۶۵٪ رزرو برای حزب، ۳۵٪ آزاد
\item \textbf{ریاست‌جمهوری:} یاروزلسکی رئیس‌جمهور می‌ماند (تضمین امنیتی)
\item \textbf{مجلس سنا:} نهاد جدید کاملاً آزاد (۱۰۰ کرسی)
\item \textbf{اصلاحات اقتصادی:} آزادسازی تدریجی
\item \textbf{اصلاح قضایی:} استقلال نسبی دادگاه‌ها
\end{enumerate}

\begin{lessonlearned}
\textbf{شگفتی انتخابات ژوئن ۱۹۸۹:} رژیم انتظار داشت با ۶۵٪ کرسی تضمین‌شده، قدرت را حفظ کند. اما همبستگی در بخش آزاد (۳۵٪ مجلس + ۱۰۰٪ سنا) \textbf{تقریباً تمام کرسی‌ها} را برد: ۹۹ از ۱۰۰ کرسی سنا + ۱۶۰ از ۱۶۱ کرسی آزاد مجلس. رژیم شوکه شد. سپس احزاب کوچک ائتلافی رژیم منشعب شدند و به همبستگی پیوستند. نتیجه: \person{تادئوش مازوویتسکی}{\lr{Tadeusz Mazowiecki}} اولین نخست‌وزیر غیرکمونیست اروپای شرقی شد (اوت ۱۹۸۹). \emphgreen{درس ایرانی:} حتی انتخابات «مهندسی‌شده» می‌تواند ابزار گذار باشد — اگر اپوزیسیون متحد و بسیج‌شده باشد.
\end{lessonlearned}

\sectiondivider

%═══════════════════════════════════════════════════════════
\section{نقش کلیسای کاتولیک: نهاد میانجی}
\label{app:poland:church}
%═══════════════════════════════════════════════════════════

\subsection{کلیسا به‌عنوان «فضای آزاد»}

\begin{table}[htbp]
\centering
\caption{نقش‌های کلیسای کاتولیک در گذار لهستان و معادل ایرانی}
\label{tab:app-poland-church}
\begin{tabularx}{\textwidth}{>{\raggedleft\arraybackslash}p{3cm} >{\raggedleft\arraybackslash}X >{\raggedleft\arraybackslash}p{3.5cm}}
\toprule
\headerrow \textbf{نقش} & \textbf{توضیح} & \textbf{معادل ایرانی} \\
\midrule
فضای فیزیکی & کلیساها محل گردهمایی و پناهگاه بودند & مساجد مستقل؟ خانه‌ها؟ فضای مجازی \\
\altrow مشروعیت‌زدایی & پاپ ژان‌پل دوم: «نترسید!» (۱۹۷۹) & روحانیون مستقل (مانند منتظری) \\
میانجی‌گری & دو ناظر کلیسا در میزگرد & ایران: نهاد میانجی‌گر مستقل \\
\altrow حمایت مالی & کمک مالی به خانواده‌های زندانیان & دیاسپورا + \lr{NGO}ها \\
شبکهٔ ارتباطی & ۱۰,۰۰۰+ کلیسا در سراسر کشور & شبکه‌های صنفی-حرفه‌ای \\
\altrow مشروعیت اخلاقی & پاپ = اقتدار اخلاقی جهانی & کیست؟ (چالش ایران) \\
\bottomrule
\end{tabularx}
\end{table}

\begin{warningbox}
\textbf{تفاوت حیاتی:} در لهستان، کلیسا \textbf{مخالف} رژیم بود و \textbf{حامی} جنبش. در ایران، نهاد مذهبی \textbf{بخشی از} رژیم است. بنابراین ایران نمی‌تواند روی «کلیسای خود» حساب کند — مگر بخش‌هایی از روحانیت مستقل (مانند مرحوم \person{منتظری}{\lr{Montazeri}} یا روحانیون منتقد). \emphred{جایگزین پیشنهادی:} نقش کلیسا در لهستان را در ایران باید ترکیبی از \textbf{تشکل‌های صنفی}، \textbf{دانشگاه‌ها}، و \textbf{فضای مجازی} ایفا کنند (\seeChapter{ch:actors}).
\end{warningbox}

\sectiondivider

%═══════════════════════════════════════════════════════════
\section{اصلاحات اقتصادی: شوک‌تراپی و پیامدها}
\label{app:poland:economy}
%═══════════════════════════════════════════════════════════

\subsection{برنامهٔ بالتسروویچ (\lr{Shock Therapy})}

دولت مازوویتسکی در ژانویه ۱۹۹۰ برنامهٔ \person{لشک بالتسروویچ}{\lr{Leszek Balcerowicz}} را اجرا کرد:

\begin{table}[htbp]
\centering
\caption{شوک‌تراپی لهستان: دستاوردها و هزینه‌ها}
\label{tab:app-poland-shock}
\begin{tabularx}{\textwidth}{>{\centering\arraybackslash}p{1cm} >{\raggedleft\arraybackslash}X >{\centering\arraybackslash}p{2cm}}
\toprule
\headerrow & \textbf{دستاوردها} & \textbf{زمان} \\
\midrule
\cmark & کاهش تورم از ۶۰۰٪ به ۳۵٪ & ۱ سال \\
\altrow \cmark & آزادسازی قیمت‌ها & فوری \\
\cmark & خصوصی‌سازی ۸,۰۰۰+ شرکت دولتی & ۵-۱۰ سال \\
\altrow \cmark & تبدیل‌پذیری ارز (زلوتی) & ۱ سال \\
\cmark & رشد اقتصادی پایدار (از ۱۹۹۲) & ۲ سال \\
\altrow \cmark & عضویت \lr{WTO} (۱۹۹۵)، \lr{OECD} (۱۹۹۶)، \lr{EU} (۲۰۰۴) & ۵-۱۵ سال \\
\midrule
\headerrow & \textbf{هزینه‌ها} & \textbf{شدت} \\
\midrule
\xmark & بیکاری از ۰٪ (رسمی) به ۱۶٪ & \riskhigh \\
\altrow \xmark & افزایش نابرابری (ضریب جینی: ۰.۲۶ → ۰.۳۵) & \riskmedium \\
\xmark & کاهش \lr{GDP} ۱۱.۶٪ (۱۹۹۰-۱۹۹۱) & \riskhigh \\
\altrow \xmark & فقر موقت (۲۰٪+ زیر خط فقر) & \riskhigh \\
\xmark & نارضایتی اجتماعی و «شکست‌خوردگان» گذار (\lr{Transition Losers}) & \riskmedium \\
\bottomrule
\end{tabularx}
\end{table}

\begin{lessonlearned}
\textbf{درس شوک‌تراپی لهستان برای ایران:} ۱) \textbf{شوک‌تراپی کار می‌کند اما دردناک است} — ایران با جمعیت ۸۵M نمی‌تواند ریسک بیکاری ۱۶٪ را بپذیرد (= ۷ میلیون بیکار جدید)؛ ۲) مدل شیلی (اصلاح تدریجی) برای ایران مناسب‌تر از مدل لهستان است؛ ۳) \textbf{«شکست‌خوردگان گذار»} خطرناک‌اند — آنها در لهستان به پوپولیسم حزب \lr{PiS} رأی دادند. در ایران ممکن است به \textbf{بازگشت اقتدارگرایی} رأی دهند (\seeChapter{ch:risks}).
\end{lessonlearned}

\sectiondivider

%═══════════════════════════════════════════════════════════
\section{لوستراسیون: پاکسازی هوشمند}
\label{app:poland:lustration}
%═══════════════════════════════════════════════════════════

\subsection{مدل لهستان: تأخیری اما مؤثر}

\termfn{لوستراسیون}{Lustration} (بررسی سوابق مقامات رژیم پیشین) در اروپای شرقی با مدل‌های متفاوتی اجرا شد:

\begin{table}[htbp]
\centering
\caption{مقایسهٔ مدل‌های لوستراسیون در اروپای شرقی}
\label{tab:app-poland-lustration}
\begin{tabularx}{\textwidth}{>{\raggedleft\arraybackslash}p{2.5cm} >{\raggedleft\arraybackslash}X >{\centering\arraybackslash}p{2cm} >{\centering\arraybackslash}p{2cm}}
\toprule
\headerrow \textbf{کشور} & \textbf{مدل لوستراسیون} & \textbf{زمان} & \textbf{شدت} \\
\midrule
جمهوری چک & جامع: ممنوعیت ۵ ساله از مناصب دولتی & ۱۹۹۱ & \riskhigh \\
\altrow آلمان شرقی & باز کردن آرشیو اشتازی (\lr{BStU}) + پاکسازی & ۱۹۹۱ & \riskhigh \\
لهستان & تأخیری: قانون ۱۹۹۷ + آرشیو \lr{IPN} (۲۰۰۰) & ۱۹۹۷ & \riskmedium \\
\altrow مجارستان & حداقلی: فقط افشاسازی داوطلبانه & ۱۹۹۴ & \risklow \\
رومانی & بسیار ضعیف: مقامات سابق در قدرت ماندند & تأخیر طولانی & \risklow \\
\bottomrule
\end{tabularx}
\end{table}

\subsection{مؤسسهٔ حافظهٔ ملی (\lr{IPN})}

\org{مؤسسهٔ حافظهٔ ملی}{\lr{Instytut Pamięci Narodowej (IPN)}} در ۱۹۹۸ تأسیس شد:

\begin{itemize}[nosep]
\item \textbf{مأموریت:} نگهداری آرشیو پلیس مخفی (\lr{SB}) + تحقیق + آموزش + تعقیب جنایات
\item \textbf{حجم آرشیو:} ۹۰ کیلومتر طولی اسناد!
\item \textbf{حق دسترسی:} هر شهروند حق دسترسی به پروندهٔ خود را دارد
\item \textbf{افشاسازی:} مقامات عمومی باید همکاری با \lr{SB} را اعتراف کنند
\end{itemize}

\begin{recommendation}
\textbf{مدل \lr{IPN} برای ایران:} تشکیل «مؤسسهٔ حافظهٔ ملی ایران» با مأموریت: ۱) حفظ آرشیو وزارت اطلاعات + سپاه + سازمان زندان‌ها — \textbf{فوری در ۷۲ ساعت اول} (جلوگیری از نابودی اسناد)؛ ۲) حق دسترسی زندانیان سیاسی سابق به پرونده‌هایشان؛ ۳) افشای شبکهٔ خبرچینان (\textbf{با احتیاط} — درس آلمان: برخی افشاسازی‌ها خانواده‌ها را ویران کرد)؛ ۴) تحقیق تاریخی مستقل؛ ۵) آموزش نسل جدید (\seeChapter{ch:requirements}).
\end{recommendation}

\sectiondivider

%═══════════════════════════════════════════════════════════
\section{مشوق عضویت \lr{EU/NATO}: قوی‌ترین ابزار تحکیم}
\label{app:poland:eu}
%═══════════════════════════════════════════════════════════

\subsection{مشروطیت اروپایی (\lr{EU Conditionality})}

مهم‌ترین عامل تحکیم دموکراسی در لهستان و اروپای شرقی، \termfn{مشروطیت اروپایی}{\lr{EU Conditionality}} بود:

\begin{table}[htbp]
\centering
\caption{مراحل ادغام لهستان در نهادهای غربی و تأثیر بر دموکراسی}
\label{tab:app-poland-eu}
\begin{tabularx}{\textwidth}{>{\centering\arraybackslash}p{2cm} >{\raggedleft\arraybackslash}X >{\centering\arraybackslash}p{2.5cm}}
\toprule
\headerrow \textbf{سال} & \textbf{رویداد} & \textbf{تأثیر} \\
\midrule
۱۹۹۱ & شورای اروپا (\lr{Council of Europe}) & معیارهای حقوق بشری \\
\altrow ۱۹۹۱ & برنامهٔ \lr{PHARE} (کمک مالی \lr{EC}) & اصلاحات اقتصادی \\
۱۹۹۳ & \textbf{معیارهای کپنهاگ:} دموکراسی + اقتصاد بازار + ظرفیت اجرایی & چارچوب الزام‌آور \\
\altrow ۱۹۹۴ & درخواست عضویت \lr{EU} & فشار اصلاحات \\
۱۹۹۷ & آغاز مذاکرات الحاق & ۳۱ فصل اصلاحات \\
\altrow ۱۹۹۹ & عضویت \lr{NATO} & تضمین امنیتی \\
۲۰۰۴ & \textbf{عضویت \lr{EU}} & تحکیم نهایی \\
\altrow ۲۰۰۷ & عضویت شنگن & آزادی رفت‌وآمد \\
\bottomrule
\end{tabularx}
\end{table}

\begin{keypoint}
\textbf{چرا مشوق \lr{EU} کار کرد:} ۱) \textbf{ملموس بودن:} مردم می‌دانستند عضویت = ویزای آزاد + کمک مالی + سرمایه‌گذاری؛ ۲) \textbf{الزام‌آور بودن:} معیارهای کپنهاگ شفاف و قابل‌اندازه‌گیری بودند؛ ۳) \textbf{مرحله‌ای بودن:} هر مرحله جایزه‌ای داشت؛ ۴) \textbf{برگشت‌ناپذیری نسبی:} عضویت \lr{EU} بازگشت به اقتدارگرایی را بسیار دشوار (اگرچه نه غیرممکن — مجارستان) کرد. \emphred{چالش ایرانی:} \textbf{معادل \lr{EU} برای ایران وجود ندارد.} هیچ سازمان منطقه‌ای مشابهی نیست. باید \textbf{بستهٔ ترکیبی} طراحی شود (\seeChapter{ch:guarantees}).
\end{keypoint}

\subsection{بستهٔ مشوق ترکیبی پیشنهادی برای ایران}

\begin{table}[htbp]
\centering
\caption{بستهٔ مشوق ترکیبی برای ایران: جایگزین عضویت \lr{EU}}
\label{tab:app-poland-iran-incentives}
\begin{tabularx}{\textwidth}{>{\raggedleft\arraybackslash}p{3cm} >{\raggedleft\arraybackslash}X >{\centering\arraybackslash}p{2cm}}
\toprule
\headerrow \textbf{مشوق} & \textbf{توضیح} & \textbf{فاز اجرا} \\
\midrule
لغو تحریم‌ها (مرحله‌ای) & هر مرحلهٔ اصلاحات = رفع بخشی از تحریم‌ها & فاز ۱-۳ \\
\altrow عضویت \lr{WTO} & ادغام در اقتصاد جهانی & فاز ۲-۳ \\
توافق مشارکت با \lr{EU} & مشابه توافق‌های شمال آفریقا & فاز ۲-۴ \\
\altrow سرمایه‌گذاری مستقیم خارجی & کنفرانس کمک‌دهندگان + صندوق بازسازی & فاز ۱-۲ \\
عضویت در شورای حقوق بشر & بازگشت به جامعهٔ بین‌المللی & فاز ۳ \\
\altrow برنامهٔ بورس تحصیلی گسترده & ۱۰,۰۰۰+ بورس/سال برای دانشجویان ایرانی & فاز ۱-۴ \\
تضمین امنیتی منطقه‌ای & پیمان عدم تجاوز + تضمین مرزها & فاز ۲-۳ \\
\altrow حل مسئلهٔ هسته‌ای & الحاق به \lr{NPT} + پروتکل الحاقی = لغو تحریم & فاز ۱-۲ \\
\bottomrule
\end{tabularx}
\end{table}

\sectiondivider

%═══════════════════════════════════════════════════════════
\section{هشدار: بازگشت اقتدارگرایی (حزب \lr{PiS})}
\label{app:poland:backsliding}
%═══════════════════════════════════════════════════════════

\subsection{لهستان ۲۰۱۵-۲۰۲۳: تضعیف دموکراسی از درون \lr{EU}}

حتی با عضویت \lr{EU}، لهستان تحت حزب \org{قانون و عدالت}{\lr{Prawo i Sprawiedliwość (PiS)}} دچار بازگشت اقتدارگرایی شد:

\begin{table}[htbp]
\centering
\caption{اقدامات ضددموکراتیک حزب \lr{PiS} (۲۰۱۵-۲۰۲۳)}
\label{tab:app-poland-pis}
\begin{tabularx}{\textwidth}{>{\centering\arraybackslash}p{2cm} >{\raggedleft\arraybackslash}X >{\centering\arraybackslash}p{2cm}}
\toprule
\headerrow \textbf{سال} & \textbf{اقدام} & \textbf{وخامت} \\
\midrule
۲۰۱۵ & فلج دادگاه قانون اساسی (انتصاب قاضیان حزبی) & \statusbad \\
\altrow ۲۰۱۶ & کنترل رسانهٔ عمومی (\lr{TVP}) & \statusbad \\
۲۰۱۷ & تلاش برای تسلط بر قوهٔ قضاییه (اعتراض \lr{EU}) & \statusbad \\
\altrow ۲۰۱۸ & محدودیت آزادی تجمع + تاریخ‌نگاری (\lr{IPN}) & \statuswarn \\
۲۰۲۰ & ممنوعیت شبه‌کامل سقط‌جنین (بحران اجتماعی) & \statuswarn \\
\altrow ۲۰۲۱ & ابزار جاسوسی \lr{Pegasus} علیه اپوزیسیون & \statusbad \\
اکتبر ۲۰۲۳ & \textbf{شکست \lr{PiS} در انتخابات}: بازگشت لیبرال‌ها (\person{تاسک}{\lr{Tusk}}) & \statusok \\
\bottomrule
\end{tabularx}
\end{table}

\begin{warningbox}
\textbf{درس بازگشت لهستانی:} حتی ۳۰ سال پس از گذار و ۲۰ سال پس از عضویت \lr{EU}، دموکراسی آسیب‌پذیر است. عوامل بازگشت: ۱) \textbf{«شکست‌خوردگان گذار»:} مناطق فقیر شرقی لهستان به \lr{PiS} رأی دادند؛ ۲) \textbf{نابرابری منطقه‌ای}؛ ۳) \textbf{پوپولیسم ناسیونالیستی-مذهبی}؛ ۴) \textbf{ابزار سازی از ارزش‌های سنتی} (ضد مهاجرت، ضد \lr{LGBT}). نکتهٔ مثبت: دموکراسی لهستان \textbf{اصلاح‌پذیر} بود — مردم در ۲۰۲۳ \lr{PiS} را کنار زدند. \emphred{درس ایرانی:} تحکیم دموکراسی فرآیندی بی‌پایان است و نیازمند نهادهای مستقل، رسانهٔ آزاد و جامعهٔ مدنی هوشیار.
\end{warningbox}

\sectiondivider

%═══════════════════════════════════════════════════════════
\section{ماتریس درس‌آموخته‌ها برای ایران}
\label{app:poland:lessons}
%═══════════════════════════════════════════════════════════

\begin{table}[htbp]
\centering
\caption{ماتریس انتقال درس‌آموخته‌های لهستان و اروپای شرقی به ایران}
\label{tab:app-poland-lessons}
\begin{tabularx}{\textwidth}{
  >{\raggedleft\arraybackslash}p{2.2cm}
  >{\raggedleft\arraybackslash}p{3.5cm}
  >{\raggedleft\arraybackslash}X
  >{\centering\arraybackslash}p{1.5cm}
}
\toprule
\headerrow \textbf{بُعد} & \textbf{درس لهستان} & \textbf{کاربرد ایرانی} & \textbf{انتقال‌پذیری} \\
\midrule
جنبش اجتماعی & همبستگی ۱۰M: سازمان‌یافته & شبکه‌های صنفی → ائتلاف فراگیر & \rating{4} \\
\altrow
مذاکرات میزگرد & ۵۹ روز، ۴۵۰+ شرکت‌کننده & «میزگرد ایران» در فاز ۱ & \rating{5} \\
نقش نهاد مذهبی & کلیسا = میانجی + فضای آزاد & روحانیون مستقل + تشکل‌های صنفی & \rating{3} \\
\altrow
لوستراسیون & \lr{IPN} + آرشیو + افشاسازی & مؤسسهٔ حافظهٔ ملی ایران & \rating{5} \\
شوک‌تراپی & مؤثر اما دردناک & اصلاح تدریجی (مدل شیلی) مناسب‌تر & \rating{2} \\
\altrow
مشوق \lr{EU/NATO} & قوی‌ترین ابزار تحکیم & بستهٔ ترکیبی (تحریم + \lr{WTO} + سرمایه‌گذاری) & \rating{3} \\
انتخابات نیمه‌آزاد & حتی مهندسی‌شده ابزار گذار شد & رفراندوم / انتخابات تحت نظارت & \rating{4} \\
\altrow
بازگشت (\lr{PiS}) & ۳۰ سال بعد هم آسیب‌پذیر & نهادسازی مستمر + رسانهٔ آزاد & \rating{4} \\
اثر دومینو & لهستان ← مجارستان ← چک ← آلمان شرقی & ایران ← منطقه؟ & \rating{3} \\
\midrule
\headerrow \multicolumn{3}{l}{\textbf{میانگین انتقال‌پذیری}} & \textbf{\rating{4}} \\
\bottomrule
\end{tabularx}
\end{table}

\sectiondivider

%═══════════════════════════════════════════════════════════
\section{مقایسهٔ مختصر: سایر کشورهای اروپای شرقی}
\label{app:poland:others}
%═══════════════════════════════════════════════════════════

\begin{table}[htbp]
\centering
\caption{مقایسهٔ مدل‌های گذار در اروپای شرقی}
\label{tab:app-poland-eastern-europe}
\begin{tabularx}{\textwidth}{
  >{\raggedleft\arraybackslash}p{2.2cm}
  >{\raggedleft\arraybackslash}p{2.5cm}
  >{\raggedleft\arraybackslash}p{2.5cm}
  >{\raggedleft\arraybackslash}X
  >{\centering\arraybackslash}p{1.5cm}
}
\toprule
\headerrow \textbf{کشور} & \textbf{نوع گذار} & \textbf{ویژگی خاص} & \textbf{درس برای ایران} & \textbf{نتیجه} \\
\midrule
لهستان & میزگرد مذاکره‌ای & همبستگی + کلیسا & مذاکره + جنبش & \statusok \\
\altrow مجارستان & اصلاح از بالا + میزگرد & تحول تدریجی + حزب حاکم منشعب شد & شکاف نخبگان & \statuswarn \\
چکسلواکی & انقلاب مخملی (۱۰ روز!) & سرعت + خشونت‌پرهیزی + هاول & رهبر اخلاقی & \statusok \\
\altrow آلمان شرقی & فروپاشی + ادغام & \lr{Stasi} = آرشیو + اتحاد با غرب & آرشیو + حمایت مالی & \statusok \\
رومانی & انقلاب خشونت‌آمیز & چائوشسکو اعدام شد + نخبگان قدیم ماندند & خشونت = عدم پاکسازی & \statuswarn \\
\altrow بلغارستان & اصلاح تدریجی & حزب کمونیست خود را اصلاح کرد & اصلاح‌طلبان درون نظام & \statuswarn \\
\bottomrule
\end{tabularx}
\end{table}

\begin{lessonlearned}
\textbf{اثر دومینو (\lr{Domino Effect}):} سقوط لهستان در ۱۹۸۹ طی ۶ ماه همهٔ اروپای شرقی را فرو ریخت: لهستان (ژوئن) ← مجارستان (اکتبر) ← آلمان شرقی (نوامبر: دیوار برلین) ← چکسلواکی (نوامبر) ← رومانی (دسامبر). \emphgreen{سؤال ایرانی:} آیا گذار ایران اثر دومینو در خاورمیانه خواهد داشت؟ تجربهٔ بهار عربی (۲۰۱۱) نشان داد که بله، اما نتایج لزوماً مثبت نیست (لیبی، سوریه، یمن). مدل ۶ باید اثرات منطقه‌ای را نیز مدیریت کند (\seeChapter{ch:risks}).
\end{lessonlearned}

\sectiondivider

%═══════════════════════════════════════════════════════════
\section{جمع‌بندی پیوست}
\label{app:poland:conclusion}
%═══════════════════════════════════════════════════════════

\begin{chaptersummary}
جمع‌بندی پیوست ث — لهستان و اروپای شرقی:

\begin{enumerate}[nosep]
\item \textbf{جنبش همبستگی} الگوی بی‌نظیر جنبش اجتماعی غیرخشونت‌آمیز است: ۱۰M عضو، ساختار سازمانی، رهبر کاریزماتیک. اپوزیسیون ایران هنوز فاصلهٔ زیادی با این مدل دارد.
\item \textbf{مذاکرات میزگرد} نشان داد که حتی رژیم‌های ایدئولوژیک در شرایط بحران اقتصادی حاضر به مذاکره می‌شوند — مشروط به فشار مردمی + شکاف نخبگان.
\item \textbf{نقش کلیسا} در لهستان بی‌نظیر بود و \textbf{مستقیماً} به ایران قابل‌انتقال نیست — جایگزین: تشکل‌های صنفی + دانشگاه‌ها + فضای مجازی.
\item \textbf{لوستراسیون و \lr{IPN}} بهترین مدل حفظ آرشیو و حافظهٔ جمعی است — «مؤسسهٔ حافظهٔ ملی ایران» باید در ۷۲ ساعت اول تأسیس شود.
\item \textbf{شوک‌تراپی} برای ایران مناسب نیست — مدل تدریجی شیلی بهتر است.
\item \textbf{مشوق عضویت \lr{EU/NATO}} قوی‌ترین ابزار تحکیم بود — ایران فاقد معادل است و باید بستهٔ ترکیبی طراحی کند.
\item حتی با عضویت \lr{EU}، \textbf{بازگشت اقتدارگرایی ممکن است} (\lr{PiS} ۲۰۱۵-۲۰۲۳) — تحکیم فرآیندی بی‌پایان است.
\item \textbf{اثر دومینو} لهستان الهام‌بخش بود اما بهار عربی نشان داد دومینو لزوماً مثبت نیست.
\end{enumerate}

\vspace{0.3cm}
\textit{مطالعهٔ تکمیلی:}
\begin{itemize}[nosep]
\item مقایسهٔ جامع ۹ نمونه: \seeChapter{app:comparison}
\item آفریقای جنوبی (مدل اصلی): \seeChapter{app:south-africa}
\item تونس (گفت‌وگوی ملی): \seeChapter{app:tunisia}
\item نهادها و بازیگران: \seeChapter{ch:actors}
\item عراق (ضد الگو): \seeChapter{app:iraq}
\end{itemize}
\end{chaptersummary}

\chapterend

%══════════════════════════════════════════════════════════════
% پایان پیوست ث
%══════════════════════════════════════════════════════════════
%══════════════════════════════════════════════════════════════
% پیوست ج: مطالعه موردی عراق — نمونهٔ منفی
% فایل: appendices/app-f-iraq.tex
% حجم هدف: ۸-۱۰ صفحه
%══════════════════════════════════════════════════════════════

\chapter{مطالعهٔ موردی: عراق — نمونهٔ منفی (۲۰۰۳-۲۰۱۰)}
\label{app:iraq}

\begin{executivesummary}
عراق \textbf{مهم‌ترین ضد الگو} (\lr{Counter-Model}) در این کتاب است. مداخلهٔ نظامی آمریکا در مارس ۲۰۰۳ رژیم \person{صدام حسین}{\lr{Saddam Hussein}} را سرنگون کرد اما به‌جای دموکراسی، \textbf{جنگ داخلی فرقه‌ای}، \textbf{فروپاشی دولت}، و \textbf{ظهور داعش} را به ارمغان آورد. هزینهٔ انسانی: بیش از ۲۰۰,۰۰۰ کشته. هزینهٔ مالی: بیش از ۲ تریلیون دلار (برای آمریکا). هر تصمیم اشتباهی که در گذار ممکن است گرفته شود، در عراق گرفته شد: مداخلهٔ نظامی بدون برنامه، انحلال ارتش، اجتثاث بدون عدالت، بی‌توجهی به بافت فرقه‌ای-قومی، و تحمیل مدل خارجی. این پیوست هفت «اشتباه مهلک» عراق را تحلیل و \textbf{ضد درس‌آموخته‌ها} (\lr{Anti-Lessons}) را برای ایران استخراج می‌کند.
\end{executivesummary}

%═══════════════════════════════════════════════════════════
\section{زمینه و بافت تاریخی}
\label{app:iraq:context}
%═══════════════════════════════════════════════════════════

\subsection{رژیم بعث: ساختار و ویژگی‌ها}

\begin{table}[htbp]
\centering
\caption{مشخصات عراق در آستانهٔ سقوط (۲۰۰۳)}
\label{tab:app-iraq-profile}
\begin{tabularx}{\textwidth}{>{\raggedleft\arraybackslash}p{4.5cm} >{\raggedleft\arraybackslash}X}
\toprule
\headerrow \textbf{شاخص} & \textbf{مقدار} \\
\midrule
جمعیت & ۲۶ میلیون نفر \\
\altrow مساحت & ۴۳۸,۰۰۰ \lr{km²} \\
ترکیب قومی-مذهبی & شیعه ۶۰٪ + سنی ۲۰٪ + کرد ۱۵٪ + سایر ۵٪ \\
\altrow \lr{GDP per capita} & $\sim$\$۹۰۰ (پس از ۱۲ سال تحریم) \\
طول عمر رژیم بعث & ۳۵ سال (۱۹۶۸-۲۰۰۳) \\
\altrow اندازهٔ نیروهای مسلح & $\sim$۴۰۰,۰۰۰ (ارتش + حرس جمهوری + فدائیان) \\
سلاح‌های کشتارجمعی & ادعای آمریکا (نادرست) — برنامهٔ هسته‌ای متوقف \\
\altrow تحریم‌های بین‌المللی & جامع از ۱۹۹۰ (قطعنامهٔ ۶۶۱) \\
اپوزیسیون & تبعیدی + پراکنده (شورای حاکمیت عراق) \\
\altrow وضعیت جامعهٔ مدنی & سرکوب‌شدهٔ کامل \\
\bottomrule
\end{tabularx}
\end{table}

\subsection{وجوه مشابهت عراق و ایران}

\begin{table}[htbp]
\centering
\caption{مقایسهٔ ساختاری عراق (۲۰۰۳) و ایران (وضع فعلی)}
\label{tab:app-iraq-iran-comparison}
\begin{tabularx}{\textwidth}{>{\raggedleft\arraybackslash}p{3cm} >{\raggedleft\arraybackslash}X >{\raggedleft\arraybackslash}X >{\centering\arraybackslash}p{1.5cm}}
\toprule
\headerrow \textbf{بُعد} & \textbf{عراق ۲۰۰۳} & \textbf{ایران فعلی} & \textbf{مشابهت} \\
\midrule
رژیم & اقتدارگرای تمامیت‌خواه & اقتدارگرای ایدئولوژیک & \rating{3} \\
\altrow ایدئولوژی & بعثیسم/ناسیونالیسم عربی & اسلام سیاسی/ولایت فقیه & \rating{2} \\
تنوع فرقه‌ای-قومی & بسیار بالا (شیعه/سنی/کرد) & بالا (فارس/ترک/کرد/بلوچ/عرب) & \rating{4} \\
\altrow نفت & ذخایر عظیم (ردهٔ ۵ جهان) & ذخایر عظیم (ردهٔ ۴ جهان) & \rating{5} \\
نیروهای مسلح موازی & حرس جمهوری + فدائیان صدام & سپاه پاسداران + بسیج & \rating{5} \\
\altrow تحریم‌ها & ۱۲ سال تحریم شدید & ۴۰+ سال تحریم متنوع & \rating{4} \\
دیاسپورا & بزرگ و سیاسی & بسیار بزرگ و فعال & \rating{4} \\
\altrow اپوزیسیون سازمان‌یافته & ضعیف و تبعیدی & ضعیف و پراکنده & \rating{4} \\
بُعد هسته‌ای & ادعایی (نادرست) & واقعی (غنی‌سازی ۶۰٪+) & \rating{3} \\
\altrow رابطه با همسایگان & متخاصم (جنگ ایران-عراق، کویت) & متخاصم با برخی + نفوذ نیابتی & \rating{4} \\
\midrule
\headerrow \multicolumn{3}{l}{\textbf{میانگین مشابهت}} & \textbf{\rating{4}} \\
\bottomrule
\end{tabularx}
\end{table}

\begin{casestudy}
\textbf{تفاوت حیاتی:} با وجود مشابهت‌های ساختاری بالا (۴ از ۵)، یک تفاوت بنیادین وجود دارد: عراق از طریق \textbf{مداخلهٔ نظامی خارجی} وارد گذار شد و ایران (در مدل ۶ پیشنهادی) نباید. مشابهت بالای ایران و عراق دقیقاً دلیل اهمیت مطالعهٔ این ضد الگوست: هر اشتباهی که در عراق شد، \textbf{در ایران با شدت بیشتری} تکرار خواهد شد (جمعیت ۳ برابر، ابعاد هسته‌ای، ژئوپلیتیک حساس‌تر).
\end{casestudy}

\sectiondivider

%═══════════════════════════════════════════════════════════
\section{هفت اشتباه مهلک: تشریح و ضد درس‌آموخته‌ها}
\label{app:iraq:mistakes}
%═══════════════════════════════════════════════════════════

\subsection{اشتباه اول: مداخلهٔ نظامی خارجی}

\begin{warningbox}
\textbf{اشتباه مهلک ۱: حملهٔ نظامی به‌جای حمایت از گذار داخلی}

\begin{itemize}[nosep]
\item \textbf{واقعیت:} آمریکا و ائتلاف «داوطلبانه» (\lr{Coalition of the Willing}) در ۲۰ مارس ۲۰۰۳ با ۱۷۷,۰۰۰ نیرو به عراق حمله کردند.
\item \textbf{بهانه:} سلاح‌های کشتارجمعی (که وجود نداشت) + ارتباط با القاعده (که نبود).
\item \textbf{بدون مجوز شورای امنیت:} قطعنامهٔ ۱۴۴۱ مجوز جنگ نمی‌داد — مداخله غیرقانونی.
\item \textbf{نتیجه:} سقوط رژیم در ۳ هفته → اما ۱۷ سال اشغال و خشونت.
\item \textbf{هزینهٔ انسانی:} ۲۰۰,۰۰۰+ کشتهٔ عراقی + ۴,۵۰۰ سرباز آمریکایی + ۵ میلیون آواره.
\item \textbf{هزینهٔ مالی:} $>$\$۲ تریلیون (برای آمریکا تنها).
\end{itemize}

\textbf{ضد درس‌آموختهٔ ایرانی:} مداخلهٔ نظامی خارجی در ایران = سناریوی \lr{E} = \textbf{مطلقاً مردود}. ایران ۳ برابر جمعیت عراق، ۳ برابر مساحت، کوهستانی‌تر، مسلح‌تر (سپاه + هسته‌ای)، و با ناسیونالیسم قوی‌تر. هزینهٔ احتمالی: $>$\$۱۰ تریلیون + $>$۵۰۰,۰۰۰ کشته (\seeChapter{ch:scenarios}).
\end{warningbox}

\subsection{اشتباه دوم: انحلال ارتش (فرمان شمارهٔ ۲)}

\begin{table}[htbp]
\centering
\caption{فرمان شمارهٔ ۲ \lr{CPA}: انحلال نیروهای مسلح عراق}
\label{tab:app-iraq-order2}
\begin{tabularx}{\textwidth}{>{\raggedleft\arraybackslash}p{4cm} >{\raggedleft\arraybackslash}X}
\toprule
\headerrow \textbf{بُعد} & \textbf{جزئیات} \\
\midrule
صادرکننده & \person{ال.پل برمر}{\lr{L. Paul Bremer III}} — حاکم \lr{CPA} \\
\altrow تاریخ & ۲۳ مه ۲۰۰۳ (۶ هفته پس از سقوط بغداد) \\
محتوا & انحلال کامل: ارتش + حرس جمهوری + وزارت دفاع + وزارت اطلاعات + تمام ساختارهای امنیتی \\
\altrow تعداد متأثران & $\sim$۴۰۰,۰۰۰ نظامی + $\sim$۳۵۰,۰۰۰ کارمند وزارتخانه‌های منحله = $\sim$۷۵۰,۰۰۰ بیکار مسلح \\
هشدارهای نادیده‌گرفته‌شده & ژنرال \lr{Garner} (پیشینیان برمر)، \lr{CIA}، وزارت خارجه — همه مخالف بودند \\
\altrow نتیجهٔ فوری & ۷۵۰,۰۰۰ مرد مسلح و خشمگین بدون درآمد و هویت \\
نتیجهٔ بلندمدت & شورشیگری سنی → القاعدهٔ عراق → \textbf{داعش} \\
\bottomrule
\end{tabularx}
\end{table}

\begin{keypoint}
\textbf{ضد درس‌آموختهٔ ایرانی:} سپاه پاسداران ایران ($\sim$۱۹۰,۰۰۰) + بسیج ($\sim$۶۰۰,۰۰۰ فعال) + ارتش ($\sim$۴۲۰,۰۰۰) = بیش از ۱.۲ میلیون نفر مسلح. انحلال = ۱.۲ میلیون بیکار مسلح. بعد از عراق، هیچ تحلیلگر جدی انحلال را توصیه نمی‌کند. مدل درست: \textbf{ادغام} (آفریقای جنوبی) + \textbf{تفکیک اقتصادی} (اندونزی) + \textbf{بازنشستگی افتخاری} + \textbf{نظارت مدنی تدریجی} (\seeChapter{ch:guarantees}).
\end{keypoint}

\subsection{اشتباه سوم: اجتثاث بعث بدون عدالت}

\begin{table}[htbp]
\centering
\caption{مقایسهٔ اجتثاث بعث عراق با مدل‌های جایگزین}
\label{tab:app-iraq-debaath}
\begin{tabularx}{\textwidth}{>{\raggedleft\arraybackslash}p{2.5cm} >{\raggedleft\arraybackslash}X >{\centering\arraybackslash}p{2cm}}
\toprule
\headerrow \textbf{مدل} & \textbf{توضیح} & \textbf{نتیجه} \\
\midrule
عراق: اجتثاث افراطی & حذف همهٔ اعضای ردهٔ ۱-۴ بعث ($\sim$۸۵,۰۰۰ نفر) از کار — بدون تفکیک عامل/عضو عادی & \statusbad فاجعه \\
\altrow آلمان: نازی‌زدایی & ابتدا افراطی (متفقین) → سپس تعدیل (دولت آلمان) & \statuswarn متوسط \\
چک: لوستراسیون هدفمند & فقط مقامات بالا + بررسی فردی & \statusok خوب \\
\altrow آفریقای جنوبی: ادغام + TRC & نه پاکسازی بلکه اعتراف + عفو مشروط & \statusok خوب \\
\textbf{پیشنهاد ایران} & \textbf{لوستراسیون هوشمند:} بررسی فردی + تفکیک عامل/عضو عادی + بازنشستگی افتخاری & -- \\
\bottomrule
\end{tabularx}
\end{table}

\begin{lessonlearned}
\textbf{اجتثاث بعث:} فرمان شمارهٔ ۱ برمر (۱۶ مه ۲۰۰۳) حدود ۸۵,۰۰۰ بعثی را از مشاغل دولتی اخراج کرد — \textbf{بدون تفکیک} بین جنایتکاران واقعی و معلمان/پزشکانی که برای داشتن شغل مجبور به عضویت بودند. نتیجه: سرمایهٔ انسانی کشور نابود شد + هزاران نفر خشمگین و بیکار به شورشیان پیوستند. \emphred{ضد درس‌آموختهٔ ایرانی:} در ایران، میلیون‌ها نفر عضو بسیج، سپاه، یا وابسته به نهادهای حکومتی هستند. پاکسازی کور = فاجعه. مدل درست: \textbf{لوستراسیون هوشمند} (بررسی فردی: ردهٔ ۱ محاکمه + ردهٔ ۲-۳ بررسی + ردهٔ ۴+ عفو مشروط).
\end{lessonlearned}

\subsection{اشتباه چهارم: فقدان برنامهٔ پس از سقوط}

\begin{enumerate}[nosep]
\item \textbf{طرح وزارت خارجه (\lr{Future of Iraq Project}):} ۱۳ جلد، ۲,۵۰۰ صفحه — \textbf{نادیده گرفته شد} توسط پنتاگون
\item \textbf{طرح ارتش (\lr{Army War College}):} ۶۰۰ صفحه هشدار — نادیده گرفته شد
\item \textbf{فرض اشتباه رامسفلد:} «مردم عراق ما را با گل استقبال می‌کنند» — نکردند
\item \textbf{فرض اشتباه چلبی:} «تبعیدیان آماده‌اند کشور را اداره کنند» — نبودند
\item \textbf{نتیجه:} بی‌نظمی کامل (غارت)، فقدان خدمات عمومی، خلأ قدرت
\end{enumerate}

\begin{warningbox}
\textbf{ضد درس‌آموختهٔ ایرانی:} \textbf{فاز ۰ (پیش‌گذار)} مهم‌ترین فاز است. بدون برنامهٔ دقیق برای ۷۲ ساعت اول، ماه اول، و سال اول — حتی بهترین نیت‌ها به فاجعه منجر می‌شوند. عراق نشان داد که \textbf{سرنگونی} آسان‌ترین بخش کار است؛ \textbf{ساختن} بسیار دشوارتر. برنامهٔ ایران باید شامل: نقشهٔ خدمات ضروری، تأمین آب/برق/غذا، جلوگیری از غارت، حفظ آرشیوها، مدیریت مرزها، و زنجیرهٔ فرماندهی موقت باشد (\seeChapter{ch:timeline}).
\end{warningbox}

\subsection{اشتباه پنجم: تحمیل مدل خارجی (فدرالیسم فرقه‌ای)}

\begin{enumerate}[nosep]
\item \textbf{نظام فرقه‌ای (\lr{Muhasasa}):} قدرت بر اساس فرقه (شیعه/سنی/کرد) تقسیم شد — نه بر اساس شهروندی
\item \textbf{فدرالیسم قومی:} منطقهٔ کردستان عملاً مستقل شد
\item \textbf{قانون اساسی ۲۰۰۵:} تحت فشار زمانی و با مشارکت محدود سنی‌ها نوشته شد — \textbf{بمب ساعتی}
\item \textbf{نتیجه:} هر انتخاباتی به جنگ فرقه‌ای تبدیل شد + سنی‌ها احساس حذف کردند → داعش
\end{enumerate}

\begin{keypoint}
\textbf{ضد درس‌آموختهٔ ایرانی:} تقسیم قدرت بر اساس \textbf{قومیت/مذهب} (مدل لبنان/عراق) = فرمول جنگ داخلی. مدل درست برای ایران: \textbf{شهروندی فراگیر} + حقوق فرهنگی-زبانی اقوام + تمرکززدایی اداری (نه قومی) + سهمیهٔ تنوع (نه سهمیهٔ فرقه‌ای). قانون اساسی آفریقای جنوبی (۱۱ زبان اما شهروندی واحد) الگوی بهتری است (\seeChapter{ch:guarantees}).
\end{keypoint}

\subsection{اشتباه ششم: بی‌توجهی به امنیت و خلأ قدرت}

\begin{table}[htbp]
\centering
\caption{تبعات خلأ امنیتی در عراق}
\label{tab:app-iraq-security}
\begin{tabularx}{\textwidth}{>{\centering\arraybackslash}p{2cm} >{\raggedleft\arraybackslash}X >{\centering\arraybackslash}p{2.5cm}}
\toprule
\headerrow \textbf{دوره} & \textbf{بحران امنیتی} & \textbf{تلفات} \\
\midrule
۲۰۰۳ & غارت گسترده + ناامنی + حمله به زیرساخت‌ها & $\sim$۷,۰۰۰ \\
\altrow ۲۰۰۴ & شورش فلوجه + شورش مقتدی صدر & $\sim$۱۶,۰۰۰ \\
۲۰۰۵ & ترور + انفجار + تخریب مرقد سامرا & $\sim$۲۰,۰۰۰ \\
\altrow ۲۰۰۶-۲۰۰۷ & \textbf{جنگ داخلی فرقه‌ای} (اوج خشونت) & $\sim$۵۵,۰۰۰ \\
۲۰۰۸-۲۰۱۱ & افزایش نیرو (\lr{Surge}) + کاهش نسبی & $\sim$۲۵,۰۰۰ \\
\altrow ۲۰۱۴-۲۰۱۷ & \textbf{داعش}: اشغال موصل + نسل‌کشی ایزدی‌ها & $\sim$۵۰,۰۰۰+ \\
\bottomrule
\end{tabularx}
\end{table}

\subsection{اشتباه هفتم: بی‌اعتنایی به فرهنگ و بافت محلی}

\begin{enumerate}[nosep]
\item \textbf{ناآگاهی فرهنگی:} مقامات \lr{CPA} زبان عربی نمی‌دانستند و تاریخ عراق را نمی‌شناختند
\item \textbf{تبعیدیان قطع‌ارتباط:} رهبران تبعیدی (چلبی، علاوی) سال‌ها خارج بودند و مشروعیت نداشتند
\item \textbf{تحمیل الگوی غربی:} تلاش برای ایجاد «دموکراسی جفرسونی» در جامعهٔ قبیله‌ای-فرقه‌ای
\item \textbf{بی‌احترامی:} ابوغریب + بازرسی خانه‌ها + عملیات‌های شبانه = تحقیر مردم
\item \textbf{نتیجه:} «آزادی‌بخشان» به «اشغالگران» تبدیل شدند
\end{enumerate}

\begin{recommendation}
\textbf{ضد درس‌آموختهٔ کلان ایرانی:} \textbf{مالکیت ملی غیرقابل‌مذاکره} است. مدل ۶ بر اصل «ایرانیان عامل اصلی» استوار است. هرگونه نظارت بین‌المللی باید: ۱) به دعوت ایرانیان باشد؛ ۲) به زبان و فرهنگ ایرانی حساس باشد؛ ۳) کارشناسان ایرانی (داخل + دیاسپورا) نقش محوری داشته باشند؛ ۴) تصمیم‌های کلیدی با ایرانیان باشد نه خارجی‌ها (\seeChapter{ch:approaches}).
\end{recommendation}

\sectiondivider

%═══════════════════════════════════════════════════════════
\section{ماتریس اشتباهات عراق و ضد درس‌آموخته‌های ایرانی}
\label{app:iraq:matrix}
%═══════════════════════════════════════════════════════════

\begin{table}[htbp]
\centering
\caption{ماتریس هفت اشتباه مهلک عراق و ضد درس‌آموخته‌های ایرانی}
\label{tab:app-iraq-lessons}
\begin{tabularx}{\textwidth}{
  >{\centering\arraybackslash}p{0.7cm}
  >{\raggedleft\arraybackslash}p{2.5cm}
  >{\raggedleft\arraybackslash}X
  >{\raggedleft\arraybackslash}p{3.5cm}
}
\toprule
\headerrow \textbf{\#} & \textbf{اشتباه عراق} & \textbf{نتیجه} & \textbf{ضد درس‌آموختهٔ ایرانی} \\
\midrule
۱ & مداخلهٔ نظامی خارجی & جنگ + ۲۰۰K کشته + \$۲T & \cellgreen{سناریوی E مطلقاً مردود} \\
\altrow
۲ & انحلال ارتش & ۷۵۰K بیکار مسلح → داعش & \cellgreen{ادغام (نه انحلال) سپاه} \\
۳ & اجتثاث بدون تفکیک & سرمایهٔ انسانی نابود شد & \cellgreen{لوستراسیون هوشمند فردی} \\
\altrow
۴ & فقدان برنامهٔ پسا-سقوط & غارت + خلأ قدرت + هرج & \cellgreen{فاز ۰ دقیق + ۷۲ ساعت اول} \\
۵ & فدرالیسم فرقه‌ای & جنگ فرقه‌ای & \cellgreen{شهروندی فراگیر + تمرکززدایی اداری} \\
\altrow
۶ & خلأ امنیتی & ترور + میلیشیا + داعش & \cellgreen{حفظ نظم: فوریت ۷۲ ساعته} \\
۷ & تحمیل مدل خارجی & «اشغالگر» نه «آزادی‌بخش» & \cellgreen{مالکیت ملی ایرانی} \\
\bottomrule
\end{tabularx}
\end{table}

\sectiondivider

%═══════════════════════════════════════════════════════════
\section{نمودار: مسیر فاجعه‌بار عراق}
\label{app:iraq:diagram}
%═══════════════════════════════════════════════════════════

\begin{figure}[htbp]
\centering
\begin{tikzpicture}[
  node distance=0.6cm,
  phase/.style={
    draw, rounded corners=5pt, minimum width=3cm,
    minimum height=1.2cm, font=\small, align=center,
    thick
  },
  bad/.style={phase, fill=MainRed!15, draw=MainRed!60},
  worse/.style={phase, fill=MainRed!30, draw=MainRed!80},
  arrow/.style={->, very thick, >=stealth, MainRed},
  mistake/.style={
    draw=MainRed, fill=MainRed!5, rounded corners=2pt,
    font=\tiny, align=center, text width=2.5cm
  }
]

% مسیر فاجعه‌بار
\node[bad] (invasion) at (0,0) {حملهٔ نظامی\\مارس ۲۰۰۳};
\node[bad] (fall) at (4,0) {سقوط بغداد\\آوریل ۲۰۰۳};
\node[worse] (orders) at (8,0) {فرمان ۱ و ۲\\مه ۲۰۰۳};
\node[worse] (chaos) at (12,0) {هرج‌ومرج\\۲۰۰۳-۲۰۰۴};

\node[worse] (insurgency) at (0,-3) {شورشیگری\\۲۰۰۴-۲۰۰۵};
\node[worse] (civilwar) at (4,-3) {جنگ داخلی\\۲۰۰۶-۲۰۰۷};
\node[bad] (surge) at (8,-3) {افزایش نیرو\\۲۰۰۷-۲۰۰۸};
\node[worse] (isis) at (12,-3) {\textbf{داعش}\\۲۰۱۴};

\draw[arrow] (invasion) -- (fall);
\draw[arrow] (fall) -- (orders);
\draw[arrow] (orders) -- (chaos);
\draw[arrow] (chaos) -- (insurgency);
\draw[arrow] (insurgency) -- (civilwar);
\draw[arrow] (civilwar) -- (surge);
\draw[arrow] (surge) -- (isis);

% اشتباهات
\node[mistake] at (0,1.5) {اشتباه ۱:\\مداخلهٔ نظامی};
\node[mistake] at (8,1.5) {اشتباه ۲+۳:\\انحلال+اجتثاث};
\node[mistake] at (12,1.5) {اشتباه ۴:\\بدون برنامه};
\node[mistake] at (0,-4.5) {اشتباه ۶:\\خلأ امنیتی};
\node[mistake] at (4,-4.5) {اشتباه ۵:\\فرقه‌گرایی};
\node[mistake] at (12,-4.5) {اشتباه ۷:\\تحمیل خارجی};

% فلش‌های اشتباه
\draw[MainRed!40, dashed, ->] (0,1.1) -- (invasion.north);
\draw[MainRed!40, dashed, ->] (8,1.1) -- (orders.north);
\draw[MainRed!40, dashed, ->] (12,1.1) -- (chaos.north);
\draw[MainRed!40, dashed, ->] (0,-4.1) -- (insurgency.south);
\draw[MainRed!40, dashed, ->] (4,-4.1) -- (civilwar.south);
\draw[MainRed!40, dashed, ->] (12,-4.1) -- (isis.south);

% هزینه‌ها
\node[font=\small\bfseries, MainRed] at (6,-6) {مجموع: ۲۰۰,۰۰۰+ کشته | \$۲+ تریلیون | ۵ میلیون آواره | ظهور داعش};

\end{tikzpicture}
\caption{مسیر فاجعه‌بار عراق: از مداخله تا داعش}
\label{fig:app-iraq-path}
\end{figure}

\sectiondivider

%═══════════════════════════════════════════════════════════
\section{هزینه-فایدهٔ مقایسه‌ای: عراق vs مدل ۶}
\label{app:iraq:cost-benefit}
%═══════════════════════════════════════════════════════════

\begin{table}[htbp]
\centering
\caption{مقایسهٔ هزینه-فایده: مدل عراقی vs مدل ۶ برای ایران}
\label{tab:app-iraq-cost}
\begin{tabularx}{\textwidth}{>{\raggedleft\arraybackslash}p{3.5cm} >{\centering\arraybackslash}X >{\centering\arraybackslash}X}
\toprule
\headerrow \textbf{شاخص} & \textbf{مدل عراقی (سناریوی E)} & \textbf{مدل ۶ پیشنهادی (سناریوی B)} \\
\midrule
هزینهٔ مالی بین‌المللی & $>$\$۶۰B (+ \$۲T آمریکا) & \$۲.۵-۵B \\
\altrow هزینهٔ انسانی & ۲۰۰,۰۰۰+ کشته & حداقلی (هدف: $<$۱,۰۰۰) \\
مدت & ۷+ سال (هنوز ناتمام) & ۵-۱۰ سال \\
\altrow نتیجهٔ دموکراتیک & \lr{V-Dem}: ۰.۲۵ & هدف: $>$۰.۶۵ \\
مشروعیت داخلی & \statusbad بسیار پایین & \statusok بالا (مالکیت ملی) \\
\altrow مشروعیت بین‌المللی & \statusbad بدون قطعنامه & \statusok با قطعنامه \\
ریسک جنگ داخلی & \riskhigh محقق شد & \risklow مدیریت‌شده \\
\altrow ریسک تروریسم & \riskhigh داعش ظهور کرد & \risklow کنترل‌شده \\
اثر منطقه‌ای & \statusbad بی‌ثباتی گسترده & \statusok ثبات‌بخشی \\
\bottomrule
\end{tabularx}
\end{table}

\sectiondivider

%═══════════════════════════════════════════════════════════
\section{جمع‌بندی پیوست}
\label{app:iraq:conclusion}
%═══════════════════════════════════════════════════════════

\begin{chaptersummary}
جمع‌بندی پیوست ج — عراق: نمونهٔ منفی:

\begin{enumerate}[nosep]
\item عراق \textbf{مهم‌ترین ضد الگوی} این کتاب است: هر اشتباهی که در گذار ممکن بود، رخ داد.
\item \textbf{هفت اشتباه مهلک:} مداخلهٔ نظامی، انحلال ارتش، اجتثاث افراطی، فقدان برنامه، فدرالیسم فرقه‌ای، خلأ امنیتی، تحمیل مدل خارجی.
\item \textbf{هزینه:} ۲۰۰,۰۰۰+ کشته + \$۲+ تریلیون + ۵M آواره + داعش — در مقابل هیچ نتیجهٔ دموکراتیک پایدار.
\item مشابهت ساختاری ایران و عراق بالاست (\rating{4})، بنابراین هر اشتباه عراقی در ایران \textbf{با شدت بیشتر} تکرار می‌شود.
\item \textbf{سناریوی E (مداخلهٔ نظامی) مطلقاً مردود} است — عراق اثبات کرد.
\item مدل ۶ دقیقاً بر مبنای \textbf{اجتناب از اشتباهات عراق} طراحی شده: مالکیت ملی، ادغام (نه انحلال)، لوستراسیون هوشمند (نه اجتثاث)، برنامهٔ دقیق پیش‌گذار، شهروندی فراگیر (نه فرقه‌گرایی).
\item \textbf{فاز ۰ (پیش‌گذار)} مهم‌ترین درس عراق است: بدون آمادگی، حتی بهترین نیت‌ها به فاجعه می‌انجامد.
\item عراق نشان داد که \textbf{سرنگونی آسان‌ترین بخش} کار است — ساختن بسیار دشوارتر.
\end{enumerate}

\vspace{0.3cm}
\textit{مطالعهٔ تکمیلی:}
\begin{itemize}[nosep]
\item مقایسهٔ جامع ۹ نمونه: \seeChapter{app:comparison}
\item سناریوی مداخلهٔ نظامی (رد): \seeChapter{ch:scenarios}
\item اصلاح بخش امنیتی: \seeChapter{ch:guarantees}
\item آسیب‌شناسی و ریسک‌ها: \seeChapter{ch:risks}
\item میانمار (ضد الگوی دیگر): \seeChapter{app:myanmar}
\end{itemize}
\end{chaptersummary}

\chapterend

%══════════════════════════════════════════════════════════════
% پایان پیوست ج
%══════════════════════════════════════════════════════════════
%══════════════════════════════════════════════════════════════
% پیوست چ: مطالعه موردی میانمار — گذار ناتمام
% فایل: appendices/app-g-myanmar.tex
% حجم هدف: ۶-۸ صفحه
%══════════════════════════════════════════════════════════════

\chapter{مطالعهٔ موردی: میانمار — گذار ناتمام (۲۰۱۰-۲۰۲۱)}
\label{app:myanmar}

\begin{executivesummary}
میانمار (برمه) نمونهٔ دردناک \textbf{گذار ناتمام} است: دهه‌ای از گشایش تدریجی (۲۰۱۰-۲۰۲۰) با کودتای نظامی ۱ فوریه ۲۰۲۱ به پایان رسید و کشور به جنگ داخلی فرو رفت. ویژگی‌های کلیدی: ۱) \textbf{گشایش کنترل‌شده از بالا} توسط همان ژنرال‌ها (بدون فشار واقعی مردمی)، ۲) \textbf{قانون اساسی ۲۰۰۸ نظامی‌نوشت} (۲۵٪ کرسی‌ها + وتوی عملی + فرماندهی مستقل)، ۳) \textbf{فقدان اصلاح واقعی بخش امنیتی} (\lr{SSR}) — \termfn{تاتمادو}{\lr{Tatmadaw}} دست‌نخورده ماند، ۴) \textbf{نسل‌کشی روهینگیا} (۲۰۱۷) حتی در دورهٔ «دموکراسی»، و ۵) \textbf{کودتای ۲۰۲۱} و بازگشت کامل به حکومت نظامی. میانمار \textbf{قوی‌ترین شاهد} بر این اصل است: بدون اصلاح واقعی بخش امنیتی، هر گشایشی موقتی و شکننده است.
\end{executivesummary}

%═══════════════════════════════════════════════════════════
\section{زمینه و بافت تاریخی}
\label{app:myanmar:context}
%═══════════════════════════════════════════════════════════

\subsection{تاتمادو: ارتشی که دولت را بلعید}

\org{تاتمادو}{\lr{Tatmadaw}} (نیروهای مسلح میانمار) از ۱۹۶۲ تا ۲۰۱۱ به‌طور مستقیم حکومت کرد — طولانی‌ترین حکومت نظامی مستقیم در آسیا:

\begin{table}[htbp]
\centering
\caption{مشخصات میانمار در آستانهٔ گشایش (۲۰۱۰)}
\label{tab:app-myanmar-profile}
\begin{tabularx}{\textwidth}{>{\raggedleft\arraybackslash}p{4.5cm} >{\raggedleft\arraybackslash}X}
\toprule
\headerrow \textbf{شاخص} & \textbf{مقدار} \\
\midrule
جمعیت & ۵۲ میلیون نفر \\
\altrow مساحت & ۶۷۶,۰۰۰ \lr{km²} \\
تنوع قومی & بسیار بالا (۱۳۵ گروه قومی رسمی) \\
\altrow \lr{GDP per capita} & $\sim$\$۱,۲۰۰ \\
طول عمر حکومت نظامی & ۴۹ سال (۱۹۶۲-۲۰۱۱) \\
\altrow اندازهٔ تاتمادو & $\sim$۴۵۰,۰۰۰ نفر \\
منافع اقتصادی ارتش & \lr{MEHL + MEC}: امپراتوری عظیم (یشم، بانک، مخابرات) \\
\altrow جنگ‌های داخلی فعال & ۱۵+ گروه مسلح قومی \\
نقش بودایی‌گری & ملی‌گرایی بودایی (ضد مسلمان) \\
\altrow تحریم‌های بین‌المللی & هدفمند (\lr{US + EU}) \\
\bottomrule
\end{tabularx}
\end{table}

\subsection{مقایسهٔ ساختاری تاتمادو و سپاه پاسداران}

\begin{table}[htbp]
\centering
\caption{مقایسهٔ تفصیلی تاتمادو و سپاه پاسداران}
\label{tab:app-myanmar-irgc}
\begin{tabularx}{\textwidth}{>{\raggedleft\arraybackslash}p{3.5cm} >{\raggedleft\arraybackslash}X >{\raggedleft\arraybackslash}X}
\toprule
\headerrow \textbf{بُعد} & \textbf{تاتمادو (میانمار)} & \textbf{سپاه پاسداران (ایران)} \\
\midrule
تعداد نیرو & $\sim$۴۵۰,۰۰۰ & $\sim$۱۹۰,۰۰۰ (+ ۶۰۰K بسیج) \\
\altrow منافع اقتصادی & \lr{MEHL + MEC}: یشم (\$۳۱B/سال)، بانک، هتل & خاتم‌الانبیاء، بنیادها، قاچاق \\
کنترل سیاسی & ۲۵٪ کرسی‌های پارلمان (بدون انتخاب) & شورای نگهبان + خبرگان + نفوذ \\
\altrow فرماندهی مستقل & کاملاً مستقل از دولت غیرنظامی & تحت فرمان رهبری (نه رئیس‌جمهور) \\
وتوی قانون اساسی & ۲۵٪+ ۱ = وتوی اصلاحیه & شورای نگهبان = وتوی قانون \\
\altrow ایدئولوژی & ناسیونالیسم بامار + بودایی‌گری & اسلام سیاسی + ضد استکبار \\
جنایات & نسل‌کشی روهینگیا (۲۰۱۷) & سرکوب اعتراضات + اعدام‌ها \\
\altrow حاضر به اصلاح؟ & خیر (کودتای ۲۰۲۱) & نامشخص \\
\bottomrule
\end{tabularx}
\end{table}

\begin{casestudy}
\textbf{تاتمادو آینهٔ سپاه:} شباهت ساختاری تاتمادو و سپاه بسیار بالاست — هر دو ارتش‌هایی هستند با منافع اقتصادی عظیم، فرماندهی مستقل، وتوی قانون اساسی، و ایدئولوژی ملی‌گرایانه. تفاوت‌ها: ۱) تاتمادو ۴۹ سال مستقیماً حکومت کرد (سپاه غیرمستقیم)؛ ۲) میانمار ۱۵+ جنگ داخلی قومی دارد (ایران ندارد)؛ ۳) تاتمادو نسل‌کشی آشکار انجام داد. نکتهٔ حیاتی: تاتمادو \textbf{گشایش} داد اما \textbf{اصلاح نشد} — و وقتی نتیجهٔ انتخابات را نپسندید، کودتا کرد. \emphred{هشدار ایرانی:} اگر سپاه بدون اصلاح واقعی باقی بماند، «کودتای ۲۰۲۱ میانمار» در ایران تکرار خواهد شد.
\end{casestudy}

\sectiondivider

%═══════════════════════════════════════════════════════════
\section{گشایش کنترل‌شده: «دموکراسی منضبط» (۲۰۱۰-۲۰۲۰)}
\label{app:myanmar:opening}
%═══════════════════════════════════════════════════════════

\subsection{قانون اساسی ۲۰۰۸: قفل‌های نهادی}

ژنرال‌ها قبل از گشایش، قانون اساسی ۲۰۰۸ را طراحی کردند که شامل \textbf{قفل‌های نهادی} بی‌سابقه بود:

\begin{table}[htbp]
\centering
\caption{قفل‌های نهادی قانون اساسی ۲۰۰۸ میانمار و مقایسه با پینوشه}
\label{tab:app-myanmar-locks}
\begin{tabularx}{\textwidth}{>{\raggedleft\arraybackslash}p{3.5cm} >{\raggedleft\arraybackslash}X >{\centering\arraybackslash}p{2cm}}
\toprule
\headerrow \textbf{قفل} & \textbf{توضیح} & \textbf{قفل پینوشه؟} \\
\midrule
۲۵٪ کرسی‌های پارلمان & نظامیان منصوب (بدون انتخاب) در هر دو مجلس & مشابه (سناتورهای منصوب) \\
\altrow وتوی اصلاحیه & اصلاح قانون اساسی نیاز به ۷۵٪+ دارد → ۲۵٪ نظامی = وتو & مشابه \\
فرماندهی مستقل & فرمانده کل ارتش خودش را منصوب می‌کند (نه رئیس‌جمهور) & مشابه \\
\altrow سه وزارت کلیدی & دفاع + داخله + مرزها فقط نظامی & \textbf{فراتر از پینوشه} \\
مصونیت & هیچ اقدام نظامی گذشته قابل تعقیب نیست & مشابه (قانون عفو ۱۹۷۸) \\
\altrow حکومت نظامی & فرمانده کل حق اعلام حکومت نظامی و قبض قدرت دارد & -- \\
ممنوعیت سوچی & شخصی که همسرش خارجی باشد نمی‌تواند رئیس‌جمهور شود (مادهٔ ۵۹-f: علیه سوچی) & -- \\
\bottomrule
\end{tabularx}
\end{table}

\begin{warningbox}
\textbf{قانون اساسی ۲۰۰۸ میانمار = مدل «چگونه قدرت را حفظ کنیم».} ژنرال‌ها دموکراسی ظاهری ایجاد کردند اما هر ابزاری برای بازگشت به قدرت نگه داشتند. این \textbf{هشدار مستقیم} برای ایران است: در هر سناریوی مذاکره‌ای، باید مراقب قانون اساسی «نظامی‌نوشت» بود. مجلس مؤسسان ایران باید واقعاً مستقل و فراگیر باشد — نه صوری (\seeChapter{ch:guarantees}).
\end{warningbox}

\subsection{گاه‌شمار گشایش و بازگشت}

\begin{table}[htbp]
\centering
\caption{گاه‌شمار گذار ناتمام میانمار (۲۰۱۰-۲۰۲۱)}
\label{tab:app-myanmar-timeline}
\begin{tabularx}{\textwidth}{>{\centering\arraybackslash}p{2.5cm} >{\raggedleft\arraybackslash}X >{\centering\arraybackslash}p{2cm}}
\toprule
\headerrow \textbf{تاریخ} & \textbf{رویداد} & \textbf{ارزیابی} \\
\midrule
نوامبر ۲۰۱۰ & انتخابات فرمایشی (حزب نظامی \lr{USDP} برنده) & \statusbad \\
\altrow مارس ۲۰۱۱ & ریاست‌جمهوری \person{تئین‌سئین}{\lr{Thein Sein}} (ژنرال بازنشسته) & \statuswarn \\
۲۰۱۱-۲۰۱۲ & آزادی زندانیان سیاسی + آزادی نسبی مطبوعات & \statusok \\
\altrow آوریل ۲۰۱۲ & \lr{NLD} در انتخابات میان‌دوره‌ای شرکت + پیروز شد & \statusok \\
۲۰۱۳-۲۰۱۴ & رشد اقتصادی + سرمایه‌گذاری خارجی + لغو تحریم‌ها & \statusok \\
\altrow نوامبر ۲۰۱۵ & \textbf{انتخابات آزاد: \lr{NLD} پیروز (۸۰٪+ کرسی‌های آزاد)} & \statusok \\
مارس ۲۰۱۶ & دولت سوچی (اما رئیس‌جمهور نشد: مادهٔ ۵۹-f) & \statuswarn \\
\altrow اوت ۲۰۱۷ & \textbf{نسل‌کشی روهینگیا:} ۷۰۰,۰۰۰ آواره → بنگلادش & \statusbad \\
۲۰۱۸-۲۰۱۹ & بحران بین‌المللی + دادگاه \lr{ICJ} & \statusbad \\
\altrow نوامبر ۲۰۲۰ & انتخابات: \lr{NLD} دوباره پیروز (بیشتر از ۲۰۱۵) & \statusok \\
\textbf{۱ فوریه ۲۰۲۱} & \textbf{کودتای تاتمادو:} بازداشت سوچی + حکومت نظامی & \statusbad \\
\altrow ۲۰۲۱-۲۰۲۴ & جنگ داخلی گسترده + \lr{NUG} (دولت وحدت ملی) + مقاومت مسلحانه & \statusbad \\
\bottomrule
\end{tabularx}
\end{table}

\sectiondivider

%═══════════════════════════════════════════════════════════
\section{چرا گذار شکست خورد: پنج علت ساختاری}
\label{app:myanmar:failure}
%═══════════════════════════════════════════════════════════

\subsection{تحلیل علّی}

\begin{table}[htbp]
\centering
\caption{پنج علت ساختاری شکست گذار میانمار}
\label{tab:app-myanmar-causes}
\begin{tabularx}{\textwidth}{
  >{\centering\arraybackslash}p{0.7cm}
  >{\raggedleft\arraybackslash}p{3cm}
  >{\raggedleft\arraybackslash}X
  >{\centering\arraybackslash}p{1.5cm}
}
\toprule
\headerrow \textbf{\#} & \textbf{علت} & \textbf{توضیح} & \textbf{ارتباط ایران} \\
\midrule
۱ & \textbf{فقدان \lr{SSR} واقعی} & تاتمادو دست‌نخورده ماند: فرماندهی مستقل + بودجهٔ مستقل + اقتصاد مستقل + ۲۵٪ پارلمان & \rating{5} \\
\altrow
۲ & \textbf{قانون اساسی نظامی‌نوشت} & قفل‌های نهادی غیرقابل‌رفع (۷۵٪ لازم + ۲۵٪ وتوی نظامی) & \rating{5} \\
۳ & \textbf{سوچی سازش بیش‌ازحد کرد} & \lr{NLD} تاتمادو را به چالش نکشید + نسل‌کشی روهینگیا را انکار کرد & \rating{4} \\
\altrow
۴ & \textbf{فقدان فشار بین‌المللی مؤثر} & چین و روسیه حامی تاتمادو + \lr{ASEAN} ضعیف + تحریم‌ها ناکافی & \rating{3} \\
۵ & \textbf{جنگ‌های قومی حل‌نشده} & ۱۵+ گروه مسلح قومی + بدون توافق صلح جامع & \rating{3} \\
\bottomrule
\end{tabularx}
\end{table}

\begin{keypoint}
\textbf{علت اول تعیین‌کننده است:} بدون اصلاح واقعی بخش امنیتی، \textbf{هر گشایشی صوری و برگشت‌پذیر} است. تاتمادو هرگز تحت نظارت مدنی واقعی قرار نگرفت: فرمانده‌اش را خودش انتخاب می‌کرد، بودجه‌اش شفاف نبود، امپراتوری اقتصادی‌اش (\lr{MEHL}: بزرگ‌ترین شرکت هلدینگ میانمار) بی‌حساب‌وکتاب بود. وقتی نتیجهٔ انتخابات ۲۰۲۰ را نپسندید، \textbf{مادهٔ ۴۱۷ قانون اساسی خودنوشته} را فعال و کودتا کرد.
\end{keypoint}

\subsection{نقد سوچی: سازش بیش‌ازحد}

\person{آنگ سان سوچی}{\lr{Aung San Suu Kyi}} — نوبل صلح ۱۹۹۱ — تصمیمات بحث‌برانگیزی گرفت:

\begin{enumerate}[nosep]
\item \textbf{پذیرش قانون اساسی ۲۰۰۸:} با وجود قفل‌های نهادی، \lr{NLD} وارد بازی شد — بدون تلاش جدی برای اصلاح
\item \textbf{عدم مقابله با ارتش:} هیچ تلاشی برای نظارت مدنی بر تاتمادو نکرد
\item \textbf{دفاع از نسل‌کشی:} در \lr{ICJ} شخصاً از ارتش دفاع کرد (۲۰۱۹) — مشروعیت بین‌المللی‌اش را از دست داد
\item \textbf{سکوت دربارهٔ جنایات:} هرگز از حقوق روهینگیا دفاع نکرد
\item \textbf{نتیجه:} تاتمادو هم از سازش بهره برد و هم وقتی خواست، سوچی را بازداشت کرد
\end{enumerate}

\begin{lessonlearned}
\textbf{دام «سازش بدون اصلاح»:} سوچی گمان کرد با \textbf{سازش تاکتیکی} با ارتش، به‌تدریج فضا باز می‌شود. اما ارتش سازش را نشانهٔ \textbf{ضعف} تلقی کرد نه حسن‌نیت. \emphred{درس ایرانی:} مذاکره با سپاه ضروری است (مدل آفریقای جنوبی)، اما مذاکره باید \textbf{از موضع قدرت} باشد و نتیجهٔ آن \textbf{اصلاح ساختاری واقعی} (ادغام + تفکیک اقتصادی + نظارت مدنی). سازش بدون اصلاح = تأخیر در فاجعه، نه جلوگیری از آن (\seeChapter{ch:guarantees}).
\end{lessonlearned}

\sectiondivider

%═══════════════════════════════════════════════════════════
\section{نسل‌کشی روهینگیا: دموکراسی بدون حقوق بشر}
\label{app:myanmar:rohingya}
%═══════════════════════════════════════════════════════════

\begin{warningbox}
\textbf{نسل‌کشی روهینگیا (اوت ۲۰۱۷)} حتی در دورهٔ «دموکراتیک» میانمار رخ داد:

\begin{itemize}[nosep]
\item \textbf{عملیات نظامی:} تاتمادو عملیات «پاکسازی» در ایالت راخین انجام داد
\item \textbf{تلفات:} حداقل ۱۰,۰۰۰ کشته + ۷۰۰,۰۰۰ آواره به بنگلادش
\item \textbf{روش‌ها:} سوزاندن روستاها، تجاوز جنسی سیستماتیک، کشتار غیرنظامیان
\item \textbf{واکنش سوچی:} سکوت + دفاع از ارتش در \lr{ICJ}
\item \textbf{تعریف حقوقی:} مأموریت تحقیق سازمان ملل: «نشانه‌های نسل‌کشی» (\lr{Genocide}) + جنایت علیه بشریت
\item \textbf{واکنش بین‌المللی:} ضعیف (چین و روسیه وتو در شورای امنیت)
\end{itemize}

\textbf{درس حیاتی:} دموکراسی بدون حقوق بشر، دموکراسی نیست. انتخابات آزاد + پارلمان + رئیس‌جمهور غیرنظامی — هیچ‌کدام مانع نسل‌کشی نشد. \emphred{هشدار ایرانی:} قانون اساسی جدید ایران باید \textbf{منشور حقوق} غیرقابل‌نقض داشته باشد + دادگاه قانون اساسی مستقل + پذیرش صلاحیت \lr{ICC} + حقوق اقلیت‌ها (بهائیان، سنی‌ها، کردها، بلوچ‌ها) بدون استثنا.
\end{warningbox}

\sectiondivider

%═══════════════════════════════════════════════════════════
\section{کودتای ۲۰۲۱: بازگشت کامل}
\label{app:myanmar:coup}
%═══════════════════════════════════════════════════════════

\subsection{چگونه کودتا شد}

\begin{table}[htbp]
\centering
\caption{عوامل کودتای ۱ فوریه ۲۰۲۱ میانمار}
\label{tab:app-myanmar-coup}
\begin{tabularx}{\textwidth}{>{\raggedleft\arraybackslash}p{3.5cm} >{\raggedleft\arraybackslash}X}
\toprule
\headerrow \textbf{عامل} & \textbf{توضیح} \\
\midrule
محرک آنی & شکست حزب نظامی \lr{USDP} در انتخابات نوامبر ۲۰۲۰ + ادعای تقلب \\
\altrow عامل ساختاری ۱ & ارتش هرگز تحت نظارت مدنی قرار نگرفت \\
عامل ساختاری ۲ & قانون اساسی ۲۰۰۸ ابزار قانونی کودتا فراهم کرد (مادهٔ ۴۱۷) \\
\altrow عامل شخصی & ژنرال \person{مین‌آنگ‌هلینگ}{\lr{Min Aung Hlaing}} در آستانهٔ بازنشستگی: یا قدرت یا محاکمه \\
عامل بین‌المللی & چین + روسیه حمایت ضمنی + \lr{ASEAN} ناتوان \\
\altrow عامل اقتصادی & ارتش حاضر به از دست دادن \lr{MEHL} و امپراتوری اقتصادی نبود \\
\bottomrule
\end{tabularx}
\end{table}

\subsection{پس از کودتا: مقاومت بی‌سابقه}

\begin{enumerate}[nosep]
\item \textbf{جنبش نافرمانی مدنی (\lr{CDM}):} صدها هزار کارمند اعتصاب کردند — پزشکان، معلمان، بانکداران
\item \textbf{نسل Z:} جوانان خلاقانه مقاومت کردند (سه‌انگشتی + فضای مجازی)
\item \textbf{\lr{NUG} (دولت وحدت ملی):} دولت موازی تشکیل شد (ترکیب \lr{NLD} + اقوام)
\item \textbf{\lr{PDF} (نیروهای دفاع مردمی):} مقاومت مسلحانه گسترش یافت
\item \textbf{سرکوب خونین:} ۱,۵۰۰+ کشته + ۱۲,۰۰۰+ بازداشتی (تا ۲۰۲۳)
\item \textbf{ائتلاف اقوام:} برای اولین بار، گروه‌های مسلح قومی با \lr{NUG} متحد شدند
\item \textbf{وضعیت ۲۰۲۴:} جنگ داخلی ادامه دارد + ارتش در برخی مناطق عقب‌نشینی کرده
\end{enumerate}

\begin{keypoint}
\textbf{تفاوت ۲۰۲۱ با ۱۹۸۸:} در کودتای قبلی (۱۹۸۸)، مردم تسلیم شدند. در ۲۰۲۱، نسل جدید تسلیم \textbf{نشد}. دلایل: ۱) یک دهه تجربهٔ آزادی نسبی؛ ۲) فضای مجازی و ارتباطات؛ ۳) الهام از هنگ‌کنگ و بهار عربی. \emphgreen{درس مثبت برای ایران:} حتی اگر گذار شکست بخورد، \textbf{خاطرهٔ آزادی} بازگشت‌ناپذیر است و مقاومت را قوی‌تر می‌کند.
\end{keypoint}

\sectiondivider

%═══════════════════════════════════════════════════════════
\section{ماتریس درس‌آموخته‌ها برای ایران}
\label{app:myanmar:lessons}
%═══════════════════════════════════════════════════════════

\begin{table}[htbp]
\centering
\caption{ماتریس انتقال درس‌آموخته‌های میانمار به ایران (عمدتاً هشدار)}
\label{tab:app-myanmar-lessons}
\begin{tabularx}{\textwidth}{
  >{\raggedleft\arraybackslash}p{2.2cm}
  >{\raggedleft\arraybackslash}p{3.5cm}
  >{\raggedleft\arraybackslash}X
  >{\centering\arraybackslash}p{1.3cm}
  >{\centering\arraybackslash}p{1cm}
}
\toprule
\headerrow \textbf{بُعد} & \textbf{درس میانمار} & \textbf{کاربرد ایرانی} & \textbf{نوع} & \textbf{اهمیت} \\
\midrule
\lr{SSR} & بدون اصلاح = بازگشت & اصلاح ساختاری سپاه اجتناب‌ناپذیر & \cellred{هشدار} & \rating{5} \\
\altrow
قانون اساسی & نظامی‌نوشت = دام & مجلس مؤسسان واقعاً مستقل & \cellred{هشدار} & \rating{5} \\
قفل‌های نهادی & ۲۵٪ = وتو = بازگشت & هیچ وتوی نهادی برای نظامیان & \cellred{هشدار} & \rating{5} \\
\altrow
اقتصاد ارتش & \lr{MEHL} دست‌نخورده ماند & تفکیک خاتم‌الانبیاء ضروری & \cellred{هشدار} & \rating{5} \\
سازش بدون اصلاح & سوچی سازش کرد → کودتا & مذاکره از موضع قدرت + اصلاح واقعی & \cellred{هشدار} & \rating{5} \\
\altrow
حقوق اقلیت‌ها & نسل‌کشی حتی در دورهٔ «دموکراسی» & منشور حقوق + \lr{ICC} + حقوق بهائیان/اقوام & \cellred{هشدار} & \rating{5} \\
مقاومت پس از شکست & نسل Z تسلیم نشد & خاطرهٔ آزادی بازگشت‌ناپذیر & \cellgreen{الگو} & \rating{3} \\
\altrow
ائتلاف اقوام & \lr{NUG} + گروه‌های قومی متحد شدند & ائتلاف فراگیر: فارس + کرد + ترک + بلوچ & \cellgreen{الگو} & \rating{4} \\
\midrule
\headerrow \multicolumn{3}{l}{\textbf{میانگین اهمیت (عمدتاً هشدار)}} & & \textbf{\rating{5}} \\
\bottomrule
\end{tabularx}
\end{table}

\sectiondivider

%═══════════════════════════════════════════════════════════
\section{نمودار: مقایسهٔ دو مسیر — گشایش واقعی vs صوری}
\label{app:myanmar:diagram}
%═══════════════════════════════════════════════════════════

\begin{figure}[htbp]
\centering
\begin{tikzpicture}[
  node distance=0.8cm,
  phase/.style={
    draw, rounded corners=5pt, minimum width=3cm,
    minimum height=1cm, font=\small, align=center,
    thick
  },
  good/.style={phase, fill=MainGreen!15, draw=MainGreen!60},
  bad/.style={phase, fill=MainRed!15, draw=MainRed!60},
  arrow/.style={->, thick, >=stealth},
  fork/.style={->, very thick, >=stealth}
]

% نقطهٔ شروع مشترک
\node[phase, fill=MainOrange!15, draw=MainOrange!60] (start) at (6,5) {گشایش سیاسی\\(تحت کنترل ارتش)};

% مسیر میانمار (شکست — راست)
\node[bad] (no-ssr) at (11,3) {بدون \lr{SSR}\\ارتش دست‌نخورده};
\node[bad] (facade) at (11,1) {دموکراسی\\صوری};
\node[bad] (coup) at (11,-1) {کودتا\\۲۰۲۱};
\node[bad] (war) at (11,-3) {جنگ داخلی\\۲۰۲۱+};

\draw[arrow, MainRed] (start.east) -| (no-ssr.north);
\draw[arrow, MainRed] (no-ssr) -- (facade);
\draw[arrow, MainRed] (facade) -- (coup);
\draw[arrow, MainRed] (coup) -- (war);

% عنوان
\node[font=\small\bfseries, MainRed] at (11,4.2) {مسیر میانمار: شکست};

% مسیر ایران پیشنهادی (موفقیت — چپ)
\node[good] (ssr) at (1,3) {\lr{SSR} واقعی\\ادغام + تفکیک};
\node[good] (const) at (1,1) {قانون اساسی\\مستقل + فراگیر};
\node[good] (consolidation) at (1,-1) {تحکیم\\نهادی};
\node[good] (democracy) at (1,-3) {دموکراسی\\پایدار};

\draw[arrow, MainGreen] (start.west) -| (ssr.north);
\draw[arrow, MainGreen] (ssr) -- (const);
\draw[arrow, MainGreen] (const) -- (consolidation);
\draw[arrow, MainGreen] (consolidation) -- (democracy);

% عنوان
\node[font=\small\bfseries, MainGreen] at (1,4.2) {مسیر پیشنهادی ایران: مدل ۶};

% نقطهٔ تفاوت
\node[draw=MainPurple, fill=MainPurple!10, rounded corners=3pt,
  font=\tiny\bfseries, align=center, text width=3cm] at (6,2.5) {نقطهٔ تفاوت:\\آیا ارتش واقعاً\\اصلاح می‌شود؟};

\draw[fork, MainGreen, dashed] (6,3.2) -- (ssr);
\draw[fork, MainRed, dashed] (6,3.2) -- (no-ssr);

\end{tikzpicture}
\caption{دو مسیر گشایش: اصلاح واقعی (مدل ۶) vs صوری (مدل میانمار)}
\label{fig:app-myanmar-paths}
\end{figure}

\sectiondivider

%═══════════════════════════════════════════════════════════
\section{جمع‌بندی پیوست}
\label{app:myanmar:conclusion}
%═══════════════════════════════════════════════════════════

\begin{chaptersummary}
جمع‌بندی پیوست چ — میانمار: گذار ناتمام:

\begin{enumerate}[nosep]
\item میانمار \textbf{قوی‌ترین هشدار} برای ایران است: گشایش بدون \lr{SSR} واقعی = بازگشت قطعی.
\item \textbf{تاتمادو آینهٔ سپاه} است: هر دو ارتش‌هایی با منافع اقتصادی عظیم، فرماندهی مستقل، و وتوی قانون اساسی.
\item \textbf{قانون اساسی نظامی‌نوشت} خطرناک‌ترین «قفل» است — مجلس مؤسسان ایران باید واقعاً مستقل باشد.
\item \textbf{سازش بدون اصلاح = تأخیر در فاجعه}: سوچی یک دهه سازش کرد اما تاتمادو هرگز اصلاح نشد و در نهایت کودتا کرد.
\item \textbf{نسل‌کشی روهینگیا} نشان داد: انتخابات + پارلمان ≠ حقوق بشر. منشور حقوق غیرقابل‌نقض ضروری است.
\item نکتهٔ مثبت: \textbf{مقاومت نسل Z} نشان داد که خاطرهٔ آزادی بازگشت‌ناپذیر است.
\item برای ایران: اصلاح ساختاری سپاه (ادغام + تفکیک اقتصادی + نظارت مدنی) \textbf{شرط لازم} هر گذار موفقی است — بدون آن، حتی بهترین قانون اساسی و انتخابات آزاد‌ترین، ناپایدار خواهد بود.
\end{enumerate}

\vspace{0.3cm}
\textit{مطالعهٔ تکمیلی:}
\begin{itemize}[nosep]
\item مقایسهٔ جامع ۹ نمونه: \seeChapter{app:comparison}
\item اصلاح بخش امنیتی: \seeChapter{ch:guarantees}
\item عراق (ضد الگوی دیگر): \seeChapter{app:iraq}
\item آفریقای جنوبی (ادغام نیروها): \seeChapter{app:south-africa}
\item اندونزی (تفکیک اقتصادی ارتش): \seeChapter{app:comparison}
\item تیمور شرقی: \seeChapter{app:timor}
\end{itemize}
\end{chaptersummary}

\chapterend

%══════════════════════════════════════════════════════════════
% پایان پیوست چ
%══════════════════════════════════════════════════════════════
%══════════════════════════════════════════════════════════════
% پیوست ح: مطالعه موردی تیمور شرقی
% فایل: appendices/app-h-timor.tex
% حجم هدف: ۶-۸ صفحه
%══════════════════════════════════════════════════════════════

\chapter{مطالعهٔ موردی: تیمور شرقی (۱۹۹۹-۲۰۰۲)}
\label{app:timor}

\begin{executivesummary}
تیمور شرقی (تیمور-لسته) یکی از معدود نمونه‌های موفق \textbf{مدیریت مستقیم بین‌المللی} (\lr{International Transitional Administration}) است که از \textbf{اشغال ۲۴ سالهٔ اندونزی} (۱۹۷۵-۱۹۹۹) به دولت مستقل دموکراتیک انتقال یافت. ویژگی‌های منحصربه‌فرد: ۱) \textbf{رفراندوم استقلال} تحت نظارت سازمان ملل (\lr{UNAMET}، ۱۹۹۹)، ۲) \textbf{خشونت وحشیانهٔ میلیشیای اندونزیایی} پس از رفراندوم، ۳) \textbf{مأموریت مدیریت انتقالی} (\lr{UNTAET}: مدل ۵ از فصل ۳)، ۴) \textbf{ساختن دولت از صفر} (بدون ارتش، بدون بوروکراسی، بدون زیرساخت)، و ۵) \textbf{کمیسیون حقیقت} (\lr{CAVR}). این تجربه هم الگو (مرحله‌بندی، رفراندوم، نهادسازی) و هم هشدار (وابستگی به کمک خارجی، بحران ۲۰۰۶) برای ایران است. \textbf{تفاوت بنیادین:} تیمور کشوری کوچک (۱ میلیون) و فقیر بود که دولت‌سازی از صفر نیاز داشت — ایران کشوری بزرگ با نهادهای موجود (هرچند آسیب‌دیده) است که نیازمند \textbf{اصلاح} (نه ساختن از صفر) است.
\end{executivesummary}

%═══════════════════════════════════════════════════════════
\section{زمینه و بافت تاریخی}
\label{app:timor:context}
%═══════════════════════════════════════════════════════════

\subsection{از استعمار پرتغال تا اشغال اندونزی}

\begin{table}[htbp]
\centering
\caption{مشخصات تیمور شرقی در آستانهٔ رفراندوم (۱۹۹۹)}
\label{tab:app-timor-profile}
\begin{tabularx}{\textwidth}{>{\raggedleft\arraybackslash}p{4.5cm} >{\raggedleft\arraybackslash}X}
\toprule
\headerrow \textbf{شاخص} & \textbf{مقدار} \\
\midrule
جمعیت & $\sim$۸۰۰,۰۰۰ نفر \\
\altrow مساحت & ۱۵,۰۰۰ \lr{km²} (نصف استان فارس) \\
استعمار پرتغال & ۱۵۱۵-۱۹۷۵ (۴۶۰ سال) \\
\altrow اشغال اندونزی & ۱۹۷۵-۱۹۹۹ (۲۴ سال) \\
تلفات اشغال & ۱۰۰,۰۰۰-۱۸۰,۰۰۰ (از ۶۰۰K جمعیت ۱۹۷۵ = تا ۳۰٪) \\
\altrow \lr{GDP per capita} & $\sim$\$۴۰۰ (یکی از فقیرترین آسیا) \\
زبان‌ها & تتوم + پرتغالی (رسمی) + بهاسا اندونزیایی + ۳۰+ زبان محلی \\
\altrow جنبش مقاومت & \lr{FRETILIN} (سیاسی) + \lr{FALINTIL} (مسلح) \\
رهبر مقاومت & \person{شاناناگوسمائو}{\lr{Xanana Gusmão}} (زندانی ۱۹۹۲-۱۹۹۹) \\
\altrow تنوع مذهبی & ۹۷٪ کاتولیک (یکدست) \\
\bottomrule
\end{tabularx}
\end{table}

\subsection{گاه‌شمار گذار}

\begin{table}[htbp]
\centering
\caption{گاه‌شمار کلیدی تیمور شرقی (۱۹۹۶-۲۰۰۲)}
\label{tab:app-timor-timeline}
\begin{tabularx}{\textwidth}{>{\centering\arraybackslash}p{2.5cm} >{\raggedleft\arraybackslash}X >{\centering\arraybackslash}p{2cm}}
\toprule
\headerrow \textbf{تاریخ} & \textbf{رویداد} & \textbf{اهمیت} \\
\midrule
۱۹۹۶ & \person{راموس‌هورتا}{\lr{Ramos-Horta}} و اسقف \person{بلو}{\lr{Belo}} نوبل صلح گرفتند & مشروعیت بین‌المللی \\
\altrow مه ۱۹۹۸ & سقوط سوهارتو در اندونزی & تغییر محیط \\
مه ۱۹۹۹ & \textbf{توافق سه‌جانبه} (اندونزی + پرتغال + سازمان ملل): رفراندوم & نقطهٔ عطف \\
\altrow ژوئن ۱۹۹۹ & استقرار \lr{UNAMET} (مأموریت رفراندوم سازمان ملل) & آغاز نظارت \\
\textbf{۳۰ اوت ۱۹۹۹} & \textbf{رفراندوم:} ۷۸.۵٪ «استقلال» / ۲۱.۵٪ «خودمختاری» & تعیین‌کننده \\
\altrow سپتامبر ۱۹۹۹ & \textbf{خشونت میلیشیا:} ویرانی ۷۰٪ زیرساخت + ۱,۴۰۰ کشته + ۳۰۰K آواره & فاجعه \\
۲۰ سپتامبر ۱۹۹۹ & \textbf{\lr{INTERFET}:} نیروی چندملیتی به رهبری استرالیا (قطعنامهٔ ۱۲۶۴) & مداخلهٔ نظامی \\
\altrow ۲۵ اکتبر ۱۹۹۹ & \textbf{\lr{UNTAET}} تأسیس شد (قطعنامهٔ ۱۲۷۲): مدیریت مستقیم سازمان ملل & مدل ۵ \\
اوت ۲۰۰۱ & انتخابات مجلس مؤسسان (مشارکت ۹۳٪) & نهادسازی \\
\altrow مارس ۲۰۰۲ & تصویب قانون اساسی & نقطهٔ عطف \\
آوریل ۲۰۰۲ & انتخابات ریاست‌جمهوری: گوسمائو ۸۳٪ & مشروعیت \\
\altrow \textbf{۲۰ مه ۲۰۰۲} & \textbf{استقلال رسمی:} تولد جمهوری دموکراتیک تیمور-لسته & موفقیت \\
\bottomrule
\end{tabularx}
\end{table}

\sectiondivider

%═══════════════════════════════════════════════════════════
\section{رفراندوم ۱۹۹۹: پیروزی و خشونت}
\label{app:timor:referendum}
%═══════════════════════════════════════════════════════════

\subsection{سازماندهی رفراندوم}

\begin{table}[htbp]
\centering
\caption{آمار رفراندوم تیمور شرقی ۱۹۹۹ و مقایسه با شیلی}
\label{tab:app-timor-referendum}
\begin{tabularx}{\textwidth}{>{\raggedleft\arraybackslash}p{4cm} >{\centering\arraybackslash}p{3cm} >{\centering\arraybackslash}p{3cm}}
\toprule
\headerrow \textbf{شاخص} & \textbf{تیمور ۱۹۹۹} & \textbf{شیلی ۱۹۸۸} \\
\midrule
سؤال & استقلال یا خودمختاری & ادامهٔ پینوشه یا نه \\
\altrow واجدین شرایط & ۴۵۱,۸۹۲ & ۷,۴۳۵,۹۱۳ \\
مشارکت & ۹۸.۶٪ & ۹۷.۵٪ \\
\altrow نتیجه & ۷۸.۵٪ استقلال & ۵۵.۹۹٪ «نه» \\
سازمان‌دهنده & سازمان ملل (\lr{UNAMET}) & خود رژیم (با ناظران) \\
\altrow ناظران بین‌المللی & ۵۰ رأی‌گیری + ۲,۰۰۰+ ناظر & $\sim$۱,۰۰۰ \\
امنیت & ناکافی (پلیس اندونزی!) & نسبتاً مطلوب \\
\altrow خشونت پس از نتیجه & \riskhigh فاجعه‌بار & \risklow حداقل \\
\bottomrule
\end{tabularx}
\end{table}

\subsection{فاجعهٔ سپتامبر ۱۹۹۹}

\begin{warningbox}
\textbf{فاجعه‌ای که قابل پیشگیری بود:} پس از اعلام نتیجهٔ رفراندوم، میلیشیاهای اندونزیایی (\lr{Aitarak, Besi Merah Putih}) با حمایت \lr{TNI} (ارتش اندونزی) به \textbf{عملیات زمین‌سوخته} دست زدند:

\begin{itemize}[nosep]
\item \textbf{۱,۴۰۰+ کشته} (برخی تخمین‌ها: ۲,۰۰۰+)
\item \textbf{۳۰۰,۰۰۰+ آواره} (از ۸۰۰K جمعیت = ۳۸٪!)
\item \textbf{۷۰٪ زیرساخت‌ها ویران شد:} ساختمان‌های دولتی، مدارس، بیمارستان‌ها، خانه‌ها
\item \textbf{کارمندان \lr{UNAMET} هم هدف حمله قرار گرفتند}
\item \textbf{سازمان ملل ابتدا عقب‌نشینی کرد} — انتقاد شدید: «رفراندوم بدون تضمین امنیت»
\end{itemize}

\textbf{درس حیاتی:} رفراندوم بدون \textbf{تضمین امنیتی کافی} می‌تواند فاجعه‌ایجاد کند. توافق سه‌جانبه امنیت را به \textbf{پلیس اندونزی} سپرده بود — همان نیرویی که با میلیشیاها همکاری می‌کرد. \emphred{هشدار ایرانی:} هر رفراندوم یا انتخاباتی در ایران باید با \textbf{تضمین امنیتی مستقل} همراه باشد. امنیت انتخابات نمی‌تواند به نیروهایی سپرده شود که منافع‌شان با نتیجه در تضاد است (\seeChapter{ch:timeline}).
\end{warningbox}

\subsection{مداخلهٔ \lr{INTERFET}: مدل واکنش سریع}

پس از فاجعه، شورای امنیت قطعنامهٔ ۱۲۶۴ را تصویب کرد و نیروی \org{اینترفت}{\lr{INTERFET (International Force East Timor)}} به رهبری \textbf{استرالیا} مستقر شد:

\begin{table}[htbp]
\centering
\caption{مشخصات نیروی \lr{INTERFET}}
\label{tab:app-timor-interfet}
\begin{tabularx}{\textwidth}{>{\raggedleft\arraybackslash}p{4cm} >{\raggedleft\arraybackslash}X}
\toprule
\headerrow \textbf{شاخص} & \textbf{جزئیات} \\
\midrule
قطعنامه & شورای امنیت ۱۲۶۴ (۱۵ سپتامبر ۱۹۹۹) \\
\altrow فرمانده & ژنرال \person{پیتر کاسگرو}{\lr{Peter Cosgrove}} (استرالیا) \\
تعداد نیرو & $\sim$۱۱,۰۰۰ (از ۲۲ کشور) \\
\altrow کشورهای اصلی & استرالیا (۵,۵۰۰)، تایلند، فیلیپین، کرهٔ جنوبی، نیوزیلند \\
مأموریت & بازگرداندن امنیت + حمایت از \lr{UNAMET} + حمایت انسانی \\
\altrow مدت & سپتامبر ۱۹۹۹ — فوریهٔ ۲۰۰۰ (سپس به نیروی \lr{UNTAET} تبدیل شد) \\
نتیجه & عقب‌نشینی اندونزی + بازگشت آوارگان + امنیت نسبی & \\
\bottomrule
\end{tabularx}
\end{table}

\begin{lessonlearned}
\textbf{\lr{INTERFET} نشان داد:} ۱) واکنش سریع بین‌المللی وقتی \textbf{ارادهٔ سیاسی} وجود داشته باشد، ممکن و مؤثر است؛ ۲) \textbf{رهبری منطقه‌ای} (استرالیا) مؤثرتر از رهبری دور (آمریکا/اروپا) است؛ ۳) مشارکت چندملیتی مشروعیت بیشتری دارد. \emphgreen{کاربرد ایرانی:} در سناریوی \lr{A} (فروپاشی ناگهانی)، ممکن است نیروی حافظ صلح لازم شود. ترکیب: سازمان ملل + کشورهای منطقه (ترکیه؟ هند؟) + ناتو — اما تنها با دعوت ایرانیان (\seeChapter{ch:scenarios}).
\end{lessonlearned}

\sectiondivider

%═══════════════════════════════════════════════════════════
\section{\lr{UNTAET}: مدل مدیریت مستقیم بین‌المللی}
\label{app:timor:untaet}
%═══════════════════════════════════════════════════════════

\subsection{ساختار و مأموریت}

\org{ادارهٔ انتقالی سازمان ملل در تیمور شرقی}{\lr{UNTAET (UN Transitional Administration in East Timor)}} (قطعنامهٔ ۱۲۷۲، اکتبر ۱۹۹۹) یکی از جامع‌ترین مأموریت‌های سازمان ملل بود — حاکمیت کامل سرزمینی:

\begin{table}[htbp]
\centering
\caption{ساختار و آمار \lr{UNTAET} (۱۹۹۹-۲۰۰۲)}
\label{tab:app-timor-untaet}
\begin{tabularx}{\textwidth}{>{\raggedleft\arraybackslash}p{5cm} >{\raggedleft\arraybackslash}X}
\toprule
\headerrow \textbf{شاخص} & \textbf{جزئیات} \\
\midrule
نمایندهٔ ویژه / حاکم موقت & \person{سرجیو ویئیرا دملو}{\lr{Sérgio Vieira de Mello}} \\
\altrow اختیارات & \textbf{تمام قوای سه‌گانه:} مقننه + مجریه + قضاییه \\
نیروی نظامی & $\sim$۸,۰۰۰ \\
\altrow پلیس مدنی & $\sim$۱,۶۴۰ \\
کارمندان غیرنظامی & $\sim$۱,۰۰۰ بین‌المللی + ۲,۰۰۰ محلی \\
\altrow بودجهٔ سالانه & $\sim$\$۵۶۳ میلیون (سال اول) \\
مأموریت & امنیت + حاکمیت + نهادسازی + آموزش + انتخابات + استقلال \\
\altrow مدت & اکتبر ۱۹۹۹ — مه ۲۰۰۲ (۳۱ ماه) \\
مأموریت جانشین & \lr{UNMISET} (۲۰۰۲-۲۰۰۵) → \lr{UNMIT} (۲۰۰۶-۲۰۱۲) \\
\bottomrule
\end{tabularx}
\end{table}

\subsection{دستاوردها و نقدها}

\begin{table}[htbp]
\centering
\caption{دستاوردها و نقدهای \lr{UNTAET}}
\label{tab:app-timor-untaet-review}
\begin{tabularx}{\textwidth}{>{\centering\arraybackslash}p{1cm} >{\raggedleft\arraybackslash}X >{\centering\arraybackslash}p{2cm}}
\toprule
\headerrow & \textbf{دستاوردها} & \textbf{امتیاز} \\
\midrule
\cmark & استقلال در ۳۱ ماه (سریع و مؤثر) & \starrating{5} \\
\altrow \cmark & انتخابات مجلس مؤسسان (مشارکت ۹۳٪) & \starrating{5} \\
\cmark & قانون اساسی دموکراتیک & \starrating{4} \\
\altrow \cmark & ایجاد نیروی دفاعی (\lr{F-FDTL}) و پلیس (\lr{PNTL}) از صفر & \starrating{3} \\
\cmark & بازگشت ۳۰۰,۰۰۰ آواره & \starrating{4} \\
\altrow \cmark & \lr{CAVR} (کمیسیون حقیقت) & \starrating{4} \\
\midrule
\headerrow & \textbf{نقدها} & \textbf{شدت} \\
\midrule
\xmark & \textbf{کم‌مشارکت‌دادن تیموری‌ها:} «مدیریت استعماری جدید»؟ & \riskhigh \\
\altrow \xmark & حقوق‌های بالای کارمندان بین‌المللی vs فقر محلی & \riskmedium \\
\xmark & سرعت خروج: ظرفیت‌سازی ناکافی → بحران ۲۰۰۶ & \riskhigh \\
\altrow \xmark & عدالت بین‌المللی ناکام: اندونزی تعقیب نشد & \riskhigh \\
\xmark & وابستگی به کمک خارجی پس از استقلال & \riskmedium \\
\bottomrule
\end{tabularx}
\end{table}

\begin{keypoint}
\textbf{نقد اصلی: «مدیریت مستقیم» ≠ «مالکیت ملی».} ویئیرا دملو (بعداً در بمب‌گذاری بغداد ۲۰۰۳ کشته شد) تلاش کرد تیموری‌ها را مشارکت دهد، اما ساختار \lr{UNTAET} ذاتاً \textbf{بالا به پایین} بود. بسیاری از تصمیمات توسط خارجی‌ها گرفته شد. رهبران تیموری (\person{گوسمائو}{\lr{Gusmão}} و \person{آلکاتیری}{\lr{Alkatiri}}) گاه احساس حاشیه‌نشینی می‌کردند. این نقد تأیید می‌کند که \textbf{مدل ۵ (مدیریت مستقیم) برای ایران مناسب نیست} (\seeChapter{ch:approaches}). مدل ۶ (ترکیبی-تطبیقی) با مالکیت ملی ایرانی طراحی شده.
\end{keypoint}

\sectiondivider

%═══════════════════════════════════════════════════════════
\section{کمیسیون حقیقت (\lr{CAVR})}
\label{app:timor:cavr}
%═══════════════════════════════════════════════════════════

\org{کمیسیون پذیرش، حقیقت و آشتی}{\lr{CAVR (Comissão de Acolhimento, Verdade e Reconciliação)}} از ۲۰۰۲ تا ۲۰۰۵ فعالیت کرد:

\begin{table}[htbp]
\centering
\caption{ساختار و آمار \lr{CAVR} تیمور شرقی}
\label{tab:app-timor-cavr}
\begin{tabularx}{\textwidth}{>{\raggedleft\arraybackslash}p{5cm} >{\raggedleft\arraybackslash}X}
\toprule
\headerrow \textbf{شاخص} & \textbf{جزئیات} \\
\midrule
مدت & ژانویه ۲۰۰۲ — اکتبر ۲۰۰۵ \\
\altrow تعداد کمیسیونرها & ۷ (همه تیموری) \\
بازهٔ زمانی & ۱۹۷۴-۱۹۹۹ (۲۵ سال) \\
\altrow تعداد شهادت‌ها & ۷,۸۲۴ \\
\termfn{فرآیند آشتی جامعه‌محور}{\lr{CRP}} & ۱,۳۷۱ فرآیند (در سطح روستا) \\
\altrow تخمین تلفات کل & ۱۰۲,۸۰۰ (± ۱۲,۰۰۰) — تحلیل آماری \\
گزارش نهایی & «\lr{Chega!}» (بس است!) — ۵ جلد، ۲,۵۰۰+ صفحه \\
\altrow نوآوری ویژه & \textbf{فرآیند آشتی محلی (\lr{CRP}):} عاملان در روستا اعتراف و خدمت اجتماعی انجام دادند \\
\bottomrule
\end{tabularx}
\end{table}

\subsection{نوآوری \lr{CRP}: عدالت جامعه‌محور}

\begin{enumerate}[nosep]
\item عاملان «جرایم سبک» (آتش‌زدن خانه، سرقت، تخریب — نه قتل) داوطلبانه به \lr{CAVR} مراجعه می‌کردند
\item در جلسهٔ عمومی روستا، \textbf{اعتراف علنی} + \textbf{عذرخواهی از قربانیان} + \textbf{توافق بر مجازات جامعه‌ای} (مثلاً بازسازی خانه، خدمت اجتماعی)
\item پس از انجام مجازات، \textbf{پذیرش مجدد در جامعه} (\lr{Acolhimento} = پذیرش)
\item ۱,۳۷۱ فرآیند انجام شد — \textbf{۸۵٪+ موفقیت‌آمیز}
\end{enumerate}

\begin{lessonlearned}
\textbf{مدل \lr{CRP} برای ایران:} فرآیند آشتی جامعه‌محور می‌تواند برای «بسیجیان عادی» یا «عوامل محلی سرکوب» استفاده شود — نه برای فرماندهان ارشد (که باید محاکمه شوند). \emphgreen{کاربرد:} در شهرها و روستاهایی که بسیجیان محلی در سرکوب مشارکت کرده‌اند، جلسات آشتی با حضور قربانیان + اعتراف + خدمت اجتماعی — به‌عنوان مکمل (نه جایگزین) محاکمهٔ فرماندهان (\seeChapter{ch:guarantees}).
\end{lessonlearned}

\sectiondivider

%═══════════════════════════════════════════════════════════
\section{بحران ۲۰۰۶: شکنندگی دولت نوپا}
\label{app:timor:crisis}
%═══════════════════════════════════════════════════════════

\subsection{ریشه‌ها و درس‌ها}

تنها ۴ سال پس از استقلال، تیمور شرقی در ۲۰۰۶ دچار بحران شدید شد:

\begin{table}[htbp]
\centering
\caption{بحران ۲۰۰۶ تیمور شرقی: علل و نتایج}
\label{tab:app-timor-crisis}
\begin{tabularx}{\textwidth}{>{\raggedleft\arraybackslash}p{3.5cm} >{\raggedleft\arraybackslash}X}
\toprule
\headerrow \textbf{بُعد} & \textbf{جزئیات} \\
\midrule
محرک آنی & اعتراض ۶۰۰ سرباز «غربی» (\lr{loromonu}) به تبعیض از سوی فرماندهان «شرقی» (\lr{lorosa'e}) \\
\altrow تنش نهادی & رقابت ارتش (\lr{F-FDTL}) و پلیس (\lr{PNTL}) — هر دو از صفر ساخته شده \\
ظرفیت‌سازی ناکافی & \lr{UNTAET} خیلی سریع خارج شد + نهادها شکننده بودند \\
\altrow نتیجه & ۱۵۰+ کشته + ۱۰۰,۰۰۰ آواره (باز هم!) + سقوط دولت آلکاتیری \\
واکنش بین‌المللی & نیروی بین‌المللی (استرالیا + نیوزیلند) بازگشت + \lr{UNMIT} تشکیل شد \\
\altrow درس & خروج زودهنگام نظارت بین‌المللی = خطر بازگشت بحران \\
\bottomrule
\end{tabularx}
\end{table}

\begin{warningbox}
\textbf{درس بحران ۲۰۰۶ برای ایران:} \textbf{خروج زودهنگام نظارت بین‌المللی} خطرناک است. \lr{UNTAET} در ۳۱ ماه خارج شد — ظرفیت‌سازی کافی نبود. مدل ۶ پیشنهادی برای ایران \textbf{۵ فاز در ۱۰ سال} پیش‌بینی کرده و خروج مشروط به رسیدن به شاخص‌های کمّی است (نه تقویم ثابت). سه سناریوی خروج در فصل ۹ تعریف شده: موفق، تعدیل، تمدید. بحران ۲۰۰۶ تأیید می‌کند که سناریوی «تمدید» باید همیشه روی میز باشد (\seeChapter{ch:timeline}).
\end{warningbox}

\sectiondivider

%═══════════════════════════════════════════════════════════
\section{ماتریس درس‌آموخته‌ها برای ایران}
\label{app:timor:lessons}
%═══════════════════════════════════════════════════════════

\begin{table}[htbp]
\centering
\caption{ماتریس انتقال درس‌آموخته‌های تیمور شرقی به ایران}
\label{tab:app-timor-lessons}
\begin{tabularx}{\textwidth}{
  >{\raggedleft\arraybackslash}p{2.5cm}
  >{\raggedleft\arraybackslash}p{3.5cm}
  >{\raggedleft\arraybackslash}X
  >{\centering\arraybackslash}p{1.5cm}
}
\toprule
\headerrow \textbf{بُعد} & \textbf{درس تیمور} & \textbf{کاربرد ایرانی} & \textbf{انتقال‌پذیری} \\
\midrule
رفراندوم & سازماندهی موفق اما بدون تضمین امنیت & رفراندوم + تضمین امنیتی مستقل & \rating{4} \\
\altrow
واکنش سریع & \lr{INTERFET}: ۱۱K نیرو در ۲ هفته & آمادگی نیروی واکنش (سناریوی \lr{A}) & \rating{3} \\
مرحله‌بندی & \lr{UNAMET} → \lr{INTERFET} → \lr{UNTAET} → \lr{UNMISET} & فاز ۰→۱→۲→۳→۴ (مدل ۶) & \rating{5} \\
\altrow
نهادسازی & ایجاد ارتش + پلیس + قضا از صفر & ایران: اصلاح نهادهای موجود (نه از صفر) & \rating{2} \\
\lr{CAVR} & حقیقت + آشتی جامعه‌محور (\lr{CRP}) & مکمل \lr{TRC} برای عوامل محلی & \rating{4} \\
\altrow
مالکیت ملی & \lr{UNTAET} انتقاد شد: «استعمار جدید» & مدل ۶: مالکیت ملی ایرانی (نه مدل ۵) & \rating{5} \\
خروج & خروج زودهنگام → بحران ۲۰۰۶ & خروج مشروط به شاخص‌ها (نه تقویم) & \rating{5} \\
\altrow
ظرفیت‌سازی & ناکافی بود & برنامهٔ ظرفیت‌سازی ۱۰ ساله & \rating{4} \\
\midrule
\headerrow \multicolumn{3}{l}{\textbf{میانگین انتقال‌پذیری}} & \textbf{\rating{4}} \\
\bottomrule
\end{tabularx}
\end{table}

\sectiondivider

%═══════════════════════════════════════════════════════════
\section{نمودار: مرحله‌بندی مأموریت‌های سازمان ملل در تیمور}
\label{app:timor:diagram}
%═══════════════════════════════════════════════════════════

\begin{figure}[htbp]
\centering
\begin{tikzpicture}[
  node distance=0.5cm,
  mission/.style={
    draw, rounded corners=5pt, minimum width=3cm,
    minimum height=1.3cm, font=\small, align=center,
    thick
  },
  arrow/.style={->, very thick, >=stealth, gray},
  timeline/.style={very thick, gray}
]

% خط زمانی
\draw[timeline, ->] (0,0) -- (15,0);
\foreach \x/\y in {0/1999, 3/2000, 5/2002, 8/2005, 11/2008, 14/2012} {
  \draw[gray] (\x,-0.2) -- (\x,0.2);
  \node[font=\tiny, below] at (\x,-0.3) {\y};
}

% مأموریت‌ها
\node[mission, fill=MainRed!15, draw=MainRed!60] (unamet) at (0.5,2) {\lr{UNAMET}\\رفراندوم};
\node[mission, fill=MainOrange!15, draw=MainOrange!60] (interfet) at (2.5,2) {\lr{INTERFET}\\امنیت};
\node[mission, fill=MainPurple!15, draw=MainPurple!60] (untaet) at (5,2) {\lr{UNTAET}\\مدیریت مستقیم};
\node[mission, fill=MainBlue!15, draw=MainBlue!60] (unmiset) at (8,2) {\lr{UNMISET}\\حمایت};
\node[mission, fill=MainGreen!15, draw=MainGreen!60] (unmit) at (11,2) {\lr{UNMIT}\\بحران ۲۰۰۶};
\node[mission, fill=MainGreen!25, draw=MainGreen!80] (exit) at (14,2) {خروج\\۲۰۱۲};

\draw[arrow] (unamet) -- (interfet);
\draw[arrow] (interfet) -- (untaet);
\draw[arrow] (untaet) -- (unmiset);
\draw[arrow] (unmiset) -- (unmit);
\draw[arrow] (unmit) -- (exit);

% شدت حضور
\draw[MainPurple!60, very thick] plot[smooth] coordinates {(0.5,-1) (2.5,-2.5) (5,-3) (8,-2) (11,-2.5) (14,-0.5)};
\node[font=\tiny, MainPurple] at (7,-3.5) {شدت حضور بین‌المللی};

% نقاط عطف
\node[font=\tiny, MainRed, anchor=south] at (0.5,3) {رفراندوم\\۳۰ اوت};
\node[font=\tiny, MainOrange, anchor=south] at (2.5,3) {خشونت\\→ مداخله};
\node[font=\tiny, MainPurple, anchor=south] at (5,3) {حاکمیت\\سازمان ملل};
\node[font=\tiny, MainBlue, anchor=south] at (8,3) {کاهش\\تدریجی};
\node[font=\tiny, MainRed, anchor=south] at (11,3) {بحران!\\بازگشت};
\node[font=\tiny, MainGreen, anchor=south] at (14,3) {خروج\\نهایی};

% معادل ایرانی
\node[font=\tiny\itshape, MainPurple, anchor=north] at (0.5,-0.5) {فاز ۰};
\node[font=\tiny\itshape, MainPurple, anchor=north] at (2.5,-0.5) {فاز ۱};
\node[font=\tiny\itshape, MainPurple, anchor=north] at (5,-0.5) {فاز ۲};
\node[font=\tiny\itshape, MainPurple, anchor=north] at (8,-0.5) {فاز ۳};
\node[font=\tiny\itshape, MainPurple, anchor=north] at (11,-0.5) {(احتمالی)};
\node[font=\tiny\itshape, MainPurple, anchor=north] at (14,-0.5) {فاز ۴};

\end{tikzpicture}
\caption{مرحله‌بندی مأموریت‌های سازمان ملل در تیمور شرقی و فازهای معادل ایرانی}
\label{fig:app-timor-phases}
\end{figure}

\sectiondivider

%═══════════════════════════════════════════════════════════
\section{وضعیت فعلی (۲۰۲۳) و ارزیابی بلندمدت}
\label{app:timor:current}
%═══════════════════════════════════════════════════════════

\begin{table}[htbp]
\centering
\caption{شاخص‌های تیمور شرقی (۲۰۲۳): ۲۱ سال پس از استقلال}
\label{tab:app-timor-current}
\begin{tabularx}{\textwidth}{>{\raggedleft\arraybackslash}X >{\centering\arraybackslash}p{3.5cm}}
\toprule
\headerrow \textbf{شاخص} & \textbf{مقدار} \\
\midrule
\lr{V-Dem Liberal Democracy Index} & ۰.۵۸ (دموکراسی انتخاباتی) \\
\altrow \lr{Freedom House} & نیمه‌آزاد (۷۲/۱۰۰) \\
\lr{GDP per capita (PPP)} & $\sim$\$۵,۰۰۰ (وابسته به نفت تیمور گپ) \\
\altrow جمعیت & ۱.۳ میلیون \\
انتقال مسالمت‌آمیز قدرت & \cmark (چندین بار: ۲۰۰۷, ۲۰۱۲, ۲۰۱۷, ۲۰۲۳) \\
\altrow عضویت بین‌المللی & سازمان ملل (۲۰۰۲) + \lr{ASEAN} (درخواست) \\
چالش اصلی & فقر + وابستگی به نفت + جوانی جمعیت (۶۵٪ زیر ۳۰) \\
\bottomrule
\end{tabularx}
\end{table}

\begin{recommendation}
\textbf{ارزیابی کلی:} تیمور شرقی با وجود تمام محدودیت‌ها (کوچکی، فقر، ویرانی ۷۰٪) به \textbf{دموکراسی نسبتاً پایدار} دست یافته — چندین انتقال مسالمت‌آمیز قدرت، قانون اساسی دموکراتیک، جامعهٔ مدنی فعال. اما \textbf{چالش‌های توسعه‌ای} جدی باقی مانده. درس: دموکراسی بدون توسعهٔ اقتصادی شکننده است — تأیید مجدد درس تونس (\seeChapter{app:tunisia}).
\end{recommendation}

\sectiondivider

%═══════════════════════════════════════════════════════════
\section{جمع‌بندی پیوست}
\label{app:timor:conclusion}
%═══════════════════════════════════════════════════════════

\begin{chaptersummary}
جمع‌بندی پیوست ح — تیمور شرقی:

\begin{enumerate}[nosep]
\item تیمور شرقی نمونهٔ موفق \textbf{مدیریت مستقیم بین‌المللی} (مدل ۵) است — اما با نقد جدی: کم‌مشارکت‌دادن محلی‌ها.
\item \textbf{مدل ۵ برای ایران مناسب نیست} — ایران کشوری بزرگ با نهادهای موجود است. مدل ۶ (ترکیبی با مالکیت ملی) مناسب‌تر است.
\item \textbf{رفراندوم ۱۹۹۹} نشان داد رفراندوم ابزار قدرتمندی است، \textbf{مشروط به تضمین امنیتی مستقل}.
\item \textbf{مرحله‌بندی مأموریت‌ها} (\lr{UNAMET→INTERFET→UNTAET→UNMISET}) الگوی خوبی برای فازبندی مدل ۶ است.
\item \textbf{فرآیند \lr{CRP}} (آشتی جامعه‌محور) نوآوری ارزشمندی است — مکمل \lr{TRC} برای عوامل محلی سرکوب.
\item \textbf{خروج زودهنگام} (۳۱ ماه) منجر به بحران ۲۰۰۶ شد — خروج باید مشروط به شاخص‌ها باشد.
\item تیمور با وجود محدودیت‌ها به دموکراسی نسبتاً پایدار رسیده — اثبات اینکه حتی در بدترین شرایط، دموکراسی ممکن است.
\end{enumerate}

\vspace{0.3cm}
\textit{مطالعهٔ تکمیلی:}
\begin{itemize}[nosep]
\item مقایسهٔ جامع ۹ نمونه: \seeChapter{app:comparison}
\item رویکردها و مدل‌های نظارت: \seeChapter{ch:approaches}
\item زمان‌بندی و فازبندی: \seeChapter{ch:timeline}
\item عراق (مدل ۵ ناموفق): \seeChapter{app:iraq}
\item آفریقای جنوبی (مدل ۳ موفق): \seeChapter{app:south-africa}
\end{itemize}
\end{chaptersummary}

\chapterend

%══════════════════════════════════════════════════════════════
% پایان پیوست ح
%══════════════════════════════════════════════════════════════
%══════════════════════════════════════════════════════════════
% پیوست خ: واژه‌نامه تخصصی دوزبانه فارسی-انگلیسی
% فایل: appendices/app-i-glossary.tex
% حجم هدف: ۵-۷ صفحه
%══════════════════════════════════════════════════════════════

\chapter{واژه‌نامهٔ تخصصی دوزبانه فارسی-انگلیسی}
\label{app:glossary}

\begin{executivesummary}
این واژه‌نامه شامل بیش از \textbf{۱۵۰ اصطلاح تخصصی} به‌کاررفته در کتاب حاضر است، دسته‌بندی‌شده در ۱۰ حوزهٔ موضوعی. هر واژه با معادل انگلیسی، تعریف مختصر فارسی، و ارجاع به فصل مرتبط ارائه شده است. مخاطبان فارسی‌زبانی که با ادبیات تخصصی گذار دموکراتیک و نظارت بین‌المللی آشنایی کمتری دارند، می‌توانند از این واژه‌نامه به‌عنوان راهنمای مرجع استفاده کنند.
\end{executivesummary}

%═══════════════════════════════════════════════════════════
\section{الف) گذار دموکراتیک و تحکیم}
\label{app:glossary:transition}
%═══════════════════════════════════════════════════════════

\begin{longtable}{>{\raggedleft\arraybackslash}p{3.5cm} >{\raggedleft\arraybackslash}p{3.5cm} >{\raggedleft\arraybackslash}p{6.5cm} >{\centering\arraybackslash}p{1.5cm}}
\caption{واژگان حوزهٔ گذار دموکراتیک}
\label{tab:glossary-transition} \\

\toprule
\headerrow \textbf{فارسی} & \textbf{انگلیسی} & \textbf{تعریف} & \textbf{فصل} \\
\midrule
\endfirsthead

\multicolumn{4}{c}{\small\textit{ادامهٔ جدول واژگان گذار دموکراتیک}} \\
\toprule
\headerrow \textbf{فارسی} & \textbf{انگلیسی} & \textbf{تعریف} & \textbf{فصل} \\
\midrule
\endhead

\bottomrule
\endlastfoot

گذار دموکراتیک & \lr{Democratic Transition} & فرآیند انتقال از رژیم اقتدارگرا به نظام دموکراتیک & ۱ \\
\altrow تحکیم دموکراسی & \lr{Democratic Consolidation} & مرحله‌ای که دموکراسی «تنها بازی شهر» شود (لینتز و استپان) & ۱ \\
آزادسازی & \lr{Liberalization} & گشایش فضای سیاسی بدون انتقال کامل قدرت & ۱ \\
\altrow دموکراتیزاسیون & \lr{Democratization} & فرآیند نهادینه‌سازی رقابت سیاسی آزاد و عادلانه & ۱ \\
موج دموکراتیزاسیون & \lr{Democratization Wave} & دوره‌های تاریخی افزایش گذارهای دموکراتیک (هانتینگتون: سه موج) & ۱ \\
\altrow بازگشت اقتدارگرایی & \lr{Democratic Backsliding} & فرآیند تدریجی تضعیف نهادهای دموکراتیک & ۷، ت \\
اقتدارگرایی رقابتی & \lr{Competitive Authoritarianism} & رژیمی با نهادهای دموکراتیک صوری اما رقابت ناعادلانه (لویتسکی و وِی) & ۱ \\
\altrow خستگی دموکراتیک & \lr{Democratic Fatigue} & سرخوردگی مردم از ناکارآمدی دموکراسی نوپا & ۷، ت \\
گذار مذاکره‌ای & \lr{Pacted Transition} & گذار طی توافق بین نخبگان رژیم و اپوزیسیون & ۴، الف \\
\altrow گذار از بالا & \lr{Top-down Transition} & تحول آغازشده توسط نخبگان درون رژیم & ۴ \\
فروپاشی & \lr{Regime Collapse} & سقوط ناگهانی رژیم بدون مذاکره & ۴ \\
\altrow شکست‌خوردگان گذار & \lr{Transition Losers} & گروه‌هایی که از فرآیند گذار آسیب می‌بینند & ث \\
اثر دومینو & \lr{Domino Effect} & سرایت گذار از یک کشور به همسایگان & ث \\
\altrow بندهای غروب & \lr{Sunset Clauses} & تضمین‌های موقت برای نخبگان رژیم پیشین & ۶، ب \\
قفل‌های نهادی & \lr{Institutional Locks} & سازوکارهای حفظ قدرت رژیم پیشین در ساختار جدید & پ، چ \\
\altrow گذار ناتمام & \lr{Incomplete Transition} & فرآیند گذار که قبل از تحکیم متوقف یا معکوس شود & چ \\

\end{longtable}

%═══════════════════════════════════════════════════════════
\section{ب) نظارت بین‌المللی}
\label{app:glossary:monitoring}
%═══════════════════════════════════════════════════════════

\begin{longtable}{>{\raggedleft\arraybackslash}p{3.5cm} >{\raggedleft\arraybackslash}p{3.5cm} >{\raggedleft\arraybackslash}p{6.5cm} >{\centering\arraybackslash}p{1.5cm}}
\caption{واژگان حوزهٔ نظارت بین‌المللی}
\label{tab:glossary-monitoring} \\

\toprule
\headerrow \textbf{فارسی} & \textbf{انگلیسی} & \textbf{تعریف} & \textbf{فصل} \\
\midrule
\endfirsthead
\bottomrule
\endlastfoot

نظارت انتخاباتی & \lr{Election Monitoring/Observation} & حضور ناظران بین‌المللی در فرآیند انتخابات & ۳ \\
\altrow نظارت ساختاری & \lr{Structural Monitoring} & نظارت بلندمدت بر نهادسازی و فرآیندهای حکمرانی & ۳ \\
مشاوره فنی & \lr{Technical Assistance} & ارائهٔ کمک تخصصی بدون مداخله در تصمیم‌گیری & ۳ \\
\altrow مدیریت مستقیم & \lr{International Administration} & ادارهٔ موقت سرزمین توسط سازمان بین‌المللی (مدل ۵) & ۳، ح \\
مدل ترکیبی-تطبیقی & \lr{Hybrid-Adaptive Model} & مدل ۶ پیشنهادی: ترکیب بهترین عناصر با تطبیق مرحله‌ای & ۳ \\
\altrow مالکیت ملی & \lr{National Ownership} & اصل تصمیم‌گیری نهایی توسط شهروندان کشور هدف & ۳، ۸ \\
حاکمیت مشروط & \lr{Conditional Sovereignty} & تعهد دولت‌ها به حقوق بشر به‌عنوان شرط حاکمیت & ۱ \\
\altrow مسئولیت حمایت & \lr{R2P (Responsibility to Protect)} & اصل حقوقی مداخلهٔ بین‌المللی در صورت جنایات گسترده & ۱ \\
شمارش موازی آرا & \lr{PVT (Parallel Vote Tabulation)} & شمارش مستقل آرا توسط ناظران مدنی & پ \\
\altrow نظارت شهروندی & \lr{Citizen Monitoring} & نظارت توسط شبکه‌های داوطلب داخلی & ۳ \\
نمایندهٔ ویژه & \lr{SRSG} & نمایندهٔ ویژهٔ دبیرکل سازمان ملل & ۸، ۹ \\
\altrow صندوق امانی & \lr{Multi-Donor Trust Fund} & صندوق مالی چندجانبه تحت مدیریت مشترک & ۱۰ \\
مأموریت ناظران & \lr{Observer Mission} & مأموریت سازمان ملل با وظیفهٔ نظارت (نه اجرا) & ۵ \\
\altrow حافظان صلح & \lr{Peacekeepers} & نیروهای نظامی-مدنی سازمان ملل برای حفظ امنیت & ۵ \\
گروه تماس & \lr{Contact Group} & گروه دولت‌های کلیدی ذی‌نفع برای هماهنگی دیپلماتیک & ۹ \\

\end{longtable}

%═══════════════════════════════════════════════════════════
\section{پ) عدالت انتقالی}
\label{app:glossary:tj}
%═══════════════════════════════════════════════════════════

\begin{longtable}{>{\raggedleft\arraybackslash}p{3.5cm} >{\raggedleft\arraybackslash}p{3.5cm} >{\raggedleft\arraybackslash}p{6.5cm} >{\centering\arraybackslash}p{1.5cm}}
\caption{واژگان حوزهٔ عدالت انتقالی}
\label{tab:glossary-tj} \\

\toprule
\headerrow \textbf{فارسی} & \textbf{انگلیسی} & \textbf{تعریف} & \textbf{فصل} \\
\midrule
\endfirsthead
\bottomrule
\endlastfoot

عدالت انتقالی & \lr{Transitional Justice} & مجموعه‌سازوکارها برای مواجهه با نقض حقوق بشر رژیم پیشین & ۶، الف \\
\altrow کمیسیون حقیقت و آشتی & \lr{TRC} & نهاد رسمی حقیقت‌یابی با اختیار عفو مشروط (الگوی آفریقای جنوبی) & ب \\
نهاد حقیقت و کرامت & \lr{IVD} & مدل تونسی عدالت انتقالی شامل بُعد اقتصادی & ت \\
\altrow فرآیند آشتی جامعه‌محور & \lr{CRP} & مکانیزم آشتی محلی: اعتراف + خدمت اجتماعی (تیمور شرقی) & ح \\
عفو مشروط & \lr{Conditional Amnesty} & عفو در ازای اعتراف کامل و علنی & ب \\
\altrow لوستراسیون & \lr{Lustration} & بررسی سوابق مقامات رژیم پیشین & ث \\
لوستراسیون هوشمند & \lr{Smart Lustration} & بررسی فردی (نه جمعی) با تفکیک عاملان اصلی از اعضای عادی & ج \\
\altrow اجتثاث & \lr{De-Ba'athification} & پاکسازی جمعی اعضای حزب/نظام (مدل عراقی ناموفق) & ج \\
غرامت & \lr{Reparations} & جبران مالی/نمادین به قربانیان & پ \\
\altrow صلاحیت جهانی & \lr{Universal Jurisdiction} & صلاحیت دادگاه هر کشور برای محاکمهٔ جنایات بین‌المللی & پ \\
عدالت تدریجی & \lr{Incremental Justice} & مدل شیلیایی: حقیقت ← غرامت ← محاکمه در طول دهه‌ها & پ \\
\altrow حافظهٔ جمعی & \lr{Collective Memory} & فرآیند اجتماعی یادآوری و مستندسازی گذشته & ث \\
مؤسسهٔ حافظهٔ ملی & \lr{IPN} & نهاد حفظ آرشیو و تحقیق تاریخی (مدل لهستان) & ث \\
\altrow اوبونتو & \lr{Ubuntu} & فلسفهٔ آفریقایی «من هستم چون ما هستیم»؛ مبنای \lr{TRC} & ب \\
ضد الگو & \lr{Counter-Model} & نمونه‌ای که درس «چه نباید کرد» می‌دهد & ج \\

\end{longtable}

%═══════════════════════════════════════════════════════════
\section{ت) اصلاح بخش امنیتی}
\label{app:glossary:ssr}
%═══════════════════════════════════════════════════════════

\begin{longtable}{>{\raggedleft\arraybackslash}p{3.5cm} >{\raggedleft\arraybackslash}p{3.5cm} >{\raggedleft\arraybackslash}p{6.5cm} >{\centering\arraybackslash}p{1.5cm}}
\caption{واژگان حوزهٔ اصلاح بخش امنیتی}
\label{tab:glossary-ssr} \\

\toprule
\headerrow \textbf{فارسی} & \textbf{انگلیسی} & \textbf{تعریف} & \textbf{فصل} \\
\midrule
\endfirsthead
\bottomrule
\endlastfoot

اصلاح بخش امنیتی & \lr{SSR (Security Sector Reform)} & بازسازی و اصلاح نهادهای نظامی، پلیسی و اطلاعاتی & ۶ \\
\altrow خلع‌سلاح و بازادغام & \lr{DDR (Disarmament, Demobilization, Reintegration)} & فرآیند خلع‌سلاح، بسیج‌زدایی و ادغام نیروها در جامعه & ۶ \\
ادغام نیروها & \lr{Force Integration} & ترکیب نیروهای رقیب در یک ارتش واحد (مدل آفریقای جنوبی) & ب \\
\altrow تفکیک اقتصادی ارتش & \lr{Military-Economic Separation} & جداسازی فعالیت‌های اقتصادی از ساختار نظامی (مدل اندونزی) & الف \\
نظارت مدنی بر ارتش & \lr{Civilian Oversight of Military} & اصل تبعیت نیروهای مسلح از مقامات غیرنظامی منتخب & ۶ \\
\altrow بازنشستگی افتخاری & \lr{Honorable Retirement} & بستهٔ مالی برای ترغیب نظامیان به ترک خدمت & ب \\
تاتمادو & \lr{Tatmadaw} & نیروهای مسلح میانمار؛ الگوی ارتش اصلاح‌ناشده & چ \\
\altrow انحلال ارتش & \lr{Military Dissolution} & منحل‌کردن کامل نیروهای مسلح (مدل عراقی فاجعه‌بار) & ج \\

\end{longtable}

%═══════════════════════════════════════════════════════════
\section{ث) حقوق بشر و قانون اساسی}
\label{app:glossary:rights}
%═══════════════════════════════════════════════════════════

\begin{longtable}{>{\raggedleft\arraybackslash}p{3.5cm} >{\raggedleft\arraybackslash}p{3.5cm} >{\raggedleft\arraybackslash}p{6.5cm} >{\centering\arraybackslash}p{1.5cm}}
\caption{واژگان حوزهٔ حقوق بشر و قانون اساسی}
\label{tab:glossary-rights} \\

\toprule
\headerrow \textbf{فارسی} & \textbf{انگلیسی} & \textbf{تعریف} & \textbf{فصل} \\
\midrule
\endfirsthead
\bottomrule
\endlastfoot

منشور حقوق & \lr{Bill of Rights} & فصل حقوق بنیادین در قانون اساسی & ب \\
\altrow دولت مدنی & \lr{Civil State / État civil} & دولتی نه دینی و نه نظامی (مفهوم تونسی) & ت \\
نمایندگی برابر & \lr{Parité / Gender Parity} & اصل برابری جنسیتی در نمایندگی سیاسی & ت \\
\altrow مجلس مؤسسان & \lr{Constituent Assembly} & نهاد منتخب برای تدوین قانون اساسی جدید & ۹ \\
مشروطیت اروپایی & \lr{EU Conditionality} & شرایط دموکراتیک و اقتصادی عضویت \lr{EU} & ث \\
\altrow معیارهای کپنهاگ & \lr{Copenhagen Criteria} & سه شرط عضویت \lr{EU}: دموکراسی + بازار + ظرفیت & ث \\
دادگاه قانون اساسی & \lr{Constitutional Court} & نهاد قضایی ناظر بر انطباق قوانین با قانون اساسی & ۶ \\
\altrow اجماع کافی & \lr{Sufficient Consensus} & اصل توافق اکثریت قریب به اتفاق (نه اتفاق آرا) & ب \\
محاصصه & \lr{Muhasasa / Sectarian Power-Sharing} & تقسیم قدرت فرقه‌ای (مدل عراقی ناموفق) & ج \\

\end{longtable}

%═══════════════════════════════════════════════════════════
\section{ج) اقتصاد و توسعه}
\label{app:glossary:economy}
%═══════════════════════════════════════════════════════════

\begin{longtable}{>{\raggedleft\arraybackslash}p{3.5cm} >{\raggedleft\arraybackslash}p{3.5cm} >{\raggedleft\arraybackslash}p{6.5cm} >{\centering\arraybackslash}p{1.5cm}}
\caption{واژگان حوزهٔ اقتصاد و توسعه}
\label{tab:glossary-economy} \\

\toprule
\headerrow \textbf{فارسی} & \textbf{انگلیسی} & \textbf{تعریف} & \textbf{فصل} \\
\midrule
\endfirsthead
\bottomrule
\endlastfoot

شوک‌تراپی & \lr{Shock Therapy} & اصلاحات اقتصادی سریع و همزمان (مدل لهستان) & ث \\
\altrow رشد با برابری & \lr{Growth with Equity} & مدل اقتصادی شیلی: رشد + عدالت اجتماعی & پ \\
هزینه-فایده & \lr{Cost-Benefit Analysis} & تحلیل مقایسه‌ای هزینه‌ها و منافع & ۱۰ \\
\altrow هزینهٔ عدم اقدام & \lr{Cost of Inaction} & هزینهٔ ناشی از عدم مداخله و ادامهٔ وضع موجود & ۱۰ \\
کنفرانس کمک‌دهندگان & \lr{Donors' Conference} & نشست بین‌المللی برای جذب کمک مالی & ۱۰ \\
\altrow لغو تحریم مشروط & \lr{Conditional Sanctions Relief} & رفع تحریم‌ها مشروط به اصلاحات مشخص & ۶ \\

\end{longtable}

%═══════════════════════════════════════════════════════════
\section{چ) مدیریت پروژه و برنامه‌ریزی}
\label{app:glossary:management}
%═══════════════════════════════════════════════════════════

\begin{longtable}{>{\raggedleft\arraybackslash}p{3.5cm} >{\raggedleft\arraybackslash}p{3.5cm} >{\raggedleft\arraybackslash}p{6.5cm} >{\centering\arraybackslash}p{1.5cm}}
\caption{واژگان حوزهٔ مدیریت و برنامه‌ریزی}
\label{tab:glossary-management} \\

\toprule
\headerrow \textbf{فارسی} & \textbf{انگلیسی} & \textbf{تعریف} & \textbf{فصل} \\
\midrule
\endfirsthead
\bottomrule
\endlastfoot

ماتریس مسئولیت & \lr{RACI Matrix} & جدول تقسیم مسئولیت: مسئول، تأییدکننده، مشاور، مطلع & ۹ \\
\altrow شاخص عملکرد کلیدی & \lr{KPI} & معیار کمّی سنجش پیشرفت & ۹ \\
چرخهٔ بهبود مستمر & \lr{PDCA (Plan-Do-Check-Act)} & چرخهٔ برنامه‌ریزی، اجرا، بررسی، اصلاح & ۹ \\
\altrow نظام هشدار زودهنگام & \lr{Early Warning System} & سیستم تشخیص علائم بحران قبل از وقوع & ۷ \\
مدل \lr{DIME+H} & \lr{DIME+H} & چارچوب تحلیل: دیپلماتیک، اطلاعاتی، نظامی، اقتصادی + انسانی & ۱ \\
\altrow نقشهٔ ذی‌نفعان & \lr{Stakeholder Mapping} & تحلیل بازیگران بر اساس قدرت و منافع & ۵ \\
نقشهٔ حرارتی ریسک & \lr{Risk Heat Map} & نمایش بصری ریسک‌ها بر اساس احتمال و شدت & ۷ \\
\altrow تاب‌آوری & \lr{Resilience} & ظرفیت سیستم برای بازیابی پس از شوک & ۷ \\
منحنی انتقال & \lr{Transition Curve} & مدل انتقال تدریجی مسئولیت (۳۰→۵۰→۸۰) & ۹ \\

\end{longtable}

%═══════════════════════════════════════════════════════════
\section{ح) نمونه‌های تاریخی — اصطلاحات ویژه}
\label{app:glossary:cases}
%═══════════════════════════════════════════════════════════

\begin{longtable}{>{\raggedleft\arraybackslash}p{3.5cm} >{\raggedleft\arraybackslash}p{3.5cm} >{\raggedleft\arraybackslash}p{6.5cm} >{\centering\arraybackslash}p{1.5cm}}
\caption{اصطلاحات ویژهٔ نمونه‌های تاریخی}
\label{tab:glossary-cases} \\

\toprule
\headerrow \textbf{فارسی} & \textbf{انگلیسی} & \textbf{تعریف} & \textbf{پیوست} \\
\midrule
\endfirsthead
\bottomrule
\endlastfoot

آپارتاید & \lr{Apartheid} & نظام جداسازی نژادی آفریقای جنوبی (۱۹۴۸-۱۹۹۴) & ب \\
\altrow کودسا & \lr{CODESA} & مذاکرات چندجانبهٔ آفریقای جنوبی (۱۹ حزب) & ب \\
پاکت‌های مونکلوا & \lr{Moncloa Pacts} & توافق‌های اسپانیا (۱۹۷۷) & الف \\
\altrow کنسرتاسیون & \lr{Concertación} & ائتلاف ۱۶ حزب اپوزیسیون شیلی & پ \\
پلبیسیت & \lr{Plebiscite} & رفراندوم شیلی ۱۹۸۸ & پ \\
\altrow کمیسیون رتیگ & \lr{Rettig Commission} & اولین کمیسیون حقیقت شیلی (۱۹۹۱) & پ \\
کمیسیون والش & \lr{Valech Commission} & دومین کمیسیون حقیقت شیلی (۲۰۰۳-۲۰۰۴) & پ \\
\altrow همبستگی & \lr{Solidarność} & اتحادیهٔ کارگری مستقل لهستان (۱۰M عضو) & ث \\
انقلاب مخملی & \lr{Velvet Revolution} & گذار مسالمت‌آمیز چکسلواکی (۱۹۸۹) & ث \\
\altrow بهار عربی & \lr{Arab Spring} & موج انقلابی ۲۰۱۰-۲۰۱۲ در خاورمیانه & ت \\
چهارگانهٔ نوبل & \lr{Tunisian Quartet} & چهار سازمان مدنی تونس (نوبل ۲۰۱۵) & ت \\
\altrow النهضه & \lr{Ennahda} & حزب اسلام‌گرای میانه‌رو تونس & ت \\
شِگا & \lr{Chega!} & گزارش نهایی کمیسیون حقیقت تیمور شرقی & ح \\
\altrow فرمان شمارهٔ ۲ & \lr{CPA Order No. 2} & فرمان انحلال ارتش عراق (فاجعه) & ج \\
پسران شیکاگو & \lr{Chicago Boys} & اقتصاددانان نئولیبرال شیلی & پ \\

\end{longtable}

\chapterend

%══════════════════════════════════════════════════════════════
% پایان پیوست خ
%══════════════════════════════════════════════════════════════
%══════════════════════════════════════════════════════════════
% پیوست د: فهرست نهادها و سازمان‌های کلیدی
% فایل: appendices/app-j-organizations.tex
% حجم هدف: ۵-۷ صفحه
%══════════════════════════════════════════════════════════════

\chapter{فهرست نهادها و سازمان‌های کلیدی}
\label{app:organizations}

\begin{executivesummary}
این پیوست فهرست جامعی از \textbf{نهادها، سازمان‌ها و نهادهای بین‌المللی} کلیدی مرتبط با فرآیند نظارت بر گذار دموکراتیک ایران ارائه می‌دهد. بیش از \textbf{۸۰ نهاد} در ۹ دسته طبقه‌بندی شده‌اند. هر نهاد با نام فارسی، مخفف انگلیسی، مأموریت مختصر، وب‌سایت، و نقش احتمالی در پروندهٔ ایران معرفی شده است. این فهرست مکمل فصل ۵ (نهادها و بازیگران) است (\seeChapter{ch:actors}).
\end{executivesummary}

%═══════════════════════════════════════════════════════════
\section{نهادهای سازمان ملل متحد}
\label{app:org:un}
%═══════════════════════════════════════════════════════════

\begin{longtable}{>{\raggedleft\arraybackslash}p{3.5cm} >{\centering\arraybackslash}p{1.5cm} >{\raggedleft\arraybackslash}p{5cm} >{\raggedleft\arraybackslash}p{4cm}}
\caption{نهادهای کلیدی سازمان ملل}
\label{tab:org-un} \\

\toprule
\headerrow \textbf{نهاد} & \textbf{مخفف} & \textbf{مأموریت} & \textbf{نقش در ایران} \\
\midrule
\endfirsthead

\multicolumn{4}{c}{\small\textit{ادامهٔ جدول نهادهای سازمان ملل}} \\
\toprule
\headerrow \textbf{نهاد} & \textbf{مخفف} & \textbf{مأموریت} & \textbf{نقش در ایران} \\
\midrule
\endhead

\bottomrule
\endlastfoot

شورای امنیت & \lr{UNSC} & حفظ صلح و امنیت بین‌المللی & قطعنامهٔ تأسیس مأموریت + مجوز \\
\altrow مجمع عمومی & \lr{UNGA} & تصمیم‌گیری عمومی و بودجه & مشروعیت‌بخشی + بودجه \\
دبیرخانه — دپارتمان صلح & \lr{DPPA/DPO} & مأموریت‌های صلح و سیاسی & طراحی و مدیریت \lr{UNMOIT} \\
\altrow \termfn{کمیساریای عالی حقوق بشر}{\lr{OHCHR}} & \lr{OHCHR} & حمایت از حقوق بشر & نظارت حقوق بشری \\
\termfn{برنامهٔ توسعه}{\lr{UNDP}} & \lr{UNDP} & توسعهٔ پایدار + حکمرانی & ظرفیت‌سازی نهادی \\
\altrow \termfn{کمیساریای پناهندگان}{\lr{UNHCR}} & \lr{UNHCR} & حمایت از پناهندگان و آوارگان & بازگشت آوارگان + دیاسپورا \\
\termfn{صندوق کودکان}{\lr{UNICEF}} & \lr{UNICEF} & حقوق کودکان & حمایت از کودکان آسیب‌دیده \\
\altrow \termfn{زنان سازمان ملل}{\lr{UN Women}} & \lr{UN Women} & برابری جنسیتی & نظارت بر سهمیهٔ زنان \\
\termfn{آژانس بین‌المللی انرژی اتمی}{\lr{IAEA}} & \lr{IAEA} & بازرسی هسته‌ای & نظارت بر برنامهٔ هسته‌ای \\
\altrow دیوان بین‌المللی دادگستری & \lr{ICJ} & حل اختلافات بین دولت‌ها & داوری حقوقی \\
دیوان بین‌المللی کیفری & \lr{ICC} & محاکمهٔ جنایات بین‌المللی & صلاحیت تکمیلی (جنایات) \\
\altrow شورای حقوق بشر & \lr{HRC} & بررسی وضعیت حقوق بشر کشورها & گزارشگر ویژهٔ ایران \\
\termfn{دفتر هماهنگی امور انسانی}{\lr{OCHA}} & \lr{OCHA} & هماهنگی کمک‌های انسان‌دوستانه & مدیریت بحران انسانی \\
\altrow \termfn{برنامهٔ جهانی غذا}{\lr{WFP}} & \lr{WFP} & امنیت غذایی & تأمین غذا در فاز بحران \\

\end{longtable}

%═══════════════════════════════════════════════════════════
\section{سازمان‌های منطقه‌ای}
\label{app:org:regional}
%═══════════════════════════════════════════════════════════

\begin{longtable}{>{\raggedleft\arraybackslash}p{3.5cm} >{\centering\arraybackslash}p{1.5cm} >{\raggedleft\arraybackslash}p{5cm} >{\raggedleft\arraybackslash}p{4cm}}
\caption{سازمان‌های منطقه‌ای مرتبط}
\label{tab:org-regional} \\

\toprule
\headerrow \textbf{نهاد} & \textbf{مخفف} & \textbf{مأموریت} & \textbf{نقش احتمالی در ایران} \\
\midrule
\endfirsthead
\bottomrule
\endlastfoot

اتحادیهٔ اروپا & \lr{EU} & ادغام اروپایی + سیاست خارجی & تحریم/مشوق + ناظران + کمک مالی \\
\altrow سازمان امنیت و همکاری اروپا & \lr{OSCE} & امنیت اروپا + نظارت انتخاباتی (\lr{ODIHR}) & ناظران انتخاباتی متخصص \\
شورای اروپا & \lr{CoE} & حقوق بشر + دموکراسی + حاکمیت قانون & کمیسیون ونیز (مشاوره قانون اساسی) \\
\altrow اتحادیهٔ آفریقا & \lr{AU} & صلح و همکاری آفریقا & تجربهٔ نظارت (الگوی آفریقای جنوبی) \\
سازمان همکاری اسلامی & \lr{OIC} & همکاری کشورهای اسلامی & مشروعیت‌بخشی منطقه‌ای \\
\altrow اتحادیهٔ عرب & \lr{AL} & همکاری کشورهای عربی & همسایگان عرب ایران \\
سازمان همکاری شانگهای & \lr{SCO} & همکاری اوراسیا (چین + روسیه) & بازیگر حساس (احتمالاً مقاوم) \\
\altrow اکو & \lr{ECO} & همکاری اقتصادی منطقه‌ای & چارچوب همکاری با همسایگان \\

\end{longtable}

%═══════════════════════════════════════════════════════════
\section{دولت‌های کلیدی}
\label{app:org:states}
%═══════════════════════════════════════════════════════════

\begin{longtable}{>{\raggedleft\arraybackslash}p{3cm} >{\raggedleft\arraybackslash}p{4.5cm} >{\raggedleft\arraybackslash}p{3.5cm} >{\centering\arraybackslash}p{2.5cm}}
\caption{دولت‌های کلیدی و نقش احتمالی}
\label{tab:org-states} \\

\toprule
\headerrow \textbf{کشور} & \textbf{نقش احتمالی} & \textbf{حساسیت/ریسک} & \textbf{اهمیت} \\
\midrule
\endfirsthead
\bottomrule
\endlastfoot

ایالات متحده & لغو تحریم + کمک مالی + فشار & تاریخچهٔ ۱۹۵۳ + بی‌اعتمادی & \rating{5} \\
\altrow بریتانیا & دیپلماسی + \lr{BBC} فارسی + حقوق & تاریخچهٔ استعماری & \rating{4} \\
فرانسه & \lr{EU} + شورای امنیت + فرهنگ & -- & \rating{3} \\
\altrow آلمان & بزرگ‌ترین شریک تجاری اروپایی & ملاحظات اقتصادی & \rating{4} \\
چین & شورای امنیت (وتو) + نفت & احتمالاً مقاوم/معارض & \rating{5} \\
\altrow روسیه & شورای امنیت (وتو) + سلاح & احتمالاً مقاوم/معارض & \rating{5} \\
ترکیه & همسایه + مدل + ناتو & رقابت منطقه‌ای & \rating{4} \\
\altrow عربستان سعودی & رقیب منطقه‌ای + نفت & رقابت ژئوپلیتیکی & \rating{4} \\
امارات & سرمایه‌گذاری + لجستیک & رقابت/فرصت & \rating{3} \\
\altrow عراق & همسایهٔ شیعه + تجربهٔ گذار & بی‌ثباتی + نفوذ ایران & \rating{4} \\
هند & همسایهٔ منطقه‌ای + انرژی & ملاحظات نفتی & \rating{3} \\
\altrow ژاپن + کره & کمک مالی + فناوری & فاصلهٔ جغرافیایی & \rating{3} \\
نروژ + سوئد & میانجی‌گری + مدل & تجربهٔ صلح + مالی & \rating{3} \\
\altrow استرالیا & تجربهٔ تیمور + ظرفیت & فاصلهٔ جغرافیایی & \rating{2} \\

\end{longtable}

%═══════════════════════════════════════════════════════════
\section{سازمان‌های غیردولتی بین‌المللی (\lr{INGO})}
\label{app:org:ngo}
%═══════════════════════════════════════════════════════════

\begin{longtable}{>{\raggedleft\arraybackslash}p{3.5cm} >{\centering\arraybackslash}p{1.5cm} >{\raggedleft\arraybackslash}p{5cm} >{\raggedleft\arraybackslash}p{4cm}}
\caption{سازمان‌های غیردولتی بین‌المللی کلیدی}
\label{tab:org-ngo} \\

\toprule
\headerrow \textbf{نهاد} & \textbf{مخفف} & \textbf{حوزهٔ فعالیت} & \textbf{نقش در ایران} \\
\midrule
\endfirsthead
\bottomrule
\endlastfoot

\termfn{دیده‌بان حقوق بشر}{\lr{Human Rights Watch}} & \lr{HRW} & مستندسازی نقض حقوق بشر & نظارت حقوق بشری \\
\altrow عفو بین‌الملل & \lr{AI} & حقوق بشر + دادخواهی & حمایت از زندانیان سیاسی \\
\termfn{مرکز بین‌المللی عدالت انتقالی}{\lr{ICTJ}} & \lr{ICTJ} & مشاوره در عدالت انتقالی & طراحی کمیسیون حقیقت \\
\altrow \termfn{گروه بحران بین‌المللی}{\lr{ICG}} & \lr{ICG} & تحلیل بحران + پیشگیری & تحلیل سناریوها + هشدار \\
مرکز کارتر & \lr{Carter Center} & نظارت انتخاباتی + میانجی‌گری & ناظران انتخاباتی \\
\altrow مؤسسهٔ صلح آمریکا & \lr{USIP} & تحقیق صلح + میانجی‌گری & مشاوره فرآیند صلح \\
بنیاد بین‌المللی نظام‌های انتخاباتی & \lr{IFES} & فنی انتخاباتی & طراحی سیستم انتخاباتی \\
\altrow \termfn{مؤسسهٔ ملی دموکراسی}{\lr{NDI}} & \lr{NDI} & آموزش دموکراسی + احزاب & ظرفیت‌سازی حزبی \\
مؤسسهٔ بین‌المللی جمهوری‌خواهان & \lr{IRI} & حمایت از دموکراسی & آموزش + نظارت \\
\altrow ایده & \lr{IDEA} & دموکراسی + انتخابات & مشاوره قانون اساسی \\
گزارشگران بدون مرز & \lr{RSF} & آزادی مطبوعات & نظارت رسانه‌ای \\
\altrow شفافیت بین‌الملل & \lr{TI} & ضد فساد & نظارت مالی + ضد فساد \\
\termfn{کمیتهٔ بین‌المللی صلیب سرخ}{\lr{ICRC}} & \lr{ICRC} & حقوق بشردوستانه & بازدید از زندان‌ها + بازگشت \\
\altrow پزشکان بدون مرز & \lr{MSF} & بهداشت در بحران & خدمات پزشکی فاز ۱ \\

\end{longtable}

%═══════════════════════════════════════════════════════════
\section{نهادهای مالی بین‌المللی}
\label{app:org:ifi}
%═══════════════════════════════════════════════════════════

\begin{longtable}{>{\raggedleft\arraybackslash}p{3.5cm} >{\centering\arraybackslash}p{1.5cm} >{\raggedleft\arraybackslash}p{5cm} >{\raggedleft\arraybackslash}p{4cm}}
\caption{نهادهای مالی بین‌المللی}
\label{tab:org-ifi} \\

\toprule
\headerrow \textbf{نهاد} & \textbf{مخفف} & \textbf{مأموریت} & \textbf{نقش در ایران} \\
\midrule
\endfirsthead
\bottomrule
\endlastfoot

بانک جهانی & \lr{WB} & توسعه + کاهش فقر & بازسازی اقتصادی + وام \\
\altrow صندوق بین‌المللی پول & \lr{IMF} & ثبات مالی + اصلاحات & مشاوره اقتصاد کلان \\
بانک توسعهٔ اسلامی & \lr{IsDB} & توسعه در جهان اسلام & تأمین مالی + مشروعیت \\
\altrow بانک سرمایه‌گذاری اروپا & \lr{EIB} & سرمایه‌گذاری توسعه‌ای & زیرساخت \\
سازمان تجارت جهانی & \lr{WTO} & تجارت آزاد & عضویت = مشوق اقتصادی \\
\altrow بانک بازسازی و توسعهٔ اروپا & \lr{EBRD} & بازسازی + خصوصی‌سازی & سرمایه‌گذاری + ظرفیت‌سازی \\

\end{longtable}

%═══════════════════════════════════════════════════════════
\section{نهادهای ایرانی (موجود / پیشنهادی)}
\label{app:org:iranian}
%═══════════════════════════════════════════════════════════

\begin{longtable}{>{\raggedleft\arraybackslash}p{4cm} >{\centering\arraybackslash}p{1.8cm} >{\raggedleft\arraybackslash}p{4.5cm} >{\raggedleft\arraybackslash}p{3.5cm}}
\caption{نهادهای ایرانی موجود و پیشنهادی}
\label{tab:org-iranian} \\

\toprule
\headerrow \textbf{نهاد} & \textbf{وضعیت} & \textbf{نقش} & \textbf{فصل مرتبط} \\
\midrule
\endfirsthead
\bottomrule
\endlastfoot

ارتش ملی واحد & پیشنهادی & ادغام سپاه + ارتش تحت نظارت مدنی & ۶، ب \\
\altrow کمیسیون حقیقت و کرامت ایران & پیشنهادی & حقیقت‌یابی + عفو مشروط + بُعد اقتصادی & ۶، ب، ت \\
مجلس مؤسسان & پیشنهادی & تدوین قانون اساسی جدید & ۹ \\
\altrow کمیسیون ملی انتخابات & پیشنهادی & مدیریت مستقل انتخابات & ۸ \\
مؤسسهٔ حافظهٔ ملی ایران & پیشنهادی & حفظ آرشیو + تحقیق تاریخی & ث \\
\altrow دادگاه ویژهٔ رسیدگی به جرایم رژیم & پیشنهادی & محاکمهٔ فرماندهان ارشد & ۸ \\
نهاد ضد فساد مستقل & پیشنهادی & نظارت مالی + شفافیت & ۸ \\
\altrow کمیسیون مستقل رسانه & پیشنهادی & آزادی مطبوعات + نظارت & ۸ \\
شورای مشورتی ملی & پیشنهادی & مشارکت جامعهٔ مدنی & ۹ \\
\altrow صندوق امانی بازسازی ایران & پیشنهادی & مدیریت کمک‌های مالی & ۱۰ \\
چهارگانهٔ ایرانی & پیشنهادی & میانجی‌گری (الگوی تونس) & ت \\
\altrow کانون وکلای دادگستری & موجود & نظارت حقوقی + دفاع از حقوق & ت \\
تشکل‌های صنفی مستقل & موجود (سرکوب) & ائتلاف‌سازی + فشار مدنی & ث \\
\altrow جنبش زن، زندگی، آزادی & موجود & حقوق زنان + فراگیری & ۲، ۶ \\

\end{longtable}

%═══════════════════════════════════════════════════════════
\section{مراکز تحقیقاتی و آکادمیک}
\label{app:org:academic}
%═══════════════════════════════════════════════════════════

\begin{longtable}{>{\raggedleft\arraybackslash}p{4cm} >{\raggedleft\arraybackslash}p{4cm} >{\raggedleft\arraybackslash}p{6cm}}
\caption{مراکز تحقیقاتی کلیدی}
\label{tab:org-academic} \\

\toprule
\headerrow \textbf{نهاد} & \textbf{حوزه} & \textbf{ارتباط با ایران} \\
\midrule
\endfirsthead
\bottomrule
\endlastfoot

\lr{V-Dem Institute} (گوتنبرگ) & شاخص‌های دموکراسی & سنجش پیشرفت ایران \\
\altrow \lr{Freedom House} & آزادی سیاسی و مدنی & رتبه‌بندی ایران \\
\lr{Chatham House} (لندن) & سیاست خارجی + ایران & برنامهٔ خاورمیانه + ایران \\
\altrow \lr{Brookings Institution} & سیاست عمومی & برنامهٔ ایران و خاورمیانه \\
\lr{Carnegie Endowment} & صلح بین‌المللی & تحلیل سیاسی ایران \\
\altrow \lr{RAND Corporation} & تحلیل راهبردی & سناریوسازی امنیتی \\
دانشگاه‌های \lr{SOAS + Oxford + Harvard} & مطالعات ایرانی & پژوهش + آموزش نخبگان \\
\altrow \lr{Institute for War \& Peace Reporting} & رسانه در بحران & آموزش خبرنگاران ایرانی \\
مؤسسهٔ هاینریش بُل (آلمان) & دموکراسی + محیط‌زیست & حمایت از جامعهٔ مدنی \\
\altrow صندوق ملی دموکراسی (\lr{NED}) & حمایت از دموکراسی & تأمین مالی نهادهای مدنی ایرانی \\

\end{longtable}

%═══════════════════════════════════════════════════════════
\section{رسانه‌های بین‌المللی فارسی‌زبان}
\label{app:org:media}
%═══════════════════════════════════════════════════════════

\begin{longtable}{>{\raggedleft\arraybackslash}p{3.5cm} >{\raggedleft\arraybackslash}p{3cm} >{\raggedleft\arraybackslash}p{4cm} >{\raggedleft\arraybackslash}p{3.5cm}}
\caption{رسانه‌های کلیدی فارسی‌زبان و نقش احتمالی}
\label{tab:org-media} \\

\toprule
\headerrow \textbf{رسانه} & \textbf{مالکیت} & \textbf{مخاطب} & \textbf{نقش احتمالی} \\
\midrule
\endfirsthead
\bottomrule
\endlastfoot

\lr{BBC Persian} & بریتانیا & ایران + دیاسپورا & پوشش خبری + اطلاع‌رسانی \\
\altrow \lr{Iran International} & خصوصی (لندن) & ایران + دیاسپورا & پوشش خبری \\
\lr{VOA Persian} & آمریکا & ایران + دیاسپورا & اطلاع‌رسانی \\
\altrow \lr{Radio Farda / RFE} & آمریکا & ایران & پوشش رادیویی \\
\lr{DW Farsi} & آلمان & ایران + دیاسپورا & اطلاع‌رسانی بی‌طرف \\
\altrow \lr{France 24 Farsi} & فرانسه & ایران + دیاسپورا & تنوع منابع \\
رسانه‌های مستقل ایرانی & خصوصی/مدنی & ایران & ضد اطلاعات نادرست + راستی‌آزمایی \\

\end{longtable}

\begin{warningbox}
\textbf{نکتهٔ حساسیت رسانه‌ای:} رسانه‌های فارسی‌زبان دولتی (آمریکا، بریتانیا) در ایران با بی‌اعتمادی مواجه هستند. مدل ۶ توصیه می‌کند: ۱) از \textbf{رسانه‌های مستقل ایرانی} حمایت مالی شود (نه دولتی)؛ ۲) \textbf{سامانهٔ راستی‌آزمایی} مستقل تأسیس شود؛ ۳) \textbf{استارلینک} و ابزارهای دسترسی آزاد به اینترنت فراهم شود (\seeChapter{ch:timeline}).
\end{warningbox}

\sectiondivider

%═══════════════════════════════════════════════════════════
\section{نمودار: شبکهٔ نهادهای کلیدی مرتبط با ایران}
\label{app:org:network}
%═══════════════════════════════════════════════════════════

\begin{figure}[htbp]
\centering
\begin{tikzpicture}[
  node distance=2cm,
  core/.style={
    draw=MainPurple, fill=MainPurple!15, rounded corners=5pt,
    minimum width=2.5cm, minimum height=1cm, font=\small\bfseries,
    align=center, thick
  },
  un/.style={
    draw=MainBlue, fill=MainBlue!10, rounded corners=3pt,
    minimum width=2cm, minimum height=0.7cm, font=\tiny,
    align=center
  },
  regional/.style={
    draw=MainGreen, fill=MainGreen!10, rounded corners=3pt,
    minimum width=2cm, minimum height=0.7cm, font=\tiny,
    align=center
  },
  ngo/.style={
    draw=MainOrange, fill=MainOrange!10, rounded corners=3pt,
    minimum width=2cm, minimum height=0.7cm, font=\tiny,
    align=center
  },
  iranian/.style={
    draw=MainRed, fill=MainRed!10, rounded corners=3pt,
    minimum width=2cm, minimum height=0.7cm, font=\tiny,
    align=center
  },
  finance/.style={
    draw=MainYellow!80!black, fill=MainYellow!10, rounded corners=3pt,
    minimum width=2cm, minimum height=0.7cm, font=\tiny,
    align=center
  },
  link/.style={-, gray!50, thin},
  stronglink/.style={-, MainPurple!60, thick}
]

% هستهٔ مرکزی
\node[core] (iran) at (0,0) {گذار\\ایران};

% سازمان ملل (بالا)
\node[un] (unsc) at (-3,3.5) {\lr{UNSC}\\شورای امنیت};
\node[un] (srsg) at (0,3.5) {\lr{SRSG/UNMOIT}\\مأموریت ایران};
\node[un] (iaea) at (3,3.5) {\lr{IAEA}\\هسته‌ای};
\node[un] (ohchr) at (-1.5,2.3) {\lr{OHCHR}\\حقوق بشر};
\node[un] (undp) at (1.5,2.3) {\lr{UNDP}\\توسعه};

% منطقه‌ای (چپ)
\node[regional] (eu) at (-5,1) {\lr{EU}\\اتحادیهٔ اروپا};
\node[regional] (osce) at (-5,-0.5) {\lr{OSCE/ODIHR}\\ناظران};
\node[regional] (oic) at (-5,-2) {\lr{OIC}\\همکاری اسلامی};

% دولت‌ها (راست)
\node[ngo] (us) at (5,1.5) {آمریکا\\تحریم/مشوق};
\node[ngo] (china) at (5,0) {چین + روسیه\\وتو/حمایت؟};
\node[ngo] (turkey) at (5,-1.5) {ترکیه + عربستان\\همسایگان};

% مالی (پایین-راست)
\node[finance] (wb) at (3,-3) {\lr{WB/IMF}\\بازسازی};
\node[finance] (fund) at (0,-3.5) {صندوق امانی\\بازسازی ایران};

% ایرانی (پایین-چپ)
\node[iranian] (const) at (-3,-3) {مجلس مؤسسان\\ایران};
\node[iranian] (trc) at (-1.5,-2.3) {کمیسیون حقیقت\\ایران};
\node[iranian] (civil) at (1.5,-2.3) {جامعهٔ مدنی\\ایران};

% NGOها (بالا-راست)
\node[ngo] (hrw) at (4.5,3) {\lr{HRW/AI}\\حقوق بشر};
\node[ngo] (ictj) at (3,2.3) {\lr{ICTJ}\\عدالت انتقالی};

% اتصالات قوی
\draw[stronglink] (iran) -- (srsg);
\draw[stronglink] (iran) -- (const);
\draw[stronglink] (iran) -- (trc);
\draw[stronglink] (iran) -- (civil);
\draw[stronglink] (iran) -- (fund);
\draw[stronglink] (srsg) -- (unsc);

% اتصالات عادی
\draw[link] (iran) -- (eu);
\draw[link] (iran) -- (osce);
\draw[link] (iran) -- (oic);
\draw[link] (iran) -- (us);
\draw[link] (iran) -- (china);
\draw[link] (iran) -- (turkey);
\draw[link] (iran) -- (wb);
\draw[link] (srsg) -- (ohchr);
\draw[link] (srsg) -- (undp);
\draw[link] (srsg) -- (iaea);
\draw[link] (srsg) -- (hrw);
\draw[link] (srsg) -- (ictj);
\draw[link] (fund) -- (wb);
\draw[link] (fund) -- (eu);
\draw[link] (trc) -- (ictj);
\draw[link] (const) -- (eu);

% راهنما
\node[font=\tiny, MainBlue] at (-5.5,3.5) {$\blacksquare$ سازمان ملل};
\node[font=\tiny, MainGreen] at (-5.5,3) {$\blacksquare$ منطقه‌ای};
\node[font=\tiny, MainOrange] at (-5.5,2.5) {$\blacksquare$ دولت‌ها/\lr{NGO}};
\node[font=\tiny, MainRed] at (-5.5,2) {$\blacksquare$ ایرانی};
\node[font=\tiny, MainYellow!80!black] at (-5.5,1.5) {$\blacksquare$ مالی};

\end{tikzpicture}
\caption{شبکهٔ نهادهای کلیدی مرتبط با نظارت بر گذار ایران}
\label{fig:app-org-network}
\end{figure}

\sectiondivider

%═══════════════════════════════════════════════════════════
\section{جمع‌بندی پیوست}
\label{app:org:conclusion}
%═══════════════════════════════════════════════════════════

\begin{chaptersummary}
جمع‌بندی پیوست د — فهرست نهادها و سازمان‌ها:

\begin{enumerate}[nosep]
\item بیش از \textbf{۸۰ نهاد} در ۹ دسته معرفی شدند: سازمان ملل (۱۴)، منطقه‌ای (۸)، دولت‌ها (۱۴)، \lr{NGO}ها (۱۴)، مالی (۶)، ایرانی (۱۴)، آکادمیک (۱۰)، رسانه‌ای (۷).
\item \textbf{شورای امنیت سازمان ملل} مرجع تصمیم‌گیری اصلی است — ریسک وتوی چین و روسیه باید مدیریت شود.
\item \textbf{\lr{SRSG} و مأموریت \lr{UNMOIT}} هستهٔ مرکزی نظارت بین‌المللی خواهد بود.
\item \textbf{نهادهای ایرانی پیشنهادی} (مجلس مؤسسان، کمیسیون حقیقت، ارتش ملی واحد) \textbf{اصلی‌ترین بازیگران} هستند — مالکیت ملی.
\item \textbf{چهارگانهٔ ایرانی} (الگوی تونس) و \textbf{تشکل‌های صنفی} نقش میانجی‌گری خواهند داشت.
\item \textbf{رسانه‌های مستقل ایرانی} باید تقویت شوند — رسانه‌های دولتی خارجی بی‌اعتمادی ایجاد می‌کنند.
\item \lr{IAEA} نقش ویژه‌ای در بُعد هسته‌ای دارد — مکمل فرآیند سیاسی.
\item صندوق امانی بازسازی ایران (\$۲.۵-۵B) با مشارکت بانک جهانی، \lr{EU}، و کشورهای کمک‌دهنده تأمین خواهد شد.
\end{enumerate}

\vspace{0.3cm}
\textit{ارجاعات:}
\begin{itemize}[nosep]
\item نهادها و بازیگران (تفصیلی): \seeChapter{ch:actors}
\item بودجه‌بندی و تأمین مالی: \seeChapter{ch:budget}
\item زمان‌بندی و تیم‌سازی: \seeChapter{ch:timeline}
\item واژه‌نامهٔ تخصصی: \seeChapter{app:glossary}
\end{itemize}
\end{chaptersummary}

\chapterend

%══════════════════════════════════════════════════════════════
% پایان پیوست د
%══════════════════════════════════════════════════════════════


% ---- منابع ----
\backmatter
\printbibliography[
    heading=bibintoc,
    title={فهرست منابع و مراجع}
]

% ---- واژه‌نامه ----
\printglossaries

% ---- نمایه ----
\printindex

\end{document}