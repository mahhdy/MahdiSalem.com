%══════════════════════════════════════════════════════════════
% پیوست ث: مطالعه موردی لهستان و اروپای شرقی
% فایل: appendices/app-e-poland.tex
% حجم هدف: ۸-۱۰ صفحه
%══════════════════════════════════════════════════════════════

\chapter{مطالعهٔ موردی: لهستان و اروپای شرقی (۱۹۸۹-۲۰۰۴)}
\label{app:poland}

\begin{executivesummary}
لهستان پیشگام گذار دموکراتیک در اروپای شرقی بود و سقوط کمونیسم در این منطقه را رقم زد. ویژگی‌های منحصربه‌فرد: ۱) \textbf{جنبش همبستگی} (\lr{Solidarność}) به‌عنوان بزرگ‌ترین جنبش اجتماعی غیرخشونت‌آمیز قرن بیستم (۱۰ میلیون عضو)، ۲) \textbf{مذاکرات میزگرد} (فوریه-آوریل ۱۹۸۹) که مدل «گذار مذاکره‌ای» در شرایط بحران اقتصادی را ارائه کرد، ۳) \textbf{نقش کلیسای کاتولیک} و شخصیت \person{پاپ ژان‌پل دوم}{\lr{Pope John Paul II}} در مشروعیت‌زدایی از رژیم، ۴) \textbf{مشوق عضویت در \lr{EU} و \lr{NATO}} به‌عنوان قوی‌ترین ابزار تحکیم دموکراسی، و ۵) هشدار اخیر \textbf{بازگشت اقتدارگرایی} (دورهٔ حزب \lr{PiS} ۲۰۱۵-۲۰۲۳). این پیوست همچنین به تجربه‌های مشابه در \textbf{جمهوری چک}، \textbf{مجارستان}، و \textbf{آلمان شرقی} اشاره می‌کند.
\end{executivesummary}

%═══════════════════════════════════════════════════════════
\section{زمینه و بافت تاریخی}
\label{app:poland:context}
%═══════════════════════════════════════════════════════════

\subsection{لهستان کمونیستی: ویژگی‌ها و تفاوت‌ها}

لهستان در بلوک شرقی وضعیت ویژه‌ای داشت — نه تمامیت‌خواه‌ترین (مانند رومانی) و نه آزادترین (مانند مجارستان):

\begin{table}[htbp]
\centering
\caption{مشخصات لهستان در آستانهٔ گذار (۱۹۸۹)}
\label{tab:app-poland-profile}
\begin{tabularx}{\textwidth}{>{\raggedleft\arraybackslash}p{4.5cm} >{\raggedleft\arraybackslash}X}
\toprule
\headerrow \textbf{شاخص} & \textbf{مقدار} \\
\midrule
جمعیت & ۳۸ میلیون نفر \\
\altrow مساحت & ۳۱۲,۰۰۰ \lr{km²} \\
تنوع قومی & بسیار پایین (۹۷٪ لهستانی) \\
\altrow \lr{GDP per capita} & $\sim$\$۱,۷۰۰ \\
طول عمر رژیم کمونیستی & ۴۵ سال (۱۹۴۴-۱۹۸۹) \\
\altrow اندازهٔ ارتش & $\sim$۴۰۰,۰۰۰ \\
حزب حاکم & \lr{PZPR} (حزب متحد کارگران لهستان) \\
\altrow وضعیت اقتصاد & بحران شدید (تورم ۶۰۰٪، صف‌های طولانی) \\
جامعهٔ مدنی & قوی‌ترین در بلوک شرقی (همبستگی) \\
\altrow نقش کلیسا & بسیار بالا (۹۵٪ کاتولیک) \\
حکومت نظامی & ۱۹۸۱-۱۹۸۳ (ژنرال \lr{Jaruzelski}) \\
\altrow سلاح‌های هسته‌ای شوروی & مستقر در خاک لهستان \\
\bottomrule
\end{tabularx}
\end{table}

\subsection{ستون‌های رژیم و مقایسه با ایران}

\begin{table}[htbp]
\centering
\caption{مقایسهٔ ساختاری رژیم کمونیستی لهستان و جمهوری اسلامی}
\label{tab:app-poland-iran-structure}
\begin{tabularx}{\textwidth}{>{\raggedleft\arraybackslash}p{3cm} >{\raggedleft\arraybackslash}X >{\raggedleft\arraybackslash}X}
\toprule
\headerrow \textbf{بُعد} & \textbf{لهستان کمونیستی} & \textbf{جمهوری اسلامی} \\
\midrule
ایدئولوژی & مارکسیسم-لنینیسم (تحمیلی) & اسلام سیاسی / ولایت فقیه \\
\altrow رهبری & حزب واحد (\lr{PZPR}) + شوروی & ولی‌فقیه + سپاه + روحانیت \\
نیروی خارجی حامی & شوروی (تعیین‌کننده) & خودبسنده (فاقد حامی مستقیم) \\
\altrow ارتش & تحت فرمان حزب + پیمان ورشو & سپاه مستقل + ارتش تحت‌فرمان \\
اقتصاد & دولتی-برنامه‌ای (بحرانی) & نیمه‌دولتی (نفت + سپاه + بنیادها) \\
\altrow مذهب & مخالف (اما ناتوان از حذف کلیسا) & پایهٔ مشروعیت (دین = قدرت) \\
جامعهٔ مدنی & قوی (همبستگی ۱۰M) & فعال اما سرکوب‌شده \\
\altrow دیاسپورا & محدود (مهاجرت بعد از گذار) & بسیار بزرگ (۴-۵M) \\
\bottomrule
\end{tabularx}
\end{table}

\begin{casestudy}
\textbf{تفاوت ساختاری کلیدی:} رژیم کمونیستی لهستان \textbf{وابسته به شوروی} بود — با فروپاشی شوروی، پایهٔ خارجی رژیم از بین رفت. جمهوری اسلامی فاقد «شوروی» است و \textbf{خودبسنده‌تر} عمل می‌کند. در عوض، تفاوت دوم مهم‌تر است: لهستان \textbf{مشوق عضویت \lr{EU} و \lr{NATO}} داشت — قوی‌ترین ابزار تحکیم. ایران فاقد معادل این مشوق است (\seeChapter{ch:guarantees}).
\end{casestudy}

\sectiondivider

%═══════════════════════════════════════════════════════════
\section{جنبش همبستگی: الگوی جنبش اجتماعی}
\label{app:poland:solidarity}
%═══════════════════════════════════════════════════════════

\subsection{تاریخچه و ویژگی‌ها}

\org{اتحادیهٔ کارگری مستقل و خودگردان همبستگی}{\lr{Niezależny Samorządny Związek Zawodowy "Solidarność"}} در اوت ۱۹۸۰ در کارخانهٔ کشتی‌سازی گدانسک تأسیس شد:

\begin{table}[htbp]
\centering
\caption{ویژگی‌های جنبش همبستگی و مقایسه با اپوزیسیون ایران}
\label{tab:app-poland-solidarity}
\begin{tabularx}{\textwidth}{>{\raggedleft\arraybackslash}p{3.5cm} >{\raggedleft\arraybackslash}X >{\raggedleft\arraybackslash}p{3.5cm}}
\toprule
\headerrow \textbf{ویژگی} & \textbf{همبستگی} & \textbf{اپوزیسیون ایران} \\
\midrule
سال تأسیس & ۱۹۸۰ & فاقد تشکیلات واحد \\
\altrow اعضا (اوج) & ۱۰ میلیون (از ۳۸M) = ۲۶٪ & فاقد عضویت رسمی \\
رهبر & \person{لخ والسا}{\lr{Lech Wałęsa}} (نوبل ۱۹۸۳) & فاقد رهبر واحد \\
\altrow ماهیت & اتحادیهٔ کارگری + جنبش اجتماعی & شبکه‌های پراکنده \\
ایدئولوژی & فراگیر (کارگر + روشنفکر + مذهبی) & متنوع (گاه متضاد) \\
\altrow پشتیبان نهادی & کلیسای کاتولیک & فاقد پشتیبان نهادی \\
ابزار & اعتصاب + نافرمانی مدنی + نشریه & اعتراض خیابانی + فضای مجازی \\
\altrow ممنوعیت & ممنوع ۱۹۸۱-۱۹۸۹ (اما فعال مخفیانه) & سرکوب مستمر \\
بازگشت & ۱۹۸۹: قانونی شد و انتخابات برد & -- \\
\bottomrule
\end{tabularx}
\end{table}

\subsection{رمز موفقیت همبستگی}

\begin{enumerate}[nosep]
\item \textbf{پایگاه طبقاتی:} ریشه در \textbf{طبقهٔ کارگر} — ایدئولوژی رژیم نمی‌توانست کارگران را «دشمن طبقاتی» بنامد
\item \textbf{فراگیری:} روشنفکران (\lr{KOR}) + کارگران + کلیسا + دانشجویان = ائتلاف فراطبقاتی
\item \textbf{حمایت نهادی کلیسا:} کلیسای کاتولیک فضا، پول، و مشروعیت فراهم کرد
\item \textbf{رهبری کاریزماتیک:} والسا نماد وحدت بود (با وجود ضعف‌هایش)
\item \textbf{خشونت‌پرهیزی:} حتی در حکومت نظامی (۱۹۸۱)، جنبش به خشونت روی نیاورد
\item \textbf{ساختار سازمانی:} حتی در دوران ممنوعیت، شبکهٔ زیرزمینی (\lr{podziemie}) فعال بود: ۵۰۰+ نشریهٔ مخفی
\end{enumerate}

\begin{keypoint}
\textbf{فاصلهٔ اپوزیسیون ایران با مدل همبستگی:} اپوزیسیون ایران فاقد سه عنصر کلیدی همبستگی است: ۱) \textbf{تشکیلات گسترده} (۱۰M عضو)؛ ۲) \textbf{رهبر واحد} (والسا)؛ ۳) \textbf{پشتیبان نهادی} (کلیسا). جنبش «زن، زندگی، آزادی» پتانسیل فراگیری دارد اما هنوز \textbf{ساختار سازمانی} ندارد. \emphred{توصیه:} تشکیل شبکه‌های صنفی-حرفه‌ای (معلمان، پرستاران، کارگران، وکلا) به‌عنوان «همبستگی‌های کوچک» که در لحظهٔ مناسب به «همبستگی بزرگ» تبدیل شوند (\seeChapter{ch:actors}).
\end{keypoint}

\sectiondivider

%═══════════════════════════════════════════════════════════
\section{مذاکرات میزگرد: مدل گذار مذاکره‌ای}
\label{app:poland:roundtable}
%═══════════════════════════════════════════════════════════

\subsection{زمینه‌سازی: چرا رژیم مذاکره کرد}

\begin{enumerate}[nosep]
\item \textbf{بحران اقتصادی شدید:} تورم ۶۰۰٪ + صف‌های طولانی + کمبود کالا
\item \textbf{موج اعتصابات ۱۹۸۸:} کارخانه‌ها، معادن، بنادر
\item \textbf{سیاست گورباچف:} \lr{Perestroika} + \lr{Glasnost} = شوروی دیگر مداخله نمی‌کند
\item \textbf{شکاف درون حزب:} جناح اصلاح‌طلب (ژنرال \person{یاروزلسکی}{\lr{Jaruzelski}} + وزیر کشور \person{کیشچاک}{\lr{Kiszczak}})
\item \textbf{واقع‌بینی:} رژیم فهمید بدون مشروعیت مردمی نمی‌تواند بحران اقتصادی را حل کند
\end{enumerate}

\subsection{ساختار میزگرد}

\begin{table}[htbp]
\centering
\caption{مذاکرات میزگرد لهستان (۶ فوریه — ۵ آوریل ۱۹۸۹)}
\label{tab:app-poland-roundtable}
\begin{tabularx}{\textwidth}{>{\raggedleft\arraybackslash}p{3cm} >{\raggedleft\arraybackslash}X}
\toprule
\headerrow \textbf{بُعد} & \textbf{جزئیات} \\
\midrule
مدت & ۵۹ روز (۶ فوریه — ۵ آوریل ۱۹۸۹) \\
\altrow مکان & کاخ نامیستنیکوفسکی، ورشو \\
شرکت‌کنندگان & ۵۷ نفر (۲۹ حکومتی + ۲۶ اپوزیسیون + ۲ ناظر کلیسا) \\
\altrow سه کارگروه اصلی & ۱) اصلاحات سیاسی، ۲) اقتصاد و سیاست اجتماعی، ۳) تکثرگرایی اتحادیه‌ای \\
تعداد کارگروه‌های فرعی & ۱۱ (آموزش، بهداشت، محیط‌زیست، قضا...) \\
\altrow مشارکت‌کنندگان کل & بیش از ۴۵۰ نفر (در همهٔ سطوح) \\
\bottomrule
\end{tabularx}
\end{table}

\subsection{توافقات میزگرد}

\begin{enumerate}[nosep]
\item \textbf{قانونی‌سازی همبستگی:} اتحادیه دوباره قانونی شد
\item \textbf{انتخابات نیمه‌آزاد:} سنا: ۱۰۰٪ آزاد | مجلس (سِیم): ۶۵٪ رزرو برای حزب، ۳۵٪ آزاد
\item \textbf{ریاست‌جمهوری:} یاروزلسکی رئیس‌جمهور می‌ماند (تضمین امنیتی)
\item \textbf{مجلس سنا:} نهاد جدید کاملاً آزاد (۱۰۰ کرسی)
\item \textbf{اصلاحات اقتصادی:} آزادسازی تدریجی
\item \textbf{اصلاح قضایی:} استقلال نسبی دادگاه‌ها
\end{enumerate}

\begin{lessonlearned}
\textbf{شگفتی انتخابات ژوئن ۱۹۸۹:} رژیم انتظار داشت با ۶۵٪ کرسی تضمین‌شده، قدرت را حفظ کند. اما همبستگی در بخش آزاد (۳۵٪ مجلس + ۱۰۰٪ سنا) \textbf{تقریباً تمام کرسی‌ها} را برد: ۹۹ از ۱۰۰ کرسی سنا + ۱۶۰ از ۱۶۱ کرسی آزاد مجلس. رژیم شوکه شد. سپس احزاب کوچک ائتلافی رژیم منشعب شدند و به همبستگی پیوستند. نتیجه: \person{تادئوش مازوویتسکی}{\lr{Tadeusz Mazowiecki}} اولین نخست‌وزیر غیرکمونیست اروپای شرقی شد (اوت ۱۹۸۹). \emphgreen{درس ایرانی:} حتی انتخابات «مهندسی‌شده» می‌تواند ابزار گذار باشد — اگر اپوزیسیون متحد و بسیج‌شده باشد.
\end{lessonlearned}

\sectiondivider

%═══════════════════════════════════════════════════════════
\section{نقش کلیسای کاتولیک: نهاد میانجی}
\label{app:poland:church}
%═══════════════════════════════════════════════════════════

\subsection{کلیسا به‌عنوان «فضای آزاد»}

\begin{table}[htbp]
\centering
\caption{نقش‌های کلیسای کاتولیک در گذار لهستان و معادل ایرانی}
\label{tab:app-poland-church}
\begin{tabularx}{\textwidth}{>{\raggedleft\arraybackslash}p{3cm} >{\raggedleft\arraybackslash}X >{\raggedleft\arraybackslash}p{3.5cm}}
\toprule
\headerrow \textbf{نقش} & \textbf{توضیح} & \textbf{معادل ایرانی} \\
\midrule
فضای فیزیکی & کلیساها محل گردهمایی و پناهگاه بودند & مساجد مستقل؟ خانه‌ها؟ فضای مجازی \\
\altrow مشروعیت‌زدایی & پاپ ژان‌پل دوم: «نترسید!» (۱۹۷۹) & روحانیون مستقل (مانند منتظری) \\
میانجی‌گری & دو ناظر کلیسا در میزگرد & ایران: نهاد میانجی‌گر مستقل \\
\altrow حمایت مالی & کمک مالی به خانواده‌های زندانیان & دیاسپورا + \lr{NGO}ها \\
شبکهٔ ارتباطی & ۱۰,۰۰۰+ کلیسا در سراسر کشور & شبکه‌های صنفی-حرفه‌ای \\
\altrow مشروعیت اخلاقی & پاپ = اقتدار اخلاقی جهانی & کیست؟ (چالش ایران) \\
\bottomrule
\end{tabularx}
\end{table}

\begin{warningbox}
\textbf{تفاوت حیاتی:} در لهستان، کلیسا \textbf{مخالف} رژیم بود و \textbf{حامی} جنبش. در ایران، نهاد مذهبی \textbf{بخشی از} رژیم است. بنابراین ایران نمی‌تواند روی «کلیسای خود» حساب کند — مگر بخش‌هایی از روحانیت مستقل (مانند مرحوم \person{منتظری}{\lr{Montazeri}} یا روحانیون منتقد). \emphred{جایگزین پیشنهادی:} نقش کلیسا در لهستان را در ایران باید ترکیبی از \textbf{تشکل‌های صنفی}، \textbf{دانشگاه‌ها}، و \textbf{فضای مجازی} ایفا کنند (\seeChapter{ch:actors}).
\end{warningbox}

\sectiondivider

%═══════════════════════════════════════════════════════════
\section{اصلاحات اقتصادی: شوک‌تراپی و پیامدها}
\label{app:poland:economy}
%═══════════════════════════════════════════════════════════

\subsection{برنامهٔ بالتسروویچ (\lr{Shock Therapy})}

دولت مازوویتسکی در ژانویه ۱۹۹۰ برنامهٔ \person{لشک بالتسروویچ}{\lr{Leszek Balcerowicz}} را اجرا کرد:

\begin{table}[htbp]
\centering
\caption{شوک‌تراپی لهستان: دستاوردها و هزینه‌ها}
\label{tab:app-poland-shock}
\begin{tabularx}{\textwidth}{>{\centering\arraybackslash}p{1cm} >{\raggedleft\arraybackslash}X >{\centering\arraybackslash}p{2cm}}
\toprule
\headerrow & \textbf{دستاوردها} & \textbf{زمان} \\
\midrule
\cmark & کاهش تورم از ۶۰۰٪ به ۳۵٪ & ۱ سال \\
\altrow \cmark & آزادسازی قیمت‌ها & فوری \\
\cmark & خصوصی‌سازی ۸,۰۰۰+ شرکت دولتی & ۵-۱۰ سال \\
\altrow \cmark & تبدیل‌پذیری ارز (زلوتی) & ۱ سال \\
\cmark & رشد اقتصادی پایدار (از ۱۹۹۲) & ۲ سال \\
\altrow \cmark & عضویت \lr{WTO} (۱۹۹۵)، \lr{OECD} (۱۹۹۶)، \lr{EU} (۲۰۰۴) & ۵-۱۵ سال \\
\midrule
\headerrow & \textbf{هزینه‌ها} & \textbf{شدت} \\
\midrule
\xmark & بیکاری از ۰٪ (رسمی) به ۱۶٪ & \riskhigh \\
\altrow \xmark & افزایش نابرابری (ضریب جینی: ۰.۲۶ → ۰.۳۵) & \riskmedium \\
\xmark & کاهش \lr{GDP} ۱۱.۶٪ (۱۹۹۰-۱۹۹۱) & \riskhigh \\
\altrow \xmark & فقر موقت (۲۰٪+ زیر خط فقر) & \riskhigh \\
\xmark & نارضایتی اجتماعی و «شکست‌خوردگان» گذار (\lr{Transition Losers}) & \riskmedium \\
\bottomrule
\end{tabularx}
\end{table}

\begin{lessonlearned}
\textbf{درس شوک‌تراپی لهستان برای ایران:} ۱) \textbf{شوک‌تراپی کار می‌کند اما دردناک است} — ایران با جمعیت ۸۵M نمی‌تواند ریسک بیکاری ۱۶٪ را بپذیرد (= ۷ میلیون بیکار جدید)؛ ۲) مدل شیلی (اصلاح تدریجی) برای ایران مناسب‌تر از مدل لهستان است؛ ۳) \textbf{«شکست‌خوردگان گذار»} خطرناک‌اند — آنها در لهستان به پوپولیسم حزب \lr{PiS} رأی دادند. در ایران ممکن است به \textbf{بازگشت اقتدارگرایی} رأی دهند (\seeChapter{ch:risks}).
\end{lessonlearned}

\sectiondivider

%═══════════════════════════════════════════════════════════
\section{لوستراسیون: پاکسازی هوشمند}
\label{app:poland:lustration}
%═══════════════════════════════════════════════════════════

\subsection{مدل لهستان: تأخیری اما مؤثر}

\termfn{لوستراسیون}{Lustration} (بررسی سوابق مقامات رژیم پیشین) در اروپای شرقی با مدل‌های متفاوتی اجرا شد:

\begin{table}[htbp]
\centering
\caption{مقایسهٔ مدل‌های لوستراسیون در اروپای شرقی}
\label{tab:app-poland-lustration}
\begin{tabularx}{\textwidth}{>{\raggedleft\arraybackslash}p{2.5cm} >{\raggedleft\arraybackslash}X >{\centering\arraybackslash}p{2cm} >{\centering\arraybackslash}p{2cm}}
\toprule
\headerrow \textbf{کشور} & \textbf{مدل لوستراسیون} & \textbf{زمان} & \textbf{شدت} \\
\midrule
جمهوری چک & جامع: ممنوعیت ۵ ساله از مناصب دولتی & ۱۹۹۱ & \riskhigh \\
\altrow آلمان شرقی & باز کردن آرشیو اشتازی (\lr{BStU}) + پاکسازی & ۱۹۹۱ & \riskhigh \\
لهستان & تأخیری: قانون ۱۹۹۷ + آرشیو \lr{IPN} (۲۰۰۰) & ۱۹۹۷ & \riskmedium \\
\altrow مجارستان & حداقلی: فقط افشاسازی داوطلبانه & ۱۹۹۴ & \risklow \\
رومانی & بسیار ضعیف: مقامات سابق در قدرت ماندند & تأخیر طولانی & \risklow \\
\bottomrule
\end{tabularx}
\end{table}

\subsection{مؤسسهٔ حافظهٔ ملی (\lr{IPN})}

\org{مؤسسهٔ حافظهٔ ملی}{\lr{Instytut Pamięci Narodowej (IPN)}} در ۱۹۹۸ تأسیس شد:

\begin{itemize}[nosep]
\item \textbf{مأموریت:} نگهداری آرشیو پلیس مخفی (\lr{SB}) + تحقیق + آموزش + تعقیب جنایات
\item \textbf{حجم آرشیو:} ۹۰ کیلومتر طولی اسناد!
\item \textbf{حق دسترسی:} هر شهروند حق دسترسی به پروندهٔ خود را دارد
\item \textbf{افشاسازی:} مقامات عمومی باید همکاری با \lr{SB} را اعتراف کنند
\end{itemize}

\begin{recommendation}
\textbf{مدل \lr{IPN} برای ایران:} تشکیل «مؤسسهٔ حافظهٔ ملی ایران» با مأموریت: ۱) حفظ آرشیو وزارت اطلاعات + سپاه + سازمان زندان‌ها — \textbf{فوری در ۷۲ ساعت اول} (جلوگیری از نابودی اسناد)؛ ۲) حق دسترسی زندانیان سیاسی سابق به پرونده‌هایشان؛ ۳) افشای شبکهٔ خبرچینان (\textbf{با احتیاط} — درس آلمان: برخی افشاسازی‌ها خانواده‌ها را ویران کرد)؛ ۴) تحقیق تاریخی مستقل؛ ۵) آموزش نسل جدید (\seeChapter{ch:requirements}).
\end{recommendation}

\sectiondivider

%═══════════════════════════════════════════════════════════
\section{مشوق عضویت \lr{EU/NATO}: قوی‌ترین ابزار تحکیم}
\label{app:poland:eu}
%═══════════════════════════════════════════════════════════

\subsection{مشروطیت اروپایی (\lr{EU Conditionality})}

مهم‌ترین عامل تحکیم دموکراسی در لهستان و اروپای شرقی، \termfn{مشروطیت اروپایی}{\lr{EU Conditionality}} بود:

\begin{table}[htbp]
\centering
\caption{مراحل ادغام لهستان در نهادهای غربی و تأثیر بر دموکراسی}
\label{tab:app-poland-eu}
\begin{tabularx}{\textwidth}{>{\centering\arraybackslash}p{2cm} >{\raggedleft\arraybackslash}X >{\centering\arraybackslash}p{2.5cm}}
\toprule
\headerrow \textbf{سال} & \textbf{رویداد} & \textbf{تأثیر} \\
\midrule
۱۹۹۱ & شورای اروپا (\lr{Council of Europe}) & معیارهای حقوق بشری \\
\altrow ۱۹۹۱ & برنامهٔ \lr{PHARE} (کمک مالی \lr{EC}) & اصلاحات اقتصادی \\
۱۹۹۳ & \textbf{معیارهای کپنهاگ:} دموکراسی + اقتصاد بازار + ظرفیت اجرایی & چارچوب الزام‌آور \\
\altrow ۱۹۹۴ & درخواست عضویت \lr{EU} & فشار اصلاحات \\
۱۹۹۷ & آغاز مذاکرات الحاق & ۳۱ فصل اصلاحات \\
\altrow ۱۹۹۹ & عضویت \lr{NATO} & تضمین امنیتی \\
۲۰۰۴ & \textbf{عضویت \lr{EU}} & تحکیم نهایی \\
\altrow ۲۰۰۷ & عضویت شنگن & آزادی رفت‌وآمد \\
\bottomrule
\end{tabularx}
\end{table}

\begin{keypoint}
\textbf{چرا مشوق \lr{EU} کار کرد:} ۱) \textbf{ملموس بودن:} مردم می‌دانستند عضویت = ویزای آزاد + کمک مالی + سرمایه‌گذاری؛ ۲) \textbf{الزام‌آور بودن:} معیارهای کپنهاگ شفاف و قابل‌اندازه‌گیری بودند؛ ۳) \textbf{مرحله‌ای بودن:} هر مرحله جایزه‌ای داشت؛ ۴) \textbf{برگشت‌ناپذیری نسبی:} عضویت \lr{EU} بازگشت به اقتدارگرایی را بسیار دشوار (اگرچه نه غیرممکن — مجارستان) کرد. \emphred{چالش ایرانی:} \textbf{معادل \lr{EU} برای ایران وجود ندارد.} هیچ سازمان منطقه‌ای مشابهی نیست. باید \textbf{بستهٔ ترکیبی} طراحی شود (\seeChapter{ch:guarantees}).
\end{keypoint}

\subsection{بستهٔ مشوق ترکیبی پیشنهادی برای ایران}

\begin{table}[htbp]
\centering
\caption{بستهٔ مشوق ترکیبی برای ایران: جایگزین عضویت \lr{EU}}
\label{tab:app-poland-iran-incentives}
\begin{tabularx}{\textwidth}{>{\raggedleft\arraybackslash}p{3cm} >{\raggedleft\arraybackslash}X >{\centering\arraybackslash}p{2cm}}
\toprule
\headerrow \textbf{مشوق} & \textbf{توضیح} & \textbf{فاز اجرا} \\
\midrule
لغو تحریم‌ها (مرحله‌ای) & هر مرحلهٔ اصلاحات = رفع بخشی از تحریم‌ها & فاز ۱-۳ \\
\altrow عضویت \lr{WTO} & ادغام در اقتصاد جهانی & فاز ۲-۳ \\
توافق مشارکت با \lr{EU} & مشابه توافق‌های شمال آفریقا & فاز ۲-۴ \\
\altrow سرمایه‌گذاری مستقیم خارجی & کنفرانس کمک‌دهندگان + صندوق بازسازی & فاز ۱-۲ \\
عضویت در شورای حقوق بشر & بازگشت به جامعهٔ بین‌المللی & فاز ۳ \\
\altrow برنامهٔ بورس تحصیلی گسترده & ۱۰,۰۰۰+ بورس/سال برای دانشجویان ایرانی & فاز ۱-۴ \\
تضمین امنیتی منطقه‌ای & پیمان عدم تجاوز + تضمین مرزها & فاز ۲-۳ \\
\altrow حل مسئلهٔ هسته‌ای & الحاق به \lr{NPT} + پروتکل الحاقی = لغو تحریم & فاز ۱-۲ \\
\bottomrule
\end{tabularx}
\end{table}

\sectiondivider

%═══════════════════════════════════════════════════════════
\section{هشدار: بازگشت اقتدارگرایی (حزب \lr{PiS})}
\label{app:poland:backsliding}
%═══════════════════════════════════════════════════════════

\subsection{لهستان ۲۰۱۵-۲۰۲۳: تضعیف دموکراسی از درون \lr{EU}}

حتی با عضویت \lr{EU}، لهستان تحت حزب \org{قانون و عدالت}{\lr{Prawo i Sprawiedliwość (PiS)}} دچار بازگشت اقتدارگرایی شد:

\begin{table}[htbp]
\centering
\caption{اقدامات ضددموکراتیک حزب \lr{PiS} (۲۰۱۵-۲۰۲۳)}
\label{tab:app-poland-pis}
\begin{tabularx}{\textwidth}{>{\centering\arraybackslash}p{2cm} >{\raggedleft\arraybackslash}X >{\centering\arraybackslash}p{2cm}}
\toprule
\headerrow \textbf{سال} & \textbf{اقدام} & \textbf{وخامت} \\
\midrule
۲۰۱۵ & فلج دادگاه قانون اساسی (انتصاب قاضیان حزبی) & \statusbad \\
\altrow ۲۰۱۶ & کنترل رسانهٔ عمومی (\lr{TVP}) & \statusbad \\
۲۰۱۷ & تلاش برای تسلط بر قوهٔ قضاییه (اعتراض \lr{EU}) & \statusbad \\
\altrow ۲۰۱۸ & محدودیت آزادی تجمع + تاریخ‌نگاری (\lr{IPN}) & \statuswarn \\
۲۰۲۰ & ممنوعیت شبه‌کامل سقط‌جنین (بحران اجتماعی) & \statuswarn \\
\altrow ۲۰۲۱ & ابزار جاسوسی \lr{Pegasus} علیه اپوزیسیون & \statusbad \\
اکتبر ۲۰۲۳ & \textbf{شکست \lr{PiS} در انتخابات}: بازگشت لیبرال‌ها (\person{تاسک}{\lr{Tusk}}) & \statusok \\
\bottomrule
\end{tabularx}
\end{table}

\begin{warningbox}
\textbf{درس بازگشت لهستانی:} حتی ۳۰ سال پس از گذار و ۲۰ سال پس از عضویت \lr{EU}، دموکراسی آسیب‌پذیر است. عوامل بازگشت: ۱) \textbf{«شکست‌خوردگان گذار»:} مناطق فقیر شرقی لهستان به \lr{PiS} رأی دادند؛ ۲) \textbf{نابرابری منطقه‌ای}؛ ۳) \textbf{پوپولیسم ناسیونالیستی-مذهبی}؛ ۴) \textbf{ابزار سازی از ارزش‌های سنتی} (ضد مهاجرت، ضد \lr{LGBT}). نکتهٔ مثبت: دموکراسی لهستان \textbf{اصلاح‌پذیر} بود — مردم در ۲۰۲۳ \lr{PiS} را کنار زدند. \emphred{درس ایرانی:} تحکیم دموکراسی فرآیندی بی‌پایان است و نیازمند نهادهای مستقل، رسانهٔ آزاد و جامعهٔ مدنی هوشیار.
\end{warningbox}

\sectiondivider

%═══════════════════════════════════════════════════════════
\section{ماتریس درس‌آموخته‌ها برای ایران}
\label{app:poland:lessons}
%═══════════════════════════════════════════════════════════

\begin{table}[htbp]
\centering
\caption{ماتریس انتقال درس‌آموخته‌های لهستان و اروپای شرقی به ایران}
\label{tab:app-poland-lessons}
\begin{tabularx}{\textwidth}{
  >{\raggedleft\arraybackslash}p{2.2cm}
  >{\raggedleft\arraybackslash}p{3.5cm}
  >{\raggedleft\arraybackslash}X
  >{\centering\arraybackslash}p{1.5cm}
}
\toprule
\headerrow \textbf{بُعد} & \textbf{درس لهستان} & \textbf{کاربرد ایرانی} & \textbf{انتقال‌پذیری} \\
\midrule
جنبش اجتماعی & همبستگی ۱۰M: سازمان‌یافته & شبکه‌های صنفی → ائتلاف فراگیر & \rating{4} \\
\altrow
مذاکرات میزگرد & ۵۹ روز، ۴۵۰+ شرکت‌کننده & «میزگرد ایران» در فاز ۱ & \rating{5} \\
نقش نهاد مذهبی & کلیسا = میانجی + فضای آزاد & روحانیون مستقل + تشکل‌های صنفی & \rating{3} \\
\altrow
لوستراسیون & \lr{IPN} + آرشیو + افشاسازی & مؤسسهٔ حافظهٔ ملی ایران & \rating{5} \\
شوک‌تراپی & مؤثر اما دردناک & اصلاح تدریجی (مدل شیلی) مناسب‌تر & \rating{2} \\
\altrow
مشوق \lr{EU/NATO} & قوی‌ترین ابزار تحکیم & بستهٔ ترکیبی (تحریم + \lr{WTO} + سرمایه‌گذاری) & \rating{3} \\
انتخابات نیمه‌آزاد & حتی مهندسی‌شده ابزار گذار شد & رفراندوم / انتخابات تحت نظارت & \rating{4} \\
\altrow
بازگشت (\lr{PiS}) & ۳۰ سال بعد هم آسیب‌پذیر & نهادسازی مستمر + رسانهٔ آزاد & \rating{4} \\
اثر دومینو & لهستان ← مجارستان ← چک ← آلمان شرقی & ایران ← منطقه؟ & \rating{3} \\
\midrule
\headerrow \multicolumn{3}{l}{\textbf{میانگین انتقال‌پذیری}} & \textbf{\rating{4}} \\
\bottomrule
\end{tabularx}
\end{table}

\sectiondivider

%═══════════════════════════════════════════════════════════
\section{مقایسهٔ مختصر: سایر کشورهای اروپای شرقی}
\label{app:poland:others}
%═══════════════════════════════════════════════════════════

\begin{table}[htbp]
\centering
\caption{مقایسهٔ مدل‌های گذار در اروپای شرقی}
\label{tab:app-poland-eastern-europe}
\begin{tabularx}{\textwidth}{
  >{\raggedleft\arraybackslash}p{2.2cm}
  >{\raggedleft\arraybackslash}p{2.5cm}
  >{\raggedleft\arraybackslash}p{2.5cm}
  >{\raggedleft\arraybackslash}X
  >{\centering\arraybackslash}p{1.5cm}
}
\toprule
\headerrow \textbf{کشور} & \textbf{نوع گذار} & \textbf{ویژگی خاص} & \textbf{درس برای ایران} & \textbf{نتیجه} \\
\midrule
لهستان & میزگرد مذاکره‌ای & همبستگی + کلیسا & مذاکره + جنبش & \statusok \\
\altrow مجارستان & اصلاح از بالا + میزگرد & تحول تدریجی + حزب حاکم منشعب شد & شکاف نخبگان & \statuswarn \\
چکسلواکی & انقلاب مخملی (۱۰ روز!) & سرعت + خشونت‌پرهیزی + هاول & رهبر اخلاقی & \statusok \\
\altrow آلمان شرقی & فروپاشی + ادغام & \lr{Stasi} = آرشیو + اتحاد با غرب & آرشیو + حمایت مالی & \statusok \\
رومانی & انقلاب خشونت‌آمیز & چائوشسکو اعدام شد + نخبگان قدیم ماندند & خشونت = عدم پاکسازی & \statuswarn \\
\altrow بلغارستان & اصلاح تدریجی & حزب کمونیست خود را اصلاح کرد & اصلاح‌طلبان درون نظام & \statuswarn \\
\bottomrule
\end{tabularx}
\end{table}

\begin{lessonlearned}
\textbf{اثر دومینو (\lr{Domino Effect}):} سقوط لهستان در ۱۹۸۹ طی ۶ ماه همهٔ اروپای شرقی را فرو ریخت: لهستان (ژوئن) ← مجارستان (اکتبر) ← آلمان شرقی (نوامبر: دیوار برلین) ← چکسلواکی (نوامبر) ← رومانی (دسامبر). \emphgreen{سؤال ایرانی:} آیا گذار ایران اثر دومینو در خاورمیانه خواهد داشت؟ تجربهٔ بهار عربی (۲۰۱۱) نشان داد که بله، اما نتایج لزوماً مثبت نیست (لیبی، سوریه، یمن). مدل ۶ باید اثرات منطقه‌ای را نیز مدیریت کند (\seeChapter{ch:risks}).
\end{lessonlearned}

\sectiondivider

%═══════════════════════════════════════════════════════════
\section{جمع‌بندی پیوست}
\label{app:poland:conclusion}
%═══════════════════════════════════════════════════════════

\begin{chaptersummary}
جمع‌بندی پیوست ث — لهستان و اروپای شرقی:

\begin{enumerate}[nosep]
\item \textbf{جنبش همبستگی} الگوی بی‌نظیر جنبش اجتماعی غیرخشونت‌آمیز است: ۱۰M عضو، ساختار سازمانی، رهبر کاریزماتیک. اپوزیسیون ایران هنوز فاصلهٔ زیادی با این مدل دارد.
\item \textbf{مذاکرات میزگرد} نشان داد که حتی رژیم‌های ایدئولوژیک در شرایط بحران اقتصادی حاضر به مذاکره می‌شوند — مشروط به فشار مردمی + شکاف نخبگان.
\item \textbf{نقش کلیسا} در لهستان بی‌نظیر بود و \textbf{مستقیماً} به ایران قابل‌انتقال نیست — جایگزین: تشکل‌های صنفی + دانشگاه‌ها + فضای مجازی.
\item \textbf{لوستراسیون و \lr{IPN}} بهترین مدل حفظ آرشیو و حافظهٔ جمعی است — «مؤسسهٔ حافظهٔ ملی ایران» باید در ۷۲ ساعت اول تأسیس شود.
\item \textbf{شوک‌تراپی} برای ایران مناسب نیست — مدل تدریجی شیلی بهتر است.
\item \textbf{مشوق عضویت \lr{EU/NATO}} قوی‌ترین ابزار تحکیم بود — ایران فاقد معادل است و باید بستهٔ ترکیبی طراحی کند.
\item حتی با عضویت \lr{EU}، \textbf{بازگشت اقتدارگرایی ممکن است} (\lr{PiS} ۲۰۱۵-۲۰۲۳) — تحکیم فرآیندی بی‌پایان است.
\item \textbf{اثر دومینو} لهستان الهام‌بخش بود اما بهار عربی نشان داد دومینو لزوماً مثبت نیست.
\end{enumerate}

\vspace{0.3cm}
\textit{مطالعهٔ تکمیلی:}
\begin{itemize}[nosep]
\item مقایسهٔ جامع ۹ نمونه: \seeChapter{app:comparison}
\item آفریقای جنوبی (مدل اصلی): \seeChapter{app:south-africa}
\item تونس (گفت‌وگوی ملی): \seeChapter{app:tunisia}
\item نهادها و بازیگران: \seeChapter{ch:actors}
\item عراق (ضد الگو): \seeChapter{app:iraq}
\end{itemize}
\end{chaptersummary}

\chapterend

%══════════════════════════════════════════════════════════════
% پایان پیوست ث
%══════════════════════════════════════════════════════════════