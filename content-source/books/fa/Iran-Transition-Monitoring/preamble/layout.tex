% ╔══════════════════════════════════════════════════════════════════╗
% ║  تنظیمات صفحه‌آرایی — نسخه ۱.۱                                  ║
% ║  ⚠️ این فایل باید پس از xepersian بارگذاری شود                  ║
% ╚══════════════════════════════════════════════════════════════════╝

% ============================================================
% سرصفحه و پاصفحه — تنظیم برای RTL
% ============================================================

\pagestyle{fancy}
\fancyhf{}

% ✅ تنظیمات RTL: در فارسی، صفحات فرد سمت چپ و زوج سمت راست هستند

% صفحات فرد (چپ در RTL)
\fancyhead[LO]{%
	\textcolor{MediumGray}{\small\rightmark}%
}
\fancyhead[RO]{%
	\textcolor{MainPurple}{\small\thepage}%
}

% صفحات زوج (راست در RTL)
\fancyhead[RE]{%
	\textcolor{MediumGray}{\small\leftmark}%
}
\fancyhead[LE]{%
	\textcolor{MainPurple}{\small\thepage}%
}

% خط سرصفحه
\renewcommand{\headrulewidth}{0.4pt}
\renewcommand{\headrule}{%
	{\color{MainPurple!30}\hrule width\headwidth height\headrulewidth}%
}

% پاصفحه
\fancyfoot[C]{%
	\vspace{2pt}%
	{\textcolor{LightGray}{\rule{3cm}{0.3pt}}}\\[-3pt]%
	{\textcolor{MediumGray}{\footnotesize نظارت بین‌المللی بر گذار دموکراتیک ایران}}%
}

% صفحات آغاز فصل
\fancypagestyle{plain}{%
	\fancyhf{}%
	\fancyfoot[C]{\textcolor{MainPurple}{\thepage}}%
	\renewcommand{\headrulewidth}{0pt}%
}

% صفحات خالی
\fancypagestyle{empty}{%
	\fancyhf{}%
	\renewcommand{\headrulewidth}{0pt}%
	\renewcommand{\footrulewidth}{0pt}%
}

% ============================================================
% تنظیم mark برای فصل و بخش
% ============================================================
\renewcommand{\chaptermark}[1]{%
	\markboth{فصل \thechapter: #1}{}%
}
\renewcommand{\sectionmark}[1]{%
	\markright{#1}%
}

% ============================================================
% عنوان‌بندی فصول — سازگار با xepersian
% ============================================================

% --- فصل ---
\titleformat{\chapter}[display]
{\normalfont\huge\bfseries}
{\textcolor{MainPurple!30}{\fontsize{60}{60}\selectfont\thechapter}}
{20pt}
{\color{MainPurple}\Huge}
[\vspace{-10pt}\textcolor{MainPurple!20}{\rule{\textwidth}{2pt}}]

\titlespacing*{\chapter}
{0pt}      % چپ
{20pt}     % قبل
{30pt}     % بعد

% --- بخش ---
\titleformat{\section}
{\normalfont\Large\bfseries\color{MainBlue}}
{\thesection}
{1em}
{}
[\vspace{-6pt}\textcolor{MainBlue!30}{\rule{\textwidth}{0.5pt}}]

\titlespacing*{\section}
{0pt}{20pt}{12pt}

% --- زیربخش ---
\titleformat{\subsection}
{\normalfont\large\bfseries\color{MainGreen!80!black}}
{\thesubsection}
{1em}
{}

\titlespacing*{\subsection}
{0pt}{16pt}{8pt}

% --- زیرزیربخش ---
\titleformat{\subsubsection}
{\normalfont\normalsize\bfseries\color{MainOrange!80!black}}
{\thesubsubsection}
{1em}
{}

\titlespacing*{\subsubsection}
{0pt}{12pt}{6pt}

% --- پاراگراف ---
\titleformat{\paragraph}[runin]
{\normalfont\normalsize\bfseries\color{DarkGray}}
{\theparagraph}
{1em}
{}
[.]

\titlespacing*{\paragraph}
{0pt}{10pt}{8pt}

% ============================================================
% فهرست مطالب
% ============================================================

\setcounter{secnumdepth}{3}
\setcounter{tocdepth}{2}

% ============================================================
% Epigraph
% ============================================================
\setlength{\epigraphwidth}{0.65\textwidth}
\setlength{\epigraphrule}{0pt}
\renewcommand{\epigraphflush}{center}
\renewcommand{\sourceflush}{center}

% ============================================================
% پاورقی — شماره‌گذاری در هر صفحه
% تنظیمات شکل و جدول
% ============================================================

% شماره‌گذاری پیوسته (نه بر اساس فصل)
% \counterwithout{figure}{chapter}
% \counterwithout{table}{chapter}

% یا شماره‌گذاری بر اساس فصل (پیش‌فرض)
% اگر می‌خواهید ۱.۱، ۱.۲ و... باشد، همین‌طور بگذارید

% ============================================================
% فاصله‌ی شناورها
% ============================================================
\setlength{\textfloatsep}{20pt plus 4pt minus 4pt}
\setlength{\floatsep}{16pt plus 4pt minus 4pt}
\setlength{\intextsep}{16pt plus 4pt minus 4pt}

% ============================================================
% تنظیمات خاص برای PDF
% ============================================================

% عنوان سند در PDF viewer
\hypersetup{
	pdftitle={نظارت بین‌المللی بر گذار دموکراتیک ایران},
	pdfauthor={مهدی سالم},
	pdfsubject={گذار دموکراتیک},
	pdfkeywords={ایران، دموکراسی، گذار، نظارت}
}