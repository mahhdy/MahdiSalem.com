%══════════════════════════════════════════════════════════════
% پیوست چ: مطالعه موردی میانمار — گذار ناتمام
% فایل: appendices/app-g-myanmar.tex
% حجم هدف: ۶-۸ صفحه
%══════════════════════════════════════════════════════════════

\chapter{مطالعهٔ موردی: میانمار — گذار ناتمام (۲۰۱۰-۲۰۲۱)}
\label{app:myanmar}

\begin{executivesummary}
میانمار (برمه) نمونهٔ دردناک \textbf{گذار ناتمام} است: دهه‌ای از گشایش تدریجی (۲۰۱۰-۲۰۲۰) با کودتای نظامی ۱ فوریه ۲۰۲۱ به پایان رسید و کشور به جنگ داخلی فرو رفت. ویژگی‌های کلیدی: ۱) \textbf{گشایش کنترل‌شده از بالا} توسط همان ژنرال‌ها (بدون فشار واقعی مردمی)، ۲) \textbf{قانون اساسی ۲۰۰۸ نظامی‌نوشت} (۲۵٪ کرسی‌ها + وتوی عملی + فرماندهی مستقل)، ۳) \textbf{فقدان اصلاح واقعی بخش امنیتی} (\lr{SSR}) — \termfn{تاتمادو}{\lr{Tatmadaw}} دست‌نخورده ماند، ۴) \textbf{نسل‌کشی روهینگیا} (۲۰۱۷) حتی در دورهٔ «دموکراسی»، و ۵) \textbf{کودتای ۲۰۲۱} و بازگشت کامل به حکومت نظامی. میانمار \textbf{قوی‌ترین شاهد} بر این اصل است: بدون اصلاح واقعی بخش امنیتی، هر گشایشی موقتی و شکننده است.
\end{executivesummary}

%═══════════════════════════════════════════════════════════
\section{زمینه و بافت تاریخی}
\label{app:myanmar:context}
%═══════════════════════════════════════════════════════════

\subsection{تاتمادو: ارتشی که دولت را بلعید}

\org{تاتمادو}{\lr{Tatmadaw}} (نیروهای مسلح میانمار) از ۱۹۶۲ تا ۲۰۱۱ به‌طور مستقیم حکومت کرد — طولانی‌ترین حکومت نظامی مستقیم در آسیا:

\begin{table}[htbp]
\centering
\caption{مشخصات میانمار در آستانهٔ گشایش (۲۰۱۰)}
\label{tab:app-myanmar-profile}
\begin{tabularx}{\textwidth}{>{\raggedleft\arraybackslash}p{4.5cm} >{\raggedleft\arraybackslash}X}
\toprule
\headerrow \textbf{شاخص} & \textbf{مقدار} \\
\midrule
جمعیت & ۵۲ میلیون نفر \\
\altrow مساحت & ۶۷۶,۰۰۰ \lr{km²} \\
تنوع قومی & بسیار بالا (۱۳۵ گروه قومی رسمی) \\
\altrow \lr{GDP per capita} & $\sim$\$۱,۲۰۰ \\
طول عمر حکومت نظامی & ۴۹ سال (۱۹۶۲-۲۰۱۱) \\
\altrow اندازهٔ تاتمادو & $\sim$۴۵۰,۰۰۰ نفر \\
منافع اقتصادی ارتش & \lr{MEHL + MEC}: امپراتوری عظیم (یشم، بانک، مخابرات) \\
\altrow جنگ‌های داخلی فعال & ۱۵+ گروه مسلح قومی \\
نقش بودایی‌گری & ملی‌گرایی بودایی (ضد مسلمان) \\
\altrow تحریم‌های بین‌المللی & هدفمند (\lr{US + EU}) \\
\bottomrule
\end{tabularx}
\end{table}

\subsection{مقایسهٔ ساختاری تاتمادو و سپاه پاسداران}

\begin{table}[htbp]
\centering
\caption{مقایسهٔ تفصیلی تاتمادو و سپاه پاسداران}
\label{tab:app-myanmar-irgc}
\begin{tabularx}{\textwidth}{>{\raggedleft\arraybackslash}p{3.5cm} >{\raggedleft\arraybackslash}X >{\raggedleft\arraybackslash}X}
\toprule
\headerrow \textbf{بُعد} & \textbf{تاتمادو (میانمار)} & \textbf{سپاه پاسداران (ایران)} \\
\midrule
تعداد نیرو & $\sim$۴۵۰,۰۰۰ & $\sim$۱۹۰,۰۰۰ (+ ۶۰۰K بسیج) \\
\altrow منافع اقتصادی & \lr{MEHL + MEC}: یشم (\$۳۱B/سال)، بانک، هتل & خاتم‌الانبیاء، بنیادها، قاچاق \\
کنترل سیاسی & ۲۵٪ کرسی‌های پارلمان (بدون انتخاب) & شورای نگهبان + خبرگان + نفوذ \\
\altrow فرماندهی مستقل & کاملاً مستقل از دولت غیرنظامی & تحت فرمان رهبری (نه رئیس‌جمهور) \\
وتوی قانون اساسی & ۲۵٪+ ۱ = وتوی اصلاحیه & شورای نگهبان = وتوی قانون \\
\altrow ایدئولوژی & ناسیونالیسم بامار + بودایی‌گری & اسلام سیاسی + ضد استکبار \\
جنایات & نسل‌کشی روهینگیا (۲۰۱۷) & سرکوب اعتراضات + اعدام‌ها \\
\altrow حاضر به اصلاح؟ & خیر (کودتای ۲۰۲۱) & نامشخص \\
\bottomrule
\end{tabularx}
\end{table}

\begin{casestudy}
\textbf{تاتمادو آینهٔ سپاه:} شباهت ساختاری تاتمادو و سپاه بسیار بالاست — هر دو ارتش‌هایی هستند با منافع اقتصادی عظیم، فرماندهی مستقل، وتوی قانون اساسی، و ایدئولوژی ملی‌گرایانه. تفاوت‌ها: ۱) تاتمادو ۴۹ سال مستقیماً حکومت کرد (سپاه غیرمستقیم)؛ ۲) میانمار ۱۵+ جنگ داخلی قومی دارد (ایران ندارد)؛ ۳) تاتمادو نسل‌کشی آشکار انجام داد. نکتهٔ حیاتی: تاتمادو \textbf{گشایش} داد اما \textbf{اصلاح نشد} — و وقتی نتیجهٔ انتخابات را نپسندید، کودتا کرد. \emphred{هشدار ایرانی:} اگر سپاه بدون اصلاح واقعی باقی بماند، «کودتای ۲۰۲۱ میانمار» در ایران تکرار خواهد شد.
\end{casestudy}

\sectiondivider

%═══════════════════════════════════════════════════════════
\section{گشایش کنترل‌شده: «دموکراسی منضبط» (۲۰۱۰-۲۰۲۰)}
\label{app:myanmar:opening}
%═══════════════════════════════════════════════════════════

\subsection{قانون اساسی ۲۰۰۸: قفل‌های نهادی}

ژنرال‌ها قبل از گشایش، قانون اساسی ۲۰۰۸ را طراحی کردند که شامل \textbf{قفل‌های نهادی} بی‌سابقه بود:

\begin{table}[htbp]
\centering
\caption{قفل‌های نهادی قانون اساسی ۲۰۰۸ میانمار و مقایسه با پینوشه}
\label{tab:app-myanmar-locks}
\begin{tabularx}{\textwidth}{>{\raggedleft\arraybackslash}p{3.5cm} >{\raggedleft\arraybackslash}X >{\centering\arraybackslash}p{2cm}}
\toprule
\headerrow \textbf{قفل} & \textbf{توضیح} & \textbf{قفل پینوشه؟} \\
\midrule
۲۵٪ کرسی‌های پارلمان & نظامیان منصوب (بدون انتخاب) در هر دو مجلس & مشابه (سناتورهای منصوب) \\
\altrow وتوی اصلاحیه & اصلاح قانون اساسی نیاز به ۷۵٪+ دارد → ۲۵٪ نظامی = وتو & مشابه \\
فرماندهی مستقل & فرمانده کل ارتش خودش را منصوب می‌کند (نه رئیس‌جمهور) & مشابه \\
\altrow سه وزارت کلیدی & دفاع + داخله + مرزها فقط نظامی & \textbf{فراتر از پینوشه} \\
مصونیت & هیچ اقدام نظامی گذشته قابل تعقیب نیست & مشابه (قانون عفو ۱۹۷۸) \\
\altrow حکومت نظامی & فرمانده کل حق اعلام حکومت نظامی و قبض قدرت دارد & -- \\
ممنوعیت سوچی & شخصی که همسرش خارجی باشد نمی‌تواند رئیس‌جمهور شود (مادهٔ ۵۹-f: علیه سوچی) & -- \\
\bottomrule
\end{tabularx}
\end{table}

\begin{warningbox}
\textbf{قانون اساسی ۲۰۰۸ میانمار = مدل «چگونه قدرت را حفظ کنیم».} ژنرال‌ها دموکراسی ظاهری ایجاد کردند اما هر ابزاری برای بازگشت به قدرت نگه داشتند. این \textbf{هشدار مستقیم} برای ایران است: در هر سناریوی مذاکره‌ای، باید مراقب قانون اساسی «نظامی‌نوشت» بود. مجلس مؤسسان ایران باید واقعاً مستقل و فراگیر باشد — نه صوری (\seeChapter{ch:guarantees}).
\end{warningbox}

\subsection{گاه‌شمار گشایش و بازگشت}

\begin{table}[htbp]
\centering
\caption{گاه‌شمار گذار ناتمام میانمار (۲۰۱۰-۲۰۲۱)}
\label{tab:app-myanmar-timeline}
\begin{tabularx}{\textwidth}{>{\centering\arraybackslash}p{2.5cm} >{\raggedleft\arraybackslash}X >{\centering\arraybackslash}p{2cm}}
\toprule
\headerrow \textbf{تاریخ} & \textbf{رویداد} & \textbf{ارزیابی} \\
\midrule
نوامبر ۲۰۱۰ & انتخابات فرمایشی (حزب نظامی \lr{USDP} برنده) & \statusbad \\
\altrow مارس ۲۰۱۱ & ریاست‌جمهوری \person{تئین‌سئین}{\lr{Thein Sein}} (ژنرال بازنشسته) & \statuswarn \\
۲۰۱۱-۲۰۱۲ & آزادی زندانیان سیاسی + آزادی نسبی مطبوعات & \statusok \\
\altrow آوریل ۲۰۱۲ & \lr{NLD} در انتخابات میان‌دوره‌ای شرکت + پیروز شد & \statusok \\
۲۰۱۳-۲۰۱۴ & رشد اقتصادی + سرمایه‌گذاری خارجی + لغو تحریم‌ها & \statusok \\
\altrow نوامبر ۲۰۱۵ & \textbf{انتخابات آزاد: \lr{NLD} پیروز (۸۰٪+ کرسی‌های آزاد)} & \statusok \\
مارس ۲۰۱۶ & دولت سوچی (اما رئیس‌جمهور نشد: مادهٔ ۵۹-f) & \statuswarn \\
\altrow اوت ۲۰۱۷ & \textbf{نسل‌کشی روهینگیا:} ۷۰۰,۰۰۰ آواره → بنگلادش & \statusbad \\
۲۰۱۸-۲۰۱۹ & بحران بین‌المللی + دادگاه \lr{ICJ} & \statusbad \\
\altrow نوامبر ۲۰۲۰ & انتخابات: \lr{NLD} دوباره پیروز (بیشتر از ۲۰۱۵) & \statusok \\
\textbf{۱ فوریه ۲۰۲۱} & \textbf{کودتای تاتمادو:} بازداشت سوچی + حکومت نظامی & \statusbad \\
\altrow ۲۰۲۱-۲۰۲۴ & جنگ داخلی گسترده + \lr{NUG} (دولت وحدت ملی) + مقاومت مسلحانه & \statusbad \\
\bottomrule
\end{tabularx}
\end{table}

\sectiondivider

%═══════════════════════════════════════════════════════════
\section{چرا گذار شکست خورد: پنج علت ساختاری}
\label{app:myanmar:failure}
%═══════════════════════════════════════════════════════════

\subsection{تحلیل علّی}

\begin{table}[htbp]
\centering
\caption{پنج علت ساختاری شکست گذار میانمار}
\label{tab:app-myanmar-causes}
\begin{tabularx}{\textwidth}{
  >{\centering\arraybackslash}p{0.7cm}
  >{\raggedleft\arraybackslash}p{3cm}
  >{\raggedleft\arraybackslash}X
  >{\centering\arraybackslash}p{1.5cm}
}
\toprule
\headerrow \textbf{\#} & \textbf{علت} & \textbf{توضیح} & \textbf{ارتباط ایران} \\
\midrule
۱ & \textbf{فقدان \lr{SSR} واقعی} & تاتمادو دست‌نخورده ماند: فرماندهی مستقل + بودجهٔ مستقل + اقتصاد مستقل + ۲۵٪ پارلمان & \rating{5} \\
\altrow
۲ & \textbf{قانون اساسی نظامی‌نوشت} & قفل‌های نهادی غیرقابل‌رفع (۷۵٪ لازم + ۲۵٪ وتوی نظامی) & \rating{5} \\
۳ & \textbf{سوچی سازش بیش‌ازحد کرد} & \lr{NLD} تاتمادو را به چالش نکشید + نسل‌کشی روهینگیا را انکار کرد & \rating{4} \\
\altrow
۴ & \textbf{فقدان فشار بین‌المللی مؤثر} & چین و روسیه حامی تاتمادو + \lr{ASEAN} ضعیف + تحریم‌ها ناکافی & \rating{3} \\
۵ & \textbf{جنگ‌های قومی حل‌نشده} & ۱۵+ گروه مسلح قومی + بدون توافق صلح جامع & \rating{3} \\
\bottomrule
\end{tabularx}
\end{table}

\begin{keypoint}
\textbf{علت اول تعیین‌کننده است:} بدون اصلاح واقعی بخش امنیتی، \textbf{هر گشایشی صوری و برگشت‌پذیر} است. تاتمادو هرگز تحت نظارت مدنی واقعی قرار نگرفت: فرمانده‌اش را خودش انتخاب می‌کرد، بودجه‌اش شفاف نبود، امپراتوری اقتصادی‌اش (\lr{MEHL}: بزرگ‌ترین شرکت هلدینگ میانمار) بی‌حساب‌وکتاب بود. وقتی نتیجهٔ انتخابات ۲۰۲۰ را نپسندید، \textbf{مادهٔ ۴۱۷ قانون اساسی خودنوشته} را فعال و کودتا کرد.
\end{keypoint}

\subsection{نقد سوچی: سازش بیش‌ازحد}

\person{آنگ سان سوچی}{\lr{Aung San Suu Kyi}} — نوبل صلح ۱۹۹۱ — تصمیمات بحث‌برانگیزی گرفت:

\begin{enumerate}[nosep]
\item \textbf{پذیرش قانون اساسی ۲۰۰۸:} با وجود قفل‌های نهادی، \lr{NLD} وارد بازی شد — بدون تلاش جدی برای اصلاح
\item \textbf{عدم مقابله با ارتش:} هیچ تلاشی برای نظارت مدنی بر تاتمادو نکرد
\item \textbf{دفاع از نسل‌کشی:} در \lr{ICJ} شخصاً از ارتش دفاع کرد (۲۰۱۹) — مشروعیت بین‌المللی‌اش را از دست داد
\item \textbf{سکوت دربارهٔ جنایات:} هرگز از حقوق روهینگیا دفاع نکرد
\item \textbf{نتیجه:} تاتمادو هم از سازش بهره برد و هم وقتی خواست، سوچی را بازداشت کرد
\end{enumerate}

\begin{lessonlearned}
\textbf{دام «سازش بدون اصلاح»:} سوچی گمان کرد با \textbf{سازش تاکتیکی} با ارتش، به‌تدریج فضا باز می‌شود. اما ارتش سازش را نشانهٔ \textbf{ضعف} تلقی کرد نه حسن‌نیت. \emphred{درس ایرانی:} مذاکره با سپاه ضروری است (مدل آفریقای جنوبی)، اما مذاکره باید \textbf{از موضع قدرت} باشد و نتیجهٔ آن \textbf{اصلاح ساختاری واقعی} (ادغام + تفکیک اقتصادی + نظارت مدنی). سازش بدون اصلاح = تأخیر در فاجعه، نه جلوگیری از آن (\seeChapter{ch:guarantees}).
\end{lessonlearned}

\sectiondivider

%═══════════════════════════════════════════════════════════
\section{نسل‌کشی روهینگیا: دموکراسی بدون حقوق بشر}
\label{app:myanmar:rohingya}
%═══════════════════════════════════════════════════════════

\begin{warningbox}
\textbf{نسل‌کشی روهینگیا (اوت ۲۰۱۷)} حتی در دورهٔ «دموکراتیک» میانمار رخ داد:

\begin{itemize}[nosep]
\item \textbf{عملیات نظامی:} تاتمادو عملیات «پاکسازی» در ایالت راخین انجام داد
\item \textbf{تلفات:} حداقل ۱۰,۰۰۰ کشته + ۷۰۰,۰۰۰ آواره به بنگلادش
\item \textbf{روش‌ها:} سوزاندن روستاها، تجاوز جنسی سیستماتیک، کشتار غیرنظامیان
\item \textbf{واکنش سوچی:} سکوت + دفاع از ارتش در \lr{ICJ}
\item \textbf{تعریف حقوقی:} مأموریت تحقیق سازمان ملل: «نشانه‌های نسل‌کشی» (\lr{Genocide}) + جنایت علیه بشریت
\item \textbf{واکنش بین‌المللی:} ضعیف (چین و روسیه وتو در شورای امنیت)
\end{itemize}

\textbf{درس حیاتی:} دموکراسی بدون حقوق بشر، دموکراسی نیست. انتخابات آزاد + پارلمان + رئیس‌جمهور غیرنظامی — هیچ‌کدام مانع نسل‌کشی نشد. \emphred{هشدار ایرانی:} قانون اساسی جدید ایران باید \textbf{منشور حقوق} غیرقابل‌نقض داشته باشد + دادگاه قانون اساسی مستقل + پذیرش صلاحیت \lr{ICC} + حقوق اقلیت‌ها (بهائیان، سنی‌ها، کردها، بلوچ‌ها) بدون استثنا.
\end{warningbox}

\sectiondivider

%═══════════════════════════════════════════════════════════
\section{کودتای ۲۰۲۱: بازگشت کامل}
\label{app:myanmar:coup}
%═══════════════════════════════════════════════════════════

\subsection{چگونه کودتا شد}

\begin{table}[htbp]
\centering
\caption{عوامل کودتای ۱ فوریه ۲۰۲۱ میانمار}
\label{tab:app-myanmar-coup}
\begin{tabularx}{\textwidth}{>{\raggedleft\arraybackslash}p{3.5cm} >{\raggedleft\arraybackslash}X}
\toprule
\headerrow \textbf{عامل} & \textbf{توضیح} \\
\midrule
محرک آنی & شکست حزب نظامی \lr{USDP} در انتخابات نوامبر ۲۰۲۰ + ادعای تقلب \\
\altrow عامل ساختاری ۱ & ارتش هرگز تحت نظارت مدنی قرار نگرفت \\
عامل ساختاری ۲ & قانون اساسی ۲۰۰۸ ابزار قانونی کودتا فراهم کرد (مادهٔ ۴۱۷) \\
\altrow عامل شخصی & ژنرال \person{مین‌آنگ‌هلینگ}{\lr{Min Aung Hlaing}} در آستانهٔ بازنشستگی: یا قدرت یا محاکمه \\
عامل بین‌المللی & چین + روسیه حمایت ضمنی + \lr{ASEAN} ناتوان \\
\altrow عامل اقتصادی & ارتش حاضر به از دست دادن \lr{MEHL} و امپراتوری اقتصادی نبود \\
\bottomrule
\end{tabularx}
\end{table}

\subsection{پس از کودتا: مقاومت بی‌سابقه}

\begin{enumerate}[nosep]
\item \textbf{جنبش نافرمانی مدنی (\lr{CDM}):} صدها هزار کارمند اعتصاب کردند — پزشکان، معلمان، بانکداران
\item \textbf{نسل Z:} جوانان خلاقانه مقاومت کردند (سه‌انگشتی + فضای مجازی)
\item \textbf{\lr{NUG} (دولت وحدت ملی):} دولت موازی تشکیل شد (ترکیب \lr{NLD} + اقوام)
\item \textbf{\lr{PDF} (نیروهای دفاع مردمی):} مقاومت مسلحانه گسترش یافت
\item \textbf{سرکوب خونین:} ۱,۵۰۰+ کشته + ۱۲,۰۰۰+ بازداشتی (تا ۲۰۲۳)
\item \textbf{ائتلاف اقوام:} برای اولین بار، گروه‌های مسلح قومی با \lr{NUG} متحد شدند
\item \textbf{وضعیت ۲۰۲۴:} جنگ داخلی ادامه دارد + ارتش در برخی مناطق عقب‌نشینی کرده
\end{enumerate}

\begin{keypoint}
\textbf{تفاوت ۲۰۲۱ با ۱۹۸۸:} در کودتای قبلی (۱۹۸۸)، مردم تسلیم شدند. در ۲۰۲۱، نسل جدید تسلیم \textbf{نشد}. دلایل: ۱) یک دهه تجربهٔ آزادی نسبی؛ ۲) فضای مجازی و ارتباطات؛ ۳) الهام از هنگ‌کنگ و بهار عربی. \emphgreen{درس مثبت برای ایران:} حتی اگر گذار شکست بخورد، \textbf{خاطرهٔ آزادی} بازگشت‌ناپذیر است و مقاومت را قوی‌تر می‌کند.
\end{keypoint}

\sectiondivider

%═══════════════════════════════════════════════════════════
\section{ماتریس درس‌آموخته‌ها برای ایران}
\label{app:myanmar:lessons}
%═══════════════════════════════════════════════════════════

\begin{table}[htbp]
\centering
\caption{ماتریس انتقال درس‌آموخته‌های میانمار به ایران (عمدتاً هشدار)}
\label{tab:app-myanmar-lessons}
\begin{tabularx}{\textwidth}{
  >{\raggedleft\arraybackslash}p{2.2cm}
  >{\raggedleft\arraybackslash}p{3.5cm}
  >{\raggedleft\arraybackslash}X
  >{\centering\arraybackslash}p{1.3cm}
  >{\centering\arraybackslash}p{1cm}
}
\toprule
\headerrow \textbf{بُعد} & \textbf{درس میانمار} & \textbf{کاربرد ایرانی} & \textbf{نوع} & \textbf{اهمیت} \\
\midrule
\lr{SSR} & بدون اصلاح = بازگشت & اصلاح ساختاری سپاه اجتناب‌ناپذیر & \cellred{هشدار} & \rating{5} \\
\altrow
قانون اساسی & نظامی‌نوشت = دام & مجلس مؤسسان واقعاً مستقل & \cellred{هشدار} & \rating{5} \\
قفل‌های نهادی & ۲۵٪ = وتو = بازگشت & هیچ وتوی نهادی برای نظامیان & \cellred{هشدار} & \rating{5} \\
\altrow
اقتصاد ارتش & \lr{MEHL} دست‌نخورده ماند & تفکیک خاتم‌الانبیاء ضروری & \cellred{هشدار} & \rating{5} \\
سازش بدون اصلاح & سوچی سازش کرد → کودتا & مذاکره از موضع قدرت + اصلاح واقعی & \cellred{هشدار} & \rating{5} \\
\altrow
حقوق اقلیت‌ها & نسل‌کشی حتی در دورهٔ «دموکراسی» & منشور حقوق + \lr{ICC} + حقوق بهائیان/اقوام & \cellred{هشدار} & \rating{5} \\
مقاومت پس از شکست & نسل Z تسلیم نشد & خاطرهٔ آزادی بازگشت‌ناپذیر & \cellgreen{الگو} & \rating{3} \\
\altrow
ائتلاف اقوام & \lr{NUG} + گروه‌های قومی متحد شدند & ائتلاف فراگیر: فارس + کرد + ترک + بلوچ & \cellgreen{الگو} & \rating{4} \\
\midrule
\headerrow \multicolumn{3}{l}{\textbf{میانگین اهمیت (عمدتاً هشدار)}} & & \textbf{\rating{5}} \\
\bottomrule
\end{tabularx}
\end{table}

\sectiondivider

%═══════════════════════════════════════════════════════════
\section{نمودار: مقایسهٔ دو مسیر — گشایش واقعی vs صوری}
\label{app:myanmar:diagram}
%═══════════════════════════════════════════════════════════

\begin{figure}[htbp]
\centering
\begin{tikzpicture}[
  node distance=0.8cm,
  phase/.style={
    draw, rounded corners=5pt, minimum width=3cm,
    minimum height=1cm, font=\small, align=center,
    thick
  },
  good/.style={phase, fill=MainGreen!15, draw=MainGreen!60},
  bad/.style={phase, fill=MainRed!15, draw=MainRed!60},
  arrow/.style={->, thick, >=stealth},
  fork/.style={->, very thick, >=stealth}
]

% نقطهٔ شروع مشترک
\node[phase, fill=MainOrange!15, draw=MainOrange!60] (start) at (6,5) {گشایش سیاسی\\(تحت کنترل ارتش)};

% مسیر میانمار (شکست — راست)
\node[bad] (no-ssr) at (11,3) {بدون \lr{SSR}\\ارتش دست‌نخورده};
\node[bad] (facade) at (11,1) {دموکراسی\\صوری};
\node[bad] (coup) at (11,-1) {کودتا\\۲۰۲۱};
\node[bad] (war) at (11,-3) {جنگ داخلی\\۲۰۲۱+};

\draw[arrow, MainRed] (start.east) -| (no-ssr.north);
\draw[arrow, MainRed] (no-ssr) -- (facade);
\draw[arrow, MainRed] (facade) -- (coup);
\draw[arrow, MainRed] (coup) -- (war);

% عنوان
\node[font=\small\bfseries, MainRed] at (11,4.2) {مسیر میانمار: شکست};

% مسیر ایران پیشنهادی (موفقیت — چپ)
\node[good] (ssr) at (1,3) {\lr{SSR} واقعی\\ادغام + تفکیک};
\node[good] (const) at (1,1) {قانون اساسی\\مستقل + فراگیر};
\node[good] (consolidation) at (1,-1) {تحکیم\\نهادی};
\node[good] (democracy) at (1,-3) {دموکراسی\\پایدار};

\draw[arrow, MainGreen] (start.west) -| (ssr.north);
\draw[arrow, MainGreen] (ssr) -- (const);
\draw[arrow, MainGreen] (const) -- (consolidation);
\draw[arrow, MainGreen] (consolidation) -- (democracy);

% عنوان
\node[font=\small\bfseries, MainGreen] at (1,4.2) {مسیر پیشنهادی ایران: مدل ۶};

% نقطهٔ تفاوت
\node[draw=MainPurple, fill=MainPurple!10, rounded corners=3pt,
  font=\tiny\bfseries, align=center, text width=3cm] at (6,2.5) {نقطهٔ تفاوت:\\آیا ارتش واقعاً\\اصلاح می‌شود؟};

\draw[fork, MainGreen, dashed] (6,3.2) -- (ssr);
\draw[fork, MainRed, dashed] (6,3.2) -- (no-ssr);

\end{tikzpicture}
\caption{دو مسیر گشایش: اصلاح واقعی (مدل ۶) vs صوری (مدل میانمار)}
\label{fig:app-myanmar-paths}
\end{figure}

\sectiondivider

%═══════════════════════════════════════════════════════════
\section{جمع‌بندی پیوست}
\label{app:myanmar:conclusion}
%═══════════════════════════════════════════════════════════

\begin{chaptersummary}
جمع‌بندی پیوست چ — میانمار: گذار ناتمام:

\begin{enumerate}[nosep]
\item میانمار \textbf{قوی‌ترین هشدار} برای ایران است: گشایش بدون \lr{SSR} واقعی = بازگشت قطعی.
\item \textbf{تاتمادو آینهٔ سپاه} است: هر دو ارتش‌هایی با منافع اقتصادی عظیم، فرماندهی مستقل، و وتوی قانون اساسی.
\item \textbf{قانون اساسی نظامی‌نوشت} خطرناک‌ترین «قفل» است — مجلس مؤسسان ایران باید واقعاً مستقل باشد.
\item \textbf{سازش بدون اصلاح = تأخیر در فاجعه}: سوچی یک دهه سازش کرد اما تاتمادو هرگز اصلاح نشد و در نهایت کودتا کرد.
\item \textbf{نسل‌کشی روهینگیا} نشان داد: انتخابات + پارلمان ≠ حقوق بشر. منشور حقوق غیرقابل‌نقض ضروری است.
\item نکتهٔ مثبت: \textbf{مقاومت نسل Z} نشان داد که خاطرهٔ آزادی بازگشت‌ناپذیر است.
\item برای ایران: اصلاح ساختاری سپاه (ادغام + تفکیک اقتصادی + نظارت مدنی) \textbf{شرط لازم} هر گذار موفقی است — بدون آن، حتی بهترین قانون اساسی و انتخابات آزاد‌ترین، ناپایدار خواهد بود.
\end{enumerate}

\vspace{0.3cm}
\textit{مطالعهٔ تکمیلی:}
\begin{itemize}[nosep]
\item مقایسهٔ جامع ۹ نمونه: \seeChapter{app:comparison}
\item اصلاح بخش امنیتی: \seeChapter{ch:guarantees}
\item عراق (ضد الگوی دیگر): \seeChapter{app:iraq}
\item آفریقای جنوبی (ادغام نیروها): \seeChapter{app:south-africa}
\item اندونزی (تفکیک اقتصادی ارتش): \seeChapter{app:comparison}
\item تیمور شرقی: \seeChapter{app:timor}
\end{itemize}
\end{chaptersummary}

\chapterend

%══════════════════════════════════════════════════════════════
% پایان پیوست چ
%══════════════════════════════════════════════════════════════