%══════════════════════════════════════════════════════════════
% پیوست ح: مطالعه موردی تیمور شرقی
% فایل: appendices/app-h-timor.tex
% حجم هدف: ۶-۸ صفحه
%══════════════════════════════════════════════════════════════

\chapter{مطالعهٔ موردی: تیمور شرقی (۱۹۹۹-۲۰۰۲)}
\label{app:timor}

\begin{executivesummary}
تیمور شرقی (تیمور-لسته) یکی از معدود نمونه‌های موفق \textbf{مدیریت مستقیم بین‌المللی} (\lr{International Transitional Administration}) است که از \textbf{اشغال ۲۴ سالهٔ اندونزی} (۱۹۷۵-۱۹۹۹) به دولت مستقل دموکراتیک انتقال یافت. ویژگی‌های منحصربه‌فرد: ۱) \textbf{رفراندوم استقلال} تحت نظارت سازمان ملل (\lr{UNAMET}، ۱۹۹۹)، ۲) \textbf{خشونت وحشیانهٔ میلیشیای اندونزیایی} پس از رفراندوم، ۳) \textbf{مأموریت مدیریت انتقالی} (\lr{UNTAET}: مدل ۵ از فصل ۳)، ۴) \textbf{ساختن دولت از صفر} (بدون ارتش، بدون بوروکراسی، بدون زیرساخت)، و ۵) \textbf{کمیسیون حقیقت} (\lr{CAVR}). این تجربه هم الگو (مرحله‌بندی، رفراندوم، نهادسازی) و هم هشدار (وابستگی به کمک خارجی، بحران ۲۰۰۶) برای ایران است. \textbf{تفاوت بنیادین:} تیمور کشوری کوچک (۱ میلیون) و فقیر بود که دولت‌سازی از صفر نیاز داشت — ایران کشوری بزرگ با نهادهای موجود (هرچند آسیب‌دیده) است که نیازمند \textbf{اصلاح} (نه ساختن از صفر) است.
\end{executivesummary}

%═══════════════════════════════════════════════════════════
\section{زمینه و بافت تاریخی}
\label{app:timor:context}
%═══════════════════════════════════════════════════════════

\subsection{از استعمار پرتغال تا اشغال اندونزی}

\begin{table}[htbp]
\centering
\caption{مشخصات تیمور شرقی در آستانهٔ رفراندوم (۱۹۹۹)}
\label{tab:app-timor-profile}
\begin{tabularx}{\textwidth}{>{\raggedleft\arraybackslash}p{4.5cm} >{\raggedleft\arraybackslash}X}
\toprule
\headerrow \textbf{شاخص} & \textbf{مقدار} \\
\midrule
جمعیت & $\sim$۸۰۰,۰۰۰ نفر \\
\altrow مساحت & ۱۵,۰۰۰ \lr{km²} (نصف استان فارس) \\
استعمار پرتغال & ۱۵۱۵-۱۹۷۵ (۴۶۰ سال) \\
\altrow اشغال اندونزی & ۱۹۷۵-۱۹۹۹ (۲۴ سال) \\
تلفات اشغال & ۱۰۰,۰۰۰-۱۸۰,۰۰۰ (از ۶۰۰K جمعیت ۱۹۷۵ = تا ۳۰٪) \\
\altrow \lr{GDP per capita} & $\sim$\$۴۰۰ (یکی از فقیرترین آسیا) \\
زبان‌ها & تتوم + پرتغالی (رسمی) + بهاسا اندونزیایی + ۳۰+ زبان محلی \\
\altrow جنبش مقاومت & \lr{FRETILIN} (سیاسی) + \lr{FALINTIL} (مسلح) \\
رهبر مقاومت & \person{شاناناگوسمائو}{\lr{Xanana Gusmão}} (زندانی ۱۹۹۲-۱۹۹۹) \\
\altrow تنوع مذهبی & ۹۷٪ کاتولیک (یکدست) \\
\bottomrule
\end{tabularx}
\end{table}

\subsection{گاه‌شمار گذار}

\begin{table}[htbp]
\centering
\caption{گاه‌شمار کلیدی تیمور شرقی (۱۹۹۶-۲۰۰۲)}
\label{tab:app-timor-timeline}
\begin{tabularx}{\textwidth}{>{\centering\arraybackslash}p{2.5cm} >{\raggedleft\arraybackslash}X >{\centering\arraybackslash}p{2cm}}
\toprule
\headerrow \textbf{تاریخ} & \textbf{رویداد} & \textbf{اهمیت} \\
\midrule
۱۹۹۶ & \person{راموس‌هورتا}{\lr{Ramos-Horta}} و اسقف \person{بلو}{\lr{Belo}} نوبل صلح گرفتند & مشروعیت بین‌المللی \\
\altrow مه ۱۹۹۸ & سقوط سوهارتو در اندونزی & تغییر محیط \\
مه ۱۹۹۹ & \textbf{توافق سه‌جانبه} (اندونزی + پرتغال + سازمان ملل): رفراندوم & نقطهٔ عطف \\
\altrow ژوئن ۱۹۹۹ & استقرار \lr{UNAMET} (مأموریت رفراندوم سازمان ملل) & آغاز نظارت \\
\textbf{۳۰ اوت ۱۹۹۹} & \textbf{رفراندوم:} ۷۸.۵٪ «استقلال» / ۲۱.۵٪ «خودمختاری» & تعیین‌کننده \\
\altrow سپتامبر ۱۹۹۹ & \textbf{خشونت میلیشیا:} ویرانی ۷۰٪ زیرساخت + ۱,۴۰۰ کشته + ۳۰۰K آواره & فاجعه \\
۲۰ سپتامبر ۱۹۹۹ & \textbf{\lr{INTERFET}:} نیروی چندملیتی به رهبری استرالیا (قطعنامهٔ ۱۲۶۴) & مداخلهٔ نظامی \\
\altrow ۲۵ اکتبر ۱۹۹۹ & \textbf{\lr{UNTAET}} تأسیس شد (قطعنامهٔ ۱۲۷۲): مدیریت مستقیم سازمان ملل & مدل ۵ \\
اوت ۲۰۰۱ & انتخابات مجلس مؤسسان (مشارکت ۹۳٪) & نهادسازی \\
\altrow مارس ۲۰۰۲ & تصویب قانون اساسی & نقطهٔ عطف \\
آوریل ۲۰۰۲ & انتخابات ریاست‌جمهوری: گوسمائو ۸۳٪ & مشروعیت \\
\altrow \textbf{۲۰ مه ۲۰۰۲} & \textbf{استقلال رسمی:} تولد جمهوری دموکراتیک تیمور-لسته & موفقیت \\
\bottomrule
\end{tabularx}
\end{table}

\sectiondivider

%═══════════════════════════════════════════════════════════
\section{رفراندوم ۱۹۹۹: پیروزی و خشونت}
\label{app:timor:referendum}
%═══════════════════════════════════════════════════════════

\subsection{سازماندهی رفراندوم}

\begin{table}[htbp]
\centering
\caption{آمار رفراندوم تیمور شرقی ۱۹۹۹ و مقایسه با شیلی}
\label{tab:app-timor-referendum}
\begin{tabularx}{\textwidth}{>{\raggedleft\arraybackslash}p{4cm} >{\centering\arraybackslash}p{3cm} >{\centering\arraybackslash}p{3cm}}
\toprule
\headerrow \textbf{شاخص} & \textbf{تیمور ۱۹۹۹} & \textbf{شیلی ۱۹۸۸} \\
\midrule
سؤال & استقلال یا خودمختاری & ادامهٔ پینوشه یا نه \\
\altrow واجدین شرایط & ۴۵۱,۸۹۲ & ۷,۴۳۵,۹۱۳ \\
مشارکت & ۹۸.۶٪ & ۹۷.۵٪ \\
\altrow نتیجه & ۷۸.۵٪ استقلال & ۵۵.۹۹٪ «نه» \\
سازمان‌دهنده & سازمان ملل (\lr{UNAMET}) & خود رژیم (با ناظران) \\
\altrow ناظران بین‌المللی & ۵۰ رأی‌گیری + ۲,۰۰۰+ ناظر & $\sim$۱,۰۰۰ \\
امنیت & ناکافی (پلیس اندونزی!) & نسبتاً مطلوب \\
\altrow خشونت پس از نتیجه & \riskhigh فاجعه‌بار & \risklow حداقل \\
\bottomrule
\end{tabularx}
\end{table}

\subsection{فاجعهٔ سپتامبر ۱۹۹۹}

\begin{warningbox}
\textbf{فاجعه‌ای که قابل پیشگیری بود:} پس از اعلام نتیجهٔ رفراندوم، میلیشیاهای اندونزیایی (\lr{Aitarak, Besi Merah Putih}) با حمایت \lr{TNI} (ارتش اندونزی) به \textbf{عملیات زمین‌سوخته} دست زدند:

\begin{itemize}[nosep]
\item \textbf{۱,۴۰۰+ کشته} (برخی تخمین‌ها: ۲,۰۰۰+)
\item \textbf{۳۰۰,۰۰۰+ آواره} (از ۸۰۰K جمعیت = ۳۸٪!)
\item \textbf{۷۰٪ زیرساخت‌ها ویران شد:} ساختمان‌های دولتی، مدارس، بیمارستان‌ها، خانه‌ها
\item \textbf{کارمندان \lr{UNAMET} هم هدف حمله قرار گرفتند}
\item \textbf{سازمان ملل ابتدا عقب‌نشینی کرد} — انتقاد شدید: «رفراندوم بدون تضمین امنیت»
\end{itemize}

\textbf{درس حیاتی:} رفراندوم بدون \textbf{تضمین امنیتی کافی} می‌تواند فاجعه‌ایجاد کند. توافق سه‌جانبه امنیت را به \textbf{پلیس اندونزی} سپرده بود — همان نیرویی که با میلیشیاها همکاری می‌کرد. \emphred{هشدار ایرانی:} هر رفراندوم یا انتخاباتی در ایران باید با \textbf{تضمین امنیتی مستقل} همراه باشد. امنیت انتخابات نمی‌تواند به نیروهایی سپرده شود که منافع‌شان با نتیجه در تضاد است (\seeChapter{ch:timeline}).
\end{warningbox}

\subsection{مداخلهٔ \lr{INTERFET}: مدل واکنش سریع}

پس از فاجعه، شورای امنیت قطعنامهٔ ۱۲۶۴ را تصویب کرد و نیروی \org{اینترفت}{\lr{INTERFET (International Force East Timor)}} به رهبری \textbf{استرالیا} مستقر شد:

\begin{table}[htbp]
\centering
\caption{مشخصات نیروی \lr{INTERFET}}
\label{tab:app-timor-interfet}
\begin{tabularx}{\textwidth}{>{\raggedleft\arraybackslash}p{4cm} >{\raggedleft\arraybackslash}X}
\toprule
\headerrow \textbf{شاخص} & \textbf{جزئیات} \\
\midrule
قطعنامه & شورای امنیت ۱۲۶۴ (۱۵ سپتامبر ۱۹۹۹) \\
\altrow فرمانده & ژنرال \person{پیتر کاسگرو}{\lr{Peter Cosgrove}} (استرالیا) \\
تعداد نیرو & $\sim$۱۱,۰۰۰ (از ۲۲ کشور) \\
\altrow کشورهای اصلی & استرالیا (۵,۵۰۰)، تایلند، فیلیپین، کرهٔ جنوبی، نیوزیلند \\
مأموریت & بازگرداندن امنیت + حمایت از \lr{UNAMET} + حمایت انسانی \\
\altrow مدت & سپتامبر ۱۹۹۹ — فوریهٔ ۲۰۰۰ (سپس به نیروی \lr{UNTAET} تبدیل شد) \\
نتیجه & عقب‌نشینی اندونزی + بازگشت آوارگان + امنیت نسبی & \\
\bottomrule
\end{tabularx}
\end{table}

\begin{lessonlearned}
\textbf{\lr{INTERFET} نشان داد:} ۱) واکنش سریع بین‌المللی وقتی \textbf{ارادهٔ سیاسی} وجود داشته باشد، ممکن و مؤثر است؛ ۲) \textbf{رهبری منطقه‌ای} (استرالیا) مؤثرتر از رهبری دور (آمریکا/اروپا) است؛ ۳) مشارکت چندملیتی مشروعیت بیشتری دارد. \emphgreen{کاربرد ایرانی:} در سناریوی \lr{A} (فروپاشی ناگهانی)، ممکن است نیروی حافظ صلح لازم شود. ترکیب: سازمان ملل + کشورهای منطقه (ترکیه؟ هند؟) + ناتو — اما تنها با دعوت ایرانیان (\seeChapter{ch:scenarios}).
\end{lessonlearned}

\sectiondivider

%═══════════════════════════════════════════════════════════
\section{\lr{UNTAET}: مدل مدیریت مستقیم بین‌المللی}
\label{app:timor:untaet}
%═══════════════════════════════════════════════════════════

\subsection{ساختار و مأموریت}

\org{ادارهٔ انتقالی سازمان ملل در تیمور شرقی}{\lr{UNTAET (UN Transitional Administration in East Timor)}} (قطعنامهٔ ۱۲۷۲، اکتبر ۱۹۹۹) یکی از جامع‌ترین مأموریت‌های سازمان ملل بود — حاکمیت کامل سرزمینی:

\begin{table}[htbp]
\centering
\caption{ساختار و آمار \lr{UNTAET} (۱۹۹۹-۲۰۰۲)}
\label{tab:app-timor-untaet}
\begin{tabularx}{\textwidth}{>{\raggedleft\arraybackslash}p{5cm} >{\raggedleft\arraybackslash}X}
\toprule
\headerrow \textbf{شاخص} & \textbf{جزئیات} \\
\midrule
نمایندهٔ ویژه / حاکم موقت & \person{سرجیو ویئیرا دملو}{\lr{Sérgio Vieira de Mello}} \\
\altrow اختیارات & \textbf{تمام قوای سه‌گانه:} مقننه + مجریه + قضاییه \\
نیروی نظامی & $\sim$۸,۰۰۰ \\
\altrow پلیس مدنی & $\sim$۱,۶۴۰ \\
کارمندان غیرنظامی & $\sim$۱,۰۰۰ بین‌المللی + ۲,۰۰۰ محلی \\
\altrow بودجهٔ سالانه & $\sim$\$۵۶۳ میلیون (سال اول) \\
مأموریت & امنیت + حاکمیت + نهادسازی + آموزش + انتخابات + استقلال \\
\altrow مدت & اکتبر ۱۹۹۹ — مه ۲۰۰۲ (۳۱ ماه) \\
مأموریت جانشین & \lr{UNMISET} (۲۰۰۲-۲۰۰۵) → \lr{UNMIT} (۲۰۰۶-۲۰۱۲) \\
\bottomrule
\end{tabularx}
\end{table}

\subsection{دستاوردها و نقدها}

\begin{table}[htbp]
\centering
\caption{دستاوردها و نقدهای \lr{UNTAET}}
\label{tab:app-timor-untaet-review}
\begin{tabularx}{\textwidth}{>{\centering\arraybackslash}p{1cm} >{\raggedleft\arraybackslash}X >{\centering\arraybackslash}p{2cm}}
\toprule
\headerrow & \textbf{دستاوردها} & \textbf{امتیاز} \\
\midrule
\cmark & استقلال در ۳۱ ماه (سریع و مؤثر) & \starrating{5} \\
\altrow \cmark & انتخابات مجلس مؤسسان (مشارکت ۹۳٪) & \starrating{5} \\
\cmark & قانون اساسی دموکراتیک & \starrating{4} \\
\altrow \cmark & ایجاد نیروی دفاعی (\lr{F-FDTL}) و پلیس (\lr{PNTL}) از صفر & \starrating{3} \\
\cmark & بازگشت ۳۰۰,۰۰۰ آواره & \starrating{4} \\
\altrow \cmark & \lr{CAVR} (کمیسیون حقیقت) & \starrating{4} \\
\midrule
\headerrow & \textbf{نقدها} & \textbf{شدت} \\
\midrule
\xmark & \textbf{کم‌مشارکت‌دادن تیموری‌ها:} «مدیریت استعماری جدید»؟ & \riskhigh \\
\altrow \xmark & حقوق‌های بالای کارمندان بین‌المللی vs فقر محلی & \riskmedium \\
\xmark & سرعت خروج: ظرفیت‌سازی ناکافی → بحران ۲۰۰۶ & \riskhigh \\
\altrow \xmark & عدالت بین‌المللی ناکام: اندونزی تعقیب نشد & \riskhigh \\
\xmark & وابستگی به کمک خارجی پس از استقلال & \riskmedium \\
\bottomrule
\end{tabularx}
\end{table}

\begin{keypoint}
\textbf{نقد اصلی: «مدیریت مستقیم» ≠ «مالکیت ملی».} ویئیرا دملو (بعداً در بمب‌گذاری بغداد ۲۰۰۳ کشته شد) تلاش کرد تیموری‌ها را مشارکت دهد، اما ساختار \lr{UNTAET} ذاتاً \textbf{بالا به پایین} بود. بسیاری از تصمیمات توسط خارجی‌ها گرفته شد. رهبران تیموری (\person{گوسمائو}{\lr{Gusmão}} و \person{آلکاتیری}{\lr{Alkatiri}}) گاه احساس حاشیه‌نشینی می‌کردند. این نقد تأیید می‌کند که \textbf{مدل ۵ (مدیریت مستقیم) برای ایران مناسب نیست} (\seeChapter{ch:approaches}). مدل ۶ (ترکیبی-تطبیقی) با مالکیت ملی ایرانی طراحی شده.
\end{keypoint}

\sectiondivider

%═══════════════════════════════════════════════════════════
\section{کمیسیون حقیقت (\lr{CAVR})}
\label{app:timor:cavr}
%═══════════════════════════════════════════════════════════

\org{کمیسیون پذیرش، حقیقت و آشتی}{\lr{CAVR (Comissão de Acolhimento, Verdade e Reconciliação)}} از ۲۰۰۲ تا ۲۰۰۵ فعالیت کرد:

\begin{table}[htbp]
\centering
\caption{ساختار و آمار \lr{CAVR} تیمور شرقی}
\label{tab:app-timor-cavr}
\begin{tabularx}{\textwidth}{>{\raggedleft\arraybackslash}p{5cm} >{\raggedleft\arraybackslash}X}
\toprule
\headerrow \textbf{شاخص} & \textbf{جزئیات} \\
\midrule
مدت & ژانویه ۲۰۰۲ — اکتبر ۲۰۰۵ \\
\altrow تعداد کمیسیونرها & ۷ (همه تیموری) \\
بازهٔ زمانی & ۱۹۷۴-۱۹۹۹ (۲۵ سال) \\
\altrow تعداد شهادت‌ها & ۷,۸۲۴ \\
\termfn{فرآیند آشتی جامعه‌محور}{\lr{CRP}} & ۱,۳۷۱ فرآیند (در سطح روستا) \\
\altrow تخمین تلفات کل & ۱۰۲,۸۰۰ (± ۱۲,۰۰۰) — تحلیل آماری \\
گزارش نهایی & «\lr{Chega!}» (بس است!) — ۵ جلد، ۲,۵۰۰+ صفحه \\
\altrow نوآوری ویژه & \textbf{فرآیند آشتی محلی (\lr{CRP}):} عاملان در روستا اعتراف و خدمت اجتماعی انجام دادند \\
\bottomrule
\end{tabularx}
\end{table}

\subsection{نوآوری \lr{CRP}: عدالت جامعه‌محور}

\begin{enumerate}[nosep]
\item عاملان «جرایم سبک» (آتش‌زدن خانه، سرقت، تخریب — نه قتل) داوطلبانه به \lr{CAVR} مراجعه می‌کردند
\item در جلسهٔ عمومی روستا، \textbf{اعتراف علنی} + \textbf{عذرخواهی از قربانیان} + \textbf{توافق بر مجازات جامعه‌ای} (مثلاً بازسازی خانه، خدمت اجتماعی)
\item پس از انجام مجازات، \textbf{پذیرش مجدد در جامعه} (\lr{Acolhimento} = پذیرش)
\item ۱,۳۷۱ فرآیند انجام شد — \textbf{۸۵٪+ موفقیت‌آمیز}
\end{enumerate}

\begin{lessonlearned}
\textbf{مدل \lr{CRP} برای ایران:} فرآیند آشتی جامعه‌محور می‌تواند برای «بسیجیان عادی» یا «عوامل محلی سرکوب» استفاده شود — نه برای فرماندهان ارشد (که باید محاکمه شوند). \emphgreen{کاربرد:} در شهرها و روستاهایی که بسیجیان محلی در سرکوب مشارکت کرده‌اند، جلسات آشتی با حضور قربانیان + اعتراف + خدمت اجتماعی — به‌عنوان مکمل (نه جایگزین) محاکمهٔ فرماندهان (\seeChapter{ch:guarantees}).
\end{lessonlearned}

\sectiondivider

%═══════════════════════════════════════════════════════════
\section{بحران ۲۰۰۶: شکنندگی دولت نوپا}
\label{app:timor:crisis}
%═══════════════════════════════════════════════════════════

\subsection{ریشه‌ها و درس‌ها}

تنها ۴ سال پس از استقلال، تیمور شرقی در ۲۰۰۶ دچار بحران شدید شد:

\begin{table}[htbp]
\centering
\caption{بحران ۲۰۰۶ تیمور شرقی: علل و نتایج}
\label{tab:app-timor-crisis}
\begin{tabularx}{\textwidth}{>{\raggedleft\arraybackslash}p{3.5cm} >{\raggedleft\arraybackslash}X}
\toprule
\headerrow \textbf{بُعد} & \textbf{جزئیات} \\
\midrule
محرک آنی & اعتراض ۶۰۰ سرباز «غربی» (\lr{loromonu}) به تبعیض از سوی فرماندهان «شرقی» (\lr{lorosa'e}) \\
\altrow تنش نهادی & رقابت ارتش (\lr{F-FDTL}) و پلیس (\lr{PNTL}) — هر دو از صفر ساخته شده \\
ظرفیت‌سازی ناکافی & \lr{UNTAET} خیلی سریع خارج شد + نهادها شکننده بودند \\
\altrow نتیجه & ۱۵۰+ کشته + ۱۰۰,۰۰۰ آواره (باز هم!) + سقوط دولت آلکاتیری \\
واکنش بین‌المللی & نیروی بین‌المللی (استرالیا + نیوزیلند) بازگشت + \lr{UNMIT} تشکیل شد \\
\altrow درس & خروج زودهنگام نظارت بین‌المللی = خطر بازگشت بحران \\
\bottomrule
\end{tabularx}
\end{table}

\begin{warningbox}
\textbf{درس بحران ۲۰۰۶ برای ایران:} \textbf{خروج زودهنگام نظارت بین‌المللی} خطرناک است. \lr{UNTAET} در ۳۱ ماه خارج شد — ظرفیت‌سازی کافی نبود. مدل ۶ پیشنهادی برای ایران \textbf{۵ فاز در ۱۰ سال} پیش‌بینی کرده و خروج مشروط به رسیدن به شاخص‌های کمّی است (نه تقویم ثابت). سه سناریوی خروج در فصل ۹ تعریف شده: موفق، تعدیل، تمدید. بحران ۲۰۰۶ تأیید می‌کند که سناریوی «تمدید» باید همیشه روی میز باشد (\seeChapter{ch:timeline}).
\end{warningbox}

\sectiondivider

%═══════════════════════════════════════════════════════════
\section{ماتریس درس‌آموخته‌ها برای ایران}
\label{app:timor:lessons}
%═══════════════════════════════════════════════════════════

\begin{table}[htbp]
\centering
\caption{ماتریس انتقال درس‌آموخته‌های تیمور شرقی به ایران}
\label{tab:app-timor-lessons}
\begin{tabularx}{\textwidth}{
  >{\raggedleft\arraybackslash}p{2.5cm}
  >{\raggedleft\arraybackslash}p{3.5cm}
  >{\raggedleft\arraybackslash}X
  >{\centering\arraybackslash}p{1.5cm}
}
\toprule
\headerrow \textbf{بُعد} & \textbf{درس تیمور} & \textbf{کاربرد ایرانی} & \textbf{انتقال‌پذیری} \\
\midrule
رفراندوم & سازماندهی موفق اما بدون تضمین امنیت & رفراندوم + تضمین امنیتی مستقل & \rating{4} \\
\altrow
واکنش سریع & \lr{INTERFET}: ۱۱K نیرو در ۲ هفته & آمادگی نیروی واکنش (سناریوی \lr{A}) & \rating{3} \\
مرحله‌بندی & \lr{UNAMET} → \lr{INTERFET} → \lr{UNTAET} → \lr{UNMISET} & فاز ۰→۱→۲→۳→۴ (مدل ۶) & \rating{5} \\
\altrow
نهادسازی & ایجاد ارتش + پلیس + قضا از صفر & ایران: اصلاح نهادهای موجود (نه از صفر) & \rating{2} \\
\lr{CAVR} & حقیقت + آشتی جامعه‌محور (\lr{CRP}) & مکمل \lr{TRC} برای عوامل محلی & \rating{4} \\
\altrow
مالکیت ملی & \lr{UNTAET} انتقاد شد: «استعمار جدید» & مدل ۶: مالکیت ملی ایرانی (نه مدل ۵) & \rating{5} \\
خروج & خروج زودهنگام → بحران ۲۰۰۶ & خروج مشروط به شاخص‌ها (نه تقویم) & \rating{5} \\
\altrow
ظرفیت‌سازی & ناکافی بود & برنامهٔ ظرفیت‌سازی ۱۰ ساله & \rating{4} \\
\midrule
\headerrow \multicolumn{3}{l}{\textbf{میانگین انتقال‌پذیری}} & \textbf{\rating{4}} \\
\bottomrule
\end{tabularx}
\end{table}

\sectiondivider

%═══════════════════════════════════════════════════════════
\section{نمودار: مرحله‌بندی مأموریت‌های سازمان ملل در تیمور}
\label{app:timor:diagram}
%═══════════════════════════════════════════════════════════

\begin{figure}[htbp]
\centering
\begin{tikzpicture}[
  node distance=0.5cm,
  mission/.style={
    draw, rounded corners=5pt, minimum width=3cm,
    minimum height=1.3cm, font=\small, align=center,
    thick
  },
  arrow/.style={->, very thick, >=stealth, gray},
  timeline/.style={very thick, gray}
]

% خط زمانی
\draw[timeline, ->] (0,0) -- (15,0);
\foreach \x/\y in {0/1999, 3/2000, 5/2002, 8/2005, 11/2008, 14/2012} {
  \draw[gray] (\x,-0.2) -- (\x,0.2);
  \node[font=\tiny, below] at (\x,-0.3) {\y};
}

% مأموریت‌ها
\node[mission, fill=MainRed!15, draw=MainRed!60] (unamet) at (0.5,2) {\lr{UNAMET}\\رفراندوم};
\node[mission, fill=MainOrange!15, draw=MainOrange!60] (interfet) at (2.5,2) {\lr{INTERFET}\\امنیت};
\node[mission, fill=MainPurple!15, draw=MainPurple!60] (untaet) at (5,2) {\lr{UNTAET}\\مدیریت مستقیم};
\node[mission, fill=MainBlue!15, draw=MainBlue!60] (unmiset) at (8,2) {\lr{UNMISET}\\حمایت};
\node[mission, fill=MainGreen!15, draw=MainGreen!60] (unmit) at (11,2) {\lr{UNMIT}\\بحران ۲۰۰۶};
\node[mission, fill=MainGreen!25, draw=MainGreen!80] (exit) at (14,2) {خروج\\۲۰۱۲};

\draw[arrow] (unamet) -- (interfet);
\draw[arrow] (interfet) -- (untaet);
\draw[arrow] (untaet) -- (unmiset);
\draw[arrow] (unmiset) -- (unmit);
\draw[arrow] (unmit) -- (exit);

% شدت حضور
\draw[MainPurple!60, very thick] plot[smooth] coordinates {(0.5,-1) (2.5,-2.5) (5,-3) (8,-2) (11,-2.5) (14,-0.5)};
\node[font=\tiny, MainPurple] at (7,-3.5) {شدت حضور بین‌المللی};

% نقاط عطف
\node[font=\tiny, MainRed, anchor=south] at (0.5,3) {رفراندوم\\۳۰ اوت};
\node[font=\tiny, MainOrange, anchor=south] at (2.5,3) {خشونت\\→ مداخله};
\node[font=\tiny, MainPurple, anchor=south] at (5,3) {حاکمیت\\سازمان ملل};
\node[font=\tiny, MainBlue, anchor=south] at (8,3) {کاهش\\تدریجی};
\node[font=\tiny, MainRed, anchor=south] at (11,3) {بحران!\\بازگشت};
\node[font=\tiny, MainGreen, anchor=south] at (14,3) {خروج\\نهایی};

% معادل ایرانی
\node[font=\tiny\itshape, MainPurple, anchor=north] at (0.5,-0.5) {فاز ۰};
\node[font=\tiny\itshape, MainPurple, anchor=north] at (2.5,-0.5) {فاز ۱};
\node[font=\tiny\itshape, MainPurple, anchor=north] at (5,-0.5) {فاز ۲};
\node[font=\tiny\itshape, MainPurple, anchor=north] at (8,-0.5) {فاز ۳};
\node[font=\tiny\itshape, MainPurple, anchor=north] at (11,-0.5) {(احتمالی)};
\node[font=\tiny\itshape, MainPurple, anchor=north] at (14,-0.5) {فاز ۴};

\end{tikzpicture}
\caption{مرحله‌بندی مأموریت‌های سازمان ملل در تیمور شرقی و فازهای معادل ایرانی}
\label{fig:app-timor-phases}
\end{figure}

\sectiondivider

%═══════════════════════════════════════════════════════════
\section{وضعیت فعلی (۲۰۲۳) و ارزیابی بلندمدت}
\label{app:timor:current}
%═══════════════════════════════════════════════════════════

\begin{table}[htbp]
\centering
\caption{شاخص‌های تیمور شرقی (۲۰۲۳): ۲۱ سال پس از استقلال}
\label{tab:app-timor-current}
\begin{tabularx}{\textwidth}{>{\raggedleft\arraybackslash}X >{\centering\arraybackslash}p{3.5cm}}
\toprule
\headerrow \textbf{شاخص} & \textbf{مقدار} \\
\midrule
\lr{V-Dem Liberal Democracy Index} & ۰.۵۸ (دموکراسی انتخاباتی) \\
\altrow \lr{Freedom House} & نیمه‌آزاد (۷۲/۱۰۰) \\
\lr{GDP per capita (PPP)} & $\sim$\$۵,۰۰۰ (وابسته به نفت تیمور گپ) \\
\altrow جمعیت & ۱.۳ میلیون \\
انتقال مسالمت‌آمیز قدرت & \cmark (چندین بار: ۲۰۰۷, ۲۰۱۲, ۲۰۱۷, ۲۰۲۳) \\
\altrow عضویت بین‌المللی & سازمان ملل (۲۰۰۲) + \lr{ASEAN} (درخواست) \\
چالش اصلی & فقر + وابستگی به نفت + جوانی جمعیت (۶۵٪ زیر ۳۰) \\
\bottomrule
\end{tabularx}
\end{table}

\begin{recommendation}
\textbf{ارزیابی کلی:} تیمور شرقی با وجود تمام محدودیت‌ها (کوچکی، فقر، ویرانی ۷۰٪) به \textbf{دموکراسی نسبتاً پایدار} دست یافته — چندین انتقال مسالمت‌آمیز قدرت، قانون اساسی دموکراتیک، جامعهٔ مدنی فعال. اما \textbf{چالش‌های توسعه‌ای} جدی باقی مانده. درس: دموکراسی بدون توسعهٔ اقتصادی شکننده است — تأیید مجدد درس تونس (\seeChapter{app:tunisia}).
\end{recommendation}

\sectiondivider

%═══════════════════════════════════════════════════════════
\section{جمع‌بندی پیوست}
\label{app:timor:conclusion}
%═══════════════════════════════════════════════════════════

\begin{chaptersummary}
جمع‌بندی پیوست ح — تیمور شرقی:

\begin{enumerate}[nosep]
\item تیمور شرقی نمونهٔ موفق \textbf{مدیریت مستقیم بین‌المللی} (مدل ۵) است — اما با نقد جدی: کم‌مشارکت‌دادن محلی‌ها.
\item \textbf{مدل ۵ برای ایران مناسب نیست} — ایران کشوری بزرگ با نهادهای موجود است. مدل ۶ (ترکیبی با مالکیت ملی) مناسب‌تر است.
\item \textbf{رفراندوم ۱۹۹۹} نشان داد رفراندوم ابزار قدرتمندی است، \textbf{مشروط به تضمین امنیتی مستقل}.
\item \textbf{مرحله‌بندی مأموریت‌ها} (\lr{UNAMET→INTERFET→UNTAET→UNMISET}) الگوی خوبی برای فازبندی مدل ۶ است.
\item \textbf{فرآیند \lr{CRP}} (آشتی جامعه‌محور) نوآوری ارزشمندی است — مکمل \lr{TRC} برای عوامل محلی سرکوب.
\item \textbf{خروج زودهنگام} (۳۱ ماه) منجر به بحران ۲۰۰۶ شد — خروج باید مشروط به شاخص‌ها باشد.
\item تیمور با وجود محدودیت‌ها به دموکراسی نسبتاً پایدار رسیده — اثبات اینکه حتی در بدترین شرایط، دموکراسی ممکن است.
\end{enumerate}

\vspace{0.3cm}
\textit{مطالعهٔ تکمیلی:}
\begin{itemize}[nosep]
\item مقایسهٔ جامع ۹ نمونه: \seeChapter{app:comparison}
\item رویکردها و مدل‌های نظارت: \seeChapter{ch:approaches}
\item زمان‌بندی و فازبندی: \seeChapter{ch:timeline}
\item عراق (مدل ۵ ناموفق): \seeChapter{app:iraq}
\item آفریقای جنوبی (مدل ۳ موفق): \seeChapter{app:south-africa}
\end{itemize}
\end{chaptersummary}

\chapterend

%══════════════════════════════════════════════════════════════
% پایان پیوست ح
%══════════════════════════════════════════════════════════════