%══════════════════════════════════════════════════════════════
% پیوست ت: مطالعه موردی تونس
% فایل: appendices/app-d-tunisia.tex
% حجم هدف: ۸-۱۰ صفحه
%══════════════════════════════════════════════════════════════

\chapter{مطالعهٔ موردی: تونس (۲۰۱۱-۲۰۲۱)}
\label{app:tunisia}

\begin{executivesummary}
تونس تنها نمونهٔ «موفقِ اولیه» در \termfn{بهار عربی}{\lr{Arab Spring}} بود که از دیکتاتوری \person{زین‌العابدین بن‌علی}{\lr{Zine El Abidine Ben Ali}} (۱۹۸۷-۲۰۱۱) به دموکراسی نوپا گذار کرد. ویژگی‌های منحصربه‌فرد: ۱) \textbf{انقلاب مردمی غیرمنتظره} (دسامبر ۲۰۱۰)، ۲) \textbf{گفت‌وگوی ملی} میان اسلام‌گرایان و سکولارها (چهارگانهٔ نوبل ۲۰۱۵)، ۳) \textbf{قانون اساسی مترقی ۲۰۱۴} (شامل مادهٔ ۴۶: برابری جنسیتی)، ۴) \textbf{نهاد حقیقت و کرامت} (\lr{IVD})، و ۵) متأسفانه \textbf{بازگشت اقتدارگرایی} پس از کودتای قیس سعید (۲۰۲۱). تونس هم \textbf{الگوی امیدبخش} (۲۰۱۱-۲۰۱۹) و هم \textbf{هشدار جدی} (۲۰۲۱+) برای ایران است. ارتباط ویژه با ایران: نزدیکی فرهنگی-منطقه‌ای (خاورمیانه/شمال آفریقا)، رابطهٔ دین و سیاست، جوانی جمعیت، و چالش اقتصادی پس از گذار.
\end{executivesummary}

%═══════════════════════════════════════════════════════════
\section{زمینه و بافت تاریخی}
\label{app:tunisia:context}
%═══════════════════════════════════════════════════════════

\subsection{رژیم بن‌علی: اقتدارگرایی مدرن}

\person{زین‌العابدین بن‌علی}{\lr{Ben Ali}} در کودتای «پزشکی» ۱۹۸۷ به قدرت رسید و تا ۲۰۱۱ حکومت کرد. رژیم او ترکیبی بود از:

\begin{enumerate}[nosep]
\item \textbf{سکولاریسم اقتدارگرا:} سرکوب همزمان اسلام‌گرایان (\lr{Ennahda}) و چپ‌ها
\item \textbf{لیبرالیسم اقتصادی ظاهری:} رشد اقتصادی اما فساد خانوادگی (خاندان طرابلسی)
\item \textbf{دولت پلیسی:} ۱۸۰,۰۰۰ نیروی امنیتی برای ۱۱ میلیون جمعیت (نسبت بسیار بالا)
\item \textbf{کنترل رسانه و اینترنت:} سانسور گسترده (لقب «عمّار ۴۰۴»)
\item \textbf{نمای دموکراتیک:} انتخابات فرمایشی با ۹۰٪+ رأی
\end{enumerate}

\begin{table}[htbp]
\centering
\caption{مشخصات تونس در آستانهٔ انقلاب (۲۰۱۰)}
\label{tab:app-tunisia-profile}
\begin{tabularx}{\textwidth}{>{\raggedleft\arraybackslash}p{4.5cm} >{\raggedleft\arraybackslash}X}
\toprule
\headerrow \textbf{شاخص} & \textbf{مقدار} \\
\midrule
جمعیت & ۱۰.۶ میلیون نفر \\
\altrow مساحت & ۱۶۳,۰۰۰ \lr{km²} \\
تنوع قومی & پایین (عرب-بربر) \\
\altrow \lr{GDP per capita} & $\sim$\$۴,۲۰۰ \\
طول عمر رژیم بن‌علی & ۲۳ سال (۱۹۸۷-۲۰۱۱) \\
\altrow نرخ بیکاری جوانان & $\sim$۳۰٪ \\
نیروهای امنیتی & $\sim$۱۸۰,۰۰۰ (پلیس + حرس ملی) \\
\altrow ارتش & $\sim$۵۰,۰۰۰ (حرفه‌ای، غیرسیاسی) \\
سطح سواد & ۸۲٪ \\
\altrow جامعهٔ مدنی & نسبتاً فعال (اتحادیهٔ کارگری \lr{UGTT} قوی) \\
\bottomrule
\end{tabularx}
\end{table}

\begin{casestudy}
\textbf{مقایسهٔ بن‌علی و جمهوری اسلامی:} مشابهت‌ها: دولت پلیسی + سانسور + فساد خانوادگی + جوانی جمعیت + بیکاری بالا + انتخابات فرمایشی. تفاوت‌ها: ۱) بن‌علی ایدئولوژی مذهبی نداشت (سکولار بود)؛ ۲) ارتش تونس \textbf{غیرسیاسی و کوچک} بود (برخلاف سپاه)؛ ۳) تونس فاقد منابع نفتی بود (وابستگی به گردشگری)؛ ۴) جمعیت بسیار کمتر (۱۱M vs ۸۵M)؛ ۵) تونس فاقد بُعد هسته‌ای بود. مهم‌ترین مشابهت: \textbf{رابطهٔ اسلام‌گرایان و سکولارها} — چالشی که تونس (موقتاً) حل کرد اما ایران حل نکرده است.
\end{casestudy}

\sectiondivider

%═══════════════════════════════════════════════════════════
\section{انقلاب یاسمن: از خودسوزی تا سقوط}
\label{app:tunisia:revolution}
%═══════════════════════════════════════════════════════════

\subsection{گاه‌شمار انقلاب (دسامبر ۲۰۱۰ — ژانویه ۲۰۱۱)}

\begin{table}[htbp]
\centering
\caption{گاه‌شمار انقلاب تونس (۲۸ روز تا سقوط)}
\label{tab:app-tunisia-revolution-timeline}
\begin{tabularx}{\textwidth}{>{\centering\arraybackslash}p{3cm} >{\raggedleft\arraybackslash}X >{\centering\arraybackslash}p{2cm}}
\toprule
\headerrow \textbf{تاریخ} & \textbf{رویداد} & \textbf{شدت} \\
\midrule
۱۷ دسامبر ۲۰۱۰ & خودسوزی \person{محمد بوعزیزی}{\lr{Mohamed Bouazizi}} در سیدی‌بوزید & نقطهٔ آغاز \\
\altrow ۱۸-۲۴ دسامبر & اعتراضات در شهرهای مرکزی (سیدی‌بوزید، قصرین، تالا) & \riskmedium \\
۲۵ دسامبر-۳ ژانویه & گسترش به شهرهای بزرگ‌تر + سرکوب خشن (۲۰+ کشته) & \riskhigh \\
\altrow ۴ ژانویه ۲۰۱۱ & مرگ بوعزیزی → تشدید اعتراضات & نقطهٔ عطف \\
۸-۱۰ ژانویه & اعتصاب عمومی \lr{UGTT} + اعتراضات سراسری & \riskhigh \\
\altrow ۱۱-۱۲ ژانویه & ارتش از شلیک به مردم \textbf{امتناع} کرد (ژنرال \person{رشید عمّار}{\lr{Rachid Ammar}}) & تعیین‌کننده \\
۱۳ ژانویه & بن‌علی سخنرانی آخر: «فهمیدم» (\lr{Fhemtkom}) & ناکافی \\
\altrow \textbf{۱۴ ژانویه ۲۰۱۱} & \textbf{فرار بن‌علی به عربستان} — سقوط رژیم ۲۸ روزه & \cmark \\
\bottomrule
\end{tabularx}
\end{table}

\subsection{نقش تعیین‌کنندهٔ ارتش}

\begin{keypoint}
\textbf{مهم‌ترین درس تونس:} ارتش \textbf{از شلیک به مردم امتناع کرد}. ژنرال \person{رشید عمّار}{\lr{Rachid Ammar}} فرمان بن‌علی را اجرا نکرد و بدین ترتیب سرنوشت انقلاب را تعیین کرد. دلایل: ۱) ارتش تونس \textbf{حرفه‌ای و غیرسیاسی} بود (برخلاف مصر و لیبی)؛ ۲) ارتش فاقد منافع اقتصادی بود؛ ۳) وفاداری به «ملت» نه «رژیم». \emphorange{سؤال کلیدی ایرانی:} آیا بخشی از سپاه حاضر است از شلیک امتناع کند؟ بدون «لحظهٔ عمّار»، سناریوی \lr{C} (انقلاب مردمی) ایران بسیار خشونت‌بارتر خواهد بود (\seeChapter{ch:scenarios}).
\end{keypoint}

\subsection{نقش فضای مجازی و رسانه}

تونس اولین «انقلاب فیسبوکی» نام گرفت:

\begin{itemize}[nosep]
\item \textbf{فیسبوک:} ۲ میلیون کاربر (از ۱۱M جمعیت) — انتشار ویدیوهای سرکوب
\item \textbf{الجزیره:} پوشش زنده اعتراضات — شکستن انحصار خبری رژیم
\item \textbf{بلاگرها:} شبکهٔ وبلاگ‌نویسان (\person{لینا بن مهنی}{\lr{Lina Ben Mhenni}}) مستندسازی کردند
\item \textbf{ویکی‌لیکس:} کابل‌های سفارت آمریکا دربارهٔ فساد خاندان طرابلسی فاش شد
\end{itemize}

\begin{lessonlearned}
\textbf{فضای مجازی سلاح دولبه است:} تونس نشان داد که شبکه‌های اجتماعی می‌توانند \textbf{بسیج سریع} و \textbf{شکستن سانسور} ایجاد کنند. اما همان ابزار بعداً برای \textbf{اطلاعات نادرست} و \textbf{قطبی‌سازی} نیز استفاده شد. \emphorange{درس ایرانی:} ۱) استارلینک و VPN برای دور زدن فیلترینگ حیاتی‌اند؛ ۲) اما باید مکانیزم \textbf{راستی‌آزمایی} و \textbf{ضد اطلاعات نادرست} از ابتدا طراحی شود (\seeChapter{ch:timeline}).
\end{lessonlearned}

\sectiondivider

%═══════════════════════════════════════════════════════════
\section{گفت‌وگوی ملی: چهارگانهٔ نوبل}
\label{app:tunisia:dialogue}
%═══════════════════════════════════════════════════════════

\subsection{بحران ۲۰۱۳: آستانهٔ شکست}

در ۲۰۱۳، تونس در آستانهٔ سقوط به مدل مصری (کودتای نظامی) یا لیبیایی (جنگ داخلی) بود:

\begin{itemize}[nosep]
\item \textbf{ترور دو رهبر سکولار:} \person{شکری بلعید}{\lr{Chokri Belaïd}} (فوریه) و \person{محمد البراهمی}{\lr{Mohamed Brahmi}} (ژوئیه)
\item \textbf{قطبی‌شدن شدید:} اسلام‌گرایان (\lr{Ennahda}) در مقابل سکولارها
\item \textbf{بحران اقتصادی:} بیکاری + تورم + کاهش گردشگری
\item \textbf{مدل مصر:} کودتای سیسی (ژوئیه ۲۰۱۳) الگوی خطرناکی ایجاد کرد
\item \textbf{اعتراضات خیابانی:} مردم خواهان انحلال مجلس مؤسسان شدند
\end{itemize}

\subsection{چهارگانهٔ گفت‌وگوی ملی (\lr{Quartet})}

در بحرانی‌ترین لحظه، چهار سازمان جامعهٔ مدنی تونس نقش \textbf{میانجی} ایفا کردند و «گفت‌وگوی ملی» (\lr{Dialogue National}) را سازماندهی کردند:

\begin{table}[htbp]
\centering
\caption{چهارگانهٔ گفت‌وگوی ملی تونس (برندهٔ نوبل صلح ۲۰۱۵)}
\label{tab:app-tunisia-quartet}
\begin{tabularx}{\textwidth}{>{\raggedleft\arraybackslash}p{3.5cm} >{\raggedleft\arraybackslash}X >{\raggedleft\arraybackslash}p{3.5cm}}
\toprule
\headerrow \textbf{سازمان} & \textbf{نقش} & \textbf{معادل ایرانی احتمالی} \\
\midrule
\org{اتحادیهٔ عمومی کارگران تونس}{\lr{UGTT}} & بزرگ‌ترین تشکل مدنی (۷۰۰K عضو) — میانجی اصلی & کانون‌های صنفی + تشکل‌های کارگری \\
\altrow \org{اتحادیهٔ صنعت و تجارت}{\lr{UTICA}} & نمایندهٔ بخش خصوصی — اعتمادسازی اقتصادی & اتاق بازرگانی + بخش خصوصی \\
\org{انجمن حقوق بشر تونس}{\lr{LTDH}} & نظارت حقوق بشری — مشروعیت‌بخشی & کانون وکلا + فعالان حقوق بشر \\
\altrow \org{کانون وکلای تونس}{\lr{ONAT}} & تضمین حقوقی — چارچوب قانونی & کانون وکلای دادگستری \\
\bottomrule
\end{tabularx}
\end{table}

\subsection{دستاوردهای گفت‌وگوی ملی}

\begin{enumerate}[nosep]
\item \textbf{استعفای دولت \lr{Ennahda}:} حزب اسلام‌گرای حاکم داوطلبانه قدرت را واگذار کرد — بی‌سابقه در جهان عرب
\item \textbf{تشکیل دولت تکنوکرات:} دولت بی‌طرف \person{مهدی جمعه}{\lr{Mehdi Jomaa}} برای مدیریت انتقال
\item \textbf{تکمیل قانون اساسی:} مجلس مؤسسان قانون اساسی ۲۰۱۴ را تصویب کرد
\item \textbf{جلوگیری از مدل مصری:} تونس کودتا را تجربه نکرد
\item \textbf{جایزهٔ نوبل صلح ۲۰۱۵:} به چهارگانه اعطا شد
\end{enumerate}

\begin{keypoint}
\textbf{نوآوری تونس: نقش جامعهٔ مدنی به‌عنوان میانجی.} نه دولت، نه ارتش، نه بازیگران خارجی — بلکه \textbf{جامعهٔ مدنی سازمان‌یافته} بحران را حل کرد. \lr{UGTT} با ۷۰۰,۰۰۰ عضو، اعتبار کافی برای نشاندن دو طرف پشت میز مذاکره داشت. \emphgreen{کاربرد ایرانی:} آیا تشکل‌های مدنی ایران (کانون وکلا، انجمن‌های صنفی، شبکه‌های زنان) می‌توانند نقش مشابهی ایفا کنند؟ چالش: سطح سازمان‌یافتگی جامعهٔ مدنی ایران پایین‌تر از تونس است (\seeChapter{ch:actors}).
\end{keypoint}

\begin{recommendation}
\textbf{مدل «چهارگانهٔ ایرانی» (پیشنهاد):} بر اساس الگوی تونس، تشکیل ائتلاف میانجی‌گر از: ۱) \textbf{تشکل‌های کارگری و صنفی} (معلمان، کارگران، پرستاران)؛ ۲) \textbf{اتاق بازرگانی و بخش خصوصی}؛ ۳) \textbf{کانون وکلای دادگستری}؛ ۴) \textbf{شبکهٔ زنان ایران} (جنبش «زن، زندگی، آزادی»). این چهارگانه می‌تواند در فاز ۱-۲ نقش میانجی بین جریان‌های سیاسی ایفا کند.
\end{recommendation}

\sectiondivider

%═══════════════════════════════════════════════════════════
\section{قانون اساسی ۲۰۱۴: پیشرفته‌ترین در جهان عرب}
\label{app:tunisia:constitution}
%═══════════════════════════════════════════════════════════

\subsection{فرآیند تدوین}

\begin{table}[htbp]
\centering
\caption{فرآیند تدوین قانون اساسی تونس (۲۰۱۱-۲۰۱۴)}
\label{tab:app-tunisia-constitution}
\begin{tabularx}{\textwidth}{>{\centering\arraybackslash}p{2.5cm} >{\raggedleft\arraybackslash}X}
\toprule
\headerrow \textbf{مرحله} & \textbf{جزئیات} \\
\midrule
اکتبر ۲۰۱۱ & انتخابات مجلس مؤسسان (\lr{ANC}): ۲۱۷ نماینده، مشارکت ۵۲٪ \\
\altrow ۲۰۱۲-۲۰۱۳ & تدوین پیش‌نویس‌های متعدد + مشاوره عمومی + بحران ۲۰۱۳ \\
ژانویه ۲۰۱۴ & تصویب نهایی: ۲۰۰ رأی موافق از ۲۱۶ (۹۳٪) \\
\altrow ویژگی مهم & اجماع بین اسلام‌گرایان و سکولارها (نه رأی اکثریت ساده) \\
\bottomrule
\end{tabularx}
\end{table}

\subsection{نوآوری‌های قانون اساسی ۲۰۱۴}

\begin{enumerate}[nosep]
\item \textbf{مادهٔ ۱:} «تونس دولتی آزاد، مستقل و با حاکمیت است؛ اسلام دین آن، عربی زبان آن و جمهوری نظام آن است» — ترکیب هویت اسلامی با جمهوریت (بدون شریعت)
\item \textbf{مادهٔ ۲:} «تونس دولتی مدنی (\lr{État civil}) است، مبتنی بر شهروندی، ارادهٔ مردم و برتری قانون» — مهم‌ترین ماده: \textbf{«دولت مدنی»} نه دینی و نه نظامی
\item \textbf{مادهٔ ۶:} آزادی وجدان و اعتقاد + حمایت از مقدسات + ممنوعیت تکفیر — \textbf{سازش خلاقانه} بین اسلام‌گرایان و سکولارها
\item \textbf{مادهٔ ۴۶:} «دولت حقوق زنان و برابری فرصت‌ها را تضمین می‌کند... و تلاش می‌کند نمایندگی برابر (\lr{parité}) در مجالس منتخب تحقق یابد» — پیشرفته‌ترین مادهٔ جنسیتی در جهان عرب
\item \textbf{مادهٔ ۴۹:} محدودیت حقوق فقط به‌موجب قانون و با رعایت تناسب — اصل تناسب
\item \textbf{نظام نیمه‌ریاستی:} تقسیم قدرت بین رئیس‌جمهور و نخست‌وزیر
\item \textbf{تمرکززدایی:} فصل هفتم: حکومت‌های محلی منتخب
\end{enumerate}

\begin{lessonlearned}
\textbf{سازش \lr{Ennahda}:} \person{راشد غنوشی}{\lr{Rached Ghannouchi}}، رهبر \lr{Ennahda}، تصمیم تاریخی گرفت: ۱) «شریعت» را به‌عنوان منبع قانون‌گذاری مطالبه نکرد؛ ۲) «دولت مدنی» را پذیرفت؛ ۳) قدرت را داوطلبانه واگذار کرد. این بی‌سابقه در تاریخ اسلام‌گرایی بود. \emphgreen{درس ایرانی:} آیا بخشی از روحانیت ایران حاضر است «مدل غنوشی» را بپذیرد — یعنی دین بدون ادعای حکمرانی؟ این یکی از کلیدی‌ترین سؤالات فاز پیش‌گذار است.
\end{lessonlearned}

\sectiondivider

%═══════════════════════════════════════════════════════════
\section{نهاد حقیقت و کرامت (\lr{IVD})}
\label{app:tunisia:ivd}
%═══════════════════════════════════════════════════════════

\subsection{ساختار و مأموریت}

\org{نهاد حقیقت و کرامت}{\lr{Instance Vérité et Dignité (IVD)}} در ژوئن ۲۰۱۴ تشکیل شد و تا دسامبر ۲۰۱۸ فعالیت کرد:

\begin{table}[htbp]
\centering
\caption{ساختار و آمار \lr{IVD} تونس}
\label{tab:app-tunisia-ivd}
\begin{tabularx}{\textwidth}{>{\raggedleft\arraybackslash}p{5cm} >{\raggedleft\arraybackslash}X}
\toprule
\headerrow \textbf{شاخص} & \textbf{جزئیات} \\
\midrule
رئیس & \person{سهام بن‌سدرین}{\lr{Sihem Bensedrine}} \\
\altrow تعداد اعضای هیئت‌رئیسه & ۱۵ کمیسیونر \\
بازهٔ زمانی پوشش & ۱ ژوئیه ۱۹۵۵ تا ۳۱ دسامبر ۲۰۱۳ (۵۸ سال!) \\
\altrow تعداد پرونده‌های دریافتی & ۶۲,۷۲۰ پرونده \\
جلسات علنی & ۱۴ جلسه (برخی پخش زندهٔ تلویزیونی) \\
\altrow دوایر تخصصی & شکنجه، ناپدیدشدگی، خشونت جنسی، فساد مالی، تبعیض منطقه‌ای \\
نوآوری ویژه & \textbf{بُعد اقتصادی:} فساد مالی و تبعیض اقتصادی منطقه‌ای بررسی شد \\
\altrow بودجه & $\sim$\$۱۵ میلیون \\
گزارش نهایی & ۲۰۱۹ (۵+ جلد، ۸۰,۰۰۰+ صفحه) \\
\bottomrule
\end{tabularx}
\end{table}

\subsection{نوآوری‌های \lr{IVD} نسبت به \lr{TRC}}

\begin{enumerate}[nosep]
\item \textbf{بُعد اقتصادی:} برخلاف \lr{TRC} آفریقای جنوبی که فقط خشونت فیزیکی را بررسی کرد، \lr{IVD} \textbf{فساد مالی و تبعیض اقتصادی} را نیز پوشش داد
\item \textbf{بازهٔ زمانی بلند:} ۵۸ سال (از استقلال تا ۲۰۱۳) — نه فقط دورهٔ بن‌علی
\item \textbf{حقوق زنان:} کمیسیون ویژهٔ خشونت جنسیتی
\item \textbf{تبعیض منطقه‌ای:} توجه به محرومیت مناطق مرکزی و جنوبی (ریشهٔ انقلاب)
\item \textbf{«آشتی اقتصادی»:} مکانیزم بازپرداخت مالی توسط فاسدان (بدون محاکمه)
\end{enumerate}

\begin{keypoint}
\textbf{ارزش نوآوری اقتصادی \lr{IVD} برای ایران:} در ایران، فساد اقتصادی (سپاه، بنیادها، خودی‌ها) \textbf{ابعاد بسیار بزرگ‌تری} از خشونت فیزیکی دارد. کمیسیون حقیقت ایران \textbf{باید} بُعد اقتصادی داشته باشد: مستندسازی فساد سیستماتیک + تبعیض منطقه‌ای (بلوچستان، کردستان، خوزستان) + مصادرهٔ اموال + قاچاق سازمان‌یافته. مدل \lr{IVD} در این بُعد از \lr{TRC} بهتر است.
\end{keypoint}

\sectiondivider

%═══════════════════════════════════════════════════════════
\section{بازگشت اقتدارگرایی: کودتای قیس سعید (۲۰۲۱)}
\label{app:tunisia:backsliding}
%═══════════════════════════════════════════════════════════

\subsection{چگونه دموکراسی شکست خورد}

در ۲۵ ژوئیه ۲۰۲۱، \person{قیس سعید}{\lr{Kais Saied}} رئیس‌جمهور تونس، پارلمان را تعلیق و قدرت را قبضه کرد. مراحل بازگشت:

\begin{table}[htbp]
\centering
\caption{مراحل بازگشت اقتدارگرایی در تونس (۲۰۱۹-۲۰۲۳)}
\label{tab:app-tunisia-backsliding}
\begin{tabularx}{\textwidth}{>{\centering\arraybackslash}p{2.5cm} >{\raggedleft\arraybackslash}X >{\centering\arraybackslash}p{2cm}}
\toprule
\headerrow \textbf{تاریخ} & \textbf{رویداد} & \textbf{وخامت} \\
\midrule
اکتبر ۲۰۱۹ & انتخاب قیس سعید (پوپولیست ضدنظام) با ۷۳٪ & \statuswarn \\
\altrow ۲۰۲۰ & بحران کووید + بحران اقتصادی + بن‌بست سیاسی & \statuswarn \\
ژوئیه ۲۰۲۱ & \textbf{تعلیق پارلمان + اخراج نخست‌وزیر} (مادهٔ ۸۰) & \statusbad \\
\altrow سپتامبر ۲۰۲۱ & تعلیق قانون اساسی ۲۰۱۴ + حکومت فردی & \statusbad \\
مارس ۲۰۲۲ & انحلال شورای عالی قضایی & \statusbad \\
\altrow ژوئیه ۲۰۲۲ & رفراندوم قانون اساسی جدید (مشارکت ۳۰٪: مشروعیت پایین) & \statusbad \\
فوریه ۲۰۲۳ & بازداشت رهبران اپوزیسیون + غنوشی & \statusbad \\
\altrow ۲۰۲۳-۲۰۲۴ & \lr{Freedom House}: «غیرآزاد» — بازگشت کامل & \statusbad \\
\bottomrule
\end{tabularx}
\end{table}

\subsection{علل شکست دموکراسی تونس}

\begin{warningbox}
\textbf{هفت علت شکست دموکراسی تونس — هشدار حیاتی برای ایران:}

\begin{enumerate}[nosep]
\item \textbf{شکست اقتصادی:} دموکراسی نتوانست وضع معیشتی مردم را بهبود دهد. بیکاری جوانان همچنان ۳۵٪+. انقلاب برای «نان و کرامت» بود اما فقط «کرامت سیاسی» آمد.

\item \textbf{بن‌بست سیاسی:} نظام نیمه‌ریاستی ناکارآمد بود. رئیس‌جمهور و نخست‌وزیر دائماً در تضاد بودند.

\item \textbf{فساد پابرجا:} نخبگان رژیم پیشین در اقتصاد باقی ماندند. اصلاحات ضدفساد ناکافی بود.

\item \textbf{خستگی دموکراتیک (\lr{Democratic Fatigue}):} مردم از بی‌ثباتی، چندپارچگی حزبی و جدال‌های سیاسی خسته شدند.

\item \textbf{پوپولیسم:} قیس سعید با شعار «ضد نظام» و «پاکدستی» رأی آورد اما استبداد آورد.

\item \textbf{ضعف جامعهٔ مدنی:} همان \lr{UGTT} که در ۲۰۱۳ ناجی بود، در ۲۰۲۱ ضعیف و منفعل شد.

\item \textbf{فقدان حمایت بین‌المللی مؤثر:} اتحادیهٔ اروپا و آمریکا واکنش ضعیفی نشان دادند.
\end{enumerate}
\end{warningbox}

\sectiondivider

%═══════════════════════════════════════════════════════════
\section{ماتریس درس‌آموخته‌ها برای ایران: الگو + هشدار}
\label{app:tunisia:lessons}
%═══════════════════════════════════════════════════════════

\begin{table}[htbp]
\centering
\caption{ماتریس انتقال درس‌آموخته‌های تونس به ایران (الگو + هشدار)}
\label{tab:app-tunisia-lessons}
\begin{tabularx}{\textwidth}{
  >{\raggedleft\arraybackslash}p{2.2cm}
  >{\raggedleft\arraybackslash}p{3.5cm}
  >{\raggedleft\arraybackslash}X
  >{\centering\arraybackslash}p{1.3cm}
  >{\centering\arraybackslash}p{1cm}
}
\toprule
\headerrow \textbf{بُعد} & \textbf{درس تونس} & \textbf{کاربرد ایرانی} & \textbf{نوع} & \textbf{اهمیت} \\
\midrule
انقلاب مردمی & سرعت (۲۸ روز) + غیرمنتظره & آمادگی برای سناریوی \lr{A/C} & الگو & \rating{4} \\
\altrow
نقش ارتش & امتناع از شلیک = سقوط & آیا بخشی از سپاه جدا می‌شود؟ & الگو & \rating{5} \\
گفت‌وگوی ملی & چهارگانه + اجماع & «چهارگانهٔ ایرانی» & الگو & \rating{5} \\
\altrow
سازش اسلام‌گرایان & غنوشی قدرت واگذار کرد & مدل «دین بدون حکمرانی» & الگو & \rating{5} \\
قانون اساسی & مجلس مؤسسان + اجماع ۹۳٪ & مجلس مؤسسان فراگیر & الگو & \rating{5} \\
\altrow
برابری جنسیتی & مادهٔ ۴۶: \lr{parité} & سهمیهٔ ۳۰٪+ زنان & الگو & \rating{5} \\
\lr{IVD} & بُعد اقتصادی عدالت انتقالی & کمیسیون حقیقت با بُعد مالی & الگو & \rating{5} \\
\altrow
فضای مجازی & فیسبوک = بسیج سریع & استارلینک + ضد اطلاعات نادرست & الگو/هشدار & \rating{4} \\
\midrule
شکست اقتصادی & بیکاری → خستگی → بازگشت & اقتصاد = مشروعیت: اولویت اول & \cellred{هشدار} & \rating{5} \\
\altrow
پوپولیسم & قیس سعید = «ناجی» → مستبد & ضد پوپولیسم: نهادسازی & \cellred{هشدار} & \rating{5} \\
بن‌بست نهادی & نظام نیمه‌ریاستی ناکارآمد & طراحی دقیق نظام سیاسی & \cellred{هشدار} & \rating{4} \\
\altrow
ضعف جامعهٔ مدنی & \lr{UGTT} ۲۰۱۳ vs ۲۰۲۱ & تقویت مستمر جامعهٔ مدنی & \cellred{هشدار} & \rating{4} \\
فقدان حمایت بین‌المللی & واکنش ضعیف \lr{EU/US} & تضمین حمایت بلندمدت & \cellred{هشدار} & \rating{4} \\
\midrule
\headerrow \multicolumn{3}{l}{\textbf{میانگین اهمیت}} & & \textbf{\rating{5}} \\
\bottomrule
\end{tabularx}
\end{table}

\sectiondivider

%═══════════════════════════════════════════════════════════
\section{نمودار: دو مسیر تونس — الگو و هشدار}
\label{app:tunisia:diagram}
%═══════════════════════════════════════════════════════════

\begin{figure}[htbp]
\centering
\begin{tikzpicture}[
  node distance=0.8cm,
  phase/.style={
    draw, rounded corners=5pt, minimum width=2.8cm,
    minimum height=1.2cm, font=\small, align=center,
    thick
  },
  good/.style={phase, fill=MainGreen!15, draw=MainGreen!60},
  bad/.style={phase, fill=MainRed!15, draw=MainRed!60},
  neutral/.style={phase, fill=MainYellow!15, draw=MainYellow!80!black},
  arrow/.style={->, thick, >=stealth},
  fork/.style={->, very thick, >=stealth, dashed}
]

% خط بالا: مسیر موفق (۲۰۱۱-۲۰۱۴)
\node[good] (rev) at (0,3) {انقلاب\\ژانویه ۲۰۱۱};
\node[good] (election) at (3.5,3) {انتخابات\\مؤسسان ۲۰۱۱};
\node[neutral] (crisis) at (7,3) {بحران\\۲۰۱۳};
\node[good] (dialogue) at (10.5,3) {گفت‌وگوی\\ملی ۲۰۱۳};
\node[good] (const) at (14,3) {قانون اساسی\\۲۰۱۴ \cmark};

\draw[arrow, MainGreen] (rev) -- (election);
\draw[arrow, MainYellow!80!black] (election) -- (crisis);
\draw[arrow, MainGreen] (crisis) -- (dialogue);
\draw[arrow, MainGreen] (dialogue) -- (const);

% عنوان
\node[font=\small\bfseries, MainGreen] at (7,4.5) {مسیر موفق: الگو برای ایران};

% خط پایین: مسیر شکست (۲۰۱۹-۲۰۲۳)
\node[neutral] (elect19) at (0,-1) {انتخابات\\۲۰۱۹};
\node[neutral] (saied) at (3.5,-1) {قیس سعید\\«ناجی»};
\node[bad] (econ) at (7,-1) {بحران\\اقتصادی};
\node[bad] (coup) at (10.5,-1) {کودتا\\ژوئیه ۲۰۲۱};
\node[bad] (auth) at (14,-1) {بازگشت\\اقتدارگرایی \xmark};

\draw[arrow, MainYellow!80!black] (elect19) -- (saied);
\draw[arrow, MainRed] (saied) -- (econ);
\draw[arrow, MainRed] (econ) -- (coup);
\draw[arrow, MainRed] (coup) -- (auth);

% عنوان
\node[font=\small\bfseries, MainRed] at (7,-2.5) {مسیر شکست: هشدار برای ایران};

% نقطهٔ انشعاب
\draw[fork, MainPurple] (const.south) -- ++(0,-0.8) -- (elect19.north);
\node[font=\tiny, MainPurple, anchor=east] at (1.5,1) {۵ سال بعد...};

% ایران
\node[draw=MainPurple, fill=MainPurple!10, rounded corners=3pt,
  minimum width=3.5cm, minimum height=1cm, font=\small\bfseries,
  align=center, thick] (iran) at (7,0.7) {ایران باید از هر دو بیاموزد};

\draw[fork, MainGreen] (iran.north) -- ++(0,0.5);
\draw[fork, MainRed] (iran.south) -- ++(0,-0.5);

\end{tikzpicture}
\caption{دو مسیر تونس: الگوی موفقیت (۲۰۱۱-۲۰۱۴) و هشدار شکست (۲۰۱۹-۲۰۲۳)}
\label{fig:app-tunisia-paths}
\end{figure}

\sectiondivider

%═══════════════════════════════════════════════════════════
\section{جمع‌بندی پیوست}
\label{app:tunisia:conclusion}
%═══════════════════════════════════════════════════════════

\begin{chaptersummary}
جمع‌بندی پیوست ت — مطالعهٔ موردی تونس:

\begin{enumerate}[nosep]
\item تونس \textbf{هم الگو و هم هشدار} برای ایران است: الگوی موفقیت (۲۰۱۱-۲۰۱۴) + هشدار شکست (۲۰۲۱+).
\item \textbf{گفت‌وگوی ملی تونس} (چهارگانهٔ نوبل) مهم‌ترین الگوی قابل‌انتقال: جامعهٔ مدنی به‌عنوان میانجی.
\item \textbf{سازش \lr{Ennahda}} بی‌سابقه بود: حزب اسلام‌گرا داوطلبانه قدرت را واگذار کرد و «دولت مدنی» را پذیرفت — مدل «دین بدون حکمرانی».
\item \textbf{قانون اساسی ۲۰۱۴} پیشرفته‌ترین در جهان عرب بود: مادهٔ ۲ (دولت مدنی) + مادهٔ ۴۶ (برابری جنسیتی) + مادهٔ ۶ (آزادی وجدان).
\item \textbf{\lr{IVD}} نوآوری ارزشمندی داشت: بُعد اقتصادی عدالت انتقالی — مستقیماً به ایران قابل‌انتقال.
\item \textbf{هفت علت شکست} تونس مهم‌ترین هشدارهای ایران هستند: شکست اقتصادی، پوپولیسم، بن‌بست نهادی، خستگی دموکراتیک، ضعف جامعهٔ مدنی، فساد پابرجا، و فقدان حمایت بین‌المللی.
\item \textbf{مهم‌ترین درس:} دموکراسی بدون \textbf{بهبود اقتصادی} پایدار نیست. اگر مردم ایران در ۵ سال اول پس از گذار بهبود معیشتی نبینند، خطر «قیس سعید ایرانی» واقعی است.
\item مدل ایران باید از مسیر اول تونس الهام بگیرد (گفت‌وگو + قانون اساسی فراگیر) و از مسیر دوم (شکست اقتصادی + پوپولیسم) اجتناب کند.
\end{enumerate}

\vspace{0.3cm}
\textit{مطالعهٔ تکمیلی:}
\begin{itemize}[nosep]
\item مقایسهٔ جامع ۹ نمونه: \seeChapter{app:comparison}
\item آفریقای جنوبی (\lr{TRC}): \seeChapter{app:south-africa}
\item شیلی (عدالت تدریجی): \seeChapter{app:chile}
\item ریسک بازگشت اقتدارگرایی: \seeChapter{ch:risks}
\item بودجه‌بندی و تأمین اقتصادی: \seeChapter{ch:budget}
\item لهستان و اروپای شرقی: \seeChapter{app:poland}
\end{itemize}
\end{chaptersummary}

\chapterend

%══════════════════════════════════════════════════════════════
% پایان پیوست ت
%══════════════════════════════════════════════════════════════