% ╔══════════════════════════════════════════════════════════════════╗
% ║  فصل ۵: نهادها، بازیگران، سازمان‌ها و نقش هر یک              ║
% ║  نقشه‌ی جامع بازیگران و تحلیل نقش‌ها                          ║
% ╚══════════════════════════════════════════════════════════════════╝

% ---- صفحه‌ی آغازین فصل ----
\chapteropening{۵}
    {نهادها، بازیگران، سازمان‌ها و نقش هر یک}
    {MainOrange}
    {در سیاست بین‌الملل، هیچ‌کس بی‌طرف نیست. 
    هنر آن است که منافع متعارض بازیگران را 
    به‌گونه‌ای مدیریت کنیم که نتیجه به نفع 
    مردم ایران باشد.}
    {داگ هامرشولد، دبیرکل سابق سازمان ملل}

\chapter{نهادها، بازیگران، سازمان‌ها و نقش هر یک}
\label{ch:actors}
\minitoc

% ---- خلاصه‌ی اجرایی ----
\begin{executivesummary}
گذار دموکراتیک ایران در خلأ رخ نمی‌دهد. 
ده‌ها نهاد بین‌المللی، دولت، سازمان غیردولتی، 
رسانه و بازیگر داخلی هر یک نقش، منافع 
و ظرفیت‌های متفاوتی دارند. این فصل 
\emphorange{نقشه‌ی جامع بازیگران} را ترسیم 
و نقش، اهمیت، محدودیت‌ها و نوع تعامل مطلوب 
با هر یک را تحلیل می‌کند. هدف آن است که 
هم بازیگران ایرانی بدانند از چه کسی 
چه انتظاری داشته باشند، و هم بازیگران 
بین‌المللی بدانند نقش‌شان چیست و 
مرز آن کجاست.
\end{executivesummary}

% ============================================================
\section{نقشه‌ی کلان بازیگران}
\label{sec:actors-map}
% ============================================================

بازیگران مرتبط با فرایند نظارت بر گذار 
ایران را در نُه دسته سازمان‌دهی می‌کنیم:

\begin{figure}[htbp]
    \centering
    \begin{tikzpicture}[
        ring/.style={
            draw=#1, fill=#1!8,
            rounded corners=3pt,
            minimum height=1cm, minimum width=3cm,
            align=center, font=\footnotesize\bfseries
        },
        center/.style={
            draw=MainPurple, fill=PurpleBG,
            circle, minimum size=2.2cm,
            align=center, font=\small\bfseries
        },
        conn/.style={thick, #1!50}
    ]
    
    % مرکز
    \node[center] (c) at (0,0) {گذار\\ایران};
    
    % ۹ دسته
    \node[ring=MainBlue] (un) at (90:4cm) 
        {۱. سازمان ملل};
    \node[ring=MainBlue] (reg) at (50:4cm) 
        {۲. سازمان‌های منطقه‌ای};
    \node[ring=MainOrange] (gov) at (10:4cm) 
        {۳. دولت‌های کلیدی};
    \node[ring=MainGreen] (ngo) at (330:4cm) 
        {۴. \lr{NGO}های بین‌المللی};
    \node[ring=MainRed] (media) at (290:4cm) 
        {۵. رسانه‌ها};
    \node[ring=DarkYellow] (ifi) at (250:4cm) 
        {۶. نهادهای مالی};
    \node[ring=MainPurple] (iran) at (210:4cm) 
        {۷. بازیگران ایرانی};
    \node[ring=DarkGray] (legal) at (170:4cm) 
        {۸. نهادهای حقوقی};
    \node[ring=MediumGray] (acad) at (130:4cm) 
        {۹. آکادمیا و اتاق فکر};
    
    % اتصالات
    \foreach \n/\col in {
        un/MainBlue, reg/MainBlue, gov/MainOrange,
        ngo/MainGreen, media/MainRed, ifi/DarkYellow,
        iran/MainPurple, legal/DarkGray, acad/MediumGray
    } {
        \draw[conn=\col] (c) -- (\n);
    }
    
    \end{tikzpicture}
    \caption{نقشه‌ی کلان نُه دسته‌ی بازیگران 
    مرتبط با نظارت بر گذار ایران}
    \label{fig:actors-map}
\end{figure}

\sectiondivider

% ============================================================
\section{دسته‌ی ۱: سازمان ملل متحد}
\label{sec:actors-un}
% ============================================================

سازمان ملل \emphblue{محوری‌ترین نهاد} 
در هر فرایند نظارت بین‌المللی است. 
اما سازمان ملل یک نهاد واحد نیست — 
مجموعه‌ای پیچیده از ارگان‌ها با 
اختیارات و فرهنگ‌های سازمانی متفاوت.

\begin{table}[htbp]
    \centering
    \caption{نهادهای کلیدی سازمان ملل 
    و نقش هر یک در گذار ایران}
    \label{tab:un-bodies}
    \tablefontsize
    \begin{tabularx}{\textwidth}{
        L{2.2cm} X C{1.8cm} L{2.5cm}
    }
        \toprule
        \headerrow
        \textbf{نهاد} & 
        \textbf{نقش احتمالی} & 
        \textbf{اهمیت} &
        \textbf{چالش اصلی} \\
        \midrule
        
        شورای امنیت &
        صدور قطعنامه، تأسیس مأموریت 
        نظارتی، تعیین مأموریت \lr{SRSG} &
        \cellred{\textbf{حیاتی}} &
        وتوی روسیه و چین \\
        \altrow
        
        مجمع عمومی &
        مشروعیت‌بخشی سیاسی، 
        تصویب قطعنامه‌ی غیرالزامی &
        مهم &
        فاقد قدرت اجرایی \\
        
        دبیرکل / \lr{SRSG} &
        میانجی‌گری، هماهنگی کل فرایند، 
        چهره‌ی عمومی مأموریت &
        \cellred{\textbf{حیاتی}} &
        انتخاب شخص مناسب \\
        \altrow
        
        \lr{UNDP} &
        ظرفیت‌سازی نهادی، طراحی 
        سیستم انتخاباتی، حکمرانی محلی &
        بالا &
        بروکراسی کند \\
        
        \lr{OHCHR} &
        نظارت بر حقوق بشر، مستندسازی 
        نقض‌ها، گزارش‌دهی مستقل &
        بالا &
        فشار سیاسی دولت‌ها \\
        \altrow
        
        \lr{DPPA} &
        تحلیل سیاسی، پیشگیری از خشونت، 
        هشدار زودهنگام &
        بالا &
        محدودیت نیرو \\
        
        \lr{UN Women} &
        تضمین حقوق زنان در قانون اساسی 
        و نهادهای جدید &
        مهم &
        مقاومت فرهنگی \\
        \altrow
        
        \lr{UNHCR} &
        مدیریت بازگشت پناهندگان و 
        آوارگان داخلی &
        مهم (فاز ۲-۳) &
        مقیاس بزرگ \\
        
        \lr{UNICEF} &
        حفاظت از کودکان در دوره‌ی 
        بی‌ثباتی &
        مهم &
        دسترسی \\
        \altrow
        
        \lr{IAEA} &
        نظارت بر تأسیسات هسته‌ای 
        در دوره‌ی گذار &
        \cellred{\textbf{حیاتی}} &
        حساسیت امنیتی \\
        
        \bottomrule
    \end{tabularx}
\end{table}

\subsection{چالش وتو در شورای امنیت}

\begin{warningbox}
\textbf{بزرگ‌ترین مانع نهادی:}
روسیه و چین هر دو روابط استراتژیک 
با جمهوری اسلامی دارند و احتمالاً 
هرگونه قطعنامه‌ی شورای امنیت علیه 
ایران یا برای تأسیس مأموریت نظارتی 
را وتو خواهند کرد.

\vspace{4pt}
\textbf{راه‌حل‌های جایگزین:}
\begin{enumerate}[itemsep=2pt, font=\small]
    \item \textbf{قطعنامه‌ی مجمع عمومی:} 
    غیرالزامی اما مشروعیت‌بخش 
    (\lr{Uniting for Peace})
    \item \textbf{دعوت دولت موقت ایران:} 
    اگر دولت جدید خودش دعوت کند، 
    نیازی به قطعنامه نیست
    \item \textbf{ائتلاف اختیاری:} 
    \lr{Coalition of the Willing} 
    مشابه گروه تماس بالکان
    \item \textbf{مکانیزم EU:} 
    اتحادیه اروپا مستقلاً 
    مأموریت نظارتی اعزام کند
    \item \textbf{فشار بر روسیه و چین:} 
    بسته‌ی تشویقی (حفظ قراردادهای 
    اقتصادی در ایران جدید)
\end{enumerate}
\end{warningbox}

\subsection{انتخاب نماینده‌ی ویژه‌ی دبیرکل}

\begin{keypoint}
انتخاب \lr{SRSG} مناسب یکی از 
تعیین‌کننده‌ترین تصمیمات فرایند است. 
پروفایل ایده‌آل:

\begin{table}[H]
    \tablefontsize
    \begin{tabularx}{\textwidth}{L{3cm} X}
        \toprule
        \headerrow
        \textbf{ویژگی} & \textbf{توضیح} \\
        \midrule
        ملیت & 
        ترجیحاً نه آمریکایی، نه اروپایی 
        غربی، نه روسی/چینی — 
        شخصیتی از جنوب جهانی 
        (آمریکای لاتین، آفریقا، 
        آسیای جنوب‌شرقی) \\
        \altrow
        تجربه & 
        سابقه‌ی میانجی‌گری، مدیریت 
        بحران، رهبری مأموریت UN \\
        زبان & 
        آشنایی با فارسی مزیت بزرگ 
        (اما الزامی نیست) \\
        \altrow
        شخصیت & 
        صبور، شنونده، قاطع اما 
        انعطاف‌پذیر \\
        شناخت منطقه & 
        درک ژئوپلیتیک خاورمیانه \\
        \altrow
        جنسیت & 
        انتخاب یک زن بسیار نمادین 
        خواهد بود (جنبش زن، زندگی، آزادی) \\
        \bottomrule
    \end{tabularx}
\end{table}
\end{keypoint}

\begin{lessonlearned}
\textbf{از تجربه‌ی \person{سرجیو ویئرا دملو}
{Sergio Vieira de Mello} در تیمور شرقی:}
دملو (برزیلی) به‌عنوان \lr{SRSG} در تیمور 
شرقی (۱۹۹۹-۲۰۰۲) یکی از موفق‌ترین 
مدیریت‌های گذار را رهبری کرد. 
رمز موفقیتش: \emphblue{شنیدن صدای مردم 
محلی + قاطعیت در تصمیمات امنیتی + 
شفافیت + برنامه‌ی مشخص خروج.} 
متأسفانه او در حمله به دفتر UN 
در بغداد (۲۰۰۳) کشته شد — 
یادآوری تلخ ریسک‌هایی که 
ناظران بین‌المللی با آن مواجه‌اند.
\end{lessonlearned}

\sectiondivider

% ============================================================
\section{دسته‌ی ۲: سازمان‌های منطقه‌ای}
\label{sec:actors-regional}
% ============================================================

\begin{table}[htbp]
    \centering
    \caption{سازمان‌های منطقه‌ای و 
    نقش احتمالی در گذار ایران}
    \label{tab:regional-orgs}
    \tablefontsize
    \begin{tabularx}{\textwidth}{
        L{2cm} X C{1.5cm} L{2.5cm}
    }
        \toprule
        \headerrow
        \textbf{سازمان} & 
        \textbf{نقش احتمالی} & 
        \textbf{اعتبار} &
        \textbf{محدودیت} \\
        \midrule
        
        اتحادیه اروپا &
        نظارت انتخاباتی (\lr{EU EOM})، 
        حمایت مالی، مشاوره حقوقی، 
        رفع تحریم‌ها مشروط &
        \starrating{4} &
        کندی تصمیم‌گیری، سیاست 
        داخلی اعضا \\
        \altrow
        
        شورای اروپا &
        \lr{Venice Commission} 
        برای قانون اساسی، 
        استانداردهای حقوق بشر &
        \starrating{5} &
        ایران عضو نیست 
        (اما مشاوره ممکن) \\
        
        \lr{OSCE} &
        نظارت انتخاباتی 
        (\lr{ODIHR})، 
        رسانه آزاد &
        \starrating{5} &
        ایران عضو نیست \\
        \altrow
        
        اتحادیه عرب &
        مشروعیت منطقه‌ای، 
        کاهش تنش‌های 
        ایران-عرب &
        \starrating{2} &
        ضعف ساختاری، 
        تعارض منافع \\
        
        \lr{OIC} &
        مشروعیت اسلامی، 
        پل فرهنگی &
        \starrating{2} &
        سیاسی‌زدگی، 
        ناکارآمدی \\
        \altrow
        
        \lr{SCO} &
        مدیریت نقش 
        روسیه و چین &
        \starrating{1} &
        ایران تازه عضو شده، 
        سازمان اقتدارگرامحور \\
        
        \bottomrule
    \end{tabularx}
\end{table}

\begin{recommendation}
\textbf{استراتژی پیشنهادی برای سازمان‌های منطقه‌ای:}
\begin{itemize}[itemsep=3pt]
    \item \textbf{EU:} شریک اصلی — 
    از ظرفیت‌های \lr{EU EOM}، 
    \lr{Venice Commission} و 
    ابزارهای مالی بهره ببرید
    \item \textbf{OSCE:} حتی بدون 
    عضویت ایران، می‌تواند 
    استانداردها و ناظران 
    ارائه دهد
    \item \textbf{اتحادیه عرب و OIC:} 
    نقش نمادین — برای کاهش 
    روایت «غرب علیه ایران»
    \item \textbf{SCO:} مدیریت کنید 
    نه نادیده بگیرید — کانال 
    ارتباط با روسیه و چین
\end{itemize}
\end{recommendation}

\sectiondivider

% ============================================================
\section{دسته‌ی ۳: دولت‌های کلیدی}
\label{sec:actors-states}
% ============================================================

دولت‌ها مهم‌ترین بازیگران واقعی 
(نه رسمی) هستند. هر دولت منافع 
خاص خود را دارد و نقشش باید 
با درک این منافع مدیریت شود.

\begin{landscape}
\begin{table}[htbp]
    \centering
    \caption{دولت‌های کلیدی: نقش، منافع، 
    ریسک و اهرم}
    \label{tab:key-states}
    \bigtablefontsize
    \setlength{\tabcolsep}{3pt}
    \begin{tabularx}{\linewidth}{
        L{1.8cm} X X L{2cm} L{2cm}
    }
        \toprule
        \headerrow
        \textbf{دولت} & 
        \textbf{منافع اصلی} & 
        \textbf{نقش احتمالی} &
        \textbf{ریسک/محدودیت} &
        \textbf{اهرم ایران} \\
        \midrule
        
        آمریکا &
        هسته‌ای، امنیت اسرائیل، 
        نفوذ منطقه‌ای، حقوق بشر &
        بزرگ‌ترین اهرم فشار و 
        حمایت مالی، رفع تحریم &
        بی‌اعتمادی تاریخی 
        ایرانیان، ابزاری شدن 
        دموکراسی &
        تحریم‌ها، 
        دارایی‌های بلوکه \\
        \altrow
        
        آلمان/فرانسه &
        ثبات منطقه، مهاجرت، 
        تجارت، هسته‌ای &
        میانجی قابل اعتمادتر، 
        حمایت مالی-فنی EU &
        کندی تصمیم‌گیری، 
        وابستگی انرژی &
        دیپلماسی، 
        بازار \\
        
        بریتانیا &
        نفوذ پسااستعماری، 
        حقوق بین‌الملل &
        تجربه حقوقی، 
        BBC فارسی &
        سابقه‌ی استعماری 
        در ذهنیت ایرانیان &
        حقوق بین‌الملل، 
        رسانه \\
        \altrow
        
        ژاپن/کره‌جنوبی &
        ثبات انرژی، 
        تجارت &
        کمک مالی بی‌طرف، 
        تجربه بازسازی &
        نفوذ سیاسی محدود &
        کم‌ریسک‌ترین 
        کمک‌کننده \\
        
        ترکیه &
        مرز مشترک، 
        رقابت منطقه‌ای، 
        کُردها &
        همسایگی، فهم فرهنگی نسبی، 
        مدیریت مرز &
        تعارض منافع، 
        اردوغان &
        مرز، تجارت، 
        کُردها \\
        \altrow
        
        روسیه &
        حفظ نفوذ، فروش سلاح، 
        ضد آمریکایی‌گری &
        \cellred{اسپویلر بالقوه} — 
        وتو در شورای امنیت &
        مانع اصلی در UN &
        باید مدیریت شود \\
        
        چین &
        نفت، قرارداد ۲۵ ساله، 
        جاده ابریشم &
        \cellred{اسپویلر بالقوه} — 
        وتو + نفوذ اقتصادی &
        منافع اقتصادی قوی‌تر 
        از ایدئولوژی &
        تضمین ادامه 
        قراردادها \\
        \altrow
        
        هند &
        بندر چابهار، انرژی، 
        توازن منطقه‌ای &
        بازیگر متوازن، 
        میانجی احتمالی &
        محافظه‌کاری ذاتی &
        چابهار، 
        روابط تاریخی \\
        
        عربستان/امارات &
        پایان تهدید ایران، 
        ثبات خلیج فارس &
        مالی و منطقه‌ای، 
        مشروعیت عربی &
        \cellred{تلاش برای 
        نفوذ بیش از حد} &
        اقتصادی، 
        منطقه‌ای \\
        \altrow
        
        اسرائیل &
        هسته‌ای، امنیت، 
        پایان تهدید &
        اطلاعاتی، فنی 
        (هسته‌ای) &
        \cellred{حضور علنی 
        سمّی برای مشروعیت} &
        پشت‌پرده، 
        نه علنی \\
        
        \bottomrule
    \end{tabularx}
\end{table}
\end{landscape}

\begin{keypoint}
\textbf{اصل طلایی مدیریت دولت‌ها:}
\begin{itemize}[itemsep=3pt]
    \item \emphpurple{هیچ دولتی 
    بی‌طرف نیست} — همه منافع دارند
    \item \emphpurple{اسپویلرها را 
    مدیریت کنید نه نادیده بگیرید} — 
    روسیه و چین را با بسته‌ی 
    تشویقی همراه کنید
    \item \emphpurple{تنوع شرکا} — 
    وابستگی به یک دولت خطرناک است
    \item \emphpurple{حضور علنی 
    برخی بازیگران سمّی است} — 
    نقش اسرائیل باید پشت‌پرده باشد
    \item \emphpurple{ایرانیان حساس‌اند} — 
    هرگونه نقش آمریکایی باید با 
    ظرافت مدیریت شود
\end{itemize}
\end{keypoint}

\sectiondivider

% ============================================================
\section{دسته‌ی ۴: سازمان‌های غیردولتی بین‌المللی}
\label{sec:actors-ngos}
% ============================================================

\lr{NGO}ها نقش‌های تخصصی و اغلب 
غیرقابل جایگزین در فرایند نظارت دارند. 
اما برخی از آن‌ها در ایران 
حساسیت‌برانگیز هستند.

\begin{table}[htbp]
    \centering
    \caption{\lr{NGO}های بین‌المللی کلیدی 
    و نقش هر یک}
    \label{tab:ngos}
    \tablefontsize
    \begin{tabularx}{\textwidth}{
        L{2cm} X C{1.5cm} C{1.5cm}
    }
        \toprule
        \headerrow
        \textbf{سازمان} & 
        \textbf{تخصص و نقش} & 
        \textbf{اعتبار} &
        \textbf{حساسیت ایران} \\
        \midrule
        
        \lr{Carter Center} &
        نظارت انتخاباتی مستقل و 
        معتبر — ۳۹ کشور &
        \starrating{5} &
        \cellgreen{کم} \\
        \altrow
        
        \lr{ICG} &
        تحلیل بحران، هشدار 
        زودهنگام، توصیه سیاستی &
        \starrating{4} &
        \cellgreen{کم} \\
        
        \lr{HRW} &
        مستندسازی نقض حقوق بشر، 
        فشار عمومی &
        \starrating{4} &
        \cellorange{متوسط} \\
        \altrow
        
        \lr{Amnesty Intl.} &
        مشابه \lr{HRW} — 
        تمرکز بر زندانیان &
        \starrating{4} &
        \cellorange{متوسط} \\
        
        \lr{Transparency Intl.} &
        ضد فساد — نظارت بر 
        شفافیت دوره گذار &
        \starrating{4} &
        \cellgreen{کم} \\
        \altrow
        
        \lr{Intl. IDEA} &
        طراحی نظام انتخاباتی، 
        دموکراسی‌سنجی &
        \starrating{4} &
        \cellgreen{کم} \\
        
        \lr{IFES} &
        زیرساخت فنی انتخابات، 
        ثبت رأی‌دهندگان &
        \starrating{4} &
        \cellgreen{کم} \\
        \altrow
        
        \lr{ICTJ} &
        عدالت انتقالی، طراحی 
        کمیسیون حقیقت &
        \starrating{5} &
        \cellgreen{کم} \\
        
        \lr{NED} &
        حمایت مالی از نهادهای 
        دموکراتیک &
        \starrating{3} &
        \cellred{بالا — تأمین 
        مالی دولت آمریکا} \\
        \altrow
        
        \lr{OSF (Soros)} &
        جامعه باز، حمایت مالی، 
        شبکه‌سازی &
        \starrating{3} &
        \cellred{بالا — 
        تئوری توطئه} \\
        
        \lr{NDI / IRI} &
        آموزش احزاب و 
        نهادهای سیاسی &
        \starrating{3} &
        \cellred{بالا — وابسته 
        به احزاب آمریکایی} \\
        
        \bottomrule
    \end{tabularx}
\end{table}

\begin{warningbox}
\textbf{مدیریت حساسیت \lr{NGO}ها:}
سازمان‌هایی مانند \lr{NED}، \lr{OSF} و 
\lr{NDI/IRI} در ایران (و حتی در 
بخش‌هایی از اپوزیسیون) 
\emphred{حساسیت‌برانگیز} هستند — 
به دلیل ارتباط با دولت آمریکا یا 
روایت‌های توطئه‌ای.

\textbf{راه‌حل:}
\begin{itemize}[itemsep=2pt]
    \item نقش عملیاتی آن‌ها 
    \textbf{پشت‌صحنه} باشد
    \item تأمین مالی از طریق 
    \textbf{صندوق‌های چندجانبه} 
    (نه مستقیم) کانالیزه شود
    \item از \textbf{نهادهای کم‌حساسیت‌تر} 
    (مانند \lr{Carter Center}، \lr{ICTJ}، 
    \lr{IDEA}) به‌عنوان چهره‌ی 
    عمومی استفاده شود
\end{itemize}
\end{warningbox}

\sectiondivider

% ============================================================
\section{دسته‌ی ۵: رسانه‌ها}
\label{sec:actors-media}
% ============================================================

رسانه‌ها در دوره‌ی گذار سه نقش حیاتی دارند: 
\textbf{اطلاع‌رسانی}، \textbf{نظارت شهروندی} 
و \textbf{شکل‌دهی روایت عمومی}.

\begin{table}[htbp]
    \centering
    \caption{رسانه‌های کلیدی و نقش‌شان 
    در گذار}
    \label{tab:media-actors}
    \tablefontsize
    \begin{tabularx}{\textwidth}{
        L{2.2cm} X L{3cm}
    }
        \toprule
        \headerrow
        \textbf{رسانه} & 
        \textbf{نقش و نفوذ} & 
        \textbf{ملاحظات} \\
        \midrule
        
        \lr{BBC} فارسی &
        بالاترین نفوذ در میان 
        رسانه‌های بین‌المللی فارسی‌زبان، 
        اعتماد نسبی مخاطبان &
        باید استقلال تحریریه حفظ شود — 
        ابزار دولت بریتانیا نباشد \\
        \altrow
        
        صدای آمریکا / فردا &
        پوشش خبری + تحلیل، 
        دسترسی گسترده &
        سابقه‌ی دولتی — 
        اعتماد کمتر از \lr{BBC} \\
        
        \lr{DW} فارسی &
        اطلاع‌رسانی متوازن، 
        اعتماد نسبی &
        منابع محدودتر \\
        \altrow
        
        ایران‌اینترنشنال &
        پربیننده‌ترین رسانه 
        فارسی‌زبان ماهواره‌ای &
        تأمین مالی عربستان → 
        چالش استقلال \\
        
        رسانه‌های مستقل 
        ایرانی آنلاین &
        \lr{IranWire}، \lr{Iranians.com} 
        و ده‌ها رسانه‌ی کوچک &
        پراکنده اما اصیل — 
        نیاز به حمایت مالی \\
        \altrow
        
        شبکه‌های اجتماعی &
        تلگرام، اینستاگرام، 
        توییتر/\lr{X} — 
        ابزار بسیج و نظارت &
        اطلاعات نادرست، 
        نیاز به سواد رسانه‌ای \\
        
        \bottomrule
    \end{tabularx}
\end{table}

\begin{operationalnote}
\textbf{اقدام فوری پیشنهادی:}
تأسیس یک \emphgreen{«هاب اطلاعاتی گذار»} 
(\lr{Transition Information Hub}) — 
پلتفرم آنلاین چندزبانه (فارسی + کردی + 
ترکی + عربی + بلوچی + انگلیسی) 
برای انتشار اطلاعات موثق درباره‌ی 
فرایند گذار، مقابله با اطلاعات نادرست 
و ارائه‌ی آموزش مدنی. مدیریت مشترک 
توسط نهاد بین‌المللی + رسانه‌های 
مستقل ایرانی.
\end{operationalnote}

\sectiondivider

% ============================================================
\section{دسته‌ی ۶: نهادهای مالی بین‌المللی}
\label{sec:actors-ifi}
% ============================================================

\begin{table}[htbp]
    \centering
    \caption{نهادهای مالی بین‌المللی و 
    نقش‌شان}
    \label{tab:ifis}
    \tablefontsize
    \begin{tabularx}{\textwidth}{
        L{2cm} X L{3cm}
    }
        \toprule
        \headerrow
        \textbf{نهاد} & 
        \textbf{نقش} & 
        \textbf{زمان ورود} \\
        \midrule
        
        \lr{IMF} &
        تثبیت اقتصاد کلان، مشاوره 
        ارزی و مالی، وام اضطراری &
        فاز ۱ — فوری \\
        \altrow
        
        \lr{World Bank} &
        بازسازی زیرساخت، 
        توسعه نهادی، 
        کاهش فقر &
        فاز ۲ — میان‌مدت \\
        
        \lr{EBRD} &
        حمایت از بخش خصوصی، 
        اصلاحات اقتصادی &
        فاز ۲-۳ \\
        \altrow
        
        \lr{ADB} &
        زیرساخت منطقه‌ای، 
        انرژی، حمل‌ونقل &
        فاز ۲-۳ \\
        
        \lr{AIIB} &
        زیرساخت — 
        کانال ارتباط با چین &
        فاز ۲-۳ \\
        
        \bottomrule
    \end{tabularx}
\end{table}

\sectiondivider

% ============================================================
\section{دسته‌ی ۷: بازیگران ایرانی}
\label{sec:actors-iranian}
% ============================================================

\begin{keypoint}
مهم‌ترین دسته‌ی بازیگران، خود ایرانیان 
هستند. اصل «مالکیت ملی» یعنی ایرانیان 
باید در مرکز فرایند باشند — 
نه حاشیه‌ی آن.
\end{keypoint}

\begin{table}[htbp]
    \centering
    \caption{بازیگران ایرانی کلیدی}
    \label{tab:iranian-actors}
    \tablefontsize
    \begin{tabularx}{\textwidth}{
        L{2.5cm} X L{2.5cm}
    }
        \toprule
        \headerrow
        \textbf{بازیگر} & 
        \textbf{نقش در گذار} & 
        \textbf{چالش اصلی} \\
        \midrule
        
        جامعه مدنی داخل &
        پایه‌ی اصلی نظارت شهروندی، 
        مشارکت در نهادهای گذار &
        سرکوب‌شدگی، 
        ضعف سازمانی \\
        \altrow
        
        جنبش زنان &
        پیشران اصلی تغییر 
        (زن، زندگی، آزادی)، 
        تضمین حقوق در قانون اساسی &
        مقاومت سنتی، 
        نیاز به نهادسازی \\
        
        جنبش‌های قومی &
        نمایندگی تنوع، مذاکره 
        خودمختاری/فدرالیسم &
        ریسک تجزیه‌طلبی، 
        مسلح بودن برخی \\
        \altrow
        
        جنبش کارگری &
        نمایندگی طبقاتی، 
        اعتصاب به‌عنوان اهرم &
        سرکوب‌شدگی شدید \\
        
        اپوزیسیون سازمان‌یافته &
        طرف مذاکره، طراحی 
        نهادهای جدید &
        پراکندگی شدید، 
        تضادهای جناحی \\
        \altrow
        
        دیاسپورا &
        منابع مالی-فنی-دیپلماتیک، 
        لابی بین‌المللی &
        فاصله از داخل، 
        تضاد نسلی \\
        
        روحانیت مستقل &
        مشروعیت‌زدایی از ولایت فقیه، 
        پل به جامعه سنتی &
        ضعف نهادی، 
        تعارض درونی \\
        \altrow
        
        نخبگان فکری &
        روایت‌سازی، طراحی 
        گفتمان دموکراتیک &
        تبعید/زندان/سکوت \\
        
        \bottomrule
    \end{tabularx}
\end{table}

\begin{casestudy}{نقش جامعه مدنی تونسی 
در گذار}
چهار سازمان مدنی تونسی در ۲۰۱۳ 
به‌عنوان «چهارتای گفت‌وگوی ملی» 
(\lr{Tunisian National Dialogue Quartet}) 
میانجی‌گری بحران سیاسی را بر عهده گرفتند 
و مانع از تکرار سناریوی مصر شدند. 
در ۲۰۱۵ جایزه نوبل صلح دریافت کردند.

\vspace{4pt}
\emphblue{درس: جامعه مدنی قوی می‌تواند 
در لحظات بحرانی نقش نجات‌بخش ایفا کند. 
تقویت جامعه مدنی ایران — حتی در شرایط 
سرکوب — سرمایه‌گذاری حیاتی برای 
آینده است.}
\end{casestudy}

\sectiondivider

% ============================================================
\section{نقشه‌ی تعامل: چه کسی با چه کسی 
چگونه کار می‌کند}
\label{sec:interaction-map}
% ============================================================

\begin{table}[htbp]
    \centering
    \caption{نوع تعامل مطلوب با هر 
    دسته‌ی بازیگران}
    \label{tab:interaction-types}
    \tablefontsize
    \begin{tabularx}{\textwidth}{
        L{2.5cm} L{2.5cm} X
    }
        \toprule
        \headerrow
        \textbf{دسته} & 
        \textbf{نوع تعامل} & 
        \textbf{ابزار اصلی} \\
        \midrule
        
        نهادهای UN &
        رسمی-دیپلماتیک &
        قطعنامه، نشست، 
        تفاهم‌نامه فنی \\
        \altrow
        
        دولت‌های کلیدی &
        دیپلماسی دو/چندجانبه &
        مذاکره، بسته تشویقی، 
        فشار/اهرم \\
        
        \lr{NGO}ها &
        قرارداد پروژه‌ای &
        \lr{MOU}، تأمین مالی 
        از صندوق مشترک \\
        \altrow
        
        رسانه‌ها &
        شفافیت + دسترسی &
        نشست خبری، اسناد عمومی، 
        هاب اطلاعاتی \\
        
        نهادهای مالی &
        برنامه‌ی مشترک &
        وام مشروط، 
        کمک فنی \\
        \altrow
        
        بازیگران ایرانی &
        مشارکت ساختاریافته &
        شورای مشورتی ملی، 
        انتخابات، رفراندوم \\
        
        آکادمیا &
        سفارش تحقیق + مشاوره &
        گرنت، کنفرانس، 
        بررسی همتا \\
        
        \bottomrule
    \end{tabularx}
\end{table}

\sectiondivider

% ============================================================
\section{جمع‌بندی فصل}
\label{sec:ch5-summary}
% ============================================================

\begin{chaptersummary}

\textbf{آنچه در این فصل آموختیم:}

\begin{enumerate}[
    label=\textcolor{DarkGray}{\bfseries\arabic*.},
    itemsep=4pt
]
    \item \textbf{نُه دسته‌ی بازیگر} شناسایی و 
    تحلیل شد: از سازمان ملل تا جامعه مدنی 
    ایرانی.
    
    \item \textbf{سازمان ملل محوری‌ترین نهاد} 
    است اما چالش وتوی روسیه و چین باید 
    با راه‌حل‌های جایگزین مدیریت شود.
    
    \item \textbf{انتخاب \lr{SRSG} مناسب} 
    یکی از تعیین‌کننده‌ترین تصمیمات است — 
    ترجیحاً شخصیتی از جنوب جهانی.
    
    \item \textbf{هیچ دولتی بی‌طرف نیست} — 
    مدیریت منافع متعارض دولت‌ها کلید 
    موفقیت است. اسپویلرها (روسیه، چین) 
    باید مدیریت شوند نه نادیده گرفته.
    
    \item \textbf{برخی \lr{NGO}ها حساسیت‌برانگیز}اند — 
    از نهادهای کم‌حساسیت‌تر به‌عنوان 
    چهره‌ی عمومی استفاده شود.
    
    \item \textbf{رسانه‌ها اکسیژن دموکراسی}اند — 
    تأسیس هاب اطلاعاتی گذار توصیه شد.
    
    \item \textbf{بازیگران ایرانی در مرکز فرایند} 
    باید باشند — نه حاشیه. جامعه مدنی، 
    جنبش زنان، اقوام و دیاسپورا 
    همه نقش‌های حیاتی دارند.
    
    \item \textbf{تنوع شرکا} — وابستگی به 
    یک دولت یا نهاد واحد خطرناک است.
\end{enumerate}

\vspace{6pt}
\begin{center}
    \textcolor{MainGreen}{
        \faArrowLeft\hspace{8pt}
        \textbf{فصل بعد: تضمین‌های موفقیت 
        و پیش‌شرط‌های ساختاری}
        \hspace{8pt}\faArrowLeft
    }
\end{center}

\end{chaptersummary}

\chapterend