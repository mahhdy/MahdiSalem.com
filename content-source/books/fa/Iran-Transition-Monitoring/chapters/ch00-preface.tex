% ╔══════════════════════════════════════════════════════════════════╗
% ║  فصل ۰: پیش‌گفتار و چکیده‌ی اجرایی                              ║
% ╚══════════════════════════════════════════════════════════════════╝

% ---- صفحه‌ی آغازین فصل ----
\chapteropening{۰}
{\rl{پیش‌گفتار و چکیده‌ی اجرایی}}
    {MainPurple}
    {\rl{دموکراسی چیزی نیست که یک‌بار به دست آید و 
    برای همیشه بماند؛ باید هر روز از نو تولدش داد.}}
    {\rl{واتسلاو هاول، رئیس‌جمهور چکسلواکی و جمهوری چک}}

\chapter*{پیش‌گفتار و چکیده‌ی اجرایی}
\addcontentsline{toc}{chapter}{\rl{پیش‌گفتار و چکیده‌ی اجرایی}}
\label{ch:preface}
\minitoc

% ============================================================
\section*{چرا این کتاب؟ چرا الان؟}
\addcontentsline{toc}{section}{چرا این کتاب؟ چرا الان؟}
% ============================================================

ایران در آستانه‌ی یکی از مهم‌ترین لحظات تاریخ معاصر خود ایستاده 
است. نظامی که بیش از چهار دهه بر این کشور حکومت کرده، با 
بحران‌های هم‌زمان مشروعیت، کارآمدی و جانشینی دست‌وپنجه نرم 
می‌کند. جنبش‌های اعتراضی پی‌درپی — از جنبش سبز ۱۳۸۸ تا 
قیام آبان ۱۳۹۸ و خیزش 
\bilingual{زن، زندگی، آزادی}{Woman, Life, Freedom} 
در ۱۴۰۱ — نشان داده‌اند که جامعه‌ی ایرانی خواهان تغییر بنیادین 
است. پرسش دیگر «آیا تغییر خواهد آمد؟» نیست، بلکه «چگونه؟»، 
«چه وقت؟» و «با چه پیامدهایی؟» است.

\begin{keypoint}
تجربه‌ی تاریخی نشان می‌دهد که لحظه‌ی سقوط یک نظام اقتدارگرا 
تنها آغاز راه است. آنچه \emphpurple{پس از سقوط} اتفاق می‌افتد — 
فرایند گذار، نهادسازی و تحکیم دموکراسی — به‌مراتب دشوارتر، 
پیچیده‌تر و تعیین‌کننده‌تر از خودِ لحظه‌ی تغییر است.
\end{keypoint}

این کتاب بر یک بُعد حیاتی و اغلب نادیده‌گرفته‌شده‌ی فرایند 
گذار تمرکز دارد: 
\emphpurple{نظارت بین‌المللی}. 
نظارتی که اگر درست طراحی و اجرا شود، می‌تواند تفاوت میان یک 
گذار موفق (مانند آفریقای جنوبی) و یک فاجعه (مانند عراق یا لیبی) 
باشد.

تدوین این سند بر یک باور بنیادین استوار است: 
\emphpurple{ایرانیان باید مالک و هدایت‌کننده‌ی فرایند گذار باشند}، 
اما جامعه‌ی بین‌المللی نیز مسئولیت و ظرفیت دارد که از این فرایند 
پشتیبانی کند — نه مدیریت آن، بلکه \emph{همراهی} و 
\emph{نظارت} بر آن.

\sectiondivider

% ============================================================
\section*{این کتاب برای چه کسی نوشته شده؟}
\addcontentsline{toc}{section}{مخاطبان سند}
% ============================================================

این سند برای طیف متنوعی از مخاطبان تدوین شده و تلاش کرده 
زبانی پیدا کند که برای همه‌ی آن‌ها قابل فهم و مفید باشد:

\begin{table}[htbp]
    \centering
    \caption{مخاطبان سند و نحوه‌ی بهره‌برداری}
    \label{tab:audiences}
    \tablefontsize
    \begin{tabularx}{\textwidth}{
        L{3cm} X L{4cm}
    }
        \toprule
        \headerrow
        \textbf{مخاطب} & 
        \textbf{نیاز اصلی} & 
        \textbf{فصول کلیدی} \\
        \midrule
        نیروهای سیاسی اپوزیسیون &
        درک مدل‌های ممکن نظارت و آمادگی برای تعامل با نهادهای 
        بین‌المللی &
        فصول ۳، ۴، ۹، ۱۱ \\
        \altrow
        فعالان جامعه‌ی مدنی &
        شناخت حقوق و ابزارهای نظارتی و نقش خود در فرایند &
        فصول ۵، ۶، ۸ \\
        سیاست‌گذاران بین‌المللی &
        چارچوب تصمیم‌گیری و تخمین هزینه‌ها و ریسک‌ها &
        فصول ۳، ۷، ۱۰ \\
        \altrow
        نهادهای بین‌المللی (\lr{UN, EU}) &
        راهنمای عملیاتی و سناریوهای احتمالی &
        فصول ۴، ۸، ۹ \\
        پژوهشگران و آکادمیا &
        مبانی نظری و تحلیل تطبیقی &
        فصول ۱، ۲، ۳ + پیوست‌ها \\
        \altrow
        دیاسپورای ایرانی &
        درک نقش خود و فرصت‌های مشارکت &
        فصول ۵، ۶، ۱۱ \\
        رسانه‌ها &
        تحلیل جامع و قابل استناد &
        فصل ۰ (چکیده) + فصل ۱۲ \\
        \bottomrule
    \end{tabularx}
\end{table}

% ============================================================
\section*{روش‌شناسی}
\addcontentsline{toc}{section}{روش‌شناسی}
% ============================================================

این سند بر مبنای چهار ستون روش‌شناختی تدوین شده است:

\begin{enumerate}[label=\textcolor{MainPurple}{\arabic*.}]
    \item \textbf{مرور ادبیات نظری:} 
    بررسی جامع مکاتب فکری مطالعات گذار دموکراتیک 
    (\lr{Transitology})، از 
    \person{ساموئل هانتینگتون}{Samuel Huntington} و 
    \person{خوان لینتز}{Juan Linz} تا 
    \person{لری دایموند}{Larry Diamond} و 
    \person{توماس کاروترز}{Thomas Carothers}.
    
    \item \textbf{تحلیل تطبیقی:} 
    مطالعه‌ی عمیق ۹ نمونه‌ی تاریخی گذار 
    (آفریقای جنوبی، شیلی، لهستان، تونس، تیمور شرقی، 
    کوزوو، عراق، میانمار و لیبی) 
    و استخراج درس‌های آموخته برای ایران.
    
    \item \textbf{تحلیل نهادی:} 
    بررسی ساختار، ظرفیت و سوابق نهادهای بین‌المللی 
    (سازمان ملل، اتحادیه‌ی اروپا، سازمان‌های غیردولتی و...) 
    در زمینه‌ی نظارت بر گذار.
    
    \item \textbf{سناریوسازی:} 
    طراحی شش سناریوی محتمل گذار در ایران و تطبیق 
    مدل نظارتی مناسب با هر سناریو.
\end{enumerate}

\begin{warningbox}
\textbf{یک تذکر مهم روش‌شناختی:} 
هیچ مدل نظارتی واحدی وجود ندارد که مستقیماً برای ایران قابل 
استفاده باشد. ایران کشوری است با ویژگی‌های منحصربه‌فرد 
(جمعیت ۸۵ میلیونی، ژئوپلیتیک پیچیده، برنامه‌ی هسته‌ای، 
تنوع قومی-مذهبی، ارتش ایدئولوژیک) که هرگونه الگوبرداری 
مکانیکی از نمونه‌های دیگر را محکوم به شکست می‌کند. 
این سند نه یک نسخه‌ی آماده، بلکه یک \emphred{چارچوب 
تحلیلی} برای تصمیم‌گیری آگاهانه ارائه می‌دهد.
\end{warningbox}

\sectiondivider

% ============================================================
\section*{نقشه‌ی خوانش سند}
\addcontentsline{toc}{section}{نقشه‌ی خوانش سند}
% ============================================================

این کتاب به‌گونه‌ای طراحی شده که هم به‌صورت خطی (از ابتدا تا 
انتها) و هم به‌صورت گزینشی (مراجعه به فصول مورد نیاز) قابل 
خوانش باشد. نمودار زیر نقشه‌ی ارتباط فصول را نشان می‌دهد:

\begin{figure}[htbp]
    \centering
    \begin{tikzpicture}[
        node distance=1.2cm and 2.5cm,
        box/.style={
            draw, rounded corners=3pt,
            minimum height=0.9cm, minimum width=3.8cm,
            font=\footnotesize\bfseries, align=center
        },
        purplebox/.style={box, fill=PurpleBG, draw=MainPurple},
        bluebox/.style={box, fill=BlueBG, draw=MainBlue},
        greenbox/.style={box, fill=GreenBG, draw=MainGreen},
        orangebox/.style={box, fill=OrangeBG, draw=MainOrange},
        redbox/.style={box, fill=RedBG, draw=MainRed},
        yellowbox/.style={box, fill=YellowBG, draw=MainYellow},
        arr/.style={-{Stealth[length=2.5mm]}, thick, gray!70}
    ]
    
    % ستون چپ: مبانی
    \node[bluebox] (ch1) {ف۱: مبانی نظری};
    \node[bluebox, below=of ch1] (ch2) {ف۲: چرا ایران؟};
    
    % ستون وسط: تحلیل
    \node[greenbox, right=of ch1] (ch3) {ف۳: رویکردها};
    \node[orangebox, below=of ch3] (ch4) {ف۴: سناریوها};
    \node[orangebox, below=of ch4] (ch5) {ف۵: بازیگران};
    
    % ستون راست: عملیاتی
    \node[greenbox, right=of ch3] (ch6) {ف۶: تضمین‌ها};
    \node[redbox, below=of ch6] (ch7) {ف۷: ریسک‌ها};
    \node[yellowbox, below=of ch7] (ch8) {ف۸: نیازمندی‌ها};
    
    % پایین
    \node[yellowbox, below=3.5cm of ch5] (ch9) {ف۹: زمان‌بندی};
    \node[yellowbox, left=of ch9] (ch10) {ف۱۰: بودجه};
    \node[purplebox, right=of ch9] (ch11) {ف۱۱: نقشه‌ی راه};
    
    % فلش‌ها
    \draw[arr] (ch1) -- (ch3);
    \draw[arr] (ch2) -- (ch4);
    \draw[arr] (ch1) -- (ch2);
    \draw[arr] (ch3) -- (ch4);
    \draw[arr] (ch3) -- (ch6);
    \draw[arr] (ch4) -- (ch5);
    \draw[arr] (ch5) -- (ch7);
    \draw[arr] (ch6) -- (ch7);
    \draw[arr] (ch7) -- (ch8);
    \draw[arr] (ch8) -- (ch9);
    \draw[arr] (ch8) -- (ch10);
    \draw[arr] (ch9) -- (ch11);
    \draw[arr] (ch10) -- (ch11);
    
    \end{tikzpicture}
    \caption{نقشه‌ی ارتباط فصول کتاب}
    \label{fig:chapter-map}
\end{figure}

\sectiondivider

% ============================================================
\section*{چکیده‌ی اجرایی}
\addcontentsline{toc}{section}{چکیده‌ی اجرایی}
% ============================================================

\begin{executivesummary}

\subsection*{یافته‌های کلیدی}

\begin{enumerate}[
    label=\textcolor{MainPurple}{\large\bfseries\arabic*.},
    leftmargin=2cm,
    itemsep=8pt
]
    \item \textbf{ایران یک مورد استاندارد نیست.}
    با جمعیت ۸۵ میلیونی، برنامه‌ی هسته‌ای، ارتش ایدئولوژیک 
    (\lr{IRGC})، شبکه‌ی نیابتی منطقه‌ای و تنوع قومی-مذهبی، 
    هیچ مدل نظارتی موجودی بدون تطبیق اساسی برای ایران کار 
    نخواهد کرد. (\seeChapter{ch:why-iran})
    
    \item \textbf{شش مدل نظارتی شناسایی شده‌اند.}
    از «نظارت انتخاباتی محدود» تا «مدیریت بین‌المللی مستقیم». 
    هیچ‌یک به‌تنهایی کافی نیست. مدل پیشنهادی این سند: 
    \emphpurple{مدل ترکیبی-تطبیقی} با فازبندی سه‌مرحله‌ای. 
    (\seeChapter{ch:approaches})
    
    \item \textbf{سناریوی گذار، مدل نظارت را تعیین می‌کند.}
    شش سناریو تحلیل شده‌اند. سناریوی مطلوب: «گذار مذاکره‌ای»؛ 
    سناریوی محتمل‌تر: «فروپاشی ناگهانی» یا «انقلاب مردمی». 
    برای هر سناریو مدل نظارتی متفاوتی لازم است. 
    (\seeChapter{ch:scenarios})
    
    \item \textbf{بزرگ‌ترین ریسک: بازگشت اقتدارگرایی.}
    تجربه‌ی مصر ۲۰۱۳ نشان داد که حتی انقلاب‌های پرشکوه 
    می‌توانند به بازگشت دیکتاتوری بینجامند. مدیریت نقش سپاه 
    پاسداران، حیاتی‌ترین چالش امنیتی گذار ایران است. 
    (\seeChapter{ch:risks})
    
    \item \textbf{هزینه‌ی تخمینی: ۳ تا ۵ میلیارد دلار در ۱۰ سال.}
    رقمی که در مقایسه با هزینه‌ی بی‌ثباتی منطقه‌ای 
    (جنگ عراق: ۲ تریلیون دلار) یک سرمایه‌گذاری بسیار 
    مقرون‌به‌صرفه است. (\seeChapter{ch:budget})
    
    \item \textbf{آمادگی باید از الان شروع شود.}
    منتظر سقوط نظام ماندن خطرناک است. شبکه‌سازی، 
    آموزش ناظران، طراحی برنامه‌ی آماده‌باش و ایجاد اجماع 
    بین‌المللی باید از همین امروز آغاز شود. 
    (\seeChapter{ch:roadmap})
\end{enumerate}

\subsection*{ده توصیه‌ی کلیدی}

\begin{table}[H]
    \centering
    \tablefontsize
    \begin{tabularx}{\textwidth}{
        C{0.8cm} X L{3cm}
    }
        \toprule
        \headerrow
        \textbf{\#} & \textbf{توصیه} & \textbf{مخاطب اصلی} \\
        \midrule
        ۱ & 
        \textbf{مالکیت ملی:} ایرانیان هدایت‌کننده‌ باشند، 
        نه موضوع نظارت &
        همه \\
        \altrow
        ۲ & 
        \textbf{آماده‌باش از الان:} برنامه‌ریزی منتظر 
        سقوط نماند &
        اپوزیسیون + بین‌الملل \\
        ۳ & 
        \textbf{مدل ترکیبی:} از هیچ مدل واحدی کپی نکنید &
        طراحان فرایند \\
        \altrow
        ۴ & 
        \textbf{فازبندی:} از نظارت سنگین شروع و تدریجاً 
        کاهش دهید &
        مأموریت بین‌المللی \\
        ۵ & 
        \textbf{فراگیری:} زنان، اقوام، جوانان و همه‌ی 
        طیف‌ها نمایندگی شوند &
        شورای مشورتی ملی \\
        \altrow
        ۶ & 
        \textbf{عدالت آشتی‌محور:} عدالت انتقالی 
        نه انتقام‌جویانه &
        کمیسیون حقیقت \\
        ۷ & 
        \textbf{اقتصاد فراموش نشود:} رفع تحریم + بسته‌ی 
        حمایتی = شرط لازم &
        قدرت‌های بزرگ \\
        \altrow
        ۸ & 
        \textbf{سپاه: مدیریت نه نابودی:} بازسازی حرفه‌ای 
        نه حذف کامل &
        طراحان \lr{SSR} \\
        ۹ & 
        \textbf{رسانه‌ی آزاد = اکسیژن:} از روز اول 
        تضمین شود &
        دولت موقت \\
        \altrow
        ۱۰ & 
        \textbf{خروج شفاف:} مأموریت بین‌المللی 
        نقطه‌ی پایان داشته باشد &
        شورای امنیت \\
        \bottomrule
    \end{tabularx}
\end{table}

\end{executivesummary}

\sectiondivider

% ============================================================
\section*{ساختار کتاب}
\addcontentsline{toc}{section}{ساختار کتاب}
% ============================================================

\begin{table}[htbp]
    \centering
    \caption{نقشه‌ی فصول کتاب}
    \label{tab:book-structure}
    \tablefontsize
    \begin{tabularx}{\textwidth}{
        C{0.7cm} C{0.4cm} X C{1.5cm}
    }
        \toprule
        \headerrow
        \textbf{فصل} & & \textbf{عنوان و محتوای اصلی} & 
        \textbf{صفحات} \\
        \midrule
        
        \cellcolor{PurpleBG} ۰ &
        \cellcolor{PurpleBG} 🟣 &
        \cellcolor{PurpleBG} پیش‌گفتار و چکیده‌ی اجرایی &
        \cellcolor{PurpleBG} ۵-۷ \\
        
        \cellcolor{BlueBG} ۱ &
        \cellcolor{BlueBG} 🔵 &
        \cellcolor{BlueBG} مبانی نظری: گذار دموکراتیک و 
        نظارت بین‌المللی &
        \cellcolor{BlueBG} ۱۲-۱۵ \\
        
        \cellcolor{BlueBG} ۲ &
        \cellcolor{BlueBG} 🔵 &
        \cellcolor{BlueBG} چرا ایران؟ ویژگی‌ها، پیچیدگی‌ها 
        و استثنائات &
        \cellcolor{BlueBG} ۱۰-۱۲ \\
        
        \cellcolor{GreenBG} ۳ &
        \cellcolor{GreenBG} 🟢 &
        \cellcolor{GreenBG} رویکردها و ساختارهای نظارت — 
        تحلیل مقایسه‌ای شش مدل &
        \cellcolor{GreenBG} ۱۸-۲۲ \\
        
        \cellcolor{OrangeBG} ۴ &
        \cellcolor{OrangeBG} 🟠 &
        \cellcolor{OrangeBG} سناریوهای گذار و مدل‌های نظارتی 
        متناظر &
        \cellcolor{OrangeBG} ۱۵-۱۸ \\
        
        \cellcolor{OrangeBG} ۵ &
        \cellcolor{OrangeBG} 🟠 &
        \cellcolor{OrangeBG} نهادها، بازیگران و نقش هر یک &
        \cellcolor{OrangeBG} ۱۸-۲۲ \\
        
        \cellcolor{GreenBG} ۶ &
        \cellcolor{GreenBG} 🟢 &
        \cellcolor{GreenBG} تضمین‌های موفقیت و پیش‌شرط‌های 
        ساختاری &
        \cellcolor{GreenBG} ۱۲-۱۵ \\
        
        \cellcolor{RedBG} ۷ &
        \cellcolor{RedBG} 🔴 &
        \cellcolor{RedBG} آسیب‌شناسی، ریسک‌ها و چالش‌ها &
        \cellcolor{RedBG} ۱۵-۱۸ \\
        
        \cellcolor{YellowBG} ۸ &
        \cellcolor{YellowBG} 🟡 &
        \cellcolor{YellowBG} نیازمندی‌ها: انسانی، نهادی، فنی، 
        حقوقی &
        \cellcolor{YellowBG} ۱۲-۱۵ \\
        
        \cellcolor{YellowBG} ۹ &
        \cellcolor{YellowBG} 🟡 &
        \cellcolor{YellowBG} زمان‌بندی، تیم‌سازی و 
        ساختارسازی &
        \cellcolor{YellowBG} ۱۰-۱۲ \\
        
        \cellcolor{YellowBG} ۱۰ &
        \cellcolor{YellowBG} 🟡 &
        \cellcolor{YellowBG} بودجه‌بندی و تأمین مالی &
        \cellcolor{YellowBG} ۸-۱۰ \\
        
        \cellcolor{PurpleBG} ۱۱ &
        \cellcolor{PurpleBG} 🟣 &
        \cellcolor{PurpleBG} نقشه‌ی راه اجرایی و توصیه‌های 
        نهایی &
        \cellcolor{PurpleBG} ۱۰-۱۲ \\
        
        \cellcolor{PurpleBG} ۱۲ &
        \cellcolor{PurpleBG} 🟣 &
        \cellcolor{PurpleBG} جمع‌بندی و کلام آخر &
        \cellcolor{PurpleBG} ۳-۵ \\
        
        \bottomrule
    \end{tabularx}
\end{table}

\begin{operationalnote}
\textbf{درباره‌ی پیوست‌ها:} 
علاوه بر ۱۳ فصل اصلی، ده پیوست تکمیلی ارائه شده شامل:
\begin{itemize}
    \item \textbf{پیوست الف:} جدول مقایسه‌ای جامع ۹ نمونه‌ی 
    تاریخی گذار
    \item \textbf{پیوست‌های ب تا ح:} هفت مطالعه‌ی موردی 
    تفصیلی (آفریقای جنوبی، شیلی، تونس، لهستان، عراق، 
    میانمار، تیمور شرقی)
    \item \textbf{پیوست خ:} واژه‌نامه‌ی تخصصی دوزبانه 
    (فارسی-انگلیسی)
    \item \textbf{پیوست د:} فهرست نهادها و سازمان‌های کلیدی
\end{itemize}
\end{operationalnote}

% ============================================================
\section*{سپاس‌گزاری}
\addcontentsline{toc}{section}{سپاس‌گزاری}
% ============================================================

\begin{pullquote}
این سند حاصل مطالعه‌ی تجربه‌ی ده‌ها کشور، صدها نهاد و 
هزاران انسانی است که در تاریخ معاصر برای ساختن جوامع آزاد 
و دموکراتیک تلاش کرده‌اند. از همه‌ی آن‌ها — موفق و ناموفق — 
آموخته‌ایم. سپاس ویژه از تمامی پژوهشگران، فعالان و 
سیاست‌گذارانی که دانش خود را در دسترس عموم قرار داده‌اند.
\end{pullquote}

\vspace{12pt}

\begin{flushright}
    \textbf{مهدی سالم}\\
    تابستان ۱۴۰۴ / ژوئن ۲۰۲۵
\end{flushright}

\chapterend