% ═══════════════════════════════════════════════════════════════════════════════
% فصل ۱۱: نقشه راه اجرایی و توصیه‌های نهایی
% فایل: chapters/ch11-roadmap.tex
% رنگ فصل: بنفش (MainPurple)
% ═══════════════════════════════════════════════════════════════════════════════

\chapteropening{۱۱}{نقشه راه اجرایی و توصیه‌های نهایی}{MainPurple}{%
بهترین زمان برای کاشتن درخت بیست سال پیش بود. دومین بهترین زمان، همین الان است.%
}{ضرب‌المثل چینی}

\chapter{نقشه راه اجرایی و توصیه‌های نهایی}
\label{ch:roadmap}

\minitoc

% ─────────────────────────────────────────────────────────────────────────────
% خلاصه اجرایی
% ─────────────────────────────────────────────────────────────────────────────

\begin{executivesummary}
این فصل، عصاره عملیاتی تمام یافته‌های کتاب است. ده توصیه کلیدی به تفکیک مخاطب، شاخص‌های کمّی سنجش موفقیت، مکانیزم پایش پنج‌سطحی، استراتژی خروج تفصیلی، و جدول اقدامات فوری/میان‌مدت/بلندمدت ارائه می‌شود. هدف: هر خواننده — از رهبر اپوزیسیون تا دیپلمات سازمان ملل تا فعال مدنی ایرانی — پس از خواندن این فصل بداند \emph{دقیقاً چه باید بکند}. نقشه راه بصری در پایان فصل، تمام عناصر را در یک نمای واحد ترکیب می‌کند.
\end{executivesummary}

\section{درآمد: از تحلیل به اقدام}
\label{sec:roadmap-intro}

در فصول پیشین، مبانی نظری (فصل \ref{ch:theoretical})، ویژگی‌های ایران (فصل \ref{ch:why-iran})، مدل‌های نظارت (فصل \ref{ch:approaches})، سناریوها (فصل \ref{ch:scenarios})، بازیگران (فصل \ref{ch:actors})، تضمین‌ها (فصل \ref{ch:guarantees})، ریسک‌ها (فصل \ref{ch:risks})، نیازمندی‌ها (فصل \ref{ch:requirements})، زمان‌بندی (فصل \ref{ch:timeline})، و بودجه (فصل \ref{ch:budget}) بررسی شدند. اکنون همه این عناصر را در یک نقشه راه عملیاتی یکپارچه ترکیب می‌کنیم.

\begin{keypoint}
این فصل سه پرسش را پاسخ می‌دهد:
\begin{enumerate}[nosep]
    \item \textbf{چه باید کرد؟} — ده توصیه کلیدی
    \item \textbf{چگونه می‌فهمیم موفق شده‌ایم؟} — شاخص‌های سنجش
    \item \textbf{اگر مسیر منحرف شد چه؟} — مکانیزم اصلاح و خروج
\end{enumerate}
\end{keypoint}

\sectiondivider

% ═══════════════════════════════════════════════════════════════════════════════
\section{ده توصیه کلیدی}
\label{sec:ten-recommendations}
% ═══════════════════════════════════════════════════════════════════════════════

\subsection{توصیه ۱: مالکیت ملی — خط قرمز غیرقابل‌مذاکره}
\label{subsec:rec-national-ownership}

\begin{recommendation}
\textbf{مخاطب اصلی: همه بازیگران}

ایرانیان باید مالک فرایند گذار باشند. نظارت بین‌المللی ابزار تسهیل است، نه مدیریت. در هر سطح تصمیم‌گیری، ایرانیان باید حرف آخر را بزنند.

\textbf{اقدامات مشخص:}
\begin{itemize}[nosep]
    \item رئیس دولت انتقالی و اکثریت شورای مشورتی ایرانی باشند
    \item هر نهاد بین‌المللی معاون ایرانی داشته باشد
    \item تصمیمات استراتژیک (قانون اساسی، انتخابات) صرفاً توسط ایرانیان گرفته شود
    \item قاعده ۳۰-۵۰-۸۰ انتقال مسئولیت رعایت شود (\seeChapter{ch:timeline})
\end{itemize}
\end{recommendation}

\subsection{توصیه ۲: آماده‌باش — همین الان شروع کنید}
\label{subsec:rec-preparedness}

\begin{recommendation}
\textbf{مخاطب اصلی: سازمان ملل، اتحادیه اروپا، اپوزیسیون}

فاز ۰ (پیش‌گذار) باید \emph{الان} آغاز شود — نه وقتی رژیم سقوط کرده. تجربه تاریخی نشان می‌دهد غافلگیری = فاجعه.

\textbf{اقدامات مشخص:}
\begin{itemize}[nosep]
    \item \lr{DPPA} تیم برنامه‌ریزی ایران تشکیل دهد
    \item فهرست ۵-۱۰ نامزد \lr{SRSG} آماده شود
    \item پیش‌نویس قطعنامه شورای امنیت نوشته شود
    \item شبکه ناظران آماده‌باش فعال شود
    \item اپوزیسیون درباره ساختار دولت انتقالی توافق کند
\end{itemize}
\end{recommendation}

\subsection{توصیه ۳: مدل ترکیبی-تطبیقی — نه حداقل، نه حداکثر}
\label{subsec:rec-hybrid-model}

\begin{recommendation}
\textbf{مخاطب اصلی: شورای امنیت، SRSG}

مدل ۶ (ترکیبی-تطبیقی) ارائه‌شده در فصل \ref{ch:approaches} بهترین گزینه برای ایران است: ترکیب عناصر مدل‌های ۲ (مشورتی)، ۳ (ساختاری)، و ۴ (تضمینی) در سه فاز.

\textbf{اصول کلیدی:}
\begin{itemize}[nosep]
    \item فاز ۱: مدل ۴ غالب (نظارت تضمینی برای تثبیت)
    \item فاز ۲: مدل ۳ غالب (نظارت ساختاری برای نهادسازی)
    \item فاز ۳: مدل ۲ غالب (مشاوره برای تحکیم)
    \item مدل ۵ (مدیریت مستقیم) قاطعانه رد شود
    \item انطباق‌پذیری: مدل باید با تحولات میدانی تعدیل شود
\end{itemize}
\end{recommendation}

\subsection{توصیه ۴: فراگیری — هیچ‌کس نباید حذف شود}
\label{subsec:rec-inclusivity}

\begin{recommendation}
\textbf{مخاطب اصلی: دولت انتقالی، اپوزیسیون}

فراگیری شرط بقای گذار است. مصادره توسط یک جناح، قومیت، جنسیت، یا طبقه = شکست حتمی.

\textbf{معیارهای کمّی:}
\begin{itemize}[nosep]
    \item حداقل ۳۰٪ زنان در همه نهادها
    \item نمایندگی همه اقوام اصلی (آذری، کرد، بلوچ، عرب، ترکمن، لر) در شورا
    \item نمایندگی اقلیت‌های مذهبی (سنّی، بهایی، مسیحی، یهودی، زرتشتی)
    \item سهمیه جوانان (زیر ۳۵): حداقل ۲۰٪
    \item دیاسپورا: حداکثر ۲۰-۳۰٪ (نه بیشتر)
\end{itemize}
\end{recommendation}

\subsection{توصیه ۵: عدالت آشتی‌محور — نه انتقام، نه فراموشی}
\label{subsec:rec-justice}

\begin{recommendation}
\textbf{مخاطب اصلی: دولت انتقالی، جامعه مدنی}

\textbf{فرمول پیشنهادی:}
\begin{itemize}[nosep]
    \item \textbf{رهبران و آمران}: محاکمه در دادگاه ویژه (۱۰۰-۵۰۰ نفر)
    \item \textbf{مجریان رده‌میانی}: کمیسیون حقیقت + عفو مشروط در ازای اعتراف
    \item \textbf{اعضای عادی}: بازگشت به زندگی عادی بدون تعقیب
    \item \textbf{قربانیان}: جبران مادی + نمادین + حق شنیده شدن
    \item \textbf{ممنوعیت}: اعدام، شکنجه، مجازات جمعی
\end{itemize}
\end{recommendation}

\begin{lessonlearned}{آفریقای جنوبی در مقابل عراق}
آفریقای جنوبی با مدل آشتی (کمیسیون حقیقت + عفو مشروط) به ثبات رسید. عراق با مدل انتقامی (بعث‌زدایی افراطی) به جنگ داخلی کشیده شد. ایران باید مدل آفریقای جنوبی را الگو قرار دهد — نه عراق.
\end{lessonlearned}

\subsection{توصیه ۶: مدیریت سپاه — بزرگ‌ترین چالش}
\label{subsec:rec-irgc}

\begin{recommendation}
\textbf{مخاطب اصلی: دولت انتقالی، شورای امنیت}

سپاه پاسداران بزرگ‌ترین بازیگر مخرب بالقوه و بزرگ‌ترین چالش گذار است. استراتژی باید ترکیبی از فشار و انگیزه باشد:

\textbf{فشار:}
\begin{itemize}[nosep]
    \item ممنوعیت قانونی از فعالیت سیاسی و اقتصادی
    \item تهدید قاطع بین‌المللی به تحریم در صورت کودتا
    \item نظارت بین‌المللی بر جابجایی نیروها و سلاح
\end{itemize}

\textbf{انگیزه:}
\begin{itemize}[nosep]
    \item برنامه \lr{DDR} با بسته مالی مناسب برای افراد رده‌پایین
    \item ادغام نیروهای فنی در ارتش حرفه‌ای جدید
    \item عدم تعقیب اعضای عادی (فقط آمران جنایات)
    \item تبدیل شرکت‌های سپاه به شرکت‌های سهامی عام (نه مصادره)
\end{itemize}
\end{recommendation}

\subsection{توصیه ۷: اقتصاد — نان مقدم بر رأی}
\label{subsec:rec-economy}

\begin{recommendation}
\textbf{مخاطب اصلی: IMF، بانک جهانی، دولت انتقالی}

بدون ثبات اقتصادی، دموکراسی بقا نخواهد داشت. مردمی که نان ندارند، به صندوق رأی اعتماد نمی‌کنند.

\textbf{اقدامات فوری (ماه ۱-۶):}
\begin{itemize}[nosep]
    \item رفع تحریم‌های بین‌المللی (فوری و بدون شرط)
    \item آزادسازی دارایی‌های بلوکه‌شده
    \item بسته کمک بشردوستانه ($۲-۵B)
    \item تثبیت نرخ ارز با حمایت \lr{IMF}
\end{itemize}

\textbf{اقدامات میان‌مدت (ماه ۶-۲۴):}
\begin{itemize}[nosep]
    \item برنامه اصلاحات اقتصادی با مشاوره \lr{IMF/WB}
    \item جذب سرمایه‌گذاری خارجی
    \item شبکه ایمنی اجتماعی برای اقشار آسیب‌پذیر
\end{itemize}
\end{recommendation}

\subsection{توصیه ۸: رسانه آزاد — اکسیژن دموکراسی}
\label{subsec:rec-media}

\begin{recommendation}
\textbf{مخاطب اصلی: دولت انتقالی، رسانه‌ها}

\textbf{اقدامات:}
\begin{itemize}[nosep]
    \item رفع فیلترینگ اینترنت در روز اول
    \item آزادی فوری روزنامه‌نگاران زندانی
    \item صدور مجوز رسانه بدون سانسور
    \item کمیسیون مستقل تنظیم رسانه (نه دولتی)
    \item حمایت مالی از رسانه‌های مستقل (بدون کنترل محتوا)
    \item آموزش سواد رسانه‌ای عمومی
    \item مقابله فعال با اطلاعات نادرست (\seeChapter{ch:timeline})
\end{itemize}
\end{recommendation}

\subsection{توصیه ۹: هسته‌ای — حل مسئله، نه تشدید بحران}
\label{subsec:rec-nuclear}

\begin{recommendation}
\textbf{مخاطب اصلی: P5+1، آژانس بین‌المللی انرژی اتمی}

\textbf{اصول:}
\begin{itemize}[nosep]
    \item مسئله هسته‌ای نباید گروگان گذار شود
    \item توافق جامع جدید (جایگزین \lr{JCPOA}) در فاز ۲
    \item تضمین حق غنی‌سازی صلح‌آمیز
    \item نظارت تقویت‌شده آژانس (پروتکل الحاقی + دسترسی گسترده)
    \item رفع کامل تحریم‌های هسته‌ای در ازای شفافیت کامل
\end{itemize}
\end{recommendation}

\subsection{توصیه ۱۰: خروج شفاف — از ابتدا بدانید کی تمام می‌شود}
\label{subsec:rec-exit}

\begin{recommendation}
\textbf{مخاطب اصلی: شورای امنیت، SRSG}

\textbf{اصول خروج:}
\begin{itemize}[nosep]
    \item معیارهای خروج از روز اول اعلام شوند (شاخص‌محور نه زمان‌محور)
    \item خروج تدریجی (۵ مرحله مطابق فصل \ref{ch:timeline})
    \item انتقال واقعی مسئولیت (نه فقط کاهش حضور)
    \item ارزیابی مستقل قبل از هر مرحله خروج
    \item حفظ رابطه پس از خروج (دفتر سیاسی \lr{UN})
\end{itemize}
\end{recommendation}

\sectiondivider

% ═══════════════════════════════════════════════════════════════════════════════
\section{توصیه‌ها به تفکیک مخاطب}
\label{sec:recommendations-by-audience}
% ═══════════════════════════════════════════════════════════════════════════════

\begin{landscape}
\begin{table}[htbp]
\centering
\bigtablefontsize
\caption{ماتریس توصیه‌ها به تفکیک مخاطب و بازه زمانی}
\label{tab:recommendations-matrix}
\begin{tabularx}{\linewidth}{>{\raggedleft\arraybackslash}p{2.5cm}
                             >{\raggedleft\arraybackslash}X
                             >{\raggedleft\arraybackslash}X
                             >{\raggedleft\arraybackslash}X}
\toprule
\headerrow مخاطب & فوری (الان — ماه ۶) & میان‌مدت (ماه ۶-۲۴) & بلندمدت (ماه ۲۴-۶۰) \\
\midrule
اپوزیسیون ایرانی & توافق بر ساختار دولت انتقالی، ائتلاف فراگیر، منشور دموکراتیک & مشارکت در مجلس مؤسسان، پذیرش نتایج انتخابات & حزب‌سازی، نهادسازی، پذیرش چرخش قدرت \\
\altrow جامعه مدنی ایران & مستندسازی حقوق بشر، آموزش ناظران محلی & نظارت شهروندی، مشارکت در قانون‌نویسی & نهاد دیده‌بان، آموزش دموکراتیک \\
سازمان ملل & تشکیل تیم برنامه‌ریزی، شناسایی \lr{SRSG} & ایجاد \lr{UNMOIT}، استقرار ناظران & کاهش تدریجی، انتقال مسئولیت \\
\altrow اتحادیه اروپا & پیش‌نویس بسته حمایتی، آموزش ناظران & نظارت انتخاباتی، مشروط‌سازی کمک & مذاکرات تجاری، پیوند نهادی \\
آمریکا & رفع تحریم‌ها، آزادسازی دارایی‌ها & حمایت مالی و فنی، مذاکره هسته‌ای & عادی‌سازی روابط، سرمایه‌گذاری \\
\altrow کشورهای منطقه & عدم مداخله، حمایت از ثبات & مشارکت در گروه تماس، کمک مالی & همکاری امنیتی و اقتصادی منطقه‌ای \\
دیاسپورا & بسیج تخصص و سرمایه، پرهیز از جناح‌گرایی & مشارکت فنی، بازگشت تدریجی & ادغام، انتقال دانش \\
\altrow رسانه‌ها & پوشش مسئولانه، ضد اطلاعات نادرست & آموزش روزنامه‌نگاری حرفه‌ای & رسانه مستقل پایدار \\
\bottomrule
\end{tabularx}
\end{table}
\end{landscape}

\sectiondivider

% ═══════════════════════════════════════════════════════════════════════════════
\section{شاخص‌های سنجش موفقیت}
\label{sec:success-indicators}
% ═══════════════════════════════════════════════════════════════════════════════

\begin{table}[htbp]
\centering
\caption{نُه شاخص کمّی سنجش موفقیت گذار}
\label{tab:success-indicators}
\begin{tabularx}{\textwidth}{>{\centering\arraybackslash}p{0.5cm}
                             >{\raggedleft\arraybackslash}p{3cm}
                             >{\raggedleft\arraybackslash}X
                             >{\centering\arraybackslash}p{2cm}
                             >{\centering\arraybackslash}p{2cm}}
\toprule
\headerrow \# & شاخص & تعریف عملیاتی & هدف سال ۳ & هدف سال ۵ \\
\midrule
۱ & آزادی سیاسی & امتیاز \lr{Freedom House} (۱-۷) & ۳.۰ & ۲.۵ \\
\altrow ۲ & آزادی مطبوعات & رتبه \lr{RSF} (از ۱۸۰) & زیر ۱۰۰ & زیر ۷۰ \\
۳ & مشارکت انتخاباتی & درصد رأی‌دهندگان & ۶۰٪ & ۶۵٪ \\
\altrow ۴ & شاخص فساد & امتیاز \lr{TI CPI} (از ۱۰۰) & ۳۵ & ۴۵ \\
۵ & حقوق بشر & تعداد موارد نقض مستند/سال & کاهش ۷۰٪ & کاهش ۹۰٪ \\
\altrow ۶ & برابری جنسیتی & درصد زنان در پارلمان & ۲۵٪ & ۳۰٪ \\
۷ & رشد اقتصادی & نرخ رشد \lr{GDP} سالانه & ۳٪ & ۵٪ \\
\altrow ۸ & اعتماد عمومی & نظرسنجی اعتماد به نهادها & ۴۵٪ & ۵۵٪ \\
۹ & امنیت & تعداد حوادث امنیتی/ماه & کاهش ۸۰٪ & کاهش ۹۵٪ \\
\bottomrule
\end{tabularx}
\end{table}

\begin{warningbox}
\textbf{هشدار درباره شاخص‌ها}: شاخص‌های کمّی ابزار مفید اما ناقصی هستند. رتبه‌بندی‌های بین‌المللی ممکن است تعصب داشته باشند. شاخص‌ها باید با تحلیل کیفی (مصاحبه، مشاهده، تحلیل روایت) تکمیل شوند. «آنچه اندازه‌گیری می‌شود مدیریت می‌شود» — اما «آنچه فقط اندازه‌گیری می‌شود تحریف می‌شود».
\end{warningbox}

\begin{casestudy}{گرجستان: موفقیت قابل‌اندازه‌گیری}
گرجستان پس از انقلاب رز (۲۰۰۳) یکی از موفق‌ترین گذارهای پساشوروی بود. شاخص‌های کمّی:
\begin{itemize}[nosep]
    \item فساد (\lr{TI CPI}): از ۱.۸ (۲۰۰۳) به ۵.۲ (۲۰۱۲) — بهبود ۱۹۰٪
    \item آزادی اقتصادی: از رتبه ۱۰۱ به ۲۱ جهانی
    \item سهولت کسب‌وکار: از رتبه ۱۱۲ به ۸ جهانی
\end{itemize}
البته گرجستان در آزادی سیاسی عقب‌گرد داشت — نشان‌دهنده اهمیت نگاه جامع به همه شاخص‌ها.
\end{casestudy}

\sectiondivider

% ═══════════════════════════════════════════════════════════════════════════════
\section{مکانیزم پایش و اصلاح}
\label{sec:monitoring-mechanism}
% ═══════════════════════════════════════════════════════════════════════════════

\subsection{پنج سطح پایش}
\label{subsec:five-level-monitoring}

\begin{figure}[htbp]
\centering
\begin{tikzpicture}[
    every node/.style={font=\small, align=center},
    level/.style={rectangle, rounded corners, minimum width=10cm, minimum height=1cm, thick},
    arrow/.style={-{Stealth[length=2.5mm]}, thick, DarkGray}
]
    % Levels (bottom to top)
    \node[level, draw=MainGreen, fill=LightGreen] (l1) at (0,0) {\textbf{سطح ۱: پایش روزانه} — شاخص‌های هشدار زودهنگام | تیم پایش};
    \node[level, draw=MainYellow, fill=LightYellow] (l2) at (0,1.5) {\textbf{سطح ۲: گزارش ماهانه} — گزارش \lr{SRSG} به دبیرکل و شورای امنیت};
    \node[level, draw=MainOrange, fill=LightOrange] (l3) at (0,3) {\textbf{سطح ۳: بازنگری فصلی} — ارزیابی شاخص‌ها، تعدیل برنامه | تیم برنامه‌ریزی};
    \node[level, draw=MainRed, fill=LightRed] (l4) at (0,4.5) {\textbf{سطح ۴: ارزیابی مستقل} — تیم خارجی هر ۶ ماه | \lr{OIOS}};
    \node[level, draw=MainPurple, fill=LightPurple] (l5) at (0,6) {\textbf{سطح ۵: بازنگری استراتژیک} — شورای امنیت، سالانه | ادامه/تعدیل/خروج};
    
    % Arrows
    \draw[arrow] (l1) -- (l2);
    \draw[arrow] (l2) -- (l3);
    \draw[arrow] (l3) -- (l4);
    \draw[arrow] (l4) -- (l5);
    
    % Side label
    \node[font=\footnotesize, rotate=90, anchor=south] at (-6,3) {\textbf{افزایش سطح اختیار تصمیم‌گیری} →};
    
\end{tikzpicture}
\caption{پنج سطح پایش و اصلاح مسیر}
\label{fig:five-monitoring-levels}
\end{figure}

\subsection{پروتکل تصمیم‌گیری}
\label{subsec:decision-protocol}

\begin{table}[htbp]
\centering
\caption{پروتکل تصمیم‌گیری بر اساس نتایج ارزیابی}
\label{tab:decision-protocol}
\begin{tabularx}{\textwidth}{>{\centering\arraybackslash}p{2cm}
                             >{\raggedleft\arraybackslash}p{3cm}
                             >{\raggedleft\arraybackslash}X
                             >{\raggedleft\arraybackslash}p{2.5cm}}
\toprule
\headerrow وضعیت & شرایط & اقدام & تصمیم‌گیر \\
\midrule
\cellgreen{سبز} & ۷+ شاخص از ۹ در مسیر هدف & ادامه طبق برنامه، بهبود جزئی & \lr{SRSG} \\
\altrow \cellorange{زرد} & ۴-۶ شاخص در مسیر، بقیه تأخیر & تعدیل برنامه، تمرکز بر حوزه‌های ضعیف & \lr{SRSG} + دبیرکل \\
\cellorange{نارنجی} & ۲-۳ شاخص در مسیر، عقب‌گرد جزئی & بازنگری جدی، ممکن است تمدید فاز & شورای امنیت \\
\altrow \cellred{قرمز} & ۱ یا کمتر در مسیر، عقب‌گرد جدی & بازنگری اساسی استراتژی & شورای امنیت \\
\cellred{مشکی} & بحران حاد (کودتا، جنگ) & فعال‌سازی طرح اضطراری (\seeChapter{ch:risks}) & دبیرکل + \lr{SC} \\
\bottomrule
\end{tabularx}
\end{table}

\begin{lessonlearned}{موزامبیک: اصلاح مسیر موفق}
در موزامبیک، وقتی فرایند \lr{DDR} کندتر از برنامه پیش رفت، سازمان ملل به‌جای اصرار بر زمان‌بندی اولیه، فاز را ۶ ماه تمدید کرد. این انعطاف باعث شد فرایند با کیفیت بهتر تکمیل شود. \textbf{درس}: اصلاح مسیر نشانه ضعف نیست، نشانه هوشمندی است.
\end{lessonlearned}

\sectiondivider

% ═══════════════════════════════════════════════════════════════════════════════
\section{استراتژی خروج تفصیلی}
\label{sec:detailed-exit}
% ═══════════════════════════════════════════════════════════════════════════════

\subsection{شرایط خروج}
\label{subsec:exit-conditions}

\begin{table}[htbp]
\centering
\caption{معیارهای خروج به تفکیک مرحله}
\label{tab:exit-criteria}
\begin{tabularx}{\textwidth}{>{\raggedleft\arraybackslash}p{3cm}
                             >{\raggedleft\arraybackslash}X
                             >{\raggedleft\arraybackslash}p{3cm}}
\toprule
\headerrow مرحله خروج & شرط لازم & شاخص تأیید \\
\midrule
تبدیل به دفتر سیاسی (ماه ۶۰) & ۲ انتخابات آزاد + ۱ انتقال مسالمت‌آمیز قدرت & تأیید ناظران بین‌المللی \\
\altrow کاهش ۸۰٪ (ماه ۷۲) & استقلال قوه قضاییه + رسانه نسبتاً آزاد & \lr{Freedom House} بالای ۳.۰ \\
کاهش ۹۵٪ (ماه ۹۶) & نهادهای ملی عملکرد مستقل & ارزیابی مستقل مثبت \\
\altrow خروج کامل (ماه ۱۲۰) & همه ۹ شاخص در سطح هدف ۵ ساله & گزارش نهایی \\
\bottomrule
\end{tabularx}
\end{table}

\subsection{چرخه تصمیم خروج}
\label{subsec:exit-decision-cycle}

\begin{figure}[htbp]
\centering
\begin{tikzpicture}[
    node distance=2cm,
    every node/.style={font=\small, align=center},
    step/.style={rectangle, rounded corners, draw=MainPurple, fill=LightPurple, minimum width=3cm, minimum height=1cm, thick},
    decision/.style={diamond, draw=MainRed, fill=LightRed, minimum width=2cm, minimum height=1.5cm, thick, aspect=2},
    outcome/.style={rectangle, rounded corners, minimum width=2.5cm, minimum height=0.8cm, thick},
    arrow/.style={-{Stealth[length=2.5mm]}, thick}
]
    \node[step] (assess) {ارزیابی\\شاخص‌ها};
    \node[decision, right=2cm of assess] (decide) {آیا معیارها\\محقق شده؟};
    \node[outcome, above right=1cm and 2cm of decide, draw=MainGreen, fill=LightGreen] (yes) {خروج مرحله‌ای\\(گام بعدی)};
    \node[outcome, right=2.5cm of decide, draw=MainOrange, fill=LightOrange] (partial) {تمدید ۶-۱۲ ماه\\تعدیل برنامه};
    \node[outcome, below right=1cm and 2cm of decide, draw=MainRed, fill=LightRed] (no) {بازنگری اساسی\\ممکن: افزایش حضور};
    
    \draw[arrow] (assess) -- (decide);
    \draw[arrow, MainGreen] (decide) -- node[above, font=\footnotesize] {بله} (yes);
    \draw[arrow, MainOrange] (decide) -- node[above, font=\footnotesize] {تا حدی} (partial);
    \draw[arrow, MainRed] (decide) -- node[below, font=\footnotesize] {خیر} (no);
    
    % Feedback
    \draw[arrow, dashed, DarkGray] (partial.south) -- ++(0,-1) -| (assess.south);
    \draw[arrow, dashed, DarkGray] (no.west) -- ++(-1,0) |- (assess.south east);
    
\end{tikzpicture}
\caption{چرخه تصمیم‌گیری خروج}
\label{fig:exit-decision-cycle}
\end{figure}

\begin{warningbox}
\textbf{دام «خروج سیاسی»}: فشار داخلی در کشورهای حامی (آمریکا، اروپا) ممکن است خروج زودهنگام را دیکته کند — مثل افغانستان ۲۰۲۱. خروج باید بر اساس شاخص‌های میدانی باشد، نه انتخابات کشورهای حامی. مکانیزم دفاع: تعهدات حقوقی چندساله + نقش \lr{SC} (نه یک کشور) در تصمیم خروج.
\end{warningbox}

\sectiondivider

% ═══════════════════════════════════════════════════════════════════════════════
\section{جدول اقدامات به تفکیک بازه زمانی}
\label{sec:action-table}
% ═══════════════════════════════════════════════════════════════════════════════

\subsection{اقدامات فوری (الان تا ماه ۶)}
\label{subsec:immediate-actions}

\begin{table}[htbp]
\centering
\caption{اقدامات فوری — اولویت حیاتی}
\label{tab:immediate-actions}
\begin{tabularx}{\textwidth}{>{\centering\arraybackslash}p{0.5cm}
                             >{\raggedleft\arraybackslash}p{3.5cm}
                             >{\raggedleft\arraybackslash}X
                             >{\raggedleft\arraybackslash}p{2.5cm}}
\toprule
\headerrow \# & اقدام & شرح & مسئول \\
\midrule
۱ & تشکیل تیم آماده‌باش \lr{UN} & ۵۰-۱۰۰ نفر، برنامه‌ریزی سناریوها & \lr{DPPA} \\
\altrow ۲ & ائتلاف اپوزیسیون & منشور دموکراتیک، ساختار دولت انتقالی & اپوزیسیون \\
۳ & پیش‌نویس قطعنامه & نسخه‌های مختلف برای سناریوهای مختلف & \lr{P3+} \\
\altrow ۴ & شبکه ناظران & فهرست آماده‌باش ۵,۰۰۰+ ناظر & \lr{EU/OSCE/Carter} \\
۵ & آموزش مدنی & آموزش ناظران داخلی، مستندسازی & جامعه مدنی \\
\altrow ۶ & نقشه ظرفیت & شناسایی ایرانیان متخصص (داخل+دیاسپورا) & \lr{UNDP} \\
۷ & پیش‌موقعیت‌یابی & ذخیره تجهیزات در کشورهای همسایه & \lr{DPKO} \\
\altrow ۸ & گفتگوی ملی & آغاز گفتگوی غیررسمی بین همه گروه‌ها & میانجیان بین‌المللی \\
\bottomrule
\end{tabularx}
\end{table}

\subsection{اقدامات میان‌مدت (ماه ۶-۲۴)}
\label{subsec:medium-term-actions}

\begin{table}[htbp]
\centering
\caption{اقدامات میان‌مدت — ساختارسازی}
\label{tab:medium-actions}
\begin{tabularx}{\textwidth}{>{\centering\arraybackslash}p{0.5cm}
                             >{\raggedleft\arraybackslash}p{3.5cm}
                             >{\raggedleft\arraybackslash}X
                             >{\raggedleft\arraybackslash}p{2.5cm}}
\toprule
\headerrow \# & اقدام & شرح & مسئول \\
\midrule
۱ & قانون اساسی & مشاوره → مجلس مؤسسان → پیش‌نویس → رفراندوم & مجلس مؤسسان \\
\altrow ۲ & انتخابات & مجلس مؤسسان + رفراندوم + پارلمان & کمیسیون انتخابات \\
۳ & اصلاح سپاه & \lr{DDR}، ادغام، نظارت & معاون امنیتی \\
\altrow ۴ & کمیسیون حقیقت & تأسیس، شروع جلسات استماع & مجلس موقت \\
۵ & اصلاحات اقتصادی & رفع تحریم، صندوق امانی، جذب سرمایه & \lr{IMF/WB} \\
\altrow ۶ & نهادسازی & قوه قضاییه مستقل، کمیسیون رسانه، ضدفساد & دولت انتقالی \\
\bottomrule
\end{tabularx}
\end{table}

\subsection{اقدامات بلندمدت (ماه ۲۴-۶۰)}
\label{subsec:long-term-actions}

\begin{table}[htbp]
\centering
\caption{اقدامات بلندمدت — تحکیم}
\label{tab:long-term-actions}
\begin{tabularx}{\textwidth}{>{\centering\arraybackslash}p{0.5cm}
                             >{\raggedleft\arraybackslash}p{3.5cm}
                             >{\raggedleft\arraybackslash}X
                             >{\raggedleft\arraybackslash}p{2.5cm}}
\toprule
\headerrow \# & اقدام & شرح & مسئول \\
\midrule
۱ & تحکیم دموکراسی & انتخابات دوم، انتقال مسالمت‌آمیز قدرت & دولت منتخب \\
\altrow ۲ & انتقال مسئولیت & ۸۰→۱۰۰٪ مدیریت ایرانی & \lr{SRSG} \\
۳ & عدالت انتقالی & تکمیل کار کمیسیون، دادگاه ویژه & نهادهای قضایی \\
\altrow ۴ & توسعه اقتصادی & خصوصی‌سازی نظارت‌شده، رشد پایدار & دولت منتخب \\
۵ & الحاق به نهادهای بین‌المللی & \lr{WTO}، شورای حقوق بشر، و... & وزارت خارجه \\
\altrow ۶ & آماده‌سازی خروج & ارزیابی، تبدیل مأموریت، خروج تدریجی & \lr{SC/SRSG} \\
\bottomrule
\end{tabularx}
\end{table}

\sectiondivider

% ═══════════════════════════════════════════════════════════════════════════════
\section{نقشه راه بصری یکپارچه}
\label{sec:visual-roadmap}
% ═══════════════════════════════════════════════════════════════════════════════

\begin{figure}[htbp]
\centering
\begin{tikzpicture}[
    font=\footnotesize,
    milestone/.style={circle, draw=#1, fill=#1!20, minimum size=0.6cm, thick},
    phase/.style={rectangle, rounded corners, draw=#1, fill=#1!10, minimum width=2cm, minimum height=0.5cm},
]
    % Timeline
    \draw[very thick, DarkGray, -{Stealth}] (0,0) -- (16,0);
    
    % Phase backgrounds
    \fill[MediumGray!15] (0,-2) rectangle (2,2);
    \fill[MainRed!10] (2,-2) rectangle (5,2);
    \fill[MainOrange!10] (5,-2) rectangle (9,2);
    \fill[MainGreen!10] (9,-2) rectangle (13,2);
    \fill[MainBlue!10] (13,-2) rectangle (16,2);
    
    % Phase labels
    \node[font=\footnotesize\bfseries, MediumGray] at (1,2.3) {فاز ۰};
    \node[font=\footnotesize\bfseries, MainRed] at (3.5,2.3) {فاز ۱};
    \node[font=\footnotesize\bfseries, MainOrange] at (7,2.3) {فاز ۲};
    \node[font=\footnotesize\bfseries, MainGreen] at (11,2.3) {فاز ۳};
    \node[font=\footnotesize\bfseries, MainBlue] at (14.5,2.3) {فاز ۴};
    
    % Time markers
    \foreach \x/\t in {0/الان, 2/ماه ۰, 3.5/ماه ۶, 5/ماه ۱۲, 7/ماه ۲۴, 9/ماه ۳۶, 11/ماه ۴۸, 13/ماه ۶۰, 16/ماه ۱۲۰} {
        \draw[DarkGray] (\x,0.1) -- (\x,-0.1);
        \node[anchor=north, font=\tiny] at (\x,-0.2) {\t};
    }
    
    % Milestones (above line)
    \node[milestone=MainRed] (m1) at (2,0.8) {};
    \node[anchor=south, font=\tiny, text width=1.5cm] at (m1.north) {تیم ارزیابی\\قطعنامه};
    
    \node[milestone=MainRed] (m2) at (3.5,0.8) {};
    \node[anchor=south, font=\tiny, text width=1.5cm] at (m2.north) {UNMOIT\\فعال};
    
    \node[milestone=MainOrange] (m3) at (5,0.8) {};
    \node[anchor=south, font=\tiny, text width=1.5cm] at (m3.north) {مجلس\\مؤسسان};
    
    \node[milestone=MainOrange] (m4) at (7,0.8) {};
    \node[anchor=south, font=\tiny, text width=1.5cm] at (m4.north) {رفراندوم\\قانون اساسی};
    
    \node[milestone=MainGreen] (m5) at (9,0.8) {};
    \node[anchor=south, font=\tiny, text width=1.5cm] at (m5.north) {انتقال\\قدرت};
    
    \node[milestone=MainGreen] (m6) at (11,0.8) {};
    \node[anchor=south, font=\tiny, text width=1.5cm] at (m6.north) {انتخابات\\دوم};
    
    \node[milestone=MainBlue] (m7) at (13,0.8) {};
    \node[anchor=south, font=\tiny, text width=1.5cm] at (m7.north) {تبدیل\\مأموریت};
    
    \node[milestone=MainPurple] (m8) at (16,0.8) {};
    \node[anchor=south, font=\tiny, text width=1.5cm] at (m8.north) {خروج\\کامل};
    
    % Activities (below line)
    \node[anchor=north, font=\tiny, text width=1.8cm] at (1,-0.5) {آماده‌باش\\برنامه‌ریزی};
    \node[anchor=north, font=\tiny, text width=1.8cm] at (3,-0.5) {تثبیت\\امنیت};
    \node[anchor=north, font=\tiny, text width=1.8cm] at (6,-0.5) {نهادسازی\\قانون‌نویسی};
    \node[anchor=north, font=\tiny, text width=1.8cm] at (8,-0.5) {انتخابات\\پارلمان};
    \node[anchor=north, font=\tiny, text width=1.8cm] at (10,-0.5) {تحکیم\\انتقال};
    \node[anchor=north, font=\tiny, text width=1.8cm] at (14,-0.5) {مشاوره\\ارزیابی};
    
    % Personnel curve
    \draw[MainPurple, very thick, dashed] plot[smooth] coordinates {
        (0,0) (1,0.2) (2,0.5) (3,1) (3.5,1.2) (5,1.5) (6,1.7) (7,1.8)
        (8,1.5) (9,1.2) (10,1) (11,0.7) (13,0.3) (16,0.05)
    };
    \node[MainPurple, font=\tiny, anchor=west] at (7.5,1.9) {حضور بین‌المللی};
    
\end{tikzpicture}
\caption{نقشه راه بصری یکپارچه — از آماده‌باش تا خروج}
\label{fig:visual-roadmap}
\end{figure}

\sectiondivider

% ═══════════════════════════════════════════════════════════════════════════════
\section{پیام نهایی به هر مخاطب}
\label{sec:final-message}
% ═══════════════════════════════════════════════════════════════════════════════

\begin{keypoint}
\textbf{به اپوزیسیون ایرانی}: وحدت یابید. اختلافات امروز، ضعف فردا خواهد بود. بر اصول توافق کنید، جزئیات را به دموکراسی بسپارید.

\textbf{به جامعه مدنی ایران}: شما ستون فقرات گذارید. آماده شوید. مستند کنید. شبکه بسازید. مهارت بیاموزید.

\textbf{به سازمان ملل}: غافلگیر نشوید. تیم بسازید. برنامه بریزید. \lr{SRSG} را شناسایی کنید. الان.

\textbf{به اتحادیه اروپا}: بزرگ‌ترین حامی صلح جهان باشید — نه فقط ناظر.

\textbf{به آمریکا}: تحریم‌ها را ابزار فشار بدانید نه هدف. آماده رفع سریع باشید.

\textbf{به کشورهای منطقه}: ثبات ایران به نفع شماست. مداخله نکنید. حمایت کنید.

\textbf{به دیاسپورا}: تخصص‌تان ارزشمند است. با فروتنی بیایید. گوش دهید قبل از حرف زدن.

\textbf{به نیروهای سپاه}: جایی برای شما در آینده هست — اگر به مردم شلیک نکنید.
\end{keypoint}

\sectiondivider

% ═══════════════════════════════════════════════════════════════════════════════
% جمع‌بندی فصل
% ═══════════════════════════════════════════════════════════════════════════════

\begin{chaptersummary}
یافته‌های کلیدی این فصل:

\begin{enumerate}
    \item \textbf{ده توصیه کلیدی}: مالکیت ملی، آماده‌باش، مدل ترکیبی، فراگیری، عدالت آشتی‌محور، مدیریت سپاه، اقتصاد، رسانه آزاد، حل مسئله هسته‌ای، خروج شفاف.
    
    \item \textbf{نُه شاخص کمّی}: آزادی سیاسی، آزادی مطبوعات، مشارکت انتخاباتی، فساد، حقوق بشر، برابری جنسیتی، رشد اقتصادی، اعتماد عمومی، امنیت — هر کدام با هدف عددی برای سال ۳ و ۵.
    
    \item \textbf{پایش پنج‌سطحی}: از روزانه تا سالانه، با پروتکل تصمیم‌گیری سبز/زرد/نارنجی/قرمز/مشکی.
    
    \item \textbf{خروج شاخص‌محور}: معیارهای مشخص برای هر مرحله خروج، نه تقویم سیاسی.
    
    \item \textbf{اقدامات به سه بازه}: فوری (الان-۶ ماه)، میان‌مدت (۶-۲۴)، بلندمدت (۲۴-۶۰).
    
    \item \textbf{توصیه‌ها به تفکیک مخاطب}: هر بازیگر — از اپوزیسیون تا سازمان ملل — می‌داند دقیقاً چه باید بکند.
    
    \item \textbf{زمان اقدام الان است}: بهترین زمان برای آمادگی، قبل از بحران است.
\end{enumerate}

\vspace{0.5cm}
\textit{در فصل پایانی (\ref{ch:conclusion})، جمع‌بندی کل کتاب و کلام آخر نویسنده ارائه خواهد شد.}
\end{chaptersummary}

\chapterend