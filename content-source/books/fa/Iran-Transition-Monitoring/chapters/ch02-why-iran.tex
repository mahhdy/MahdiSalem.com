% ╔══════════════════════════════════════════════════════════════════╗
% ║  فصل ۲: چرا ایران؟                                             ║
% ║  ویژگی‌ها، پیچیدگی‌ها و استثنائات                               ║
% ╚══════════════════════════════════════════════════════════════════╝

% ---- صفحه‌ی آغازین فصل ----
\chapteropening{۲}
    {چرا ایران؟ ویژگی‌ها، پیچیدگی‌ها و استثنائات}
    {MainBlue}
    {ایران یک کشور نیست، یک تمدن است 
    که در قالب یک کشور گنجانده شده.}
    {ریچارد فرای، ایران‌شناس آمریکایی}

\chapter{چرا ایران؟ ویژگی‌ها، پیچیدگی‌ها و استثنائات}
\label{ch:why-iran}
\minitoc

% ---- خلاصه‌ی اجرایی فصل ----
\begin{executivesummary}
این فصل تبیین می‌کند که \emphblue{ایران یک مورد 
استاندارد گذار نیست}. ترکیب منحصربه‌فرد ویژگی‌های 
ژئوپلیتیکی (قدرت منطقه‌ای، بازیگر هسته‌ای)، 
جامعه‌شناختی (تنوع قومی، جوانی جمعیت، دیاسپورای 
عظیم)، و ساختار سیاسی (نظام دوگانه، سپاه پاسداران، 
روحانیت نهادینه) ایران را به یکی از پیچیده‌ترین 
موارد احتمالی گذار دموکراتیک در تاریخ معاصر 
تبدیل می‌کند. هرگونه الگوبرداری مکانیکی از 
نمونه‌های دیگر محکوم به شکست است.
\end{executivesummary}

% ============================================================
\section{ویژگی‌های ژئوپلیتیکی: ایران در تقاطع بحران‌ها}
\label{sec:geopolitics}
% ============================================================

ایران در یکی از حساس‌ترین نقاط ژئوپلیتیکی جهان 
قرار دارد. این ویژگی هر فرایند گذاری را تحت تأثیر 
قرار می‌دهد زیرا بازیگران منطقه‌ای و بین‌المللی 
منافع حیاتی در ایران دارند.

\subsection{قدرت منطقه‌ای و بازیگر هسته‌ای}

\begin{statsbox}
\begin{minipage}[t]{0.48\textwidth}
    \begin{center}
        {\statisticfont ۸۵}\\[2pt]
        {\small میلیون نفر جمعیت}\\[8pt]
        {\statisticfont ۱.۶}\\[2pt]
        {\small میلیون کیلومتر مربع}\\[8pt]
        {\statisticfont ۴}\\[2pt]
        {\small نیروی نظامی میلیونی 
        (فعال + ذخیره)}
    \end{center}
\end{minipage}%
\hfill
\begin{minipage}[t]{0.48\textwidth}
    \begin{center}
        {\statisticfont ۴}\\[2pt]
        {\small درصد ذخایر نفت جهان}\\[8pt]
        {\statisticfont ۱۷}\\[2pt]
        {\small درصد ذخایر گاز جهان}\\[8pt]
        {\statisticfont ۱۵}\\[2pt]
        {\small کشور همسایه (زمینی و دریایی)}
    \end{center}
\end{minipage}
\end{statsbox}

ایران صرفاً یک کشور اقتدارگرای کوچک مانند 
تونس یا حتی یک کشور متوسط مانند لهستان نیست. 
ایران یک \emphblue{قدرت منطقه‌ای} با ویژگی‌های 
زیر است:

\begin{itemize}[itemsep=4pt]
    \item \textbf{بازیگر هسته‌ای:} 
    برنامه‌ی هسته‌ای ایران — چه مسالمت‌آمیز و 
    چه نظامی — هر فرایند گذاری را به مسئله‌ای 
    بین‌المللی تبدیل می‌کند. آمریکا، اسرائیل و 
    اروپا نمی‌توانند نسبت به سرنوشت برنامه‌ی 
    هسته‌ای در دوره‌ی گذار بی‌تفاوت باشند.
    
    \item \textbf{شبکه‌ی نیابتی منطقه‌ای:} 
    حزب‌الله لبنان، حشد الشعبی عراق، حوثی‌های 
    یمن و گروه‌های فلسطینی — سقوط نظام ایران 
    تأثیر فوری بر تمام این بازیگران و در نتیجه 
    بر کل ژئوپلیتیک خاورمیانه خواهد داشت.
    
    \item \textbf{تنگه‌ی هرمز:} 
    حدود ۲۰ درصد نفت جهان از این تنگه عبور 
    می‌کند. بی‌ثباتی در ایران مستقیماً بر بازار 
    جهانی انرژی تأثیر می‌گذارد.
    
    \item \textbf{رقابت قدرت‌های بزرگ:} 
    آمریکا، چین و روسیه منافع متعارض در ایران 
    دارند. چین قرارداد ۲۵ ساله با ایران دارد؛ 
    روسیه ایران را شریک استراتژیک می‌داند؛ 
    آمریکا خواهان تغییر رفتار (یا تغییر رژیم) است.
\end{itemize}

\begin{warningbox}
\textbf{پیامد برای نظارت بین‌المللی:}
بر خلاف تیمور شرقی یا حتی آفریقای جنوبی، 
گذار در ایران یک \emphred{رویداد ژئوپلیتیکی 
جهانی} است. وتوی احتمالی چین و روسیه در 
شورای امنیت، تلاش بازیگران منطقه‌ای برای 
تأثیرگذاری بر نتیجه، و حساسیت بازار انرژی 
همگی فرایند نظارت را پیچیده می‌کنند. 
هر مدل نظارتی باید این واقعیت‌ها را از 
ابتدا در نظر بگیرد.
\end{warningbox}

\subsection{جغرافیای بحران: همسایگان ناآرام}

\begin{figure}[htbp]
    \centering
    \begin{tikzpicture}[
        country/.style={
            draw, rounded corners=2pt,
            minimum height=0.8cm,
            align=center, font=\tiny\bfseries
        },
        stable/.style={country, fill=GreenBG, draw=MainGreen},
        unstable/.style={country, fill=OrangeBG, draw=MainOrange},
        crisis/.style={country, fill=RedBG, draw=MainRed},
        rival/.style={country, fill=BlueBG, draw=MainBlue},
        iran/.style={
            draw=MainPurple, fill=PurpleBG,
            rounded corners=3pt,
            minimum height=1.5cm, minimum width=2.5cm,
            font=\small\bfseries
        }
    ]
    
    % ایران در مرکز
    \node[iran] (iran) at (0,0) {ایران};
    
    % همسایگان
    \node[crisis] (iraq) at (-3.5, 1) {عراق\\بی‌ثباتی};
    \node[crisis] (afghan) at (3.5, 1) {افغانستان\\بحران};
    \node[unstable] (pak) at (3.5, -1) {پاکستان\\شکننده};
    \node[rival] (turkey) at (-2, 2.5) {ترکیه\\رقیب};
    \node[rival] (saudi) at (-2, -2.5) {عربستان\\رقیب};
    \node[unstable] (azer) at (0, 2.5) {آذربایجان\\تنش};
    \node[stable] (turkmen) at (2, 2.5) {ترکمنستان\\اقتدارگرا};
    \node[stable] (arm) at (-1, 2.5) {ارمنستان\\شکننده};
    
    % دریایی
    \node[rival] (uae) at (-0.5, -2.5) 
        {امارات\\رقیب};
    \node[rival] (kuwait) at (-3, -1) 
        {کویت};
    
    % فلش‌ها
    \draw[MainRed, thick, ->] (iraq) -- (iran) 
        node[midway, above, font=\tiny] {شبه‌نظامیان};
    \draw[MainRed, thick, ->] (afghan) -- (iran) 
        node[midway, above, font=\tiny] {مهاجرت، مواد};
    \draw[MainOrange, thick, <->] (turkey) -- (iran) 
        node[midway, left, font=\tiny] {رقابت};
    \draw[MainOrange, thick, <->] (saudi) -- (iran) 
        node[midway, left, font=\tiny] {رقابت};
    
    \end{tikzpicture}
    \caption{جغرافیای ژئوپلیتیکی ایران: 
    همسایگان و تنش‌ها}
    \label{fig:geopolitical-map}
\end{figure}

\begin{lessonlearned}
\textbf{از تجربه‌ی عراق:}
پس از سقوط صدام (۲۰۰۳)، نبود امنیت مرزی 
منجر به سرازیر شدن جنگجویان خارجی، تأسیس 
\lr{ISIS} و سال‌ها جنگ داخلی شد. ایران با ۱۵ 
همسایه‌ی زمینی و دریایی — چندین مورد آن‌ها 
بی‌ثبات — با چالش مشابه اما در مقیاس بزرگ‌تر 
روبروست. \emphblue{امنیت مرزی باید اولویت 
نخست هر برنامه‌ی نظارتی باشد.}
\end{lessonlearned}

% ============================================================
\section{ویژگی‌های جامعه‌شناختی: سرمایه و چالش}
\label{sec:sociology}
% ============================================================

\subsection{تنوع قومی-زبانی و مذهبی}

ایران یکی از متنوع‌ترین کشورهای خاورمیانه است. 
این تنوع هم یک سرمایه‌ی ملی و هم یک چالش 
بالقوه برای دوره‌ی گذار است:

\begin{table}[htbp]
    \centering
    \caption{ترکیب قومی-زبانی و مذهبی ایران 
    (تخمینی)}
    \label{tab:ethnic-composition}
    \tablefontsize
    \begin{tabularx}{\textwidth}{
        L{2cm} C{1.5cm} X L{2.5cm}
    }
        \toprule
        \headerrow
        \textbf{گروه} & 
        \textbf{درصد تخمینی} & 
        \textbf{مناطق اصلی} & 
        \textbf{حساسیت سیاسی} \\
        \midrule
        
        فارس &
        ۵۰-۵۵\% &
        مرکز، شرق، جنوب &
        اکثریت — هویت مسلط \\
        \altrow
        
        ترک آذربایجانی &
        ۱۵-۲۰\% &
        شمال‌غرب (آذربایجان) &
        حساس — مسئله‌ی هویت \\
        
        کُرد &
        ۷-۱۰\% &
        غرب (کردستان، کرمانشاه، ایلام) &
        بسیار حساس — سابقه مسلح \\
        \altrow
        
        لُر و بختیاری &
        ۶-۸\% &
        غرب و جنوب‌غرب &
        متوسط \\
        
        عرب &
        ۲-۳\% &
        جنوب‌غرب (خوزستان) &
        حساس — مسئله‌ی منابع نفتی \\
        \altrow
        
        بلوچ &
        ۲-۳\% &
        جنوب‌شرق (سیستان) &
        بسیار حساس — محرومیت \\
        
        ترکمن &
        ۱-۲\% &
        شمال‌شرق &
        متوسط \\
        \altrow
        
        گیلک و مازندرانی &
        ۵-۷\% &
        شمال (سواحل خزر) &
        پایین \\
        
        سایر &
        ۳-۵\% &
        پراکنده &
        متفاوت \\
        
        \bottomrule
    \end{tabularx}
\end{table}

\begin{comparisonbox}{تنوع قومی ایران و نمونه‌های تاریخی}
\begin{itemize}[itemsep=3pt]
    \item \textbf{آفریقای جنوبی:} 
    تنوع نژادی شدید (سیاه، سفید، رنگین‌پوست، هندی) 
    اما مدل آشتی ملی مبتنی بر «ملت رنگین‌کمان» 
    موفق بود. \emphgreen{درس: هویت فراگیر ملی 
    کلید است.}
    
    \item \textbf{یوگسلاوی:} 
    تنوع قومی-مذهبی مشابه، اما فقدان هویت ملی 
    فراگیر منجر به تجزیه و جنگ داخلی شد. 
    \emphred{هشدار: اگر هویت ایرانی تضعیف شود.}
    
    \item \textbf{اسپانیا:} 
    تنوع منطقه‌ای (کاتالان، باسک، گالیسی) با مدل 
    خودمختاری مدیریت شد. 
    \emphgreen{درس: فدرالیسم/خودمختاری 
    می‌تواند تنوع را مدیریت کند.}
    
    \item \textbf{عراق:} 
    تنوع قومی-مذهبی (شیعه، سنی، کرد) با نظام 
    سهمیه‌ای مدیریت شد — نتیجه: فلج نهادی و 
    فرقه‌گرایی. 
    \emphred{هشدار: نظام سهمیه‌ای قومی-مذهبی 
    برای ایران مناسب نیست.}
\end{itemize}
\end{comparisonbox}

\subsection{جوانی جمعیت و شکاف نسلی}

\begin{statsbox}
\begin{center}
    {\statisticfont ۶۰\%}\\[2pt]
    {\small از جمعیت ایران زیر ۳۵ سال هستند}\\[8pt]
    {\statisticfont ۹۸\%}\\[2pt]
    {\small نرخ باسوادی (یکی از بالاترین 
    در خاورمیانه)}\\[8pt]
    {\statisticfont ۴.۵}\\[2pt]
    {\small میلیون دانشجو (بالاترین نسبت 
    به جمعیت در منطقه)}
\end{center}
\end{statsbox}

جوانی جمعیت ایران هم فرصت و هم چالش است:

\begin{table}[htbp]
    \centering
    \caption{جوانی جمعیت: فرصت‌ها و چالش‌ها 
    برای گذار}
    \label{tab:youth-opportunities}
    \begin{tabularx}{\textwidth}{
        C{0.5cm} X X
    }
        \toprule
        \headerrow
        & \textbf{\textcolor{MainGreen}{فرصت}} & 
        \textbf{\textcolor{MainRed}{چالش}} \\
        \midrule
        
        ۱ &
        \cellgreen{انرژی بالا برای تغییر و 
        مشارکت مدنی} &
        \cellred{بیکاری جوانان = بمب ساعتی 
        اجتماعی} \\
        \altrow
        
        ۲ &
        \cellgreen{آشنایی با فناوری و 
        شبکه‌های اجتماعی} &
        \cellred{آسیب‌پذیری در برابر 
        اطلاعات نادرست} \\
        
        ۳ &
        \cellgreen{تحصیلات بالا = ظرفیت 
        نهادسازی} &
        \cellred{انتظارات بالا = ناامیدی سریع 
        اگر تغییر کُند باشد} \\
        \altrow
        
        ۴ &
        \cellgreen{ارتباط با جهان از طریق 
        \lr{VPN} و دیاسپورا} &
        \cellred{شکاف نسلی عمیق = تعارض 
        بر سر ارزش‌ها} \\
        
        ۵ &
        \cellgreen{رهبری جنبش «زن، زندگی، 
        آزادی» توسط زنان جوان} &
        \cellred{خطر رادیکالیزه شدن بخشی 
        از جوانان} \\
        
        \bottomrule
    \end{tabularx}
\end{table}

\subsection{دیاسپورای ایرانی: پل یا شکاف؟}

\begin{statsbox}
\begin{center}
    {\statisticfont ۴-۵}\\[2pt]
    {\small میلیون ایرانی در خارج از کشور}\\[8pt]
    {\statisticfont ۱}\\[2pt]
    {\small میلیون نفر فقط در آمریکا 
    (بزرگ‌ترین جمعیت دیاسپورا)}
\end{center}
\end{statsbox}

دیاسپورای ایرانی یکی از تحصیل‌کرده‌ترین و 
موفق‌ترین جوامع مهاجر جهان است. اما رابطه‌ی 
دیاسپورا با داخل پیچیده است:

\begin{table}[htbp]
    \centering
    \caption{نقش دوگانه‌ی دیاسپورا در گذار}
    \label{tab:diaspora-dual}
    \tablefontsize
    \begin{tabularx}{\textwidth}{
        L{2.5cm} X X
    }
        \toprule
        \headerrow
        \textbf{بُعد} & 
        \textbf{\textcolor{MainGreen}{ظرفیت}} & 
        \textbf{\textcolor{MainRed}{ریسک}} \\
        \midrule
        
        مالی &
        منابع مالی عظیم برای حمایت از گذار &
        تلاش برای «خرید» نفوذ سیاسی \\
        \altrow
        
        فنی &
        تخصص در حوزه‌های حقوقی، فنی، 
        مدیریتی &
        ناآشنایی با واقعیت‌های روزمره‌ی 
        داخل ایران \\
        
        سیاسی &
        لابی بین‌المللی و جلب حمایت 
        دولت‌ها &
        تضادهای جناحی شدید بین گروه‌های 
        دیاسپورا \\
        \altrow
        
        فرهنگی &
        پل ارتباطی با جهان، 
        انتقال ارزش‌های دموکراتیک &
        فاصله‌ی فرهنگی فزاینده با 
        نسل جدید داخل \\
        
        انسانی &
        بازگشت نخبگان و متخصصان &
        تنش بازگشتی‌ها/مقیمان 
        (تجربه‌ی افغانستان) \\
        
        \bottomrule
    \end{tabularx}
\end{table}

\begin{lessonlearned}
\textbf{از تجربه‌ی افغانستان:}
پس از سقوط طالبان (۲۰۰۱)، بسیاری از 
افغان‌های دیاسپورا به مناصب کلیدی منصوب شدند. 
نتیجه: شکاف عمیق بین «غرب‌رفته‌ها» و مردم 
محلی، اتهام فساد و ناکارآمدی، و در نهایت 
فروپاشی ۲۰۲۱. \emphblue{مشارکت دیاسپورا 
باید ساختاریافته، شفاف و تابع اراده‌ی مردم 
داخل باشد — نه جایگزین آن.}
\end{lessonlearned}

% ============================================================
\section{ویژگی‌های نظام سیاسی: 
ساختار دوگانه‌ی قدرت}
\label{sec:political-structure}
% ============================================================

\subsection{معماری قدرت در جمهوری اسلامی}

نظام سیاسی ایران ساختاری منحصربه‌فرد دارد: 
ترکیبی از عناصر 
\termfn{جمهوری}{Republican} 
(انتخابات، مجلس، رئیس‌جمهور) و عناصر 
\termfn{تئوکراتیک-اقتدارگرا}{Theocratic-Authoritarian} 
(رهبر، شورای نگهبان، سپاه). این دوگانگی 
فرایند گذار را پیچیده می‌کند.

\begin{figure}[htbp]
    \centering
    \begin{tikzpicture}[
        inst/.style={
            draw, rounded corners=2pt,
            minimum height=1cm,
            align=center, font=\footnotesize
        },
        elected/.style={inst, fill=GreenBG, draw=MainGreen},
        appointed/.style={inst, fill=RedBG, draw=MainRed},
        military/.style={inst, fill=OrangeBG, draw=MainOrange},
        supreme/.style={
            inst, fill=DarkRed!80, text=white,
            font=\small\bfseries,
            minimum height=1.3cm, minimum width=3cm
        },
        arr/.style={-{Stealth[length=2mm]}, thick},
        appoints/.style={arr, MainRed, dashed},
        elects/.style={arr, MainGreen}
    ]
    
    % رهبر
    \node[supreme] (leader) at (0, 4.5) 
        {رهبر (ولی فقیه)};
    
    % نهادهای انتصابی
    \node[appointed, minimum width=2.5cm] (gc) at (-4, 2.5) 
        {شورای نگهبان};
    \node[appointed, minimum width=2.5cm] (jud) at (4, 2.5) 
        {قوه‌ی قضاییه};
    \node[appointed, minimum width=2.5cm] (exp) at (-4, 0.5) 
        {مجمع تشخیص};
    \node[military, minimum width=2.5cm] (irgc) at (4, 0.5) 
        {سپاه پاسداران};
    
    % نهادهای انتخابی
    \node[elected, minimum width=2.5cm] (pres) at (-1.5, -1.5) 
        {رئیس‌جمهور};
    \node[elected, minimum width=2.5cm] (parl) at (1.5, -1.5) 
        {مجلس شورا};
    
    % مردم
    \node[font=\small\bfseries, MainBlue] (people) at (0, -3.5) 
        {مردم (رأی‌دهندگان)};
    
    % فلش‌ها — انتصاب
    \draw[appoints] (leader) -- (gc) 
        node[midway, above, font=\tiny] {انتصاب};
    \draw[appoints] (leader) -- (jud)
        node[midway, above, font=\tiny] {انتصاب};
    \draw[appoints] (leader) -- (irgc)
        node[midway, right, font=\tiny] {فرماندهی};
    \draw[appoints] (leader) -- (exp);
    
    % فلش‌ها — انتخاب
    \draw[elects] (people) -- (pres)
        node[midway, left, font=\tiny] {انتخاب};
    \draw[elects] (people) -- (parl)
        node[midway, right, font=\tiny] {انتخاب};
    
    % فلش — فیلتر
    \draw[MainRed, ultra thick, ->] (gc) -- (pres)
        node[midway, left, font=\tiny\color{MainRed}] 
        {فیلتر نامزدها};
    \draw[MainRed, ultra thick, ->] (gc) -- (parl)
        node[midway, right, font=\tiny\color{MainRed}] 
        {فیلتر نامزدها};
    
    % کادر تقسیم‌بندی
    \draw[MainRed, thick, dashed] 
        (-6, 1.5) -- (6, 1.5);
    \node[font=\tiny\bfseries, MainRed] at (5.5, 1.8) 
        {اقتدارگرا};
    \node[font=\tiny\bfseries, MainGreen] at (5.5, 1.2) 
        {شبه‌دموکراتیک};
    
    \end{tikzpicture}
    \caption{ساختار دوگانه‌ی قدرت در 
    جمهوری اسلامی ایران}
    \label{fig:dual-power}
\end{figure}

\begin{keypoint}
\textbf{چرا این ساختار گذار را پیچیده می‌کند:}
\begin{enumerate}[itemsep=3pt, font=\small]
    \item هیچ «مرکز واحد قدرت» وجود ندارد 
    که با آن مذاکره شود.
    \item سپاه پاسداران هم بازیگر نظامی، هم 
    اقتصادی و هم سیاسی است.
    \item شورای نگهبان مکانیزم «وتوی ساختاری» 
    بر هرگونه تغییر درون‌نظامی است.
    \item نقش رهبر شخصی و غیرنهادی است — 
    بحران جانشینی می‌تواند خود محرک گذار باشد.
    \item بخش «انتخابی» نظام سوابق و 
    زیرساخت‌هایی دارد که می‌تواند در گذار 
    مفید باشد.
\end{enumerate}
\end{keypoint}

\subsection{سپاه پاسداران: دولت در دولت}

سپاه پاسداران انقلاب اسلامی (\lr{IRGC}) 
مهم‌ترین بازیگر در هر سناریوی گذار است. 
درک ابعاد قدرت سپاه برای طراحی هر مدل 
نظارتی ضروری است:

\begin{table}[htbp]
    \centering
    \caption{ابعاد قدرت سپاه پاسداران}
    \label{tab:irgc-dimensions}
    \begin{tabularx}{\textwidth}{
        L{2cm} X L{3cm}
    }
        \toprule
        \headerrow
        \textbf{بُعد} & 
        \textbf{توضیح} & 
        \textbf{مقایسه} \\
        \midrule
        
        نظامی &
        ۱۵۰,۰۰۰+ نیروی فعال، موشک‌های 
        بالستیک، نیروی دریایی، هوافضا، 
        نیروی قدس (عملیات خارجی) &
        بزرگ‌تر از ارتش 
        بسیاری کشورها \\
        \altrow
        
        اقتصادی &
        کنترل تخمینی ۲۰-۴۰\% اقتصاد 
        ایران از طریق بنیادها، شرکت‌ها و 
        قراردادهای دولتی &
        شبیه ارتش 
        مصر یا پاکستان \\
        
        اطلاعاتی &
        سازمان اطلاعات سپاه، 
        کنترل فضای سایبری، 
        بسیج مستضعفین (۵+ میلیون) &
        \lr{KGB} + 
        شبکه‌ی بسیج محلی \\
        \altrow
        
        سیاسی &
        حضور مستقیم و غیرمستقیم 
        سپاهیان در مجلس، دولت و 
        نهادهای محلی &
        حزب حاکم + ارتش 
        ترکیبی \\
        
        ایدئولوژیک &
        تعهد به ولایت فقیه و 
        صدور انقلاب &
        نیروی ایدئولوژیک 
        (نه حرفه‌ای صرف) \\
        
        \bottomrule
    \end{tabularx}
\end{table}

\begin{warningbox}
\textbf{مهم‌ترین درس تاریخی:}
\begin{itemize}[itemsep=3pt]
    \item \textbf{عراق (دی‌بعثی‌سازی ۲۰۰۳):} 
    انحلال کامل ارتش و حزب بعث → ۵۰۰,۰۰۰ 
    مرد مسلح بیکار و خشمگین → ظهور داعش. 
    \emphred{نابودی کامل سپاه فاجعه‌بار خواهد بود.}
    
    \item \textbf{مصر (۲۰۱۱-۲۰۱۳):} 
    ارتش نقش «محافظ گذار» ایفا کرد اما سپس 
    خودش قدرت را تصاحب کرد. 
    \emphred{اعتماد کورکورانه به نظامیان خطرناک است.}
    
    \item \textbf{اندونزی (۱۹۹۸):} 
    بازسازی تدریجی ارتش از نقش سیاسی به 
    نقش حرفه‌ای طی ۱۰-۱۵ سال. 
    \emphgreen{بهترین الگو برای مدیریت سپاه.}
\end{itemize}
\end{warningbox}

% ============================================================
\section{چرا مدل‌های موجود ناکافی‌اند: 
ماتریس استثنائات ایران}
\label{sec:iran-exceptions}
% ============================================================

با جمع‌بندی ویژگی‌های ایران، اکنون می‌توانیم 
نشان دهیم چرا هیچ مدل نظارتی موجودی 
بدون تطبیق اساسی برای ایران کار نمی‌کند:

\begin{table}[htbp]
    \centering
    \caption{ماتریس استثنائات ایران نسبت به 
    نمونه‌های تاریخی}
    \label{tab:iran-exceptions}
    \tablefontsize
    \begin{tabularx}{\textwidth}{
        L{3cm} C{1cm} X
    }
        \toprule
        \headerrow
        \textbf{ویژگی} & 
        \textbf{مشابه؟} & 
        \textbf{نزدیک‌ترین مشابه و تفاوت‌ها} \\
        \midrule
        
        جمعیت ۸۵ میلیون &
        \xmark &
        بزرگ‌تر از همه‌ی نمونه‌ها به جز 
        اندونزی (۲۷۰M) \\
        \altrow
        
        برنامه‌ی هسته‌ای &
        \xmark &
        هیچ نمونه‌ی گذاری با بازیگر 
        هسته‌ای وجود ندارد \\
        
        ارتش ایدئولوژیک + اقتصادی &
        \statuswarn &
        مصر (ارتش اقتصادی) + میانمار 
        (ارتش ایدئولوژیک) \\
        \altrow
        
        شبکه‌ی نیابتی منطقه‌ای &
        \xmark &
        هیچ نمونه‌ای ندارد \\
        
        تنوع قومی + وحدت ملی نسبی &
        \statuswarn &
        آفریقای جنوبی (تنوع + وحدت)، 
        عراق (تنوع - وحدت) \\
        \altrow
        
        دیاسپورای بزرگ و ثروتمند &
        \statuswarn &
        افغانستان (بزرگ اما فقیر)، 
        کوبا (ثروتمند اما کوچک‌تر) \\
        
        تمدن ۳۰۰۰ ساله &
        \statuswarn &
        مصر (تمدنی اما متفاوت) \\
        \altrow
        
        نظام تئوکراتیک نهادینه &
        \xmark &
        هیچ نمونه‌ی مشابهی وجود ندارد 
        (طالبان و داعش = غیرنهادی) \\
        
        تاریخ شکست‌خورده‌ی دموکراسی &
        \statuswarn &
        مشابه مصر و روسیه \\
        \altrow
        
        اهمیت انرژی جهانی &
        \statuswarn &
        عراق (نفت)، لیبی (نفت) \\
        
        \bottomrule
    \end{tabularx}
    
    {\footnotesize 
    \cmark = مشابه \hspace{1cm} 
    \statuswarn = تا حدی مشابه \hspace{1cm} 
    \xmark = بی‌سابقه}
\end{table}

\begin{recommendation}
\textbf{نتیجه‌گیری عملیاتی:}
ایران نیاز به یک \emphblue{مدل نظارتی 
طراحی‌شده‌ی سفارشی} دارد — نه کپی از 
هیچ نمونه‌ی موجود. این مدل باید:
\begin{enumerate}[itemsep=3pt]
    \item ابعاد ژئوپلیتیکی را در 
    طراحی خود لحاظ کند
    \item برای مدیریت سپاه 
    استراتژی مشخص داشته باشد
    \item مکانیزم‌های مدیریت تنوع 
    قومی را شامل شود
    \item نقش دیاسپورا را 
    ساختاریافته تعریف کند
    \item بُعد هسته‌ای را مدیریت کند
    \item بر فراگیری (زنان، جوانان، اقوام) 
    تأکید ویژه داشته باشد
\end{enumerate}
فصل ۳ چنین مدلی را ارائه خواهد کرد.
\end{recommendation}

\sectiondivider

% ============================================================
\section{جمع‌بندی فصل}
\label{sec:ch2-summary}
% ============================================================

\begin{chaptersummary}

\begin{chaptersummary}

\textbf{آنچه در این فصل آموختیم:}

\begin{enumerate}[
    label=\textcolor{DarkGray}{\bfseries\arabic*.},
    itemsep=4pt
]
    \item ایران یک \textbf{قدرت منطقه‌ای با 
    ویژگی‌های ژئوپلیتیکی منحصربه‌فرد} است: 
    بازیگر هسته‌ای، شبکه‌ی نیابتی منطقه‌ای، 
    کنترل‌کننده‌ی تنگه‌ی هرمز و میدان رقابت 
    قدرت‌های بزرگ.
    
    \item \textbf{تنوع قومی-زبانی} ایران هم سرمایه 
    و هم چالش است. مدیریت این تنوع در دوره‌ی 
    گذار حیاتی است — نه با حذف و نه با 
    سهمیه‌بندی قومی، بلکه با مدل‌های 
    خودمختاری و فدرالیسم مذاکره‌ای.
    
    \item \textbf{جوانی جمعیت و تحصیلات بالا} 
    سرمایه‌ی عظیمی برای گذار است، 
    اما بیکاری و انتظارات برآورده‌نشده 
    می‌تواند آن را به تهدید تبدیل کند.
    
    \item \textbf{دیاسپورای ایرانی} ظرفیت مالی، 
    فنی و دیپلماتیک دارد اما مشارکتش باید 
    ساختاریافته و تابع اراده‌ی داخل باشد.
    
    \item \textbf{ساختار دوگانه‌ی قدرت} 
    (جمهوری + تئوکراتیک) و نقش بی‌سابقه‌ی 
    \textbf{سپاه پاسداران} به‌عنوان بازیگر 
    نظامی-اقتصادی-سیاسی-ایدئولوژیک، 
    مهم‌ترین چالش‌های ساختاری گذار هستند.
    
    \item \textbf{هیچ مدل نظارتی موجودی} 
    بدون تطبیق اساسی برای ایران کار 
    نخواهد کرد. ایران به مدل سفارشی 
    نیاز دارد.
\end{enumerate}

\vspace{6pt}
\begin{center}
    \textcolor{MainGreen}{
        \faArrowLeft\hspace{8pt}
        \textbf{فصل بعد: رویکردها، ساختارها و 
        محدوده‌های نظارت — تحلیل مقایسه‌ای}
        \hspace{8pt}\faArrowLeft
    }
\end{center}

\end{chaptersummary}

\chapterend