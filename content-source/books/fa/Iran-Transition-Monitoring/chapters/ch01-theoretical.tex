% ╔══════════════════════════════════════════════════════════════════╗
% ║  فصل ۱: مبانی نظری و مفهومی                                    ║
% ║  گذار دموکراتیک و نظارت بین‌المللی                              ║
% ╚══════════════════════════════════════════════════════════════════╝

% ---- صفحه‌ی آغازین فصل ----
\chapteropening{۱}
    {مبانی نظری و مفهومی}
    {MainBlue}
    {دموکراسی بدترین شکل حکومت است، 
    به‌جز تمام شکل‌های دیگری که تاکنون آزموده شده‌اند.}
    {وینستون چرچیل}

\chapter{مبانی نظری و مفهومی: گذار دموکراتیک و نظارت بین‌المللی}
\label{ch:theoretical}
\minitoc

% ---- خلاصه‌ی اجرایی فصل ----
\begin{executivesummary}
این فصل سه وظیفه‌ی اصلی دارد: نخست، ایجاد 
\emphblue{زبان مشترک} از طریق تعریف دقیق مفاهیم کلیدی؛ 
دوم، مرور \emphblue{مکاتب فکری} مطالعات گذار دموکراتیک 
از دهه‌ی ۱۹۷۰ تا امروز؛ و سوم، تبیین 
\emphblue{مبانی نظری و حقوقی نظارت بین‌المللی} بر فرایند 
گذار. خواننده پس از مطالعه‌ی این فصل درکی روشن از 
«گذار دموکراتیک چیست؟»، «نظارت بین‌المللی چه معنایی 
دارد؟» و «چه چارچوب نظری‌ای برای تحلیل مورد ایران 
مناسب است؟» خواهد داشت.
\end{executivesummary}

% ============================================================
\section{گذار دموکراتیک: تعریف و مرزبندی مفهومی}
\label{sec:transition-definition}
% ============================================================

\begin{definitionbox}{گذار دموکراتیک}
\termfn{گذار دموکراتیک}{Democratic Transition} 
فاصله‌ی زمانی میان فروپاشی یا تضعیف یک نظام 
اقتدارگرا و استقرار یک نظام دموکراتیک است. این فرایند 
لزوماً خطی، یک‌سویه یا تضمین‌شده نیست و ممکن است 
به شکست، بازگشت یا رکود بینجامد.
\end{definitionbox}

برای درک دقیق‌تر، باید گذار دموکراتیک را از چند مفهوم 
مشابه اما متفاوت تمیز دهیم:

\begin{table}[htbp]
    \centering
    \caption{تمایز مفهومی: گذار دموکراتیک و مفاهیم مرتبط}
    \label{tab:concept-distinction}
    \tablefontsize
    \begin{tabularx}{\textwidth}{
        L{2.5cm} X X
    }
        \toprule
        \headerrow
        \textbf{مفهوم} & 
        \textbf{تعریف} & 
        \textbf{تفاوت با گذار دموکراتیک} \\
        \midrule
        
        \textbf{آزادسازی سیاسی}
        \newline\lr{\footnotesize Liberalization} &
        کاهش محدودیت‌های سیاسی و مدنی توسط نظام 
        اقتدارگرا بدون تغییر بنیادین ساختار قدرت &
        ممکن است بدون گذار واقعی رخ دهد؛ 
        ابزار نظام برای مدیریت فشار \\
        \altrow
        
        \textbf{دموکراتیزاسیون}
        \newline\lr{\footnotesize Democratization} &
        فرایند گسترده‌تر و بلندمدت‌تر ساختن نهادها و 
        فرهنگ دموکراتیک &
        گذار یک مقطع زمانی است؛ دموکراتیزاسیون 
        فرایندی مستمر \\
        
        \textbf{تحکیم}
        \newline\lr{\footnotesize Consolidation} &
        مرحله‌ای که دموکراسی نهادینه شده و بازگشت 
        به اقتدارگرایی بعید است &
        تحکیم پس از گذار می‌آید؛ بسیاری از کشورها 
        در گذار متوقف می‌شوند \\
        \altrow
        
        \textbf{تغییر رژیم}
        \newline\lr{\footnotesize Regime Change} &
        هرگونه تغییر بنیادین در ماهیت نظام سیاسی 
        (نه لزوماً به سمت دموکراسی) &
        تغییر رژیم می‌تواند از دیکتاتوری به 
        دیکتاتوری باشد \\
        
        \textbf{انقلاب}
        \newline\lr{\footnotesize Revolution} &
        تغییر سریع و بنیادین ساختار قدرت، اغلب 
        با خشونت یا بسیج توده‌ای &
        انقلاب یک \emph{مکانیزم} تغییر است؛ 
        گذار دموکراتیک یک \emph{مسیر} \\
        
        \bottomrule
    \end{tabularx}
\end{table}

\subsection{سه مرحله‌ی کلاسیک گذار}
\label{subsec:three-phases}

ادبیات کلاسیک 
(\person{اودانل و اشمیتر}{O'Donnell \& Schmitter}، ۱۹۸۶) 
گذار را به سه مرحله تقسیم می‌کند:

\begin{figure}[htbp]
    \centering
    \begin{tikzpicture}[
        phase/.style={
            draw, rounded corners=4pt,
            minimum height=2cm, minimum width=4cm,
            align=center, font=\small\bfseries,
            drop shadow={shadow xshift=1mm, shadow yshift=-1mm}
        },
        arr/.style={
            -{Stealth[length=3mm]}, ultra thick
        }
    ]
    
    % مراحل
    \node[phase, fill=LightBlue, draw=MainBlue] (lib) 
        {مرحله‌ی ۱\\[2pt]آزادسازی\\[2pt]
        \lr{\footnotesize Liberalization}};
    
    \node[phase, fill=LightGreen, draw=MainGreen, 
        right=2.5cm of lib] (trans) 
        {مرحله‌ی ۲\\[2pt]گذار\\[2pt]
        \lr{\footnotesize Transition}};
    
    \node[phase, fill=LightPurple, draw=MainPurple, 
        right=2.5cm of trans] (cons) 
        {مرحله‌ی ۳\\[2pt]تحکیم\\[2pt]
        \lr{\footnotesize Consolidation}};
    
    % فلش‌ها
    \draw[arr, MainBlue] (lib) -- (trans);
    \draw[arr, MainGreen] (trans) -- (cons);
    
    % فلش بازگشت
    \draw[arr, MainRed, dashed, bend right=40] 
        (trans.south) to 
        node[below, font=\footnotesize\color{MainRed}]
        {بازگشت اقتدارگرایانه} 
        (lib.south);
    
    \draw[arr, MainRed, dashed, bend right=30] 
        (cons.south east) to 
        node[below, font=\footnotesize\color{MainRed}, 
        text width=2.5cm, align=center]
        {پس‌رفت\\دموکراتیک} 
        ([yshift=-5mm]trans.south east);
    
    % برچسب‌های توضیحی
    \node[below=2.5cm of lib, text width=3.5cm, 
        align=center, font=\tiny\color{DarkGray}] {
        کاهش سرکوب\\
        آزادی نسبی مطبوعات\\
        آزادی زندانیان
    };
    
    \node[below=2.5cm of trans, text width=3.5cm, 
        align=center, font=\tiny\color{DarkGray}] {
        مذاکره یا فروپاشی\\
        انتخابات بنیادین\\
        قانون اساسی جدید
    };
    
    \node[below=2.5cm of cons, text width=3.5cm, 
        align=center, font=\tiny\color{DarkGray}] {
        نهادینه‌شدن قواعد\\
        فرهنگ دموکراتیک\\
        انتقال مسالمت‌آمیز قدرت
    };
    
    \end{tikzpicture}
    \caption{سه مرحله‌ی کلاسیک گذار دموکراتیک 
    و مسیرهای بازگشت}
    \label{fig:three-phases}
\end{figure}

\begin{lessonlearned}
\textbf{از تجربه‌ی مصر (۲۰۱۱-۲۰۱۳):}
مصر نشان داد که حتی پس از انقلاب پرشکوه و انتخابات آزاد 
(مرحله‌ی ۲)، بدون تحکیم نهادی (مرحله‌ی ۳) بازگشت 
به اقتدارگرایی نه‌تنها ممکن بلکه محتمل است. 
ارتش مصر در ژوئیه ۲۰۱۳ کودتا کرد و دموکراسی نوپا را 
در نطفه خفه کرد. دلیل اصلی: فقدان اجماع ملی بر 
قواعد بازی و نبود نظارت بین‌المللی مؤثر.
\end{lessonlearned}

% ============================================================
\section{مکاتب فکری مطالعات گذار}
\label{sec:schools-of-thought}
% ============================================================

مطالعات گذار دموکراتیک طی نیم‌قرن اخیر مسیر پرفراز 
و نشیبی را پیموده‌اند. در این بخش، مهم‌ترین مکاتب و 
نسل‌های فکری را مرور می‌کنیم.

\subsection{نسل اول: نظریه‌ی مدرنیزاسیون (دهه‌های ۱۹۵۰-۱۹۷۰)}
\label{subsec:modernization}

\person{سیمور مارتین لیپست}{Seymour Martin Lipset} 
در مقاله‌ی تأثیرگذار خود (۱۹۵۹) استدلال کرد که 
\emphblue{توسعه‌ی اقتصادی پیش‌شرط دموکراسی است}. 
بر اساس این دیدگاه، رشد اقتصادی منجر به گسترش طبقه‌ی 
متوسط، افزایش سطح تحصیلات و تقاضا برای مشارکت سیاسی 
می‌شود.

\person{ساموئل هانتینگتون}{Samuel Huntington} 
در کتاب بنیادین خود 
\emph{\lr{Political Order in Changing Societies}} (۱۹۶۸) 
هشدار داد که مدرنیزاسیون سریع بدون نهادسازی سیاسی 
به بی‌ثباتی و خشونت می‌انجامد — نکته‌ای که برای ایران 
بسیار آموزنده است.

\begin{table}[htbp]
    \centering
    \caption{خلاصه‌ی نظریه‌ی مدرنیزاسیون}
    \label{tab:modernization}
    \begin{tabularx}{\textwidth}{L{3cm} X}
        \toprule
        \headerrow
        \textbf{عنصر} & \textbf{توضیح} \\
        \midrule
        فرض بنیادین & 
        توسعه‌ی اقتصادی → طبقه‌ی متوسط → تقاضای 
        دموکراسی \\
        \altrow
        متفکران اصلی & 
        \lr{Lipset (1959)}, \lr{Rostow (1960)}, 
        \lr{Huntington (1968)} \\
        قوت‌ها & 
        همبستگی آماری قوی بین درآمد سرانه و دموکراسی؛ 
        توان پیش‌بینی بلندمدت \\
        \altrow
        ضعف‌ها & 
        نمی‌تواند «چرا الان؟» و «چگونه؟» را توضیح دهد؛ 
        موارد نقض فراوان (چین ثروتمند اما اقتدارگرا؛ 
        هند فقیر اما دموکراتیک) \\
        ربط به ایران & 
        ایران از نظر شاخص‌های مدرنیزاسیون (تحصیلات، 
        شهرنشینی، ارتباطات) آماده‌ی دموکراسی است، 
        اما ساختار سیاسی آن را مسدود کرده \\
        \bottomrule
    \end{tabularx}
\end{table}

\subsection{نسل دوم: مطالعات انتقال‌شناسی (دهه‌های ۱۹۸۰-۱۹۹۰)}
\label{subsec:transitology}

با موج سوم دموکراتیزاسیون 
(\person{هانتینگتون}{Huntington}، ۱۹۹۱) — 
سقوط دیکتاتوری‌ها در جنوب اروپا (پرتغال ۱۹۷۴، اسپانیا 
۱۹۷۵، یونان ۱۹۷۴)، آمریکای لاتین (دهه‌ی ۱۹۸۰) و 
اروپای شرقی (۱۹۸۹) — نسل جدیدی از مطالعات شکل 
گرفت که \termfn{انتقال‌شناسی}{Transitology} نام گرفت.

بنیان‌گذاران این مکتب 
\person{گیلرمو اودانل}{Guillermo O'Donnell} و 
\person{فیلیپه اشمیتر}{Philippe Schmitter} 
در اثر بنیادین خود 
\emph{\lr{Transitions from Authoritarian Rule}} (۱۹۸۶) 
چند اصل کلیدی مطرح کردند:

\begin{enumerate}[
    label=\textcolor{MainBlue}{\bfseries\arabic*.},
    itemsep=6pt
]
    \item \textbf{اصل عدم قطعیت 
    (\lr{Uncertainty}):} 
    نتیجه‌ی گذار از پیش معلوم نیست. بازیگران در شرایط 
    عدم اطمینان تصمیم می‌گیرند.
    
    \item \textbf{اصل عاملیت 
    (\lr{Agency over Structure}):} 
    انتخاب‌های نخبگان سیاسی — نه ساختارهای اقتصادی 
    یا اجتماعی — تعیین‌کننده‌ی نتیجه‌ی گذار هستند.
    
    \item \textbf{اصل پیمان‌سازی 
    (\lr{Pact-Making}):} 
    گذارهای موفق اغلب حاصل پیمان بین بخشی از نظام 
    قدیم (\lr{softliners}) و بخشی از اپوزیسیون 
    (\lr{moderates}) هستند.
    
    \item \textbf{اصل شکاف نخبگان 
    (\lr{Elite Splitting}):} 
    گذار زمانی ممکن می‌شود که بین تندروها 
    (\lr{hardliners}) و نرم‌روها 
    (\lr{softliners}) در درون نظام شکاف ایجاد شود.
\end{enumerate}

\person{خوان لینتز}{Juan Linz} و 
\person{آلفرد استپان}{Alfred Stepan} 
در \emph{\lr{Problems of Democratic Transition 
and Consolidation}} (۱۹۹۶) این تحلیل را عمیق‌تر 
کردند و پنج «عرصه‌ی تحکیم دموکراتیک» را 
شناسایی کردند:

\begin{figure}[htbp]
    \centering
    \begin{tikzpicture}[
        arena/.style={
            draw=MainBlue, fill=BlueBG, 
            rounded corners=3pt,
            minimum height=1.4cm, minimum width=3.5cm,
            align=center, font=\small\bfseries
        },
        center/.style={
            draw=MainPurple, fill=PurpleBG,
            circle, minimum size=2.5cm,
            align=center, font=\small\bfseries
        },
        conn/.style={thick, MainBlue!60}
    ]
    
    % مرکز
    \node[center] (dem) {تحکیم\\دموکراسی};
    
    % پنج عرصه
    \node[arena] (cs) at (90:4cm) 
        {جامعه‌ی مدنی\\
        \lr{\tiny Civil Society}};
    \node[arena] (ps) at (162:4cm) 
        {جامعه‌ی سیاسی\\
        \lr{\tiny Political Society}};
    \node[arena] (rl) at (234:4cm) 
        {حاکمیت قانون\\
        \lr{\tiny Rule of Law}};
    \node[arena] (sb) at (306:4cm) 
        {دستگاه دولتی\\
        \lr{\tiny State Bureaucracy}};
    \node[arena] (es) at (18:4cm) 
        {جامعه‌ی اقتصادی\\
        \lr{\tiny Economic Society}};
    
    % اتصالات
    \draw[conn] (dem) -- (cs);
    \draw[conn] (dem) -- (ps);
    \draw[conn] (dem) -- (rl);
    \draw[conn] (dem) -- (sb);
    \draw[conn] (dem) -- (es);
    
    % اتصالات بین عرصه‌ها
    \draw[conn, dashed, MainBlue!30] (cs) -- (ps);
    \draw[conn, dashed, MainBlue!30] (ps) -- (rl);
    \draw[conn, dashed, MainBlue!30] (rl) -- (sb);
    \draw[conn, dashed, MainBlue!30] (sb) -- (es);
    \draw[conn, dashed, MainBlue!30] (es) -- (cs);
    
    \end{tikzpicture}
    \caption{پنج عرصه‌ی تحکیم دموکراتیک 
    (لینتز و استپان، ۱۹۹۶)}
    \label{fig:five-arenas}
\end{figure}

\begin{keypoint}
مدل لینتز-استپان نشان می‌دهد که 
\emphblue{انتخابات آزاد به‌تنهایی دموکراسی نیست}. 
تحکیم واقعی مستلزم رشد هم‌زمان پنج عرصه است. 
نظارت بین‌المللی باید همه‌ی این عرصه‌ها را پوشش دهد، 
نه فقط انتخابات را.
\end{keypoint}

\subsection{نسل سوم: نقد و بازنگری (دهه‌ی ۲۰۰۰ به بعد)}
\label{subsec:critique}

از اوایل دهه‌ی ۲۰۰۰ موجی از نقد بر انتقال‌شناسی 
شکل گرفت. مهم‌ترین نقدها:

\subsubsection{نقد کاروترز: پایان پارادایم گذار}

\person{توماس کاروترز}{Thomas Carothers} 
در مقاله‌ی مشهور 
\emph{\lr{The End of the Transition Paradigm}} 
(۲۰۰۲) استدلال کرد که اکثر کشورهای به‌اصطلاح 
«در حال گذار» در واقع در یک 
\termfn{منطقه‌ی خاکستری}{Gray Zone} 
گیر کرده‌اند — نه کاملاً اقتدارگرا و نه واقعاً دموکراتیک.

\subsubsection{اقتدارگرایی رقابتی}

\person{استیون لویتسکی}{Steven Levitsky} و 
\person{لوکان وِی}{Lucan Way} 
مفهوم 
\termfn{اقتدارگرایی رقابتی}{Competitive Authoritarianism} 
را معرفی کردند (۲۰۱۰): نظام‌هایی که انتخابات برگزار 
می‌کنند اما زمین بازی کاملاً ناعادلانه است.

\begin{warningbox}
\textbf{ربط مستقیم به ایران:}
جمهوری اسلامی ایران نمونه‌ی بارز 
\emphred{اقتدارگرایی انتخاباتی} است — 
نظامی که ظاهر انتخاباتی دارد اما نامزدها 
پیش‌دستی توسط شورای نگهبان فیلتر می‌شوند، 
رسانه‌ها کنترل می‌شوند و اپوزیسیون واقعی 
اجازه‌ی فعالیت ندارد. هرگونه نظارت بین‌المللی 
باید از این واقعیت آغاز کند.
\end{warningbox}

\subsubsection{بازگشت اقتدارگرایانه}

\person{لری دایموند}{Larry Diamond} 
از مفهوم 
\termfn{رکود دموکراتیک}{Democratic Recession} 
(۲۰۱۵) سخن گفت: پس از سه دهه گسترش، 
دموکراسی در جهان عقب‌نشینی کرده است. مجارستان، 
ترکیه، روسیه و ونزوئلا نمونه‌هایی هستند.

\begin{table}[htbp]
    \centering
    \caption{خلاصه‌ی سه نسل مطالعات گذار دموکراتیک}
    \label{tab:three-generations}
    \tablefontsize
    \begin{tabularx}{\textwidth}{
        L{2cm} C{2cm} X X L{2.5cm}
    }
        \toprule
        \headerrow
        \textbf{نسل} & 
        \textbf{دوره} & 
        \textbf{پرسش محوری} & 
        \textbf{پاسخ} &
        \textbf{متفکران} \\
        \midrule
        اول: مدرنیزاسیون &
        ۵۰-۷۰ &
        \emph{چه شرایطی} دموکراسی را ممکن می‌کند؟ &
        توسعه‌ی اقتصادی و اجتماعی &
        \lr{Lipset, Huntington} \\
        \altrow
        دوم: انتقال‌شناسی &
        ۸۰-۹۰ &
        \emph{چگونه} گذار رخ می‌دهد و \emph{چه کسی} 
        آن را هدایت می‌کند؟ &
        انتخاب‌های نخبگان، پیمان‌سازی &
        \lr{O'Donnell, Schmitter, Linz, Stepan} \\
        سوم: نقد و بازنگری &
        ۲۰۰۰+ &
        \emph{چرا} بسیاری از گذارها شکست 
        می‌خورند؟ &
        منطقه‌ی خاکستری، اقتدارگرایی جدید، 
        عوامل ساختاری &
        \lr{Carothers, Levitsky, Diamond} \\
        \bottomrule
    \end{tabularx}
\end{table}

\sectiondivider

% ============================================================
\section{نظارت بین‌المللی: مفهوم، تکامل و انواع}
\label{sec:monitoring-concept}
% ============================================================

\begin{definitionbox}{نظارت بین‌المللی بر گذار}
حضور سازمان‌یافته‌ی نهادها و بازیگران بین‌المللی 
در فرایند گذار یک کشور با هدف \emphblue{مشاهده، 
ارزیابی، گزارش‌دهی و در برخی موارد تسهیل‌گری یا 
تضمین} احترام به اصول دموکراتیک، حقوق بشر و 
حاکمیت قانون.
\end{definitionbox}

\subsection{تکامل تاریخی نظارت بین‌المللی}
\label{subsec:monitoring-evolution}

نظارت بین‌المللی بر فرایندهای سیاسی داخلی 
کشورها پدیده‌ای نسبتاً جدید است:

\begin{figure}[htbp]
    \centering
    \begin{tikzpicture}[
        era/.style={
            draw, rounded corners=2pt,
            minimum height=1.2cm, minimum width=2.8cm,
            align=center, font=\tiny\bfseries
        },
        timeline/.style={ultra thick, gray!50}
    ]
    
    % خط زمان
    \draw[timeline, -{Stealth}] (0,0) -- (15,0);
    
    % علامت‌های زمانی
    \foreach \x/\y in {
        1/۱۹۴۵, 3/۱۹۶۰, 5/۱۹۷۵, 
        7/۱۹۸۹, 9/۲۰۰۰, 11/۲۰۱۱, 13/۲۰۲۰
    } {
        \draw[thick] (\x, 0.15) -- (\x, -0.15) 
            node[below, font=\tiny] {\y};
    }
    
    % دوره‌ها
    \node[era, fill=BlueBG, draw=MainBlue] at (2, 1.5) {
        تأسیس UN\\حاکمیت مطلق};
    
    \node[era, fill=BlueBG, draw=MainBlue] at (4, -1.5) {
        استعمارزدایی\\خودمختاری};
    
    \node[era, fill=GreenBG, draw=MainGreen] at (6, 1.5) {
        نظارت اولیه\\اروپای جنوبی};
    
    \node[era, fill=GreenBG, draw=MainGreen] at (8, -1.5) {
        پایان جنگ سرد\\موج سوم};
    
    \node[era, fill=OrangeBG, draw=MainOrange] at (10, 1.5) {
        نظارت حرفه‌ای\\مأموریت‌های UN};
    
    \node[era, fill=OrangeBG, draw=MainOrange] at (12, -1.5) {
        بهار عربی\\چالش‌های جدید};
    
    \node[era, fill=RedBG, draw=MainRed] at (14, 1.5) {
        رکود دموکراتیک\\بازنگری};
    
    \end{tikzpicture}
    \caption{تکامل تاریخی نظارت بین‌المللی بر گذار}
    \label{fig:monitoring-evolution}
\end{figure}

\subsection{انواع نظارت بین‌المللی}
\label{subsec:monitoring-types}

نظارت بین‌المللی طیف گسترده‌ای دارد. 
مهم است که این انواع را از هم تمیز دهیم 
زیرا هر یک ابزارها، نیروی انسانی و 
پیامدهای متفاوتی دارد:

\begin{table}[htbp]
    \centering
    \caption{انواع نظارت بین‌المللی بر گذار}
    \label{tab:monitoring-types}
    \tablefontsize
    \begin{tabularx}{\textwidth}{
        L{2.2cm} X C{2cm} C{1.8cm}
    }
        \toprule
        \headerrow
        \textbf{نوع نظارت} & 
        \textbf{توضیح} & 
        \textbf{مثال تاریخی} &
        \textbf{تهاجم به حاکمیت} \\
        \midrule
        
        \textbf{نظارت انتخاباتی}
        \newline\lr{\tiny Election Monitoring} &
        مشاهده و ارزیابی فرایند انتخابات از ثبت‌نام 
        تا شمارش آرا &
        ناظران \lr{OSCE} در اروپا &
        \cellgreen{پایین} \\
        \altrow
        
        \textbf{نظارت حقوق بشری}
        \newline\lr{\tiny HR Monitoring} &
        مستندسازی و گزارش‌دهی نقض حقوق بشر &
        \lr{OHCHR} در کلمبیا &
        \cellgreen{پایین-متوسط} \\
        
        \textbf{نظارت مشورتی}
        \newline\lr{\tiny Advisory Monitoring} &
        ارائه‌ی مشاوره‌ی فنی بدون قدرت اجرایی &
        \lr{Venice Commission} &
        \cellgreen{پایین} \\
        \altrow
        
        \textbf{نظارت ساختاری}
        \newline\lr{\tiny Structural Oversight} &
        نظارت بر اصلاح نهادها، قانون اساسی 
        و بخش امنیتی &
        \lr{EU} در اروپای شرقی &
        \cellorange{متوسط} \\
        
        \textbf{نظارت اجرایی}
        \newline\lr{\tiny Executive Oversight} &
        قدرت اجرایی محدود در برخی حوزه‌ها 
        (مثلاً امنیت یا مالیه) &
        \lr{UNTAET} تیمور شرقی &
        \cellorange{بالا} \\
        \altrow
        
        \textbf{مدیریت مستقیم}
        \newline\lr{\tiny Direct Administration} &
        کنترل کامل بین‌المللی بر حکمرانی &
        \lr{CPA} عراق &
        \cellred{بسیار بالا} \\
        
        \bottomrule
    \end{tabularx}
\end{table}

\begin{recommendation}
\textbf{برای ایران:}
هیچ‌یک از انواع فوق به‌تنهایی مناسب ایران نیست. 
مدل‌های ابتدایی (نظارت انتخاباتی صرف) ناکافی‌اند 
و مدل‌های انتهایی (مدیریت مستقیم) نه ممکن‌اند و 
نه مطلوب. فصل ۳ مدل ترکیبی-تطبیقی پیشنهادی 
را تشریح خواهد کرد. (\seeChapter{ch:approaches})
\end{recommendation}

\sectiondivider

% ============================================================
\section{مبانی حقوقی نظارت بین‌المللی}
\label{sec:legal-foundations}
% ============================================================

نظارت بین‌المللی در تنش دائمی با اصل 
\termfn{حاکمیت ملی}{National Sovereignty} 
قرار دارد. این بخش مبانی حقوقی‌ای را بررسی 
می‌کند که چنین نظارتی را مشروع و قانونی می‌سازد.

\subsection{حاکمیت ملی: مطلق یا مشروط؟}

تعریف کلاسیک حاکمیت (وستفالیایی، ۱۶۴۸) 
هرگونه مداخله‌ی خارجی در امور داخلی را 
ممنوع می‌داند. اما تحولات قرن بیستم و بیست‌ویکم 
این مفهوم را اصلاح کرده‌اند:

\begin{enumerate}[
    label=\textcolor{MainBlue}{\bfseries\alph*)},
    itemsep=6pt
]
    \item \textbf{اعلامیه‌ی جهانی حقوق بشر (۱۹۴۸):}
    حقوق بشر فراتر از مرزهای ملی است.
    
    \item \textbf{میثاقین بین‌المللی (۱۹۶۶):}
    حق تعیین سرنوشت و حق مشارکت سیاسی 
    جزو حقوق بنیادین انسان‌هاست.
    
    \item \textbf{مسئولیت حمایت 
    (\lr{R2P}، ۲۰۰۵):}
    اگر دولتی از حمایت شهروندانش در برابر 
    نسل‌کشی، جرایم جنگی، پاکسازی قومی و 
    جرایم علیه بشریت ناتوان یا ناخواسته باشد، 
    مسئولیت به جامعه‌ی بین‌المللی منتقل می‌شود.
    
    \item \textbf{رویه‌ی عملی شورای امنیت:}
    صدور قطعنامه‌های متعدد برای تأسیس 
    مأموریت‌های نظارتی (کامبوج، تیمور شرقی، 
    کوزوو، لیبی و...).
\end{enumerate}

\begin{table}[htbp]
    \centering
    \caption{تکامل مفهوم حاکمیت در حقوق بین‌الملل}
    \label{tab:sovereignty-evolution}
    \begin{tabularx}{\textwidth}{
        L{2.5cm} C{2cm} X
    }
        \toprule
        \headerrow
        \textbf{دوره} & 
        \textbf{مفهوم حاکمیت} & 
        \textbf{پیامد برای نظارت} \\
        \midrule
        
        وستفالیایی (۱۶۴۸+) &
        مطلق &
        هرگونه نظارت = مداخله \\
        \altrow
        
        پس از جنگ دوم (۱۹۴۵+) &
        مشروط به حقوق بشر &
        نظارت حقوق بشری مشروع \\
        
        پس از جنگ سرد (۱۹۸۹+) &
        مسئولیت‌محور &
        نظارت انتخاباتی و ساختاری \\
        \altrow
        
        قرن ۲۱ (۲۰۰۵+) &
        مسئولیت حمایت &
        نظارت جامع + مداخله در 
        موارد استثنایی \\
        
        \bottomrule
    \end{tabularx}
\end{table}

\subsection{حق مردم بر نظارت بین‌المللی}

مفهوم نوظهوری در حال شکل‌گیری است: 
\emphblue{حق مردم — نه دولت — بر درخواست 
نظارت بین‌المللی}. این مفهوم بر آن است که 
وقتی حکومتی مشروعیت مردمی ندارد، 
\emph{مردم} حق دارند از جامعه‌ی بین‌المللی 
بخواهند که بر فرایند تغییر نظارت کند.

\begin{reflectionbox}
\textbf{پرسش تأملی:}
آیا مردم ایران — که در خیزش‌های پی‌درپی خواست 
خود برای تغییر را اعلام کرده‌اند — حق دارند 
مستقیماً از جامعه‌ی بین‌المللی درخواست نظارت کنند، 
حتی اگر حکومت مخالف باشد؟ مبنای حقوقی چنین 
درخواستی چیست؟
\end{reflectionbox}

\sectiondivider

% ============================================================
\section{چارچوب تحلیلی این کتاب}
\label{sec:analytical-framework}
% ============================================================

با توجه به مرور مکاتب فکری و انواع نظارت، 
این کتاب از یک \emphblue{چارچوب تحلیلی سه‌بعدی} 
استفاده می‌کند که سه متغیر اصلی را در تعامل با 
یکدیگر بررسی می‌کند:

\begin{enumerate}[
    label=\textcolor{MainBlue}{\bfseries بُعد \arabic*:},
    itemsep=8pt
]
    \item \textbf{عمق نظارت 
    (\lr{Depth of Monitoring}):}
    از نظارت سطحی انتخاباتی تا مدیریت مستقیم 
    بین‌المللی — نظارت چقدر «عمیق» در ساختار 
    قدرت نفوذ می‌کند؟
    
    \item \textbf{گستره‌ی نظارت 
    (\lr{Scope of Monitoring}):}
    چه حوزه‌هایی تحت نظارت قرار می‌گیرند؟ 
    فقط انتخابات؟ یا امنیت، اقتصاد، حقوق بشر، 
    قانون اساسی و رسانه هم؟
    
    \item \textbf{مدت نظارت 
    (\lr{Duration of Monitoring}):}
    نظارت چقدر طول می‌کشد؟ سه ماه؟ سه سال؟ 
    ده سال؟ مدت زمان رابطه‌ی مستقیم با موفقیت 
    تحکیم دموکراسی دارد.
\end{enumerate}

\begin{figure}[htbp]
    \centering
    \begin{tikzpicture}[
        scale=1.2,
        axis/.style={-{Stealth}, thick, MainBlue},
        label/.style={font=\small\bfseries, MainBlue}
    ]
    
    % محورها
    \draw[axis] (0,0) -- (5.5,0) 
        node[right, label] {گستره};
    \draw[axis] (0,0) -- (0,5.5) 
        node[above, label] {عمق};
    \draw[axis] (0,0) -- (-3,-2.5) 
        node[below left, label] {مدت};
    
    % مقیاس عمق
    \foreach \y/\t in {
        1/انتخاباتی,
        2/مشورتی,
        3/ساختاری,
        4/اجرایی,
        5/مدیریت مستقیم
    } {
        \draw[gray!40] (-0.1,\y) -- (0.1,\y);
        \node[left, font=\tiny, text=MediumGray] 
            at (-0.2,\y) {\t};
    }
    
    % نقاط نمونه‌های تاریخی
    \filldraw[MainGreen, opacity=0.8] (3,2) circle (4pt) 
        node[right, font=\tiny, black] 
        {آفریقای جنوبی};
    \filldraw[MainGreen, opacity=0.8] (2,1.5) circle (4pt) 
        node[right, font=\tiny, black] 
        {شیلی};
    \filldraw[MainOrange, opacity=0.8] (4,3.5) circle (4pt) 
        node[right, font=\tiny, black] 
        {تیمور شرقی};
    \filldraw[MainRed, opacity=0.8] (5,5) circle (4pt) 
        node[right, font=\tiny, black] 
        {عراق (CPA)};
    \filldraw[MainBlue, opacity=0.8] (2,1) circle (4pt) 
        node[right, font=\tiny, black] 
        {تونس};
    
    % ناحیه‌ی پیشنهادی ایران
    \draw[MainPurple, ultra thick, dashed, 
        fill=MainPurple!10, opacity=0.5] 
        (2.5,2) -- (4.5,3) -- (4,4) -- (2,3) -- cycle;
    \node[font=\small\bfseries, MainPurple] 
        at (3.2,3) {ایران؟};
    
    \end{tikzpicture}
    \caption{چارچوب تحلیلی سه‌بعدی: 
    جایگاه نمونه‌های تاریخی و محدوده‌ی پیشنهادی 
    برای ایران}
    \label{fig:3d-framework}
\end{figure}

\subsection{ترکیب چارچوب تحلیلی با مکاتب فکری}

چارچوب سه‌بعدی فوق با سه بینش نظری ترکیب می‌شود:

\begin{table}[htbp]
    \centering
    \caption{ترکیب چارچوب تحلیلی با بینش‌های نظری}
    \label{tab:framework-theory}
    \begin{tabularx}{\textwidth}{
        L{2.5cm} X L{3.5cm}
    }
        \toprule
        \headerrow
        \textbf{بینش نظری} & 
        \textbf{کاربرد در این کتاب} & 
        \textbf{فصول مرتبط} \\
        \midrule
        
        مدرنیزاسیون &
        ارزیابی آمادگی ساختاری ایران: آیا جامعه‌ی 
        ایرانی از نظر تحصیلات، شهرنشینی و طبقه‌ی 
        متوسط «آماده» است؟ &
        فصل ۲ \\
        \altrow
        
        انتقال‌شناسی &
        تحلیل سناریوهای گذار: نقش نخبگان، 
        شکاف درون نظام، پیمان‌سازی &
        فصول ۴ و ۵ \\
        
        نقد و بازنگری &
        آسیب‌شناسی: ریسک‌های بازگشت 
        اقتدارگرایانه، منطقه‌ی خاکستری، 
        شکست‌های نظارت &
        فصل ۷ \\
        
        \bottomrule
    \end{tabularx}
\end{table}

\subsection{مدل تحلیلی DIME+H}

برای سازماندهی تحلیل‌ها در فصول آتی، 
از مدل \lr{DIME+H} استفاده می‌کنیم که 
پنج حوزه‌ی کلیدی گذار را پوشش می‌دهد:

\begin{figure}[htbp]
    \centering
    \begin{tikzpicture}[
        hex/.style={
            regular polygon, regular polygon sides=6,
            draw=#1, fill=#1!10,
            minimum size=2.5cm,
            font=\small\bfseries,
            align=center,
            inner sep=0pt
        }
    ]
    
    % شش‌ضلعی مرکزی
    \node[hex=MainPurple] (center) at (0,0) 
        {گذار\\دموکراتیک\\ایران};
    
    % پنج حوزه
    \node[hex=MainBlue] (D) at (90:3.2cm) 
        {D\\[2pt]دیپلماسی\\
        \lr{\tiny Diplomacy}};
    
    \node[hex=MainRed] (I) at (162:3.2cm) 
        {I\\[2pt]اطلاعات\\
        \lr{\tiny Information}};
    
    \node[hex=MainGreen] (M) at (234:3.2cm) 
        {M\\[2pt]نظامی\\
        \lr{\tiny Military}};
    
    \node[hex=MainOrange] (E) at (306:3.2cm) 
        {E\\[2pt]اقتصاد\\
        \lr{\tiny Economy}};
    
    \node[hex=DarkYellow] (H) at (18:3.2cm) 
        {H\\[2pt]انسانی\\
        \lr{\tiny Human}};
    
    % خطوط اتصال
    \foreach \n in {D,I,M,E,H} {
        \draw[gray!40, thick] (center) -- (\n);
    }
    
    \end{tikzpicture}
    \caption{مدل تحلیلی \lr{DIME+H} 
    برای بررسی ابعاد نظارت بر گذار}
    \label{fig:dimeh-model}
\end{figure}

\begin{operationalnote}
هر فصل از فصل ۳ به بعد، تحلیل‌های خود را 
بر مبنای این پنج حوزه سازمان‌دهی خواهد کرد 
تا خواننده بتواند ابعاد مختلف هر موضوع را 
به‌صورت نظام‌مند بررسی کند.
\end{operationalnote}

\sectiondivider

% ============================================================
\section{موج‌های دموکراتیزاسیون و جایگاه ایران}
\label{sec:waves}
% ============================================================

\person{ساموئل هانتینگتون}{Samuel Huntington} 
در کتاب \emph{\lr{The Third Wave}} (۱۹۹۱) 
سه موج بزرگ دموکراتیزاسیون را شناسایی کرد. 
برخی پژوهشگران از موج چهارم نیز سخن گفته‌اند:

\begin{table}[htbp]
    \centering
    \caption{موج‌های دموکراتیزاسیون و موج‌های 
    بازگشت اقتدارگرایانه}
    \label{tab:waves}
    \tablefontsize
    \begin{tabularx}{\textwidth}{
        C{1.2cm} C{2cm} X X C{1.5cm}
    }
        \toprule
        \headerrow
        \textbf{موج} & 
        \textbf{دوره} & 
        \textbf{نمونه‌ها} & 
        \textbf{موج بازگشت} &
        \textbf{ایران} \\
        \midrule
        
        اول &
        ۱۸۲۸-۱۹۲۶ &
        آمریکا، بریتانیا، فرانسه، 
        استرالیا، کانادا &
        ۱۹۲۲-۱۹۴۲: فاشیسم در اروپا &
        مشروطه ۱۲۸۵ (ناتمام) \\
        \altrow
        
        دوم &
        ۱۹۴۳-۱۹۶۲ &
        آلمان، ژاپن، ایتالیا (پس از جنگ)، 
        هند، اسرائیل &
        ۱۹۵۸-۱۹۷۵: کودتاها در 
        آفریقا و آمریکای لاتین &
        ملی‌شدن نفت ۱۳۳۲ → کودتای ۲۸ مرداد \\
        
        سوم &
        ۱۹۷۴-۱۹۹۵ &
        پرتغال، اسپانیا، آمریکای لاتین، 
        اروپای شرقی، آفریقای جنوبی &
        ۱۹۹۵-۲۰۱۰: روسیه، 
        اقتدارگرایی جدید &
        انقلاب ۱۳۵۷: ضددموکراتیک \\
        \altrow
        
        چهارم؟ &
        ۲۰۱۰-؟ &
        تونس، (بهار عربی)، 
        سودان ۲۰۱۹؟ &
        مصر ۲۰۱۳، لیبی، سوریه، 
        سودان ۲۰۲۳ &
        \cellcolor{PurpleBG}
        \textbf{۱۴۰۱-؟} \\
        
        \bottomrule
    \end{tabularx}
\end{table}

\begin{keypoint}
ایران در تمام موج‌های دموکراتیزاسیون حضور داشته 
اما هر بار ناکام مانده است: مشروطه (۱۲۸۵) → استبداد 
صغیر و سپس رضاخان؛ ملی‌شدن نفت (۱۳۳۲) → 
کودتای ۲۸ مرداد؛ انقلاب ۱۳۵۷ → جمهوری اسلامی. 
\emphblue{آیا خیزش «زن، زندگی، آزادی» آغازگر 
موفقیت ایران در موج چهارم خواهد بود؟} پاسخ تا حد 
زیادی به طراحی درست فرایند گذار و نظارت بر آن 
بستگی دارد.
\end{keypoint}

\sectiondivider

% ============================================================
\section{جمع‌بندی فصل و پل به فصل بعد}
\label{sec:ch1-summary}
% ============================================================

\begin{chaptersummary}

\textbf{آنچه در این فصل آموختیم:}

\begin{enumerate}[
    label=\textcolor{DarkGray}{\bfseries\arabic*.},
    itemsep=4pt
]
    \item \textbf{گذار دموکراتیک} فرایندی غیرخطی، 
    نامطمئن و چندبُعدی است — نه یک رویداد واحد.
    
    \item \textbf{سه نسل فکری} مطالعات گذار 
    وجود دارد: مدرنیزاسیون (چرا؟)، انتقال‌شناسی 
    (چگونه؟) و نقد (چرا شکست؟). هر سه برای 
    تحلیل ایران لازم‌اند.
    
    \item \textbf{نظارت بین‌المللی} طیف گسترده‌ای 
    دارد: از نظارت انتخاباتی ساده تا مدیریت مستقیم 
    بین‌المللی. انتخاب مدل مناسب حیاتی است.
    
    \item \textbf{مبانی حقوقی} نظارت تکامل یافته‌اند: 
    حاکمیت مطلق جای خود را به حاکمیت مشروط و 
    مسئولیت حمایت داده است.
    
    \item \textbf{چارچوب تحلیلی سه‌بعدی} 
    (عمق × گستره × مدت) به همراه مدل \lr{DIME+H} 
    ابزار تحلیل این کتاب خواهند بود.
    
    \item \textbf{ایران در تمام موج‌های دموکراتیزاسیون 
    حضور داشته} اما هر بار ناکام مانده. 
    فصل بعد به «چرایی» و «ویژگی‌های خاص ایران» 
    خواهد پرداخت.
\end{enumerate}

\vspace{6pt}
\begin{center}
    \textcolor{MainBlue}{
        \faArrowLeft\hspace{8pt}
        \textbf{فصل بعد: چرا ایران؟ 
        ویژگی‌ها، پیچیدگی‌ها و استثنائات}
        \hspace{8pt}\faArrowLeft
    }
\end{center}

\end{chaptersummary}

\chapterend