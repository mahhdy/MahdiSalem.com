% ═══════════════════════════════════════════════════════════════════════════════
% فصل ۱۰: بودجه‌بندی و تأمین مالی
% فایل: chapters/ch10-budget.tex
% رنگ فصل: زرد (MainYellow)
% ═══════════════════════════════════════════════════════════════════════════════

\chapteropening{۱۰}{بودجه‌بندی و تأمین مالی}{MainYellow}{%
اگر فکر می‌کنید هزینه صلح گران است، هزینه جنگ را حساب کنید.%
}{کوفی عنان}

\chapter{بودجه‌بندی و تأمین مالی}
\label{ch:budget}

\minitoc

% ─────────────────────────────────────────────────────────────────────────────
% خلاصه اجرایی
% ─────────────────────────────────────────────────────────────────────────────

\begin{executivesummary}
برآورد کل بودجه نظارت بین‌المللی بر گذار ایران در بازه ۱۰ ساله، \textbf{۲.۵ تا ۵ میلیارد دلار} است — رقمی قابل‌توجه اما در مقایسه با هزینه‌های مداخله نظامی (عراق: ۶۰+ میلیارد دلار فقط برای بازسازی) یا هزینه عدم اقدام (بحران پناهندگی، تروریسم، بی‌ثباتی منطقه‌ای) بسیار معقول. این فصل بودجه را به تفکیک فاز، بخش، و نوع هزینه ارائه می‌دهد. منابع مالی شامل مشارکت سازمان ملل (۳۰٪)، اتحادیه اروپا (۲۰٪)، آمریکا (۱۵٪)، ژاپن/کره (۱۰٪)، کشورهای خلیج فارس (۱۰٪)، دارایی‌های آزادشده ایران (۱۰٪)، و بخش خصوصی (۵٪) است. شفافیت مالی و حسابرسی مستقل، ستون‌های اعتمادسازی در این حوزه‌اند.
\end{executivesummary}

\section{درآمد: اقتصاد صلح‌سازی}
\label{sec:budget-intro}

بودجه‌بندی نظارت بین‌المللی صرفاً یک تمرین حسابداری نیست؛ بیان \emph{اولویت‌های استراتژیک} در قالب ارقام است. تخصیص منابع مالی نشان می‌دهد که جامعه بین‌المللی واقعاً چه چیزی را مهم می‌داند.

\begin{keypoint}
\textbf{اقتصاد پیشگیری}: هر ۱ دلار سرمایه‌گذاری در پیشگیری از بحران، ۱۶ دلار در هزینه‌های واکنش صرفه‌جویی می‌کند.\footnote{\lr{United Nations/World Bank. \emph{Pathways for Peace: Inclusive Approaches to Preventing Violent Conflict}. 2018.}} نظارت مؤثر بر گذار ایران، سرمایه‌گذاری در پیشگیری از بحرانی است که می‌تواند ده‌ها میلیارد دلار و صدها هزار جان انسانی هزینه داشته باشد.
\end{keypoint}

\begin{table}[htbp]
\centering
\caption{مقایسه هزینه‌های مداخله بین‌المللی در نمونه‌های تاریخی}
\label{tab:cost-comparison}
\begin{tabularx}{\textwidth}{>{\raggedleft\arraybackslash}p{3cm}
                             >{\centering\arraybackslash}p{2cm}
                             >{\centering\arraybackslash}p{2cm}
                             >{\centering\arraybackslash}p{2cm}
                             >{\raggedleft\arraybackslash}X}
\toprule
\headerrow کشور/مأموریت & جمعیت (M) & مدت (سال) & هزینه کل (\$B) & هزینه سرانه \\
\midrule
عراق (بازسازی) & ۳۸ & ۸ & ۶۰+ & ۱,۵۸۰ \\
\altrow افغانستان (کل) & ۳۸ & ۲۰ & ۱۴۵+ & ۳,۸۰۰ \\
تیمور شرقی (\lr{UNTAET}) & ۰.۸ & ۳ & ۳ & ۳,۷۵۰ \\
\altrow کوزوو (\lr{UNMIK}) & ۱.۸ & ۲۰+ & ۵+ & ۲,۸۰۰ \\
بوسنی (\lr{OHR}+\lr{SFOR}) & ۳.۵ & ۲۵+ & ۱۵+ & ۴,۳۰۰ \\
\altrow کامبوج (\lr{UNTAC}) & ۱۰ & ۲ & ۱.۶ & ۱۶۰ \\
\midrule
\textbf{ایران (پیشنهادی)} & \textbf{۸۵} & \textbf{۱۰} & \textbf{۲.۵-۵} & \textbf{۳۰-۶۰} \\
\bottomrule
\end{tabularx}
\end{table}

\begin{lessonlearned}{عراق: هزینه عدم برنامه‌ریزی}
آمریکا بیش از ۶۰ میلیارد دلار صرف «بازسازی» عراق کرد، اما حسابرسی \lr{SIGIR} نشان داد که حداقل ۸ میلیارد دلار به هدر رفته یا سرقت شده است. دلیل: عدم برنامه‌ریزی پیشینی، نبود مکانیزم نظارت مالی، و عجله سیاسی. هزینه سرانه بازسازی عراق (۱,۵۸۰ دلار) ۲۵-۵۰ برابر بیشتر از برآورد ایران (۳۰-۶۰ دلار) است — زیرا مدل ایران مبتنی بر نظارت است نه مداخله.
\end{lessonlearned}

\sectiondivider

% ═══════════════════════════════════════════════════════════════════════════════
\section{برآورد بودجه به تفکیک فاز}
\label{sec:budget-by-phase}
% ═══════════════════════════════════════════════════════════════════════════════

\subsection{بودجه کلی فازها}
\label{subsec:phase-budgets}

\begin{table}[htbp]
\centering
\caption{برآورد بودجه به تفکیک فاز (میلیون دلار)}
\label{tab:budget-by-phase}
\begin{tabularx}{\textwidth}{>{\raggedleft\arraybackslash}p{3cm}
                             >{\centering\arraybackslash}p{2cm}
                             >{\centering\arraybackslash}p{2cm}
                             >{\centering\arraybackslash}p{2cm}
                             >{\raggedleft\arraybackslash}X}
\toprule
\headerrow فاز & مدت & حداقل (\$M) & حداکثر (\$M) & توضیح \\
\midrule
فاز ۰ (پیش‌گذار) & ۶-۱۲ ماه & ۵۰ & ۱۰۰ & برنامه‌ریزی، آموزش، پیش‌موقعیت \\
\altrow فاز ۱ (تثبیت) & ۶ ماه & ۴۰۰ & ۸۰۰ & استقرار سریع، امنیت، نظم اولیه \\
فاز ۲ (نهادسازی) & ۱۸ ماه & ۸۰۰ & ۱,۵۰۰ & انتخابات، قانون اساسی، نهادها \\
\altrow فاز ۳ (تحکیم) & ۳۶ ماه & ۶۰۰ & ۱,۲۰۰ & تحکیم، انتقال، انتخابات دوم \\
فاز ۴ (پی‌گیری) & ۶۰ ماه & ۱۵۰ & ۴۰۰ & مشاوره، ارزیابی، خروج \\
\midrule
\textbf{مجموع} & \textbf{۱۰ سال} & \textbf{۲,۰۰۰} & \textbf{۴,۰۰۰} & \\
\altrow \textbf{+ ذخیره احتیاطی ۲۵٪} & & \textbf{۵۰۰} & \textbf{۱,۰۰۰} & پوشش ریسک و تغییرات \\
\midrule
\textbf{مجموع نهایی} & & \textbf{۲,۵۰۰} & \textbf{۵,۰۰۰} & \\
\bottomrule
\end{tabularx}
\end{table}

\begin{casestudy}{ذخیره احتیاطی: درس افغانستان}
بودجه اولیه \lr{ISAF} در افغانستان بارها بازنگری شد — هر بار افزایش. عوامل غیرقابل‌پیش‌بینی (شورش طالبان، فساد، بلایای طبیعی) بودجه را ۲-۳ برابر تخمین اولیه بالا بردند. ذخیره احتیاطی ۲۵٪ ضروری و حداقلی است.
\end{casestudy}

\subsection{نمودار توزیع زمانی بودجه}
\label{subsec:budget-timeline-chart}

\begin{figure}[htbp]
\centering
\begin{tikzpicture}
\begin{axis}[
    ybar,
    width=14cm,
    height=7cm,
    bar width=1.2cm,
    ylabel={میلیون دلار},
    xlabel={سال},
    symbolic x coords={سال ۰, سال ۱, سال ۲, سال ۳, سال ۴, سال ۵, سال ۶-۷, سال ۸-۱۰},
    xtick=data,
    x tick label style={font=\footnotesize},
    y tick label style={font=\footnotesize},
    ymin=0,
    ymax=1100,
    nodes near coords,
    every node near coord/.append style={font=\tiny},
    legend style={at={(0.5,-0.25)}, anchor=north, legend columns=2, font=\footnotesize},
    axis lines=left,
    enlarge x limits=0.1,
]
    \addplot[fill=MainRed!60] coordinates {
        (سال ۰, 75) (سال ۱, 800) (سال ۲, 600) (سال ۳, 400) 
        (سال ۴, 300) (سال ۵, 250) (سال ۶-۷, 150) (سال ۸-۱۰, 100)
    };
    \addplot[fill=MainYellow!60] coordinates {
        (سال ۰, 0) (سال ۱, 200) (سال ۲, 300) (سال ۳, 200) 
        (سال ۴, 100) (سال ۵, 50) (سال ۶-۷, 25) (سال ۸-۱۰, 0)
    };
    \legend{هزینه‌های عملیاتی, هزینه‌های سرمایه‌ای}
\end{axis}
\end{tikzpicture}
\caption{توزیع زمانی بودجه (سناریوی میانه)}
\label{fig:budget-timeline}
\end{figure}

\sectiondivider

% ═══════════════════════════════════════════════════════════════════════════════
\section{برآورد بودجه به تفکیک بخش}
\label{sec:budget-by-sector}
% ═══════════════════════════════════════════════════════════════════════════════

\begin{landscape}
\begin{table}[htbp]
\centering
\bigtablefontsize
\caption{برآورد بودجه به تفکیک بخش و فاز (میلیون دلار — سناریوی میانه)}
\label{tab:budget-by-sector}
\begin{tabularx}{\linewidth}{>{\raggedleft\arraybackslash}p{3cm}
                             >{\centering\arraybackslash}p{1.5cm}
                             >{\centering\arraybackslash}p{1.5cm}
                             >{\centering\arraybackslash}p{1.5cm}
                             >{\centering\arraybackslash}p{1.5cm}
                             >{\centering\arraybackslash}p{1.5cm}
                             >{\centering\arraybackslash}p{1.5cm}
                             >{\centering\arraybackslash}p{1.2cm}
                             >{\raggedleft\arraybackslash}X}
\toprule
\headerrow بخش & فاز ۰ & فاز ۱ & فاز ۲ & فاز ۳ & فاز ۴ & مجموع & \% & توضیح \\
\midrule
نیروی انسانی (حقوق و مزایا) & ۲۰ & ۲۵۰ & ۵۰۰ & ۳۵۰ & ۱۰۰ & ۱,۲۲۰ & ۳۶٪ & بزرگ‌ترین قلم \\
\altrow نظارت انتخاباتی & ۵ & ۳۰ & ۲۵۰ & ۱۰۰ & ۲۰ & ۴۰۵ & ۱۲٪ & شامل ثبت‌نام و رأی‌گیری \\
حقوق بشر و عدالت انتقالی & ۵ & ۴۰ & ۸۰ & ۸۰ & ۳۰ & ۲۳۵ & ۷٪ & کمیسیون حقیقت + دادگاه \\
\altrow اصلاح بخش امنیتی & ۰ & ۸۰ & ۱۲۰ & ۸۰ & ۲۰ & ۳۰۰ & ۹٪ & \lr{DDR}، آموزش، نظارت \\
لجستیک و زیرساخت & ۱۰ & ۱۰۰ & ۸۰ & ۵۰ & ۲۰ & ۲۶۰ & ۸٪ & ساختمان، حمل‌ونقل، تجهیزات \\
\altrow فناوری اطلاعات & ۱۰ & ۵۰ & ۶۰ & ۳۰ & ۱۰ & ۱۶۰ & ۵٪ & سامانه‌ها، ارتباطات، سایبری \\
آموزش و ظرفیت‌سازی & ۱۰ & ۲۰ & ۵۰ & ۶۰ & ۳۰ & ۱۷۰ & ۵٪ & ایرانی و بین‌المللی \\
\altrow رسانه و ارتباطات & ۵ & ۲۰ & ۴۰ & ۳۰ & ۱۰ & ۱۰۵ & ۳٪ & ارتباطات عمومی، ضد اطلاعات نادرست \\
امنیت کارکنان & ۵ & ۳۰ & ۵۰ & ۳۰ & ۱۰ & ۱۲۵ & ۴٪ & حفاظت، بیمه، تخلیه \\
\altrow هماهنگی و مدیریت & ۵ & ۳۰ & ۴۰ & ۳۰ & ۱۰ & ۱۱۵ & ۳٪ & دفتر \lr{SRSG}، ارزیابی \\
ذخیره احتیاطی & --- & --- & --- & --- & --- & ۶۵۵ & ۸٪ (مبنا) & \\
\midrule
\textbf{مجموع} & \textbf{۷۵} & \textbf{۶۵۰} & \textbf{۱,۲۷۰} & \textbf{۸۴۰} & \textbf{۲۶۰} & \textbf{۳,۷۵۰} & \textbf{۱۰۰٪} & سناریوی میانه \\
\bottomrule
\end{tabularx}
\end{table}
\end{landscape}

\subsection{تحلیل اقلام بودجه‌ای}
\label{subsec:budget-analysis}

\subsubsection{نیروی انسانی (۳۶٪ — بزرگ‌ترین قلم)}

\begin{table}[htbp]
\centering
\caption{تفکیل هزینه‌های نیروی انسانی}
\label{tab:hr-costs}
\begin{tabularx}{\textwidth}{>{\raggedleft\arraybackslash}p{3.5cm}
                             >{\centering\arraybackslash}p{2cm}
                             >{\centering\arraybackslash}p{2cm}
                             >{\raggedleft\arraybackslash}X}
\toprule
\headerrow دسته & تعداد (اوج) & هزینه سالانه (\$K/نفر) & مجموع ۱۰ ساله (\$M) \\
\midrule
کارکنان ارشد بین‌المللی (\lr{D/P5}) & ۲۰۰ & ۲۰۰-۳۰۰ & ۱۵۰ \\
\altrow کارکنان حرفه‌ای بین‌المللی (\lr{P2-P4}) & ۲,۰۰۰ & ۱۲۰-۲۰۰ & ۴۰۰ \\
کارکنان عمومی بین‌المللی (\lr{GS}) & ۱,۵۰۰ & ۶۰-۱۰۰ & ۱۵۰ \\
\altrow مشاوران کوتاه‌مدت & ۱,۰۰۰ & ۱۵۰-۲۵۰ & ۱۲۰ \\
کارکنان ایرانی (دائم) & ۲۰,۰۰۰ & ۱۵-۳۰ & ۳۰۰ \\
\altrow ناظران موقت انتخابات & ۱۰۰,۰۰۰ & ۰.۵-۱ (روزانه) & ۱۰۰ \\
\midrule
\textbf{مجموع} & & & \textbf{۱,۲۲۰} \\
\bottomrule
\end{tabularx}
\end{table}

\begin{warningbox}
تفاوت حقوق کارکنان بین‌المللی و ایرانی (۵-۱۰ برابر) می‌تواند منبع نارضایتی باشد. راهکارها:
\begin{itemize}[nosep]
    \item شفافیت در توضیح ساختار حقوقی \lr{UN}
    \item فوق‌العاده سختی کار و ریسک برای کارکنان ایرانی
    \item مسیر ارتقا برای کارکنان ایرانی به پست‌های بین‌المللی
    \item هزینه زندگی محلی در تعیین حقوق ایرانیان لحاظ شود
\end{itemize}
\end{warningbox}

\subsubsection{نظارت انتخاباتی (۱۲٪)}

\begin{table}[htbp]
\centering
\caption{تفکیک هزینه‌های انتخاباتی}
\label{tab:election-costs}
\begin{tabularx}{\textwidth}{>{\raggedleft\arraybackslash}p{3.5cm}
                             >{\centering\arraybackslash}p{2.5cm}
                             >{\raggedleft\arraybackslash}X}
\toprule
\headerrow قلم & هزینه (\$M) & توضیح \\
\midrule
ثبت‌نام رأی‌دهندگان & ۸۰ & سامانه دیجیتال + محلی، ۶۰M+ نفر \\
\altrow تجهیزات رأی‌گیری & ۶۰ & صندوق، برگه، جوهر، حمل‌ونقل \\
آموزش کارکنان انتخاباتی & ۴۰ & ۱۰۰,۰۰۰+ نفر \\
\altrow اطلاع‌رسانی عمومی & ۳۰ & رسانه، آموزش رأی‌دهندگان \\
نظارت بین‌المللی & ۸۰ & ناظران \lr{OSCE/EU/Carter} \\
\altrow لجستیک انتخاباتی & ۵۰ & حمل‌ونقل مواد به ۶۵,۰۰۰+ شعبه \\
فناوری شمارش و گزارش & ۳۰ & انتقال نتایج، پایگاه داده \\
\altrow رسیدگی به شکایات & ۱۵ & هیئت‌های حل اختلاف \\
ناظران داخلی & ۲۰ & آموزش و تجهیز \\
\midrule
\textbf{مجموع} & \textbf{۴۰۵} & برای ۳-۴ انتخابات \\
\bottomrule
\end{tabularx}
\end{table}

\begin{keypoint}
هزینه سرانه انتخابات در ایران حدود ۵-۸ دلار به ازای هر رأی‌دهنده برآورد می‌شود. مقایسه: عراق ۲۰۰۵ حدود ۱۲ دلار، افغانستان ۲۰۰۴ حدود ۲۰ دلار، تیمور شرقی ۱۹۹۹ حدود ۳۵ دلار. هزینه پایین‌تر ایران به دلیل زیرساخت‌های بهتر و باسوادی بالاتر است.
\end{keypoint}

\sectiondivider

% ═══════════════════════════════════════════════════════════════════════════════
\section{منابع مالی}
\label{sec:funding-sources}
% ═══════════════════════════════════════════════════════════════════════════════

\subsection{ترکیب پیشنهادی منابع}
\label{subsec:funding-mix}

\begin{figure}[htbp]
\centering
\begin{tikzpicture}
\begin{axis}[
    width=10cm,
    height=10cm,
    xbar,
    bar width=0.6cm,
    xlabel={درصد از بودجه کل},
    symbolic y coords={
        بخش خصوصی,
        دارایی‌های ایران,
        کشورهای خلیج فارس,
        ژاپن و کره,
        آمریکا,
        اتحادیه اروپا,
        سازمان ملل
    },
    ytick=data,
    y tick label style={font=\small},
    xmin=0,
    xmax=35,
    nodes near coords,
    every node near coord/.append style={font=\small},
    axis lines=left,
    enlarge y limits=0.15,
]
    \addplot[fill=MainBlue!60] coordinates {
        (30, سازمان ملل)
        (20, اتحادیه اروپا)
        (15, آمریکا)
        (10, ژاپن و کره)
        (10, کشورهای خلیج فارس)
        (10, دارایی‌های ایران)
        (5, بخش خصوصی)
    };
\end{axis}
\end{tikzpicture}
\caption{ترکیب پیشنهادی منابع مالی}
\label{fig:funding-sources}
\end{figure}

\subsection{تحلیل هر منبع}
\label{subsec:source-analysis}

\begin{table}[htbp]
\centering
\caption{تحلیل تفصیلی منابع مالی}
\label{tab:funding-analysis}
\begin{tabularx}{\textwidth}{>{\raggedleft\arraybackslash}p{2.5cm}
                             >{\centering\arraybackslash}p{1.5cm}
                             >{\centering\arraybackslash}p{1.5cm}
                             >{\raggedleft\arraybackslash}X
                             >{\raggedleft\arraybackslash}p{2.5cm}}
\toprule
\headerrow منبع & سهم (\%) & مبلغ (\$M) & توجیه & ریسک اصلی \\
\midrule
بودجه ارزیابی‌شده \lr{UN} & ۳۰ & ۷۵۰-۱,۵۰۰ & تعهد جمعی، مشروعیت & تأخیر تصویب \\
\altrow اتحادیه اروپا & ۲۰ & ۵۰۰-۱,۰۰۰ & بزرگ‌ترین کمک‌دهنده صلح & بروکراسی \\
آمریکا (دوجانبه) & ۱۵ & ۳۷۵-۷۵۰ & منفعت استراتژیک & تغییر دولت \\
\altrow ژاپن + کره جنوبی & ۱۰ & ۲۵۰-۵۰۰ & سابقه مشارکت + منافع انرژی & محدودیت قانون اساسی \\
کشورهای خلیج فارس & ۱۰ & ۲۵۰-۵۰۰ & ثبات منطقه‌ای & شرط‌گذاری سیاسی \\
\altrow دارایی‌های آزادشده ایران & ۱۰ & ۲۵۰-۵۰۰ & دارایی‌های بلوکه‌شده ۱۰۰B+ & اختلاف حقوقی \\
بخش خصوصی + بنیادها & ۵ & ۱۲۵-۲۵۰ & \lr{OSF}, \lr{Gates}, شرکت‌ها & پایداری نامطمئن \\
\bottomrule
\end{tabularx}
\end{table}

\begin{warningbox}
\textbf{اصل تنوع مالی}: هیچ منبع واحدی نباید بیش از ۳۰٪ بودجه را تأمین کند. وابستگی مالی = وابستگی سیاسی. در عراق، تأمین ۹۰٪+ بودجه توسط آمریکا، مأموریت را عملاً آمریکایی کرد و مشروعیت آن را تضعیف نمود.
\end{warningbox}

\subsection{دارایی‌های آزادشده ایران}
\label{subsec:iranian-assets}

\begin{table}[htbp]
\centering
\caption{برآورد دارایی‌های بلوکه‌شده ایران}
\label{tab:frozen-assets}
\begin{tabularx}{\textwidth}{>{\raggedleft\arraybackslash}p{3.5cm}
                             >{\centering\arraybackslash}p{2.5cm}
                             >{\raggedleft\arraybackslash}X}
\toprule
\headerrow نوع دارایی & برآورد (\$B) & توضیح \\
\midrule
ذخایر ارزی بلوکه‌شده & ۵۰-۸۰ & در بانک‌های خارجی \\
\altrow دارایی‌های بانک مرکزی & ۲۰-۴۰ & طلا و ارز خارجی \\
درآمد نفتی معوقه & ۲۰-۳۰ & فروش‌های بلوکه‌شده \\
\altrow دارایی‌های سپاه/بنیادها خارجی & ۵-۱۵ & شرکت‌ها و حساب‌های خارجی \\
\midrule
\textbf{مجموع} & \textbf{۹۵-۱۶۵} & \\
\bottomrule
\end{tabularx}
\end{table}

\begin{recommendation}
اختصاص ۱-۳٪ از دارایی‌های آزادشده (۱-۵ میلیارد دلار) به صندوق امانی گذار:
\begin{itemize}[nosep]
    \item مشروعیت بالا: «پول ایرانی‌ها برای ایرانی‌ها»
    \item کاهش وابستگی به حامیان خارجی
    \item مدیریت مشترک (ایرانی-بین‌المللی) صندوق امانی
    \item شفافیت کامل در مصرف
\end{itemize}
\end{recommendation}

\sectiondivider

% ═══════════════════════════════════════════════════════════════════════════════
\section{مکانیزم‌های مالی}
\label{sec:financial-mechanisms}
% ═══════════════════════════════════════════════════════════════════════════════

\subsection{صندوق امانی چندجانبه}
\label{subsec:multi-donor-trust}

\begin{figure}[htbp]
\centering
\begin{tikzpicture}[
    node distance=1.2cm,
    every node/.style={font=\small, align=center},
    source/.style={rectangle, rounded corners, draw=MainBlue, fill=LightBlue, minimum width=2cm, minimum height=0.7cm},
    fund/.style={rectangle, rounded corners, draw=MainYellow, fill=LightYellow, minimum width=4cm, minimum height=1.2cm, thick},
    dest/.style={rectangle, rounded corners, draw=MainGreen, fill=LightGreen, minimum width=2.5cm, minimum height=0.7cm},
    audit/.style={rectangle, rounded corners, draw=MainRed, fill=LightRed, minimum width=3cm, minimum height=0.7cm},
    arrow/.style={-{Stealth[length=2.5mm]}, thick}
]
    % Sources
    \node[source] (s1) {\lr{UN}};
    \node[source, below=0.4cm of s1] (s2) {\lr{EU}};
    \node[source, below=0.4cm of s2] (s3) {آمریکا};
    \node[source, below=0.4cm of s3] (s4) {ژاپن/کره};
    \node[source, below=0.4cm of s4] (s5) {خلیج فارس};
    \node[source, below=0.4cm of s5] (s6) {دارایی‌های ایران};
    \node[source, below=0.4cm of s6] (s7) {بخش خصوصی};
    
    % Trust Fund
    \node[fund, right=3cm of s4] (fund) {\textbf{صندوق امانی}\\{\footnotesize هیئت امنا: ایرانی + بین‌المللی}};
    
    % Destinations
    \node[dest, right=3cm of fund] (d1) {انتخابات};
    \node[dest, above=0.3cm of d1] (d2) {حقوق بشر};
    \node[dest, above=0.3cm of d2] (d3) {امنیت};
    \node[dest, below=0.3cm of d1] (d4) {قضایی};
    \node[dest, below=0.3cm of d4] (d5) {ظرفیت‌سازی};
    
    % Audit
    \node[audit, below=1.5cm of fund] (audit) {حسابرسی مستقل\\{\footnotesize + گزارش عمومی}};
    
    % Arrows - sources to fund
    \foreach \s in {s1,s2,s3,s4,s5,s6,s7} {
        \draw[arrow, MainBlue] (\s.east) -- ++(0.5,0) |- (fund.west);
    }
    
    % Arrows - fund to destinations
    \foreach \d in {d1,d2,d3,d4,d5} {
        \draw[arrow, MainGreen] (fund.east) -- ++(0.5,0) |- (\d.west);
    }
    
    % Audit arrow
    \draw[arrow, MainRed, dashed] (fund) -- (audit);
    \draw[arrow, MainRed, dashed] (audit.east) -- ++(2,0) node[right, font=\footnotesize] {گزارش عمومی};
    
\end{tikzpicture}
\caption{نمودار جریان مالی صندوق امانی}
\label{fig:trust-fund-flow}
\end{figure}

\subsection{مکانیزم‌های تخصیص بودجه}
\label{subsec:allocation-mechanisms}

\begin{table}[htbp]
\centering
\caption{سه مکانیزم تخصیص بودجه}
\label{tab:allocation-mechanisms}
\begin{tabularx}{\textwidth}{>{\raggedleft\arraybackslash}p{3cm}
                             >{\raggedleft\arraybackslash}X
                             >{\raggedleft\arraybackslash}X
                             >{\centering\arraybackslash}p{2cm}}
\toprule
\headerrow مکانیزم & مزایا & معایب & سهم پیشنهادی \\
\midrule
بودجه ارزیابی‌شده \lr{UN} & پایدار، الزام‌آور & کُند، بوروکراتیک & ۳۰-۴۰٪ \\
\altrow صندوق امانی چندجانبه & انعطاف‌پذیر، هماهنگ & وابسته به حسن‌نیت حامیان & ۴۰-۵۰٪ \\
کمک‌های دوجانبه & سریع، انعطاف‌پذیر & خطر سوگیری، ناهماهنگی & ۱۰-۲۰٪ \\
\bottomrule
\end{tabularx}
\end{table}

\begin{lessonlearned}{تیمور شرقی: موفقیت صندوق امانی}
در تیمور شرقی، صندوق امانی مدیریت‌شده توسط بانک جهانی (\lr{TFET}) یکی از مؤثرترین مکانیزم‌ها بود: ۱۸۰ میلیون دلار از ۱۱ کشور جمع‌آوری شد، با حسابرسی مستقل و گزارش‌دهی شفاف. نکته کلیدی: مالکیت تیموری‌ها بر اولویت‌گذاری هزینه‌ها.
\end{lessonlearned}

\sectiondivider

% ═══════════════════════════════════════════════════════════════════════════════
\section{شفافیت مالی و ضدفساد}
\label{sec:financial-transparency}
% ═══════════════════════════════════════════════════════════════════════════════

\subsection{اصول شفافیت مالی}
\label{subsec:transparency-principles}

\begin{keypoint}
\textbf{شش اصل شفافیت مالی:}
\begin{enumerate}[nosep]
    \item \textbf{انتشار عمومی}: بودجه، هزینه‌ها، و قراردادها به‌صورت آنلاین منتشر شوند
    \item \textbf{حسابرسی مستقل}: حداقل سالانه توسط شرکت بین‌المللی معتبر
    \item \textbf{دسترسی رسانه}: روزنامه‌نگاران حق پرسش و دسترسی به اسناد مالی را دارند
    \item \textbf{مکانیزم شکایت}: کانال امن برای گزارش فساد توسط کارکنان و شهروندان
    \item \textbf{استاندارد \lr{IATI}}: گزارش‌دهی مطابق ابتکار شفافیت کمک بین‌المللی
    \item \textbf{پاسخگویی دوسویه}: هم به حامیان مالی و هم به مردم ایران
\end{enumerate}
\end{keypoint}

\subsection{ساختار نظارت مالی}
\label{subsec:financial-oversight}

\begin{table}[htbp]
\centering
\caption{لایه‌های نظارت مالی}
\label{tab:financial-oversight}
\begin{tabularx}{\textwidth}{>{\raggedleft\arraybackslash}p{0.5cm}
                             >{\raggedleft\arraybackslash}p{3cm}
                             >{\raggedleft\arraybackslash}X
                             >{\raggedleft\arraybackslash}p{2.5cm}}
\toprule
\headerrow \# & لایه & وظیفه & فرکانس \\
\midrule
۱ & حسابرسی داخلی \lr{UNMOIT} & بررسی روزانه تراکنش‌ها & مستمر \\
\altrow ۲ & بازرسی داخلی \lr{UN} (\lr{OIOS}) & حسابرسی مستقل از مدیریت مأموریت & شش‌ماهه \\
۳ & حسابرسی خارجی & شرکت بین‌المللی (مثل \lr{Deloitte/PwC}) & سالانه \\
\altrow ۴ & هیئت امنای صندوق & نظارت بر تخصیص و مصرف & ماهانه \\
۵ & کمیته بودجه شورای امنیت & بررسی بودجه ارزیابی‌شده & سالانه \\
\altrow ۶ & نظارت عمومی (رسانه/مدنی) & افشاگری، پرسش، پیگیری & مستمر \\
\bottomrule
\end{tabularx}
\end{table}

\begin{warningbox}
\textbf{فساد در مأموریت‌های \lr{UN} واقعی است}:
\begin{itemize}[nosep]
    \item رسوایی «نفت در برابر غذا» عراق: ۱.۸ میلیارد دلار سوءاستفاده
    \item گزارش‌های فساد در \lr{UNMIK} کوزوو و \lr{MONUC} کنگو
    \item حیف‌ومیل در پروژه‌های بازسازی افغانستان
\end{itemize}
اعتماد مردم ایران — که سال‌ها شاهد فساد نهادهای حکومتی بوده‌اند — بسیار شکننده است. هر مورد فساد، حتی کوچک، می‌تواند کل مأموریت را بی‌اعتبار کند.
\end{warningbox}

\begin{recommendation}
\textbf{سیاست تحمل‌صفر با فساد:}
\begin{enumerate}[nosep]
    \item اخراج فوری کارکنان متخلف
    \item ارجاع پرونده‌ها به مراجع قضایی (بین‌المللی و ملی)
    \item جبران کامل خسارت از محل حقوق/ضمانت متخلف
    \item انتشار عمومی نتایج تحقیقات (بدون سانسور)
    \item مکانیزم حفاظت از افشاگران (\lr{Whistleblower Protection})
\end{enumerate}
\end{recommendation}

\sectiondivider

% ═══════════════════════════════════════════════════════════════════════════════
\section{تحلیل هزینه-فایده}
\label{sec:cost-benefit}
% ═══════════════════════════════════════════════════════════════════════════════

\subsection{هزینه عدم اقدام}
\label{subsec:cost-of-inaction}

\begin{table}[htbp]
\centering
\caption{برآورد هزینه عدم اقدام (سناریوی فروپاشی بدون نظارت)}
\label{tab:cost-of-inaction}
\begin{tabularx}{\textwidth}{>{\raggedleft\arraybackslash}p{3.5cm}
                             >{\centering\arraybackslash}p{2.5cm}
                             >{\raggedleft\arraybackslash}X}
\toprule
\headerrow حوزه & هزینه برآوردی (\$B) & توضیح \\
\midrule
بحران پناهندگی & ۲۰-۵۰ & ۵-۱۰M آواره، هزینه برای کشورهای همسایه و اروپا \\
\altrow بی‌ثباتی نفتی & ۵۰-۲۰۰ & افزایش ۳۰-۵۰٪ قیمت نفت جهانی \\
تروریسم & ۵-۲۰ & گسترش گروه‌های افراطی، عملیات ضدتروریسم \\
\altrow بحران هسته‌ای & غیرقابل‌محاسبه & ریسک اشاعه یا استفاده \\
از دست رفتن بازار & ۱۰-۳۰ & بازار ۸۵M نفری غیرقابل‌دسترس \\
\altrow هزینه بشردوستانه & ۵-۱۵ & کمک غذایی، پزشکی، سرپناه \\
\midrule
\textbf{مجموع} & \textbf{۹۰-۳۱۵} & بسیار بیشتر از بودجه نظارت \\
\bottomrule
\end{tabularx}
\end{table}

\begin{keypoint}
\textbf{نسبت هزینه-فایده}: بودجه ۲.۵-۵ میلیارد دلاری نظارت، در مقابل هزینه ۹۰-۳۱۵ میلیارد دلاری عدم اقدام. حتی اگر نظارت فقط ۵٪ احتمال فاجعه را کاهش دهد، بازگشت سرمایه ۱:۱-۱:۳ خواهد بود. در واقعیت، نظارت مؤثر می‌تواند احتمال موفقیت را ۳۰-۵۰٪ افزایش دهد.
\end{keypoint}

\subsection{فواید اقتصادی گذار موفق}
\label{subsec:benefits-of-success}

\begin{table}[htbp]
\centering
\caption{فواید اقتصادی گذار موفق ایران}
\label{tab:benefits}
\begin{tabularx}{\textwidth}{>{\raggedleft\arraybackslash}p{3.5cm}
                             >{\centering\arraybackslash}p{2.5cm}
                             >{\raggedleft\arraybackslash}X}
\toprule
\headerrow حوزه & ارزش برآوردی (\$B/سال) & توضیح \\
\midrule
رفع تحریم‌ها و صادرات نفت & ۶۰-۱۰۰ & بازگشت به ظرفیت ۴-۵ M بشکه/روز \\
\altrow جذب سرمایه‌گذاری خارجی & ۱۰-۳۰ & بازار بکر ۸۵M نفری \\
گردشگری & ۵-۱۵ & میراث فرهنگی عظیم \\
\altrow بازگشت سرمایه فراری & ۲۰-۵۰ & سرمایه ایرانیان خارج \\
رشد اقتصادی & ۵-۱۰٪/سال & اثر آزادسازی اقتصادی \\
\bottomrule
\end{tabularx}
\end{table}

\sectiondivider

% ═══════════════════════════════════════════════════════════════════════════════
\section{کنفرانس کمک‌دهندگان}
\label{sec:donors-conference}
% ═══════════════════════════════════════════════════════════════════════════════

\begin{casestudy}{کنفرانس‌های موفق: بن (افغانستان) و پاریس (لبنان)}
کنفرانس بن ۲۰۰۱ (افغانستان) و کنفرانس‌های پاریس (لبنان، \lr{CEDRE 2018}) نشان دادند که کنفرانس کمک‌دهندگان می‌تواند میلیاردها دلار تعهد جذب کند — \emph{به شرط آنکه}:
\begin{itemize}[nosep]
    \item طرح روشن و قابل‌اعتمادی وجود داشته باشد
    \item نمایندگان قانونی کشور میزبان حضور داشته باشند
    \item مکانیزم شفافیت و پاسخگویی تعریف شده باشد
    \item تعهدات مالی به برنامه‌های مشخص متصل باشند
\end{itemize}
\end{casestudy}

\begin{table}[htbp]
\centering
\caption{طرح پیشنهادی کنفرانس کمک‌دهندگان ایران}
\label{tab:donors-conference}
\begin{tabularx}{\textwidth}{>{\raggedleft\arraybackslash}p{3cm}
                             >{\raggedleft\arraybackslash}X}
\toprule
\headerrow عنصر & شرح \\
\midrule
زمان & ماه ۲-۳ پس از گذار \\
\altrow مکان & ژنو یا وین (بی‌طرف) \\
میزبان & دبیرکل \lr{UN} + دولت انتقالی ایران \\
\altrow مدعوین & ۵۰+ کشور، نهادهای مالی بین‌المللی، بخش خصوصی \\
هدف مالی & ۳-۵ میلیارد دلار تعهد اولیه \\
\altrow ساختار & ۱) جلسه عمومی + ۲) میزهای تخصصی + ۳) تعهدات \\
پیگیری & کنفرانس سالانه بازنگری \\
\bottomrule
\end{tabularx}
\end{table}

\begin{recommendation}
برای موفقیت کنفرانس:
\begin{enumerate}[nosep]
    \item ارائه طرح جامع با ارقام مشخص (همین سند می‌تواند مبنا باشد)
    \item حضور دولت انتقالی ایران با مشروعیت
    \item تعهد شفاف به حسابرسی و گزارش‌دهی
    \item پیوند تعهدات مالی به شاخص‌های پیشرفت
    \item مکانیزم پیگیری تعهدات (بسیاری از تعهدات کنفرانس‌ها محقق نمی‌شوند)
\end{enumerate}
\end{recommendation}

\sectiondivider

% ═══════════════════════════════════════════════════════════════════════════════
% جمع‌بندی فصل
% ═══════════════════════════════════════════════════════════════════════════════

\begin{chaptersummary}
یافته‌های کلیدی این فصل:

\begin{enumerate}
    \item \textbf{بودجه ۲.۵-۵ میلیارد دلار معقول و توجیه‌پذیر است}: در مقایسه با عراق (۶۰B+)، افغانستان (۱۴۵B+)، و هزینه عدم اقدام (۹۰-۳۱۵B)، این رقم سرمایه‌گذاری عاقلانه‌ای است.

    \item \textbf{هزینه سرانه پایین}: ۳۰-۶۰ دلار به ازای هر ایرانی در ۱۰ سال — کمتر از قیمت یک وعده غذا در رستوران نیویورک.

    \item \textbf{نیروی انسانی بزرگ‌ترین قلم بودجه (۳۶٪)}: کنترل هزینه‌های حقوقی و تعادل بین کارکنان بین‌المللی و ایرانی حیاتی است.

    \item \textbf{تنوع منابع مالی ضروری}: حداکثر ۳۰٪ از هر منبع واحد. ترکیب \lr{UN} + \lr{EU} + دوجانبه + دارایی‌های ایران + خصوصی.

    \item \textbf{صندوق امانی چندجانبه مکانیزم مرکزی}: با مدیریت مشترک ایرانی-بین‌المللی و حسابرسی مستقل.

    \item \textbf{شفافیت مالی شرط اعتماد}: شش اصل شفافیت، شش لایه نظارت، و سیاست تحمل‌صفر با فساد.

    \item \textbf{هزینه عدم اقدام بسیار بیشتر است}: بحران پناهندگی، بی‌ثباتی نفتی، تروریسم — مجموعاً ۹۰-۳۱۵ میلیارد دلار.

    \item \textbf{کنفرانس کمک‌دهندگان در ماه ۲-۳}: با طرح روشن، حضور دولت انتقالی، و مکانیزم پیگیری تعهدات.
\end{enumerate}

\vspace{0.5cm}
\textit{در فصل بعد (\ref{ch:roadmap})، نقشه راه اجرایی و توصیه‌های نهایی برای همه ذی‌نفعان ارائه خواهد شد.}
\end{chaptersummary}

\chapterend