% ═══════════════════════════════════════════════════════════════════════════════
%                    تاریخ تحولات فرانسه - فصل [X]
% ═══════════════════════════════════════════════════════════════════════════════

\documentclass[12pt,a4paper]{book}

% ─────────────────────────── پکیج‌ها ───────────────────────────
\usepackage{amsmath,amssymb}
\usepackage{geometry}
\geometry{top=2.5cm, bottom=2.5cm, left=2cm, right=2.5cm, headheight=15pt}
\usepackage{graphicx}
\usepackage{array,booktabs,longtable,multirow,colortbl}
\usepackage{xcolor}
\usepackage{tikz}
\usetikzlibrary{shapes.geometric, arrows.meta, positioning, calc, backgrounds, 
	fit, decorations.pathmorphing, shadows, patterns}
\usepackage{pgfplots}
\pgfplotsset{compat=1.18}
\usepackage{tcolorbox}
\tcbuselibrary{skins,breakable}
\usepackage{enumitem}
\usepackage{fancyhdr}
\usepackage{pdflscape}
\usepackage{setspace}
\usepackage{titlesec}
\usepackage{float}
\usepackage{pdfpages}
\usepackage{pdflscape}  % برای صفحات landscape
\usepackage{hyperref}

% ─────────────────────────── رنگ‌ها ───────────────────────────
\definecolor{bleurepublique}{RGB}{0, 35, 149}
\definecolor{rougerevolution}{RGB}{237, 41, 57}
\definecolor{orroyal}{RGB}{255, 215, 0}
\definecolor{vertnapoleon}{RGB}{0, 100, 0}
\definecolor{violetempire}{RGB}{128, 0, 128}
\definecolor{fondclair}{RGB}{255, 253, 240}
\definecolor{gris}{RGB}{128, 128, 128}
\definecolor{grisclair}{RGB}{245, 245, 245}
\definecolor{noirsombre}{RGB}{30, 30, 30}

% رنگ‌های کمکی
\definecolor{bleulight}{RGB}{230, 235, 250}
\definecolor{rougelight}{RGB}{253, 235, 237}
\definecolor{vertlight}{RGB}{235, 250, 235}
\definecolor{violetlight}{RGB}{245, 235, 250}
\definecolor{orroyallight}{RGB}{255, 250, 230}
\definecolor{grislight}{RGB}{248, 248, 248}
\definecolor{bleumid}{RGB}{180, 195, 230}
\definecolor{rougemid}{RGB}{245, 180, 185}
\definecolor{vertmid}{RGB}{180, 220, 180}
\definecolor{violetmid}{RGB}{210, 180, 220}
\definecolor{orroyalmid}{RGB}{255, 240, 180}
\definecolor{orroyaldark}{RGB}{200, 170, 0}

% ─────────────────────────── فونت فارسی ───────────────────────────
\usepackage{fontspec}
\setmainfont{Vazirmatn}
\usepackage{xepersian}
\settextfont{Vazirmatn}
\setdigitfont{Vazirmatn}

% ─────────────────────────── هایپرلینک ───────────────────────────
\hypersetup{
	colorlinks=true,
	linkcolor=bleurepublique,
	urlcolor=bleurepublique,
	citecolor=vertnapoleon
}

% ─────────────────────────── کادرها ───────────────────────────
\newtcolorbox{kholasebox}[1][]{enhanced,breakable,colback=bleulight,
	colframe=bleurepublique,coltitle=white,fonttitle=\bfseries\large,
	title={#1},boxrule=2pt,arc=4pt,left=10pt,right=10pt,top=10pt,bottom=10pt,
	drop shadow={opacity=0.3}}

\newtcolorbox{naghlbox}[1][]{enhanced,breakable,colback=orroyallight,
	colframe=orroyaldark,coltitle=black,fonttitle=\bfseries,title={#1},
	boxrule=1.5pt,arc=3pt,borderline west={4pt}{0pt}{orroyal},
	left=15pt,right=10pt,top=8pt,bottom=8pt}

\newtcolorbox{olgoobox}[1][]{enhanced,breakable,colback=vertlight,
	colframe=vertnapoleon,coltitle=white,fonttitle=\bfseries,title={#1},
	boxrule=1.5pt,arc=4pt,left=10pt,right=10pt,top=8pt,bottom=8pt,
	before upper={\parindent15pt}}

\newtcolorbox{enghelabbox}[1][]{enhanced,breakable,colback=rougelight,
	colframe=rougerevolution,coltitle=white,fonttitle=\bfseries,title={#1},
	boxrule=2pt,arc=4pt,left=10pt,right=10pt,top=8pt,bottom=8pt}

\newtcolorbox{empirebox}[1][]{enhanced,breakable,colback=violetlight,
	colframe=violetempire,coltitle=white,fonttitle=\bfseries,title={#1},
	boxrule=1.5pt,arc=4pt,left=10pt,right=10pt,top=8pt,bottom=8pt}

\newtcolorbox{noktebox}[1][]{enhanced,colback=grisclair,colframe=gris,
	fonttitle=\bfseries,title={#1},boxrule=1pt,arc=3pt,left=8pt,right=8pt}

% ─────────────────────────── صفحه‌آرایی ───────────────────────────
\pagestyle{fancy}
\fancyhf{}
\fancyhead[RO]{\leftmark}
\fancyhead[LE]{\rightmark}
\fancyfoot[C]{\thepage}
\renewcommand{\headrulewidth}{1pt}
\renewcommand{\footrulewidth}{0.5pt}
\setstretch{1.5}

\titleformat{\chapter}[display]
{\normalfont\huge\bfseries\color{bleurepublique}}
{\chaptertitlename\ \thechapter}{20pt}{\Huge}
\titleformat{\section}
{\normalfont\Large\bfseries\color{bleurepublique}}{\thesection}{1em}{}
\titleformat{\subsection}
{\normalfont\large\bfseries\color{bleurepublique}}{\thesubsection}{1em}{}

% ═══════════════════════════════════════════════════════════════════════════════
\begin{document}



%══════════════════════════════════════════════════════════════════════════════
% ادامه فصل ۷ — بخش ۲
%══════════════════════════════════════════════════════════════════════════════
%══════════════════════════════════════════════════════════════════════════════
% فصل ۷: ویشی و جمهوری چهارم (۱۹۴۰-۱۹۵۸)
%══════════════════════════════════════════════════════════════════════════════

\chapter{ویشی و جمهوری چهارم (۱۹۴۰-۱۹۵۸)}
\label{chap:vichy-fourth}

\begin{kholasebox}
	\textbf{خلاصه فصل:}
	
	دوره ۱۹۴۰-۱۹۵۸ یکی از دردناک‌ترین و پرتنش‌ترین دوره‌های تاریخ فرانسه است. این کشور که در ۱۹۴۰ ظرف شش هفته شکست خورد، چهار سال تحت اشغال و حکومت همکار ویشی زیست، سپس آزاد شد و کوشید با جمهوری چهارم به ثبات دست یابد—اما بار دیگر در بحران استعماری الجزایر فروپاشید.
	
	\textbf{دوره‌بندی:}
	\begin{itemize}[nosep,rightmargin=0pt]
		\item \textbf{فرانسه ویشی (۱۹۴۰-۱۹۴۴):} «دولت فرانسوی»، همکاری، انقلاب ملی
		\item \textbf{مقاومت (۱۹۴۰-۱۹۴۴):} فرانسه آزاد، مقاومت داخلی
		\item \textbf{آزادی و دولت موقت (۱۹۴۴-۱۹۴۶):} تصفیه، بازسازی
		\item \textbf{جمهوری چهارم (۱۹۴۶-۱۹۵۸):} بی‌ثباتی، جنگ‌های استعماری، سقوط
	\end{itemize}
	
	\textbf{مفاهیم کلیدی:} همکاری (\lr{collaboration})، مقاومت (\lr{Résistance})، تصفیه (\lr{épuration})، سه‌جانبه‌گرایی، استعمارزدایی، بحران الجزایر.
\end{kholasebox}

%──────────────────────────────────────────────────────────────────────────────
\section{فرانسه ویشی (۱۹۴۰-۱۹۴۴)}
%──────────────────────────────────────────────────────────────────────────────

\subsection{تولد رژیم}

در ۱۰ ژوئیه ۱۹۴۰، در کازینوی شهر آبگرم ویشی، ۵۶۹ نماینده و سناتور (از ۶۴۹ حاضر) به لایحه‌ای رأی دادند که «تمام اختیارات را به دولت جمهوری، تحت امضا و اقتدار مارشال پتن» واگذار می‌کرد. فقط ۸۰ نفر مخالفت کردند—آنها که بعداً «هشتاد تن» نامیده شدند.

\begin{table}[htbp]
	\centering
	\caption{رأی‌گیری ۱۰ ژوئیه ۱۹۴۰}
	\label{tab:vichy-vote}
	\begin{tabular}{|r|c|c|}
		\hline
		\rowcolor{gris}
		\textcolor{white}{\textbf{گزینه}} & \textcolor{white}{\textbf{تعداد}} & \textcolor{white}{\textbf{درصد حاضران}} \\
		\hline
		موافق & ۵۶۹ & ۸۸٪ \\
		\hline
		\rowcolor{grisclair}
		مخالف & ۸۰ & ۱۲٪ \\
		\hline
		ممتنع & ۱۷ & — \\
		\hline
		\rowcolor{grisclair}
		غایب & ۱۷۶ & — \\
		\hline
	\end{tabular}
\end{table}

\begin{noktebox}
	\textbf{چرا چنین رأی یک‌طرفه‌ای؟}
	
	\begin{itemize}[nosep]
		\item \textbf{شوک شکست:} باور به پایان فرانسه قدیم
		\item \textbf{اعتبار پتن:} «قهرمان وردن»، پدر ملت
		\item \textbf{فشار:} حضور آلمان‌ها، فضای رعب
		\item \textbf{فرصت‌طلبی:} امید به جایگاه در نظم جدید
		\item \textbf{غیبت:} ۲۷ نماینده در کشتی «ماسیلیا» به مراکش رفته بودند
	\end{itemize}
\end{noktebox}

\subsection{ساختار رژیم}

\begin{tikzpicture}[
	every node/.style={font=\small},
	box/.style={rectangle, rounded corners, draw=gris, fill=grisclair,
		minimum width=3.5cm, minimum height=1.5cm, align=center}
	]
	% Title
	\node[font=\bfseries\large] at (7,8) {ساختار «دولت فرانسوی» (ویشی)};
	
	% Pétain at top
	\node[box, fill=orroyalmid, draw=orroyaldark, text=white, minimum width=5cm] (petain) at (7,6) 
	{\textbf{مارشال پتن}\\«رئیس دولت فرانسوی»\\قدرت مطلقه};
	
	% Key positions
	\node[box] (laval) at (2.5,3.5) {\textbf{لاوال}\\معاون (۱۹۴۰)\\نخست‌وزیر (۱۹۴۲-۴۴)};
	
	\node[box] (darlan) at (7,3.5) {\textbf{دارلان}\\جانشین (۱۹۴۱-۴۲)\\دریاسالار};
	
	\node[box] (ministers) at (11.5,3.5) {\textbf{وزرا}\\منصوب توسط پتن\\مسئول در برابر او};
	
	% Characteristics
	\node[rectangle, draw=rougerevolution, fill=rougelight, rounded corners, align=center] at (3,1) 
	{بدون مجلس\\بدون احزاب\\بدون انتخابات};
	
	\node[rectangle, draw=orroyaldark, fill=orroyallight, rounded corners, align=center] at (7,1) 
	{شعار: «کار، خانواده، میهن»\\به جای «آزادی، برابری، برادری»};
	
	\node[rectangle, draw=violetempire, fill=violetlight, rounded corners, align=center] at (11,1) 
	{پایتخت: ویشی\\منطقه «آزاد» (تا ۱۹۴۲)};
	
	% Arrows
	\draw[->, thick, orroyaldark] (petain) -- (laval);
	\draw[->, thick, orroyaldark] (petain) -- (darlan);
	\draw[->, thick, orroyaldark] (petain) -- (ministers);
	
\end{tikzpicture}

\subsection{تقسیم فرانسه}

آتش‌بس ۲۲ ژوئن ۱۹۴۰ فرانسه را به چند منطقه تقسیم کرد:

\begin{table}[htbp]
	\centering
	\caption{تقسیمات فرانسه تحت اشغال}
	\label{tab:france-zones}
	\begin{tabular}{|r|p{7cm}|c|}
		\hline
		\rowcolor{gris}
		\textcolor{white}{\textbf{منطقه}} & \textcolor{white}{\textbf{توضیح}} & \textcolor{white}{\textbf{مساحت}} \\
		\hline
		منطقه اشغالی شمال & تحت کنترل مستقیم آلمان، پاریس & ۶۰٪ \\
		\hline
		\rowcolor{grisclair}
		منطقه «آزاد» جنوب & تحت حاکمیت ویشی (تا نوامبر ۱۹۴۲) & ۴۰٪ \\
		\hline
		آلزاس-لورن & الحاق عملی به آلمان & — \\
		\hline
		\rowcolor{grisclair}
		منطقه ممنوعه شمال‌شرق & برای مهاجرت آلمانی آینده & — \\
		\hline
		منطقه ایتالیایی & جنوب‌شرق (تا ۱۹۴۳) & کوچک \\
		\hline
	\end{tabular}
\end{table}

\begin{tikzpicture}[
	every node/.style={font=\small}
	]
	% Title
	\node[font=\bfseries\large] at (6,8) {نقشه فرانسه تقسیم‌شده (۱۹۴۰-۱۹۴۲)};
	
	% Simplified map outline
	\draw[very thick] (0,0) -- (4,0) -- (5,2) -- (6,1) -- (8,2) -- (9,4) -- 
	(10,5) -- (9,7) -- (6,7.5) -- (3,7) -- (1,5) -- (0,3) -- cycle;
	
	% Demarcation line
	\draw[ultra thick, rougerevolution, dashed] (0,3.5) -- (3,3) -- (6,4) -- (9,3.5);
	
	% Zones
	\fill[bleulight, opacity=0.5] (0,3.5) -- (3,3) -- (6,4) -- (9,3.5) -- 
	(9,7) -- (6,7.5) -- (3,7) -- (1,5) -- (0,3) -- cycle;
	\fill[orroyallight, opacity=0.5] (0,0) -- (4,0) -- (5,2) -- (6,1) -- (8,2) -- 
	(9,3.5) -- (6,4) -- (3,3) -- (0,3.5) -- cycle;
	
	% Labels
	\node at (5,5.5) {\textbf{منطقه اشغالی}};
	\node[font=\footnotesize] at (5,5) {(تحت کنترل آلمان)};
	
	\node at (5,1.5) {\textbf{منطقه «آزاد»}};
	\node[font=\footnotesize] at (5,1) {(ویشی)};
	
	% Cities
	\fill (6,6) circle (0.1);
	\node[font=\footnotesize, right] at (6.1,6) {پاریس};
	
	\fill (4,1.8) circle (0.1);
	\node[font=\footnotesize, right] at (4.1,1.8) {ویشی};
	
	% Line label
	\node[font=\footnotesize, rougerevolution] at (1,2.5) {خط مرزی};
	
	% Note
	\node[font=\footnotesize, align=center] at (6,-1) 
	{نوامبر ۱۹۴۲: آلمان منطقه «آزاد» را نیز اشغال کرد};
	
\end{tikzpicture}

\subsection{«انقلاب ملی»}

پتن فقط نمی‌خواست فرانسه را اداره کند؛ می‌خواست آن را دگرگون سازد. «انقلاب ملی» (\lr{Révolution nationale}) پروژه‌ای بود برای بازسازی فرانسه بر پایه‌های ضد جمهوری:

\begin{tikzpicture}[
	every node/.style={font=\small},
	value/.style={rectangle, rounded corners, draw=orroyaldark, fill=orroyallight,
		minimum width=3cm, minimum height=1.5cm, align=center}
	]
	% Title
	\node[font=\bfseries\large] at (7,7.5) {«انقلاب ملی» ویشی};
	
	% Old values (crossed out)
	\node[rectangle, draw=rougerevolution, fill=rougelight, rounded corners,
	minimum width=8cm, minimum height=1.2cm, align=center] at (7,6) 
	{{آزادی، برابری، برادری} — «شعارهای دروغین»};
	
	% New values
	\node[value] (travail) at (2,3.5) {\textbf{کار}\\(\lr{Travail})\\ضد بیکاری، ضد اعتصاب};
	
	\node[value] (famille) at (7,3.5) {\textbf{خانواده}\\(\lr{Famille})\\ضد فردگرایی، نقش زن};
	
	\node[value] (patrie) at (12,3.5) {\textbf{میهن}\\(\lr{Patrie})\\ضد بین‌المللی‌گرایی};
	
	% Enemies
	\node[rectangle, draw=gris, fill=grisclair, rounded corners, align=center] at (7,1) {
		\textbf{«دشمنان فرانسه»:} فراماسون‌ها، کمونیست‌ها، یهودیان،\\
		معلمان لائیک، «ضد فرانسه» (\lr{Anti-France})
	};
	
\end{tikzpicture}

\subsubsection{سیاست‌های انقلاب ملی}

\begin{table}[htbp]
	\centering
	\caption{اقدامات انقلاب ملی}
	\label{tab:national-revolution}
	\begin{tabular}{|r|p{9cm}|}
		\hline
		\rowcolor{orroyalmid}
		\textcolor{white}{\textbf{حوزه}} & \textcolor{white}{\textbf{اقدامات}} \\
		\hline
		\textbf{ضد جمهوری} & انحلال احزاب، ممنوعیت فراماسونری، تصفیه معلمان \\
		\hline
		\rowcolor{orroyallight}
		\textbf{ضد یهود} & قوانین یهودستیزانه (اکتبر ۱۹۴۰، ژوئن ۱۹۴۱) \\
		\hline
		\textbf{کار} & منشور کار (۱۹۴۱)، ممنوعیت اتحادیه‌های آزاد \\
		\hline
		\rowcolor{orroyallight}
		\textbf{خانواده} & محدودیت طلاق، تشویق زایمان، جرم‌انگاری سقط جنین \\
		\hline
		\textbf{جوانان} & «شانتیه دو ژونس» (اردوگاه‌های کار جوانان) \\
		\hline
		\rowcolor{orroyallight}
		\textbf{روستا} & ایدئولوژی «بازگشت به زمین» (\lr{retour à la terre}) \\
		\hline
	\end{tabular}
\end{table}

\subsection{همکاری با آلمان}

همکاری (\lr{collaboration}) در چند سطح صورت گرفت:

\begin{tikzpicture}[
	every node/.style={font=\small},
	level/.style={rectangle, rounded corners, minimum width=4.5cm, minimum height=1.8cm, align=center}
	]
	% Title
	\node[font=\bfseries\large] at (7,8) {سطوح همکاری با آلمان نازی};
	
	% Levels
	\node[level, draw=gris, fill=grisclair] (state) at (3,5.5) 
	{\textbf{همکاری دولتی}\\مذاکره با آلمان\\«محافظت» از فرانسه};
	
	\node[level, draw=orroyaldark, fill=orroyallight] (admin) at (8,5.5) 
	{\textbf{همکاری اداری}\\اجرای دستورات آلمان\\پلیس، بوروکراسی};
	
	\node[level, draw=violetempire, fill=violetlight] (econ) at (13,5.5) 
	{\textbf{همکاری اقتصادی}\\کار اجباری، صنایع\\۴۰٪ تولید برای آلمان};
	
	\node[level, draw=rougerevolution, fill=rougemid, text=white] (ideol) at (5.5,2.5) 
	{\textbf{همکاری ایدئولوژیک}\\فاشیست‌های فرانسوی\\دِآ، دوریو، براسیاک};
	
	\node[level, draw=rougerevolution, fill=rougemid, text=white] (milit) at (10.5,2.5) 
	{\textbf{همکاری نظامی}\\لژیون ضدبلشویکی\\واحد اس‌اس شارلمانی};
	
	% Spectrum arrow
	\draw[->, ultra thick] (0,0) -- (14,0);
	\node at (0,0.4) {کمتر داوطلبانه};
	\node at (14,0.4) {کاملاً داوطلبانه};
	
\end{tikzpicture}

\subsubsection{دیدار مونتوار (۲۴ اکتبر ۱۹۴۰)}

\begin{naghlbox}
	«من امروز راه همکاری را در پیش می‌گیرم... این همکاری باید صادقانه باشد... من این همکاری را در افتخار می‌پذیرم.»
	
	\hfill --- \textit{مارشال پتن، ۳۰ اکتبر ۱۹۴۰}
\end{naghlbox}

دست‌دادن پتن و هیتلر در مونتوار، نماد «همکاری» شد—تصویری که تا ابد با ویشی گره خورد.

\subsection{یهودستیزی و هولوکاست}

ویشی منتظر دستور آلمان نماند؛ خودش ابتکار عمل را در دست گرفت:

\begin{enghelabbox}
	\textbf{قوانین یهودستیزانه ویشی}
	
	\textbf{قانون ۳ اکتبر ۱۹۴۰ (بدون درخواست آلمان):}
	\begin{itemize}[nosep]
		\item تعریف «یهودی» (نژادی)
		\item ممنوعیت مشاغل دولتی، ارتش، آموزش، رسانه
		\item محدودیت در مشاغل آزاد (پزشکی، حقوق)
	\end{itemize}
	
	\textbf{قانون ۲ ژوئن ۱۹۴۱:}
	\begin{itemize}[nosep]
		\item سرشماری یهودیان
		\item «آریایی‌سازی» اموال یهودی
		\item محدودیت‌های بیشتر
	\end{itemize}
	
	\textbf{رویداد وِل دیو (۱۶-۱۷ ژوئیه ۱۹۴۲):}
	\begin{itemize}[nosep]
		\item دستگیری ۱۳,۱۵۲ یهودی در پاریس توسط پلیس فرانسه
		\item از جمله ۴,۰۰۰ کودک
		\item زندانی در ولودروم زمستانی (ورزشگاه)
		\item تحویل به آلمان‌ها، ارسال به آشویتس
	\end{itemize}
	
	\textbf{آمار کلی:}
	\begin{itemize}[nosep]
		\item ۷۶,۰۰۰ یهودی از فرانسه به اردوگاه‌ها فرستاده شدند
		\item فقط ۲,۵۰۰ نفر بازگشتند
		\item حدود ۲۵٪ یهودیان فرانسه
	\end{itemize}
\end{enghelabbox}

\begin{noktebox}
	\textbf{نکته مهم:}
	
	ویشی یهودیان «خارجی» را راحت‌تر تحویل داد تا یهودیان «فرانسوی». این «معامله» نشان می‌داد که ویشی یهودستیزی را پذیرفته بود، فقط می‌خواست «یهودیان خودش» را نگه دارد—که البته نتوانست.
\end{noktebox}

\subsection{کار اجباری (\lr{STO})}

از فوریه ۱۹۴۳، «خدمت کار اجباری» (\lr{Service du travail obligatoire}) برقرار شد:

\begin{itemize}
	\item \textbf{هدف:} ارسال کارگر به آلمان
	\item \textbf{تعداد:} ۶۵۰,۰۰۰ فرانسوی به آلمان فرستاده شدند
	\item \textbf{پیامد:} هزاران جوان به جنگل‌ها گریختند و به «ماکی» (مقاومت روستایی) پیوستند
\end{itemize}

%──────────────────────────────────────────────────────────────────────────────
\section{مقاومت (۱۹۴۰-۱۹۴۴)}
%──────────────────────────────────────────────────────────────────────────────

\subsection{فرانسه آزاد و دوگل}

در ۱۸ ژوئن ۱۹۴۰، ژنرال شارل دوگل از لندن ندای مقاومت سر داد:

\begin{naghlbox}
	\textbf{ندای ۱۸ ژوئن ۱۹۴۰}
	
	«آیا آخرین کلمه گفته شده است؟ آیا امید باید ناپدید شود؟ آیا شکست قطعی است؟ نه!
	
	باور کنید مرا، من که با آگاهی کامل سخن می‌گویم، و به شما می‌گویم که هیچ چیز از دست نرفته است برای فرانسه. همان ابزارهایی که ما را شکست دادند، می‌توانند روزی پیروزی را بیاورند.
	
	زیرا فرانسه تنها نیست!... این جنگ، جنگ جهانی است.
	
	من، ژنرال دوگل، که اکنون در لندن هستم، از افسران و سربازان فرانسوی... از مهندسان و کارگران متخصص صنایع تسلیحاتی... دعوت می‌کنم که با من تماس بگیرند.
	
	هر چه پیش آید، شعله مقاومت فرانسه نباید خاموش شود و خاموش نخواهد شد.»
	
	\hfill --- \textit{شارل دوگل، ۱۸ ژوئن ۱۹۴۰}
\end{naghlbox}

\begin{table}[htbp]
	\centering
	\caption{شارل دوگل در ۱۹۴۰}
	\label{tab:de-gaulle-1940}
	\begin{tabular}{|r|p{10cm}|}
		\hline
		\rowcolor{bleumid}
		\textcolor{white}{\textbf{ویژگی}} & \textcolor{white}{\textbf{توضیح}} \\
		\hline
		\textbf{سن} & ۴۹ سال \\
		\hline
		\rowcolor{bleulight}
		\textbf{درجه} & ژنرال بریگاد (تازه ارتقا یافته) \\
		\hline
		\textbf{سابقه} & فرمانده تانک در ۱۹۴۰، معاون وزیر جنگ (چند روز) \\
		\hline
		\rowcolor{bleulight}
		\textbf{شهرت} & نویسنده نظامی، طرفدار جنگ مکانیزه \\
		\hline
		\textbf{وضعیت قانونی} & محکوم به اعدام غیابی توسط ویشی \\
		\hline
		\rowcolor{bleulight}
		\textbf{منابع اولیه} & چند صد داوطلب، حمایت محدود چرچیل \\
		\hline
	\end{tabular}
\end{table}

\subsection{تحول فرانسه آزاد}

\begin{tikzpicture}[
	every node/.style={font=\small},
	phase/.style={rectangle, rounded corners, draw=bleurepublique, fill=bleulight,
		minimum width=5cm, minimum height=2cm, align=center}
	]
	% Title
	\node[font=\bfseries\large] at (7,8) {تحول فرانسه آزاد به فرانسه مبارز};
	
	% Timeline
	\draw[very thick, bleurepublique] (0,5.5) -- (14,5.5);
	
	% Phases
	\node[phase] (p1) at (2.5,3) {\textbf{۱۹۴۰-۱۹۴۲}\\«فرانسه آزاد»\\نیروهای محدود\\آفریقای استوایی};
	
	\node[phase] (p2) at (7.5,3) {\textbf{۱۹۴۲-۱۹۴۳}\\«فرانسه مبارز»\\گسترش، الجزایر\\ادغام با ژیرو};
	
	\node[phase] (p3) at (12,3) {\textbf{۱۹۴۳-۱۹۴۴}\\«\lr{CFLN/GPRF}»\\دولت موقت\\به رسمیت شناخته شده};
	
	% Markers
	\fill[bleurepublique] (2.5,5.5) circle (0.12);
	\fill[bleurepublique] (7.5,5.5) circle (0.12);
	\fill[bleurepublique] (12,5.5) circle (0.12);
	
	% Key events
	\node[font=\footnotesize] at (2.5,6.2) {ژوئن ۱۹۴۰};
	\node[font=\footnotesize] at (7.5,6.2) {نوامبر ۱۹۴۲};
	\node[font=\footnotesize] at (12,6.2) {ژوئن ۱۹۴۴};
	
	% Arrows
	\draw[->, very thick, bleurepublique] (p1) -- (p2);
	\draw[->, very thick, bleurepublique] (p2) -- (p3);
	
\end{tikzpicture}

\subsection{مقاومت داخلی}

مقاومت داخلی از گروه‌های کوچک و پراکنده شروع شد و تدریجاً سازمان‌یافته شد:

\begin{table}[htbp]
	\centering
	\caption{جنبش‌های اصلی مقاومت داخلی}
	\label{tab:resistance-movements}
	\begin{tabular}{|r|c|p{5cm}|}
		\hline
		\rowcolor{rougemid}
		\textcolor{white}{\textbf{جنبش}} & \textcolor{white}{\textbf{منطقه}} & \textcolor{white}{\textbf{گرایش}} \\
		\hline
		کمبا (\lr{Combat}) & جنوب & مسیحی دموکرات، ملی \\
		\hline
		\rowcolor{rougelight}
		فران-تیرور (\lr{Franc-Tireur}) & جنوب & سوسیالیست \\
		\hline
		لیبراسیون (\lr{Libération}) & شمال و جنوب & چپ، سندیکالیست \\
		\hline
		\rowcolor{rougelight}
		\lr{FTP} & سراسری & کمونیست (از ۱۹۴۱) \\
		\hline
		دفاع از فرانسه & شمال & گلیست \\
		\hline
	\end{tabular}
\end{table}

\subsubsection{ژان مولَن و شورای ملی مقاومت}

\begin{olgoobox}
	\textbf{ژان مولَن (۱۸۹۹-۱۹۴۳)}
	
	\begin{itemize}[nosep]
		\item استاندار جوان، مقاوم از ۱۹۴۰
		\item فرستاده دوگل به فرانسه (ژانویه ۱۹۴۲)
		\item مأموریت: متحد کردن مقاومت تحت رهبری دوگل
		\item \textbf{۲۷ مه ۱۹۴۳:} تشکیل شورای ملی مقاومت (\lr{CNR})
		\item \textbf{۲۱ ژوئن ۱۹۴۳:} دستگیری در کالویر
		\item \textbf{۸ ژوئیه ۱۹۴۳:} مرگ زیر شکنجه کلاوس باربی
	\end{itemize}
	
	خاکستر مولَن در ۱۹۶۴ به پانتئون منتقل شد.
\end{olgoobox}

\begin{naghlbox}
	«او کسی بود که شعله یک چراغ ضعیف را حفظ کرد؛ و مهم نیست: شعله روشن بود.»
	
	\hfill --- \textit{آندره مالرو، در مراسم پانتئون، ۱۹۶۴}
\end{naghlbox}

\subsection{برنامه شورای ملی مقاومت (مارس ۱۹۴۴)}

\begin{olgoobox}
	\textbf{برنامه \lr{CNR} — «روزهای خوش»}
	
	برنامه‌ای برای فرانسه پس از آزادی، توافق‌شده میان همه گرایش‌ها (از گلیست‌ها تا کمونیست‌ها):
	
	\textbf{اقدامات فوری:}
	\begin{itemize}[nosep]
		\item برقراری مجدد آزادی‌های جمهوری
		\item مجازات خائنان
		\item مصادره اموال همکاران
	\end{itemize}
	
	\textbf{اصلاحات ساختاری:}
	\begin{itemize}[nosep]
		\item ملی‌سازی انرژی، بیمه، بانک‌های بزرگ
		\item برنامه‌ریزی اقتصادی
		\item حق مشارکت کارگران در مدیریت
		\item تأمین اجتماعی جامع
		\item حق کار و استراحت
		\item آموزش برابر برای همه
	\end{itemize}
	
	این برنامه، اساس دولت رفاه فرانسه شد.
\end{olgoobox}

\subsection{نبرد و آزادی}

\begin{tikzpicture}[
	every node/.style={font=\small},
	event/.style={rectangle, rounded corners, draw=bleurepublique, fill=bleulight,
		minimum width=3.5cm, minimum height=1.3cm, align=center}
	]
	% Title
	\node[font=\bfseries\large] at (7,7.5) {آزادی فرانسه (۱۹۴۴)};
	
	% Timeline
	\draw[very thick, bleurepublique] (0,5.5) -- (14,5.5);
	
	% Events
	\node[event] (e1) at (2,3.5) {\textbf{۶ ژوئن}\\پیاده‌روی نرماندی\\روز دی};
	
	\node[event] (e2) at (5.5,3.5) {\textbf{۱۵ اوت}\\پیاده‌روی\\پروونس};
	
	\node[event] (e3) at (9,3.5) {\textbf{۱۹-۲۵ اوت}\\قیام پاریس\\آزادی پایتخت};
	
	\node[event] (e4) at (12.5,3.5) {\textbf{پاییز ۱۹۴۴}\\آزادی کامل\\(جز «جیب‌ها»)};
	
	% Markers
	\foreach \x in {2, 5.5, 9, 12.5} {
		\fill[bleurepublique] (\x,5.5) circle (0.1);
		\draw[->, thick, bleurepublique] (\x,5.5) -- (\x,4.3);
	}
	
	% De Gaulle's entry
	\node[rectangle, draw=vertnapoleon, fill=vertlight, rounded corners, align=center] at (7,1) {
		\textbf{۲۶ اوت ۱۹۴۴:} رژه دوگل در شانزه‌لیزه\\
		تثبیت مشروعیت دولت موقت\\
		جلوگیری از قدرت‌گیری کمونیست‌ها یا متفقین
	};
	
\end{tikzpicture}

%──────────────────────────────────────────────────────────────────────────────
\section{دولت موقت و تصفیه (۱۹۴۴-۱۹۴۶)}
%──────────────────────────────────────────────────────────────────────────────

\subsection{دولت موقت جمهوری فرانسه (\lr{GPRF})}

دوگل در ۲۵ اوت ۱۹۴۴ وارد پاریس شد و دولت موقت را تشکیل داد—ائتلافی از گلیست‌ها، کمونیست‌ها، سوسیالیست‌ها، و دموکرات‌مسیحی‌ها (\lr{MRP}).

\begin{table}[htbp]
	\centering
	\caption{اقدامات دولت موقت (۱۹۴۴-۱۹۴۶)}
	\label{tab:gprf-actions}
	\begin{tabular}{|c|p{9cm}|}
		\hline
		\rowcolor{bleumid}
		\textcolor{white}{\textbf{سال}} & \textcolor{white}{\textbf{اقدام}} \\
		\hline
		۱۹۴۴ & حق رأی زنان (اولین بار در تاریخ فرانسه) \\
		\hline
		\rowcolor{bleulight}
		۱۹۴۴-۴۵ & ملی‌سازی‌ها: رنو، ایر فرانس، زغال‌سنگ، بانک فرانسه \\
		\hline
		۱۹۴۵ & تأسیس تأمین اجتماعی (\lr{Sécurité sociale}) \\
		\hline
		\rowcolor{bleulight}
		۱۹۴۵ & تأسیس کمیساریای برنامه‌ریزی (ژان مونه) \\
		\hline
		۱۹۴۵ & تأسیس مدرسه ملی مدیریت (\lr{ENA}) \\
		\hline
		\rowcolor{bleulight}
		۱۹۴۶ & قانون اساسی جدید (پس از دو همه‌پرسی) \\
		\hline
	\end{tabular}
\end{table}

\subsection{تصفیه (\lr{Épuration})}

پس از آزادی، موج انتقام از «همکاران» آغاز شد:

\begin{enghelabbox}
	\textbf{تصفیه: آمار و ابعاد}
	
	\textbf{تصفیه غیررسمی («وحشی»):}
	\begin{itemize}[nosep]
		\item اعدام‌های بدون محاکمه: ۹,۰۰۰ تا ۱۰,۰۰۰ نفر
		\item تراشیدن سر زنان («همکاران افقی»): حدود ۲۰,۰۰۰
		\item خشونت، غارت، انتقام‌جویی
	\end{itemize}
	
	\textbf{تصفیه قانونی:}
	\begin{itemize}[nosep]
		\item بازداشت‌ها: بیش از ۱۲۰,۰۰۰ نفر
		\item محاکمات: حدود ۱۲۵,۰۰۰ نفر
		\item احکام اعدام: ۷,۰۳۷ (اجرا شده: ۱,۵۰۰)
		\item حبس: ۳۸,۰۰۰
		\item محرومیت از حقوق مدنی: ۴۹,۰۰۰
	\end{itemize}
	
	\textbf{محاکمات مهم:}
	\begin{itemize}[nosep]
		\item \textbf{پتن:} محکوم به اعدام، تبدیل به حبس ابد (مرگ ۱۹۵۱)
		\item \textbf{لاوال:} اعدام (اکتبر ۱۹۴۵)
		\item \textbf{براسیاک:} اعدام (فوریه ۱۹۴۵)
	\end{itemize}
\end{enghelabbox}

\begin{naghlbox}
	«تصفیه ناقص بود، اما اگر کامل‌تر بود، فرانسه را پاره‌پاره می‌کرد.»
	
	\hfill --- \textit{تحلیل رایج}
\end{naghlbox}

\subsection{استعفای دوگل (ژانویه ۱۹۴۶)}

در ۲۰ ژانویه ۱۹۴۶، دوگل به طور ناگهانی استعفا داد. دلیل: مخالفت با «رژیم احزاب» که در حال شکل‌گیری بود.

\begin{noktebox}
	\textbf{چرا دوگل رفت؟}
	
	\begin{itemize}[nosep]
		\item مخالفت با قانون اساسی پارلمانتاریستی در حال تدوین
		\item اعتقاد به ضرورت قوه مجریه قوی
		\item امتناع از سازش با احزاب
		\item امید (نادرست) به اینکه ملت او را فرا خواهد خواند
	\end{itemize}
	
	دوگل ۱۲ سال در «بیابان» سیاسی ماند تا بحران الجزایر او را بازگرداند.
\end{noktebox}

%──────────────────────────────────────────────────────────────────────────────
\section{جمهوری چهارم (۱۹۴۶-۱۹۵۸)}
%──────────────────────────────────────────────────────────────────────────────

\subsection{تأسیس}

پس از رد اولین پیش‌نویس قانون اساسی در همه‌پرسی مه ۱۹۴۶، دومین پیش‌نویس با اکثریت ضعیف (۵۳٪) در اکتبر ۱۹۴۶ تصویب شد.

\begin{table}[htbp]
	\centering
	\caption{مقایسه جمهوری سوم و چهارم}
	\label{tab:third-fourth-comparison}
	\begin{tabular}{|r|p{5cm}|p{5cm}|}
		\hline
		\rowcolor{bleumid}
		\textcolor{white}{\textbf{ویژگی}} & \textcolor{white}{\textbf{جمهوری سوم}} & \textcolor{white}{\textbf{جمهوری چهارم}} \\
		\hline
		نظام & پارلمانی & پارلمانی (تقویت‌شده) \\
		\hline
		\rowcolor{bleulight}
		مجلسین & هر دو قوی & مجلس ملی قوی‌تر \\
		\hline
		رئیس‌جمهور & تشریفاتی & تشریفاتی \\
		\hline
		\rowcolor{bleulight}
		دولت & ناپایدار & ناپایدار \\
		\hline
		حق رأی & مردان & همگانی (زنان از ۱۹۴۴) \\
		\hline
		\rowcolor{bleulight}
		مدت & ۷۰ سال & ۱۲ سال \\
		\hline
		تعداد دولت‌ها & ۱۰۸ & ۲۴ \\
		\hline
	\end{tabular}
\end{table}

\begin{tikzpicture}[
	every node/.style={font=\small},
	box/.style={rectangle, rounded corners, draw=bleurepublique, fill=bleulight,
		minimum width=3.5cm, minimum height=1.5cm, align=center}
	]
	% Title
	\node[font=\bfseries\large] at (7,7.5) {ساختار جمهوری چهارم};
	
	% President
	\node[box, fill=grisclair, draw=gris] (pres) at (7,6) 
	{\textbf{رئیس‌جمهور}\\انتخاب توسط مجلسین\\تشریفاتی};
	
	% National Assembly
	\node[box] (assembly) at (3,3.5) 
	{\textbf{مجلس ملی}\\انتخاب مستقیم\\قدرت اصلی};
	
	% Council of Republic
	\node[box, fill=grisclair, draw=gris] (council) at (11,3.5) 
	{\textbf{شورای جمهوری}\\جایگزین سنا\\اختیارات محدود};
	
	% Government
	\node[box, fill=rougelight, draw=rougerevolution] (gov) at (7,1) 
	{\textbf{دولت}\\مسئول در برابر مجلس\\«سرمایه‌گذاری» + رأی اعتماد};
	
	% Arrows
	\draw[->, thick] (assembly) -- (gov) node[midway, left] {کنترل};
	\draw[->, thick] (pres) -- (gov) node[midway, right] {انتصاب نخست‌وزیر};
	\draw[<->, thick, dashed] (assembly) -- (council);
	
\end{tikzpicture}

\subsection{سه‌جانبه‌گرایی (۱۹۴۴-۱۹۴۷)}

از آزادی تا ۱۹۴۷، سه حزب بزرگ با هم حکومت کردند:

\begin{tikzpicture}[
	every node/.style={font=\small},
	party/.style={rectangle, rounded corners, minimum width=3.5cm, minimum height=2cm, align=center}
	]
	% Title
	\node[font=\bfseries\large] at (7,7) {سه‌جانبه‌گرایی (\lr{Tripartisme})};
	
	% Parties
	\node[party, draw=rougerevolution, fill=rougemid, text=white] (pcf) at (2,4) 
	{\textbf{\lr{PCF}}\\حزب کمونیست\\۲۸٪ آرا (۱۹۴۶)};
	
	\node[party, draw=rougerevolution, fill=rougelight] (sfio) at (7,4) 
	{\textbf{\lr{SFIO}}\\سوسیالیست\\۲۱٪ آرا};
	
	\node[party, draw=bleurepublique, fill=bleulight] (mrp) at (12,4) 
	{\textbf{\lr{MRP}}\\دموکرات‌مسیحی\\۲۶٪ آرا};
	
	% Coalition
	\node[rectangle, draw=vertnapoleon, fill=vertlight, rounded corners,
	minimum width=10cm, minimum height=1.2cm, align=center] at (7,1.5) 
	{\textbf{ائتلاف حاکم:} بیش از ۷۵٪ آرا\\اجماع بر بازسازی، تأمین اجتماعی، ملی‌سازی};
	
	% Arrows
	\draw[->, thick] (pcf) -- (7,2.2);
	\draw[->, thick] (sfio) -- (7,2.2);
	\draw[->, thick] (mrp) -- (7,2.2);
	
\end{tikzpicture}

\subsection{جنگ سرد و پایان سه‌جانبه‌گرایی (۱۹۴۷)}

در مه ۱۹۴۷، نخست‌وزیر رامادیه وزرای کمونیست را برکنار کرد:

\begin{itemize}
	\item \textbf{دلیل رسمی:} اختلاف بر سر سیاست دستمزد
	\item \textbf{دلیل واقعی:} جنگ سرد، فشار آمریکا (طرح مارشال)
	\item \textbf{نتیجه:} \lr{PCF} تا ۱۹۸۱ از دولت بیرون ماند
\end{itemize}

\subsection{«نیروی سوم» و بی‌ثباتی}

پس از ۱۹۴۷، جمهوری چهارم میان دو اپوزیسیون گیر افتاد:

\begin{tikzpicture}[
	every node/.style={font=\small},
	opp/.style={rectangle, rounded corners, draw=rougerevolution, fill=rougelight,
		minimum width=4cm, minimum height=2cm, align=center}
	]
	% Title
	\node[font=\bfseries\large] at (7,7) {جمهوری چهارم میان دو آتش};
	
	% Left opposition
	\node[opp, fill=rougemid, text=white] (left) at (2,4) 
	{\textbf{چپ:}\\حزب کمونیست\\۲۵-۲۸٪ آرا\\«حزب شوروی»};
	
	% Center - government
	\node[rectangle, rounded corners, draw=bleurepublique, fill=bleulight,
	minimum width=5cm, minimum height=2cm, align=center] (center) at (7,4) 
	{\textbf{«نیروی سوم»:}\\سوسیالیست‌ها + \lr{MRP} + رادیکال‌ها\\حکومت با اکثریت شکننده};
	
	% Right opposition
	\node[opp, fill=violetlight, draw=violetempire] (right) at (12,4) 
	{\textbf{راست:}\\گلیست‌ها (\lr{RPF})\\پوژادیست‌ها\\ضد نظام};
	
	% Pressure arrows
	\draw[->, ultra thick, rougerevolution] (left) -- (center);
	\draw[->, ultra thick, violetempire] (right) -- (center);
	
	% Result
	\node[rectangle, draw=gris, fill=grisclair, rounded corners, align=center] at (7,1) 
	{\textbf{نتیجه:} دولت‌های ائتلافی ضعیف\\۲۴ دولت در ۱۲ سال — میانگین ۶ ماه};
	
\end{tikzpicture}

\subsection{دستاوردهای اقتصادی}

علی‌رغم بی‌ثباتی سیاسی، جمهوری چهارم دوره رشد اقتصادی بود:

\begin{table}[htbp]
	\centering
	\caption{شاخص‌های اقتصادی جمهوری چهارم}
	\label{tab:fourth-republic-economy}
	\begin{tabular}{|r|c|c|c|}
		\hline
		\rowcolor{vertmid}
		\textcolor{white}{\textbf{شاخص}} & \textcolor{white}{\textbf{۱۹۴۶}} & \textcolor{white}{\textbf{۱۹۵۸}} & \textcolor{white}{\textbf{رشد}} \\
		\hline
		تولید صنعتی (شاخص) & ۷۰ & ۱۸۰ & +۱۵۷٪ \\
		\hline
		\rowcolor{vertlight}
		تولید ناخالص داخلی & پایه & +۵۰٪ & ۴.۵٪ سالانه \\
		\hline
		خودرو سالانه (هزار) & ۸۵ & ۱,۱۰۰ & +۱,۲۰۰٪ \\
		\hline
		\rowcolor{vertlight}
		مصرف برق & پایه & ۳ برابر & — \\
		\hline
	\end{tabular}
\end{table}

\begin{olgoobox}
	\textbf{عوامل رشد اقتصادی}
	
	\begin{itemize}[nosep]
		\item \textbf{طرح مارشال:} ۲.۵ میلیارد دلار کمک آمریکا (۱۹۴۸-۱۹۵۲)
		\item \textbf{برنامه‌ریزی:} طرح‌های مونه (مدرنیزاسیون)
		\item \textbf{ملی‌سازی:} سرمایه‌گذاری دولتی در زیرساخت
		\item \textbf{جامعه زغال و فولاد:} آغاز همگرایی اروپایی (۱۹۵۱)
		\item \textbf{رشد جمعیت:} بیبی‌بوم پس از جنگ
	\end{itemize}
\end{olgoobox}

\subsection{آغاز اتحادیه اروپا}

فرانسه نقش محوری در ساختن اروپای متحد داشت:

\begin{table}[htbp]
	\centering
	\caption{گام‌های اروپایی جمهوری چهارم}
	\label{tab:european-steps}
	\begin{tabular}{|c|p{8cm}|}
		\hline
		\rowcolor{bleumid}
		\textcolor{white}{\textbf{سال}} & \textcolor{white}{\textbf{رویداد}} \\
		\hline
		۱۹۵۰ & اعلامیه شومان: پیشنهاد جامعه زغال و فولاد \\
		\hline
		\rowcolor{bleulight}
		۱۹۵۱ & معاهده پاریس: تأسیس \lr{CECA} (شش کشور) \\
		\hline
		۱۹۵۴ & رد ارتش اروپایی (\lr{CED}) توسط مجلس فرانسه \\
		\hline
		\rowcolor{bleulight}
		۱۹۵۷ & معاهده رم: تأسیس جامعه اقتصادی اروپا (\lr{CEE}) \\
		\hline
	\end{tabular}
\end{table}

%──────────────────────────────────────────────────────────────────────────────
\section{جنگ‌های استعماری}
%──────────────────────────────────────────────────────────────────────────────

\subsection{هندوچین (۱۹۴۶-۱۹۵۴)}

\begin{enghelabbox}
	\textbf{جنگ هندوچین}
	
	\begin{itemize}[nosep]
		\item \textbf{آغاز:} بمباران هایفونگ (نوامبر ۱۹۴۶)
		\item \textbf{دشمن:} ویت‌مین به رهبری هو شی مین
		\item \textbf{ماهیت:} جنگ چریکی، کمونیسم vs استعمار
		\item \textbf{تلفات فرانسه:} ۹۲,۰۰۰ (از جمله ۲۰,۰۰۰ فرانسوی)
		\item \textbf{هزینه:} بیش از بودجه طرح مارشال برای فرانسه
		\item \textbf{فاجعه نهایی:} دین‌بین‌فو (مارس-مه ۱۹۵۴)
		\item \textbf{نتیجه:} قراردادهای ژنو، خروج فرانسه، تقسیم ویتنام
	\end{itemize}
\end{enghelabbox}

\begin{naghlbox}
	«دین‌بین‌فو نه فقط یک شکست نظامی بود، بلکه پایان توهم فرانسه بزرگ استعماری بود.»
	
	\hfill --- \textit{تحلیل مورخان}
\end{naghlbox}

\subsection{شروع جنگ الجزایر (۱۹۵۴)}

در ۱ نوامبر ۱۹۵۴—همان روز امضای قراردادهای ژنو—جبهه آزادی‌بخش ملی الجزایر با حملات هماهنگ در سراسر الجزایر، قیام مسلحانه را آغاز کرد.

\begin{table}[htbp]
	\centering
	\caption{الجزایر در نظام استعماری فرانسه}
	\label{tab:algeria-status}
	\begin{tabular}{|r|p{9cm}|}
		\hline
		\rowcolor{rougemid}
		\textcolor{white}{\textbf{ویژگی}} & \textcolor{white}{\textbf{توضیح}} \\
		\hline
		\textbf{وضعیت حقوقی} & سه «دپارتمان» فرانسه (نه مستعمره)—بخشی از خاک فرانسه \\
		\hline
		\rowcolor{rougelight}
		\textbf{جمعیت (۱۹۵۴)} & ۹ میلیون مسلمان، ۱ میلیون اروپایی‌تبار («پیه‌نوار») \\
		\hline
		\textbf{سابقه استعمار} & از ۱۸۳۰ (۱۲۴ سال) \\
		\hline
		\rowcolor{rougelight}
		\textbf{مهاجران فرانسوی} & نسل سوم و چهارم، الجزایر را «وطن» می‌دانستند \\
		\hline
		\textbf{نابرابری} & مسلمانان شهروند درجه دو، محرومیت اقتصادی و سیاسی \\
		\hline
	\end{tabular}
\end{table}

\begin{noktebox}
	\textbf{چرا الجزایر متفاوت بود؟}
	
	\begin{itemize}[nosep]
		\item \textbf{وضعیت حقوقی:} الجزایر «فرانسه» بود، نه مستعمره
		\item \textbf{جمعیت اروپایی:} یک میلیون «پیه‌نوار» با نفوذ سیاسی قوی
		\item \textbf{احساسات:} پس از شکست هندوچین، «دیگر عقب‌نشینی نه»
		\item \textbf{ارتش:} افسران می‌خواستند «شکست را جبران کنند»
		\item \textbf{نفت:} کشف نفت صحرا در ۱۹۵۶
	\end{itemize}
\end{noktebox}

\subsubsection{تشدید جنگ}

\begin{tikzpicture}[
	every node/.style={font=\small},
	event/.style={rectangle, rounded corners, draw=rougerevolution, fill=rougelight,
		minimum width=3cm, minimum height=1.2cm, align=center}
	]
	% Title
	\node[font=\bfseries\large] at (7,7.5) {تشدید جنگ الجزایر (۱۹۵۴-۱۹۵۸)};
	
	% Timeline
	\draw[very thick, rougerevolution] (0,5.5) -- (14,5.5);
	
	% Events
	\node[event] (e1) at (1.5,3.5) {\textbf{نوامبر ۱۹۵۴}\\«توسن سرخ»\\آغاز قیام};
	
	\node[event] (e2) at (5,3.5) {\textbf{اوت ۱۹۵۵}\\کشتار فیلیپ‌ویل\\قسطنطنیه};
	
	\node[event] (e3) at (8.5,3.5) {\textbf{۱۹۵۶-۵۷}\\نبرد الجزیره\\شکنجه سیستماتیک};
	
	\node[event] (e4) at (12,3.5) {\textbf{فوریه ۱۹۵۸}\\بمباران ساکیه\\بحران بین‌المللی};
	
	% Markers
	\foreach \x in {1.5, 5, 8.5, 12} {
		\fill[rougerevolution] (\x,5.5) circle (0.1);
		\draw[->, thick, rougerevolution] (\x,5.5) -- (\x,4.3);
	}
	
	% Statistics box
	\node[rectangle, draw=gris, fill=grisclair, rounded corners, align=center] at (7,1) {
		\textbf{نیروی فرانسه:} ۴۰۰,۰۰۰ سرباز (۱۹۵۶)\\
		\textbf{سربازان وظیفه:} برای اولین بار در جنگ استعماری\\
		\textbf{روش‌ها:} جمع‌آوری جمعیت، شکنجه، اعدام‌های فراقانونی
	};
	
\end{tikzpicture}

\subsubsection{شکنجه و بحران اخلاقی}

\begin{enghelabbox}
	\textbf{مسئله شکنجه}
	
	در جریان «نبرد الجزیره» (ژانویه-اکتبر ۱۹۵۷)، ارتش فرانسه به شکنجه سیستماتیک روی آورد:
	
	\begin{itemize}[nosep]
		\item برق‌گرفتگی («ژژن»)
		\item غرق‌کردن («باینوار»)
		\item شکنجه جنسی
		\item «ناپدیدشدن» زندانیان
	\end{itemize}
	
	\textbf{واکنش‌ها:}
	\begin{itemize}[nosep]
		\item افشاگری‌ها: کتاب «سؤال» (هانری آلگ، ۱۹۵۸)
		\item اعتراض روشنفکران: سارتر، موریاک، کامو
		\item استعفای ژنرال دو بولاردیه (تنها ژنرال معترض)
		\item بحران وجدان ملی
	\end{itemize}
\end{enghelabbox}

\begin{naghlbox}
	«نه قربانیان و نه شکنجه‌گران فراموش نخواهند کرد. شکنجه در خاطره ملت‌ها حک می‌شود.»
	
	\hfill --- \textit{ژان-پل سارتر}
\end{naghlbox}

%──────────────────────────────────────────────────────────────────────────────
\section{سقوط جمهوری چهارم (۱۹۵۸)}
%──────────────────────────────────────────────────────────────────────────────

\subsection{بحران مه ۱۹۵۸}

جمهوری چهارم در بحران الجزایر غرق شد. دولت‌ها یکی پس از دیگری سقوط می‌کردند، بدون آنکه راه‌حلی بیابند.

\begin{tikzpicture}[
	every node/.style={font=\small},
	event/.style={rectangle, rounded corners, draw=rougerevolution, fill=rougelight,
		minimum width=4cm, minimum height=1.5cm, align=center}
	]
	% Title
	\node[font=\bfseries\large] at (7,8) {بحران مه ۱۹۵۸: سقوط جمهوری چهارم};
	
	% Timeline
	\draw[very thick, rougerevolution] (0,6) -- (14,6);
	
	% Events
	\node[event] (e1) at (2,4) {\textbf{۱۳ مه}\\شورش در الجزیره\\تصرف ساختمان دولتی\\«کمیته نجات عمومی»};
	
	\node[event] (e2) at (7,4) {\textbf{۱۵ مه}\\ژنرال سالان:\\«دوگل زنده باد!»\\ارتش خواهان دوگل};
	
	\node[event] (e3) at (12,4) {\textbf{۲۴ مه}\\«عملیات احیا»\\چترباران کُرس\\تهدید پاریس};
	
	% Result
	\node[rectangle, draw=bleurepublique, fill=bleumid, text=white, rounded corners,
	minimum width=10cm, minimum height=1.5cm, align=center] (result) at (7,1) {
		\textbf{۱ ژوئن ۱۹۵۸:} دوگل نخست‌وزیر شد (۳۲۹ رأی موافق، ۲۲۴ مخالف)\\
		\textbf{۲ ژوئن:} اختیارات تام برای ۶ ماه + تدوین قانون اساسی جدید
	};
	
	% Arrows
	\draw[->, ultra thick, rougerevolution] (e1) -- (e2);
	\draw[->, ultra thick, rougerevolution] (e2) -- (e3);
	\draw[->, ultra thick, bleurepublique] (7,2.5) -- (result);
	
\end{tikzpicture}

\subsection{چگونه دوگل بازگشت؟}

\begin{table}[htbp]
	\centering
	\caption{مراحل بازگشت دوگل به قدرت}
	\label{tab:de-gaulle-return}
	\begin{tabular}{|c|p{9cm}|}
		\hline
		\rowcolor{bleumid}
		\textcolor{white}{\textbf{تاریخ}} & \textcolor{white}{\textbf{رویداد}} \\
		\hline
		۱۳ مه & شورش نظامی-مدنی در الجزیره، دولت پفلیملن ناتوان \\
		\hline
		\rowcolor{bleulight}
		۱۵ مه & دوگل: «آماده‌ام مسئولیت جمهوری را بپذیرم» \\
		\hline
		۱۹ مه & کنفرانس مطبوعاتی دوگل: رد کودتا، اما آماده برای قدرت \\
		\hline
		\rowcolor{bleulight}
		۲۴ مه & چتربازان کُرس را گرفتند—تهدید «عملیات احیا» علیه پاریس \\
		\hline
		۲۸ مه & استعفای پفلیملن \\
		\hline
		\rowcolor{bleulight}
		۲۹ مه & رئیس‌جمهور کوتی از دوگل برای نخست‌وزیری دعوت کرد \\
		\hline
		۱ ژوئن & مجلس به دولت دوگل رأی داد \\
		\hline
		\rowcolor{bleulight}
		۳ ژوئن & قانون اختیارات تام + حق تدوین قانون اساسی جدید \\
		\hline
	\end{tabular}
\end{table}

\begin{noktebox}
	\textbf{کودتا یا قانونی؟}
	
	بازگشت دوگل همیشه بحث‌برانگیز بوده:
	
	\begin{itemize}[nosep]
		\item \textbf{از نظر شکلی:} مجلس به دوگل رأی داد—قانونی
		\item \textbf{از نظر واقعی:} تحت تهدید کودتای نظامی—فشار
		\item \textbf{دیدگاه دوگل:} «من قانونی آمدم، جمهوری را نجات دادم»
		\item \textbf{دیدگاه چپ:} «کودتای آرام با کمک ارتش»
	\end{itemize}
	
	فرانسوا میتران کتابی نوشت با عنوان «کودتای دائم» (۱۹۶۴).
\end{noktebox}

\subsection{پایان جمهوری چهارم}

\begin{naghlbox}
	«جمهوری چهارم به دست خودش نمُرد. کُشته شد—توسط الجزایر.»
	
	\hfill --- \textit{تحلیل رایج}
\end{naghlbox}

در سپتامبر ۱۹۵۸، قانون اساسی جدید با ۷۹٪ آرا تصویب شد. در دسامبر ۱۹۵۸، دوگل رئیس‌جمهور شد. جمهوری پنجم آغاز شده بود.

%──────────────────────────────────────────────────────────────────────────────
\section{خط زمانی جامع}
%──────────────────────────────────────────────────────────────────────────────

\begin{landscape}
	\begin{tikzpicture}[
		every node/.style={font=\footnotesize},
		event/.style={rectangle, rounded corners, minimum width=1.3cm, minimum height=0.6cm, align=center}
		]
		% Title
		\node[font=\bfseries\large] at (12,10) {خط زمانی ویشی و جمهوری چهارم (۱۹۴۰-۱۹۵۸)};
		
		% Main timeline
		\draw[ultra thick, black] (0,7) -- (24,7);
		
		% Year markers
		\foreach \x/\year in {0/۱۹۴۰, 4/۱۹۴۴, 8/۱۹۴۶, 12/۱۹۵۰, 16/۱۹۵۴, 20/۱۹۵۶, 24/۱۹۵۸} {
			\draw[thick] (\x,6.7) -- (\x,7.3);
			\node at (\x,6.3) {\year};
		}
		
		% Period bars
		\fill[orroyallight] (0,8) rectangle (4,8.5);
		\node[font=\tiny] at (2,8.25) {ویشی/اشغال};
		\draw[orroyaldark, thick] (0,8) rectangle (4,8.5);
		
		\fill[rougelight] (4,8) rectangle (8,8.5);
		\node[font=\tiny] at (6,8.25) {آزادی/موقت};
		\draw[rougerevolution, thick] (4,8) rectangle (8,8.5);
		
		\fill[bleulight] (8,8) rectangle (24,8.5);
		\node[font=\tiny] at (16,8.25) {جمهوری چهارم};
		\draw[bleurepublique, thick] (8,8) rectangle (24,8.5);
		
		% Key events - above
		\node[event, draw=orroyaldark, fill=orroyallight] at (0.5,9.3) {ویشی\\ژوئیه ۴۰};
		\node[event, draw=bleurepublique, fill=bleulight] at (2,9.3) {ندای\\دوگل};
		\node[event, draw=rougerevolution, fill=rougelight] at (3,9.3) {وِل\\دیو ۴۲};
		\node[event, draw=bleurepublique, fill=bleulight] at (5,9.3) {آزادی\\پاریس};
		\node[event, draw=vertnapoleon, fill=vertlight] at (7,9.3) {تأمین\\اجتماعی};
		\node[event, draw=rougerevolution, fill=rougelight] at (16,9.3) {دین‌بین‌فو};
		\node[event, draw=rougerevolution, fill=rougelight] at (18,9.3) {سوئز\\۵۶};
		\node[event, draw=rougerevolution, fill=rougemid, text=white] at (24,9.3) {۱۳ مه\\۵۸};
		
		% Key events - below
		\node[event, draw=orroyaldark, fill=orroyallight] at (1,5) {مونتوار};
		\node[event, draw=rougerevolution, fill=rougelight] at (2.5,5) {\lr{STO}\\۴۳};
		\node[event, draw=bleurepublique, fill=bleulight] at (4.5,5) {\lr{CNR}};
		\node[event, draw=gris, fill=grisclair] at (6,5) {تصفیه};
		\node[event, draw=rougerevolution, fill=rougelight] at (9,5) {جنگ سرد\\۴۷};
		\node[event, draw=bleurepublique, fill=bleulight] at (12,5) {\lr{CECA}\\۵۱};
		\node[event, draw=bleurepublique, fill=bleulight] at (15,5) {\lr{CEE}\\۵۷};
		\node[event, draw=rougerevolution, fill=rougelight] at (17.5,5) {الجزایر\\۵۴-};
		
	\end{tikzpicture}
\end{landscape}

%──────────────────────────────────────────────────────────────────────────────
\section{تحلیل: ویشی در حافظه ملی}
%──────────────────────────────────────────────────────────────────────────────

\begin{tikzpicture}[
	every node/.style={font=\small},
	phase/.style={rectangle, rounded corners, minimum width=5.5cm, minimum height=2cm, align=center}
	]
	% Title
	\node[font=\bfseries\large] at (7,8.5) {تحول حافظه ویشی در فرانسه};
	
	% Phases
	\node[phase, draw=gris, fill=grisclair] (p1) at (3,6) 
	{\textbf{۱۹۴۴-۱۹۷۰}\\«اسطوره رزیستانسیالیستی»\\همه مقاومت کردند\\(دوگل، کمونیست‌ها)};
	
	\node[phase, draw=rougerevolution, fill=rougelight] (p2) at (11,6) 
	{\textbf{۱۹۷۰-۱۹۹۰}\\«آینه شکسته»\\فیلم «غم و ترحم» (۱۹۷۱)\\افشای همکاری};
	
	\node[phase, draw=bleurepublique, fill=bleulight] (p3) at (3,2.5) 
	{\textbf{۱۹۹۰-۲۰۰۰}\\«اعتراف»\\شیراک ۱۹۹۵: مسئولیت دولت\\محاکمه پاپون (۱۹۹۷)};
	
	\node[phase, draw=vertnapoleon, fill=vertlight] (p4) at (11,2.5) 
	{\textbf{۲۰۰۰-حال}\\«یادآوری متوازن»\\تکریم «عادلان»\\تداوم بحث‌ها};
	
	% Arrows
	\draw[->, very thick] (p1) -- (p2);
	\draw[->, very thick] (p2) -- (p3);
	\draw[->, very thick] (p3) -- (p4);
	
\end{tikzpicture}

\begin{naghlbox}
	\textbf{سخنرانی شیراک (۱۶ ژوئیه ۱۹۹۵)}
	
	«این روزها که فرانسه جشن می‌گیرد، یادآور لحظات تاریک تاریخ ماست... فرانسه، سرزمین روشنگری و حقوق بشر، سرزمین پذیرش و پناه‌دهی، فرانسه در آن روز کاری جبران‌ناپذیر کرد. فرانسه به تعهدش خیانت کرد و تحویل‌دادنی‌هایش را به جلادانشان سپرد.»
	
	\hfill --- \textit{ژاک شیراک، پنجاهمین سالگرد وِل دیو}
\end{naghlbox}

%──────────────────────────────────────────────────────────────────────────────
\section{الگوها و درس‌ها}
%──────────────────────────────────────────────────────────────────────────────

\begin{olgoobox}
	\textbf{الگوهای کلیدی دوره ۱۹۴۰-۱۹۵۸}
	
	\begin{enumerate}
		\item \textbf{الگوی «شکست به‌مثابه فرصت برای ارتجاع»:}
		\begin{itemize}[nosep]
			\item ویشی از شکست ۱۹۴۰ برای برچیدن جمهوری استفاده کرد
			\item «انقلاب ملی» پروژه‌ای قدیمی بود که جنگ فرصتش را داد
			\item شباهت با ۱۸۱۵ (بازگشت بوربون‌ها پس از شکست ناپلئون)
		\end{itemize}
		
		\item \textbf{الگوی «مشروعیت دوگانه»:}
		\begin{itemize}[nosep]
			\item ویشی: مشروعیت قانونی (رأی مجلس)
			\item دوگل: مشروعیت تاریخی (ادامه جنگ، ارزش‌های جمهوری)
			\item تنش میان قانونیت و مشروعیت—موضوعی تکرارشونده
		\end{itemize}
		
		\item \textbf{الگوی «استعمار به‌مثابه بن‌بست»:}
		\begin{itemize}[nosep]
			\item هندوچین، الجزایر: هزینه سنگین، شکست نهایی
			\item جمهوری چهارم قربانی ناتوانی در استعمارزدایی شد
			\item دوگل همان کسی بود که توانست الجزایر را رها کند
		\end{itemize}
		
		\item \textbf{الگوی «بی‌ثباتی پارلمانی»:}
		\begin{itemize}[nosep]
			\item جمهوری چهارم تکرار ضعف جمهوری سوم بود
			\item ۲۴ دولت در ۱۲ سال
			\item این تجربه، قانون اساسی جمهوری پنجم را شکل داد
		\end{itemize}
		
		\item \textbf{الگوی «بحران به‌مثابه مامای تغییر»:}
		\begin{itemize}[nosep]
			\item جمهوری سوم در شکست ۱۹۴۰ مُرد
			\item جمهوری چهارم در بحران الجزایر مُرد
			\item جمهوری پنجم از بحران زاده شد
		\end{itemize}
	\end{enumerate}
\end{olgoobox}

%──────────────────────────────────────────────────────────────────────────────
\section{نقشه مفهومی: مسئله مشروعیت}
%──────────────────────────────────────────────────────────────────────────────

\begin{tikzpicture}[
	every node/.style={font=\small},
	entity/.style={rectangle, rounded corners, minimum width=4cm, minimum height=1.8cm, align=center}
	]
	% Title
	\node[font=\bfseries\large] at (7,9) {مسئله مشروعیت در دوره ۱۹۴۰-۱۹۴۴};
	
	% Entities
	\node[entity, draw=orroyaldark, fill=orroyallight] (vichy) at (3,6) 
	{\textbf{ویشی}\\مشروعیت قانونی\\رأی مجلس ۱۹۴۰\\«دولت فرانسوی»};
	
	\node[entity, draw=bleurepublique, fill=bleulight] (fl) at (11,6) 
	{\textbf{فرانسه آزاد}\\مشروعیت تاریخی\\ادامه جنگ\\«جمهوری واقعی»};
	
	% Question
	\node[rectangle, draw=rougerevolution, fill=rougelight, rounded corners,
	minimum width=6cm, minimum height=1.5cm, align=center] at (7,3) {
		\textbf{سؤال:} کدام‌یک «فرانسه واقعی» بود؟\\
		پاسخ تاریخ: فرانسه آزاد
	};
	
	% Vichy claims
	\node[font=\footnotesize, align=center] at (3,3.5) {
		حفاظت از فرانسویان\\
		جلوگیری از بدتر شدن\\
		ادامه دولت
	};
	
	% Free France claims
	\node[font=\footnotesize, align=center] at (11,3.5) {
		افتخار ملی\\
		ارزش‌های جمهوری\\
		پیروزی نهایی
	};
	
	% Arrows
	\draw[->, thick, orroyaldark] (vichy) -- (7,4.2);
	\draw[->, thick, bleurepublique] (fl) -- (7,4.2);
	
	% Post-war resolution
	\node[rectangle, draw=vertnapoleon, fill=vertlight, rounded corners,
	minimum width=10cm, minimum height=1cm, align=center] at (7,0.5) {
		\textbf{راه‌حل پس از جنگ:} ویشی «هیچ» بود—جمهوری هرگز متوقف نشد (دکترین دوگل)
	};
	
\end{tikzpicture}

%──────────────────────────────────────────────────────────────────────────────
\section{جمع‌بندی فصل}
%──────────────────────────────────────────────────────────────────────────────

\begin{kholasebox}
	\textbf{جمع‌بندی: ویشی و جمهوری چهارم (۱۹۴۰-۱۹۵۸)}
	
	\textbf{دوره ویشی (۱۹۴۰-۱۹۴۴):}
	\begin{itemize}[nosep]
		\item رژیم اقتدارگرا، ضد جمهوری، همکار با نازی‌ها
		\item «انقلاب ملی»: کار، خانواده، میهن
		\item یهودستیزی فعال: ۷۶,۰۰۰ یهودی به اردوگاه‌ها
		\item همکاری اقتصادی، نظامی، پلیسی
		\item نقطه سیاه تاریخ فرانسه
	\end{itemize}
	
	\textbf{مقاومت (۱۹۴۰-۱۹۴۴):}
	\begin{itemize}[nosep]
		\item فرانسه آزاد دوگل: از لندن و آفریقا
		\item مقاومت داخلی: کمونیست‌ها، گلیست‌ها، دیگران
		\item ژان مولَن و شورای ملی مقاومت
		\item برنامه \lr{CNR}: اساس دولت رفاه
	\end{itemize}
	
	\textbf{آزادی و تصفیه (۱۹۴۴-۱۹۴۶):}
	\begin{itemize}[nosep]
		\item اعدام‌های بدون محاکمه و قانونی
		\item محاکمه پتن و لاوال
		\item ملی‌سازی‌ها، تأمین اجتماعی، حق رأی زنان
	\end{itemize}
	
	\textbf{جمهوری چهارم (۱۹۴۶-۱۹۵۸):}
	\begin{itemize}[nosep]
		\item پارلمانتاریسم ادامه جمهوری سوم
		\item بی‌ثباتی: ۲۴ دولت در ۱۲ سال
		\item اما: رشد اقتصادی، آغاز اروپا، دولت رفاه
		\item جنگ‌های استعماری: هندوچین (شکست)، الجزایر (بحران)
		\item سقوط در ۱۳ مه ۱۹۵۸: بازگشت دوگل
	\end{itemize}
	
	\textbf{میراث:}
	\begin{itemize}[nosep]
		\item یادآوری دردناک ویشی و همکاری
		\item تأمین اجتماعی و دولت رفاه
		\item آغاز همگرایی اروپایی
		\item درس بی‌ثباتی پارلمانی برای جمهوری پنجم
		\item زخم الجزایر (تا امروز)
	\end{itemize}
\end{kholasebox}

%──────────────────────────────────────────────────────────────────────────────
\section*{منابع فصل}
%──────────────────────────────────────────────────────────────────────────────
\addcontentsline{toc}{section}{منابع فصل}

\begin{itemize}[nosep]
	\item Azéma, Jean-Pierre. \textit{From Munich to the Liberation, 1938-1944}. Cambridge: Cambridge UP, 1984.
	\item Burrin, Philippe. \textit{France under the Germans: Collaboration and Compromise}. New York: New Press, 1996.
	\item Gildea, Robert. \textit{Marianne in Chains: Daily Life in the Heart of France during the German Occupation}. New York: Metropolitan Books, 2002.
	\item Jackson, Julian. \textit{France: The Dark Years, 1940-1944}. Oxford: Oxford UP, 2001.
	\item Kedward, H.R. \textit{Resistance in Vichy France}. Oxford: Oxford UP, 1978.
	\item Paxton, Robert O. \textit{Vichy France: Old Guard and New Order, 1940-1944}. New York: Columbia UP, 1972.
	\item Rioux, Jean-Pierre. \textit{The Fourth Republic, 1944-1958}. Cambridge: Cambridge UP, 1987.
	\item Rousso, Henry. \textit{The Vichy Syndrome: History and Memory in France since 1944}. Cambridge: Harvard UP, 1991.
	\item Shepard, Todd. \textit{The Invention of Decolonization: The Algerian War and the Remaking of France}. Ithaca: Cornell UP, 2006.
	\item Wall, Irwin M. \textit{France, the United States, and the Algerian War}. Berkeley: UC Press, 2001.
\end{itemize}

%══════════════════════════════════════════════════════════════════════════════
% پایان فصل ۷
%══════════════════════════════════════════════════════════════════════════════

\end{document}