%══════════════════════════════════════════════════════════════════════════════
% فصل ۸: جمهوری پنجم (۱۹۵۸-۲۰۲۴)
%══════════════════════════════════════════════════════════════════════════════

\documentclass[12pt,a4paper]{book}

%──────────────────────────────────────────────────────────────────────────────
% پکیج‌های اصلی
%──────────────────────────────────────────────────────────────────────────────
\usepackage{fontspec}
\usepackage{geometry}
\usepackage{graphicx}
\usepackage{tikz}
\usepackage{tcolorbox}
\usepackage{booktabs}
\usepackage{array}
\usepackage{colortbl}
\usepackage{multirow}
\usepackage{enumitem}
\usepackage{fancyhdr}
\usepackage{pdflscape}
\usepackage{xcolor}
\usepackage{hyperref}

%──────────────────────────────────────────────────────────────────────────────
% تنظیمات صفحه
%──────────────────────────────────────────────────────────────────────────────
\geometry{
	a4paper,
	top=2.5cm,
	bottom=2.5cm,
	left=2.5cm,
	right=2.5cm,
	headheight=15pt
}

%──────────────────────────────────────────────────────────────────────────────
% کتابخانه‌های TikZ
%──────────────────────────────────────────────────────────────────────────────
\usetikzlibrary{shapes.geometric, arrows.meta, positioning, calc, decorations.pathreplacing, backgrounds}

%──────────────────────────────────────────────────────────────────────────────
% تعریف رنگ‌های اصلی
%──────────────────────────────────────────────────────────────────────────────
\definecolor{bleurepublique}{RGB}{0, 85, 164}
\definecolor{rougerevolution}{RGB}{190, 30, 45}
\definecolor{orroyal}{RGB}{218, 165, 32}
\definecolor{orroyaldark}{RGB}{184, 134, 11}
\definecolor{vertnapoleon}{RGB}{0, 102, 51}
\definecolor{violetempire}{RGB}{102, 51, 153}
\definecolor{gris}{RGB}{128, 128, 128}
\definecolor{grisclair}{RGB}{220, 220, 220}

%──────────────────────────────────────────────────────────────────────────────
% تعریف رنگ‌های کمکی (برای fill در TikZ)
%──────────────────────────────────────────────────────────────────────────────
\definecolor{bleulight}{RGB}{230, 240, 250}
\definecolor{bleumid}{RGB}{100, 149, 237}
\definecolor{rougelight}{RGB}{255, 230, 230}
\definecolor{rougemid}{RGB}{220, 100, 100}
\definecolor{vertlight}{RGB}{230, 250, 230}
\definecolor{vertmid}{RGB}{100, 180, 100}
\definecolor{violetlight}{RGB}{245, 235, 255}
\definecolor{violetmid}{RGB}{180, 140, 200}
\definecolor{orroyallight}{RGB}{255, 248, 220}
\definecolor{orroyalmid}{RGB}{238, 201, 0}
\definecolor{grislight}{RGB}{245, 245, 245}

%──────────────────────────────────────────────────────────────────────────────
% تنظیمات tcolorbox
%──────────────────────────────────────────────────────────────────────────────
\tcbuselibrary{skins, breakable}

% کادر خلاصه (آبی)
\newtcolorbox{kholasebox}{
	colback=bleulight,
	colframe=bleurepublique,
	fonttitle=\bfseries,
	title={},
	breakable,
	enhanced,
	boxrule=1pt,
	arc=3pt,
	left=10pt,
	right=10pt,
	top=10pt,
	bottom=10pt
}

% کادر نقل قول (طلایی)
\newtcolorbox{naghlbox}{
	colback=orroyallight,
	colframe=orroyaldark,
	fonttitle=\bfseries,
	title={},
	breakable,
	enhanced,
	boxrule=1pt,
	arc=3pt,
	left=10pt,
	right=10pt,
	top=10pt,
	bottom=10pt
}

% کادر الگو/درس (سبز)
\newtcolorbox{olgoobox}{
	colback=vertlight,
	colframe=vertnapoleon,
	fonttitle=\bfseries,
	title={},
	breakable,
	enhanced,
	boxrule=1pt,
	arc=3pt,
	left=10pt,
	right=10pt,
	top=10pt,
	bottom=10pt
}

% کادر انقلاب/بحران (قرمز)
\newtcolorbox{enghelabbox}{
	colback=rougelight,
	colframe=rougerevolution,
	fonttitle=\bfseries,
	title={},
	breakable,
	enhanced,
	boxrule=1pt,
	arc=3pt,
	left=10pt,
	right=10pt,
	top=10pt,
	bottom=10pt
}

% کادر امپراتوری (بنفش)
\newtcolorbox{empirebox}{
	colback=violetlight,
	colframe=violetempire,
	fonttitle=\bfseries,
	title={},
	breakable,
	enhanced,
	boxrule=1pt,
	arc=3pt,
	left=10pt,
	right=10pt,
	top=10pt,
	bottom=10pt
}

% کادر نکته (خاکستری)
\newtcolorbox{noktebox}{
	colback=grislight,
	colframe=gris,
	fonttitle=\bfseries,
	title={},
	breakable,
	enhanced,
	boxrule=1pt,
	arc=3pt,
	left=10pt,
	right=10pt,
	top=10pt,
	bottom=10pt
}

%──────────────────────────────────────────────────────────────────────────────
% دستور برای متن لاتین
%──────────────────────────────────────────────────────────────────────────────
%\newcommand{\lr}[1]{\textdir TLT #1}

%──────────────────────────────────────────────────────────────────────────────
% فونت و زبان فارسی (آخرین پکیج)
%──────────────────────────────────────────────────────────────────────────────
\usepackage{xepersian}
\settextfont{Vazirmatn}
\setdigitfont{Vazirmatn}

%══════════════════════════════════════════════════════════════════════════════
% شروع سند
%══════════════════════════════════════════════════════════════════════════════

\begin{document}
	
	\chapter{جمهوری پنجم (۱۹۵۸-۲۰۲۴)}
	\label{chap:fifth-republic}
	
	\begin{kholasebox}
		\textbf{خلاصه فصل:}
		
		جمهوری پنجم، که از بحران الجزایر زاده شد، طولانی‌ترین و باثبات‌ترین رژیم فرانسه پس از انقلاب است. این نظام که بر اساس ایده‌های دوگل طراحی شد، قدرت اجرایی قوی را با دموکراسی پارلمانی ترکیب کرده و توانسته از چالش‌های متعددی—از جنگ الجزایر تا مه ۱۹۶۸، از بحران‌های اقتصادی تا چرخش قدرت میان چپ و راست—سربلند بیرون آید.
		
		\textbf{دوره‌بندی:}
		\begin{itemize}[nosep,rightmargin=0pt]
			\item \textbf{دوره دوگل (۱۹۵۸-۱۹۶۹):} تأسیس، الجزایر، استقلال ملی، مه ۶۸
			\item \textbf{تداوم گلیسم (۱۹۶۹-۱۹۸۱):} پمپیدو، ژیسکاردستن، مدرنیزاسیون
			\item \textbf{چپ در قدرت (۱۹۸۱-۱۹۹۵):} میتران، همزیستی، پایان جنگ سرد
			\item \textbf{دوره شیراک (۱۹۹۵-۲۰۰۷):} همزیستی، «نه» به عراق
			\item \textbf{دوره معاصر (۲۰۰۷-۲۰۲۴):} سارکوزی، اولاند، ماکرون، چالش‌های جدید
		\end{itemize}
		
		\textbf{مفاهیم کلیدی:} ریاست‌جمهوری قوی، همزیستی، گلیسم، چرخش، دور دوم، انتخابات مستقیم، حوزه محفوظ.
	\end{kholasebox}
	
	%──────────────────────────────────────────────────────────────────────────────
	\section{تأسیس جمهوری پنجم}
	%──────────────────────────────────────────────────────────────────────────────
	
	\subsection{قانون اساسی ۱۹۵۸}
	
	قانون اساسی جدید در تابستان ۱۹۵۸ توسط تیمی به رهبری میشل دوبره (با نظارت دوگل) تدوین شد و در همه‌پرسی ۲۸ سپتامبر ۱۹۵۸ با ۷۹٪ آرا تصویب شد.
	
	\begin{tikzpicture}[
		every node/.style={font=\small},
		box/.style={rectangle, rounded corners, draw=bleurepublique, fill=bleulight,
			minimum width=3.5cm, minimum height=1.5cm, align=center}
		]
		% Title
		\node[font=\bfseries\large] at (7,9) {ساختار نهادی جمهوری پنجم};
		
		% President at top
		\node[box, fill=bleumid, text=white, minimum width=5cm, minimum height=1.8cm] (pres) at (7,7) 
		{\textbf{رئیس‌جمهور}\\انتخاب مستقیم (از ۱۹۶۲)\\۷ سال (۵ سال از ۲۰۰۰)};
		
		% Government
		\node[box] (pm) at (3,4) {\textbf{نخست‌وزیر}\\منصوب توسط رئیس‌جمهور};
		
		\node[box] (gov) at (7,4) {\textbf{دولت}\\وزرا منصوب\\توسط رئیس‌جمهور};
		
		% Parliament
		\node[box] (an) at (3,1) {\textbf{مجلس ملی}\\۵۷۷ نماینده\\انتخاب مستقیم};
		
		\node[box] (senat) at (11,1) {\textbf{سنا}\\۳۴۸ سناتور\\انتخاب غیرمستقیم};
		
		% Constitutional Council
		\node[box, fill=vertlight, draw=vertnapoleon] (cc) at (11,4) {\textbf{شورای قانون اساسی}\\۹ عضو\\کنترل قوانین};
		
		% Arrows
		\draw[->, thick, bleurepublique] (pres) -- (pm) node[midway, left] {\footnotesize انتصاب};
		\draw[->, thick, bleurepublique] (pres) -- (gov);
		\draw[->, thick, bleurepublique] (pm) -- (gov) node[midway, below] {\footnotesize پیشنهاد};
		\draw[<->, thick, rougerevolution] (gov) -- (an) node[midway, left] {\footnotesize مسئولیت};
		\draw[->, thick, dashed] (an) -- (senat) node[midway, below] {\footnotesize قانون‌گذاری};
		\draw[->, thick, vertnapoleon] (cc) -- (7,2.5) node[midway, right] {\footnotesize کنترل};
		
		% Key power
		\draw[->, ultra thick, rougerevolution] (pres) -- (11,7) -- (11,5.5) node[midway, right] {\footnotesize انحلال مجلس};
		
	\end{tikzpicture}
	
	\begin{table}[htbp]
		\centering
		\caption{اختیارات رئیس‌جمهور در جمهوری پنجم}
		\label{tab:presidential-powers}
		\begin{tabular}{|r|p{7cm}|c|}
			\hline
			\rowcolor{bleumid}
			\textcolor{white}{\textbf{اختیار}} & \textcolor{white}{\textbf{توضیح}} & \textcolor{white}{\textbf{ماده}} \\
			\hline
			انتصاب نخست‌وزیر & بدون نیاز به تأیید مجلس & ۸ \\
			\hline
			\rowcolor{bleulight}
			انحلال مجلس ملی & یک‌بار در سال، بدون شرط خاص & ۱۲ \\
			\hline
			همه‌پرسی & ارجاع مستقیم به ملت & ۱۱ \\
			\hline
			\rowcolor{bleulight}
			اختیارات اضطراری & در شرایط بحرانی، قدرت تام & ۱۶ \\
			\hline
			فرماندهی کل قوا & رئیس ارتش‌ها & ۱۵ \\
			\hline
			\rowcolor{bleulight}
			سیاست خارجی & «حوزه محفوظ» رئیس‌جمهور & سنت \\
			\hline
			عفو & عفو محکومان & ۱۷ \\
			\hline
		\end{tabular}
	\end{table}
	
	\begin{noktebox}
		\textbf{نوآوری اصلی: «پارلمانتاریسم عقلانی‌شده»}
		
		قانون اساسی ۱۹۵۸ قدرت پارلمان را محدود کرد:
		\begin{itemize}[nosep]
			\item ماده ۳۴: قانون‌گذاری فقط در حوزه‌های مشخص
			\item ماده ۳۷: بقیه حوزه‌ها در اختیار دولت (مقررات)
			\item ماده ۴۹-۳: تصویب قانون بدون رأی‌گیری (مسئله اعتماد)
			\item دستور کار مجلس: در اختیار دولت
		\end{itemize}
	\end{noktebox}
	
	\subsection{انتخاب مستقیم رئیس‌جمهور (۱۹۶۲)}
	
	در ۱۹۶۲، دوگل با همه‌پرسی (و نه از طریق مجلس) قانون اساسی را اصلاح کرد: انتخاب مستقیم رئیس‌جمهور توسط مردم. این اصلاح، که به شدت مورد مخالفت احزاب قرار گرفت، ماهیت رژیم را تغییر داد.
	
	\begin{tikzpicture}[
		every node/.style={font=\small},
		box/.style={rectangle, rounded corners, minimum width=4cm, minimum height=1.5cm, align=center}
		]
		% Title
		\node[font=\bfseries\large] at (7,7) {تحول مشروعیت ریاست‌جمهوری};
		
		% Before
		\node[box, draw=gris, fill=grisclair] (before) at (3,4.5) 
		{\textbf{قبل از ۱۹۶۲}\\انتخاب توسط\\هیئت بزرگ انتخاباتی};
		
		% After
		\node[box, draw=bleurepublique, fill=bleumid, text=white] (after) at (11,4.5) 
		{\textbf{پس از ۱۹۶۲}\\انتخاب مستقیم\\توسط مردم};
		
		% Arrow
		\draw[->, ultra thick, rougerevolution] (before) -- (after) 
		node[midway, above] {همه‌پرسی اکتبر ۱۹۶۲};
		
		% Result
		\node[box, draw=vertnapoleon, fill=vertlight] at (7,1.5) 
		{\textbf{نتیجه:} رئیس‌جمهور = تنها نماینده کل ملت\\تقویت جایگاه در برابر مجلس و احزاب};
		
		% Vote result
		\node[font=\footnotesize, align=center] at (7,5.5) 
		{نتیجه همه‌پرسی: ۶۲٪ آری};
		
	\end{tikzpicture}
	
	\begin{naghlbox}
		«فرانسه رژیم احزاب را نمی‌خواهد و باز نخواهد گشت. رئیس‌جمهور باید از ملت باشد، نه از پارلمان.»
		
		\hfill --- \textit{شارل دوگل، ۱۹۶۲}
	\end{naghlbox}
	
	%──────────────────────────────────────────────────────────────────────────────
	\section{دوره دوگل (۱۹۵۸-۱۹۶۹)}
	%──────────────────────────────────────────────────────────────────────────────
	
	\subsection{پایان جنگ الجزایر}
	
	دوگل به قدرت آمد با وعده «الجزایر فرانسوی»، اما تدریجاً به استقلال رسید:
	
	\begin{tikzpicture}[
		every node/.style={font=\small},
		event/.style={rectangle, rounded corners, draw=rougerevolution, fill=rougelight,
			minimum width=3.2cm, minimum height=1.3cm, align=center}
		]
		% Title
		\node[font=\bfseries\large] at (7,8) {مسیر استقلال الجزایر (۱۹۵۸-۱۹۶۲)};
		
		% Timeline
		\draw[very thick, rougerevolution] (0,5.5) -- (14,5.5);
		
		% Events
		\node[event] (e1) at (1.5,3.5) {\textbf{ژوئن ۱۹۵۸}\\«فهمیدمتان»\\ابهام‌گویی};
		
		\node[event] (e2) at (4.5,3.5) {\textbf{سپتامبر ۱۹۵۹}\\«خودمختاری»\\چرخش};
		
		\node[event] (e3) at (7.5,3.5) {\textbf{ژانویه ۱۹۶۰}\\«هفته سنگرها»\\شورش پیه‌نوارها};
		
		\node[event] (e4) at (10.5,3.5) {\textbf{آوریل ۱۹۶۱}\\«کودتای ژنرال‌ها»\\شکست نظامیان};
		
		\node[event, fill=bleumid, draw=bleurepublique, text=white] (e5) at (13.5,3.5) {\textbf{مارس ۱۹۶۲}\\قراردادهای اِویان\\استقلال};
		
		% Markers
		\foreach \x in {1.5, 4.5, 7.5, 10.5, 13.5} {
			\fill[rougerevolution] (\x,5.5) circle (0.1);
			\draw[->, thick] (\x,5.5) -- (\x,4.3);
		}
		
		% OAS note
		\node[rectangle, draw=gris, fill=grisclair, rounded corners, align=center] at (7,1) {
			\textbf{OAS} (سازمان ارتش مخفی): ترور علیه استقلال\\
			سوءقصدهای متعدد به دوگل — پتی-کلامار (اوت ۱۹۶۲)
		};
		
	\end{tikzpicture}
	
	\begin{enghelabbox}
		\textbf{قراردادهای اِویان (۱۸ مارس ۱۹۶۲)}
		
		\begin{itemize}[nosep]
			\item \textbf{آتش‌بس:} ۱۹ مارس ۱۹۶۲
			\item \textbf{همه‌پرسی:} ۱ ژوئیه ۱۹۶۲ — ۹۹.۷٪ آری به استقلال
			\item \textbf{استقلال رسمی:} ۵ ژوئیه ۱۹۶۲
			\item \textbf{خروج پیه‌نوارها:} ۹۰۰,۰۰۰ نفر به فرانسه
			\item \textbf{سرنوشت حرکی‌ها:} مسلمانان طرفدار فرانسه، بسیاری قتل‌عام شدند
		\end{itemize}
		
		\textbf{آمار جنگ:}
		\begin{itemize}[nosep]
			\item کشته‌های الجزایری: ۳۰۰,۰۰۰ تا ۱,۰۰۰,۰۰۰ (برآوردهای متفاوت)
			\item کشته‌های فرانسوی: ۲۵,۰۰۰ نظامی + ۵,۰۰۰ غیرنظامی
			\item آوارگان: میلیون‌ها
		\end{itemize}
	\end{enghelabbox}
	
	\begin{naghlbox}
		«استعمارزدایی منافع ماست، و در نتیجه سیاست ماست.»
		
		\hfill --- \textit{شارل دوگل}
	\end{naghlbox}
	
	\subsection{سیاست خارجی: استقلال ملی}
	
	دوگل فرانسه را به عنوان قدرتی مستقل میان دو ابرقدرت معرفی کرد:
	
	\begin{table}[htbp]
		\centering
		\caption{سیاست خارجی دوگل}
		\label{tab:de-gaulle-foreign}
		\begin{tabular}{|c|p{8cm}|}
			\hline
			\rowcolor{bleumid}
			\textcolor{white}{\textbf{سال}} & \textcolor{white}{\textbf{اقدام}} \\
			\hline
			۱۹۶۰ & اولین آزمایش اتمی فرانسه (رگان، صحرا) \\
			\hline
			\rowcolor{bleulight}
			۱۹۶۳ & وتوی عضویت انگلستان در جامعه اروپایی \\
			\hline
			۱۹۶۳ & معاهده دوستی فرانسه-آلمان (الیزه) \\
			\hline
			\rowcolor{bleulight}
			۱۹۶۴ & به رسمیت شناختن چین کمونیست \\
			\hline
			۱۹۶۶ & خروج از فرماندهی نظامی ناتو \\
			\hline
			\rowcolor{bleulight}
			۱۹۶۷ & «زنده باد کبک آزاد!» (مونترال) \\
			\hline
			۱۹۶۷ & انتقاد از اسرائیل پس از جنگ شش‌روزه \\
			\hline
		\end{tabular}
	\end{table}
	
	\begin{noktebox}
		\textbf{دکترین دوگل در سیاست خارجی:}
		
		\begin{itemize}[nosep]
			\item \textbf{استقلال ملی:} نه وابستگی به آمریکا، نه به شوروی
			\item \textbf{بازدارندگی اتمی:} نیروی ضربتی مستقل
			\item \textbf{«اروپای میهن‌ها»:} نه فدرالیسم، همکاری دولت‌های مستقل
			\item \textbf{«از آتلانتیک تا اورال»:} اروپا بزرگ‌تر از غرب
			\item \textbf{جهان سوم:} روابط با کشورهای غیرمتعهد
		\end{itemize}
	\end{noktebox}
	
	\subsection{مدرنیزاسیون اقتصادی}
	
	دهه ۱۹۶۰ دوره رشد سریع اقتصادی بود—«سی سال باشکوه»:
	
	\begin{table}[htbp]
		\centering
		\caption{شاخص‌های اقتصادی دوره دوگل}
		\label{tab:de-gaulle-economy}
		\begin{tabular}{|r|c|c|c|}
			\hline
			\rowcolor{vertmid}
			\textcolor{white}{\textbf{شاخص}} & \textcolor{white}{\textbf{۱۹۵۸}} & \textcolor{white}{\textbf{۱۹۶۹}} & \textcolor{white}{\textbf{تغییر}} \\
			\hline
			رشد سالانه & — & — & میانگین ۵.۵٪ \\
			\hline
			\rowcolor{vertlight}
			خودرو (میلیون) & ۵ & ۱۲.۵ & +۱۵۰٪ \\
			\hline
			تلویزیون (میلیون خانوار) & ۱ & ۱۰ & +۹۰۰٪ \\
			\hline
			\rowcolor{vertlight}
			یخچال (درصد خانوارها) & ۲۵٪ & ۸۰٪ & +۲۲۰٪ \\
			\hline
			دستمزد واقعی & پایه & +۴۵٪ & — \\
			\hline
		\end{tabular}
	\end{table}
	
	\subsection{مه ۱۹۶۸}
	
	بزرگ‌ترین بحران دوره دوگل، جنبش مه ۱۹۶۸ بود:
	
	\begin{tikzpicture}[
		every node/.style={font=\small},
		phase/.style={rectangle, rounded corners, draw=rougerevolution, fill=rougelight,
			minimum width=4cm, minimum height=1.8cm, align=center}
		]
		% Title
		\node[font=\bfseries\large] at (7,8.5) {رویدادهای مه ۱۹۶۸};
		
		% Timeline
		\draw[very thick, rougerevolution] (0,6) -- (14,6);
		
		% Phases
		\node[phase] (p1) at (2,3.5) {\textbf{مارس-آوریل}\\جنبش دانشجویی\\نانتر، کوهن-بندیت\\«ممنوع است ممنوع کردن»};
		
		\node[phase] (p2) at (7,3.5) {\textbf{اوایل مه}\\سرکوب، سنگربندی\\«شب سنگرها»\\۱۰-۱۱ مه};
		
		\node[phase] (p3) at (12,3.5) {\textbf{۱۳-۳۰ مه}\\اعتصاب عمومی\\۱۰ میلیون کارگر\\فلج کشور};
		
		% Key dates
		\node[font=\footnotesize] at (2,6.5) {۲۲ مارس};
		\node[font=\footnotesize] at (7,6.5) {۱۰-۱۱ مه};
		\node[font=\footnotesize] at (12,6.5) {۱۳-۳۰ مه};
		
		\fill[rougerevolution] (2,6) circle (0.1);
		\fill[rougerevolution] (7,6) circle (0.1);
		\fill[rougerevolution] (12,6) circle (0.1);
		
		% Resolution
		\node[rectangle, draw=bleurepublique, fill=bleulight, rounded corners,
		minimum width=10cm, minimum height=1.5cm, align=center] at (7,0.5) {
			\textbf{۳۰ مه:} سخنرانی دوگل — انحلال مجلس — تظاهرات گلیست‌ها\\
			\textbf{ژوئن:} پیروزی قاطع گلیست‌ها در انتخابات
		};
		
	\end{tikzpicture}
	
	\begin{enghelabbox}
		\textbf{مه ۱۹۶۸: ابعاد و خواسته‌ها}
		
		\textbf{جنبش دانشجویی:}
		\begin{itemize}[nosep]
			\item نقد جامعه مصرفی و اقتدارگرایی
			\item آزادی جنسی و فردی
			\item شعارها: «زیر سنگ‌فرش، ساحل است»، «واقع‌بین باشید، محال را بخواهید»
		\end{itemize}
		
		\textbf{جنبش کارگری:}
		\begin{itemize}[nosep]
			\item بزرگ‌ترین اعتصاب تاریخ فرانسه: ۱۰ میلیون نفر
			\item خواسته‌های سنتی: دستمزد، ساعت کار، شرایط
			\item قراردادهای گرِنِل: افزایش ۳۵٪ حداقل دستمزد
		\end{itemize}
		
		\textbf{نتیجه فوری:}
		\begin{itemize}[nosep]
			\item دوگل پیروز شد (انتخابات ژوئن)
			\item اما جامعه تغییر کرد (آزادسازی اخلاقی، فرهنگی)
		\end{itemize}
	\end{enghelabbox}
	
	\begin{naghlbox}
		«اصلاحات آری، تخت‌خواب‌گندگی نه!»
		
		\hfill --- \textit{شارل دوگل، مه ۱۹۶۸}
	\end{naghlbox}
	
	\subsection{رفتن دوگل (۱۹۶۹)}
	
	در آوریل ۱۹۶۹، دوگل همه‌پرسی درباره اصلاحات منطقه‌ای و سنا برگزار کرد و اعلام کرد در صورت رأی منفی استعفا خواهد داد. نتیجه: ۵۲٪ نه.
	
	\begin{naghlbox}
		«من وظایف خود را به عنوان رئیس‌جمهور متوقف می‌کنم. این تصمیم از نیمه‌شب امروز مؤثر است.»
		
		\hfill --- \textit{شارل دوگل، ۲۸ آوریل ۱۹۶۹}
	\end{naghlbox}
	
	دوگل در ۹ نوامبر ۱۹۷۰ درگذشت.
	
	%──────────────────────────────────────────────────────────────────────────────
	\section{تداوم گلیسم (۱۹۶۹-۱۹۸۱)}
	%──────────────────────────────────────────────────────────────────────────────
	
	\subsection{ژرژ پمپیدو (۱۹۶۹-۱۹۷۴)}
	
	\begin{table}[htbp]
		\centering
		\caption{ریاست‌جمهوری پمپیدو}
		\label{tab:pompidou}
		\begin{tabular}{|r|p{9cm}|}
			\hline
			\rowcolor{bleumid}
			\textcolor{white}{\textbf{جنبه}} & \textcolor{white}{\textbf{توضیح}} \\
			\hline
			\textbf{سابقه} & نخست‌وزیر دوگل (۱۹۶۲-۱۹۶۸)، بانکدار روچیلد \\
			\hline
			\rowcolor{bleulight}
			\textbf{شعار} & «تداوم و گشایش» \\
			\hline
			\textbf{اقتصاد} & صنعتی‌سازی، پروژه‌های بزرگ، هواپیمایی \\
			\hline
			\rowcolor{bleulight}
			\textbf{اروپا} & پذیرش انگلستان در جامعه اروپایی (۱۹۷۳) \\
			\hline
			\textbf{داخلی} & نوسازی پاریس، مرکز پمپیدو \\
			\hline
			\rowcolor{bleulight}
			\textbf{پایان} & مرگ در قدرت (آوریل ۱۹۷۴)—سرطان \\
			\hline
		\end{tabular}
	\end{table}
	
	\subsection{والری ژیسکاردستن (۱۹۷۴-۱۹۸۱)}
	
	ژیسکار اولین رئیس‌جمهور غیرگلیست جمهوری پنجم بود:
	
	\begin{tikzpicture}[
		every node/.style={font=\small},
		reform/.style={rectangle, rounded corners, draw=bleurepublique, fill=bleulight,
			minimum width=3.5cm, minimum height=1.2cm, align=center}
		]
		% Title
		\node[font=\bfseries\large] at (7,7.5) {اصلاحات دوره ژیسکاردستن};
		
		% Reforms
		\node[reform] (r1) at (2,5.5) {کاهش سن رأی\\به ۱۸ سال\\(۱۹۷۴)};
		
		\node[reform] (r2) at (6,5.5) {قانون سقط جنین\\(قانون وِی)\\(۱۹۷۵)};
		
		\node[reform] (r3) at (10,5.5) {طلاق توافقی\\(۱۹۷۵)};
		
		\node[reform] (r4) at (2,3) {ارجاع به شورای\\قانون اساسی توسط\\۶۰ نماینده (۱۹۷۴)};
		
		\node[reform] (r5) at (6,3) {منشی دولت\\برای وضعیت زنان\\(۱۹۷۴)};
		
		\node[reform] (r6) at (10,3) {شورای اروپایی\\انتخابات مستقیم\\(۱۹۷۹)};
		
		% Context
		\node[rectangle, draw=rougerevolution, fill=rougelight, rounded corners,
		minimum width=10cm, minimum height=1.2cm, align=center] at (6,0.5) {
			\textbf{زمینه:} بحران نفتی ۱۹۷۳، پایان «سی سال باشکوه»\\
			رکود، بیکاری، تورم — شعار «تغییر بدون ریسک» جواب نداد
		};
		
	\end{tikzpicture}
	
	\begin{naghlbox}
		«فرانسه می‌خواهد از مرکز حکومت شود.»
		
		\hfill --- \textit{والری ژیسکاردستن}
	\end{naghlbox}
	
	\begin{noktebox}
		\textbf{قانون وِی (۱۷ ژانویه ۱۹۷۵)}
		
		قانون‌گذاری سقط جنین، که توسط سیمون وِی (وزیر بهداشت) ارائه شد، با مقاومت شدید راست مواجه شد. وِی، خود بازمانده آشویتس، مجبور شد توهین‌های دردناکی را تحمل کند. قانون فقط با آرای چپ تصویب شد—اکثریت راست مخالف بود.
	\end{noktebox}
	
	%──────────────────────────────────────────────────────────────────────────────
	\section{چپ در قدرت (۱۹۸۱-۱۹۹۵)}
	%──────────────────────────────────────────────────────────────────────────────
	
	\subsection{پیروزی تاریخی میتران (۱۹۸۱)}
	
	در ۱۰ مه ۱۹۸۱، فرانسوا میتران اولین رئیس‌جمهور سوسیالیست جمهوری پنجم شد:
	
	\begin{tikzpicture}[
		every node/.style={font=\small}
		]
		% Title
		\node[font=\bfseries\large] at (7,7) {انتخابات ریاست‌جمهوری ۱۹۸۱ — دور دوم};
		
		% Results
		\fill[rougelight] (2,2) rectangle (6,5.5);
		\fill[bleulight] (8,2) rectangle (12,4.5);
		
		\draw[rougerevolution, thick] (2,2) rectangle (6,5.5);
		\draw[bleurepublique, thick] (8,2) rectangle (12,4.5);
		
		\node[align=center] at (4,4.5) {\textbf{فرانسوا میتران}\\سوسیالیست};
		\node[font=\large\bfseries, rougerevolution] at (4,3) {۵۱.۸٪};
		
		\node[align=center] at (10,4) {\textbf{والری ژیسکاردستن}\\راست};
		\node[font=\large\bfseries, bleurepublique] at (10,3) {۴۸.۲٪};
		
		% Significance
		\node[rectangle, draw=vertnapoleon, fill=vertlight, rounded corners,
		minimum width=10cm, minimum height=1.2cm, align=center] at (7,0.5) {
			\textbf{اهمیت:} اولین «چرخش» در جمهوری پنجم\\
			اثبات اینکه قانون اساسی ۱۹۵۸ فقط برای راست نیست
		};
		
	\end{tikzpicture}
	
	\begin{table}[htbp]
		\centering
		\caption{فرانسوا میتران (۱۹۱۶-۱۹۹۶)}
		\label{tab:mitterrand-bio}
		\begin{tabular}{|r|p{10cm}|}
			\hline
			\rowcolor{rougemid}
			\textcolor{white}{\textbf{ویژگی}} & \textcolor{white}{\textbf{توضیح}} \\
			\hline
			\textbf{سابقه} & وزیر در جمهوری چهارم، ضد دوگل \\
			\hline
			\rowcolor{rougelight}
			\textbf{تلاش‌های قبلی} & شکست در ۱۹۶۵ و ۱۹۷۴ \\
			\hline
			\textbf{استراتژی} & اتحاد چپ (سوسیالیست‌ها + کمونیست‌ها) \\
			\hline
			\rowcolor{rougelight}
			\textbf{کتاب ضد دوگل} & «کودتای دائم» (۱۹۶۴) \\
			\hline
			\textbf{مدت} & دو دوره — طولانی‌ترین ریاست‌جمهوری (۱۴ سال) \\
			\hline
		\end{tabular}
	\end{table}
	
	\subsection{اصلاحات ۱۹۸۱-۱۹۸۳}
	
	\begin{olgoobox}
		\textbf{برنامه ۱۱۰ پیشنهاد میتران}
		
		\textbf{اصلاحات اجتماعی:}
		\begin{itemize}[nosep]
			\item هفته ۳۹ ساعته (از ۴۰)
			\item هفته پنجم مرخصی با حقوق
			\item بازنشستگی در ۶۰ سالگی
			\item افزایش حداقل دستمزد (۱۰٪)
		\end{itemize}
		
		\textbf{اصلاحات سیاسی:}
		\begin{itemize}[nosep]
			\item لغو مجازات اعدام (سپتامبر ۱۹۸۱—رابرت بادنتر)
			\item آزادی رادیوهای محلی
			\item تمرکززدایی (قانون دِفِر ۱۹۸۲)
		\end{itemize}
		
		\textbf{ملی‌سازی‌ها:}
		\begin{itemize}[nosep]
			\item ۵ گروه صنعتی بزرگ
			\item ۳۶ بانک
			\item ۲ شرکت مالی
		\end{itemize}
	\end{olgoobox}
	
	\subsection{چرخش ۱۹۸۳: «ریاضت اقتصادی»}
	
	سیاست‌های انبساطی اولیه با واقعیت اقتصادی برخورد کرد:
	
	\begin{itemize}
		\item \textbf{تورم:} بیش از ۱۰٪
		\item \textbf{کسری تجاری:} رشد سریع
		\item \textbf{کاهش ارزش فرانک:} سه بار در دو سال
		\item \textbf{تصمیم مارس ۱۹۸۳:} ماندن در نظام پولی اروپا = ریاضت
	\end{itemize}
	
	\begin{naghlbox}
		«ما باید بین دو اروپا انتخاب کنیم: اروپای پیشرفت یا اروپای عقب‌ماندگی. فرانسه نمی‌تواند تنها به سوسیالیسم برود.»
		
		\hfill --- \textit{ژاک دولور، وزیر دارایی}
	\end{naghlbox}
	
	\subsection{اولین همزیستی (۱۹۸۶-۱۹۸۸)}
	
	در انتخابات ۱۹۸۶، راست پیروز شد. میتران ژاک شیراک (گلیست) را به نخست‌وزیری منصوب کرد. این اولین «همزیستی» بود:
	
	\begin{tikzpicture}[
		every node/.style={font=\small},
		box/.style={rectangle, rounded corners, minimum width=4cm, minimum height=2cm, align=center}
		]
		% Title
		\node[font=\bfseries\large] at (7,7) {همزیستی: تقسیم قدرت};
		
		% President
		\node[box, draw=rougerevolution, fill=rougelight] (pres) at (3,4) 
		{\textbf{رئیس‌جمهور}\\میتران (چپ)\\سیاست خارجی، دفاع\\«حوزه محفوظ»};
		
		% PM
		\node[box, draw=bleurepublique, fill=bleulight] (pm) at (11,4) 
		{\textbf{نخست‌وزیر}\\شیراک (راست)\\سیاست داخلی، اقتصاد\\اکثریت مجلس};
		
		% Tension
		\draw[<->, ultra thick, rougerevolution] (pres) -- (pm) 
		node[midway, above] {رقابت و تنش};
		
		% Precedent
		\node[rectangle, draw=gris, fill=grisclair, rounded corners, align=center] at (7,1) {
			\textbf{نتیجه:} قانون اساسی انعطاف‌پذیر بود\\
			همزیستی ممکن شد بدون بحران نظام
		};
		
	\end{tikzpicture}
	
	\subsection{دور دوم میتران (۱۹۸۸-۱۹۹۵)}
	
	میتران در ۱۹۸۸ مجدداً انتخاب شد (با شعار «فرانسه متحد»). دور دوم با چالش‌های جدید:
	
	\begin{itemize}
		\item \textbf{سقوط دیوار برلین (۱۹۸۹):} پایان جنگ سرد
		\item \textbf{وحدت آلمان (۱۹۹۰):} نگرانی‌های فرانسه
		\item \textbf{معاهده ماستریخت (۱۹۹۲):} همه‌پرسی با ۵۱٪ آری
		\item \textbf{دومین همزیستی (۱۹۹۳-۱۹۹۵):} با بالادور
		\item \textbf{بیماری:} سرطان پنهان از ۱۹۸۱
	\end{itemize}
	
	میتران در ژانویه ۱۹۹۶ درگذشت.
	
	%──────────────────────────────────────────────────────────────────────────────
	\section{دوره شیراک (۱۹۹۵-۲۰۰۷)}
	%──────────────────────────────────────────────────────────────────────────────
	
	\subsection{ریاست‌جمهوری پرفرازونشیب}
	
	ژاک شیراک پس از دو شکست (۱۹۸۱، ۱۹۸۸) سرانجام در ۱۹۹۵ رئیس‌جمهور شد:
	
	\begin{tikzpicture}[
		every node/.style={font=\small},
		event/.style={rectangle, rounded corners, draw=bleurepublique, fill=bleulight,
			minimum width=3.5cm, minimum height=1.3cm, align=center}
		]
		% Title
		\node[font=\bfseries\large] at (7,8) {رویدادهای کلیدی دوره شیراک};
		
		% Timeline
		\draw[very thick, bleurepublique] (0,5.5) -- (14,5.5);
		
		% Events
		\node[event] (e1) at (2,3.5) {\textbf{۱۹۹۵}\\سخنرانی وِل دیو\\اعتراف به\\مسئولیت ویشی};
		
		\node[event, fill=rougelight, draw=rougerevolution] (e2) at (5.5,3.5) {\textbf{۱۹۹۵}\\اعتصابات بزرگ\\ضد اصلاح\\بازنشستگی};
		
		\node[event, fill=rougelight, draw=rougerevolution] (e3) at (9,3.5) {\textbf{۱۹۹۷}\\انحلال مجلس\\شکست راست\\سومین همزیستی};
		
		\node[event, fill=vertlight, draw=vertnapoleon] (e4) at (12.5,3.5) {\textbf{۲۰۰۲}\\«زلزله» ۲۱ آوریل\\لوپن در دور دوم\\۸۲٪ برای شیراک};
		
		% Markers
		\foreach \x in {2, 5.5, 9, 12.5} {
			\fill[bleurepublique] (\x,5.5) circle (0.1);
			\draw[->, thick] (\x,5.5) -- (\x,4.3);
		}
		
	\end{tikzpicture}
	
	\subsection{سومین همزیستی (۱۹۹۷-۲۰۰۲)}
	
	شیراک در ۱۹۹۷ مجلس را منحل کرد به امید تقویت اکثریت—اما چپ پیروز شد و لیونل ژوسپن (سوسیالیست) نخست‌وزیر شد. این طولانی‌ترین همزیستی بود:
	
	\begin{olgoobox}
		\textbf{دستاوردهای دولت ژوسپن (۱۹۹۷-۲۰۰۲)}
		
		\begin{itemize}[nosep]
			\item \textbf{هفته ۳۵ ساعته:} قانون اوبری (۱۹۹۸-۲۰۰۰)
			\item \textbf{پکس:} اتحاد مدنی همجنس‌گرایان (۱۹۹۹)
			\item \textbf{برابری:} سهمیه زنان در انتخابات (۲۰۰۰)
			\item \textbf{کاهش دوره ریاست‌جمهوری:} از ۷ به ۵ سال (۲۰۰۰)
			\item \textbf{پایان سربازی اجباری}
			\item \textbf{رشد اقتصادی و کاهش بیکاری}
		\end{itemize}
	\end{olgoobox}
	
	\subsection{«زلزله» ۲۱ آوریل ۲۰۰۲}
	
	در دور اول انتخابات ۲۰۰۲، ژان-ماری لوپن (جبهه ملی، راست افراطی) از ژوسپن جلو افتاد:
	
	\begin{table}[htbp]
		\centering
		\caption{نتایج دور اول انتخابات ۲۰۰۲}
		\label{tab:2002-first-round}
		\begin{tabular}{|c|r|c|}
			\hline
			\rowcolor{gris}
			\textcolor{white}{\textbf{رتبه}} & \textcolor{white}{\textbf{نامزد}} & \textcolor{white}{\textbf{درصد}} \\
			\hline
			۱ & ژاک شیراک (گلیست) & ۱۹.۹٪ \\
			\hline
			\rowcolor{grisclair}
			\textbf{۲} & \textbf{ژان-ماری لوپن (راست افراطی)} & \textbf{۱۶.۹٪} \\
			\hline
			۳ & لیونل ژوسپن (سوسیالیست) & ۱۶.۲٪ \\
			\hline
		\end{tabular}
	\end{table}
	
	\begin{enghelabbox}
		\textbf{واکنش به ۲۱ آوریل}
		
		\begin{itemize}[nosep]
			\item تظاهرات گسترده علیه لوپن (۱ مه: یک میلیون نفر)
			\item فراخوان همه احزاب برای رأی به شیراک
			\item شعار: «رأی بدهید به کلاهبردار، نه به فاشیست»
			\item نتیجه دور دوم: شیراک ۸۲٪، لوپن ۱۸٪
		\end{itemize}
		
		\textbf{درس:} پراکندگی چپ می‌تواند به فاجعه بینجامد
	\end{enghelabbox}
	
	\subsection{سیاست خارجی: «نه» به جنگ عراق}
	
	مهم‌ترین میراث دور دوم شیراک، مخالفت با جنگ عراق بود:
	
	\begin{naghlbox}
		«جنگ همیشه شکست بشریت است... فرانسه مخالف اقدام نظامی است که توجیهی ندارد.»
		
		\hfill --- \textit{دومینیک دو ویلپن، شورای امنیت، ۱۴ فوریه ۲۰۰۳}
	\end{naghlbox}
	
	%──────────────────────────────────────────────────────────────────────────────
	\section{دوره معاصر (۲۰۰۷-۲۰۲۴)}
	%──────────────────────────────────────────────────────────────────────────────
	
	\subsection{نیکلا سارکوزی (۲۰۰۷-۲۰۱۲)}
	
	\begin{table}[htbp]
		\centering
		\caption{ریاست‌جمهوری سارکوزی}
		\label{tab:sarkozy}
		\begin{tabular}{|r|p{9cm}|}
			\hline
			\rowcolor{bleumid}
			\textcolor{white}{\textbf{جنبه}} & \textcolor{white}{\textbf{توضیح}} \\
			\hline
			\textbf{سبک} & «ابَررئیس‌جمهور»—حضور همه‌جایی، فعال \\
			\hline
			\rowcolor{bleulight}
			\textbf{اصلاحات} & بازنشستگی (از ۶۰ به ۶۲)، دانشگاه‌ها، نقشه قضایی \\
			\hline
			\textbf{بحران ۲۰۰۸} & بحران مالی جهانی، سیاست‌های مقابله \\
			\hline
			\rowcolor{bleulight}
			\textbf{اروپا} & معاهده لیسبون (۲۰۰۷)—جایگزین قانون اساسی رد شده \\
			\hline
			\textbf{لیبی} & مداخله نظامی ۲۰۱۱ \\
			\hline
			\rowcolor{bleulight}
			\textbf{جنجال‌ها} & سبک زندگی، ماجرای بتانکور \\
			\hline
%══════════════════════════════════════════════════════════════════════════════
% ادامه فصل ۸
%══════════════════════════════════════════════════════════════════════════════

\textbf{شکست ۲۰۱۲} & اولین رئیس‌جمهور شکست‌خورده در انتخاب مجدد (پس از ژیسکار) \\
\hline
\end{tabular}
\end{table}

\subsection{فرانسوا اولاند (۲۰۱۲-۲۰۱۷)}

\begin{table}[htbp]
\centering
\caption{ریاست‌جمهوری اولاند}
\label{tab:hollande}
\begin{tabular}{|r|p{9cm}|}
\hline
\rowcolor{rougemid}
\textcolor{white}{\textbf{جنبه}} & \textcolor{white}{\textbf{توضیح}} \\
\hline
\textbf{شعار} & «رئیس‌جمهور عادی» \\
\hline
\rowcolor{rougelight}
\textbf{اصلاحات} & ازدواج همجنس‌گرایان (۲۰۱۳)—«ازدواج برای همه» \\
\hline
\textbf{تروریسم} & حملات ۲۰۱۵ (شارلی ابدو، نوامبر پاریس)، ۲۰۱۶ (نیس) \\
\hline
\rowcolor{rougelight}
\textbf{حالت اضطراری} & دو سال حالت اضطراری \\
\hline
\textbf{اقتصاد} & بیکاری بالا، رشد ضعیف \\
\hline
\rowcolor{rougelight}
\textbf{محبوبیت} & پایین‌ترین محبوبیت تاریخ جمهوری پنجم \\
\hline
\textbf{انصراف} & اولین رئیس‌جمهور که از نامزدی مجدد انصراف داد \\
\hline
\end{tabular}
\end{table}

\subsection{امانوئل ماکرون (۲۰۱۷-۲۰۲۴)}

\begin{tikzpicture}[
every node/.style={font=\small},
box/.style={rectangle, rounded corners, draw=bleurepublique, fill=bleulight,
minimum width=4cm, minimum height=1.5cm, align=center}
]
% Title
\node[font=\bfseries\large] at (7,8) {پدیده ماکرون};

% Background
\node[box] (bg) at (3,5.5) {\textbf{پیش‌زمینه}\\بانکدار، وزیر اقتصاد\\نه چپ، نه راست\\جنبش «رو به جلو»};

% 2017
\node[box, fill=vertlight, draw=vertnapoleon] (e2017) at (11,5.5) {\textbf{انتخابات ۲۰۱۷}\\۶۶٪ مقابل مارین لوپن\\فروپاشی احزاب سنتی\\جوان‌ترین رئیس‌جمهور};

% Challenges
\node[box, fill=rougelight, draw=rougerevolution] (ch1) at (3,2.5) {\textbf{جلیقه‌زردها}\\۲۰۱۸-۲۰۱۹\\اعتراض مالیاتی\\خشونت، امتیازات};

\node[box, fill=rougelight, draw=rougerevolution] (ch2) at (7,2.5) {\textbf{کووید-۱۹}\\۲۰۲۰-۲۰۲۱\\قرنطینه‌ها\\بحران اقتصادی};

\node[box, fill=rougelight, draw=rougerevolution] (ch3) at (11,2.5) {\textbf{اصلاح بازنشستگی}\\۲۰۲۳\\اعتراضات گسترده\\تصویب با ماده ۴۹-۳};

% 2022
\node[rectangle, draw=bleurepublique, fill=bleumid, text=white, rounded corners,
minimum width=8cm, minimum height=1cm, align=center] at (7,0) 
{\textbf{۲۰۲۲:} انتخاب مجدد (۵۹٪ مقابل لوپن)—اما بدون اکثریت مجلس};

\end{tikzpicture}

\begin{noktebox}
\textbf{ویژگی‌های دوره ماکرون:}

\begin{itemize}[nosep]
\item \textbf{پایان دوقطبی چپ-راست:} ظهور سه‌قطبی (ماکرونیسم، راست افراطی، چپ رادیکال)
\item \textbf{بحران نمایندگی:} احزاب سنتی فروپاشیده
\item \textbf{تنش اجتماعی:} جلیقه‌زردها، اصلاح بازنشستگی
\item \textbf{چالش‌های جهانی:} کووید، جنگ اوکراین، تورم
\item \textbf{مجلس بدون اکثریت:} از ۲۰۲۲، حکومت دشوار
\end{itemize}
\end{noktebox}

%──────────────────────────────────────────────────────────────────────────────
\section{روسای جمهور جمهوری پنجم}
%──────────────────────────────────────────────────────────────────────────────

\begin{table}[htbp]
\centering
\caption{روسای جمهور جمهوری پنجم}
\label{tab:fifth-republic-presidents}
\begin{tabular}{|c|r|c|c|p{4cm}|}
\hline
\rowcolor{bleumid}
\textcolor{white}{\textbf{ش.}} & \textcolor{white}{\textbf{نام}} & \textcolor{white}{\textbf{دوره}} & \textcolor{white}{\textbf{گرایش}} & \textcolor{white}{\textbf{ویژگی اصلی}} \\
\hline
۱ & شارل دوگل & ۱۹۵۹-۱۹۶۹ & گلیست & بنیان‌گذار، الجزایر \\
\hline
\rowcolor{bleulight}
۲ & ژرژ پمپیدو & ۱۹۶۹-۱۹۷۴ & گلیست & صنعتی‌سازی \\
\hline
۳ & والری ژیسکاردستن & ۱۹۷۴-۱۹۸۱ & لیبرال & مدرنیزاسیون اجتماعی \\
\hline
\rowcolor{rougelight}
۴ & فرانسوا میتران & ۱۹۸۱-۱۹۹۵ & سوسیالیست & چرخش، همزیستی \\
\hline
۵ & ژاک شیراک & ۱۹۹۵-۲۰۰۷ & گلیست & «نه» به عراق \\
\hline
\rowcolor{bleulight}
۶ & نیکلا سارکوزی & ۲۰۰۷-۲۰۱۲ & محافظه‌کار & بحران ۲۰۰۸ \\
\hline
۷ & فرانسوا اولاند & ۲۰۱۲-۲۰۱۷ & سوسیالیست & تروریسم \\
\hline
\rowcolor{bleulight}
۸ & امانوئل ماکرون & ۲۰۱۷-... & مرکز & فراتر از چپ و راست \\
\hline
\end{tabular}
\end{table}

%──────────────────────────────────────────────────────────────────────────────
\section{تحول نظام حزبی}
%──────────────────────────────────────────────────────────────────────────────

\begin{tikzpicture}[
every node/.style={font=\small},
era/.style={rectangle, rounded corners, minimum width=6cm, minimum height=3cm, align=center}
]
% Title
\node[font=\bfseries\large] at (7,9) {تحول نظام حزبی جمهوری پنجم};

% Era 1
\node[era, draw=bleurepublique, fill=bleulight] (e1) at (3,6) {
\textbf{۱۹۵۸-۱۹۸۱}\\[0.3em]
تسلط راست گلیست\\
چپ در اپوزیسیون\\
دوقطبی نامتقارن
};

% Era 2
\node[era, draw=rougerevolution, fill=rougelight] (e2) at (11,6) {
\textbf{۱۹۸۱-۲۰۱۷}\\[0.3em]
چرخش قدرت\\
دوقطبی متقارن\\
چپ-راست کلاسیک
};

% Era 3
\node[era, draw=vertnapoleon, fill=vertlight] (e3) at (7,2) {
\textbf{۲۰۱۷-حال}\\[0.3em]
فروپاشی دوقطبی\\
سه‌قطبی جدید\\
بحران احزاب سنتی
};

% Arrows
\draw[->, very thick] (e1) -- (e2);
\draw[->, very thick] (e2) -- (e3);

\end{tikzpicture}

\begin{table}[htbp]
\centering
\caption{احزاب اصلی و تحول آن‌ها}
\label{tab:party-evolution}
\begin{tabular}{|r|p{3.5cm}|p{3.5cm}|p{3.5cm}|}
\hline
\rowcolor{gris}
\textcolor{white}{\textbf{موقعیت}} & \textcolor{white}{\textbf{۱۹۵۸-۱۹۸۰}} & \textcolor{white}{\textbf{۱۹۸۰-۲۰۱۷}} & \textcolor{white}{\textbf{۲۰۱۷-حال}} \\
\hline
راست افراطی & حاشیه‌ای & جبهه ملی (رشد) & تجمع ملی (قدرت اصلی) \\
\hline
\rowcolor{grisclair}
راست & گلیست & گلیست به جمهوری‌خواهان & فروپاشیده \\
\hline
مرکز & کوچک & کوچک & ماکرونیسم \\
\hline
\rowcolor{grisclair}
چپ میانه & ضعیف & حزب سوسیالیست & فروپاشیده \\
\hline
چپ رادیکال & کمونیست‌ها & افول کمونیست‌ها & فرانسه نافرمان \\
\hline
\end{tabular}
\end{table}

%──────────────────────────────────────────────────────────────────────────────
\section{همزیستی‌ها: آزمون انعطاف قانون اساسی}
%──────────────────────────────────────────────────────────────────────────────

\begin{table}[htbp]
\centering
\caption{سه دوره همزیستی در جمهوری پنجم}
\label{tab:cohabitations}
\begin{tabular}{|c|c|c|c|p{4cm}|}
\hline
\rowcolor{bleumid}
\textcolor{white}{\textbf{دوره}} & \textcolor{white}{\textbf{رئیس‌جمهور}} & \textcolor{white}{\textbf{نخست‌وزیر}} & \textcolor{white}{\textbf{مدت}} & \textcolor{white}{\textbf{ویژگی}} \\
\hline
اول & میتران (چپ) & شیراک (راست) & ۱۹۸۶-۱۹۸۸ & رقابت شدید \\
\hline
\rowcolor{bleulight}
دوم & میتران (چپ) & بالادور (راست) & ۱۹۹۳-۱۹۹۵ & همکاری نسبی \\
\hline
سوم & شیراک (راست) & ژوسپن (چپ) & ۱۹۹۷-۲۰۰۲ & طولانی‌ترین \\
\hline
\end{tabular}
\end{table}

\begin{noktebox}
\textbf{چرا دیگر همزیستی نداشتیم؟}

اصلاح ۲۰۰۰ (دوره ۵ ساله) و برگزاری انتخابات مجلس بلافاصله پس از انتخابات ریاست‌جمهوری، احتمال همزیستی را کاهش داد—رئیس‌جمهور تازه‌انتخاب معمولاً اکثریت مجلس را هم می‌برد.

البته در ۲۰۲۲ ماکرون اکثریت مطلق نداشت—شکلی جدید از حکومت دشوار، اما نه همزیستی کلاسیک.
\end{noktebox}

%──────────────────────────────────────────────────────────────────────────────
\section{چالش‌های معاصر جمهوری پنجم}
%──────────────────────────────────────────────────────────────────────────────

\begin{tikzpicture}[
every node/.style={font=\small},
challenge/.style={rectangle, rounded corners, draw=rougerevolution, fill=rougelight,
minimum width=4cm, minimum height=1.8cm, align=center}
]
% Title
\node[font=\bfseries\large] at (7,8) {چالش‌های جمهوری پنجم در قرن بیست‌ویکم};

% Challenges
\node[challenge] (c1) at (2,5.5) {\textbf{بحران نمایندگی}\\بی‌اعتمادی به سیاست\\تحریم انتخابات\\پوپولیسم};

\node[challenge] (c2) at (7,5.5) {\textbf{راست افراطی}\\رشد مداوم\\۴۰٪+ در انتخابات\\تهدید دموکراسی؟};

\node[challenge] (c3) at (12,5.5) {\textbf{تنش اجتماعی}\\جلیقه‌زردها\\اصلاح بازنشستگی\\نابرابری};

\node[challenge] (c4) at (2,2) {\textbf{هویت و لائیسیته}\\اسلام در فرانسه\\تروریسم\\«جدایی‌طلبی»};

\node[challenge] (c5) at (7,2) {\textbf{محیط زیست}\\تغییر اقلیم\\انتقال انرژی\\جنبش‌های سبز};

\node[challenge] (c6) at (12,2) {\textbf{جایگاه اروپایی}\\برگزیت\\رقابت آلمان\\حاکمیت اروپایی};

\end{tikzpicture}

\subsection{صعود راست افراطی}

\begin{table}[htbp]
\centering
\caption{تحول آرای راست افراطی در انتخابات ریاست‌جمهوری}
\label{tab:far-right-evolution}
\begin{tabular}{|c|c|c|c|}
\hline
\rowcolor{rougemid}
\textcolor{white}{\textbf{سال}} & \textcolor{white}{\textbf{نامزد}} & \textcolor{white}{\textbf{دور اول}} & \textcolor{white}{\textbf{دور دوم}} \\
\hline
۱۹۸۸ & ژان-ماری لوپن & ۱۴.۴٪ & — \\
\hline
\rowcolor{rougelight}
۱۹۹۵ & ژان-ماری لوپن & ۱۵٪ & — \\
\hline
۲۰۰۲ & ژان-ماری لوپن & ۱۶.۹٪ & ۱۷.۸٪ \\
\hline
\rowcolor{rougelight}
۲۰۱۲ & مارین لوپن & ۱۷.۹٪ & — \\
\hline
۲۰۱۷ & مارین لوپن & ۲۱.۳٪ & ۳۳.۹٪ \\
\hline
\rowcolor{rougelight}
۲۰۲۲ & مارین لوپن & ۲۳.۲٪ & ۴۱.۵٪ \\
\hline
\end{tabular}
\end{table}

\begin{enghelabbox}
\textbf{از جبهه ملی به تجمع ملی}

\begin{itemize}[nosep]
\item \textbf{ژان-ماری لوپن (۱۹۷۲-۲۰۱۱):} بنیان‌گذار، نوستالژی ویشی، یهودستیزی
\item \textbf{مارین لوپن (۲۰۱۱-حال):} «زدایش شیطانی»، تمرکز بر مهاجرت و امنیت
\item \textbf{تغییر نام (۲۰۱۸):} جبهه ملی به تجمع ملی
\item \textbf{موفقیت‌ها:} ۸۹ نماینده مجلس (۲۰۲۲)، اولین گروه پارلمانی
\end{itemize}

\textbf{دلایل رشد:}
\begin{itemize}[nosep]
\item بحران اقتصادی و نابرابری منطقه‌ای
\item مهاجرت و مسائل هویتی
\item بی‌اعتمادی به «نخبگان»
\item شکست احزاب سنتی
\end{itemize}
\end{enghelabbox}

\subsection{بحران لائیسیته و اسلام}

\begin{table}[htbp]
\centering
\caption{رویدادها و قوانین مرتبط با لائیسیته}
\label{tab:laicite-events}
\begin{tabular}{|c|p{9cm}|}
\hline
\rowcolor{bleumid}
\textcolor{white}{\textbf{سال}} & \textcolor{white}{\textbf{رویداد/قانون}} \\
\hline
۱۹۸۹ & «ماجرای روسری» اول — سه دانش‌آموز با حجاب اخراج \\
\hline
\rowcolor{bleulight}
۲۰۰۴ & قانون ممنوعیت نمادهای مذهبی «آشکار» در مدارس دولتی \\
\hline
۲۰۱۰ & قانون ممنوعیت پوشش کامل صورت در فضای عمومی \\
\hline
\rowcolor{bleulight}
۲۰۱۵ & حمله به شارلی ابدو — «من شارلی هستم» \\
\hline
۲۰۱۵ & حملات ۱۳ نوامبر پاریس — ۱۳۰ کشته \\
\hline
\rowcolor{bleulight}
۲۰۲۰ & قتل ساموئل پتی (معلم) به دلیل نشان‌دادن کاریکاتور \\
\hline
۲۰۲۱ & قانون «تقویت احترام به اصول جمهوری» \\
\hline
\end{tabular}
\end{table}

\begin{naghlbox}
«لائیسیته نه ضد مذهب است و نه بی‌تفاوت به مذهب. لائیسیته آزادی باور داشتن یا نداشتن است، با برابری همه شهروندان در برابر قانون.»

\hfill --- \textit{تعریف رسمی}
\end{naghlbox}

%──────────────────────────────────────────────────────────────────────────────
\section{الگوها و درس‌ها}
%──────────────────────────────────────────────────────────────────────────────

\begin{olgoobox}
\textbf{الگوهای کلیدی جمهوری پنجم}

\begin{enumerate}
\item \textbf{الگوی «ثبات از طریق ریاست‌جمهوری قوی»:}
\begin{itemize}[nosep]
	\item جمهوری سوم و چهارم: رئیس‌جمهور ضعیف، بی‌ثباتی
	\item جمهوری پنجم: رئیس‌جمهور قوی، ثبات نسبی
	\item اما: تمرکز قدرت می‌تواند به «ابَرریاست‌جمهوری» بینجامد
\end{itemize}

\item \textbf{الگوی «انعطاف‌پذیری قانون اساسی»:}
\begin{itemize}[nosep]
	\item همزیستی نشان داد نظام می‌تواند با تقسیم قدرت کار کند
	\item چرخش چپ-راست بدون بحران نظام
	\item قانون اساسی ۲۴ بار اصلاح شده (تا ۲۰۲۴)
\end{itemize}

\item \textbf{الگوی «شخصیت‌محوری»:}
\begin{itemize}[nosep]
	\item دوگل نظام را بر شخصیت خود ساخت
	\item هر رئیس‌جمهور سبک خاص خود را دارد
	\item نظام به کاریزما وابسته است؟
\end{itemize}

\item \textbf{الگوی «بحران به‌مثابه محرک تغییر»:}
\begin{itemize}[nosep]
	\item الجزایر: تولد جمهوری پنجم
	\item مه ۶۸: تحول اجتماعی
	\item جلیقه‌زردها: بازتعریف قرارداد اجتماعی؟
\end{itemize}
\end{enumerate}
\end{olgoobox}

%──────────────────────────────────────────────────────────────────────────────
\section{جمع‌بندی فصل}
%──────────────────────────────────────────────────────────────────────────────

\begin{kholasebox}
\textbf{جمع‌بندی: جمهوری پنجم (۱۹۵۸-۲۰۲۴)}

\textbf{ویژگی‌های ساختاری:}
\begin{itemize}[nosep]
\item ریاست‌جمهوری قوی با انتخاب مستقیم
\item «پارلمانتاریسم عقلانی‌شده»
\item قدرت اجرایی دوگانه (رئیس‌جمهور/نخست‌وزیر)
\item امکان همزیستی
\item ثبات نهادی (در مقایسه با جمهوری‌های قبلی)
\end{itemize}

\textbf{دوره‌ها:}
\begin{itemize}[nosep]
\item دوگل (۱۹۵۸-۱۹۶۹): تأسیس، الجزایر، استقلال ملی، مه ۶۸
\item پمپیدو/ژیسکار (۱۹۶۹-۱۹۸۱): تداوم و مدرنیزاسیون
\item میتران (۱۹۸۱-۱۹۹۵): چرخش چپ، همزیستی، اروپا
\item شیراک (۱۹۹۵-۲۰۰۷): همزیستی سوم، ۲۱ آوریل، عراق
\item سارکوزی/اولاند/ماکرون (۲۰۰۷-۲۰۲۴): بحران‌ها و تحولات
\end{itemize}

\textbf{رویدادهای کلیدی:}
\begin{itemize}[nosep]
\item استقلال الجزایر (۱۹۶۲)
\item انتخاب مستقیم رئیس‌جمهور (۱۹۶۲)
\item مه ۱۹۶۸
\item چرخش ۱۹۸۱ (میتران)
\item سه همزیستی
\item «زلزله» ۲۱ آوریل ۲۰۰۲
\item انتخاب ماکرون و فروپاشی احزاب سنتی (۲۰۱۷)
\end{itemize}

\textbf{چالش‌های معاصر:}
\begin{itemize}[nosep]
\item صعود راست افراطی
\item بحران نمایندگی سیاسی
\item تنش‌های اجتماعی
\item لائیسیته و هویت
\item جایگاه در اروپا و جهان
\end{itemize}

\textbf{میراث:}
\begin{itemize}[nosep]
\item طولانی‌ترین و باثبات‌ترین رژیم پس از انقلاب
\item الگوی «نیمه‌ریاستی» برای دیگر کشورها
\item تعادل (شکننده؟) میان اقتدار و دموکراسی
\end{itemize}
\end{kholasebox}

%──────────────────────────────────────────────────────────────────────────────
\section*{منابع فصل}
%──────────────────────────────────────────────────────────────────────────────
\addcontentsline{toc}{section}{منابع فصل}

\begin{itemize}[nosep]
\item Berstein, Serge, and Jean-Pierre Rioux. \textit{The Pompidou Years, 1969-1974}. Cambridge: Cambridge UP, 2000.
\item Cole, Alistair. \textit{French Politics and Society}. 3rd ed. London: Routledge, 2017.
\item Elgie, Robert. \textit{Political Institutions in Contemporary France}. Oxford: Oxford UP, 2003.
\item Gildea, Robert. \textit{France Since 1945}. 2nd ed. Oxford: Oxford UP, 2002.
\item Hazareesingh, Sudhir. \textit{In the Shadow of the General}. Oxford: Oxford UP, 2012.
\item Jackson, Julian. \textit{De Gaulle}. Cambridge: Harvard UP, 2018.
\item Knapp, Andrew, and Vincent Wright. \textit{The Government and Politics of France}. 5th ed. London: Routledge, 2006.
\item Lacouture, Jean. \textit{De Gaulle}. 3 vols. Paris: Seuil, 1984-1986.
\item Perrineau, Pascal. \textit{Le Vote disruptif}. Paris: Presses de Sciences Po, 2017.
\item Shields, James. \textit{The Extreme Right in France}. London: Routledge, 2007.
\end{itemize}

\end{document}